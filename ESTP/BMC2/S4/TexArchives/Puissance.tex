
\section{Puissance de matrices}

\subsection{La méthode de la conjecture}

\begin{Meth}\textbf{Conjecture et récurrence}
\vspace{1em}

Pour calculer $A^n$, on peut :
\begin{enumerate}
    \item Calculer les premières puissances de $A$ : $A^2$, $A^3$, ...
    \item Conjecturer la forme générale de $A^n$.
    \item Démontrer cette conjecture par récurrence.
\end{enumerate}
\end{Meth}

\vspace{1em}
\hrule
\vspace{1em}

\exo[1]{Conjecture classique}

Soit la matrice dont tous les coefficients valent $1$ :
$$J = \begin{pmatrix}
1 & 1 & 1 \\
1 & 1 & 1 \\
1 & 1 & 1
\end{pmatrix}$$

\begin{enumerate}
    \item Calculer $J^2$ et $J^3$.
    \item Conjecturer la forme de $J^k$ pour tout $k \in \mathbb{N}^*$.
    \item Démontrer par récurrence que $\forall k \in \mathbb{N}^*, J^k = 3^{k-1} J$.
\end{enumerate}

\vspace{1em}
\hrule
\vspace{1em}

\begin{Rmq}
\textbf{Démonstration :}\textbf{Solution de l'exercice précédent}
\vspace{1em}

\textbf{Initialisation :} Pour $k = 1$, $J^1 = J = 3^0 J$. Vrai.

\textbf{Hérédité :} Supposons que $J^k = 3^{k-1} J$. Alors :
$$J^{k+1} = J^k \cdot J = (3^{k-1} J) \cdot J = 3^{k-1} J^2$$
Or $J^2 = 3J$, donc :
$$J^{k+1} = 3^{k-1} \cdot 3J = 3^k J$$
\end{Rmq}


\subsection{Utilisation des relations algébriques}

\begin{Meth}\textbf{Décomposition $A = B + I$}
\vspace{1em}

Si $B = A - I$ vérifie $B^k = 0$ pour un certain $k$, alors on utilise le binôme de Newton :
$$A^n = (B + I)^n = \sum_{j=0}^n \binom{n}{j} B^j$$
Cette somme est finie car $B^j = 0$ pour $j \geq k$.
\end{Meth}

\vspace{1em}
\hrule
\vspace{1em}

\exo[2]{Méthode avec $B = A - I$}

Soit $A = \begin{pmatrix}
4 & -3 \\
3 & -2
\end{pmatrix}$.

\begin{enumerate}
    \item Poser $B = A - I_2$ et calculer $B^2$.
    \item En déduire $A^n$ pour tout $n \in \mathbb{N}$.
\end{enumerate}

\textit{Indication :} On trouve $B^2 = 0$, donc $A^n = (B + I)^n = I + nB$.

\vspace{1em}
\hrule
\vspace{1em}

\subsection{Matrices diagonales et nilpotentes}

\begin{Prop}\textbf{Puissance d'une matrice diagonale}
\vspace{1em}

Si $A$ est diagonale :
$$A = \begin{pmatrix}
a & 0 & 0 \\
0 & b & 0 \\
0 & 0 & c
\end{pmatrix}
\quad \Rightarrow \quad
A^n = \begin{pmatrix}
a^n & 0 & 0 \\
0 & b^n & 0 \\
0 & 0 & c^n
\end{pmatrix}$$
\end{Prop}

\begin{Def}\textbf{Matrice nilpotente}
\vspace{1em}

Une matrice $N$ est \textbf{nilpotente} s'il existe un entier $k$ tel que $N^k = 0$.
\end{Def}

\begin{Rmq}$\,$

Pour une matrice nilpotente $N$ avec $N^k = 0$, on a $N^n = 0$ pour tout $n \geq k$.
\end{Rmq}

\subsection{Méthode avec polynôme annulateur}

\begin{Meth}\textbf{Division euclidienne}
\vspace{1em}

Si $P(A) = 0$ pour un polynôme $P$ de degré $d$, on effectue la division euclidienne de $X^n$ par $P(X)$ :
$$X^n = Q(X) \cdot P(X) + R(X) \quad \text{avec } \deg R < d$$
Alors $A^n = R(A)$.
\end{Meth}

\vspace{1em}
\hrule
\vspace{1em}

\section{Exercices}

\vspace{1em}
\hrule
\vspace{1em}

\exo[1]{Puissance $n$-ième par récurrence}

On considère les matrices :
$$A = \begin{pmatrix}
1 & -1 \\
-1 & 1
\end{pmatrix}, \quad
B = \begin{pmatrix}
1 & 1 \\
0 & 2
\end{pmatrix}$$

\begin{enumerate}
    \item Calculer $A^2$, $A^3$. En déduire $A^n$ pour tout $n \geq 1$.
    \item Calculer $B^2$, $B^3$. En déduire $B^n$ pour tout $n \geq 1$.
\end{enumerate}

\vspace{1em}
\hrule
\vspace{1em}

\exo[2]{Puissance $n$-ième avec le binôme}

Soit :
$$A = \begin{pmatrix}
1 & 1 & 0 \\
0 & 1 & 1 \\
0 & 0 & 1
\end{pmatrix}, \quad
I = \begin{pmatrix}
1 & 0 & 0 \\
0 & 1 & 0 \\
0 & 0 & 1
\end{pmatrix}, \quad
B = A - I$$

\begin{enumerate}
    \item Calculer $B^n$ pour tout $n \in \mathbb{N}$.
    \item En déduire $A^n$.
\end{enumerate}

\vspace{1em}
\hrule
\vspace{1em}

\exo[2]{Puissance $k$-ième sans division euclidienne}

Soit la matrice :
$$U = \begin{pmatrix}
0 & 1 & 1 & 1 \\
1 & 0 & 1 & 1 \\
1 & 1 & 0 & 1 \\
1 & 1 & 1 & 0
\end{pmatrix}$$

\begin{enumerate}
    \item Calculer $U^2$ et en déduire une relation simple liant $U^2$, $U$ et $I_4$.
    
    \item Soit $(\alpha_k)$ et $(\beta_k)$ les suites définies par :
    $$\alpha_0 = 1, \quad \beta_0 = 0, \quad \alpha_{k+1} = 3\beta_k, \quad \beta_{k+1} = \alpha_k + 2\beta_k$$
    Démontrer que pour tout $k \in \mathbb{N}$ :
    $$U^k = \begin{pmatrix}
    \alpha_k & \beta_k & \beta_k & \beta_k \\
    \beta_k & \alpha_k & \beta_k & \beta_k \\
    \beta_k & \beta_k & \alpha_k & \beta_k \\
    \beta_k & \beta_k & \beta_k & \alpha_k
    \end{pmatrix}$$
    
    \item Démontrer que pour tout $k \in \mathbb{N}$, $\beta_{k+2} = 2\beta_{k+1} + 3\beta_k$.
    
    \item En déduire que :
    $$\beta_k = \frac{3^k - (-1)^k}{4} \quad \text{et} \quad \alpha_k = \frac{3^k + 3(-1)^k}{4}$$
\end{enumerate}

\vspace{1em}
\hrule
\vspace{1em}

\exo[2]{Puissance $n$-ième avec polynôme annulateur}

\begin{enumerate}
    \item Pour $n \geq 2$, déterminer le reste de la division euclidienne de $X^n$ par $X^2 - 3X + 2$.
    
    \item Soit $A = \begin{pmatrix}
    0 & 1 & -1 \\
    -1 & 2 & -1 \\
    1 & -1 & 2
    \end{pmatrix}$.
    
    Vérifier que $A^2 - 3A + 2I = 0$ et en déduire $A^n$ pour $n \geq 2$.
\end{enumerate}

\vspace{1em}
\hrule
\vspace{1em}

\exo[2]{Puissance $k$-ième avec polynôme annulateur}

Soit la matrice :
$$U = \begin{pmatrix}
0 & 1 & 1 & 1 \\
1 & 0 & 1 & 1 \\
1 & 1 & 0 & 1 \\
1 & 1 & 1 & 0
\end{pmatrix}$$

\begin{enumerate}
    \item Déterminer une relation simple liant $I_4$, $U$ et $U^2$.
    \item En déduire, pour $k \geq 0$, la valeur de $U^k$.
\end{enumerate}

\vspace{1em}
\hrule
\vspace{1em}

\exo[2]{Matrice de rotation}

Soit la matrice de rotation :
$$R_\theta = \begin{pmatrix}
\cos\theta & -\sin\theta \\
\sin\theta & \cos\theta
\end{pmatrix}$$

\begin{enumerate}
    \item Montrer que $R_\theta^n = R_{n\theta}$ pour tout $n \in \mathbb{N}$.
    \item En déduire une formule pour $\cos(n\theta)$ et $\sin(n\theta)$.
\end{enumerate}

\vspace{1em}
\hrule
\vspace{1em}

\exo[3]{Fibonacci et matrices}

Soit la matrice :
$$F = \begin{pmatrix}
1 & 1 \\
1 & 0
\end{pmatrix}$$

\begin{enumerate}
    \item Calculer $F^2$, $F^3$, $F^4$.
    \item Montrer que $F^n = \begin{pmatrix}
    f_{n+1} & f_n \\
    f_n & f_{n-1}
    \end{pmatrix}$ où $(f_n)$ est la suite de Fibonacci.
    \item En diagonalisant $F$, retrouver la formule de Binet :
    $$f_n = \frac{1}{\sqrt{5}}\left[\left(\frac{1+\sqrt{5}}{2}\right)^n - \left(\frac{1-\sqrt{5}}{2}\right)^n\right]$$
\end{enumerate}

\vspace{1em}
\hrule
\vspace{1em}

\exo[2]{Suite récurrente linéaire}

Soit $(u_n)$ la suite définie par $u_0 = 1$, $u_1 = 2$ et $u_{n+2} = 3u_{n+1} - 2u_n$.

\begin{enumerate}
    \item Écrire ce système sous forme matricielle $X_{n+1} = AX_n$ avec $X_n = \begin{pmatrix} u_{n+1} \\ u_n \end{pmatrix}$.
    \item Diagonaliser $A$ et en déduire $A^n$.
    \item Exprimer $u_n$ en fonction de $n$.
\end{enumerate}

\vspace{1em}
\hrule
\vspace{1em}

\exo[3]{Matrice trigonalisable et puissance}

Soit :
$$A = \begin{pmatrix}
2 & 1 & 0 \\
0 & 2 & 1 \\
0 & 0 & 2
\end{pmatrix}$$

\begin{enumerate}
    \item Montrer que $A$ n'est pas diagonalisable.
    \item Écrire $A = 2I + N$ et calculer $N^2$, $N^3$.
    \item En déduire $A^n$ pour tout $n \in \mathbb{N}$.
\end{enumerate}

\vspace{1em}
\hrule
\vspace{1em}

\exo[2]{Application aux suites}

Soit $(x_n, y_n, z_n)$ la suite définie par :
$$\begin{cases}
x_{n+1} = 2x_n + y_n \\
y_{n+1} = x_n + 2y_n \\
z_{n+1} = z_n
\end{cases}$$
avec $(x_0, y_0, z_0) = (1, 0, 1)$.

\begin{enumerate}
    \item Écrire ce système sous forme matricielle.
    \item Calculer les valeurs propres de la matrice.
    \item Exprimer $x_n$, $y_n$, $z_n$ en fonction de $n$.
\end{enumerate}

