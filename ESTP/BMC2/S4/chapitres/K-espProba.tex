


\section{Espace probabilisé et probabilisable}


\begin{Def}\textbf{Ensemble dénombrable}\\

On dit qu'un ensemble est dénombrable s'il est en bijection avec l'ensemble $\mathbb{N}$.\\
\end{Def}
\begin{Ex}
Les ensembles $\mathbb{N}$, $\mathbb{N}^*$.
Les ensembles $\mathbb{Z}$ et $\mathbb{Q}$ sont dénombrables, mais $\mathbb{R}$ n'est pas dénombrable.\\
\end{Ex}

\begin{Rmq} 
Les indices de sommation d'une série forment un ensemble dénombrable.\\
\end{Rmq} 

\begin{Def}\textbf{Univers des possibles $\Omega$}\\
On appelle univers des possibles l'ensemble $\Omega$ des résultats possibles décrivant une expérience aléatoire.\\ 
On ne considérera dans ce chapitre que des univers discrets, i.e. finis ou dénombrables.\\
\end{Def}

\begin{Ex}
\begin{itemize}
    \item On lance un dé. On peut alors choisir $\Omega = [1, 6]$.
    \item On lance une pièce. On choisit alors $\Omega = \{P, F\}$.
    \item On lance une pièce $n$ fois. On choisit alors $\Omega = \{P, F\}^n$.
    \item On lance une pièce indéfiniment. On choisit alors $\Omega = \{P, F\}^{\mathbb{N}^*}$.
    \item Une urne contient une boule blanche, notée $B$, et quatre boules rouges, notées $R$.
    \begin{itemize}
        \item On tire successivement deux boules avec remise : $\Omega = \{(B, B), (B, R), (R, B), (R, R)\}$.
        \item On tire successivement deux boules sans remise : $\Omega = \{(B, R), (R, B), (R, R)\}$.
        \item On tire simultanément deux boules : $\Omega = \{\{B, R\}, \{R, R\}\}$.\\
    \end{itemize}
\end{itemize}
\end{Ex}

\begin{Def}\textbf{Tribu (ou $\sigma$-algèbre)}\\

On appelle tribu (ou $\sigma$-algèbre) sur un univers $\Omega$ tout sous-ensemble $T$ de l'ensemble $\mathcal{P}(\Omega)$ des parties de $\Omega$ vérifiant les propriétés suivantes :
\begin{itemize}
    \item[(i)] la tribu $T$ contient l'univers : $\Omega \in T$ ;
    \item[(ii)] la tribu $T$ est stable par complémentaire : $\forall A \in T, A^c \in T$ ;
    \item[(iii)] la tribu $T$ est stable par réunion d'une suite d'événements de la tribu :
    $\forall (B_n)_{n \in \mathbb{N}} \in T^{\mathbb{N}}, \bigcup_{n \in \mathbb{N}} B_n \in T$.
\end{itemize}
On définit alors les éléments de la tribu $T$ :
\begin{itemize}
    \item les éléments de $T$ sont appelés événements de l'expérience aléatoire,
    \item les singletons de $T$ sont appelés événements élémentaires,
    \item l'événement $\Omega$ est appelé l'événement certain, l'événement $\emptyset$ est appelé l'événement impossible.
\end{itemize}
On dit alors que le couple $(\Omega, T)$ est un espace probabilisable.\\
\end{Def}


\begin{Ex}
Soit $\Omega$ un univers et soit $A \subset \Omega$. Les ensembles $T_1 = \{\Omega, \emptyset\}$, $T_2 = \{\Omega, A, A^c, \emptyset\}$ et $\mathcal{P}(\Omega)$ sont des tribus sur $\Omega$.

\end{Ex}


\begin{Prop}
Soit $(\Omega, \mathcal{T})$ un espace probabilisable. Alors :\\
\begin{enumerate}
    \item $\emptyset \in \mathcal{T}$,
    \item Pour tout $(A, B) \in \mathcal{T}^2$, $A \cap B, A \cup B$ et $A \setminus B$ sont des événements, i.e. $A \cap B \in \mathcal{T}$, $A \cup B \in \mathcal{T}$ et $A \setminus B \in \mathcal{T}$.
    \item La tribu $\mathcal{T}$ est stable par intersection dénombrable.\\
\end{enumerate}
\end{Prop}

\begin{Def}\textbf{Expérience aléatoire}\\

On appelle expérience aléatoire une expérience dont le résultat n'est pas connu a priori, c'est-à-dire qu'il ne dépend que du hasard. L'ensemble des résultats possibles d'une expérience aléatoire est appelé univers.\\
 On le note usuellement $\Omega$ et ses éléments, appelées issues, sont notés $\omega$. Lors de chaque expérience, une seule issue $\omega$ de $\Omega$ est réalisée. Dans ce chapitre, on ne traitera que le cas où $\Omega$ est un ensemble fini.\\
\end{Def}

\begin{Ex}
\begin{enumerate}
    \item On lance un dé à six faces numérotées de 1 à 6, on a $\Omega = \{1, 2, 3, 4, 5, 6\}$.
    \item On lance une pièce, on a $\Omega = \{\text{Pile}, \text{Face}\}$.
    \item On lance 3 fois une pièce, on a 
    $\Omega = \{\text{PPP}, \text{PPF}, \text{PFP}, \text{PFF}, \text{FPP}, \text{FPF}, \text{FFP}, \text{FFF}\}$.
\end{enumerate}
\end{Ex}


\begin{Def}\textbf{Évènement}\\

Dans le cas où l'univers d'une expérience est fini, on appelle évènement toute partie de $\Omega$. Le couple $(\Omega, \mathcal{P}(\Omega))$ et alors appelé espace probabilisable.\\

\end{Def}


\begin{Ex}
Pour le lancer d'un dé à 6 faces, l'événement “le nombre obtenu est pair” est $\{2, 4, 6\}$.
\end{Ex}

\begin{Def}\textbf{Évènement impossible, certain, élémentaire}\\
\begin{itemize}
    \item $\emptyset$ est appelé l'événement impossible.
    \item $\Omega$ est appelé l'événement certain.
    \item  Les événements $\{w\}$ (les singletons) sont appelés événements élémentaires.\\
\end{itemize}

\end{Def}


\section{Opérations sur les évènements}

\subsection{Opérations sur les évènements}
Dans le cadre des probabilités, on a l'interprétation suivante des opérations sur les parties : 

\begin{Def} \textbf{Opérations sur les évènements}\\

Soit $A$ et $B$ deux événements.
\begin{enumerate}
    \item $A^c$ (le complémentaire de $A$ dans $\Omega$) est appelé événement contraire de $A$.
    \item L'événement $A \cap B$ est réalisé si les événements $A$ et $B$ sont réalisés.
    \item L'événement $A \cup B$ est réalisé si au moins un des deux événements $A$ ou $B$ est réalisé.
    \item $A \subset B$ signifie que l'événement $A$ implique l'événement $B$.
    \item Si $A \cap B = \emptyset$, alors on dit que $A$ et $B$ sont incompatibles.\\
\end{enumerate}
\end{Def}

\begin{Def}\textbf{Système complet d'évènements}\\

Soit $(A_i)_{i \in I}$ une famille d'événements (où $I$ est un ensemble fini de $\mathbb{N}$). On dit que cette famille forme un système complet d'événements de $\Omega$ si :
\begin{enumerate}
    \item $\forall(i, j) \in I^2$ tels que $i \neq j$, $A_i \cap A_j = \emptyset$.
    \item $\bigcup_{i \in I} A_i = \Omega$.\\
\end{enumerate}
\end{Def}

\begin{Ex}
\begin{enumerate}
\item Pour tout événement $A$, $\{A, A^c\}$ est un système complet d'événements.
\item Dans le cas d'un lancé de dé, si on note :
\begin{itemize}
    \item $A$: "le résultat est pair"
    \item $B$: "le résultat est égal à 3 ou 5"
    \item $C$: "le résultat est égal à 1"
\end{itemize}
Alors $\{A, B, C\}$ est un s.c.e.
\item On tire 3 fois et avec remise une boule dans une urne contenant des boules noires et blanches. On note $X$ le rang de la première boule blanche tirée en convenant que si on ne tire finalement pas de boule blanche alors $X$ vaut 0. Alors 
\[
\{(X = 0), (X = 1), (X = 2), (X = 3)\}
\] 
est un système complet d'événements.
\end{enumerate}
\end{Ex}



\subsection{Définition et premières propriétés}

\begin{Def}\textbf{Probabilité - Espace probabilisé fini }\\

Soit $\Omega$ un ensemble fini. On appelle probabilité sur l'espace probabilisable $(\Omega, \mathcal{P}(\Omega))$ toute application $\mathbb{P} : \mathcal{P}(\Omega) \rightarrow [0, 1]$ vérifiant :
\begin{itemize}
    \item[1.] $\mathbb{P}(\Omega) = 1$
    \item[2.] $\forall A, B \in \mathcal{P}(\Omega)$ tels que $A \cap B = \emptyset$, $\mathbb{P}(A \cup B) = \mathbb{P}(A) + \mathbb{P}(B)$.
\end{itemize}
Le triplet $(\Omega, \mathcal{P}(\Omega), \mathbb{P})$ est alors appelé espace probabilisé fini.\\
\end{Def}

\begin{Rmq} 
On retiendra surtout de cette définition que si $A$ et $B$ sont incompatibles, $\mathbb{P}(A \cup B) = \mathbb{P}(A) + \mathbb{P}(B)$.\\
\end{Rmq} 

\begin{Prop}\textbf{Propriété de base d'une probabilité (admis)}\\

Soit $\mathbb{P}$ une probabilité sur $\Omega$ et $A$ et $B$ deux événements de $\Omega$.
\begin{itemize}
    \item[(I)] $\mathbb{P}(A) = 1 - \mathbb{P}(A^c)$
    \item[(II)] Si $A \subset B$ alors $\mathbb{P}(A) \leq \mathbb{P}(B)$
    \item[(III)] Pour toute famille $(A_i)_{1 \leq i \leq n}$ d'événements deux à deux incompatibles, on a
    \[
    \mathbb{P}\left( \bigcup_{i=1}^{n} A_i \right) = \sum_{i=1}^{n} \mathbb{P}(A_i)
    \]
    Attention ! (I) On écrira donc $\mathbb{P}(A_1 \cup A_2 \cup \cdots \cup A_n) = \mathbb{P}(A_1) + \mathbb{P}(A_2) + \cdots + \mathbb{P}(A_n)$ que si l'on sait que $A_1, A_2, \ldots, A_n$ sont deux à deux incompatibles.\\
\end{itemize}
\end{Prop}

\subsection{Probabilité et Événements Élémentaires}

\begin{Prop}\textbf{Définition d'une probabilité par les événements élémentaires (admis)}\\

Une probabilité est entièrement déterminée par la connaissance des probabilités des événements élémentaires.\\
 On peut donc définir une probabilité en attribuant à chaque événement élémentaire une probabilité de telle sorte que la somme des probabilités des événements élémentaires soit égale à $1$.\\
\end{Prop}



\begin{Prop}\textbf{Probabilité d'un événement et événements élémentaires (admis)}\\

Soit $(\Omega, \mathcal{P}(\Omega), \mathbb{P})$ un espace probabilisé fini. La probabilité d'un événement est égale à la somme des probabilités de ses événements élémentaires. Autrement dit :
\[
\forall A \in \mathcal{P}(\Omega), \quad \mathbb{P}(A) = \sum_{\omega \in A} \mathbb{P}(\{\omega\}).
\]
\end{Prop}

\begin{Rmq} 
On pourra se contenter de retenir la phrase suivante, moins rigoureuse : la probabilité d'un événement est égale à la somme des probabilités de ses issues.\\


\end{Rmq} 

\exo[1]{Dé pipé}

On lance un dé à 6 faces. On note, pour tout $i \in [1 ; 6]$, $p_i = \mathbb{P}(\{i\})$. Ce dé est pipé de telle façon que :
\begin{align*}
    p_1 &= p_2 = p_3 = \frac{1}{12}, \\
    p_4 &= \frac{1}{2}, \\
    p_5 &= p_6.
\end{align*}
Déterminer la probabilité d'obtenir un nombre pair avec ce dé.

\begin{Def}\textbf{Equiprobabilité } \\

On dit que l'on est en situation d'équiprobabilité si les issues ont toutes la même probabilité.\\

\end{Def}

\begin{Prop}\textbf{Probabilité d'un évènement en situation d'équiprobabilité (admis)} \\

En situation d'équiprobabilité, on a :
\[
\mathbb{P}(A) = \frac{\text{card } A}{\text{card } \Omega}.
\]

\end{Prop}

\subsection{Formule du crible de Poincaré}
\bigskip

\begin{Prop}\textbf{Formule du crible de Poincaré (admis)} \\
Soit $A$, $B$ et $C$ trois événements.
\[
\mathbb{P}(A \cup B) = \mathbb{P}(A) + \mathbb{P}(B) - \mathbb{P}(A \cap B)
\]
et
\[
\mathbb{P}(A \cup B \cup C) = \mathbb{P}(A) + \mathbb{P}(B) + \mathbb{P}(C) - \mathbb{P}(A \cap B) - \mathbb{P}(A \cap C) - \mathbb{P}(B \cap C) + \mathbb{P}(A \cap B \cap C).
\]

\end{Prop}

\exo[2]{Urne et boules}

Une urne contient trois boules (une rouge, une verte et une jaune) indiscernables au toucher. On fait $r$ tirages avec remise, ($r \geq 3$). Déterminer la probabilité que chacune des boules ait été tirée au moins une fois. Vérifier ensuite que cette probabilité tend vers $1$ lorsque $r$ tend vers $+\infty$.\\




\section{Probabilités conditionnelles}
\subsection{Définition et propriétés de base}

\begin{Def}\textbf{Probabilité sachant A}\\

 Soit $(\Omega, \mathcal{P}(\Omega), \mathbb{P})$ un espace probabilisé et $A$ un événement de probabilité non nulle. Alors l'application $\mathbb{P}P_A$ définie sur $\mathcal{P}(\Omega)$ par 
\[
\mathbb{P}_A(B) = \frac{\mathbb{P}(B \cap A)}{\mathbb{P}(A)}
\]
est une probabilité appelée probabilité conditionnelle relative à $A$ ou probabilité sachant $A$.\\
\end{Def}

\begin{Rmq}  Le minimum vital consistera ici à retenir que 
\[
\mathbb{P}_A(B)= \mathbb{P}(B/A) = \frac{\mathbb{P}(B \cap A)}{\mathbb{P}(A)}.
\]
$\mathbb{P}_A$ étant une probabilité, on retrouve les formules valables pour toute probabilité, appliquées à $\mathbb{P}_A$.\\
\end{Rmq} 

\begin{Prop}\textbf{Propriétés de $\mathbb{P}_A$ (admis)}\\

\begin{enumerate}
    \item $\mathbb{P}_A(\bar{B}) = 1 - \mathbb{P}_A(B)$
    \item Si $B \subset C$ alors $\mathbb{P}_A(B) \leq \mathbb{P}_A(C)$
    \item Formule du crible (appliquée à $\mathbb{P}_A$) : 
    \[
    \mathbb{P}_A(B \cup C) = \mathbb{P}_A(B) + \mathbb{P}_A(C) - \mathbb{P}_A(B \cap C).
    \]
\end{enumerate}
\end{Prop}

\begin{Rmq} 
 Il faut connaître le sens des probabilités que l'on rencontre sur les branches d'un arbre de probabilité.\\
\end{Rmq} 

\exo[1]{Tirages sans remise}

Une urne contient 3 boules blanches, 2 boules rouges et 5 boules noires. On effectue deux tirages sans remise et on note $B_i$ (resp. $R_i$, $N_i$) l'évènement ``la $i$-ième boule tirée est blanche (resp. rouge, noire)''. Déterminer la probabilité d'obtenir deux boules blanches.\\


\subsection{Formule des probabilités composées}

\begin{Prop}\textbf{Formule des probabilités composées (admis)}\\

\begin{enumerate}
    \item Si $\mathbb{P}(A) \neq 0$, alors 
    \[
    \mathbb{P}(A \cap B) = \mathbb{P}(A)\mathbb{P}_A(B).
    \]
    \item Si $\mathbb{P}(A_1 \cap A_2 \cap \ldots \cap A_{n-1}) \neq 0$, alors 
    \[
    \mathbb{P}\left(\bigcap_{i=1}^n A_i\right) = \mathbb{P}(A_1 \cap A_2 \cap \ldots \cap A_n) = \mathbb{P}(A_1)\mathbb{P}_{A_1}(A_2) \ldots \mathbb{P}_{A_1 \cap A_2 \cap \ldots \cap A_{n-1}}(A_n).
    \]
\end{enumerate}
\end{Prop}

\exo[1]{Trois tirages sans remise}

On reprend l'urne de l'exemple précédent. On effectue maintenant trois tirages sans remise. Déterminer la probabilité d'obtenir trois boules blanches.\\






\subsection{Formule des probabilités totales}
\begin{Prop}
\textbf{Formule des probabilités totales (admis)}\\

Soit $(A_1, A_2, \ldots, A_n)$ un système complet d'événements tel que $\forall i \in \{1, \ldots, n\}, \mathbb{P}(A_i) \neq 0$.

Alors pour tout événement $B$, on a :
\[
\mathbb{P}(B) = \sum_{i=1}^{n} \mathbb{P}(B \cap A_i) = \sum_{i=1}^{n} \mathbb{P}(A_i) \mathbb{P}_{A_i}(B)
\]
\end{Prop}

\begin{Rmq} On utilise ce théorème quand on manque d'information ; selon l'information manquante, on sait quel système complet choisir.\\
\end{Rmq} 

\subsection{Formule de Bayes}


\begin{Prop}
\textbf{Formule de Bayes (admis)}\\

1. Soit $A$ et $B$ deux événements de probabilité non nulle, alors :
\[
\mathbb{P}_B(A) = \frac{\mathbb{P}(A) \mathbb{P}_A(B)}{\mathbb{P}(B)}
\]

2. Soit $A_1, \ldots, A_n$ un système complet d'événements et $B$ un événement, alors $\forall i \in \{1, \ldots, n\}$ :
\[
\mathbb{P}_B(A_i) = \frac{\mathbb{P}(A_i) \mathbb{P}_{A_i}(B)}{\sum_{j=1}^{n} \mathbb{P}(A_j) \mathbb{P}_{A_j}(B)}
\]

\end{Prop}

\begin{Rmq}  On utilise ce théorème pour échanger le conditionnement, mais on peut s'en passer en utilisant systématiquement la formule des probabilités totales.\\
\end{Rmq} 

% \exo{

% On reprend l'exemple précédent. Sachant que la deuxième boule tirée est blanche, quelle est la probabilité que la première soit blanche ?\\
% }







\section{Indépendance d'évènements}

\subsection{Indépendance de deux évènements}

\begin{Def}
\textbf{Indépendance de deux évènements}\\
Soit $(\Omega, \mathcal{P}(\Omega), \mathbb{P})$ un espace probabilisé fini. On dit que deux événements $A$ et $B$ sont indépendants pour la probabilité $\mathbb{P}$ si
\[
\mathbb{P} (A \cap B) = \mathbb{P} (A) \times \mathbb{P} (B).
\]
Si $\mathbb{P} (A) \neq 0$, cela revient à dire que $\mathbb{P}_A(B) = \mathbb{P} (B)$.\\
\textbf{Attention !} L'indépendance dépend de la probabilité !\\
\end{Def}

\begin{Ex}
On dispose d'un dé équilibré, d'une pièce équilibrée et d'une pièce truquée (2 piles). On lance le dé : si on obtient 1, on lance deux fois la pièce équilibrée, sinon on lance deux fois la pièce truquée. On note $A_i$ "obtenir pile au $i$-ième lancer" et $C$ "le lancer du dé donne 1".\\
 Montrons que $A_1$ et $A_2$ sont indépendants pour $P_C$ mais pas pour $P$.\\
\end{Ex}

\subsection{Indépendance mutuelle d'un ensemble d'évènements}

\begin{Def}\textbf{Indépendance mutuelle d'un ensemble d'évènements}\\
On dit que $A_1, \ldots, A_n$ sont mutuellement indépendants pour la probabilité $\mathbb{P}$ si pour tout ensemble d'indice $I \subset J_{1,n}$,
\[
\mathbb{P} \left( \bigcap_{i \in I} A_i \right) = \prod_{i \in I} \mathbb{P} (A_i).\\
\]
\end{Def}

\begin{Ex}  Trois évènements $A_1, A_2, A_3$ sont mutuellement indépendants si et seulement si :\\
$$
\begin{array}{ll}
\mathbb{P} (A_1 \cap A_2) & = \mathbb{P} (A_1)\mathbb{P} (A_2) \\
\mathbb{P} (A_1 \cap A_3) & = \mathbb{P} (A_1)\mathbb{P} (A_3) \\
\mathbb{P} (A_2 \cap A_3) & = \mathbb{P} (A_2)\mathbb{P} (A_3) \\
\mathbb{P} (A_1 \cap A_2 \cap A_3) & = \mathbb{P} (A_1)\mathbb{P} (A_2)\mathbb{P} (A_3).
\end{array}
$$

\end{Ex}

\exo[1]{\textbf{Dé}}

On lance deux fois un dé équilibré.\\
 Soit les événements : $A_1$: “le premier nombre obtenu est pair” et $A_2$: “le deuxième nombre obtenu est pair” et $A_3$: “la somme des deux nombres obtenus est paire”. Vérifier que les $A_i$ sont indépendants deux à deux mais pas mutuellement indépendants.\\






\section{Exercices}






% Exercice 2016

\exo[2]{\textbf{Vrai/Faux}}

Les assertions suivantes sont-elles vraies ou fausses?
\begin{enumerate}
\item Deux événements incompatibles sont indépendants.
\item Deux événements indépendants sont incompatibles.
\item Si $P(A)+P(B)=1$, alors $A=\bar B$.
\item Si $A$ et $B$ sont deux événements indépendants, alors $P(A\cup B)=P(A)+P(B)$.
\item Soit $(A_n)_{n\in\mathbb N}$ et $(B_p)_{p\in\mathbb N}$ deux systèmes complets d'événements. Alors $(A_n\cap B_p)_{(n,p)\in\mathbb N^2}$ est un système complet d'événements.\\
\end{enumerate}


% Exercice 1227



\exo[1]{\textbf{Écriture ensembliste}}

Soit $\Omega$ un univers et soient $A,B,C$ trois événements de $\Omega$. Traduire en termes ensemblistes
(en utilisant uniquement les symboles d'union, d'intersection et de passage au complémentaire, ainsi que $A$, $B$ 
et $C$) les événements suivants :
\begin{multicols}{2}
\begin{enumerate}
 \item Seul $A$ se réalise;
\item $A$ et $B$ se réalisent, mais pas $C$.
\item les trois événements se réalisent;
\item au moins l'un des trois événements se réalise;
\item au moins deux des trois événements se réalisent;
\item aucun ne se réalise;
\item au plus l'un des trois se réalise;
\item exactement deux des trois se réalisent;\\
\end{enumerate}

\end{multicols}





\exo[1]{\textbf{Vocabulaire 1}}

\begin{enumerate}
\item On pose, pour tout $n \geq 1$, $A_n = \left[-\frac{1}{n}, \frac{1}{n}\right]$.\\
Déterminer:
\begin{multicols}{2}
\begin{itemize}
    \item $\bigcup_{k=1}^{n} A_k$ ;
    \item $\bigcap_{k=1}^{n} A_k$ ;
    \item $\bigcup_{k=1}^{+\infty} A_k$ ;

    \item $\bigcap_{k=1}^{+\infty} A_k$ ;

\end{itemize}
\end{multicols}
\item Même question avec $A_n =\left [\frac{1}{n}, +\infty\right[$.\\
\end{enumerate}


\exo[2]{\textbf{Vocabulaire 2}}

Soit $(A_i)_{i \in \mathbb{N}}$ une suite d'évènements et $n \in \mathbb{N}^*$. Écrire, à l'aide des opérations ensemblistes $\cup$ et $\cap$ les évènements suivants :
\begin{multicols}{2}
\begin{enumerate}
    \item L'un des évènements $A_1, \ldots, A_n$ se réalise.
    \item L'un des évènements $A_i$ se réalise.
    \item Aucun des évènements $A_i$ ne se réalise.
    \item $A_1$ et $A_1$ seul se réalise.
    \item $A_1$ et $A_2$ et eux seuls se réalisent.
    \item (* ) Un et un seul des $A_i$ se réalise.
    \item (* ) À partir d'un certain rang, tous les $A_i$ se réalisent.
    \item (* ) Une infinité d'$A_i$ se réalise.
    \item (* ) Un nombre fin de $A_i$ se réalise.\\
\end{enumerate}
\end{multicols}



\exo[2]{\textbf{Vocabulaire 3}}

On tire successivement et avec remise des boules dans une urne contenant 1 rouge et 9 blanches. On note $R_n$ l'évènement « la $n$-ième boule tirée est rouge ». Traduire en langage courant les évènements suivants :\\
\begin{multicols}{3}
\begin{itemize}
    \item $\bigcup_{k=1}^{n} R_k$ ;
    \item $\bigcap_{k=1}^{n} R_k$ ;
    \item $\bigcup_{k=1}^{+\infty} R_k$ ;
    \item $\bigcup_{k=1}^{+\infty} \overline{R_k}$ ;
    \item $\bigcap_{k=1}^{+\infty} R_k$ ;
    \item $\bigcap_{k=5}^{+\infty} \overline{R_k}$ ;
    \item $\bigcup_{i=1}^{+\infty} \bigcap_{k=i}^{+\infty} R_k$ ;
    \item $\bigcap_{i=1}^{+\infty} \bigcup_{k=i}^{+\infty} R_k$.\\
\end{itemize}
\end{multicols}


\exo[2]{\textbf{Réunion disjointe}}

Deux joueurs A et B s'affrontent dans un jeu de dés. On lance deux dés équilibrés jusqu'à ce qu'une somme égale à 5 ou 7 apparaisse. Le joueur A gagne si la somme 5 apparaît et le joueur B gagne si c'est la somme 7 qui apparaît. On note $A_n$ l'évènement « le joueur A gagne au $n$-ième tour » et $B_n$ l'évènement « le joueur B gagne au $n$-ième tour ». On pose $G_n = A_n \cup B_n$. On note $A$ (resp. $B$) l'évènement « le joueur $A$ (resp. $B$) gagne ».
\begin{enumerate}
    \item Déterminer $\mathbb{P}(A_n)$ pour tout entier naturel $n$ non nul.
    \item En déduire $\mathbb{P}(A)$.
    \item Déterminer $\mathbb{P}(B)$.
    \item Le jeu s'arrête-t-il presque sûrement ?\\
\end{enumerate}



\exo[2]{\textbf{Limite monotone - 1}}

Une urne contient initialement 1 boule rouge et 1 boule blanche. On tire une boule, on note sa couleur, puis on la remet dans l'urne avec une boule de la même couleur. On répète cette opération à l'infini. On note, pour $n \geq 1$ quelconque :
\begin{itemize}
    \item $R_n$ l'évènement « la $n$-ième boule tirée est rouge » ;
    \item $A_n$ l'évènement « On ne tire aucune boule rouge au cours des $n$ premiers tirages. » ;
    \item $B$ l'évènement « On tire au moins une boule rouge lors de la suite infinie de tirages ».
\end{itemize}
\begin{enumerate}
    \item Déterminer, pour $n \geq 1$, $\mathbb{P}(A_n)$.
    \item En déduire $\mathbb{P}(B)$.\\
\end{enumerate}



\exo[2]{\textbf{Limite monotone - 2}}

Une urne contient initialement 1 boule rouge et 1 boule blanche. On tire une boule, on note sa couleur, puis on la remet dans l'urne avec deux boules de la même couleur. On répète cette opération à l'infini. On note, pour $n \geq 1$ quelconque :\\
\begin{itemize}
    \item $R_n$ l'évènement « la $n$-ième boule tirée est rouge » ;
    \item $A_n$ l'évènement « On ne tire aucune boule rouge au cours des $n$ premiers tirages. » ;
    \item $B$ l'évènement « On tire au moins une boule rouge lors de la suite infinie de tirages ».
\end{itemize}
\begin{enumerate}
    \item Montrer que pour tout $n \in \mathbb{N}^*$, $\mathbb{P}(A_n) = \prod_{k=1}^{n} \left(1 - \frac{1}{2k}\right)$.
    \item Justifier que la suite $(\ln(\mathbb{P}(A_n)))_{n \geq 1}$ diverge vers $-\infty$.
    \item En déduire $\mathbb{P}(B)$.\\
\end{enumerate}




\exo[2]{\textbf{Limite monotone - 3}}

Un joueur dispose de 2 euros et joue au jeu suivant : il lance une pièce de monnaie équilibrée et perd 1 euro s'il obtient Pile et gagne un bonbon s'il obtient Face. Il recommence ainsi indéfiniment jusqu'à avoir perdu ses 2 euros. Pour modéliser cette expérience, on considèrera qu'il répète en fait indéfiniment cette expérience mais que lorsqu'il n'a plus d'argent, à chaque lancer il ne perd plus d'argent ni ne gagne de bonbon.\\

Pour tout entier $k \geq 1$, on note :
\begin{itemize}
    \item $X_k$ la variable aléatoire donnant le nombre de Piles obtenus lors des $k$ premiers lancers.
    \item $A_k$ l'évènement « le joueur a encore 2 Euros à l'issue du $k$-ième lancer ».
    \item $B_k$ l'évènement « le joueur a encore exactement 1 Euro à l'issue du $k$-ième lancer ».
    \item $R_k$ l'évènement « le joueur a 0 euros (et est donc ruiné) à l'issue du $k$-ième lancer ».
    \item $R$ l'évènement « le joueur finit ruiné » autrement dit « le jeu finit par s'arrêter ».
\end{itemize}

\begin{enumerate}
    \item Soit $k \in \mathbb{N}^*$. Déterminer la loi de $X_k$.\\
     On précisera la valeur de $\mathbb{P}(X_k = i)$ pour $i \in X_k(\Omega)$.
    \item Soit $k \in \mathbb{N}^*$.
    \begin{enumerate}
        \item Déterminer $\mathbb{P}(A_k)$.
        \item Déterminer $\mathbb{P}(B_k)$.
        \item En déduire $\mathbb{P}(R_k)$.
    \end{enumerate}
    \item En déduire que le joueur finira presque sûrement ruiné.
\end{enumerate}




\exo[2]{\textbf{Joueur}}

Un joueur joue au jeu suivant à l'aide d'une pièce équilibrée.
\begin{itemize}
    \item À la première étape, il lance une fois la pièce. Si elle tombe sur Pile, il a gagné, sinon il passe à la seconde étape.
    \item À la seconde étape, il lance deux fois la pièce. S'il obtient deux Pile, il a gagné, sinon il passe à l'étape suivante.
    \item Il recommence ainsi de suite, à l'infini : À la $k$-ième étape, il lance $k$ fois la pièce. Si elle tombe $k$ fois sur Pile, il a gagné. Sinon il passe à l'étape suivante.
\end{itemize}

On note $A_n$ l'évènement « le joueur n'a toujours pas gagné à l'issue de la $n$-ième étape ».
\begin{enumerate}
    \item Montrer que pour tout $n \in \mathbb{N}^*$, 
    \[
    \mathbb{P}(A_n) = \prod_{k=1}^{n} \left(1 - \frac{1}{2^k}\right).
    \]
    \item Justifier que la suite $(\ln(\mathbb{P}(A_n)))_{n \geq 1}$ tend vers une limite finie $a$.
    \item En déduire que le joueur a une probabilité non nulle de ne jamais gagner.\\
\end{enumerate}

% \begin{small}
% \begin{Sol}
% \begin{enumerate}
%     \item
% Soit \( n \in \mathbb{N}^* \). On a les inclusions suivantes : 
% \[
% A_n \subset A_{n-1} \subset \dots \subset A_2 \subset A_1
% \]
% donc 
% \[
% A_n = A_1 \cap A_2 \cap \dots \cap A_n.
% \]
% Ainsi, d'après la formule des probabilités composées :
% \[
% \mathbb{P}(A_n) = \mathbb{P}(A_1 \cap A_2 \cap \dots \cap A_n) = \mathbb{P}(A_1) P(A_2 | A_1) \dots \mathbb{P}(A_n | A_1 \cap A_2 \cap \dots \cap A_{n-1}).
% \]
% Or, pour tout \( k \geq 2 \), si \( A_1 \cap A_2 \cap \dots \cap A_{k-1} \) est réalisé alors le joueur passe à la \( k \)-ième
% étape, il lance \( k \) fois la pièce et la probabilité qu'il gagne est \( \left(\frac{1}{2}\right)^k \), donc la probabilité qu'il
% ne gagne pas est \( 1 - \frac{1}{2^k} \). Dès lors,

% \[
% p_n = \mathbb{P}(A_n) = \frac{1}{2} \left( 1 - \frac{1}{2^2} \right) \dots \left( 1 - \frac{1}{2^n} \right)
% = \frac{2 - 1}{2} \times \frac{2^2 - 1}{2^2} \times \dots \times \frac{2^n - 1}{2^n}.
% \]

% \item
% Pour tout \( n \in \mathbb{N}^* \), on prend le logarithme :

% \[
% \ln \left( \frac{1}{p_n} \right) = \ln \left( \frac{2}{2 - 1} \times \frac{2^2}{2^2 - 1} \times \dots \times \frac{2^n}{2^n - 1} \right)
% \]

% \[
% = \sum_{k=1}^{n} \ln \left( \frac{2^k}{2^k - 1} \right)
% = \sum_{k=1}^{n} \ln \left( 1 + \frac{1}{2^k - 1} \right).
% \]

% Or, on sait que :

% \[
% \ln \left( 1 + \frac{1}{2^k - 1} \right) \sim_{k \to +\infty} \frac{1}{2^k}.
% \]

% La série de terme général \( \frac{1}{2^k} \) converge car \( 0 \leq \frac{1}{2} < 1 \), donc la série :

% \[
% \sum_{k=1}^{\infty} \ln \left( 1 + \frac{1}{2^k - 1} \right)
% \]

% converge vers une limite \( a \). On en déduit que :

% \[
% \lim_{n \to +\infty} \ln \left( \frac{1}{p_n} \right) = a.
% \]

% \item

% Pour tout \( n \in \mathbb{N}^* \) :

% \[
% \ln \left( \frac{1}{p_n} \right) = -\ln (p_n),
% \]

% d'où :

% \[
% p_n = \exp \left( - \ln \left( \frac{1}{p_n} \right) \right).
% \]

% Sachant que \( \lim_{n \to +\infty} \ln \left( \frac{1}{p_n} \right) = a \), on a :

% \[
% \lim_{n \to +\infty} -\ln \left( \frac{1}{p_n} \right) = -a.
% \]

% Par continuité de la fonction exponentielle sur \( \mathbb{R} \), on obtient :

% \[
% \lim_{n \to +\infty} p_n = e^{-a}.
% \]

% Ainsi, la suite \( (p_n) \) converge vers une limite \( \ell = e^{-a} > 0 \).

% \end{enumerate}
% \end{Sol}
% \end{small}

% Exercice 1758




\exo[3]{\textbf{Tribu engendrée}}

Soit $\Omega=\mathbb Z$. On considère $\mathcal T$ la tribu engendrée
par les ensembles $S_n=\{n,n+1,n+2\}$ avec $n\in\mathbb Z.$
Quels sont les éléments de la tribu $\mathcal T$?\\


% Exercice 2014




\exo[3]{\textbf{Tribu image réciproque}}

Soit $E$ et $F$ deux ensembles, $\mathcal T$ une tribu sur $F$ et $\phi:E\to F$ une application. Montrer que $\mathcal T'=\{\phi^{-1}(A);\ A\in\mathcal T\}$ est une tribu sur $E$.\\


% Exercice 1760




\exo[3]{\textbf{Tribu engendrée par une partition}}

Soit $X$ un ensemble non-vide et $A_1,\dots,A_n$ une partition de $X$.
On note 
$$\mathcal T=\left\{\bigcup_{i\in J}A_i;\ J\subset\{1,\dots,n\}\right\}.$$
Démontrer que $\mathcal T$ est la tribu engendrée par $A_1,\dots,A_n$.\\


% Exercice 2015


\exo[3]{\textbf{Tribu engendrée par une partition}}

Soit $E$ un ensemble infini et $(A_n)_{n\in\mathbb N}$ une partition de $E$. Pour toute partie $J$ de $\mathbb N$, on pose $B_J=\bigcup_{j\in J}A_j$. 
\begin{enumerate}
\item Démontrer que $\mathcal T=\{B_J;\ J\in\mathcal P(\mathbb N)\}$ est une tribu sur $E$ et que c'est la plus petite tribu contenant tous les $A_n$.
\item Trouver une partition $(A_n)_{n\in\mathbb N}$ de $\mathbb N$ de sorte que, pour tout $n\in\mathbb N$, $A_n$ n'est pas fini.
\item Trouver une tribu incluse dans $\mathcal P(\mathbb N)$, de cardinal infini, dont tous les éléments, sauf l'ensemble vide, sont de cardinal infini.\\
\end{enumerate}


% Exercice 1229



\exo[2]{\textbf{Sur la probabilité de l'intersection}}

Soient $A$ et $B$ deux événements d'un espace probabilisé. Démontrer que 
$$\max\big(0,\mathbb{P}(A)+\mathbb{P}(B)-1\big)\leq \mathbb{P}(A\cap B)\leq \min\big(\mathbb{P}(A),\mathbb{P}(B)\big).$$


% Exercice 3479



\exo[3]{\textbf{Majorer la probabilité d'une réunion}}

Soit $(\Omega,\mathcal A,\mathbb{P})$ un espace probabilisé, et $A_1,\dots,A_n$ des événements. Démontrer que 
$$\mathbb{P}\left(\bigcup_{i=1}^n A_i\right)\leq \min_{1\leq k\leq n}\left(\sum_{i=1}^n \mathbb{P}(A_i)-\sum_{\substack{1\leq i\leq n\\ i\neq k}}\mathbb{P}(A_i\cap A_k)\right).$$


% Exercice 1230



% \begin{Sol}



% Pour tout \( k \in \{1, \dots, n\} \), on a :
% \[
% \mathbb{P}\left( \bigcup_{i=1}^{n} A_i \right) \leq \sum_{i=1}^{n} \mathbb{P}(A_i) - \sum_{\substack{1 \leq i \leq n \\ i \neq k}} \mathbb{P}(A_i \cap A_k).
% \]

% Pour\( k = n \). Pour \( i = 1, \dots, n-1 \), on a :

% \[
% \mathbb{P}(A_i) - \mathbb{P}(A_i \cap A_n) = \mathbb{P}(A_i \setminus A_n),
% \]

% et donc :

% \[
% \sum_{i=1}^{n} \mathbb{P}(A_i) - \sum_{i=1}^{n-1} P(A_i \cap A_n) = \mathbb{P}(A_n) + \sum_{i=1}^{n-1} (\mathbb{P}(A_i) - \mathbb{P}(A_i \cap A_n)).
% \]

% Ce qui s'écrit aussi :

% \[
% \mathbb{P}(A_n) + \sum_{i=1}^{n-1} \mathbb{P}(A_i \setminus A_n) \geq \mathbb{P}\left(A_n \cup \bigcup_{i=1}^{n-1} (A_i \setminus A_n)\right).
% \]

% Or,

% \[
% A_n \cup \bigcup_{i=1}^{n-1} (A_i \setminus A_n) = \bigcup_{i=1}^{n} A_i.
% \]

% Ainsi,

% \[
% \mathbb{P}\left(A_n \cup \bigcup_{i=1}^{n-1} (A_i \setminus A_n)\right) = v\left(\bigcup_{i=1}^{n} A_i\right),
% \]


% \end{Sol}

\exo[2]{\textbf{Inégalité de Bonferroni}}
Soient $(\Omega,\mathcal F,\mathbb \mathbb{P})$ un espace probabilisé, et $A_1,\dots,A_n$
des événements. Démontrer que 
$$\mathbb \mathbb{P}(A_1\cap\dots\cap A_n)\geq \left(\sum_{i=1}^n \mathbb \mathbb{P}(A_i)\right)-(n-1).$$


% Exercice 3489



\exo[3]{\textbf{Somme compliquée}}

Soit $(\Omega,\mathcal T,\mathbb{P})$ un espace probabilisé.
\begin{enumerate}
\item Soit $A_1,\dots,A_n\in\mathcal T.$ Calculer 
$$\mathbb{P}(A_1\cup A_2)+\mathbb{P}(\overline{A_1}\cup A_2)+\mathbb{P}(A_1\cup \overline{A_2})+\mathbb{P}(\overline{A_1}\cup\overline{A_2}).$$
\item Soit $n\geq 2$ et $A_1,\dots,A_n\in\mathcal T.$ On pose 
$$\Gamma_n=\{A_1,\overline{A_1}\}\times\cdots\times \{A_n,\overline{A_n}\}.$$
Calculer 
$$\sum_{(B_1,\dots,B_n)\in\Gamma_n}\mathbb{P}(B_1\cup\cdots\cup B_n).$$
\end{enumerate}


% Exercice 1231

\exo[2]{\textbf{Limites supérieures et inférieures d'ensembles}}

Soit $\Omega$ un ensemble et $(A_n)_{n\in\mathbb N}$ une suite de parties de $\Omega$. On appelle limite supérieure des $A_n$, et on note $\limsup_n A_n$ l'ensemble des éléments de $\Omega$ qui appartiennent à une infinité de $A_n$.\\ 
On appelle limite inférieure des $A_n$, et on note $\liminf_n A_n$, l'ensemble des éléments de $\Omega$ qui appartiennent à tous les $A_n$, sauf un nombre fini d'entre eux.\\
\begin{enumerate}
 \item Déterminer les ensembles $\limsup_n A_n$ et $\liminf_n A_n$ dans les cas suivants :
\begin{enumerate}
\item $A_n=]-\infty,n]$;
\item $A_n=]-\infty,-n]$;
\item $A_{2n}=A$, $A_{2n+1}=B$;
\item $A_n=]-\infty,(-1)^n]$. 
\end{enumerate}
\item \'Ecrire les définitions de $\liminf_n A_n$ et $\limsup_n A_n$ avec les quantificateurs
$\forall$ et $\exists$. Les traduire en termes ensemblistes à l'aide de $\bigcap$ et $\bigcup$.\\
\end{enumerate}


% Exercice 1233


\exo[3]{\textbf{Premier lemme de Borel-Cantelli}}

Soit $(\Omega,\mathcal F,\mathbb \mathbb{P})$ un espace probabilisé. Soit $(A_n)_{n\geq 0}$ une suite d'événements.\\
On note $A=\limsup_n A_n=\bigcap_{n\geq 0}\bigcup_{k\geq n}A_k$. On suppose que $\sum_n \mathbb \mathbb{P}(A_n)<+\infty$. Pour $n\geq 1$, on note 
$D_n=\bigcup_{k=n}^{+\infty}A_k$.
\begin{enumerate}
 \item Démontrer que $\lim_{n\to+\infty}\mathbb \mathbb{P}(D_n)=0$;
\item En déduire que $\mathbb \mathbb{P}(A)=0$. Interpréter ce résultat.\\
\end{enumerate}

