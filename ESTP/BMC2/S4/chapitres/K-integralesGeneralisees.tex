
\thispagestyle{empty}

\section{Définitions et premières propriétés}

\subsection{Intégrales sur un intervalle non borné}

\begin{Def}
\textbf{Intégrale impropre sur $[a, +\infty[$}

Soit $f:[a,+\infty[\rightarrow \mathbb{R}$ une fonction continue par morceaux. On dit que l'intégrale :
$$\int_{a}^{+\infty} f(t) \, dt$$
est \textbf{convergente} si la fonction $F : x \mapsto \int_{a}^{x} f(t) \, dt$ admet une limite finie lorsque $x$ tend vers $+\infty$.

Dans ce cas, on note :
$$\int_{a}^{+\infty} f(t) \, dt = \lim_{x \to +\infty} \int_{a}^{x} f(t) \, dt$$

Une telle intégrale est appelée \textbf{intégrale généralisée} ou \textbf{intégrale impropre}.
\end{Def}

\begin{Ex}
Étudions la convergence de $\displaystyle\int_{1}^{+\infty} \frac{1}{t^2} \, dt$.

Pour $x > 1$, on a :
$$\int_{1}^{x} \frac{1}{t^2} \, dt = \left[-\frac{1}{t}\right]_{1}^{x} = -\frac{1}{x} + 1 = 1 - \frac{1}{x}$$

Quand $x \to +\infty$, cette expression tend vers $1$. Donc l'intégrale converge et :
$$\int_{1}^{+\infty} \frac{1}{t^2} \, dt = 1$$
\end{Ex}

\begin{Ex}
Étudions la convergence de $\displaystyle\int_{1}^{+\infty} \frac{1}{t} \, dt$.

Pour $x > 1$, on a :
$$\int_{1}^{x} \frac{1}{t} \, dt = \left[\ln t\right]_{1}^{x} = \ln x$$

Quand $x \to +\infty$, $\ln x \to +\infty$. L'intégrale \textbf{diverge}.
\end{Ex}

\begin{Rmq}$\,$

Lorsque l'intégrale ne converge pas, on dit qu'elle \textbf{diverge}. Elle peut diverger vers $+\infty$, $-\infty$, ou ne pas avoir de limite.
\end{Rmq}

\subsection{Intégrales au voisinage d'un point}

\begin{Def}
\textbf{Intégrale impropre sur $[a, b[$}

Soit $f:[a, b[\rightarrow \mathbb{R}$ une fonction continue par morceaux. On dit que l'intégrale :
$$\int_{a}^{b} f(t) \, dt$$
est \textbf{convergente} si la fonction $F : x \mapsto \int_{a}^{x} f(t) \, dt$ admet une limite finie lorsque $x$ tend vers $b^{-}$.

Dans ce cas, on note :
$$\int_{a}^{b} f(t) \, dt = \lim_{x \to b^{-}} \int_{a}^{x} f(t) \, dt$$
\end{Def}

\begin{Ex}
Étudions la convergence de $\displaystyle\int_{0}^{1} \frac{1}{\sqrt{t}} \, dt$.

La fonction $t \mapsto \frac{1}{\sqrt{t}}$ n'est pas définie en $0$. Pour $0 < \varepsilon < 1$, on a :
$$\int_{\varepsilon}^{1} \frac{1}{\sqrt{t}} \, dt = \left[2\sqrt{t}\right]_{\varepsilon}^{1} = 2 - 2\sqrt{\varepsilon}$$

Quand $\varepsilon \to 0^{+}$, cette expression tend vers $2$. Donc l'intégrale converge et :
$$\int_{0}^{1} \frac{1}{\sqrt{t}} \, dt = 2$$
\end{Ex}

\subsection{Intégrales sur un intervalle ouvert}

\begin{Def}
\textbf{Intégrale impropre sur $]a, b[$}

Soit $f:]a, b[\rightarrow \mathbb{R}$ une fonction continue par morceaux. On dit que l'intégrale :
$$\int_{a}^{b} f(t) \, dt$$
est \textbf{convergente} s'il existe $c \in ]a, b[$ tel que les deux intégrales $\displaystyle\int_{c}^{b} f(t) \, dt$ et $\displaystyle\int_{a}^{c} f(t) \, dt$ convergent.

Dans ce cas, on note :
$$\int_{a}^{b} f(t) \, dt = \lim_{x \rightarrow a^{+}} \int_{x}^{c} f(t) \, dt + \lim_{x \rightarrow b^{-}} \int_{c}^{x} f(t) \, dt$$

Cette valeur ne dépend pas du choix de $c$.
\end{Def}

\begin{Rmq}$\,$

Pour étudier la convergence d'une intégrale sur un intervalle $]a, b[$, on doit vérifier \textbf{séparément} la convergence aux deux bornes. L'intégrale converge si et seulement si elle converge en chaque borne.
\end{Rmq}

\subsection{Propriétés fondamentales}

Dans la suite, on considère $I = (a, b)$ un intervalle de $\mathbb{R}$ ouvert ou semi-ouvert et $f, g$ deux fonctions continues par morceaux sur cet intervalle.

\begin{Prop}
\textbf{Positivité}

Si $\displaystyle\int_{I} f$ converge et si $f \geq 0$ sur $I$, alors $\displaystyle\int_{I} f \geq 0$.
\end{Prop}

\begin{Prop}
\textbf{Linéarité}

Si $\displaystyle\int_{I} f$ et $\displaystyle\int_{I} g$ convergent, alors pour tout $\lambda \in \mathbb{R}$, $\displaystyle\int_{I}(f + \lambda g)$ converge et :
$$\int_{I}(f + \lambda g) = \int_{I} f + \lambda \int_{I} g$$
\end{Prop}

\begin{Prop}
\textbf{Relation de Chasles}

Si $\displaystyle\int_{I} f$ converge, alors pour tout $c \in ]a, b[$, les intégrales $\displaystyle\int_{a}^{c} f$ et $\displaystyle\int_{c}^{b} f$ convergent et on a :
$$\int_{a}^{b} f = \int_{a}^{c} f + \int_{c}^{b} f$$
\end{Prop}

\begin{Thm}
\textbf{Calcul par primitive}

Si $F$ est une primitive de $f$ sur $]a, b[$, alors $\displaystyle\int_{a}^{b} f$ converge si et seulement si $F$ admet des limites finies en $a^{+}$ et en $b^{-}$. Dans ce cas :
$$\int_{a}^{b} f(t) \, dt = \lim_{x \rightarrow b^{-}} F(x) - \lim_{x \rightarrow a^{+}} F(x)$$
\end{Thm}

\begin{Ex}
Calculons $\displaystyle\int_{0}^{+\infty} e^{-t} \, dt$.

Une primitive de $e^{-t}$ est $F(t) = -e^{-t}$. On a :
\begin{itemize}
\item $\displaystyle\lim_{t \to 0^{+}} F(t) = -1$
\item $\displaystyle\lim_{t \to +\infty} F(t) = 0$
\end{itemize}

Donc l'intégrale converge et :
$$\int_{0}^{+\infty} e^{-t} \, dt = 0 - (-1) = 1$$
\end{Ex}

\subsection{Intégrales de référence}

\begin{Thm}
\textbf{Intégrales de Riemann}

Les intégrales de référence suivantes sont fondamentales :

\begin{enumerate}
\item \textbf{En $+\infty$ (exponentielle) :} $\displaystyle\int_{0}^{+\infty} e^{-\alpha t} \, dt$ converge $\Leftrightarrow \alpha > 0$

\item \textbf{En $+\infty$ (puissance) :} $\displaystyle\int_{1}^{+\infty} \frac{dt}{t^{\alpha}}$ converge $\Leftrightarrow \alpha > 1$

\item \textbf{En un point fini :} $\displaystyle\int_{a}^{b} \frac{dt}{(t-a)^{\alpha}}$ converge $\Leftrightarrow \alpha < 1$
\end{enumerate}
\end{Thm}

\begin{Rmq}$\,$

Attention, $\displaystyle\int_{0}^{+\infty} \frac{dt}{t^{\alpha}}$ n'est \textbf{jamais} convergente, quelle que soit la valeur de $\alpha$. En effet, si $\alpha \leq 1$, l'intégrale diverge en $+\infty$, et si $\alpha > 1$, elle diverge en $0$.
\end{Rmq}

\begin{Ex}
\textbf{Calcul de l'intégrale de Riemann en $+\infty$}

Pour $\alpha \neq 1$ et $x > 1$ :
$$\int_{1}^{x} \frac{dt}{t^{\alpha}} = \left[\frac{t^{1-\alpha}}{1-\alpha}\right]_{1}^{x} = \frac{x^{1-\alpha} - 1}{1-\alpha}$$

\begin{itemize}
\item Si $\alpha > 1$ : $1-\alpha < 0$, donc $x^{1-\alpha} \to 0$ quand $x \to +\infty$, et l'intégrale converge vers $\dfrac{1}{\alpha - 1}$.
\item Si $\alpha < 1$ : $1-\alpha > 0$, donc $x^{1-\alpha} \to +\infty$, et l'intégrale diverge.
\end{itemize}
\end{Ex}

\begin{Meth}
Pour étudier une intégrale généralisée :
\begin{enumerate}
\item Identifier les points où la fonction n'est pas définie ou les bornes infinies.
\item Découper l'intégrale si nécessaire (relation de Chasles).
\item Essayer de calculer explicitement une primitive.
\item Si le calcul direct est impossible, utiliser les théorèmes de comparaison.
\end{enumerate}
\end{Meth}

\subsubsection*{QCM}

\begin{enumerate}
\item L'intégrale $\displaystyle\int_{1}^{+\infty} \frac{dt}{t^2}$ :
\begin{enumerate}
\item diverge
\item converge vers $1$
\item converge vers $2$
\item converge vers $\frac{1}{2}$
\end{enumerate}

\item L'intégrale $\displaystyle\int_{0}^{1} \frac{dt}{\sqrt{t}}$ :
\begin{enumerate}
\item diverge car $\frac{1}{\sqrt{t}} \to +\infty$ quand $t \to 0$
\item converge vers $2$
\item converge vers $1$
\item n'est pas définie
\end{enumerate}

\item L'intégrale de Riemann $\displaystyle\int_{1}^{+\infty} \frac{dt}{t^{\alpha}}$ converge si et seulement si :
\begin{enumerate}
\item $\alpha > 0$
\item $\alpha < 1$
\item $\alpha > 1$
\item $\alpha = 1$
\end{enumerate}

\item L'intégrale $\displaystyle\int_{0}^{+\infty} e^{-2t} \, dt$ vaut :
\begin{enumerate}
\item $2$
\item $1$
\item $\frac{1}{2}$
\item diverge
\end{enumerate}
\end{enumerate}

\section{Fonctions intégrables}

\begin{Def}
\textbf{Intégrabilité}

Une fonction $f$ est dite \textbf{intégrable} sur $I$, ou l'intégrale $\displaystyle\int_{I} f$ est dite \textbf{absolument convergente}, si $\displaystyle\int_{I}|f|$ converge.
\end{Def}

\begin{Thm}
\textbf{Convergence absolue implique convergence}

Si $f$ est intégrable sur $I$, alors $\displaystyle\int_{I} f(t) \, dt$ converge.
\end{Thm}

\begin{Rmq}$\,$

La réciproque est fausse ! Une intégrale peut converger sans converger absolument. On parle alors de \textbf{convergence semi-convergente}.
\end{Rmq}

\begin{Ex}
L'intégrale $\displaystyle\int_{1}^{+\infty} \frac{\sin t}{t} \, dt$ converge (semi-convergence), mais $\displaystyle\int_{1}^{+\infty} \frac{|\sin t|}{t} \, dt$ diverge.

La fonction $\frac{\sin t}{t}$ n'est donc pas intégrable sur $[1, +\infty[$.
\end{Ex}

\begin{Prop}
\textbf{Inégalité triangulaire}

Si $f$ et $g$ sont intégrables sur $I$, alors $f + g$ est intégrable sur $I$ et on a :
$$\left|\int_{I}(f + g)\right| \leq \int_{I}|f + g| \leq \int_{I}|f| + \int_{I}|g|$$
\end{Prop}

\begin{Prop}
\textbf{Fonction continue intégrable}

Si $f$ est continue et intégrable sur $I$, alors :
$$\int_{I}|f(t)| \, dt = 0 \Rightarrow f \equiv 0$$
\end{Prop}

\begin{Thm}
\textbf{Critère de convergence pour les fonctions positives}

Si $f$ est positive sur $[a, b[$, alors $\displaystyle\int_{a}^{b} f$ converge si et seulement si la fonction $x \mapsto \int_{a}^{x} f(t) \, dt$ est majorée sur $]a, b[$.
\end{Thm}

\begin{Ex}
Montrons que $\displaystyle\int_{0}^{+\infty} e^{-t^2} \, dt$ converge.

Pour $t \geq 1$, on a $t^2 \geq t$, donc $e^{-t^2} \leq e^{-t}$.

Comme $\displaystyle\int_{1}^{+\infty} e^{-t} \, dt$ converge (vers $e^{-1}$), et $e^{-t^2} \geq 0$, on en déduit que $\displaystyle\int_{1}^{+\infty} e^{-t^2} \, dt$ converge.

Sur $[0, 1]$, $e^{-t^2}$ est continue bornée, donc $\displaystyle\int_{0}^{1} e^{-t^2} \, dt$ converge.

Par relation de Chasles, $\displaystyle\int_{0}^{+\infty} e^{-t^2} \, dt$ converge.
\end{Ex}

\subsubsection*{QCM}

\begin{enumerate}
\item Une fonction intégrable sur $I$ est une fonction telle que :
\begin{enumerate}
\item $\displaystyle\int_{I} f$ converge
\item $\displaystyle\int_{I} |f|$ converge
\item $f$ est bornée
\item $f$ est continue
\end{enumerate}

\item Si $\displaystyle\int_{I} f$ converge mais $\displaystyle\int_{I} |f|$ diverge, on dit que l'intégrale est :
\begin{enumerate}
\item absolument convergente
\item divergente
\item semi-convergente
\item indéterminée
\end{enumerate}

\item L'intégrale $\displaystyle\int_{1}^{+\infty} \frac{\cos t}{t^2} \, dt$ :
\begin{enumerate}
\item diverge
\item converge absolument
\item est semi-convergente
\item n'existe pas
\end{enumerate}
\end{enumerate}

\section{Intégrabilité et comparaison}

\subsection{Théorèmes de comparaison}

\begin{Thm}
\textbf{Comparaison de fonctions positives}

Soient $f$ et $g$ deux fonctions continues par morceaux sur $[a, b[$ telles que $0 \leq f \leq g$.

Si $\displaystyle\int_{a}^{b} g$ converge, alors $\displaystyle\int_{a}^{b} f$ converge.

Contraposée : Si $\displaystyle\int_{a}^{b} f$ diverge, alors $\displaystyle\int_{a}^{b} g$ diverge.
\end{Thm}

\begin{Ex}
Étudions $\displaystyle\int_{1}^{+\infty} \frac{\ln t}{t^2} \, dt$.

Pour $t \geq 1$, on a $\ln t \leq t$, donc $\dfrac{\ln t}{t^2} \leq \dfrac{1}{t}$.

Mais cette majoration ne suffit pas car $\displaystyle\int_{1}^{+\infty} \frac{dt}{t}$ diverge.

Pour $t$ assez grand, $\ln t \leq \sqrt{t}$, donc $\dfrac{\ln t}{t^2} \leq \dfrac{1}{t^{3/2}}$.

Comme $\dfrac{3}{2} > 1$, $\displaystyle\int_{1}^{+\infty} \frac{dt}{t^{3/2}}$ converge, donc $\displaystyle\int_{1}^{+\infty} \frac{\ln t}{t^2} \, dt$ converge.
\end{Ex}

\begin{Thm}
\textbf{Comparaison par équivalents}

Soient $f$ et $g$ deux fonctions continues par morceaux sur $[a, b[$ telles que $g$ garde un signe constant au voisinage de $b$ et $f(x) \sim_{b} g(x)$.

Alors $\displaystyle\int_{a}^{b} f$ et $\displaystyle\int_{a}^{b} g$ sont de même nature (toutes deux convergentes ou toutes deux divergentes).
\end{Thm}

\begin{Ex}
Étudions $\displaystyle\int_{0}^{1} \frac{1}{1 - \cos t} \, dt$.

Au voisinage de $0$ : $1 - \cos t \sim \dfrac{t^2}{2}$, donc $\dfrac{1}{1 - \cos t} \sim \dfrac{2}{t^2}$.

Comme $\displaystyle\int_{0}^{1} \frac{dt}{t^2}$ diverge (exposant $2 \geq 1$), l'intégrale $\displaystyle\int_{0}^{1} \frac{dt}{1 - \cos t}$ diverge.
\end{Ex}

\subsection{Critères de Riemann}

\begin{Thm}
\textbf{Critères de Riemann en $+\infty$}

Soit $f$ une fonction continue par morceaux sur $[a, +\infty[$.

\begin{enumerate}
\item S'il existe $\alpha > 1$ tel que $t^{\alpha} f(t) \to 0$ quand $t \to +\infty$, alors $f$ est intégrable sur $[a, +\infty[$.

\item S'il existe $c > 0$ tel que $t f(t) \geq c$ quand $t \to +\infty$, alors $\displaystyle\int_{a}^{+\infty} f(t) \, dt$ diverge.
\end{enumerate}
\end{Thm}

\begin{Thm}
\textbf{Critères de Riemann en un point}

Soit $f$ une fonction continue par morceaux sur $]a, b]$.

\begin{enumerate}
\item S'il existe $\alpha < 1$ tel que $(t-a)^{\alpha} f(t) \to 0$ quand $t \to a^{+}$, alors $f$ est intégrable sur $]a, b]$.

\item S'il existe $c > 0$ tel que $(t-a) f(t) \geq c$ quand $t \to a^{+}$, alors $\displaystyle\int_{a}^{b} f(t) \, dt$ diverge.
\end{enumerate}
\end{Thm}

\begin{Ex}
Étudions $\displaystyle\int_{2}^{+\infty} \frac{dt}{t(\ln t)^2}$.

Calculons $t^{\alpha} \cdot \dfrac{1}{t(\ln t)^2} = \dfrac{t^{\alpha - 1}}{(\ln t)^2}$.

Pour $\alpha = 1$ : $\dfrac{1}{(\ln t)^2} \to 0$ quand $t \to +\infty$.

Ce n'est pas suffisant (il faut $\alpha > 1$).

Pour tout $\alpha > 1$ : $\dfrac{t^{\alpha - 1}}{(\ln t)^2} \to +\infty$ car $t^{\alpha - 1}$ croît plus vite que $(\ln t)^2$.

Le critère de Riemann ne s'applique pas directement. Utilisons un changement de variable $u = \ln t$ :
$$\int_{2}^{x} \frac{dt}{t(\ln t)^2} = \int_{\ln 2}^{\ln x} \frac{du}{u^2} = \left[-\frac{1}{u}\right]_{\ln 2}^{\ln x} = \frac{1}{\ln 2} - \frac{1}{\ln x} \to \frac{1}{\ln 2}$$

L'intégrale converge.
\end{Ex}

\begin{Meth}
\textbf{Pour utiliser les critères de Riemann :}
\begin{itemize}
\item En $+\infty$ : chercher un $\alpha > 1$ tel que $t^{\alpha} |f(t)| \to 0$.
\item En un point $a$ : chercher un $\alpha < 1$ tel que $(t-a)^{\alpha} |f(t)| \to 0$.
\item Si on trouve le bon $\alpha$, l'intégrale converge absolument.
\end{itemize}
\end{Meth}

\subsubsection*{QCM}

\begin{enumerate}
\item Pour montrer que $\displaystyle\int_{1}^{+\infty} f(t) \, dt$ converge avec $f \geq 0$, on peut :
\begin{enumerate}
\item trouver $g \leq f$ avec $\int g$ convergente
\item trouver $g \geq f$ avec $\int g$ convergente
\item trouver $g \sim f$ avec $\int g$ divergente
\item montrer que $f$ est bornée
\end{enumerate}

\item L'intégrale $\displaystyle\int_{1}^{+\infty} \frac{dt}{t^2 + 1}$ :
\begin{enumerate}
\item diverge
\item converge car $\frac{1}{t^2+1} \leq \frac{1}{t^2}$ et $\int \frac{dt}{t^2}$ converge
\item converge car $\frac{1}{t^2+1} \sim \frac{1}{t^2}$
\item les réponses b) et c) sont correctes
\end{enumerate}

\item Pour $\displaystyle\int_{0}^{1} \frac{dt}{t^{\alpha}}$, la condition de convergence est :
\begin{enumerate}
\item $\alpha > 1$
\item $\alpha < 1$
\item $\alpha = 1$
\item $\alpha \leq 0$
\end{enumerate}

\item L'intégrale $\displaystyle\int_{0}^{1} \frac{dt}{\sqrt{1-t}}$ :
\begin{enumerate}
\item diverge car $\frac{1}{\sqrt{1-t}} \to +\infty$
\item converge car $\frac{1}{\sqrt{1-t}} = \frac{1}{(1-t)^{1/2}}$ avec $\frac{1}{2} < 1$
\item converge vers $1$
\item converge vers $2$
\end{enumerate}
\end{enumerate}

\section{Techniques de calcul}

\subsection{Intégration par parties}

\begin{Thm}
\textbf{Intégration par parties généralisée}

Soient $u$ et $v$ deux fonctions de classe $\mathcal{C}^1$ sur $]a, b[$. Si deux des trois quantités suivantes existent (sont finies) :
\begin{itemize}
\item $\displaystyle\int_{a}^{b} u'(t) v(t) \, dt$
\item $\displaystyle\int_{a}^{b} u(t) v'(t) \, dt$
\item $\displaystyle\lim_{t \to a^{+}} [u(t)v(t)] - \lim_{t \to b^{-}} [u(t)v(t)]$
\end{itemize}
alors la troisième existe aussi et on a :
$$\int_{a}^{b} u'(t) v(t) \, dt = \left[u(t)v(t)\right]_{a}^{b} - \int_{a}^{b} u(t) v'(t) \, dt$$
\end{Thm}

\begin{Ex}
Calculons $\displaystyle\int_{0}^{+\infty} t e^{-t} \, dt$.

Posons $u'(t) = e^{-t}$ et $v(t) = t$, donc $u(t) = -e^{-t}$ et $v'(t) = 1$.

On a $[u(t)v(t)]_{0}^{+\infty} = \left[-te^{-t}\right]_{0}^{+\infty} = 0 - 0 = 0$.

Donc :
$$\int_{0}^{+\infty} t e^{-t} \, dt = 0 - \int_{0}^{+\infty} (-e^{-t}) \, dt = \int_{0}^{+\infty} e^{-t} \, dt = 1$$
\end{Ex}

\subsection{Changement de variable}

\begin{Thm}
\textbf{Changement de variable}

Soit $\phi : [\alpha, \beta] \to [a, b]$ une fonction de classe $\mathcal{C}^1$, bijective et strictement monotone. Soit $f$ une fonction continue par morceaux sur $[a, b]$.

Si $\displaystyle\int_{a}^{b} f(t) \, dt$ converge, alors $\displaystyle\int_{\alpha}^{\beta} f(\phi(u)) \phi'(u) \, du$ converge et :
$$\int_{a}^{b} f(t) \, dt = \int_{\alpha}^{\beta} f(\phi(u)) \phi'(u) \, du$$
\end{Thm}

\begin{Ex}
Calculons $\displaystyle\int_{0}^{+\infty} \frac{\ln t}{1 + t^2} \, dt$.

Décomposons : $\displaystyle\int_{0}^{+\infty} = \int_{0}^{1} + \int_{1}^{+\infty}$.

Pour $\displaystyle\int_{1}^{+\infty} \frac{\ln t}{1 + t^2} \, dt$, posons $u = \frac{1}{t}$, donc $t = \frac{1}{u}$ et $dt = -\frac{du}{u^2}$.

Quand $t \to 1^{+}$, $u \to 1^{-}$ et quand $t \to +\infty$, $u \to 0^{+}$.

$$\int_{1}^{+\infty} \frac{\ln t}{1 + t^2} \, dt = \int_{1}^{0} \frac{\ln(1/u)}{1 + 1/u^2} \cdot \left(-\frac{1}{u^2}\right) \, du = \int_{0}^{1} \frac{-\ln u}{u^2 + 1} \, du = -\int_{0}^{1} \frac{\ln u}{1 + u^2} \, du$$

Donc :
$$\int_{0}^{+\infty} \frac{\ln t}{1 + t^2} \, dt = \int_{0}^{1} \frac{\ln t}{1 + t^2} \, dt - \int_{0}^{1} \frac{\ln t}{1 + t^2} \, dt = 0$$
\end{Ex}

\section{Exercices}

\vspace{1em}
\hrule
\vspace{1em}

\exo[1]{Convergence simple}

Les intégrales impropres suivantes sont-elles convergentes ? Justifier.

\begin{enumerate}
\item $\displaystyle\int_{0}^{1} \ln t \, dt$
\item $\displaystyle\int_{0}^{+\infty} e^{-t^{2}} \, dt$
\item $\displaystyle\int_{0}^{+\infty} x(\sin x) e^{-x} \, dx$
\item $\displaystyle\int_{0}^{+\infty}(\ln t) e^{-t} \, dt$
\item $\displaystyle\int_{0}^{1} \frac{dt}{(1-t) \sqrt{t}}$
\end{enumerate}

\vspace{1em}
\hrule
\vspace{1em}

\exo[1]{Nature d'intégrales}

Les intégrales impropres suivantes sont-elles convergentes ?

\begin{enumerate}
\item $\displaystyle\int_{0}^{+\infty} \frac{dt}{e^{t}-1}$
\item $\displaystyle\int_{0}^{+\infty} \frac{t e^{-\sqrt{t}}}{1+t^{2}} \, dt$
\item $\displaystyle\int_{0}^{1} \cos ^{2}\left(\frac{1}{t}\right) \, dt$
\end{enumerate}

\vspace{1em}
\hrule
\vspace{1em}

\exo[2]{Convergence et calculs}

Les intégrales impropres suivantes sont-elles convergentes ?

\begin{enumerate}
\item $\displaystyle\int_{0}^{+\infty} \frac{\ln t}{t^{2}+1} \, dt$
\item $\displaystyle\int_{1}^{+\infty} \frac{\sqrt{\ln t}}{(t-1) \sqrt{t}} \, dt$
\item $\displaystyle\int_{1}^{+\infty} e^{-\sqrt{\ln t}} \, dt$
\end{enumerate}

\vspace{1em}
\hrule
\vspace{1em}

\exo[2]{Intégrales de Bertrand}

On s'intéresse aux intégrales de Bertrand :
$$\int_{e}^{+\infty} \frac{dx}{x^{\alpha}(\ln x)^{\beta}}$$
On souhaite déterminer sa nature en fonction de $\alpha, \beta$ réels.

\begin{enumerate}
\item On suppose $\alpha > 1$. En comparant avec une intégrale de Riemann, montrer que l'intégrale converge.
\item On suppose $\alpha = 1$. Calculer pour $X > e$ :
$$\int_{e}^{X} \frac{dx}{x(\ln x)^{\beta}}$$
En déduire les valeurs de $\beta$ pour lesquelles l'intégrale converge.
\item On suppose $\alpha < 1$. En comparant à $\frac{1}{x}$, démontrer que l'intégrale diverge.
\end{enumerate}

\vspace{1em}
\hrule
\vspace{1em}

\exo[2]{Discussion selon un paramètre}

Discuter selon les valeurs de $\alpha$ réel, la convergence des intégrales suivantes :

\begin{enumerate}
\item $\displaystyle\int_{0}^{+\infty} \frac{t \ln t}{(1+t^{2})^{\alpha}} \, dt$
\item $\displaystyle\int_{0}^{+\infty} x^{\alpha} \ln \left(x+e^{\alpha x}\right) \, dx$
\end{enumerate}

\vspace{1em}
\hrule
\vspace{1em}

\exo[2]{Calcul par changement de variable}

\begin{enumerate}
\item Montrer que $\displaystyle\int_{0}^{+\infty} \frac{\ln t}{1+t^{2}} \, dt$ converge.
\item Avec le changement de variable $u = 1/t$, montrer que :
$$\int_{0}^{+\infty} \frac{\ln t}{1+t^{2}} \, dt = 0$$
\item Soit $\alpha > 0$, calculer :
$$\int_{0}^{+\infty} \frac{\ln t}{\alpha^{2}+t^{2}} \, dt$$
\end{enumerate}

\vspace{1em}
\hrule
\vspace{1em}

\exo[2]{Calculs d'intégrales}

Justifier la convergence et calculer la valeur des intégrales suivantes :

\begin{enumerate}
\item $\displaystyle\int_{0}^{1} \frac{\ln t}{\sqrt{1-t}} \, dt$
\item $\displaystyle\int_{0}^{+\infty} t e^{-\sqrt{t}} \, dt$
\end{enumerate}

\vspace{1em}
\hrule
\vspace{1em}

\exo[1]{Intégrale et paramètre}

Pour quelles valeurs de $\alpha \in \mathbb{R}$, l'intégrale suivante converge-t-elle ?
$$\int_{0}^{1} \frac{1 - \cos t}{t^{\alpha}} \, dt$$

\vspace{1em}
\hrule
\vspace{1em}

\exo[1]{Équivalents et convergence}

Déterminer un équivalent simple au voisinage de la borne problématique et en déduire la nature de l'intégrale :

\begin{enumerate}
\item $\displaystyle\int_{0}^{1} \frac{dt}{\sqrt{t(1-t)}}$
\item $\displaystyle\int_{1}^{+\infty} \frac{dt}{\sqrt{t^3 - 1}}$
\item $\displaystyle\int_{0}^{+\infty} \frac{t^2}{(1+t)^4} \, dt$
\end{enumerate}

\vspace{1em}
\hrule
\vspace{1em}

\exo[2]{Intégration par parties}

À l'aide d'une intégration par parties, calculer les intégrales suivantes :

\begin{enumerate}
\item $\displaystyle\int_{0}^{+\infty} t^2 e^{-t} \, dt$
\item $\displaystyle\int_{0}^{1} t \ln t \, dt$
\item $\displaystyle\int_{1}^{+\infty} \frac{\ln t}{t^2} \, dt$
\end{enumerate}

\vspace{1em}
\hrule
\vspace{1em}

\exo[3]{Intégrale de Gauss}

On admet que $\displaystyle\int_{0}^{+\infty} e^{-t^2} \, dt = \frac{\sqrt{\pi}}{2}$.

\begin{enumerate}
\item Calculer $\displaystyle\int_{0}^{+\infty} e^{-at^2} \, dt$ pour $a > 0$.
\item À l'aide d'une intégration par parties, calculer $\displaystyle\int_{0}^{+\infty} t^2 e^{-t^2} \, dt$.
\item Plus généralement, pour $n \in \mathbb{N}$, exprimer $I_n = \displaystyle\int_{0}^{+\infty} t^{2n} e^{-t^2} \, dt$ en fonction de $I_{n-1}$.
\end{enumerate}

\vspace{1em}
\hrule
\vspace{1em}

\exo[3]{Fonction Gamma}

On définit la fonction Gamma par :
$$\Gamma(x) = \int_{0}^{+\infty} t^{x-1} e^{-t} \, dt$$

\begin{enumerate}
\item Montrer que cette intégrale converge pour tout $x > 0$.
\item À l'aide d'une intégration par parties, montrer que pour tout $x > 0$ :
$$\Gamma(x+1) = x \Gamma(x)$$
\item Calculer $\Gamma(1)$ et en déduire que $\Gamma(n+1) = n!$ pour tout $n \in \mathbb{N}$.
\item Montrer que $\Gamma(1/2) = \sqrt{\pi}$.
\end{enumerate}

\vspace{1em}
\hrule
\vspace{1em}

\exo[2]{Convergence et divergence}

Pour chacune des intégrales suivantes, déterminer si elle converge ou diverge :

\begin{enumerate}
\item $\displaystyle\int_{2}^{+\infty} \frac{dt}{t \ln t}$
\item $\displaystyle\int_{2}^{+\infty} \frac{dt}{t (\ln t)^2}$
\item $\displaystyle\int_{0}^{1} \frac{e^t - 1}{t^2} \, dt$
\item $\displaystyle\int_{0}^{+\infty} \frac{\sin^2 t}{t^2} \, dt$
\end{enumerate}

\vspace{1em}
\hrule
\vspace{1em}

\exo[1]{Calcul direct}

Calculer les intégrales suivantes (après avoir justifié leur convergence) :

\begin{enumerate}
\item $\displaystyle\int_{1}^{+\infty} \frac{dt}{t(t+1)}$
\item $\displaystyle\int_{0}^{+\infty} \frac{dt}{(1+t)(1+t^2)}$
\item $\displaystyle\int_{0}^{1} \frac{dt}{\sqrt{t}(1+t)}$
\end{enumerate}

\vspace{1em}
\hrule
\vspace{1em}

\exo[2]{Intégrale dépendant d'un paramètre}

Soit $a > 0$. On pose :
$$F(a) = \int_{0}^{+\infty} \frac{e^{-at} - e^{-t}}{t} \, dt$$

\begin{enumerate}
\item Montrer que cette intégrale est bien définie.
\item À l'aide du changement de variable $u = at$, montrer que $F(a) = \ln a$.
\end{enumerate}

\vspace{1em}
\hrule
\vspace{1em}

\exo[3]{Intégrale de Dirichlet}

On pose $I = \displaystyle\int_{0}^{+\infty} \frac{\sin t}{t} \, dt$.

\begin{enumerate}
\item Montrer que cette intégrale converge (semi-convergence).
\item On admet que $I = \frac{\pi}{2}$. En déduire la valeur de $\displaystyle\int_{0}^{+\infty} \frac{\sin^2 t}{t^2} \, dt$.

\textit{Indication : utiliser l'identité $\sin^2 t = \frac{1 - \cos(2t)}{2}$ et une intégration par parties.}
\end{enumerate}

