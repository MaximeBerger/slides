
\section{Réduction d'endomorphisme}

\textit{Toutes les définitions et tous les résultats sont énoncés pour un espace vectoriel sur le corps des réels, sauf mention du contraire. Tout reste vrai pour n'importe quel corps.}

\subsection{Valeurs propres et vecteurs propres}

\begin{Def}\textbf{Valeur propre et vecteur propre}\\

Soit $E$ un $K$-espace vectoriel et $f : E \to E$ un endomorphisme.
\begin{itemize}
    \item Un vecteur $v \in E$ est un \textbf{vecteur propre} de $f$ si $v \neq 0_E$ et $f(v)$ est colinéaire à $v$ :
    $$v \neq 0_E \quad \text{et} \quad \exists \lambda \in K, \, f(v) = \lambda v$$
    \item Un scalaire $\lambda \in K$ est une \textbf{valeur propre} de $f$ s'il existe un vecteur $v \in E$ tel que $v \neq 0_E$ et $f(v) = \lambda v$.
    \item L'ensemble des valeurs propres de $f$ est appelé le \textbf{spectre} de $f$ et est noté $\mathrm{Sp}(f)$.
\end{itemize}
\end{Def}

\begin{Rmq}$\,$

Si $v \neq 0_E$ et $f(v) = \lambda v$, alors $v$ est un \textbf{vecteur propre} associé à la \textbf{valeur propre} $\lambda$.
\end{Rmq}

\begin{Ex}\textbf{Exemples de valeurs propres}\\

\begin{itemize}
    \item Pour l'identité $\mathrm{Id}_E$, tout vecteur non nul est vecteur propre pour la valeur propre $1$.
    \item Pour une homothétie de rapport $k$, tout vecteur non nul est vecteur propre pour la valeur propre $k$.
    \item Pour une symétrie par rapport à un sous-espace $F$, les vecteurs propres sont les vecteurs de $F$ (valeur propre $1$) et les vecteurs de $F^\perp$ (valeur propre $-1$).
\end{itemize}
\end{Ex}

\begin{Prop}\textbf{Caractérisation des valeurs propres}\\

$$\lambda \text{ valeur propre de } f \Longleftrightarrow \ker(f - \lambda \mathrm{Id}) \neq \{0_E\} \Longleftrightarrow f - \lambda \mathrm{Id} \text{ n'est pas injectif}$$
\end{Prop}

\begin{Rmq}$\,$

Si $v$ est vecteur propre de $f$ pour la valeur propre $\lambda$, alors pour tout polynôme $P \in K[X]$, on a $P(f)(v) = P(\lambda) v$. Ainsi $P(\lambda)$ est valeur propre de $P(f)$ et $v$ est vecteur propre associé.
\end{Rmq}

\begin{Thm}\textbf{Indépendance des vecteurs propres}\\

Soient $\lambda_1, \ldots, \lambda_p$ des valeurs propres distinctes de $f$. Si $v_i$ est un vecteur propre pour la valeur propre $\lambda_i$, alors les vecteurs $v_1, \ldots, v_p$ sont linéairement indépendants.
\end{Thm}

\begin{Rmq}
\textbf{Démonstration :}
Par récurrence sur le nombre $p$ de vecteurs propres.

\textbf{Initialisation :} Pour $p = 1$, $(v_1)$ est libre car $v_1 \neq 0_E$.

\textbf{Hérédité :} Supposons $(v_1, \ldots, v_{p-1})$ libre. Si $\alpha_1 v_1 + \cdots + \alpha_p v_p = 0_E$, en appliquant $f$ :
$$\alpha_1 \lambda_1 v_1 + \cdots + \alpha_p \lambda_p v_p = 0_E$$
En soustrayant $\lambda_p$ fois l'égalité initiale :
$$\alpha_1 (\lambda_1 - \lambda_p) v_1 + \cdots + \alpha_{p-1} (\lambda_{p-1} - \lambda_p) v_{p-1} = 0_E$$
Par hypothèse de récurrence, $\alpha_i (\lambda_i - \lambda_p) = 0$ pour $i < p$. Comme $\lambda_i \neq \lambda_p$, on a $\alpha_i = 0$. Puis $\alpha_p v_p = 0_E$ donne $\alpha_p = 0$.
\end{Rmq}


\vspace{1em}
\hrule
\vspace{1em}

\exo[1]{Application directe}

Soit $f$ l'endomorphisme de $\mathbb{R}^2$ ayant pour matrice dans la base canonique :
$$A = \begin{pmatrix}
1 & 1 \\
0 & 2
\end{pmatrix}$$
\begin{enumerate}
    \item Trouver les valeurs propres de $f$.
    \item Déterminer une base de vecteurs propres associés.
    \item Écrire la matrice de $f$ dans cette nouvelle base.
\end{enumerate}

\vspace{1em}
\hrule
\vspace{1em}

\exo[1]{Valeurs propres complexes}

Soit $g$ l'endomorphisme de $\mathbb{R}^2$ ayant pour matrice dans la base canonique :
$$B = \begin{pmatrix}
0 & -1 \\
1 & 0
\end{pmatrix}$$
\begin{enumerate}
    \item L'endomorphisme $g$ admet-il des valeurs propres réelles ?
    \item Quelles sont les valeurs propres de $g$ dans $\mathbb{C}$ ?
    \item Interpréter géométriquement l'endomorphisme $g$.
\end{enumerate}

\vspace{1em}
\hrule
\vspace{1em}

\exo[1]{Un spectre égal à $\mathbb{R}$}

L'ensemble $C^\infty(\mathbb{R}, \mathbb{R})$ des fonctions infiniment dérivables est un $\mathbb{R}$-espace vectoriel de dimension infinie. La dérivation $D : f \mapsto f'$ est un endomorphisme.
\begin{enumerate}
    \item Montrer que chaque réel $\lambda$ est une valeur propre de $D$ en exhibant un vecteur propre associé.
    \item En déduire que $\mathrm{Sp}(D) = \mathbb{R}$.
\end{enumerate}

\vspace{1em}
\hrule
\vspace{1em}

\exo[1]{Un spectre vide}

Montrer que l'endomorphisme $\mathbb{R}[X] \to \mathbb{R}[X]$, $P(X) \mapsto X \cdot P(X)$ ne possède aucun vecteur propre.

\vspace{1em}
\hrule
\vspace{1em}

\subsection{Polynôme caractéristique}

\begin{Def}\textbf{Polynôme caractéristique}\\

Soit $A \in M_n(K)$ la matrice d'un endomorphisme $f$ de dimension $n$. Le \textbf{polynôme caractéristique} de $A$ (ou de $f$) est défini par :
$$\chi_A(\lambda) = \det(\lambda I_n - A)$$
C'est un polynôme en $\lambda$ de degré $n$.
\end{Def}

\begin{Prop}\textbf{Propriétés du polynôme caractéristique}\\

On a la formule :
$$\chi_A(\lambda) = \lambda^n - (\mathrm{tr}\,A)\lambda^{n-1} + \cdots + (-1)^n \det A$$

En particulier :
\begin{itemize}
    \item Le coefficient dominant est $1$ (polynôme unitaire).
    \item Le coefficient de $\lambda^{n-1}$ est $-\mathrm{tr}\,A$.
    \item Le terme constant est $(-1)^n \det A$.
\end{itemize}
\end{Prop}

\begin{Rmq}$\,$

Le polynôme caractéristique est un invariant de similitude : deux matrices semblables ont le même polynôme caractéristique. En particulier, $\chi_A$ ne dépend pas de la base choisie pour représenter $f$.
\end{Rmq}

\begin{Thm}\textbf{Valeurs propres et polynôme caractéristique}\\

Les valeurs propres de $f$ sont exactement les racines du polynôme caractéristique $\chi_f$ :
$$\lambda \in \mathrm{Sp}(f) \Longleftrightarrow \chi_f(\lambda) = 0$$
\end{Thm}

\begin{Rmq}
\textbf{Démonstration :}
$\lambda$ est valeur propre $\iff \exists v \neq 0, f(v) = \lambda v \iff \exists v \neq 0, (\lambda \mathrm{Id} - f)(v) = 0$
$\iff (\lambda \mathrm{Id} - f)$ non injectif $\iff \det(\lambda I_n - A) = 0$.
\end{Rmq}


\begin{Def}\textbf{Multiplicité algébrique}\\

Soit $\lambda$ une valeur propre de $f$. La \textbf{multiplicité algébrique} de $\lambda$, notée $m_\lambda$, est la multiplicité de $\lambda$ comme racine de $\chi_f$.
\end{Def}

\begin{Ex}\textbf{Calcul de polynôme caractéristique}\\

Pour $A = \begin{pmatrix} 2 & 1 \\ 0 & 2 \end{pmatrix}$, on a :
$$\chi_A(\lambda) = \det \begin{pmatrix} \lambda - 2 & -1 \\ 0 & \lambda - 2 \end{pmatrix} = (\lambda - 2)^2$$
La seule valeur propre est $\lambda = 2$ avec multiplicité algébrique $m_2 = 2$.
\end{Ex}

\vspace{1em}
\hrule
\vspace{1em}

\exo[1]{Matrice triangulaire}

Soit $f$ l'endomorphisme de $\mathbb{R}^3$ dont la matrice dans la base canonique est :
$$D = \begin{pmatrix}
3 & 1 & 4 \\
0 & 2 & 5 \\
0 & 0 & 1
\end{pmatrix}$$
\begin{enumerate}
    \item Donner directement les valeurs propres de $f$ et leurs multiplicités algébriques.
    \item Trouver un vecteur propre associé à chaque valeur propre.
\end{enumerate}

\vspace{1em}
\hrule
\vspace{1em}

\exo[2]{Polynôme caractéristique paramétré}

Soit $m$ un réel et $f$ l'endomorphisme de $\mathbb{R}^3$ dont la matrice dans la base canonique est :
$$A = \begin{pmatrix}
1 & 0 & 1 \\
-1 & 2 & 1 \\
2-m & m-2 & m
\end{pmatrix}$$
\begin{enumerate}
    \item Calculer le polynôme caractéristique de $f$.
    \item Déterminer les valeurs propres de $f$ en fonction de $m$.
    \item Préciser le nombre de valeurs propres distinctes selon les valeurs de $m$.
\end{enumerate}

\vspace{1em}
\hrule
\vspace{1em}

\exo[2]{Trace et déterminant}

Soit $A = \begin{pmatrix}
a & b \\
c & d
\end{pmatrix}$ une matrice $2 \times 2$.
\begin{enumerate}
    \item Montrer que $\chi_A(\lambda) = \lambda^2 - \mathrm{tr}(A)\lambda + \det(A)$.
    \item En déduire que si $A$ a deux valeurs propres $\lambda_1$ et $\lambda_2$, alors $\lambda_1 + \lambda_2 = \mathrm{tr}(A)$ et $\lambda_1 \lambda_2 = \det(A)$.
    \item Généraliser au cas $n \times n$ : montrer que la somme des valeurs propres (comptées avec multiplicité) est égale à la trace.
\end{enumerate}

\vspace{1em}
\hrule
\vspace{1em}

\subsection{Sous-espaces propres}

\begin{Def}\textbf{Sous-espace propre}\\

Soit $\lambda$ une valeur propre de $f$. Le \textbf{sous-espace propre} de $f$ pour la valeur propre $\lambda$ est :
$$E_\lambda = \ker(f - \lambda \mathrm{Id}) = \{v \in E : f(v) = \lambda v\}$$
\end{Def}

\begin{Def}\textbf{Multiplicité géométrique}\\

La \textbf{multiplicité géométrique} de $\lambda$ est la dimension de son sous-espace propre : $\dim E_\lambda$.
\end{Def}

\begin{Prop}\textbf{Propriétés des sous-espaces propres}\\

\begin{enumerate}
    \item Les sous-espaces propres sont en somme directe :
    $$\bigoplus_{\lambda \in \mathrm{Sp}(f)} E_\lambda$$
    \item Pour toute valeur propre $\lambda$ :
    $$1 \leq \dim E_\lambda \leq m_\lambda$$
    où $m_\lambda$ est la multiplicité algébrique de $\lambda$.
\end{enumerate}
\end{Prop}

\begin{Meth}\textbf{Calcul d'un sous-espace propre}\\

Pour calculer $E_\lambda = \ker(A - \lambda I_n)$ :
\begin{enumerate}
    \item Former la matrice $A - \lambda I_n$.
    \item Résoudre le système homogène $(A - \lambda I_n)X = 0$.
    \item Les solutions forment le sous-espace propre $E_\lambda$.
\end{enumerate}
\end{Meth}

\vspace{1em}
\hrule
\vspace{1em}

\exo[1]{Calcul de sous-espaces propres}

Soit la matrice :
$$A = \begin{pmatrix}
5 & -6 \\
3 & -4
\end{pmatrix}$$
\begin{enumerate}
    \item Calculer le polynôme caractéristique et les valeurs propres.
    \item Déterminer les sous-espaces propres associés.
    \item Vérifier que la somme des sous-espaces propres est directe et égale à $\mathbb{R}^2$.
\end{enumerate}

\vspace{1em}
\hrule
\vspace{1em}

\exo[2]{Multiplicités algébrique et géométrique}

Soit la matrice :
$$A = \begin{pmatrix}
2 & 1 & 0 \\
0 & 2 & 0 \\
0 & 0 & 3
\end{pmatrix}$$
\begin{enumerate}
    \item Calculer le polynôme caractéristique et les valeurs propres avec leurs multiplicités algébriques.
    \item Calculer les sous-espaces propres et leurs dimensions (multiplicités géométriques).
    \item Comparer multiplicités algébriques et géométriques.
\end{enumerate}

\vspace{1em}
\hrule
\vspace{1em}

\subsection{Diagonalisation}

\begin{Def}\textbf{Endomorphisme et matrice diagonalisable}\\

\begin{itemize}
    \item Un endomorphisme $f$ est \textbf{diagonalisable} s'il existe une base de $E$ formée de vecteurs propres de $f$. Dans cette base, la matrice de $f$ est diagonale.
    \item Une matrice $A$ est \textbf{diagonalisable} s'il existe $P$ inversible telle que $P^{-1}AP$ est diagonale.
\end{itemize}
\end{Def}

\begin{Thm}\textbf{Critères de diagonalisation}\\

Soit $f$ un endomorphisme de $E$ de dimension $n$. Les assertions suivantes sont équivalentes :
\begin{enumerate}
    \item $f$ est diagonalisable.
    \item $E = \bigoplus_{\lambda \in \mathrm{Sp}(f)} E_{\lambda}$
    \item $\dim E = \sum_{\lambda \in \mathrm{Sp}(f)} \dim E_{\lambda}$
    \item $\chi_f$ est scindé et $\forall \lambda \in \mathrm{Sp}(f), \dim E_{\lambda} = m_{\lambda}$ (multiplicité géométrique = multiplicité algébrique).
\end{enumerate}
\end{Thm}

\begin{Prop}\textbf{Condition suffisante de diagonalisabilité}\\

Si $f$ admet $n = \dim E$ valeurs propres distinctes, alors $f$ est diagonalisable.
\end{Prop}

\begin{Rmq}
\textbf{Démonstration :}
Si $\lambda_1, \ldots, \lambda_n$ sont $n$ valeurs propres distinctes, il existe $n$ vecteurs propres $v_1, \ldots, v_n$ linéairement indépendants d'après le théorème d'indépendance. Ces $n$ vecteurs forment une base de $E$ composée de vecteurs propres.
\end{Rmq}


\begin{Rmq}$\,$

La réciproque est fausse : $I_n$ est diagonalisable mais $\mathrm{Sp}(I_n) = \{1\}$ (une seule valeur propre).
\end{Rmq}

\begin{Meth}\textbf{Méthode de diagonalisation}\\

Pour diagonaliser une matrice $A$ :
\begin{enumerate}
    \item Calculer $\chi_A$ et ses racines (valeurs propres $\lambda_1, \ldots, \lambda_r$).
    \item Pour chaque $\lambda_i$, calculer $E_{\lambda_i} = \ker(A - \lambda_i I)$.
    \item Vérifier : $\sum \dim E_{\lambda_i} = n$.
    \item Former $P$ avec les vecteurs propres en colonnes.
    \item $D = P^{-1}AP$ est diagonale avec les valeurs propres sur la diagonale.
\end{enumerate}
\end{Meth}

\vspace{1em}
\hrule
\vspace{1em}

\exo[1]{Diagonalisation de matrices $2 \times 2$}

Diagonaliser (si possible) les matrices suivantes :
$$A = \begin{pmatrix}
1 & 5 \\
2 & 4
\end{pmatrix}, \quad
B = \begin{pmatrix}
2 & 5 \\
4 & 3
\end{pmatrix}, \quad
C = \begin{pmatrix}
5 & 3 \\
-8 & -6
\end{pmatrix}$$

\vspace{1em}
\hrule
\vspace{1em}

\exo[2]{Diagonalisation de matrices $3 \times 3$}

Diagonaliser (si possible) les matrices suivantes :
$$D = \begin{pmatrix}
0 & 2 & -1 \\
3 & -2 & 0 \\
-2 & 2 & 1
\end{pmatrix}, \quad
E = \begin{pmatrix}
0 & 3 & 2 \\
-2 & 5 & 2 \\
2 & -3 & 0
\end{pmatrix}, \quad
F = \begin{pmatrix}
1 & 0 & 0 \\
0 & 1 & 0 \\
1 & -1 & 2
\end{pmatrix}$$

\vspace{1em}
\hrule
\vspace{1em}

\exo[2]{Diagonalisabilité avec paramètre}

Pour quelles valeurs de $m$, la matrice 
$$A(m) = \begin{pmatrix}
1 & 1-m \\
1 & 1+m
\end{pmatrix}$$ 
est-elle diagonalisable ?

\vspace{1em}
\hrule
\vspace{1em}

\exo[1]{Matrice symétrique $2 \times 2$}

Soit la matrice symétrique :
$$A = \begin{pmatrix}
a & c \\
c & d
\end{pmatrix}$$
\begin{enumerate}
    \item Calculer le polynôme caractéristique.
    \item Montrer que le discriminant est toujours positif ou nul.
    \item En déduire que $A$ est diagonalisable sur $\mathbb{R}$.
\end{enumerate}

\vspace{1em}
\hrule
\vspace{1em}

\exo[2]{Diagonalisation sur $\mathbb{C}$}

Montrer que la matrice suivante est diagonalisable sur $\mathbb{C}$ :
$$A = \begin{pmatrix}
2 & 0 & 0 & -2 \\
5 & 0 & 0 & -4 \\
-2 & 1 & -1 & 1 \\
2 & 0 & 0 & -1
\end{pmatrix}$$

\vspace{1em}
\hrule
\vspace{1em}

\subsection{Trigonalisation}

\begin{Def}\textbf{Endomorphisme trigonalisable}\\

Un endomorphisme $f$ est \textbf{trigonalisable} s'il existe une base dans laquelle la matrice de $f$ est triangulaire supérieure.
\end{Def}

\begin{Thm}\textbf{Critère de trigonalisation}\\

$$f \text{ trigonalisable} \Longleftrightarrow \chi_f \text{ est scindé}$$
\end{Thm}

\begin{Thm}\textbf{Trigonalisation sur $\mathbb{C}$}\\

Tout endomorphisme d'un $\mathbb{C}$-espace vectoriel de dimension finie est trigonalisable (car tout polynôme à coefficients complexes est scindé, d'après le théorème de d'Alembert-Gauss).
\end{Thm}

\begin{Rmq}$\,$

\begin{itemize}
    \item Les coefficients diagonaux d'une matrice trigonalisée sont les valeurs propres (avec multiplicité).
    \item On peut choisir une matrice triangulaire inférieure en inversant l'ordre des vecteurs de la base.
\end{itemize}
\end{Rmq}

\begin{Meth}\textbf{Méthode de trigonalisation}\\

Pour trigonaliser une matrice $A$ :
\begin{enumerate}
    \item Chercher une base de $E_\lambda = \ker(A - \lambda I)$ pour chaque valeur propre $\lambda$.
    \item Si $\dim E_\lambda < m_\lambda$, compléter avec des vecteurs de $\ker(A - \lambda I)^2$, puis $\ker(A - \lambda I)^3$, etc.
    \item Répéter pour toutes les valeurs propres.
    \item La famille obtenue forme une base dans laquelle la matrice est triangulaire.
\end{enumerate}
\end{Meth}

\vspace{1em}
\hrule
\vspace{1em}

\exo[2]{Trigonalisation}

Trigonaliser les matrices suivantes :
$$A = \begin{pmatrix}
1 & 0 & 1 \\
-1 & 2 & 1 \\
1 & -1 & 1
\end{pmatrix}, \quad
B = \begin{pmatrix}
0 & 1 & 0 \\
-4 & 4 & 0 \\
-2 & 1 & 2
\end{pmatrix}, \quad
C = \begin{pmatrix}
1 & 0 & 0 \\
0 & 0 & -1 \\
0 & 1 & 2
\end{pmatrix}$$

\vspace{1em}
\hrule
\vspace{1em}

\exo[2]{Trigonalisation explicite}

Soit $f$ l'endomorphisme ayant pour matrice :
$$A = \begin{pmatrix}
2 & 1 & -1 \\
3 & 3 & -4 \\
3 & 1 & -2
\end{pmatrix}$$
\begin{enumerate}
    \item Calculer le polynôme caractéristique et montrer que $f$ est trigonalisable.
    \item Déterminer les sous-espaces propres.
    \item Trouver une base de trigonalisation et écrire la matrice triangulaire.
\end{enumerate}

\vspace{1em}
\hrule
\vspace{1em}

\subsection{Puissance d'une matrice}

\begin{Def}\textbf{Matrice nilpotente}\\

Une matrice $N$ est dite \textbf{nilpotente} s'il existe un entier $k$ tel que $N^k = 0$. Le plus petit tel entier est l'indice de nilpotence.
\end{Def}

\begin{Prop}\textbf{Puissance d'une matrice diagonalisable}\\

Si $A = PDP^{-1}$ avec $D = \mathrm{diag}(\lambda_1, \ldots, \lambda_n)$, alors :
$$A^k = PD^kP^{-1} \quad \text{avec} \quad D^k = \mathrm{diag}(\lambda_1^k, \ldots, \lambda_n^k)$$
\end{Prop}

\begin{Prop}\textbf{Puissance d'une matrice trigonalisable}\\

Si $A = PTP^{-1}$ avec $T$ triangulaire, alors $A^k = PT^kP^{-1}$.

Pour calculer $T^k$ : décomposer $T = D + N$ avec $D$ diagonale et $N$ nilpotente strictement triangulaire, puis utiliser le binôme de Newton.
\end{Prop}

\begin{Ex}\textbf{Puissance d'une matrice diagonale}\\

Si $D = \begin{pmatrix} 2 & 0 \\ 0 & 3 \end{pmatrix}$, alors $D^k = \begin{pmatrix} 2^k & 0 \\ 0 & 3^k \end{pmatrix}$.
\end{Ex}

\vspace{1em}
\hrule
\vspace{1em}

\exo[1]{Puissance $n$-ième par récurrence}

Soit la matrice dont tous les coefficients valent $1$ :
$$J = \begin{pmatrix}
1 & 1 & 1 \\
1 & 1 & 1 \\
1 & 1 & 1
\end{pmatrix}$$
\begin{enumerate}
    \item Calculer $J^2$ et $J^3$.
    \item Conjecturer la forme de $J^k$ pour tout $k \in \mathbb{N}^*$.
    \item Démontrer cette conjecture par récurrence.
\end{enumerate}

\vspace{1em}
\hrule
\vspace{1em}

\exo[2]{Calcul de puissance par diagonalisation}

Soit $A = \begin{pmatrix}
0 & 1 & 2 \\
1 & 0 & 2 \\
0 & 0 & 3
\end{pmatrix}$.
\begin{enumerate}
    \item Montrer que $A$ est diagonalisable et la diagonaliser.
    \item Calculer $A^n$ pour tout $n \in \mathbb{N}$.
\end{enumerate}

\vspace{1em}
\hrule
\vspace{1em}

\exo[2]{Inverse par le polynôme caractéristique}

Soit la matrice :
$$A = \begin{pmatrix}
1 & 1 & 0 \\
-1 & 0 & 0 \\
2 & 0 & -1
\end{pmatrix}$$
\begin{enumerate}
    \item Calculer le polynôme caractéristique de $A$.
    \item En utilisant le théorème de Cayley-Hamilton, exprimer $A^{-1}$ comme polynôme en $A$.
    \item Calculer explicitement $A^{-1}$.
\end{enumerate}

\vspace{1em}
\hrule
\vspace{1em}

\section{Exercices récapitulatifs}

\vspace{1em}
\hrule
\vspace{1em}

\exo[3]{Étude complète}

Soit $f$ l'endomorphisme de $\mathbb{R}^4$ dont la matrice dans la base canonique est :
$$A = \begin{pmatrix}
-8 & -3 & -3 & 1 \\
6 & 3 & 2 & -1 \\
26 & 7 & 10 & -2 \\
0 & 0 & 0 & 2
\end{pmatrix}$$
\begin{enumerate}
    \item Montrer que $1$ et $2$ sont valeurs propres de $f$.
    \item Calculer les sous-espaces propres.
    \item $f$ est-elle diagonalisable ? Si oui, diagonaliser.
    \item Calculer $A^n$ pour tout $n \in \mathbb{N}$.
\end{enumerate}

\vspace{1em}
\hrule
\vspace{1em}

\exo[3]{Diagonalisabilité selon un paramètre}

Soit $a$ un réel. Soit $f$ l'endomorphisme de $\mathbb{R}^4$ dont la matrice dans la base canonique est :
$$A = \begin{pmatrix}
1 & 0 & 1 & 1 \\
0 & 1 & 1 & 1 \\
0 & 0 & a & 0 \\
0 & 0 & 0 & -a
\end{pmatrix}$$
\begin{enumerate}
    \item Combien $f$ a-t-elle de valeurs propres en fonction de $a$ ?
    \item Trouver une base de chaque espace propre.
    \item Pour quelles valeurs de $a$, $f$ est-elle diagonalisable ?
\end{enumerate}

\vspace{1em}
\hrule
\vspace{1em}

\exo[3]{Diagonalisation et applications aux suites}

Soit $A = \begin{pmatrix}
0 & 1 & 2 \\
1 & 0 & 2 \\
0 & 0 & 3
\end{pmatrix}$ et $B = \begin{pmatrix}
7 & 1 \\
0 & 7
\end{pmatrix}$.

\begin{enumerate}
    \item Montrer que $A$ est diagonalisable et la diagonaliser. Montrer que $B$ n'est pas diagonalisable.
    \item Montrer que pour tout $n \in \mathbb{N}$ :
    $$A^n = \begin{pmatrix}
    \frac{1+(-1)^n}{2} & \frac{1-(-1)^n}{2} & -1+3^n \\
    \frac{1-(-1)^n}{2} & \frac{1+(-1)^n}{2} & -1+3^n \\
    0 & 0 & 3^n
    \end{pmatrix}$$
    \item Déterminer toutes les suites $(u_n)$, $(v_n)$ et $(w_n)$ telles que, pour tout $n \in \mathbb{N}$ :
    $$\begin{cases}
    u_{n+1} = v_n + 2w_n \\
    v_{n+1} = u_n + 2w_n \\
    w_{n+1} = 3w_n
    \end{cases}$$
\end{enumerate}

\vspace{1em}
\hrule
\vspace{1em}

\exo[2]{Fibonacci et matrices}

Soit la matrice :
$$F = \begin{pmatrix}
1 & 1 \\
1 & 0
\end{pmatrix}$$
\begin{enumerate}
    \item Calculer $F^2$, $F^3$, $F^4$.
    \item Montrer que $F^n = \begin{pmatrix}
    f_{n+1} & f_n \\
    f_n & f_{n-1}
    \end{pmatrix}$ où $(f_n)$ est la suite de Fibonacci définie par $f_0 = 0$, $f_1 = 1$ et $f_{n+2} = f_{n+1} + f_n$.
    \item En diagonalisant $F$, retrouver la formule de Binet :
    $$f_n = \frac{1}{\sqrt{5}}\left[\left(\frac{1+\sqrt{5}}{2}\right)^n - \left(\frac{1-\sqrt{5}}{2}\right)^n\right]$$
\end{enumerate}

\vspace{1em}
\hrule
\vspace{1em}

\exo[2]{Suite récurrente linéaire}

Soit $(u_n)$ la suite définie par $u_0 = 1$, $u_1 = 2$ et $u_{n+2} = 3u_{n+1} - 2u_n$.
\begin{enumerate}
    \item Écrire ce système sous forme matricielle $X_{n+1} = AX_n$ avec $X_n = \begin{pmatrix} u_{n+1} \\ u_n \end{pmatrix}$.
    \item Diagonaliser $A$ et en déduire $A^n$.
    \item Exprimer $u_n$ en fonction de $n$.
\end{enumerate}

\vspace{1em}
\hrule
\vspace{1em}

\exo[2]{Matrice de rotation}

Soit la matrice de rotation :
$$R_\theta = \begin{pmatrix}
\cos\theta & -\sin\theta \\
\sin\theta & \cos\theta
\end{pmatrix}$$
\begin{enumerate}
    \item Montrer que $R_\theta$ n'est pas diagonalisable sur $\mathbb{R}$ (sauf cas particuliers).
    \item Diagonaliser $R_\theta$ sur $\mathbb{C}$.
    \item En déduire que $R_\theta^n = R_{n\theta}$ pour tout $n \in \mathbb{N}$.
\end{enumerate}

\vspace{1em}
\hrule
\vspace{1em}

\exo[3]{Matrice trigonalisable et puissance}

Soit :
$$A = \begin{pmatrix}
2 & 1 & 0 \\
0 & 2 & 1 \\
0 & 0 & 2
\end{pmatrix}$$
\begin{enumerate}
    \item Montrer que $A$ n'est pas diagonalisable.
    \item Écrire $A = 2I + N$ et calculer $N^2$, $N^3$.
    \item En utilisant le binôme de Newton, calculer $A^n$ pour tout $n \in \mathbb{N}$.
\end{enumerate}

\vspace{1em}
\hrule
\vspace{1em}

\exo[2]{Valeurs propres avec paramètre}

Soit $m$ un réel et $f$ l'endomorphisme dont la matrice dans la base canonique est :
$$A = \begin{pmatrix}
2 & 0 & 1-m \\
-1 & 1 & m-1 \\
m-1 & 0 & 2m
\end{pmatrix}$$
\begin{enumerate}
    \item Calculer le polynôme caractéristique.
    \item Préciser le nombre de valeurs propres de $f$ en fonction de $m$.
    \item Pour quelles valeurs de $m$ la matrice est-elle diagonalisable ?
\end{enumerate}

\vspace{1em}
\hrule
\vspace{1em}

\exo[2]{Application aux suites}

Soit $(x_n, y_n, z_n)$ la suite définie par :
$$\begin{cases}
x_{n+1} = 2x_n + y_n \\
y_{n+1} = x_n + 2y_n \\
z_{n+1} = z_n
\end{cases}$$
avec $(x_0, y_0, z_0) = (1, 0, 1)$.
\begin{enumerate}
    \item Écrire ce système sous forme matricielle.
    \item Calculer les valeurs propres de la matrice.
    \item Exprimer $x_n$, $y_n$, $z_n$ en fonction de $n$.
\end{enumerate}

\vspace{1em}
\hrule
\vspace{1em}

\exo[1]{QCM - Valeurs propres}

Pour chaque question, une seule réponse est correcte.

\begin{enumerate}
\item Si $\lambda$ est valeur propre de $A$, alors $\lambda^2$ est valeur propre de :
\begin{multicols}{4}
\begin{enumerate}[label=\alph*.]
    \item $2A$
    \item $A^2$
    \item $A + I$
    \item $A^{-1}$
\end{enumerate}
\end{multicols}

\item Une matrice $3 \times 3$ ayant pour polynôme caractéristique $(\lambda - 1)^2(\lambda - 2)$ :
\begin{multicols}{2}
\begin{enumerate}[label=\alph*.]
    \item est toujours diagonalisable
    \item n'est jamais diagonalisable
    \item peut être diagonalisable ou non
    \item a exactement 3 valeurs propres distinctes
\end{enumerate}
\end{multicols}

\item Si $A$ et $B$ sont semblables, alors :
\begin{multicols}{2}
\begin{enumerate}[label=\alph*.]
    \item $A = B$
    \item $\det(A) = \det(B)$ et $\mathrm{tr}(A) = \mathrm{tr}(B)$
    \item $A$ et $B$ ont les mêmes coefficients
    \item $AB = BA$
\end{enumerate}
\end{multicols}
\end{enumerate}

\vspace{1em}
\hrule
\vspace{1em}

\exo[3]{Commutant d'une matrice diagonalisable}

Soit $A$ une matrice diagonalisable à valeurs propres distinctes.
\begin{enumerate}
    \item Montrer que si $B$ commute avec $A$ (i.e. $AB = BA$), alors tout vecteur propre de $A$ est aussi vecteur propre de $B$.
    \item En déduire que $B$ est diagonalisable dans la même base que $A$.
    \item Montrer que l'ensemble des matrices qui commutent avec $A$ est l'ensemble des polynômes en $A$.
\end{enumerate}

\vspace{1em}
\hrule
\vspace{1em}

\exo[3]{Matrices semblables sur $\mathbb{R}$ et sur $\mathbb{C}$}

Soit $A = \begin{pmatrix} 0 & -1 \\ 1 & 0 \end{pmatrix}$ et $B = \begin{pmatrix} 0 & 1 \\ -1 & 0 \end{pmatrix}$.
\begin{enumerate}
    \item Calculer les polynômes caractéristiques de $A$ et $B$.
    \item Montrer que $A$ et $B$ ne sont pas semblables sur $\mathbb{R}$.
    \item Montrer que $A$ et $B$ sont semblables sur $\mathbb{C}$.
\end{enumerate}

\vspace{1em}
\hrule
\vspace{1em}

