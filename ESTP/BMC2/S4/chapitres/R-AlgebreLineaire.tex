
\section{Algèbre linéaire : compléments}

\textit{Dans tout le chapitre, $E$ désigne un espace vectoriel de dimension finie $n$ sur un corps $K$ (généralement $\mathbb{R}$ ou $\mathbb{C}$). Toutes les propriétés énoncées restent vraies pour n'importe quel corps.}

%==============================================================================
\subsection{Somme de sous-espaces vectoriels}
%==============================================================================

\begin{Def}\textbf{Somme de sous-espaces vectoriels}\\

Soient $E_1, \ldots, E_r$ des sous-espaces vectoriels de $E$. La \textbf{somme} de ces sous-espaces est l'ensemble :
$$E_1 + E_2 + \cdots + E_r = \left\{ x_1 + x_2 + \cdots + x_r \mid x_i \in E_i \text{ pour tout } i \right\}$$
\end{Def}

\begin{Prop}\textbf{La somme est un sous-espace vectoriel}\\

La somme $E_1 + E_2 + \cdots + E_r$ est un sous-espace vectoriel de $E$. C'est le plus petit sous-espace vectoriel de $E$ contenant tous les $E_i$.
\end{Prop}

\begin{Def}\textbf{Somme directe}\\

Les sous-espaces $E_1, \ldots, E_r$ sont dits \textbf{en somme directe} si pour tout vecteur $x \in E_1 + \cdots + E_r$, la décomposition $x = x_1 + \cdots + x_r$ avec $x_i \in E_i$ est \textbf{unique}.

On note alors la somme directe :
$$E_1 \oplus E_2 \oplus \cdots \oplus E_r$$
\end{Def}

\begin{Prop}\textbf{Caractérisation de la somme directe}\\

Les sous-espaces $E_1, \ldots, E_r$ sont en somme directe si et seulement si :
$$x_1 + x_2 + \cdots + x_r = 0_E \quad \text{avec } x_i \in E_i \quad \Longrightarrow \quad x_1 = x_2 = \cdots = x_r = 0_E$$
\end{Prop}

\begin{Prop}\textbf{Cas de deux sous-espaces}\\

Pour deux sous-espaces $E_1$ et $E_2$, on a :
$$E_1 \oplus E_2 \Longleftrightarrow E_1 \cap E_2 = \{0_E\}$$
\end{Prop}

\begin{Rmq}$\,$

\textbf{Attention :} Pour $r \geq 3$ sous-espaces, la condition $E_i \cap E_j = \{0_E\}$ pour tout $i \neq j$ n'est \textbf{pas suffisante} pour que la somme soit directe.
\end{Rmq}

\begin{Ex}\textbf{Contre-exemple}\\

Dans $\mathbb{R}^2$, considérons :
\begin{itemize}
    \item $E_1 = \text{Vect}((1, 0))$ (axe des abscisses)
    \item $E_2 = \text{Vect}((0, 1))$ (axe des ordonnées)
    \item $E_3 = \text{Vect}((1, 1))$ (première bissectrice)
\end{itemize}
On a $E_i \cap E_j = \{0\}$ pour $i \neq j$, mais $(1,1) + (-1,0) + (0,-1) = (0,0)$ donc la somme n'est pas directe.
\end{Ex}

\begin{Thm}\textbf{Dimension de la somme}\\

Soient $E_1, \ldots, E_r$ des sous-espaces vectoriels de $E$. Alors :
$$\dim(E_1 + E_2 + \cdots + E_r) \leq \dim E_1 + \dim E_2 + \cdots + \dim E_r$$

On a \textbf{égalité si et seulement si} les sous-espaces sont en somme directe.
\end{Thm}

\begin{Prop}\textbf{Formule pour deux sous-espaces}\\

Pour deux sous-espaces $E_1$ et $E_2$ :
$$\dim(E_1 + E_2) = \dim E_1 + \dim E_2 - \dim(E_1 \cap E_2)$$
\end{Prop}

\begin{Def}\textbf{Sous-espaces supplémentaires}\\

Deux sous-espaces $E_1$ et $E_2$ sont dits \textbf{supplémentaires dans $E$} si :
$$E = E_1 \oplus E_2$$
c'est-à-dire si $E_1 + E_2 = E$ et $E_1 \cap E_2 = \{0_E\}$.

On dit que $E_2$ est un \textbf{supplémentaire} de $E_1$ dans $E$.
\end{Def}

\begin{Rmq}$\,$

Un supplémentaire n'est \textbf{pas unique}. En dimension finie, tout sous-espace admet au moins un supplémentaire.
\end{Rmq}

\begin{Prop}\textbf{Définition d'applications linéaires par morceaux}\\

Si $E = E_1 \oplus E_2 \oplus \cdots \oplus E_r$ et si on se donne des applications linéaires $f_i : E_i \to F$, alors il existe une \textbf{unique} application linéaire $f : E \to F$ telle que :
$$\forall x \in E_i, \quad f(x) = f_i(x)$$
\end{Prop}

\vspace{1em}
\hrule
\vspace{1em}

\exo[1]{Somme de sous-espaces de $\mathbb{R}^3$}

Dans $\mathbb{R}^3$, on considère :
$$E_1 = \{(x, y, z) \in \mathbb{R}^3 \mid x + y = 0\} \quad \text{et} \quad E_2 = \text{Vect}((1, 1, 0))$$
\begin{enumerate}
    \item Déterminer $\dim E_1$ et une base de $E_1$.
    \item Montrer que $E_1$ et $E_2$ sont en somme directe.
    \item A-t-on $E_1 \oplus E_2 = \mathbb{R}^3$ ?
\end{enumerate}

\vspace{1em}
\hrule
\vspace{1em}

\exo[1]{Intersection et somme}

Soient $F = \{(x, y, z, t) \in \mathbb{R}^4 \mid x + y = z + t = 0\}$ et $G = \text{Vect}((1, 0, 1, 0), (0, 1, 0, 1))$.
\begin{enumerate}
    \item Déterminer $\dim F$ et $\dim G$.
    \item Calculer $\dim(F \cap G)$.
    \item En déduire $\dim(F + G)$.
\end{enumerate}

\vspace{1em}
\hrule
\vspace{1em}

\exo[2]{Somme directe de trois sous-espaces}

Dans $\mathbb{R}^3$, on pose :
$$E_1 = \text{Vect}((1, 0, 1)), \quad E_2 = \text{Vect}((0, 1, 1)), \quad E_3 = \text{Vect}((1, 1, 0))$$
\begin{enumerate}
    \item Montrer que $E_i \cap E_j = \{0\}$ pour $i \neq j$.
    \item Les sous-espaces $E_1$, $E_2$, $E_3$ sont-ils en somme directe ?
    \item Déterminer $E_1 + E_2 + E_3$.
\end{enumerate}

\vspace{1em}
\hrule
\vspace{1em}

%==============================================================================
\subsection{Endomorphismes}
%==============================================================================

\begin{Def}\textbf{Endomorphisme}\\

Un \textbf{endomorphisme} de $E$ est une application linéaire de $E$ dans lui-même :
$$f : E \longrightarrow E$$
L'ensemble des endomorphismes de $E$ est noté $\mathcal{L}(E)$ ou $\text{End}(E)$.
\end{Def}

\begin{Def}\textbf{Puissances d'un endomorphisme}\\

Soit $f \in \mathcal{L}(E)$. On définit par récurrence les \textbf{puissances} de $f$ :
\begin{itemize}
    \item $f^0 = \text{Id}_E$ (l'endomorphisme identité)
    \item $f^1 = f$
    \item $f^{k+1} = f \circ f^k$ pour $k \geq 0$
\end{itemize}
Ainsi $f^k$ désigne la composition de $f$ avec elle-même $k$ fois.
\end{Def}

\begin{Prop}\textbf{Algèbre des endomorphismes}\\

L'ensemble $\mathcal{L}(E)$ muni de l'addition et de la composition est une \textbf{algèbre} :
\begin{itemize}
    \item $(\mathcal{L}(E), +, \cdot)$ est un espace vectoriel
    \item La composition $\circ$ est bilinéaire et associative
    \item $\text{Id}_E$ est élément neutre pour $\circ$
\end{itemize}
De plus, pour $f, g \in \mathcal{L}(E)$ et $k, \ell \in \mathbb{N}$ :
$$f^k \circ f^\ell = f^{k+\ell} \quad \text{et} \quad (f^k)^\ell = f^{k\ell}$$
\end{Prop}

\begin{Rmq}$\,$

La composition n'est \textbf{pas commutative} en général : $f \circ g \neq g \circ f$.
\end{Rmq}

\vspace{1em}
\hrule
\vspace{1em}

\exo[1]{Commutativité}

Soit $f : \mathbb{R}^2 \to \mathbb{R}^2$ défini par $f(x, y) = (y, x)$ et $g : \mathbb{R}^2 \to \mathbb{R}^2$ défini par $g(x, y) = (2x, 0)$.
\begin{enumerate}
    \item Calculer $f \circ g$ et $g \circ f$.
    \item Les endomorphismes $f$ et $g$ commutent-ils ?
\end{enumerate}

\vspace{1em}
\hrule
\vspace{1em}

%==============================================================================
\subsection{Déterminant et trace d'un endomorphisme}
%==============================================================================

\begin{Def}\textbf{Déterminant d'un endomorphisme}\\

Soit $f \in \mathcal{L}(E)$ et $A$ la matrice de $f$ dans une base quelconque de $E$. Le \textbf{déterminant} de $f$ est défini par :
$$\det f = \det A$$
\end{Def}

\begin{Prop}\textbf{Invariance du déterminant}\\

Le déterminant d'un endomorphisme est un \textbf{invariant algébrique} : il ne dépend pas de la base choisie pour représenter $f$.

En effet, si $A$ et $B$ sont les matrices de $f$ dans deux bases différentes, alors $B = P^{-1}AP$ pour une matrice de passage $P$ inversible, et :
$$\det B = \det(P^{-1}AP) = \det(P^{-1}) \cdot \det A \cdot \det P = \det A$$
\end{Prop}

\begin{Def}\textbf{Trace d'un endomorphisme}\\

Soit $f \in \mathcal{L}(E)$ et $A = (a_{ij})$ la matrice de $f$ dans une base. La \textbf{trace} de $f$ est :
$$\text{tr}(f) = \text{tr}(A) = \sum_{i=1}^{n} a_{ii}$$
(somme des coefficients diagonaux)
\end{Def}

\begin{Prop}\textbf{Propriétés de la trace}\\

\begin{enumerate}
    \item La trace est un \textbf{invariant algébrique} : elle ne dépend pas de la base.
    \item La trace est \textbf{linéaire} : $\text{tr}(\alpha f + \beta g) = \alpha \, \text{tr}(f) + \beta \, \text{tr}(g)$
    \item $\text{tr}(f \circ g) = \text{tr}(g \circ f)$ (cyclicité)
    \item $\text{tr}(\text{Id}_E) = n = \dim E$
\end{enumerate}
\end{Prop}

\begin{Ex}\textbf{Calcul de trace}\\

Soit $A = \begin{pmatrix} 3 & 1 & -2 \\ 0 & 5 & 4 \\ 1 & -1 & 2 \end{pmatrix}$. On a $\text{tr}(A) = 3 + 5 + 2 = 10$.
\end{Ex}

\vspace{1em}
\hrule
\vspace{1em}

\exo[1]{Trace et déterminant}

Soit $A = \begin{pmatrix} 2 & 1 \\ 3 & 4 \end{pmatrix}$.
\begin{enumerate}
    \item Calculer $\text{tr}(A)$ et $\det(A)$.
    \item Calculer $\text{tr}(A^2)$ et $\det(A^2)$.
    \item Vérifier que $\det(A^2) = (\det A)^2$.
\end{enumerate}

\vspace{1em}
\hrule
\vspace{1em}

\exo[2]{Trace d'un projecteur}

Soit $p$ un projecteur de $E$ (c'est-à-dire $p^2 = p$).
\begin{enumerate}
    \item Montrer que les seules valeurs propres possibles de $p$ sont $0$ et $1$.
    \item Montrer que $E = \ker p \oplus \text{Im}\, p$.
    \item En déduire que $\text{tr}(p) = \text{rg}(p) = \dim(\text{Im}\, p)$.
\end{enumerate}

\vspace{1em}
\hrule
\vspace{1em}

%==============================================================================
\subsection{Rang d'une application linéaire}
%==============================================================================

\begin{Def}\textbf{Rang}\\

Soit $f : E \to F$ une application linéaire. Le \textbf{rang} de $f$ est la dimension de l'image de $f$ :
$$\text{rg}(f) = \dim(\text{Im}\, f)$$
\end{Def}

\begin{Thm}\textbf{Théorème du rang}\\

Soit $f : E \to F$ une application linéaire. On a :
$$\dim E = \dim(\ker f) + \text{rg}(f)$$
\end{Thm}

\begin{Prop}\textbf{Bornes du rang}\\

Pour $f : E \to F$ :
$$0 \leq \text{rg}(f) \leq \min(\dim E, \dim F)$$
\end{Prop}

\begin{Prop}\textbf{Invariance du rang}\\

Le rang est un \textbf{invariant algébrique}. Si $A$ est la matrice de $f$ dans des bases et $B$ inversible, alors :
$$\text{rg}(AB) = \text{rg}(A) = \text{rg}(f)$$
\end{Prop}

\begin{Meth}\textbf{Calcul du rang d'une matrice}\\

Pour calculer le rang d'une matrice $A$, on peut :
\begin{enumerate}
    \item \textbf{Méthode des pivots de Gauss :} Échelonner la matrice et compter les pivots non nuls.
    \item \textbf{Méthode des mineurs :} Chercher la taille maximale d'une sous-matrice carrée inversible.
\end{enumerate}
\end{Meth}

\begin{Ex}\textbf{Calcul de rang par échelonnement}\\

Soit $A = \begin{pmatrix} 1 & 2 & 3 \\ 2 & 3 & 4 \\ 3 & 4 & 5 \end{pmatrix}$.

Par opérations élémentaires : $L_2 \leftarrow L_2 - 2L_1$, $L_3 \leftarrow L_3 - 3L_1$ :
$$A \sim \begin{pmatrix} 1 & 2 & 3 \\ 0 & -1 & -2 \\ 0 & -2 & -4 \end{pmatrix}$$
Puis $L_3 \leftarrow L_3 - 2L_2$ :
$$A \sim \begin{pmatrix} 1 & 2 & 3 \\ 0 & -1 & -2 \\ 0 & 0 & 0 \end{pmatrix}$$
Donc $\text{rg}(A) = 2$ (deux pivots non nuls).
\end{Ex}

\vspace{1em}
\hrule
\vspace{1em}

%==============================================================================
\subsection{Relation entre rang et déterminant}
%==============================================================================

\begin{Def}\textbf{Matrice extraite}\\

Soit $A \in M_{n,p}(K)$. Une \textbf{matrice extraite} de $A$ est une matrice obtenue en supprimant un certain nombre de lignes et un certain nombre de colonnes de $A$.
\end{Def}

\begin{Thm}\textbf{Caractérisation du rang par les mineurs}\\

Soit $A \in M_{n,p}(K)$ et $r$ un entier tel que $1 \leq r \leq \min(n, p)$. Alors :
$$\text{rg}(A) \geq r \Longleftrightarrow \text{il existe une matrice extraite } r \times r \text{ de } A \text{ de déterminant non nul}$$
\end{Thm}

\begin{Prop}\textbf{Rang maximal et inversibilité}\\

Pour $A \in M_n(K)$ :
$$\text{rg}(A) = n \Longleftrightarrow \det(A) \neq 0 \Longleftrightarrow A \text{ est inversible}$$
\end{Prop}

\begin{Ex}\textbf{Rang par extraction}\\

Soit $A = \begin{pmatrix} 1 & -4 & 3 & 2 \\ 0 & 6 & 0 & 3 \\ 5 & 5 & 6 & 4 \\ 0 & 0 & 0 & 5 \end{pmatrix}$.

La sous-matrice $\begin{pmatrix} 1 & -4 & 2 \\ 0 & 6 & 3 \\ 0 & 0 & 5 \end{pmatrix}$ (colonnes 1, 2, 4) est triangulaire avec déterminant $1 \times 6 \times 5 = 30 \neq 0$.

Donc $\text{rg}(A) \geq 3$. Comme $\det(A) = 0$ (à vérifier), on a $\text{rg}(A) = 3$.
\end{Ex}

\vspace{1em}
\hrule
\vspace{1em}

%==============================================================================
\subsection{Équations de l'image}
%==============================================================================

\begin{Meth}\textbf{Trouver les équations de l'image}\\

Soit $f : E \to F$ une application linéaire. Pour trouver les équations cartésiennes de $\text{Im}\, f$ :
\begin{enumerate}
    \item Écrire la matrice $A$ de $f$ dans les bases canoniques.
    \item Un vecteur $y = (y_1, \ldots, y_m)$ est dans $\text{Im}\, f$ si et seulement si le système $Ax = y$ a une solution.
    \item Appliquer le pivot de Gauss au système augmenté $(A \mid y)$.
    \item Les équations de compatibilité obtenues donnent les équations de $\text{Im}\, f$.
\end{enumerate}
\end{Meth}

\begin{Ex}\textbf{Équations de l'image}\\

Soit $f : \mathbb{R}^3 \to \mathbb{R}^4$ de matrice $A = \begin{pmatrix} 5 & -1 & -3 \\ 2 & 1 & -4 \\ 3 & -2 & 1 \\ 1 & 2 & -5 \end{pmatrix}$.

On échelonne $(A \mid y)$ où $y = (y_1, y_2, y_3, y_4)^T$ :
$$\begin{pmatrix} 5 & -1 & -3 & y_1 \\ 2 & 1 & -4 & y_2 \\ 3 & -2 & 1 & y_3 \\ 1 & 2 & -5 & y_4 \end{pmatrix} \sim \cdots$$

On trouve $\text{rg}(A) = 2$, donc $\text{Im}\, f$ est un plan dans $\mathbb{R}^4$ défini par 2 équations.
\end{Ex}

\vspace{1em}
\hrule
\vspace{1em}

%==============================================================================
\section{Exercices}
%==============================================================================

\exo[1]{Rang de matrices}

Calculer le rang des matrices suivantes :
$$A = \begin{pmatrix} 1 & 2 & 3 \\ 2 & 3 & 4 \\ 3 & 4 & 5 \end{pmatrix}, \quad
B = \begin{pmatrix} 1 & 1 & 1 \\ 1 & 2 & 4 \\ 1 & 3 & 9 \end{pmatrix}, \quad
C = \begin{pmatrix} 1 & 2 & 3 & 3 \\ 2 & 3 & 4 & 2 \\ 3 & 4 & 5 & 2 \end{pmatrix}$$

\vspace{1em}
\hrule
\vspace{1em}

\exo[1]{Rang en fonction de paramètres}

Soit $D = \begin{pmatrix} 1 & 2 & 1 & 2 \\ -2 & -3 & 0 & -5 \\ 4 & 9 & 6 & 7 \\ 1 & -1 & -5 & 5 \end{pmatrix}$.

Calculer le rang de $D$.

\vspace{1em}
\hrule
\vspace{1em}

\exo[2]{Rang maximal}

Soient $\alpha, \beta$ deux réels. 
$$A = \begin{pmatrix} 1 & 3 & \alpha & \beta \\ 2 & -1 & 2 & 1 \\ -1 & 1 & 2 & 0 \end{pmatrix}$$
Déterminer les valeurs de $\alpha$ et $\beta$ pour que la matrice $A$ soit de rang maximal.

\vspace{1em}
\hrule
\vspace{1em}

\exo[2]{Rang et déterminant paramétrés}

Soit $m \in \mathbb{R}$ et $A(m) = \begin{pmatrix} m-1 & 1 & -1 & 1-m \\ -1 & m+1 & -1-m & 1 \\ m+4 & m & 1 & -3 \\ 3m+1 & 2m+3 & -2m-3 & -2m \end{pmatrix}$.
\begin{enumerate}
    \item Calculer $\det(A(m))$ en fonction de $m$.
    \item Déterminer le rang de $A(m)$ selon les valeurs de $m$.
    \item Pour $m = -1$, soit $f : \mathbb{R}^4 \to \mathbb{R}^4$ l'application linéaire de matrice $A(-1)$.\\
    Déterminer les équations de $\text{Im}\, f$ et une base de $\ker f$.
\end{enumerate}

\vspace{1em}
\hrule
\vspace{1em}

\exo[1]{Noyau et image}

Soit $f : \mathbb{R}^3 \to \mathbb{R}^3$ l'endomorphisme de matrice dans la base canonique :
$$A = \begin{pmatrix} 1 & 2 & 1 \\ 2 & 4 & 2 \\ 1 & 2 & 1 \end{pmatrix}$$
\begin{enumerate}
    \item Déterminer $\ker f$ et $\text{Im}\, f$.
    \item Vérifier le théorème du rang.
    \item L'endomorphisme $f$ est-il un projecteur ?
\end{enumerate}

\vspace{1em}
\hrule
\vspace{1em}

\exo[2]{Somme directe et supplémentaire}

Dans $\mathbb{R}^4$, on considère :
$$F = \{(x, y, z, t) \mid x + y = 0 \text{ et } z - t = 0\}$$
$$G = \text{Vect}((1, 0, 1, 0), (0, 1, 0, 1))$$
\begin{enumerate}
    \item Déterminer une base de $F$ et de $G$.
    \item Montrer que $F$ et $G$ sont supplémentaires dans $\mathbb{R}^4$.
    \item Exprimer le vecteur $(1, 2, 3, 4)$ comme somme d'un vecteur de $F$ et d'un vecteur de $G$.
\end{enumerate}

\vspace{1em}
\hrule
\vspace{1em}

\exo[2]{Projecteur associé à une somme directe}

On considère $E = \mathbb{R}^3$ avec $F = \text{Vect}((1, 1, 0))$ et $G = \{(x, y, z) \mid x + y = 0\}$.
\begin{enumerate}
    \item Montrer que $E = F \oplus G$.
    \item Déterminer la matrice du projecteur $p$ sur $F$ parallèlement à $G$.
    \item Vérifier que $p^2 = p$.
\end{enumerate}

\vspace{1em}
\hrule
\vspace{1em}

\exo[1]{Trace et commutateur}

Soient $A, B \in M_n(\mathbb{R})$. Le \textbf{commutateur} de $A$ et $B$ est $[A, B] = AB - BA$.
\begin{enumerate}
    \item Montrer que $\text{tr}([A, B]) = 0$.
    \item En déduire qu'il n'existe pas de matrices $A, B \in M_n(\mathbb{R})$ telles que $AB - BA = I_n$.
\end{enumerate}

\vspace{1em}
\hrule
\vspace{1em}

\exo[2]{Rang d'un produit}

Soient $A \in M_{n,p}(K)$ et $B \in M_{p,q}(K)$.
\begin{enumerate}
    \item Montrer que $\text{rg}(AB) \leq \min(\text{rg}(A), \text{rg}(B))$.
    \item Donner un exemple où $\text{rg}(AB) < \text{rg}(A)$ et $\text{rg}(AB) < \text{rg}(B)$.
    \item Montrer que si $B$ est inversible, alors $\text{rg}(AB) = \text{rg}(A)$.
\end{enumerate}

\vspace{1em}
\hrule
\vspace{1em}

\exo[3]{Matrices nilpotentes et trace}

Une matrice $A \in M_n(K)$ est dite \textbf{nilpotente} s'il existe $k \in \mathbb{N}^*$ tel que $A^k = 0$.
\begin{enumerate}
    \item Montrer que si $A$ est nilpotente, alors $\text{tr}(A) = 0$.
    \item Montrer que si $A$ est nilpotente, alors $\det(A) = 0$.
    \item La réciproque est-elle vraie ?
    \item Montrer que si $A$ est nilpotente, alors $\text{tr}(A^k) = 0$ pour tout $k \geq 1$.
\end{enumerate}

\vspace{1em}
\hrule
\vspace{1em}

\exo[2]{Image et noyau d'un endomorphisme}

Soit $f$ l'endomorphisme de $\mathbb{R}^4$ de matrice dans la base canonique :
$$A = \begin{pmatrix} 1 & 1 & 1 & 1 \\ 1 & 1 & -1 & -1 \\ 1 & -1 & 1 & -1 \\ 1 & -1 & -1 & 1 \end{pmatrix}$$
\begin{enumerate}
    \item Calculer le rang de $f$.
    \item Déterminer une base de $\ker f$ et une base de $\text{Im}\, f$.
    \item Montrer que $\mathbb{R}^4 = \ker f \oplus \text{Im}\, f$.
\end{enumerate}

\vspace{1em}
\hrule
\vspace{1em}

\exo[3]{Caractérisation de la somme directe}

Soient $E_1, E_2, E_3$ trois sous-espaces vectoriels de $E$. On suppose que $E_i \cap E_j = \{0\}$ pour $i \neq j$.
\begin{enumerate}
    \item Montrer que cette condition n'implique pas que $E_1, E_2, E_3$ sont en somme directe.
    \item Montrer que $E_1, E_2, E_3$ sont en somme directe si et seulement si :
    $$E_1 \cap (E_2 + E_3) = \{0\}$$
    \item Généraliser au cas de $r$ sous-espaces.
\end{enumerate}

\vspace{1em}
\hrule
\vspace{1em}

\exo[2]{Endomorphisme induit}

Soit $f \in \mathcal{L}(E)$ et $F$ un sous-espace vectoriel de $E$ stable par $f$ (c'est-à-dire $f(F) \subset F$).
\begin{enumerate}
    \item Montrer que la restriction $f|_F : F \to F$ est un endomorphisme de $F$.
    \item Soit $G$ un supplémentaire de $F$ dans $E$, stable par $f$. Montrer que si $\mathcal{B}_F$ et $\mathcal{B}_G$ sont des bases de $F$ et $G$, alors la matrice de $f$ dans $\mathcal{B}_F \cup \mathcal{B}_G$ est diagonale par blocs.
    \item En déduire que $\det(f) = \det(f|_F) \cdot \det(f|_G)$ et $\text{tr}(f) = \text{tr}(f|_F) + \text{tr}(f|_G)$.
\end{enumerate}

\vspace{1em}
\hrule
\vspace{1em}

