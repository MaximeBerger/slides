\section{Combinatoire}



\begin{enumerate}
    \item 
Dans un bureau de vingt-cinq personnes, combien existe-t-il de façons
de choisir un président, un vice -président, un secrétaire et un trésorier ?\\

    \item

Quel est le nombre de choix possibles de 13 étudiants parmi un groupe
de 120 étudiants ?\\

    \item 
Une équipe de 8 personnes est amenée à occuper 8 postes de
travail distincts.\\
Combien peut-on envisager de répartitions distinctes des 8
personnes.\\
\end{enumerate}


\section{Statistiques}


On considère la série statistique suivante:\\


 \begin{tabular}{|c|c|c|c|c|c|}
 \hline
 valeur&10&15&16&20&21\\
 \hline
 effectif &2& 3&7&7&1\\
 \hline
 \end{tabular}

\begin{enumerate}
    \item 
    Donner l'étendue.

    \item
 Donner l'écart inter-quartile.

    \item 
  Dessiner la boîte à moustaches.\\
\end{enumerate}


\section{Probabilités}


\begin{enumerate}
    \item 

Donner la définition d'une tribu.
 \item
 Donner la tribu triviale.
\end{enumerate}



Donner la définition d'un système complet d'événements.



% \section*{Solutions}

% \begin{enumerate}
%     \item \textbf{Définition d'une tribu :} \\
    
%     Soit $\Omega$ un univers. Une \textbf{tribu} $\mathcal{F}$ sur $\Omega$ est une collection de parties de $\Omega$ qui vérifie les propriétés suivantes :
%     \begin{enumerate}
%         \item $\Omega \in \mathcal{F}$.
%         \item Si $A \in \mathcal{F}$ alors son complémentaire $A^c \in \mathcal{F}$.
%         \item Si $\{A_n\}_{n \in \mathbb{N}}$ est une suite d'éléments de $\mathcal{F}$, alors leur union $\bigcup\limits_{n=1}^{\infty} A_n$ appartient aussi à $\mathcal{F}$.
%     \end{enumerate}
    
%     \item \textbf{La tribu triviale :} \\
    
%     La \textbf{tribu triviale} sur un univers $\Omega$ est la plus petite tribu possible. Elle est donnée par :
%     \[
%     \mathcal{F} = \{\emptyset, \Omega\}.
%     \]

% \end{enumerate}

% \textbf{Définition d'un système complet d'événements :} \\

% Un ensemble fini ou dénombrable d'événements $\{A_i\}_{i \in I}$ est un \textbf{système complet d'événements} si :
% \begin{enumerate}
%     \item Les événements sont deux à deux disjoints : $A_i \cap A_j = \emptyset$ pour tout $i \neq j$.
%     \item Leur union couvre tout l'univers : $\bigcup\limits_{i \in I} A_i = \Omega$.
% \end{enumerate}