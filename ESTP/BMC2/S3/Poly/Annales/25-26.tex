\section{Intégrale curviligne}


\noindent\textbf{Exercice  : 5 points}
\vspace{2em}

Vous choisirez deux questions parmi cette liste.
A chaque fois, vous illustrerez la situation par un dessin. 

\vspace{1em}

\begin{enumerate}[itemsep=1.5em]
    \item 
    Soit $\Gamma$ l'arc de cercle orienté positivement défini par $x(t) = \cos t$, $y(t) = \sin t$, pour $t \in [0, \pi]$. Calculer $\displaystyle \int_\Gamma x\,dx + y\,dy$.

    \item 
    Soit $C$ le segment orienté du point $A(0,0)$ au point $B(1,1)$. Calculer $\displaystyle \int_C (x + y)\,dx + (x - y)\,dy$.

    \item 
    Soit $\Gamma$ le demi-cercle supérieur de rayon 1 centré à l'origine, paramétré par $x(t) = \cos t$, $y(t) = \sin t$ pour $t \in [0, \pi]$. Calculer $\displaystyle \int_\Gamma x^2\,dy$.

    \item 
    Soit $C$ le contour du triangle orienté positivement dont les sommets sont $O(0,0)$, $A(1,0)$, $B(1,1)$. Calculer $\displaystyle \int_C x\,dy - y\,dx$.

    \item 
    Soit $\Gamma$ la courbe paramétrée par $x(t) = t$, $y(t) = t^2$, pour $t \in [0, 1]$. Calculer $\displaystyle \int_\Gamma (x^2 + y)\,ds$, où $ds$ est l'élément de longueur.

    \item 
    Montrer que l'intégrale curviligne $\displaystyle \int_\Gamma y\,dx - x\,dy$ sur un cercle orienté positivement centré à l'origine est égale à $2\pi R^2$, où $R$ est le rayon du cercle.

    \item 
    Soit $\Gamma$ la portion de la parabole $y = x^2$ allant de $(-1,1)$ à $(1,1)$. Calculer $\displaystyle \int_\Gamma x\,dy$ en utilisant une paramétrisation adaptée.

    \item 
    Soit $\vec{F}(x, y) = (y, -x)$ et soit $\Gamma$ le bord du carré de sommets $(\pm1, \pm1)$ orienté positivement. Calculer $\displaystyle \int_\Gamma \vec{F} \cdot d\vec{r}$.

    \item 
    Soit $\vec{F}(x, y) = (x^2, y^2)$ et $\Gamma$ le cercle unité orienté positivement. Calculer $\displaystyle \int_\Gamma \vec{F} \cdot d\vec{r}$.

    \item 
    Soit $\Gamma$ une courbe paramétrée par $x(t) = \cos t$, $y(t) = \sin 2t$ pour $t \in [0, \pi]$. Calculer $\displaystyle \int_\Gamma x\,dy - y\,dx$.
\end{enumerate}

\newpage

\section{Séries numériques}


 
Trois questions seront choisies dans la liste suivante.\\
\noindent\textbf{Exercice  : 5 points}\\


\begin{enumerate}
    \item 
    En utilisant un critère de comparaison, étudier la convergence de la série $\displaystyle\sum_{n=2}^{+\infty} \frac{1}{n(\ln n)^2}$.\\

    \item
    En utilisant un critère de Riemann, déterminer la nature de la série $\displaystyle\sum_{n=1}^{+\infty} \frac{1}{n^\alpha}$ selon les valeurs du réel $\alpha$.\\

    \item 
    En utilisant le critère de la série géométrique, étudier la convergence de la série $\displaystyle\sum_{n=0}^{+\infty} \left(\frac{2}{3}\right)^n$.\\

    \item 
    En utilisant un critère de comparaison série-intégrale, déterminer la nature de la série $\displaystyle\sum_{n=2}^{+\infty} \frac{1}{n\ln n}$.\\

    \item 
    Étudier la convergence absolue ou non de la série $\displaystyle\sum_{n=1}^{+\infty} \frac{(-1)^n}{n}$.\\

    \item 
    En appliquant le critère des séries alternées, déterminer la nature de la série $\displaystyle\sum_{n=1}^{+\infty} \frac{(-1)^n}{n \sqrt{n}}$.\\

    \item 
    La série $\displaystyle\sum_{n=1}^{+\infty} \frac{1}{n(n+1)}$ est-elle convergente ? Peut-on calculer sa somme en utilisant la méthode des séries télescopiques ?\\

    \item 
    En développant le terme général, déterminer la nature de la série $\displaystyle\sum_{n=1}^{+\infty} \ln\left(1 + \frac{1}{n^2}\right)$.\\

    \item 
    Soit la série $\displaystyle\sum_{n=1}^{+\infty} u_n$ avec $u_n = \frac{n^2 + 3}{n^3 + 1}$. Étudier la convergence à l'aide d'un équivalent du terme général.\\

    \item 
    Soit $S = \displaystyle\sum_{n=0}^{+\infty} \frac{(-1)^n x^{2n+1}}{(2n+1)!}$. Montrer que cette série converge pour tout $x \in \mathbb{R}$ et identifier sa somme.\\
\end{enumerate}


\section{Développements limités}



\begin{enumerate}
    \item 
    Donner les développements limités des fonctions $e^x$ et $\ln(1+x)$ à l'ordre $3$ au voisinage de $0$.\\

    \item
    Utiliser les développements limités obtenus pour calculer le développement limité de $f(x) = e^x \cdot \ln(1+x)$ à l'ordre $3$ au voisinage de $0$.\\

    \item 
    En partant du développement limité de $\dfrac{1}{1+x}$, déterminer le développement limité de $g(x) = \dfrac{x}{1+2x}$ à l'ordre $3$ au voisinage de $0$ en détaillant la démarche.\\
\end{enumerate}

\section{Courbes paramétrées (polaires)}

\noindent\textbf{Exercice  : 10 points}\\


On considère la courbe définie en coordonnées polaires par :
$$
r(\theta) = 1 + \cos \theta, \quad \theta \in [0 ; 2\pi]
$$

\begin{enumerate}
    \item 
    Étudier les symétries de la courbe et réduire si possible l'intervalle d'étude. \textbf{0 1  2 point}\\

    \item 
    Donner les points remarquables (tangentes verticales, tangentes horizontales, ...). \textbf{0 1 2 3 points}\\

    \item 
    Calculer les coordonnées cartésiennes du point correspondant à $\theta = \frac{\pi}{2}$. \textbf{0 1  2 point}\\

    \item 
    Déterminer la longueur de la courbe sur l'intervalle $[0 ; 2\pi]$. \textbf{0 1 2 3  points}\\
\end{enumerate}



%\textbf{Correction :}
%\begin{enumerate}
%    \item 
%    On a $r(\theta) = 1 + \cos \theta$.
%
%    La fonction $\cos \theta$ est paire : $\cos(-\theta) = \cos(\theta)$, donc $r(-\theta) = r(\theta)$.
%
%    Cela implique que la courbe est **symétrique par rapport à l'axe polaire** (c'est-à-dire l'axe des abscisses). 
%
%    On peut donc réduire l'étude à l'intervalle $[0 ; \pi]$.
%
%    \item 
%    
%
%    - Pour $\theta = 0$, $r(0) = 1 + 1 = 2$.\\
%    - Pour $\theta = \frac{\pi}{2}$, $r = 1$.\\
%    - Pour $\theta = \pi$, $r = 1 + \cos(\pi) = 1 - 1 = 0$.\\
%
%\begin{tikzpicture}[scale=3]
%  \draw[->] (-1.2,0) -- (2.2,0) node[right] {$x$};
%  \draw[->] (0,-1.2) -- (0,1.2) node[above] {$y$};
%  
%  \draw[domain=0:360,smooth,variable=\t,samples=200,blue,thick]
%    plot ({(1+cos(\t))*cos(\t)},{(1+cos(\t))*sin(\t)});
%  
%  \node at (1.4,1) {$r(\theta) = 1 + \cos(\theta)$};
%\end{tikzpicture}
%
%
%    \item 
%    À $\theta = \frac{\pi}{2}$, on a :
%    \[
%    r = 1 + \cos\left(\frac{\pi}{2}\right) = 1
%    \]
%    Les coordonnées cartésiennes sont données par :
%    \[
%    x = r \cos \theta = 1 \cdot \cos\left(\frac{\pi}{2}\right) = 0,\quad y = r \sin \theta = 1 \cdot \sin\left(\frac{\pi}{2}\right) = 1
%    \]
%    Donc le point a pour coordonnées : $(0 ; 1)$.
%
%    \item 
%    La formule de la longueur d'une courbe en coordonnées polaires est :
%    \[
%    L = \int_{0}^{2\pi} \sqrt{r(\theta)^2 + \left(r'(\theta)\right)^2} \, d\theta
%    \]
%    On a :
%    \[
%    r(\theta) = 1 + \cos \theta,\quad r'(\theta) = -\sin \theta
%    \]
%    Donc :
%    \[
%    r^2 + (r')^2 = (1 + \cos \theta)^2 + \sin^2 \theta = 1 + 2\cos \theta + \cos^2 \theta + \sin^2 \theta = 2 + 2\cos \theta
%    \]
%    Donc :
%    \[
%    L = \int_{0}^{2\pi} \sqrt{2(1 + \cos \theta)} \, d\theta
%    \]
%    On reconnaît $1 + \cos \theta = 2 \cos^2\left(\frac{\theta}{2}\right)$, donc :
%    \[
%    \sqrt{2(1 + \cos \theta)} = \sqrt{4 \cos^2\left(\frac{\theta}{2}\right)} = 2 \left|\cos\left(\frac{\theta}{2}\right)\right|
%    \]
%
%    Comme $\cos\left(\frac{\theta}{2}\right) \ge 0$ pour $\theta \in [0, \pi]$ et $\le 0$ pour $\theta \in [\pi, 2\pi]$, on obtient :
%    \[
%    L = \int_0^\pi 2\cos\left(\frac{\theta}{2}\right) d\theta - \int_\pi^{2\pi} 2\cos\left(\frac{\theta}{2}\right) d\theta
%    \]
%
%    Posons $u = \frac{\theta}{2}$, $d\theta = 2du$, les bornes deviennent respectivement :
%    \[
%    \theta \in [0 ; \pi] \Rightarrow u \in [0 ; \frac{\pi}{2}],\quad \theta \in [\pi ; 2\pi] \Rightarrow u \in [\frac{\pi}{2} ; \pi]
%    \]
%    Donc :
%    \[
%    L = 2 \cdot 2 \int_0^{\frac{\pi}{2}} \cos(u) du - 2 \cdot 2 \int_{\frac{\pi}{2}}^{\pi} \cos(u) du
%    \]
%    \[
%    = 4 \left[ \sin(u) \right]_0^{\frac{\pi}{2}} - 4 \left[ \sin(u) \right]_{\frac{\pi}{2}}^{\pi} = 4(\sin(\frac{\pi}{2}) - 0) - 4(\sin(\pi) - \sin(\frac{\pi}{2})) = 4(1) - 4(0 - 1) = 4 + 4 = 8
%    \]
%
%    Donc la **longueur de la courbe** est :
%    \[
%    \boxed{L = 8}
%    \]
%\end{enumerate}