\section{Développements limités}



\begin{enumerate}
    \item 
    Donner les développements limités des fonctions $\sin(x)$ et $\cos(x)$ à l'ordre $3$ au voisinage de $0$.

    \item
    Utiliser les développements limités obtenus pour calculer le développement limité de $h(x) = \sin(x) \cdot \cos(x)$ à l'ordre $3$.

    \item 
    En partant du développement limité de $\dfrac{1}{1-x}$, calculer $\ln\left(1-x\right)$ à l'ordre $3$ au voisinage de $0$ en détaillant la démarche.
\end{enumerate}


\section{Courbes paramétrées}

L'astroïde est la courbe de coordonnées cartésiennes (où $t \in \mathbb{R}$) :
$$
\left \{
\begin{array}{lc}
x(t) =& \cos^3(t)\\
y(t) = &\sin^3(t)\\
\end{array}
\right.
$$

\begin{enumerate}
    \item 
   Réduire l'intervalle d'étude. \textbf{ 0 1 point}
\item Calculer les tangentes de la courbe pour $t \in \left[0;\frac{\pi}{2} \right]$.\textbf{0 1 2 3 points}
\item En déduire les points stationnaires.\textbf{0 1 2 points}
    \item Calculer la longueur de la courbe.\textbf{0 1 2 3 4 points}


\end{enumerate}



% \section*{Solutions}
% Pour rappel, la correction est extrêmement détaillée pour vous aider, un tel niveau n'était pas attendu.\\



% \begin{enumerate}

%     \item
%     Les fonctions $x$ et $y$ sont définies sur $\mathbb{R}$.\\
    
%     L'équation paramétrique est $2\pi$-périodique. L'ensemble d'étude est un intervalle de longueur $2\pi$.\\
    
%     $\forall t \in \mathbb{R}, \qquad -t \in\mathbb{R}, \qquad x(-t)= x(t) \text{ et } y(-t)=y(t)$.\\
%     Les point $M_t$ et $M_{-t}$ sont symétriques par rapport à l'axe des abscisses.\\
%     On réduit l'ensemble d'étude à $[0\pi]$.\\
    
%     On peut également remarquer que :\\
%     \begin{itemize} \item 
%     $\forall t \in \mathbb{R}, \qquad t+\pi \in\mathbb{R}, \qquad x(t+\pi)=- x(t) \text{ et } y(t+\pi)=-y(t)$.\\
%     Les point $M_t$ et $M_{t+\pi}$ sont symétriques par rapport à l'origine.\\
%     \item
%        $\forall t \in \mathbb{R}, \qquad \frac{\pi}{2}-t \in\mathbb{R}, \qquad x( \frac{\pi}{2}-t )=y(t) \text{ et } y( \frac{\pi}{2}-t )=x(t)$.\\ 
%         Les point $M_t$ et $M_{\frac{\pi}{2}-t}$ sont symétriques par rapport à la droite d'équation $y=x$.\\
%     \end{itemize}
    
%     On peut donc réduire l'ensemble d'étude à $t \in \left[0, \frac{\pi}{4}\right]$.\\
    
%     \item 
    
% Les fonctions $x$ et $y$ sont dérivables sur $\left[0;\frac{\pi}{2} \right]$
%     \[
%  x'(t) = -3\cos^2(t)\sin(t) \quad \text{et} \quad y'(t) = 3\sin^2(t)\cos(t).
%     \]

%     Pour $t \in \left[0, \frac{\pi}{2}\right]$, la tangente en chaque point $(x(t), y(t))$ a pour équation :
%     \[
%     y - \sin^3(t) = -\tan(t)(x - \cos^3(t)).
%     \]
    
%     \item
    
%     Les points stationnaires correspondent à $x'(t) = 0$ et $y'(t) = 0$ avec
%       \[
%  x'(t) = -3\cos^2(t)\sin(t) \quad \text{et} \quad y'(t) = 3\sin^2(t)\cos(t).
%     \]  
%     On obtient 
%     $$t=0 +k\pi \text{ et } t=\frac{\pi}{2}+k\pi \text{ avec } k \in \mathbb{R}$$
%    Restreignons nous uniquement à l'intervalle  d'étude $t \in \left[0, \frac{\pi}{2}\right]$, les autres points peuvent être obtenus par symétrie.\\
%    $t=0$ et $t=\frac{\pi}{2}$
%     Pour $t = 0$, le point est $(x, y) = (1, 0)$. Pour $t = \frac{\pi}{2}$, le point est $(x, y) = (0, 1)$.
%     Les points stationnaires ont pour coordonnées $(1, 0)$ et $(0, 1)$, puis par symétrie $(1, -1)$ et $(1, 1)$.\\

%     \item
%     La longueur de la courbe pour $t \in \left[0, \frac{\pi}{2}\right]$ est donnée par :
%     \[
%     L = \int_0^{\frac{\pi}{2}} \sqrt{\left(x'(t)\right)^2 + \left(y'(t)\right)^2} \, dt.
%     \]
%     Or
%     \[
%     \forall t \in \mathbb{R} \qquad
%     \sqrt{\left(x'(t)\right)^2 + \left(y'(t)\right)^2} = 3|\cos(t)\sin(t)|\sqrt{\cos^2(t) + \sin^2(t)} = 3|\cos(t)\sin(t)|.
%     \]
%     Donc :
%     \[
%     L = \int_0^{\frac{\pi}{2}} 3\cos(t)\sin(t) \, dt = 3\int_0^{\frac{\pi}{2}} \frac{\sin(2t)}{2} \, dt = \frac{3}{2} \int_0^{\frac{\pi}{2}} \sin(2t) \, dt
%  \frac{3}{2} \left[-\frac{1}{2}\cos(2t)\right]_0^{\frac{\pi}{2}} = \frac{3}{2} \cdot \frac{1}{2} \cdot (1 + 1) = \frac{3}{2}.
%     \]
%     La longueur de la courbe est $L = \frac{3}{2}$.
    
% \end{enumerate}

\section{Courbes paramétrées, 2eme version}

On considère la courbe de coordonnées cartésiennes (où $t \in \mathbb{R}$) :
$$
\left \{
\begin{array}{lc}
x(t) =& \cos^2(t)\\
y(t) = &\sin^2(t)\\
\end{array}
\right.
$$

\begin{enumerate}
    \item 
   Réduire l'intervalle d'étude. \textbf{ 0 1 point}\\
\item Calculer les tangentes de la courbe pour $t \in \left[0;\pi \right]$.\textbf{0 1 2 3 points}\\
\item En déduire les points stationnaires.\textbf{0 1 2 points}\\
    \item
   Calculer la longueur de la courbe.\textbf{0 1 2 3 4 points}\\


\section{Séries numériques}


Trois questions pouvaient être tirées parmi le lot suivant

\begin{enumerate}
    \item 
    Soit la série $\displaystyle\sum_{n=1}^{+\infty} \frac{1}{n^2 + 3n + 2}$. Montrer qu'il s'agit d'une série à termes positifs et conclure sur sa convergence par comparaison.\\

    \item 
    Étudier la convergence de la série $\displaystyle\sum_{n=2}^{+\infty} \frac{1}{n (\ln n)^p}$ selon les valeurs du paramètre $p > 0$.\\

    \item 
    Étudier la convergence de la série $\displaystyle\sum_{n=1}^{+\infty} \frac{n^2}{n^3 + x}$ selon la valeur du réel $x$.\\

    \item 
    Pour quelles valeurs de $x \in \mathbb{R}$ la série $\displaystyle\sum_{n=1}^{+\infty} \frac{x^n}{n}$ converge-t-elle ?\\

    \item 
    Étudier la nature de la série $\displaystyle\sum_{n=1}^{+\infty} \frac{(-1)^n}{n^p}$ selon les valeurs du paramètre réel $p > 0$.\\

    \item 
    Soit $f(x) = \displaystyle\sum_{n=1}^{+\infty} \frac{x^n}{n^2}$. Déterminer l'ensemble de convergence de cette série, puis étudier la continuité de $f$ sur cet ensemble.\\

    \item 
    On considère la série $\displaystyle\sum_{n=1}^{+\infty} \frac{(-1)^n}{n^2 + x}$. Pour quelles valeurs de $x$ la série est-elle convergente ? Converge-t-elle absolument ?\\

    \item 
    Étudier la convergence de la série $\displaystyle\sum_{n=1}^{+\infty} \frac{1}{n^\alpha \ln^\beta n}$ selon les valeurs des paramètres $\alpha > 0$ et $\beta \in \mathbb{R}$.\\

    \item 
    Pour $x > 0$, étudier la convergence de la série $\displaystyle\sum_{n=1}^{+\infty} \frac{1}{n^x + n}$.\\

    \item 
    Soit $S(x) = \displaystyle\sum_{n=1}^{+\infty} \left(\frac{x}{1+x}\right)^n$. Donner l'ensemble de convergence de cette série en fonction de $x$. Calculer ensuite $S(x)$ lorsque la série converge.\\
\end{enumerate}

\end{enumerate}

