\section{Espaces vectoriels}

    \subsection{Des ensembles que vous connaissez déjà}

    \subsection{Définition d'un espace vectoriel}

        \begin{Def}\textbf{Espace vectoriel}\\
        Soit \(K\) un corps (toujours \(\R\) ou \(\C\)).\\
        Un \textbf{espace vectoriel} sur \(K\) est un ensemble \(E\) muni de deux opérations :
        \begin{itemize}
            \item une addition interne \( + : E \times E \to E \),
            \item une multiplication externe \( \cdot : K \times E \to E \)
        \end{itemize}
        telles que :
        \begin{enumerate}
            \item l'addition est commutative : \(u + v = v + u\),
            \item il existe un élément neutre appelé \textbf{vecteur nul} noté \(\vec{0}\),
            \item tout élément \(u\) admet un opposé noté \(-u\), leur somme donne le vecteur nul. 
            \item \(\forall \lambda, \mu \in K\) et \(\forall u, v \in E\) :
            \begin{align*}
                \lambda \cdot (u + v) &= \lambda \cdot u + \lambda \cdot v, \\
                (\lambda + \mu) \cdot u &= \lambda \cdot u + \mu \cdot u, \\
                (\lambda \mu) \cdot u &= \lambda \cdot (\mu \cdot u), \\
                1_K \cdot u &= u.
            \end{align*}
        \end{enumerate}
        \end{Def}

        \begin{Ex}
        \begin{itemize}
            \item $\emptyset$ n'est pas un espace vectoriel sur $\mathbb{R}$.
            \item $\{0\}$ est un espace vectoriel sur $\mathbb{R}$.
            \item \(\R^n\) est un espace vectoriel sur \(\R\).
            \item L'ensemble des suites réelles est un espace vectoriel sur $\mathbb{R}$. (qui pourrait s'écrire $\mathbb{R}^{\infty}$, on l'écrit généralement $\mathbb{R}^{\mathbb{N}}$).
            \item L'ensemble \(\mathcal{M}_{n,p}(\R)\) des matrices réelles de taille \(n \times p\) est un espace vectoriel sur \(\R\).
            \item L'ensemble des fonctions continues sur \([0,1]\), noté \(\mathcal{C}([0,1],\R)\), est un espace vectoriel.
        \end{itemize}
        \end{Ex}

        Faire trouver le vecteur nul, l'opposé d'un vecteur, etc. 

        Faire trouver un vecteur qui n'est pas dans l'espace vectoriel. 

        Au-delà des vecteurs \og géométriques \fg , d'autres ensembles possèdent les mêmes propriétés et constituent des espaces vectoriels :
        \begin{itemize}
            \item l'ensemble des polynômes à coefficients réels,
            \item l'ensemble des matrices réelles de taille donnée,
            \item l'ensemble des suites réelles,
            \item l'ensemble des fonctions définies sur un domaine fixé,
            \item l'ensemble des images en niveaux de gris de format \(400 \times 400\) pixels (considérées comme des matrices de réels).
        \end{itemize}

        Ainsi, on peut :
        \begin{itemize}
            \item calculer le milieu de deux polynômes par la formule du milieu utilisée sur les vecteurs,
            \item définir un barycentre pour trois matrices,
            \item mais il n'existe pas toujours de notion de distance ou d'angle entre vecteurs dans un espace vectoriel sans structure supplémentaire (produit scalaire).
        \end{itemize}

        \begin{Rmq}
        Pour définir des notions de distance ou d'angle (comme dans le plan usuel), il faut introduire un \textbf{produit scalaire}, ce qui n'est pas inclus dans la définition d'espace vectoriel. Sans cela, aucune base n'est meilleure qu'une autre — la notion de base orthogonale n'existe pas toujours.
        \end{Rmq}

    

    \subsection{Sous-espaces vectoriels}

        \begin{Def}\textbf{Sous-espace vectoriel}\\
        Un sous-ensemble \(F\) d'un espace vectoriel \(E\) est appelé \textbf{sous-espace vectoriel} si :
        \begin{itemize}
            \item \(\vec{0} \in F\),
            \item \(F\) est stable par addition : \(u, v \in F \implies u+v \in F\),
            \item \(F\) est stable par multiplication externe : \(\lambda \in K, u \in F \implies \lambda u \in F\).
        \end{itemize}
        \end{Def}

        \begin{Ex}
        \begin{itemize}
            \item L'ensemble des vecteurs du plan \(xy\) dans \(\R^3\) est un sous-espace vectoriel de \(\R^3\).
            \item L'ensemble des polynômes de degré au plus 2 est un sous-espace vectoriel de l'espace des polynômes.
        \end{itemize}
        \end{Ex}

        \begin{Rmq}
        L'intersection de deux sous-espaces vectoriels est toujours un sous-espace vectoriel. La réunion de deux sous-espaces vectoriels n'est pas toujours un sous-espace vectoriel.
        \end{Rmq}

        Sommes de sous-espaces vectoriels. 
        \begin{Def}\textbf{Somme de sous-espaces vectoriels}\\
        Soient \(F\) et \(G\) deux sous-espaces vectoriels d'un espace vectoriel \(E\). La \textbf{somme} de \(F\) et \(G\) est l'ensemble :
        \[
        F + G = \{ u + v \mid u \in F, v \in G \}.
        \]
        C'est le plus petit sous-espace vectoriel de $E$ contenant $F$ et $G$. 
        \end{Def}
        Des exemples.

        Somme directe ?
        Une somme qui n'est pas directe ? 

\section{Dimension}
    \subsection{Combinaisons linéaires et familles de vecteurs}
        \begin{itemize}
            \item En dimension \(1\), tous les vecteurs sont colinéaires : 
            on peut choisir celui qu'on veut (sauf le vecteur nul) et décrire tous les autres comme des multiples de ce vecteur.
            \item En dimension \(2\), il faut choisir deux vecteurs non colinéaires pour former une base. Il y a une infinité de choix possibles.
            \item En dimension \(3\), même principe : il faut trois vecteurs non coplanaires.
            \item En dimension \(n\), il faut \(n\) vecteurs non coplanaires.
        \end{itemize}

        Une fois une base choisie, on décrit un vecteur par ses \textbf{coordonnées}, c'est-à-dire une liste de scalaires. Ces coordonnées s'écrivent souvent sous forme de colonne (ce qui rapproche la notion de vecteur de celle de matrice colonne).

        Les opérations sur les vecteurs se traduisent alors par des opérations sur leurs coordonnées :
        \begin{itemize}
            \item l'addition de vecteurs correspond à l'addition coordonnée par coordonnée,
            \item la multiplication par un scalaire également.
        \end{itemize}

        \begin{Def}\textbf{Combinaison linéaire}

        Soient \(u_1, \dots, u_p \in E\). Tout vecteur de la forme
        \[
        \lambda_1 u_1 + \cdots + \lambda_p u_p, \quad \lambda_i \in K
        \]
        est appelé une \textbf{combinaison linéaire} de \(u_1, \dots, u_p\).
        \end{Def}

        \begin{Def}\textbf{Sous-espace engendré}\\
        L'ensemble de toutes les combinaisons linéaires de \(u_1, \dots, u_p\) est appelé le sous-espace vectoriel \textbf{engendré} par \(u_1, \dots, u_p\) et se note :
        \[
        \text{Vect}(u_1, \dots, u_p).
        \]
        \end{Def}
        \begin{Def} Famille génératrice
        On dit que la famille \((u_1, \dots, u_p)\) est \textbf{génératrice} de \(E\) si :
        \[
        \text{Vect}(u_1, \dots, u_p) = E.
        \]
        \end{Def}

        
        \begin{Def}\textbf{Indépendance linéaire}\\
        Une famille \((u_1, \dots, u_p)\) de vecteurs est dite \textbf{libre} (ou linéairement indépendante) si :
        \[
        \lambda_1 u_1 + \cdots + \lambda_p u_p = \vec{0}
        \implies \lambda_1 = \cdots = \lambda_p = 0.
        \]
        \end{Def}

        \begin{Rmq}
        Toute sous-famille d'une famille libre est libre.
        \end{Rmq}

        \begin{Def}\textbf{Base et dimension}\\
        Une famille \((u_1, \dots, u_p)\) est une \textbf{base} de \(E\) si elle est :
        \begin{itemize}
            \item libre,
            \item génératrice de \(E\).
        \end{itemize}
        Dans ce cas, le nombre \(p\) de vecteurs est appelé la \textbf{dimension} de \(E\).
        \end{Def}

        \begin{Thm}\textbf{Théorème de la base incomplète}\\
        Toute famille libre de vecteurs dans un espace vectoriel \(E\) peut être complétée en une base de \(E\).
        \end{Thm}

        \begin{Thm}\textbf{Théorème de la dimension}\\
        Toutes les bases d'un espace vectoriel \(E\) ont le même nombre d'éléments.
        \end{Thm}

        \begin{Rmq}
        Pour montrer qu'une famille est une base d'un espace vectoriel, il suffit de prouver :
        \begin{itemize}
            \item soit qu'elle est libre et qu'elle contient \(\dim E\) vecteurs,
            \item soit qu'elle est génératrice et qu'elle contient \(\dim E\) vecteurs.
        \end{itemize}
        \end{Rmq}


        \begin{Def} Famille libre
        On dit que la famille \((u_1, \dots, u_p)\) est \textbf{libre} si :
        \[
        \lambda_1 u_1 + \cdots + \lambda_p u_p = \vec{0}
        \implies \lambda_1 = \cdots = \lambda_p = 0.
        \]
        \end{Def}

        \begin{Def} Famille liée
        On dit que la famille \((u_1, \dots, u_p)\) est \textbf{liée} si elle n'est pas libre.
        \end{Def}

       Famille liée

        \begin{Def} Base
        Une famille \((u_1, \dots, u_p)\) est une \textbf{base} de \(E\) si elle est libre et génératrice.
        \end{Def}

        \begin{Thm} Théorème de la dimension

            Toutes les bases d'un espace vectoriel \(E\) ont le même nombre d'éléments.
            Ce nombre est appelé la \textbf{dimension} de \(E\).
        \end{Thm}

    \subsection{Exercices}


        \exo[2]{Sous-espace vectoriel}

        Montrer que l'ensemble \(F = \{ (x,y,z)\in \mathbb{R}^3 \mid x - y + 2z = 0 \}\) est un sous-espace vectoriel de \(\mathbb{R}^3\). Déterminer une base de \(F\) et sa dimension.

        \vspace{1em}
        \hrule
        \vspace{1em}

        \exo[1]{Liberté d'une famille}

        Déterminer si la famille \(\big( (1,2), (3,6) \big)\) est libre dans \(\mathbb{R}^2\). Même question dans \(\mathbb{R}^3\).

        \vspace{1em}
        \hrule
        \vspace{1em}

        \exo[2]{Dimension d'un espace de polynômes}

        Montrer que l'espace des polynômes réels de degré au plus 4 est de dimension 5. Proposer une base.

        \vspace{1em}
        \hrule
        \vspace{1em}

        \exo[2]{Famille génératrice}

        Soit \( E = \mathbb{R}^3 \). La famille \(\big\{ (1,0,1), (0,1,1), (1,1,2)\big\}\) est-elle une base de \( E \) ?

        \vspace{1em}
        \hrule
        \vspace{1em}
    
\section{Applications linéaires}
\section{Matrices}
