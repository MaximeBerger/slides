\documentclass[10pt]{article}

\usepackage[utf8]{inputenc}
\usepackage[french]{babel}
\usepackage{graphicx}
%\usepackage{wrapfig}
\usepackage{dsfont}
\usepackage[margin=2cm]{geometry}
\usepackage{amsmath,amssymb}
%\usepackage{tikz}
\usepackage{multicol}
\newcommand{\xB}{{\cal B}}
\newcommand{\xC}{{\cal C}}
\newcommand{\R}{\mathbf{R}}

\begin{document}

\pagestyle{empty}

\noindent
\begin{minipage}[l]{8cm}
  \scriptsize{ESTP, S3\\Ann\'ee universitaire 2025/2026}
\end{minipage}

\begin{center}
  {\large\textbf{Petit test } \\
10 minutes\\}
  \bigskip
  

 
\end{center}

\bigskip

\begin{center}
  \fbox{%
    \begin{minipage}{0.95\linewidth}
      NOM: \hspace*{5cm} Pr\'enom: \hspace*{4cm} Groupe: TD 
    \end{minipage}
  }
  \end{center}

\bigskip
 
  Deux questions peuvent être prises parmi cette liste.

\noindent\textbf{Exercice  : 5 points}\\


A chaque fois, un dessin illustrera la situation.\\

\begin{enumerate}
    \item 
    Soit $\Gamma$ l'arc de cercle orienté positivement défini par $x(t) = \cos t$, $y(t) = \sin t$, pour $t \in [0, \pi]$. Calculer $\displaystyle \int_\Gamma x\,dx + y\,dy$.\\

    \item 
    Soit $C$ le segment orienté du point $A(0,0)$ au point $B(1,1)$. Calculer $\displaystyle \int_C (x + y)\,dx + (x - y)\,dy$.\\

    \item 
    Soit $\Gamma$ le demi-cercle supérieur de rayon 1 centré à l'origine, paramétré par $x(t) = \cos t$, $y(t) = \sin t$ pour $t \in [0, \pi]$. Calculer $\displaystyle \int_\Gamma x^2\,dy$.\\

    \item 
    Soit $C$ le contour du triangle orienté positivement dont les sommets sont $O(0,0)$, $A(1,0)$, $B(1,1)$. Calculer $\displaystyle \int_C x\,dy - y\,dx$.\\

    \item 
    Soit $\Gamma$ la courbe paramétrée par $x(t) = t$, $y(t) = t^2$, pour $t \in [0, 1]$. Calculer $\displaystyle \int_\Gamma (x^2 + y)\,ds$, où $ds$ est l'élément de longueur.\\

    \item 
    Montrer que l'intégrale curviligne $\displaystyle \int_\Gamma y\,dx - x\,dy$ sur un cercle orienté positivement centré à l'origine est égale à $2\pi R^2$, où $R$ est le rayon du cercle.\\

    \item 
    Soit $\Gamma$ la portion de la parabole $y = x^2$ allant de $(-1,1)$ à $(1,1)$. Calculer $\displaystyle \int_\Gamma x\,dy$ en utilisant une paramétrisation adaptée.\\

    \item 
    Soit $\vec{F}(x, y) = (y, -x)$ et soit $\Gamma$ le bord du carré de sommets $(\pm1, \pm1)$ orienté positivement. Calculer $\displaystyle \int_\Gamma \vec{F} \cdot d\vec{r}$.\\

    \item 
    Soit $\vec{F}(x, y) = (x^2, y^2)$ et $\Gamma$ le cercle unité orienté positivement. Calculer $\displaystyle \int_\Gamma \vec{F} \cdot d\vec{r}$.\\

    \item 
    Soit $\Gamma$ une courbe paramétrée par $x(t) = \cos t$, $y(t) = \sin 2t$ pour $t \in [0, \pi]$. Calculer $\displaystyle \int_\Gamma x\,dy - y\,dx$.\\
\end{enumerate}


\end{document}
