\documentclass[10pt]{article}

\usepackage[utf8]{inputenc}
\usepackage[french]{babel}
\usepackage{graphicx}
%\usepackage{wrapfig}
\usepackage{dsfont}
\usepackage[margin=2cm]{geometry}
\usepackage{amsmath,amssymb}
%\usepackage{tikz}
\usepackage{multicol}
\newcommand{\xB}{{\cal B}}
\newcommand{\xC}{{\cal C}}
\newcommand{\R}{\mathbf{R}}

\begin{document}

\pagestyle{empty}

\noindent
\begin{minipage}[l]{8cm}
  \scriptsize{ESTP, S3\\Ann\'ee universitaire 2024/2025}
\end{minipage}

\begin{center}
  {\large\textbf{Petit test } \\
10 minutes\\}
  \bigskip
  

 
\end{center}

\bigskip

\begin{center}
  \fbox{%
    \begin{minipage}{0.95\linewidth}
      NOM: \hspace*{5cm} Pr\'enom: \hspace*{4cm} Groupe: TD 
    \end{minipage}
  }
  \end{center}

\bigskip
 
 

\noindent\textbf{Exercice  : 3 points}\\


\begin{enumerate}
    \item 

Donner la définition d'une tribu.
 \item
 Donner la tribu triviale.
\end{enumerate}

\noindent\textbf{Exercice  : 2 points}\\


Donner la définition d'un système complet d'événements.



\section*{Solutions}

\begin{enumerate}
    \item \textbf{Définition d'une tribu :} \\
    
    Soit $\Omega$ un univers. Une \textbf{tribu} $\mathcal{F}$ sur $\Omega$ est une collection de parties de $\Omega$ qui vérifie les propriétés suivantes :
    \begin{enumerate}
        \item $\Omega \in \mathcal{F}$.
        \item Si $A \in \mathcal{F}$ alors son complémentaire $A^c \in \mathcal{F}$.
        \item Si $\{A_n\}_{n \in \mathbb{N}}$ est une suite d'éléments de $\mathcal{F}$, alors leur union $\bigcup\limits_{n=1}^{\infty} A_n$ appartient aussi à $\mathcal{F}$.
    \end{enumerate}
    
    \item \textbf{La tribu triviale :} \\
    
    La \textbf{tribu triviale} sur un univers $\Omega$ est la plus petite tribu possible. Elle est donnée par :
    \[
    \mathcal{F} = \{\emptyset, \Omega\}.
    \]

\end{enumerate}

\noindent\textbf{Exercice : 2 points} \\

\textbf{Définition d'un système complet d'événements :} \\

Un ensemble fini ou dénombrable d'événements $\{A_i\}_{i \in I}$ est un \textbf{système complet d'événements} si :
\begin{enumerate}
    \item Les événements sont deux à deux disjoints : $A_i \cap A_j = \emptyset$ pour tout $i \neq j$.
    \item Leur union couvre tout l'univers : $\bigcup\limits_{i \in I} A_i = \Omega$.
\end{enumerate}

\end{document}



\end{document}
