\documentclass[10pt]{article}

\usepackage[utf8]{inputenc}
\usepackage[french]{babel}
\usepackage{graphicx}
%\usepackage{wrapfig}
\usepackage{dsfont}
\usepackage[margin=2cm]{geometry}
\usepackage{amsmath,amssymb}
%\usepackage{tikz}
\usepackage{multicol}
\newcommand{\xB}{{\cal B}}
\newcommand{\xC}{{\cal C}}
\newcommand{\R}{\mathbf{R}}

\begin{document}

\pagestyle{empty}

\noindent
\begin{minipage}[l]{8cm}
  \scriptsize{ESTP, S3\\Ann\'ee universitaire 2025/2026}
\end{minipage}

\begin{center}
  {\large\textbf{Petit test \no 1} \\
10 minutes\\}
  \bigskip
  

 
\end{center}

\bigskip

\begin{center}
  \fbox{%
    \begin{minipage}{0.95\linewidth}
      NOM: \hspace*{5cm} Pr\'enom: \hspace*{4cm} Groupe: TD 
    \end{minipage}
  }
  \end{center}

\bigskip
 
 

\noindent\textbf{Exercice  : 5 points}\\



\begin{enumerate}
    \item 
    Donner les développements limités des fonctions $e^x$ et $\ln(1+x)$ à l'ordre $3$ au voisinage de $0$.\\

    \item
    Utiliser les développements limités obtenus pour calculer le développement limité de $f(x) = e^x \cdot \ln(1+x)$ à l'ordre $3$ au voisinage de $0$.\\

    \item 
    En partant du développement limité de $\dfrac{1}{1+x}$, déterminer le développement limité de $g(x) = \dfrac{x}{1+2x}$ à l'ordre $3$ au voisinage de $0$ en détaillant la démarche.\\
\end{enumerate}


\end{document}
