\documentclass[10pt]{article}

\usepackage[utf8]{inputenc}
\usepackage[french]{babel}
\usepackage{graphicx}
%\usepackage{wrapfig}
\usepackage{dsfont}
\usepackage[margin=2cm]{geometry}
\usepackage{amsmath,amssymb}
%\usepackage{tikz}
\usepackage{multicol}
\newcommand{\xB}{{\cal B}}
\newcommand{\xC}{{\cal C}}
\newcommand{\R}{\mathbf{R}}

\begin{document}

\pagestyle{empty}

\noindent
\begin{minipage}[l]{8cm}
  \scriptsize{ESTP, S3\\Ann\'ee universitaire 2025/2026}
\end{minipage}

\begin{center}
  {\large\textbf{Petit test } \\
10 minutes\\}
  \bigskip
  

 
\end{center}

\bigskip

\begin{center}
  \fbox{%
    \begin{minipage}{0.95\linewidth}
      NOM: \hspace*{5cm} Pr\'enom: \hspace*{4cm} Groupe: TD 
    \end{minipage}
  }
  \end{center}

\bigskip
 
 
Trois questions seront choisies dans la liste suivante.\\
\noindent\textbf{Exercice  : 5 points}\\


\begin{enumerate}
    \item 
    En utilisant un critère de comparaison, étudier la convergence de la série $\displaystyle\sum_{n=2}^{+\infty} \frac{1}{n(\ln n)^2}$.\\

    \item
    En utilisant un critère de Riemann, déterminer la nature de la série $\displaystyle\sum_{n=1}^{+\infty} \frac{1}{n^\alpha}$ selon les valeurs du réel $\alpha$.\\

    \item 
    En utilisant le critère de la série géométrique, étudier la convergence de la série $\displaystyle\sum_{n=0}^{+\infty} \left(\frac{2}{3}\right)^n$.\\

    \item 
    En utilisant un critère de comparaison série-intégrale, déterminer la nature de la série $\displaystyle\sum_{n=2}^{+\infty} \frac{1}{n\ln n}$.\\

    \item 
    Étudier la convergence absolue ou non de la série $\displaystyle\sum_{n=1}^{+\infty} \frac{(-1)^n}{n}$.\\

    \item 
    En appliquant le critère des séries alternées, déterminer la nature de la série $\displaystyle\sum_{n=1}^{+\infty} \frac{(-1)^n}{n \sqrt{n}}$.\\

    \item 
    La série $\displaystyle\sum_{n=1}^{+\infty} \frac{1}{n(n+1)}$ est-elle convergente ? Peut-on calculer sa somme en utilisant la méthode des séries télescopiques ?\\

    \item 
    En développant le terme général, déterminer la nature de la série $\displaystyle\sum_{n=1}^{+\infty} \ln\left(1 + \frac{1}{n^2}\right)$.\\

    \item 
    Soit la série $\displaystyle\sum_{n=1}^{+\infty} u_n$ avec $u_n = \frac{n^2 + 3}{n^3 + 1}$. Étudier la convergence à l’aide d’un équivalent du terme général.\\

    \item 
    Soit $S = \displaystyle\sum_{n=0}^{+\infty} \frac{(-1)^n x^{2n+1}}{(2n+1)!}$. Montrer que cette série converge pour tout $x \in \mathbb{R}$ et identifier sa somme.\\
\end{enumerate}


\end{document}
