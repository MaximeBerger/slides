\documentclass[10pt]{article}

\usepackage[utf8]{inputenc}
\usepackage[french]{babel}
\usepackage{graphicx}
%\usepackage{wrapfig}
\usepackage{dsfont}
\usepackage[margin=2cm]{geometry}
\usepackage{amsmath,amssymb}
%\usepackage{tikz}
\usepackage{multicol}
\newcommand{\xB}{{\cal B}}
\newcommand{\xC}{{\cal C}}
\newcommand{\R}{\mathbf{R}}

\begin{document}

\pagestyle{empty}

\noindent
\begin{minipage}[l]{8cm}
  \scriptsize{ESTP, S3\\Ann\'ee universitaire 2024/2025}
\end{minipage}

\begin{center}
  {\large\textbf{Petit test \no 2} \\
20 minutes\\}
  \bigskip
  

 
\end{center}

\bigskip

\begin{center}
  \fbox{%
    \begin{minipage}{0.95\linewidth}
      NOM: \hspace*{5cm} Pr\'enom: \hspace*{4cm} Groupe: TD 
    \end{minipage}
  }
  \end{center}

\bigskip
 
 

\noindent\textbf{Exercice  : 10 points}\\


On considère la courbe de coordonnées cartésiennes (où $t \in \mathbb{R}$) :
$$
\left \{
\begin{array}{lc}
x(t) =& \cos^2(t)\\
y(t) = &\sin^2(t)\\
\end{array}
\right.
$$

\begin{enumerate}
    \item 
   Réduire l'intervalle d'étude. \textbf{ 0 1 point}\\
\item Calculer les tangentes de la courbe pour $t \in \left[0;\pi \right]$.\textbf{0 1 2 3 points}\\
\item En déduire les points stationnaires.\textbf{0 1 2 points}\\
    \item
   Calculer la longueur de la courbe.\textbf{0 1 2 3 4 points}\\


\end{enumerate}



\end{document}
