\documentclass[10pt]{article}

\usepackage[utf8]{inputenc}
\usepackage[french]{babel}
\usepackage{graphicx}
%\usepackage{wrapfig}
\usepackage{dsfont}
\usepackage[margin=2cm]{geometry}
\usepackage{amsmath,amssymb}
%\usepackage{tikz}
\usepackage{multicol}
\newcommand{\xB}{{\cal B}}
\newcommand{\xC}{{\cal C}}
\newcommand{\R}{\mathbf{R}}
\usepackage{amsmath, tikz}
\begin{document}

\pagestyle{empty}

\noindent
\begin{minipage}[l]{8cm}
  \scriptsize{ESTP, S2\\Ann\'ee universitaire 2024/2025}
\end{minipage}

\begin{center}
  {\large\textbf{Petit test } \\
15 minutes\\}
  \bigskip
  

 
\end{center}

\bigskip

\begin{center}
  \fbox{%
    \begin{minipage}{0.95\linewidth}
      NOM: \hspace*{5cm} Pr\'enom: \hspace*{4cm} Groupe: TD 
    \end{minipage}
  }
  \end{center}

\bigskip
 
 

\noindent\textbf{Exercice  : 5 points}\\


Calculer l'intégrale suivante après avoir tracé le domaine

\begin{equation}
I = \int\int_D e^{-x-y} \,dx\,dy, \quad \text{où } D = \{(x, y) \in \mathbb{R}^2 \mid 0 \leq x \leq 1, \ 1 \leq y \leq x+3\}.
\end{equation}




%\section*{Correction}
%1 point tracé
%1 point ordre d'intégration
%1 point primitive
% 2 points fin de calcul
%\begin{center}
%\begin{tikzpicture}[scale=1.5]
%    % Axes
%    \draw[->] (-0.5,0) -- (2,0) node[right] {$x$};
%    \draw[->] (0,-0.5) -- (0,4.5) node[above] {$y$};
%    
%    % Domaine
%    \fill[blue!20, opacity=0.5] (0,1) -- (1,4) -- (1,1) -- cycle;
%    
%    % Bords du domaine
%    \draw[thick, blue] (0,1) -- (1,4);
%    \draw[thick, blue] (1,1) -- (1,4);
%    \draw[thick, blue] (0,1) -- (1,1);
%    
%    % Labels
%    \node[left] at (0,1) {1};
%    \node[below] at (1,0) {1};
%    %\node[right] at (1,4) {$(1,4)$};
%    
%    % Equation de la droite
%    \draw[dashed] (0,3) -- (1,4);
%    \node[right] at (0.5,3.5) {\small $y = x+3$};
%\end{tikzpicture}
%\end{center}

%\section*{Calcul de l'intégrale}
%
%L'intégrale double est donnée par :
%\begin{align*}
%I &= \int_0^1 \int_1^{x+3} e^{-x-y} \, dy \, dx.
%\end{align*}
%
%Calculons l'intégrale intérieure :
%\begin{align*}
%\int_1^{x+3} e^{-x-y} \, dy &= e^{-x} \int_1^{x+3} e^{-y} \, dy \\
%&= e^{-x} \left[ -e^{-y} \right]_1^{x+3} \\
%&= e^{-x} \left( -e^{-(x+3)} + e^{-1} \right) \\
%&= e^{-x} e^{-1} - e^{-x} e^{-(x+3)} \\
%&= e^{-x-1} - e^{-2x-3}.
%\end{align*}
%
%Intégrons maintenant par rapport à \( x \) :
%\begin{align*}
%I &= \int_0^1 \left( e^{-x-1} - e^{-2x-3} \right) dx \\
%&= e^{-1} \int_0^1 e^{-x} \,dx - e^{-3} \int_0^1 e^{-2x} \,dx.
%\end{align*}
%
%Les intégrales élémentaires donnent :
%\begin{align*}
%\int_0^1 e^{-x} \,dx &= \left[ -e^{-x} \right]_0^1 = 1 - e^{-1}, \\
%\int_0^1 e^{-2x} \,dx &= \left[ -\frac{e^{-2x}}{2} \right]_0^1 = \frac{1 - e^{-2}}{2}.
%\end{align*}
%
%Ainsi, on obtient :
%\begin{align*}
%I &= e^{-1} (1 - e^{-1}) - e^{-3} \frac{1 - e^{-2}}{2} \\
%&= e^{-1} - e^{-2} - \frac{e^{-3} (1 - e^{-2})}{2} \\
%&= e^{-1} - e^{-2} - \frac{e^{-3} - e^{-5}}{2}.
%\end{align*}

\end{document}

