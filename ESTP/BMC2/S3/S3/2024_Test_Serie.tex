\documentclass[10pt]{article}

\usepackage[utf8]{inputenc}
\usepackage[french]{babel}
\usepackage{graphicx}
%\usepackage{wrapfig}
\usepackage{dsfont}
\usepackage[margin=2cm]{geometry}
\usepackage{amsmath,amssymb}
%\usepackage{tikz}
\usepackage{multicol}
\newcommand{\xB}{{\cal B}}
\newcommand{\xC}{{\cal C}}
\newcommand{\R}{\mathbf{R}}

\begin{document}

\pagestyle{empty}

\noindent
\begin{minipage}[l]{8cm}
  \scriptsize{ESTP, S3\\Ann\'ee universitaire BLANC}
\end{minipage}

\begin{center}
  {\large\textbf{Petit test } \\
10 minutes\\}
  \bigskip
  

 
\end{center}

\bigskip

\begin{center}
  \fbox{%
    \begin{minipage}{0.95\linewidth}
      NOM: \hspace*{5cm} Pr\'enom: \hspace*{4cm} Groupe: TD 
    \end{minipage}
  }
  \end{center}

\bigskip
 
 Trois questions peuvent être prises parmi le lot suivant

\noindent\textbf{Exercice  : 5 points}\\

\begin{enumerate}
    \item 
    Soit la série $\displaystyle\sum_{n=1}^{+\infty} \frac{1}{n^2 + 3n + 2}$. Montrer qu’il s’agit d’une série à termes positifs et conclure sur sa convergence par comparaison.\\

    \item 
    Étudier la convergence de la série $\displaystyle\sum_{n=2}^{+\infty} \frac{1}{n (\ln n)^p}$ selon les valeurs du paramètre $p > 0$.\\

    \item 
    Étudier la convergence de la série $\displaystyle\sum_{n=1}^{+\infty} \frac{n^2}{n^3 + x}$ selon la valeur du réel $x$.\\

    \item 
    Pour quelles valeurs de $x \in \mathbb{R}$ la série $\displaystyle\sum_{n=1}^{+\infty} \frac{x^n}{n}$ converge-t-elle ?\\

    \item 
    Étudier la nature de la série $\displaystyle\sum_{n=1}^{+\infty} \frac{(-1)^n}{n^p}$ selon les valeurs du paramètre réel $p > 0$.\\

    \item 
    Soit $f(x) = \displaystyle\sum_{n=1}^{+\infty} \frac{x^n}{n^2}$. Déterminer l’ensemble de convergence de cette série, puis étudier la continuité de $f$ sur cet ensemble.\\

    \item 
    On considère la série $\displaystyle\sum_{n=1}^{+\infty} \frac{(-1)^n}{n^2 + x}$. Pour quelles valeurs de $x$ la série est-elle convergente ? Converge-t-elle absolument ?\\

    \item 
    Étudier la convergence de la série $\displaystyle\sum_{n=1}^{+\infty} \frac{1}{n^\alpha \ln^\beta n}$ selon les valeurs des paramètres $\alpha > 0$ et $\beta \in \mathbb{R}$.\\

    \item 
    Pour $x > 0$, étudier la convergence de la série $\displaystyle\sum_{n=1}^{+\infty} \frac{1}{n^x + n}$.\\

    \item 
    Soit $S(x) = \displaystyle\sum_{n=1}^{+\infty} \left(\frac{x}{1+x}\right)^n$. Donner l’ensemble de convergence de cette série en fonction de $x$. Calculer ensuite $S(x)$ lorsque la série converge.\\
\end{enumerate}



\end{document}
