\documentclass[12pt]{article}

% Packages pour les marges
\usepackage[
    top=1.5cm,
    bottom=1.5cm,
    left=1.5cm,
    right=1.5cm
]{geometry}

% Police sans serif
% \usepackage{helvet}
% \renewcommand{\familydefault}{\sfdefault}

% Packages existants
\usepackage[french]{babel}
\usepackage[utf8]{inputenc}
\usepackage[T1]{fontenc}
\usepackage{amsmath}
\usepackage{amsfonts}
\usepackage{amssymb}
\usepackage{mathtools}
\usepackage{array}
\usepackage[version=4]{mhchem}
\usepackage{stmaryrd}
\usepackage{enumitem}
\usepackage{ifthen}
\usepackage{eurosym}
\usepackage{textcomp}
\usepackage{graphicx}
\usepackage{xcolor}
\usepackage{multicol}
\definecolor{Theme}{HTML}{0E7490} % teal-700
\definecolor{ThemeLight}{HTML}{E0F2F1}
\definecolor{Accent}{HTML}{F59E0B} % amber-500
\definecolor{Gray}{HTML}{374151}
\usepackage[colorlinks=true,linkcolor=Theme,urlcolor=Theme,citecolor=Theme]{hyperref}

\usepackage{mdframed}
\usepackage[sf]{titlesec}
\usepackage{environ}

% Définition de la variable pour afficher les corrections
\newboolean{showSolutions}
% Décommentez la ligne suivante pour afficher les solutions
\input \jobname.adr

\title{Mathématiques : Examen}

\author{}
\date{}

\newenvironment{solution}
{
    \vspace{0.5em}
    \begin{mdframed}[backgroundcolor=ThemeLight,leftmargin=0,rightmargin=0,skipabove=0.2em,skipbelow=0.2em]
    \textbf{Solution.}\\[0.5em]
}
{
    \end{mdframed}
    \vspace{0.5em}
}
\begin{document}
\sffamily

\begin{center}
    \renewcommand{\arraystretch}{1.5} % Ajuste l'espacement vertical des lignes
    \begin{tabular}{|>{\centering\arraybackslash}m{4cm}|>{\centering\arraybackslash}m{6cm}|>{\centering\arraybackslash}m{4cm}|}
        \hline 
        \vspace{5mm} \hspace{5mm}\raisebox{-0.2\height}{\includegraphics[width=3cm]{Logo-ESTP.png}} \vspace{5mm}  & 
        \textbf{Contrôle de connaissances et de compétences} & 
        \textbf{FO-002-VLA-XX-001} \\
        \hline
        \textbf{30/01/2026}  &  & \textbf{Page 1/2} \\
        \hline
    \end{tabular}
\end{center}
\vspace{0.5em}

\begin{center}
    \renewcommand{\arraystretch}{1.5}
    \begin{tabular}{|c|m{10cm}|}
        \hline 
        \multicolumn{2}{|c|}{\textbf{ANNÉE SCOLAIRE 2025-2026 -- Semestre 3}} \\
        \hline 
        \textbf{Nom de l'enseignant} & Maxime Berger \& Karine Serier \\
        \hline 
        \textbf{Promotion} & BMC2 - S3 \\
        \hline 
        \textbf{Matière} & Mathématiques  \\
        \hline 
        \textbf{Durée de l'examen} & 3h00 \\
        \hline 
        \textbf{Consignes} & 
        \vspace{0.5em}
        \begin{itemize}
            \item Calculatrice \textbf{NON} autorisée
            \item Aucun document n'est autorisé \vspace{1em}
        \end{itemize}\\
        \hline
    \end{tabular}
\end{center}


\begin{center}
    \renewcommand{\arraystretch}{1.3}
    \begin{tabular}{|c|c|c|c|c|c||c|}
        \hline
        \textbf{Exercice} & \textbf{1} & \textbf{2} & \textbf{3} & \textbf{4} & \textbf{5} & \textbf{Total} \\
        \hline
        \textbf{Barème} & 4 pts & 4 pts & 4 pts & 3 pts & 5 pts & \textbf{20 pts} \\
        \hline
    \end{tabular}
\end{center}

\vspace{1em}
%==============================================================================
\section*{Exercice 1 : Convergence de séries \hfill \normalfont\textit{(4 points)}}
%==============================================================================

Étudier la nature (convergence ou divergence) des séries suivantes. Justifier soigneusement chaque réponse.
\ifthenelse{\boolean{showSolutions}}{}{
\begin{multicols}{2}
}

\begin{enumerate}
    \item $\displaystyle \sum_{n=0}^{+\infty} (-1)^n\frac{3^n}{5^n}$ \textit{(1 pt)}
    
    \ifthenelse{\boolean{showSolutions}}{
    \begin{solution}
    C'est une série géométrique de raison $r = \frac{-3}{5}$.
    
    Comme $|r| = \frac{3}{5} < 1$, la série \textbf{converge}.
    
    Sa somme vaut : $S = \frac{1}{1 - \frac{-3}{5}} = \frac{1}{\frac{8}{5}} = \frac{5}{8}$
    \end{solution}
    }{}
    
    \item $\displaystyle \sum_{n=1}^{+\infty} \frac{(-1)^{n+1}}{2n}$ \textit{(1 pt)}
    
    \ifthenelse{\boolean{showSolutions}}{
    \begin{solution}
    C'est une série alternée de la forme $\sum (-1)^{n+1} a_n$ avec $a_n = \frac{1}{2n}$.
    
    Vérifions le critère des séries alternées (Leibniz) :
    \begin{itemize}
        \item $(a_n)$ est décroissante : $\frac{1}{n+1} < \frac{1}{n}$ \checkmark
        \item $\lim_{n \to +\infty} a_n = 0$ \checkmark
    \end{itemize}
    
    Par le critère de Leibniz, la série \textbf{converge}.
    
    \end{solution}
    }{}
    
    \item $\displaystyle \sum_{n=0}^{+\infty} \frac{2^n}{n!}$ \textit{(1 pt)}
    
    \ifthenelse{\boolean{showSolutions}}{
    \begin{solution}
    Utilisons le critère de D'Alembert. Posons $u_n = \frac{2^n}{n!}$.
    \[
    \frac{u_{n+1}}{u_n} = \frac{2^{n+1}}{(n+1)!} \cdot \frac{n!}{2^n} = \frac{2 \cdot 2^n}{(n+1) \cdot n!} \cdot \frac{n!}{2^n} = \frac{2}{n+1}
    \]
    
    Donc $\lim_{n \to +\infty} \frac{u_{n+1}}{u_n} = 0 < 1$.
    
    Par le critère de D'Alembert, la série \textbf{converge}.
    
    \end{solution}
    }{}
    
    \item $\displaystyle \sum_{n=1}^{+\infty} \frac{n+1}{n^3 + 2n + \cos(n)}$ \textit{(1 pt)}
    
    \ifthenelse{\boolean{showSolutions}}{
    \begin{solution}
    Cherchons un équivalent du terme général quand $n \to +\infty$ :
    \[
    \frac{n+1}{n^3 + 2n + \cos(n)} = \frac{n(1 + \frac{1}{n})}{n^3(1 + \frac{2}{n^2} + \frac{\cos(n)}{n^3})} = \frac{1 + \frac{1}{n}}{n^2(1 + \frac{2}{n^2} + \frac{\cos(n)}{n^3})} \sim \frac{1}{n^2} \quad \text{quand } n \to +\infty
    \]
    
    Or $\sum \frac{1}{n^2}$ est une série de Riemann avec $\alpha = 2 > 1$, donc convergente.
    
    Par équivalence de séries à termes positifs, la série $\sum \frac{n+1}{n^3+2n+\cos(n)}$ \textbf{converge}.
    \end{solution}
    }{}
\end{enumerate}
\ifthenelse{\boolean{showSolutions}}{}{
\end{multicols}
}

\vspace{1em}

%==============================================================================
\section*{Exercice 2 : Noyau et image \hfill \normalfont\textit{(4 points)}}
%==============================================================================

Soit $h : \mathbb{R}^3 \to \mathbb{R}^2$ l'application linéaire définie par :
\[
h(x, y, z) = (x - y + 2z, 3x - 3y + 6z)
\]

\begin{enumerate}
    \item Écrire la matrice $B$ de $h$ dans les bases canoniques. \textit{(0.5 pt)}
    
    \ifthenelse{\boolean{showSolutions}}{
    \begin{solution}
    \[
    B = \begin{pmatrix} 1 & -1 & 2 \\ 3 & -3 & 6 \end{pmatrix}
    \]
    \end{solution}
    }{}
    
    \item Déterminer le noyau $\ker(h)$. Donner une base et la dimension. \textit{(2 pts)}
    
    \ifthenelse{\boolean{showSolutions}}{
    \begin{solution}
    $(x, y, z) \in \ker(h) \Leftrightarrow h(x, y, z) = (0, 0)$
    
    On résout le système :
    \[
    \begin{cases}
    x - y + 2z = 0 \\
    3x - 3y + 6z = 0
    \end{cases}
    \]
    
    La deuxième équation est le triple de la première, donc on a une seule contrainte. On choisira deux paramètres libres $y$ et $z$. 
    On exprime alors $x$ en fonction de $y$ et $z$ : $x = y - 2z$.
    
    Les solutions sont :
    \[
    (x, y, z) = (y - 2z, y, z) = y(1, 1, 0) + z(-2, 0, 1)
    \]
    
    \textbf{Base de $\ker(h)$ :} $\{(1, 1, 0), (-2, 0, 1)\}$
    
    \textbf{Dimension :} $\dim(\ker(h)) = 2$
    \end{solution}
    }{}
    
    \item Déterminer l'image $\text{Im}(h)$. Donner une base et la dimension. \textit{(1.5 pts)}
    
    \ifthenelse{\boolean{showSolutions}}{
    \begin{solution}
    L'image de $h$ est engendrée par les colonnes de $B$ :
    \[
    \text{Im}(h) = \text{Vect}\left\{\begin{pmatrix} 1 \\ 3 \end{pmatrix}, \begin{pmatrix} -1 \\ -3 \end{pmatrix}, \begin{pmatrix} 2 \\ 6 \end{pmatrix}\right\}
    \]
    
    On remarque que $(-1, -3) = -(1, 3)$ et $(2, 6) = 2(1, 3)$.
    
    Donc $\text{Im}(h) = \text{Vect}\{(1, 3)\}$.
    
    \textbf{Base de $\text{Im}(h)$ :} $\{(1, 3)\}$
    
    \textbf{Dimension :} $\dim(\text{Im}(h)) = 1$
    
    \textit{Vérification par le théorème du rang :} $\dim(\ker(h)) + \dim(\text{Im}(h)) = 2 + 1 = 3 = \dim(\mathbb{R}^3)$ \checkmark
    \end{solution}
    }{}
\end{enumerate}

\newpage

\begin{center}
    \renewcommand{\arraystretch}{1.5} 
    \begin{tabular}{|>{\centering\arraybackslash}m{4cm}|>{\centering\arraybackslash}m{6cm}|>{\centering\arraybackslash}m{4cm}|}
        \hline
            \hspace{4cm}&\hspace{6cm} & \textbf{Page 2/2}\\
            \hline
    \end{tabular}
\end{center}


%==============================================================================
\section*{Exercice 3 : Algèbre linéaire \hfill \normalfont\textit{(4 points)}}
%==============================================================================

Soit $F = \{(x, y, z) \in \mathbb{R}^3 \mid 2x - y + z = 0\}$.

\begin{enumerate}
    \item Montrer que $F$ est un sous-espace vectoriel de $\mathbb{R}^3$. \textit{(1.5 pts)}
    
    \ifthenelse{\boolean{showSolutions}}{
    \begin{solution}
    Vérifions les trois axiomes d'un sous-espace vectoriel :
    
    \textbf{1) Non vide :} $(0, 0, 0) \in F$ car $2(0) - 0 + 0 = 0$. \checkmark
    
    \textbf{2) Stabilité par addition :} Soient $u = (x_1, y_1, z_1)$ et $v = (x_2, y_2, z_2)$ dans $F$.
    \begin{itemize}
        \item $2x_1 - y_1 + z_1 = 0$ et $2x_2 - y_2 + z_2 = 0$
        \item $u + v = (x_1 + x_2, y_1 + y_2, z_1 + z_2)$
        \item $2(x_1 + x_2) - (y_1 + y_2) + (z_1 + z_2) = (2x_1 - y_1 + z_1) + (2x_2 - y_2 + z_2) = 0$
    \end{itemize}
    Donc $u + v \in F$. \checkmark
    
    \textbf{3) Stabilité par multiplication :} Soit $\lambda \in \mathbb{R}$ et $u = (x, y, z) \in F$.
    \begin{itemize}
        \item $\lambda u = (\lambda x, \lambda y, \lambda z)$
        \item $2\lambda x - \lambda y + \lambda z = \lambda(2x - y + z) = 0$
    \end{itemize}
    Donc $\lambda u \in F$. \checkmark
    
    $F$ est bien un sous-espace vectoriel de $\mathbb{R}^3$.
    \end{solution}
    }{}
    
    \item Écrire $F$ avec le mot clé "Vect", donner une base et la dimension. \textit{(1 pt)}
    
    \ifthenelse{\boolean{showSolutions}}{
    \begin{solution}
    De $2x - y + z = 0$, on tire $y = 2x + z$. Donc on peut choisir $x$ et $z$ comme paramètres libres :
    \[
    (x, y, z) = (x, 2x+z, z) = x(1, 2, 0) + z(0, 1, 1)
    \]

    Donc $F = \text{Vect}\{(1, 2, 0), (0, 1, 1)\}$.

    Les vecteurs $e_1 = (1, 2, 0)$ et $e_2 = (0, 1, 1)$ sont clairement linéairement indépendants (le premier a une composante en $x$ non nulle et le second a $x=0$).
    
    \textbf{Base de $F$ :} $\mathcal{B}_F = \{(1, 2, 0), (0, 1, 1)\}$
    
    \textbf{Dimension :} $\dim(F) = 2$
    \end{solution}
    }{}
    
    \item Soit $\varphi : \mathbb{R}^2 \to \mathbb{R}^3$ définie par $\varphi(x, y) = (2x - y, x + 3y, x - y)$.
    
    Montrer que $\varphi$ est une application linéaire. \textit{(1 pt)}
    
    \ifthenelse{\boolean{showSolutions}}{
    \begin{solution}
    Soient $(x_1, y_1), (x_2, y_2) \in \mathbb{R}^2$ et $\lambda \in \mathbb{R}$.
    
    \textbf{Additivité :}
    \begin{align*}
    \varphi((x_1, y_1) + (x_2, y_2)) &= \varphi(x_1 + x_2, y_1 + y_2) \\
    &= (2(x_1+x_2) - (y_1+y_2), (x_1+x_2) + 3(y_1+y_2), (x_1+x_2) - (y_1+y_2)) \\
    &= (2x_1-y_1 + 2x_2-y_2, x_1+3y_1 + x_2+3y_2, x_1-y_1 + x_2-y_2) \\
    &= (2x_1-y_1, x_1+3y_1, x_1-y_1) + (2x_2-y_2, x_2+3y_2, x_2-y_2) \\
    &= \varphi(x_1, y_1) + \varphi(x_2, y_2) \quad \checkmark
    \end{align*}
    
    \textbf{Homogénéité :}
    \begin{align*}
    \varphi(\lambda(x, y)) &= \varphi(\lambda x, \lambda y) \\
    &= (2\lambda x - \lambda y, \lambda x + 3\lambda y, \lambda x - \lambda y) \\
    &= \lambda(2x - y, x + 3y, x - y) = \lambda \varphi(x, y) \quad \checkmark
    \end{align*}
    
    $\varphi$ est bien une application linéaire.
    \end{solution}
    }{}
    
    \item Écrire la matrice $A$ de $\varphi$ dans les bases canoniques de $\mathbb{R}^2$ et $\mathbb{R}^3$. \textit{(0.5 pt)}
    
    \ifthenelse{\boolean{showSolutions}}{
    \begin{solution}
    On calcule les images des vecteurs de la base canonique de $\mathbb{R}^2$ :
    \begin{itemize}
        \item $\varphi(1, 0) = (2, 1, 1)$
        \item $\varphi(0, 1) = (-1, 3, -1)$
    \end{itemize}
    
    La matrice de $\varphi$ est :
    \[
    A = \begin{pmatrix} 2 & -1 \\ 1 & 3 \\ 1 & -1 \end{pmatrix}
    \]
    \end{solution}
    }{}
    
    \item Soit $\mathcal{B} = \{(1, 1), (1, -1)\}$ une base de $\mathbb{R}^2$ et $\mathcal{B}' = \{(1, 0, 0), (1, 1, 0), (1, 1, 1)\}$ une base de $\mathbb{R}^3$.
    
    Calculer la matrice $A'$ de $\varphi$ dans les bases $\mathcal{B}$ (départ) et $\mathcal{B}'$ (arrivée). \textit{(0.5 pt)}
    
    \ifthenelse{\boolean{showSolutions}}{
    \begin{solution}
    \textbf{Étape 1 :} Calculons les images des vecteurs de $\mathcal{B}$ :
    \begin{itemize}
        \item $\varphi(1, 1) = (2 \cdot 1 - 1, 1 + 3 \cdot 1, 1 - 1) = (1, 4, 0)$
        \item $\varphi(1, -1) = (2 \cdot 1 - (-1), 1 + 3 \cdot (-1), 1 - (-1)) = (3, -2, 2)$
    \end{itemize}
    
    \textbf{Étape 2 :} Exprimons ces images dans la base $\mathcal{B}'$.
    
    Pour $(1, 4, 0) = \alpha(1, 0, 0) + \beta(1, 1, 0) + \gamma(1, 1, 1)$ :
    \[
    \begin{cases}
    \alpha + \beta + \gamma = 1 \\
    \beta + \gamma = 4 \\
    \gamma = 0
    \end{cases}
    \implies \gamma = 0, \; \beta = 4, \; \alpha = -3
    \]
    Donc $[\varphi(1, 1)]_{\mathcal{B}'} = \begin{pmatrix} -3 \\ 4 \\ 0 \end{pmatrix}$
    
    Pour $(3, -2, 2) = \alpha(1, 0, 0) + \beta(1, 1, 0) + \gamma(1, 1, 1)$ :
    \[
    \begin{cases}
    \alpha + \beta + \gamma = 3 \\
    \beta + \gamma = -2 \\
    \gamma = 2
    \end{cases}
    \implies \gamma = 2, \; \beta = -4, \; \alpha = 5
    \]
    Donc $[\varphi(1, -1)]_{\mathcal{B}'} = \begin{pmatrix} 5 \\ -4 \\ 2 \end{pmatrix}$
    
    \textbf{Matrice de $\varphi$ dans les bases $\mathcal{B}$ et $\mathcal{B}'$ :}
    \[
    A' = \begin{pmatrix} -3 & 5 \\ 4 & -4 \\ 0 & 2 \end{pmatrix}
    \]
    \end{solution}
    }{}
\end{enumerate}

\vspace{2em}


%==============================================================================
\section*{Exercice 4 : Série télescopique et comparaison \hfill \normalfont\textit{(3 points)}}
%==============================================================================

\begin{enumerate}
    \item Décomposer $\frac{1}{n(n+2)}$ en éléments simples. \textit{(0.5 pt)}
    
    \textit{Montrer que : $\frac{1}{n(n+2)} = \frac{1}{2}\left(\frac{1}{n} - \frac{1}{n+2}\right)$}
    
    \ifthenelse{\boolean{showSolutions}}{
    \begin{solution}
    \[
    \frac{1}{n(n+2)} = \frac{A}{n} + \frac{B}{n+2}
    \]
    En multipliant par $n(n+2)$ : $1 = A(n+2) + Bn$
    \begin{itemize}
        \item $n = 0$ : $A = \frac{1}{2}$
        \item $n = -2$ : $B = -\frac{1}{2}$
    \end{itemize}
    Donc $\frac{1}{n(n+2)} = \frac{1}{2}\left(\frac{1}{n} - \frac{1}{n+2}\right)$
    \end{solution}
    }{}
    
    \item En déduire la valeur de la somme partielle $S_N = \sum_{n=1}^{N} \frac{1}{n(n+2)}$. \textit{(1 pt)}
    
    \ifthenelse{\boolean{showSolutions}}{
    \begin{solution}
    \[
    S_N = \frac{1}{2}\sum_{n=1}^{N} \left(\frac{1}{n} - \frac{1}{n+2}\right)
    \]
    C'est une série télescopique. En développant :
    \begin{align*}
    S_N &= \frac{1}{2}\left[\left(1 - \frac{1}{3}\right) + \left(\frac{1}{2} - \frac{1}{4}\right) + \left(\frac{1}{3} - \frac{1}{5}\right) + \cdots + \left(\frac{1}{N} - \frac{1}{N+2}\right)\right]\\
    &= \frac{1}{2}\left(1 + \frac{1}{2} - \frac{1}{N+1} - \frac{1}{N+2}\right) = \frac{1}{2}\left(\frac{3}{2} - \frac{1}{N+1} - \frac{1}{N+2}\right)
    \end{align*}
    \end{solution}
    }{}
    
    \item Calculer $\sum_{n=1}^{+\infty} \frac{1}{n(n+2)}$. \textit{(0.5 pt)}
    
    \ifthenelse{\boolean{showSolutions}}{
    \begin{solution}
    \[
    \sum_{n=1}^{+\infty} \frac{1}{n(n+2)} = \lim_{N \to +\infty} S_N = \lim_{N \to +\infty} \frac{1}{2}\left(\frac{3}{2} - \frac{1}{N+1} - \frac{1}{N+2}\right) = \frac{3}{4}
    \]
    \end{solution}
    }{}
    
    \item À l'aide d'une comparaison série/intégrale, montrer que $\sum_{n=2}^{+\infty} \frac{1}{n \ln^2(n)}$ converge. \textit{(1 pt)}
    
    \textit{Indication : On pourra montrer que pour $x \in [n, n+1]$ : $\dfrac{1}{(n+1) \ln^2(n+1)} \leq \dfrac{1}{x \ln^2(x)} \leq \dfrac{1}{n \ln^2(n)}$, puis utiliser le changement de variable $u = \ln(x)$ dans l'intégrale.}
    
    \ifthenelse{\boolean{showSolutions}}{
    \begin{solution}
    La fonction $f(x) = \frac{1}{x \ln^2(x)}$ est positive, continue et décroissante sur $[2, +\infty[$.

    On a, pour $x \in [n, n+1]$, $\dfrac{1}{(n+1) \ln^2(n+1)} \leq \dfrac{1}{x \ln^2(x)} \leq \dfrac{1}{n \ln^2(n)}$.
    
    En intégrant cette inégalité entre $n$ et $n+1$, on obtient :
    \[
    \frac{1}{(n+1) \ln^2(n+1)} \leq \int_n^{n+1} \frac{1}{x \ln^2(x)} dx \leq \frac{1}{n \ln^2(n)}
    \]

    Gardons seulement la partie gauche de cette inégalité, et sommons pour $n$ allant de 2 à $N$ :
    \[
    \sum_{n=2}^{N} \frac{1}{(n+1) \ln^2(n+1)} \leq \int_2^{N+1} \frac{1}{x \ln^2(x)} dx
    \]

    On peut calculer cette intégrale avec le changement de variable $u = \ln(x)$, $du = \frac{dx}{x}$ : 
    \[
    \int_2^{N+1} \frac{1}{x \ln^2(x)} dx = \int_{\ln 2}^{\ln(N+1)} \frac{1}{u^2} du = \left[-\frac{1}{u}\right]_{\ln 2}^{\ln(N+1)} = \frac{1}{\ln 2} - \frac{1}{\ln(N+1)}
    \]

    Cette intégrale converge vers $\frac{1}{\ln 2}$ quand $N \to +\infty$.
    
    La somme partielle $\sum_{n=3}^{N+1} \frac{1}{n \ln^2(n)}$ est donc bornée.
    
    Comme la série est à termes positifs et ses sommes partielles sont bornées, la série $\sum \frac{1}{n \ln^2(n)}$ \textbf{converge}.

    \end{solution}
    }{}
\end{enumerate}

\vspace{2em}

%==============================================================================
\section*{Exercice 5 : Calcul différentiel et intégrales curvilignes  \hfill \normalfont\textit{(5 points)}}
%==============================================================================

\begin{enumerate}
    \item Soit $f(x, y) = x^3 - xy^2 + \ln(xy)$ (définie pour $xy > 0$). Calculer la différentielle $df$. \textit{(1 pt)}
    
    \ifthenelse{\boolean{showSolutions}}{
    \begin{solution}
    \[
    \frac{\partial f}{\partial x} = 3x^2 - y^2 + \frac{1}{x}, \qquad \frac{\partial f}{\partial y} = -2xy + \frac{1}{y}
    \]
    
    Donc :
    \[
    df = \left(3x^2 - y^2 + \frac{1}{x}\right)dx + \left(-2xy + \frac{1}{y}\right)dy
    \]
    \end{solution}
    }{}
    
    \item Soit $g(x, y, z) = x^2y + yz^2 - xz$. Calculer le gradient $\nabla g$. \textit{(0.5 pt)}
    
    \ifthenelse{\boolean{showSolutions}}{
    \begin{solution}
    \[
    \nabla g = \begin{pmatrix} 
    \dfrac{\partial g}{\partial x} \\[0.8em] 
    \dfrac{\partial g}{\partial y} \\[0.8em] 
    \dfrac{\partial g}{\partial z} 
    \end{pmatrix} = \begin{pmatrix} 2xy - z \\[0.5em] x^2 + z^2 \\[0.5em] 2yz - x \end{pmatrix}
    \]
    \end{solution}
    }{}
    
    \item Soit $\vec{F}(x, y, z) = (xy, y^2z, xz^2)$. Calculer la matrice jacobienne de $\vec{F}$ et la divergence $\text{div}(\vec{F})$. \textit{(1.5 pts)}
    
    \ifthenelse{\boolean{showSolutions}}{
    \begin{solution}
    La matrice jacobienne est :
    \[
    J_{\vec{F}} = \begin{pmatrix}
    \dfrac{\partial F_1}{\partial x} & \dfrac{\partial F_1}{\partial y} & \dfrac{\partial F_1}{\partial z} \\[0.8em]
    \dfrac{\partial F_2}{\partial x} & \dfrac{\partial F_2}{\partial y} & \dfrac{\partial F_2}{\partial z} \\[0.8em]
    \dfrac{\partial F_3}{\partial x} & \dfrac{\partial F_3}{\partial y} & \dfrac{\partial F_3}{\partial z}
    \end{pmatrix} = \begin{pmatrix}
    y & x & 0 \\
    0 & 2yz & y^2 \\
    z^2 & 0 & 2xz
    \end{pmatrix}
    \]
    
    La divergence est :
    \[
    \text{div}(\vec{F}) = \frac{\partial F_1}{\partial x} + \frac{\partial F_2}{\partial y} + \frac{\partial F_3}{\partial z} = y + 2yz + 2xz
    \]
    \end{solution}
    }{}
    
    \item Soit $\vec{H}(x, y) = (3x^2 + 2y, 2x - 4y)$. 
    
    Montrer que $\vec{H}$ dérive d'un potentiel scalaire et déterminer ce potentiel. \textit{(1 pt)}
    
    \ifthenelse{\boolean{showSolutions}}{
    \begin{solution}
    Vérifions la condition d'irrotationnalité : $\frac{\partial H_1}{\partial y} = \frac{\partial H_2}{\partial x}$
    \[
    \frac{\partial H_1}{\partial y} = 2, \qquad \frac{\partial H_2}{\partial x} = 2 \quad \checkmark
    \]
    
    Donc $\vec{H}$ dérive d'un potentiel $\varphi$ tel que $\nabla \varphi = \vec{H}$.
    
    On intègre :
    \[
    \frac{\partial \varphi}{\partial x} = 3x^2 + 2y \implies \varphi(x, y) = x^3 + 2xy + h(y)
    \]
    
    On vérifie avec la deuxième composante :
    \[
    \frac{\partial \varphi}{\partial y} = 2x + h'(y) = 2x - 4y \implies h'(y) = -4y \implies h(y) = -2y^2 + C
    \]
    
    \textbf{Potentiel :} $\varphi(x, y) = x^3 + 2xy - 2y^2 + C$
    \end{solution}
    }{}
    
    \item Calculer l'intégrale curviligne $\displaystyle \int_{C^+} (x + y) \, dx + (x - y) \, dy$
    
    où $C$ est le segment de droite allant de $(1, 0)$ à $(0, 2)$, parcouru dans ce sens. \textit{(1 pt)}
    
    \ifthenelse{\boolean{showSolutions}}{
    \begin{solution}
    Paramétrons le segment : $\gamma(t) = (1-t, 2t)$ pour $t \in [0, 1]$.
    
    On a : $x = 1-t$, $y = 2t$, $dx = -dt$, $dy = 2dt$.
    
    \begin{align*}
    \int_{C^+} (x + y) \, dx + (x - y) \, dy &= \int_0^1 ((1-t) + 2t)(-dt) + ((1-t) - 2t)(2dt) \\
    &= \int_0^1 -(1+t) \, dt + \int_0^1 2(1 - 3t) \, dt \\
    &= \int_0^1 (-1 - t + 2 - 6t) \, dt \\
    &= \int_0^1 (1 - 7t) \, dt \\
    &= \left[t - \frac{7t^2}{2}\right]_0^1 \\
    &= 1 - \frac{7}{2} = -\frac{5}{2}
    \end{align*}
    \end{solution}
    }{}
\end{enumerate}


\end{document}
