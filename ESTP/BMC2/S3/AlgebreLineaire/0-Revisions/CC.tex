
Appliquer le pivot de Gauss pour transformer ces systèmes en systèmes échelonnés. Indiquez pour chacun s'il possède aucune solution, une unique solution ou une infinité de solutions.

$$
\left\{\begin{array}{l}
    2 x+3 y-z=1 \\
    4 x+y+2 z=6 \\
    x-3 y+z=2
\end{array}\right.
$$

\ifthenelse{\boolean{showSolutions}}{
    On applique les opérations suivantes pour enlever les $x$ des lignes 2 et 3 :
    \begin{align*}
        L_2 &\leftarrow L_2 - 2L_1 \\
        L_3 &\leftarrow 2L_3 - L_1
    \end{align*}
    On obtient le système suivant :
    $$
    \left\{\begin{array}{l}
        2 x+3 y-z=1 \\
        \quad -5 y+4 z=4 \\
        \quad -9 y+3 z=3
    \end{array}\right.
    $$
    Pour retirer le $y$ de la ligne 3, on applique $L_3 \leftarrow 5L_3 + 9L_2$ :
    $$
    \left\{\begin{array}{l}
        2 x+3 y-z=1 \\
        \quad -5 y+4 z=4 \\
        \quad \quad -21z=-21
    \end{array}\right.
    $$
    Le système est parfaitement échelonné, il admet une unique solution. 
}{}

$$
\left\{\begin{array}{l}
    2 x+y=1 \\
    x+y=0 \\
    3 x+4 y=-1
\end{array}\right.
$$
\ifthenelse{\boolean{showSolutions}}{
    On applique les opérations suivantes pour enlever les $x$ des lignes 2 et 3 :
    \begin{align*}
        L_2 &\leftarrow 2L_2 - L_1 \\
        L_3 &\leftarrow 2L_3 - 3L_1
    \end{align*}
    On obtient le système suivant :
    $$
    \left\{\begin{array}{l}
        2 x+y=1 \\
        \quad y=-1 \\
        \quad 5 y=-5
    \end{array}\right.
    $$
    Les deux dernières équations sont les mêmes, on se ramène donc au système :
    $$
    \left\{\begin{array}{l}
        2 x+y=1 \\
        \quad y=-1
    \end{array}\right.
    $$
    Le système est parfaitement échelonné, il admet une unique solution. 
}{}
$$
\left\{\begin{array}{l}
    x+y+z+t=3 \\
    x+y+z-t=3 \\
    x-y-z-t=-1
\end{array}\right.
$$
\ifthenelse{\boolean{showSolutions}}{
    On applique les opérations suivantes pour enlever les $x$ des lignes 2 et 3 :
    \begin{align*}
        L_2 &\leftarrow L_2 - L_1 \\
        L_3 &\leftarrow L_3 - L_1
    \end{align*}
    On obtient le système suivant :
    $$
    \left\{\begin{array}{l}
        x+y+z+t=3 \\
        \quad \quad \quad -2 t=6 \\
        \quad -2y -2z -2 t=-4
    \end{array}\right.
    $$
    En échangeant les lignes 2 et 3, on obtient un système échelonné :
    $$
    \left\{\begin{array}{l}
        x+y+z+t=3 \\
        \quad -2y -2z -2 t=-4 \\
        \quad \quad \quad -2 t=6
    \end{array}\right.
    $$
    Le système échelonné possède plus d'inconnues que d'équations, on peut donc garder un paramètre libre 
}{}
