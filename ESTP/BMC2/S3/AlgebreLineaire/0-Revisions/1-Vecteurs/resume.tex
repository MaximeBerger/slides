  \begin{resumeBox}
    \textbf{Objectif en une phrase :} Savoir manipuler les vecteurs de $\mathbb{R}^n$ et les opérations associées.\\[0.25em]
    \emph{À retenir :} 
    \begin{niceitemize}
      \item Les vecteurs sont équivalents aux points de l'ensemble $\mathbb{R}^n$.
      \item Opérations sur les vecteurs : addition, multiplication par un scalaire.
      \item Produit scalaire, norme, distance, angle.
    \end{niceitemize}
  \end{resumeBox}
  \begin{formulesBox}
    % Placez ici vos formules, figures TikZ ou images.
    % Exemple : $F(\omega)=\int_{-\infty}^{+\infty} f(t)\,e^{-i\omega t}\,dt$
    % \begin{center}\includegraphics[width=.8\linewidth]{exemple.jpg}\end{center}
  \end{formulesBox}
  \begin{rappelsBox}
    \begin{niceitemize}
      \item Définition / notation utile A.
      \item Propriété ou théorème B.
      \item Piège classique à éviter.
    \end{niceitemize}
  \end{rappelsBox}
  \begin{competencesBox}
    \begin{niceitemize}
      \item Savoir calculer le produit scalaire, la norme, la distance et l'angle entre deux vecteurs.
      \item Savoir déterminer si deux vecteurs sont orthogonaux ou non.
      \item Savoir déterminer si deux vecteurs sont colinéaires ou non.
    \end{niceitemize}
  \end{competencesBox}
  \begin{exempleBox}
    \textbf{Énoncé.} ...\\
    \textbf{Solution (esquisse).} ...
  \end{exempleBox}
  \begin{exerciceBox}
    \textbf{Consigne.} ...\\
    \textit{Indice.} ...\\
    \textit{Réponse attendue (à compléter par l'étudiant).}
  \end{exerciceBox}