
\begin{resumeBox}
  \emph{À retenir dans une semaine :} 
  \begin{niceitemize}
    \item Nous interprèterons les vecteurs plutôt comme des points, et pas comme des flèches.
    \item Pour résoudre un système linéaire, on le rend \textbf{échelonné} avec le pivot de Gauss.
    \item On peut interpréter un système linéaire de 3 façons différentes : 
      \begin{itemize}
        \item[$\bullet$] Comme une intersection d'éléments géométriques (droites, plans, etc.).
        \item[$\bullet$] Comme une combinaison linéaire de vecteurs.
        \item[$\bullet$] Comme une équation matricielle.
      \end{itemize}
  \end{niceitemize}
\end{resumeBox}
% \begin{formulesBox}
%   % Placez ici vos formules, figures TikZ ou images.
%   % Exemple : $F(\omega)=\int_{-\infty}^{+\infty} f(t)\,e^{-i\omega t}\,dt$
%   % \begin{center}\includegraphics[width=.8\linewidth]{exemple.jpg}\end{center}
% \end{formulesBox}
\begin{rappelsBox}
  \begin{niceitemize}
    \item Que signifie qu'un système soit échelonné ?
    \item Quelles sont les manipulations autorisées pour le pivot de Gauss ?
    \item Comment interpréter un système linéaire comme une combinaison de vecteurs ?
  \end{niceitemize}
\end{rappelsBox}

\section{Exercices}
  \subsection{Equations de réactions Chimiques}
  Les équations de réactions chimiques peuvent être interprétées comme des systèmes linéaires. \newline
  Y a-t-il toujours une infinité de façon d'équilibrer l'équation ?


  \vspace{1em}

  Transformer le problème suivant en système linéaire. Sans le résoudre, combien a-t-il de solutions ?
  \begin{center}
    \includegraphics[width=0.8\linewidth]{0-Revisions/2-PivotDeGauss/equilibre.png}
  \end{center}

  \newpage

  \vspace{2em}

  \subsection{Combien de solutions ?}
Ces systèmes admettent-ils zéro, une ou une infinité de solutions ?

\begin{multicols}{3}
\begin{enumerate}[label=\alph*)]
\item $$\begin{cases}
    -3x + 2y &= 0\\
    -2y &= 0
  \end{cases}$$

\item $$\begin{cases}
    x + y &= 4\\
    y - z &= 0
  \end{cases}$$

\item 
$$\begin{cases}
    x + y &= 4\\
    y - z &= 0\\
    0 &= 0
  \end{cases}$$

\item 
$$\begin{cases}
    x + y &= 4\\
    0 &= 4
  \end{cases}$$

\item 
$$\begin{cases}
    3x + 6y + z &= -0.5\\
    -z &= 2.5
  \end{cases}$$

\item 
$$\begin{cases}
    x - 3y &= 2\\
    0 &= 0
  \end{cases}$$

\item 
$$\begin{cases}
    2x + 2y &= 4\\
    y &= 1\\
    0 &= 4
  \end{cases}$$

\item 
$$\begin{cases}
    2x + y &= 0
  \end{cases}$$

\item 
$$\begin{cases}
    x - y &= -1\\
    0 &= 0\\
    0 &= 4
  \end{cases}$$

\item 
$$\begin{cases}
    x + y - 3z &= -1\\
    y - z &= 2\\
    z &= 0\\
    0 &= 0
  \end{cases}$$
\end{enumerate}
\end{multicols}

\vspace{2em}
\subsection{Systèmes d'équations linéaires}
Résoudre les systèmes suivants :

\begin{multicols}{3}
\begin{enumerate}[label={}]
\item 
$$\begin{cases}
  2x + 2y = 5 \\
    x - 4y = 0
  \end{cases}$$

\item 
  $$\begin{cases}
    -x + y = 1 \\
    x + y = 2
  \end{cases}$$

\item 
$$\begin{cases}
    x - 3y + z = 1 \\
    x + y + 2z = 14
  \end{cases}$$


\item 
$$\begin{cases}
    -x - y = 1 \\
    -3x - 3y = 2
  \end{cases}$$


\item 
$$\begin{cases}
    4y + z = 20 \\
    2x - 2y + z = 0 \\
    x + z = 5 \\
    x + y - z = 10
  \end{cases}$$


\item 
$$\begin{cases}
    2x + z + w = 5 \\
    y - w = -1 \\
    3x - z - w = 0 \\
    4x + y + 2z + w = 9
  \end{cases}$$

\end{enumerate}
\end{multicols}

\vspace{2em}

\subsection{Approfondissement}
Résoudre 
$$\begin{cases}
  2 \sin \alpha-\cos \beta+3 \tan \gamma &= 3 \\
  4 \sin \alpha+2 \cos \beta-2 \tan \gamma &= 10 \\
  6 \sin \alpha-3 \cos \beta+\tan \gamma &= 9
\end{cases}$$

\newpage
\vspace{2em}

\subsection{Manipulation}

\textit{La méthode de Gauss consiste à combiner les équations d'un système pour en former de nouvelles.}

\medskip
\begin{enumerate}[label=\alph*)]
\item Peut-on obtenir l'équation \(3x - 2y = 5\) par une suite d'opérations de réduction de Gauss à partir des équations de ce système ?
\[
\begin{cases}
x + y = 1 \\
4x - y = 6
\end{cases}
\]

\item Peut-on obtenir l'équation \(5x - 3y = 2\) par une suite d'opérations de réduction de Gauss à partir des équations de ce système ?
\[
\begin{cases}
2x + 2y = 5 \\
3x + y = 4
\end{cases}
\]

\item Peut-on obtenir \(6x - 9y + 5z = -2\) par une suite d'opérations de réduction de Gauss à partir des équations de ce système ?
\[
\begin{cases}
2x + y - z = 4 \\
6x - 3y + z = 5
\end{cases}
\]
\end{enumerate}

\vspace{2em}
\subsection{Interprétation}
Choisir 3 systèmes linéaire dans un exercice précédent et l'écrire des 2 manières différentes : 
\begin{enumerate}
\item Comme une combinaison linéaires de vecteurs.
\item Comme une equation matricielle.
\end{enumerate}

Par exemple, le système 
$$\begin{cases}
x + y = 1 \\
4x - y = 6
\end{cases}$$
peut s'écrire comme une combinaison linéaire de vecteurs :
$$x\begin{pmatrix}1\\4\end{pmatrix} + y\begin{pmatrix}1\\-1\end{pmatrix} = \begin{pmatrix}1\\6\end{pmatrix}$$
ou comme une equation matricielle :
$$\begin{pmatrix}1&1\\4&-1\end{pmatrix}\begin{pmatrix}x\\y\end{pmatrix} = \begin{pmatrix}1\\6\end{pmatrix}$$

\vspace{2em}
\subsection{Pour ceux qui s'ennuient}
Une boîte contenant des pennies, des nickels et des dimes renferme treize pièces d'une valeur totale de 83 cents.
Combien y a-t-il de pièces de chaque type dans la boîte ?
(Ce sont des pièces américaines : un penny vaut 1 cent, un nickel 5 cents et un dime 10 cents.)
