

\subsection*{Combinaisons linéaires}

Les vecteurs $u$ suivants sont-ils combinaison linéaire des vecteurs $e_i$ ?
\begin{itemize}
    \item $E=\mathbb{R}^2, u=(1,2), e_1=(1,-2), e_2=(2,3)$;
    \item $E=\mathbb{R}^2, u=(1,2), e_1=(1,-2), e_2=(2,3), e_3=(-4,5)$;
    \item $E=\mathbb{R}^3, u=(2,5,3), e_1=(1,3,2), e_2=(1,-1,4)$;
    \item $E=\mathbb{R}^3, u=(3,1, m), e_1=(1,3,2), e_2=(1,-1,4)$ (discuter suivant la valeur de $m$ ).
\end{itemize}

\vspace{3em}
\subsection*{Sous-espace engendré}


Dans $\mathbb{R}^3$, on pose $u_1=\left(\begin{array}{c}1 \\ -1 \\ 2\end{array}\right)$ et $u_2=\left(\begin{array}{c}1 \\ 1 \\ -1\end{array}\right)$.
\begin{itemize}
    \item Les vecteurs $v_1=\left(\begin{array}{l}3 \\ 1 \\ 0\end{array}\right)$ et $v_2=\left(\begin{array}{c}1 \\ 5 \\ -1\end{array}\right)$ sont-ils combinaison linéaire de $u_1$ et $u_2$ ?
    \item Soit $a, b, c \in \mathbb{R}$. Démontrer que $v=\left(\begin{array}{l}a \\ b \\ c\end{array}\right)$ est combinaison linéaire de $u_1$ et $u_2$ si et seulement si $-a+3 b+2 c=0$.
    \item En déduire un vecteur de $\mathbb{R}^3$ qui n'est pas combinaison linéaire de $\boldsymbol{u}_1$ et de $\boldsymbol{u}_2$.
\end{itemize}



\vspace{3em}
\subsection*{Familles libres}
Une famille $(e_1, \cdots, e_p)$ est dite \emph{libre} si la seule combinaison linéaire $\lambda_1 e_1 + \cdots + \lambda_p e_p$ qui donne le vecteur nul est la combinaison $\lambda_1 = \cdots = \lambda_p = 0$.

Les familles suivantes sont-elles libres dans $\mathbb{R}^3$ ?
\begin{itemize}
    \item $(u, v)$ avec $u=(1,2,3)$ et $v=(-1,4,6)$;
    \item $(u, v, w)$ avec $u=(1,2,-1), v=(1,0,1)$ et $w=(0,0,1)$;
    \item $(u, v, w)$ avec $u=(1,2,-1), v=(1,0,1)$ et $w=(-1,2,-3)$;
\end{itemize}

\section*{Bases}
