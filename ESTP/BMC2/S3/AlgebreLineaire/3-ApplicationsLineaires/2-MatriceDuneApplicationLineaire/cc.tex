Que représente la $j$-ième colonne de la matrice d'une application linéaire $f$ dans une base $\mathcal{B} = (e_1, \ldots, e_n)$ ?

\vspace{2em}

On considère l'application linéaire $f : \mathbb{R}^3 \to \mathbb{R}^2$ définie par :
$$ f(x, y, z) = (2x - y + z, \, x + 3y - 2z) $$

\begin{enumerate}
    \item Écrire la matrice de $f$ dans les bases canoniques de $\mathbb{R}^3$ et $\mathbb{R}^2$.
    
    \vspace{4em}
    
    \item On pose $\mathcal{B}' = \left( \begin{pmatrix} 1 \\ 1 \\ 0 \end{pmatrix}, \begin{pmatrix} 1 \\ 0 \\ 1 \end{pmatrix}, \begin{pmatrix} 0 \\ 1 \\ 1 \end{pmatrix} \right)$ une base de $\mathbb{R}^3$ et $\mathcal{B}'' = \left( \begin{pmatrix} 1 \\ 1 \end{pmatrix}, \begin{pmatrix} 1 \\ -1 \end{pmatrix} \right)$ une base de $\mathbb{R}^2$.
    
    Écrire la matrice de $f$ dans les bases $\mathcal{B}'$ et $\mathcal{B}''$.
\end{enumerate}

