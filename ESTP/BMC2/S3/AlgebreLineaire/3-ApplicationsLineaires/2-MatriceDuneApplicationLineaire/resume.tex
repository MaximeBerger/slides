

\section{Matrice d'une application linéaire}
\textbf{idée générale :} \newline 
Pour connaître une application linéaire, il suffit de connaître les images des vecteurs d'une base.

Toute application linéaire $f: E \to F$ peut être décrite par un tableau de nombres, codant les images des vecteurs de la base de $E$.


\begin{formulesBox}
    \textbf{Comment construire la matrice d'une application linéaire ?}
        
    \begin{enumerate}[itemsep=1em]
        \item \textbf{Identifier une base de l'espace de départ} $E$. Par exemple, la base canonique:
        \begin{itemize}
            \item Pour $\mathbb{R}^2$ : $e_1 = (1, 0)$ et $e_2 = (0, 1)$.
            \item Pour $\mathbb{R}^3$ : $e_1 = (1, 0, 0)$, $e_2 = (0, 1, 0)$ et $e_3 = (0, 0, 1)$.
            \item Pour $\mathbb{R}_2[X]$ : $e_1 = 1$, $e_2 = X$, $e_3 = X^2$.
        \end{itemize}
        
        \item \textbf{Calculer les images} de ces vecteurs par l'application $f$ :
        \[ f(e_1), \quad f(e_2), \quad f(e_3), \quad \dots \]

        \item \textbf{Décomposer ces images dans une base} de l'espace d'arrivée. \newline
        Quelles sont les coordonnées de $f(e_1)$ dans la base de $F$ ?
        
        \item \textbf{Remplir les colonnes} de la matrice avec ces coordonnées :\newline
        La première colonne contient les coordonnées du vecteur $f(e_1)$, la deuxième les coordonnées de $f(e_2)$, etc.

        \textit{Si $f$ va de $\mathbb{R}^n$ dans $\mathbb{R}^m$, la matrice aura $m$ lignes et $n$ colonnes.}
    \end{enumerate}
\end{formulesBox}

\newpage

\section{Exercices d'application}

\textbf{Exercice 1 : Le cas standard}

Soit $f : \mathbb{R}^2 \to \mathbb{R}^2$ définie par $f(x, y) = (2x - 3y, x + 4y)$.
Déterminer la matrice $A$ de $f$ dans la base canonique.

\emph{Calculez d'abord $f(1, 0)$, puis $f(0, 1)$.}

\ifthenelse{\boolean{showSolutions}}{
\begin{solution}
Dans la base canonique $(e_1,e_2)$ avec $e_1=(1,0)$ et $e_2=(0,1)$ :
\[
f(e_1)=f(1,0)=(2,1),\qquad f(e_2)=f(0,1)=(-3,4).
\]
La matrice de $f$ dans la base canonique a pour colonnes les coordonnées de $f(e_1)$ puis de $f(e_2)$ :
\[
A=\begin{pmatrix}
2 & -3\\
1 & 4
\end{pmatrix}.
\]
\end{solution}
}{}

\vspace{0.5cm}

\textbf{Exercice 2 : Changement de dimension}

Soit $g : \mathbb{R}^3 \to \mathbb{R}^2$ définie par $g(x, y, z) = (x + y, y - z)$.
Déterminer la matrice $B$ associée à $g$.

\emph{Indication : L'espace de départ est $\mathbb{R}^3$, il y aura donc 3 colonnes ($g(e_1), g(e_2), g(e_3)$). L'espace d'arrivée est $\mathbb{R}^2$, il y aura donc 2 lignes.}

\ifthenelse{\boolean{showSolutions}}{
\begin{solution}
Dans les bases canoniques, avec $e_1=(1,0,0)$, $e_2=(0,1,0)$, $e_3=(0,0,1)$ :
\[
g(e_1)=g(1,0,0)=(1,0),\quad
g(e_2)=g(0,1,0)=(1,1),\quad
g(e_3)=g(0,0,1)=(0,-1).
\]
La matrice $B$ (2 lignes, 3 colonnes) a ces vecteurs comme colonnes :
\[
B=\begin{pmatrix}
1 & 1 & 0\\
0 & 1 & -1
\end{pmatrix}.
\]
\end{solution}
}{}


\begin{formulesBox}
    \textbf{Comment calculer l'image d'un vecteur par une application linéaire ?}
    Pour calculer l'image d'un vecteur $X$ via une application linéaire $f$ représentée par une matrice~$A$, il suffit de faire le produit matrice-vecteur :
    \[
        f(X) = A \cdot X
    \]
    où $X$ est vu comme une colonne de coordonnées dans la base choisie.

    \begin{itemize}
        \item Écrire le vecteur $X$ sous forme de colonne : \[
        X = \begin{pmatrix} x_1 \\ x_2 \\ \vdots \\ x_n \end{pmatrix}
        \]
        \item Multiplier la matrice~$A$ par $X$ pour obtenir le vecteur image~$Y$ :
        \[
        Y = A \cdot X
        \]
        \item Les coordonnées de $Y$ donnent $f(X)$ dans la base de l'espace d'arrivée.
    \end{itemize}

    \textbf{Exemple~:} Si 
    $
        A = \begin{pmatrix} 2 & -3 \\ 1 & 4 \end{pmatrix}
    $ 
    et $X = \begin{pmatrix} x \\ y \end{pmatrix}$, alors
    \[
        f(x, y) = A \begin{pmatrix} x \\ y \end{pmatrix} = \begin{pmatrix} 2x - 3y \\ x + 4y \end{pmatrix}
    \]

      
\end{formulesBox}

\textbf{Exercice 3 : De la matrice à la formule}

Soit $h$ une application linéaire de $\mathbb{R}^2$ dans $\mathbb{R}^2$ dont la matrice dans la base canonique est :
\[
C = \begin{pmatrix} 0 & 1 \\ -1 & 0 \end{pmatrix}
\]
Donner l'expression de $h(x, y)$.

\ifthenelse{\boolean{showSolutions}}{
\begin{solution}
Si la matrice de $h$ dans la base canonique est
\(
C=\begin{pmatrix}0&1\\-1&0\end{pmatrix},
\)
alors pour tout $(x,y)\in\mathbb{R}^2$ :
\[
h(x,y)=C\begin{pmatrix}x\\y\end{pmatrix}
=\begin{pmatrix}0\cdot x+1\cdot y\\-1\cdot x+0\cdot y\end{pmatrix}
=\begin{pmatrix}y\\-x\end{pmatrix}.
\]
Donc $\boxed{h(x,y)=(y,-x)}$.
\end{solution}
}{}

\vspace{0.5cm}

\textbf{Exercice 4 : Cas général}

Soit $k : \mathbb{R}^n \to \mathbb{R}$ définie par $k(x_1, \dots, x_n) = x_1$.
Quelle est la forme de la matrice associée ?

\emph{Indication : C'est une application qui va dans $\mathbb{R}$ (une seule composante), la matrice n'aura donc qu'une seule ligne (matrice ligne).}

\ifthenelse{\boolean{showSolutions}}{
\begin{solution}
On se place dans les bases canoniques. On calcule l'image des vecteurs de base $(e_1,\dots,e_n)$ :
\[
k(e_1)=1,\qquad k(e_2)=0,\qquad \dots,\qquad k(e_n)=0.
\]
La matrice associée (matrice ligne $1\times n$) a donc pour colonnes ces nombres, ce qui revient à :
\[
\boxed{\begin{pmatrix}1&0&0&\cdots&0\end{pmatrix}}.
\]
\end{solution}
}{}

