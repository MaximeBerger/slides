

\section{Matrice de passage d'une base à une autre}
\textbf{Idée générale :} \newline 
Lorsqu'on travaille dans un espace vectoriel, on peut choisir différentes bases pour exprimer les vecteurs. La \textbf{matrice de passage} permet de traduire les coordonnées d'un vecteur d'une base à une autre.

C'est comme un "dictionnaire" qui convertit les coordonnées : si on connaît les coordonnées d'un vecteur dans une base $\mathcal{B}$, la matrice de passage nous donne ses coordonnées dans une autre base $\mathcal{B'}$.


\begin{formulesBox}
    \textbf{Comment construire la matrice de passage ?}
        
    Soient $\mathcal{B} = (e_1, e_2, \dots, e_n)$ et $\mathcal{B'} = (e'_1, e'_2, \dots, e'_n)$ deux bases d'un espace vectoriel $E$.

    La \textbf{matrice de passage} de $\mathcal{B}$ vers $\mathcal{B'}$, notée $P_{\mathcal{B} \to \mathcal{B'}}$, se construit ainsi :

    \begin{enumerate}[itemsep=1em]
        \item \textbf{Exprimer chaque vecteur de la nouvelle base $\mathcal{B'}$} dans l'ancienne base $\mathcal{B}$ :
        \begin{itemize}
            \item Quelles sont les coordonnées de $e'_1$ dans $\mathcal{B}$ ?
            \item Quelles sont les coordonnées de $e'_2$ dans $\mathcal{B}$ ?
            \item etc.
        \end{itemize}
        
        \item \textbf{Remplir les colonnes} de la matrice :
        \begin{itemize}
            \item La 1ère colonne contient les coordonnées de $e'_1$ dans $\mathcal{B}$.
            \item La 2ème colonne contient les coordonnées de $e'_2$ dans $\mathcal{B}$.
            \item etc.
        \end{itemize}
    \end{enumerate}

    \textbf{Attention à l'ordre !} Les colonnes contiennent les vecteurs de la \textbf{nouvelle} base exprimés dans l'\textbf{ancienne} base.
\end{formulesBox}

\newpage

\section{Exercices d'application}

\textbf{Exercice 1 : Cas simple dans $\mathbb{R}^2$}

Dans $\mathbb{R}^2$, on considère la base canonique $\mathcal{B} = (e_1, e_2)$ avec $e_1 = (1,0)$ et $e_2 = (0,1)$, et une nouvelle base $\mathcal{B'} = (v_1, v_2)$ avec $v_1 = (1,1)$ et $v_2 = (1,-1)$.

Déterminer la matrice de passage $P$ de $\mathcal{B}$ vers $\mathcal{B'}$.

\emph{Indication : Il faut exprimer $v_1$ et $v_2$ comme combinaisons linéaires de $e_1$ et $e_2$.}

\ifthenelse{\boolean{showSolutions}}{
\begin{solution}
C'est simple car on exprime les vecteurs de $\mathcal{B'}$ dans la base canonique $\mathcal{B}$ !

$v_1 = (1,1) = 1 \cdot e_1 + 1 \cdot e_2$, donc ses coordonnées dans $\mathcal{B}$ sont $(1,1)$.

$v_2 = (1,-1) = 1 \cdot e_1 + (-1) \cdot e_2$, donc ses coordonnées dans $\mathcal{B}$ sont $(1,-1)$.

La matrice de passage de $\mathcal{B}$ vers $\mathcal{B'}$ est :
\[
P = \begin{pmatrix} 1 & 1 \\ 1 & -1 \end{pmatrix}
\]

\textbf{Remarque :} Quand on part de la base canonique, les colonnes sont simplement les coordonnées des vecteurs de la nouvelle base !
\end{solution}
}{}

\vspace{0.5cm}

\textbf{Exercice 2 : Matrice de passage inverse}

Avec les mêmes bases que l'exercice 1, déterminer la matrice de passage $Q$ de $\mathcal{B'}$ vers $\mathcal{B}$.

\emph{Indication : Cette fois, il faut exprimer $e_1$ et $e_2$ comme combinaisons linéaires de $v_1$ et $v_2$.}

\ifthenelse{\boolean{showSolutions}}{
\begin{solution}
On cherche $\alpha, \beta$ tels que $e_1 = \alpha v_1 + \beta v_2$, c'est-à-dire :
\[
(1,0) = \alpha(1,1) + \beta(1,-1) = (\alpha + \beta, \alpha - \beta)
\]
On obtient le système : $\alpha + \beta = 1$ et $\alpha - \beta = 0$, d'où $\alpha = \beta = \frac{1}{2}$.

Donc $e_1 = \frac{1}{2}v_1 + \frac{1}{2}v_2$, ses coordonnées dans $\mathcal{B'}$ sont $\left(\frac{1}{2}, \frac{1}{2}\right)$.

De même, pour $e_2 = \gamma v_1 + \delta v_2$ :
\[
(0,1) = \gamma(1,1) + \delta(1,-1) = (\gamma + \delta, \gamma - \delta)
\]
On obtient : $\gamma + \delta = 0$ et $\gamma - \delta = 1$, d'où $\gamma = \frac{1}{2}$ et $\delta = -\frac{1}{2}$.

Donc $e_2 = \frac{1}{2}v_1 - \frac{1}{2}v_2$, ses coordonnées dans $\mathcal{B'}$ sont $\left(\frac{1}{2}, -\frac{1}{2}\right)$.

La matrice de passage de $\mathcal{B'}$ vers $\mathcal{B}$ est :
\[
Q = \begin{pmatrix} \frac{1}{2} & \frac{1}{2} \\ \frac{1}{2} & -\frac{1}{2} \end{pmatrix}
\]

\textbf{Remarque :} On peut vérifier que $P \cdot Q = I_2$ (matrice identité), ce qui confirme que $Q = P^{-1}$.
\end{solution}
}{}


\begin{formulesBox}
    \textbf{Comment changer les coordonnées d'un vecteur ?}

    Soit $v$ un vecteur de $E$, et soient $X$ ses coordonnées dans $\mathcal{B}$ et $X'$ ses coordonnées dans $\mathcal{B'}$.

    Si $P$ est la matrice de passage de $\mathcal{B}$ vers $\mathcal{B'}$, alors :
    \[
        \boxed{X = P \cdot X'}
    \]

    Autrement dit : pour obtenir les coordonnées dans l'ancienne base, on multiplie les nouvelles coordonnées par la matrice de passage.

    \textbf{Propriétés importantes :}
    \begin{itemize}
        \item La matrice de passage est \textbf{toujours inversible}.
        \item $(P_{\mathcal{B} \to \mathcal{B'}})^{-1} = P_{\mathcal{B'} \to \mathcal{B}}$
        \item Pour obtenir les nouvelles coordonnées : $X' = P^{-1} \cdot X$
    \end{itemize}
      
\end{formulesBox}

\vspace{0.5cm}

\textbf{Exercice 3 : Changement de coordonnées}

On reprend les bases de l'exercice 1. Soit $v = (3, 1)$ un vecteur de $\mathbb{R}^2$ (coordonnées dans la base canonique $\mathcal{B}$).

Quelles sont les coordonnées de $v$ dans la base $\mathcal{B'} = (v_1, v_2)$ ?

\ifthenelse{\boolean{showSolutions}}{
\begin{solution}
On utilise la formule $X' = P^{-1} \cdot X$ avec $X = \begin{pmatrix} 3 \\ 1 \end{pmatrix}$ et la matrice $P$ calculée à l'exercice 1.

On calcule d'abord $P^{-1}$. Le déterminant de $P = \begin{pmatrix} 1 & 1 \\ 1 & -1 \end{pmatrix}$ est $\det(P) = -1 - 1 = -2$, donc :
\[
P^{-1} = \frac{1}{-2} \begin{pmatrix} -1 & -1 \\ -1 & 1 \end{pmatrix} = \begin{pmatrix} \frac{1}{2} & \frac{1}{2} \\ \frac{1}{2} & -\frac{1}{2} \end{pmatrix}
\]

Puis :
\[
X' = P^{-1} \cdot X = \begin{pmatrix} \frac{1}{2} & \frac{1}{2} \\ \frac{1}{2} & -\frac{1}{2} \end{pmatrix} \begin{pmatrix} 3 \\ 1 \end{pmatrix}
= \begin{pmatrix} \frac{3}{2} + \frac{1}{2} \\ \frac{3}{2} - \frac{1}{2} \end{pmatrix}
= \begin{pmatrix} 2 \\ 1 \end{pmatrix}
\]

Donc les coordonnées de $v$ dans $\mathcal{B'}$ sont $(2, 1)$.

\textbf{Vérification :} $2 \cdot v_1 + 1 \cdot v_2 = 2(1,1) + 1(1,-1) = (2,2) + (1,-1) = (3,1)$ ✓
\end{solution}
}{}

\vspace{0.5cm}

\textbf{Exercice 4 : Dans $\mathbb{R}^3$}

Dans $\mathbb{R}^3$, on considère la base canonique $\mathcal{B} = (e_1, e_2, e_3)$ et la base $\mathcal{B'} = (f_1, f_2, f_3)$ avec :
\[
f_1 = (1, 0, 1), \quad f_2 = (0, 1, 1), \quad f_3 = (1, 1, 0)
\]

Déterminer la matrice de passage $P$ de $\mathcal{B}$ vers $\mathcal{B'}$.

\ifthenelse{\boolean{showSolutions}}{
\begin{solution}
On exprime les vecteurs de $\mathcal{B'}$ dans la base canonique $\mathcal{B}$ :
\begin{itemize}
    \item $f_1 = (1, 0, 1)$ a pour coordonnées $(1, 0, 1)$ dans $\mathcal{B}$.
    \item $f_2 = (0, 1, 1)$ a pour coordonnées $(0, 1, 1)$ dans $\mathcal{B}$.
    \item $f_3 = (1, 1, 0)$ a pour coordonnées $(1, 1, 0)$ dans $\mathcal{B}$.
\end{itemize}

La matrice de passage de $\mathcal{B}$ vers $\mathcal{B'}$ est donc :
\[
P = \begin{pmatrix} 1 & 0 & 1 \\ 0 & 1 & 1 \\ 1 & 1 & 0 \end{pmatrix}
\]

\textbf{Remarque :} Quand on part \textbf{de} la base canonique, les colonnes sont simplement les coordonnées des vecteurs de la nouvelle base !
\end{solution}
}{}

\newpage

\begin{formulesBox}
    \textbf{Formule de changement de base pour une matrice}

    Soit $f : E \to E$ une application linéaire. Si $A$ est la matrice de $f$ dans la base $\mathcal{B}$ et $A'$ est la matrice de $f$ dans la base $\mathcal{B'}$, alors :
    \[
        \boxed{A' = P^{-1} A P}
    \]
    où $P$ est la matrice de passage de $\mathcal{B}$ vers $\mathcal{B'}$.

    \textbf{Interprétation :} Pour exprimer $f$ dans une nouvelle base :
    \begin{enumerate}
        \item On traduit de $\mathcal{B'}$ vers $\mathcal{B}$ (multiplication par $P$, car $X = P \cdot X'$)
        \item On applique $f$ dans $\mathcal{B}$ (multiplication par $A$)
        \item On traduit de $\mathcal{B}$ vers $\mathcal{B'}$ (multiplication par $P^{-1}$)
    \end{enumerate}
      
\end{formulesBox}

\vspace{0.5cm}

\textbf{Exercice 5 : Changement de base d'une matrice}

Soit $f : \mathbb{R}^2 \to \mathbb{R}^2$ l'application linéaire dont la matrice dans la base canonique $\mathcal{B}$ est :
\[
A = \begin{pmatrix} 3 & 1 \\ 1 & 3 \end{pmatrix}
\]

On considère la base $\mathcal{B'} = (v_1, v_2)$ avec $v_1 = (1, 1)$ et $v_2 = (1, -1)$.

Déterminer la matrice $A'$ de $f$ dans la base $\mathcal{B'}$.

\emph{Indication : Utiliser la formule $A' = P^{-1} A P$ avec $P$ la matrice de passage de $\mathcal{B}$ vers $\mathcal{B'}$.}

\ifthenelse{\boolean{showSolutions}}{
\begin{solution}
La matrice de passage de $\mathcal{B}$ vers $\mathcal{B'}$ est (voir exercice 1) :
\[
P = \begin{pmatrix} 1 & 1 \\ 1 & -1 \end{pmatrix}
\]

On calcule $P^{-1}$. Le déterminant de $P$ est $\det(P) = -1 - 1 = -2$, donc :
\[
P^{-1} = \frac{1}{-2} \begin{pmatrix} -1 & -1 \\ -1 & 1 \end{pmatrix} = \begin{pmatrix} \frac{1}{2} & \frac{1}{2} \\ \frac{1}{2} & -\frac{1}{2} \end{pmatrix}
\]

Calculons $AP$ :
\[
AP = \begin{pmatrix} 3 & 1 \\ 1 & 3 \end{pmatrix} \begin{pmatrix} 1 & 1 \\ 1 & -1 \end{pmatrix} = \begin{pmatrix} 4 & 2 \\ 4 & -2 \end{pmatrix}
\]

Puis $A' = P^{-1}(AP)$ :
\[
A' = \begin{pmatrix} \frac{1}{2} & \frac{1}{2} \\ \frac{1}{2} & -\frac{1}{2} \end{pmatrix} \begin{pmatrix} 4 & 2 \\ 4 & -2 \end{pmatrix} = \begin{pmatrix} 4 & 0 \\ 0 & 2 \end{pmatrix}
\]

\textbf{Remarque :} La matrice $A'$ est diagonale ! Cela signifie que $v_1$ et $v_2$ sont des vecteurs propres de $f$, avec pour valeurs propres $4$ et $2$ respectivement.
\end{solution}
}{}

\vspace{0.5cm}

\textbf{Exercice 6 : Application géométrique}

Soit $r$ la rotation d'angle $\frac{\pi}{2}$ dans $\mathbb{R}^2$. Sa matrice dans la base canonique $\mathcal{B}$ est :
\[
R = \begin{pmatrix} 0 & -1 \\ 1 & 0 \end{pmatrix}
\]

On considère la base $\mathcal{B'} = (u_1, u_2)$ avec $u_1 = (1, 1)$ et $u_2 = (-1, 1)$.

\begin{enumerate}
    \item Déterminer la matrice de passage $P$ de $\mathcal{B}$ vers $\mathcal{B'}$.
    \item Calculer $P^{-1}$.
    \item Déterminer la matrice $R'$ de $r$ dans la base $\mathcal{B'}$.
    \item Interpréter géométriquement le résultat.
\end{enumerate}

\ifthenelse{\boolean{showSolutions}}{
\begin{solution}
\textbf{1.} On exprime les vecteurs de $\mathcal{B'}$ dans la base canonique :
\[
P = \begin{pmatrix} 1 & -1 \\ 1 & 1 \end{pmatrix}
\]

\textbf{2.} Le déterminant est $\det(P) = 1 \times 1 - (-1) \times 1 = 2$, donc :
\[
P^{-1} = \frac{1}{2} \begin{pmatrix} 1 & 1 \\ -1 & 1 \end{pmatrix}
\]

\textbf{3.} Calculons $RP$ :
\[
RP = \begin{pmatrix} 0 & -1 \\ 1 & 0 \end{pmatrix} \begin{pmatrix} 1 & -1 \\ 1 & 1 \end{pmatrix} = \begin{pmatrix} -1 & -1 \\ 1 & -1 \end{pmatrix}
\]

Puis $R' = P^{-1}(RP)$ :
\[
R' = \frac{1}{2} \begin{pmatrix} 1 & 1 \\ -1 & 1 \end{pmatrix} \begin{pmatrix} -1 & -1 \\ 1 & -1 \end{pmatrix} = \frac{1}{2} \begin{pmatrix} 0 & -2 \\ 2 & 0 \end{pmatrix} = \begin{pmatrix} 0 & -1 \\ 1 & 0 \end{pmatrix}
\]

\textbf{4.} On remarque que $R' = R$ : la matrice de la rotation est la même dans les deux bases ! C'est parce que $\mathcal{B'}$ est aussi une base orthonormée (à un facteur $\sqrt{2}$ près) obtenue par rotation de $\frac{\pi}{4}$ de la base canonique. Une rotation commute avec un changement de base qui est lui-même une rotation.
\end{solution}
}{}

