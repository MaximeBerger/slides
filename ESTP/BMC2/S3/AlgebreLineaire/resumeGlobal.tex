% =====================================================================
% Template LaTeX – Traces distribuées aux étudiants
% Auteur : (à compléter)
% Compilation : pdflatex/xelatex (pdflatex recommandé ici)
% =====================================================================
\documentclass[11pt,a4paper]{report}

% -------------------- Encodage & langue --------------------
\usepackage[T1]{fontenc}
\usepackage[utf8]{inputenc}
\usepackage[french]{babel}
\usepackage{lmodern}
\usepackage{microtype}
\usepackage{amsmath, amssymb}
\usepackage{multicol}
\usepackage{enumitem}


% -------------------- Mise en page --------------------------
\usepackage[a4paper,margin=2cm]{geometry}
\usepackage{fancyhdr}
\usepackage{parskip}      % espace entre paragraphes
\setlength{\parindent}{0pt}

% -------------------- Couleurs & liens ----------------------
\usepackage{xcolor}
\definecolor{Theme}{HTML}{0E7490} % teal-700
\definecolor{ThemeLight}{HTML}{E0F2F1}
\definecolor{Accent}{HTML}{F59E0B} % amber-500
\definecolor{Gray}{HTML}{374151}
\usepackage[colorlinks=true,linkcolor=Theme,urlcolor=Theme,citecolor=Theme]{hyperref}

% -------------------- Graphiques / décor --------------------
\usepackage{tikz}
\usetikzlibrary{patterns,positioning,calc}
\usepackage{graphicx}
\usepackage{tcolorbox}
\tcbuselibrary{skins,breakable,hooks,most}
\usepackage{fontawesome5}

% -------------------- Titres -------------------------------
\usepackage{titlesec}
\titleformat{\chapter}[display]
  {\Huge\bfseries\color{Theme}}
  {\filright\rule{0.75\linewidth}{1.2pt}\\[3pt]{Algèbre linéaire - Chapitre~\thechapter}}
  {0.2ex}
  {\filright}
  [\vspace{0.1ex}\rule{0.35\linewidth}{1.2pt}]

\titleformat{\section}
  {\Large\bfseries\color{Gray}}
  {\thesection}{0.6em}{}

% -------------------- En-têtes / pieds ---------------------
\pagestyle{fancy}
\fancyhf{}
\fancyhead[L]{\color{Gray}\leftmark}
\fancyhead[R]{\color{Gray}\textit{BMC2}}
\fancyfoot[L]{\color{Gray}\small Auteur~: \textit{M. Berger}}
\fancyfoot[R]{\color{Gray}\small p.\ \thepage}
\renewcommand{\headrulewidth}{0pt}
\renewcommand{\footrulewidth}{0pt}

% -------------------- Macros utilitaires -------------------


% Tcolorboxes stylisées
\tcbset{tracebox/.style={breakable,enhanced,sharp corners,boxrule=0pt,frame hidden,arc=2mm,
  colback=white,coltitle=black,fonttitle=\bfseries\large,
  borderline west={2mm}{0pt}{Theme},
  before skip=8pt,after skip=8pt,drop fuzzy shadow}}

\newtcolorbox{resumeBox}{tracebox,title={\faStickyNote\quad Résumé des idées}}
\newtcolorbox{rappelsBox}{tracebox,title={\faRedo\quad Ce que je dois savoir }}
\newtcolorbox{exempleBox}{tracebox,title={\faChalkboardTeacher\quad Exemple vu ensemble}}

% Encadré « Formules & illustrations »
\newtcolorbox{formulesBox}{tracebox,title={\faCalculator\quad Méthode},colback=ThemeLight}

% Astuce : puces clean
\newenvironment{niceitemize}{\begin{itemize}\setlength{\itemsep}{0.25em}\color{Gray}}{\end{itemize}}

% Raccourci pour une « Trace » complète
% Usage : \TraceSection{Titre}{Objectif court}
\newcommand{\TraceSection}[2]{%
  
}

\makeatletter
\renewcommand{\thesubsection}{\arabic{subsection}}
\renewcommand{\p@subsection}{}
\makeatother
% -------------------- Page de titre ------------------------
\title{\textbf{Traces de cours}\\\large (résumés, formules, exemples, mini-exercices)}
\author{ BMC2 }
\date{\today}


\usepackage{mdframed}
\usepackage{ifthen}

% -------------------- Corrections (solutions) ---------------
% Copié depuis : ESTP/PGE1/S5/AnalyseAlgebre/8-TD4/index.tex
\newenvironment{solution}
{
    \vspace{0.5em}
    \begin{mdframed}[backgroundcolor=ThemeLight,leftmargin=0,rightmargin=0,skipabove=0.2em,skipbelow=0.2em]
    \textbf{Solution.}\\[0.5em]
}
{
    \end{mdframed}
    \vspace{0.5em}
}

% \usepackage[sf]{titlesec}
% Définition de la variable pour afficher les corrections
\newboolean{showSolutions}
% Décommentez la ligne suivante pour afficher les solutions
\input \jobname.adr

% -------------------- Document ----------------------------
\begin{document}




% ================== Séquence 1 ==================

% \chapter{Les vecteurs de $\mathbb{R}^n$}
% \section*{Les fonctions périodiques}

\subsection{Les fonctions complexes périodiques}

Les fonctions réelles suivantes sont-elles périodiques et si oui, quelle est leur période ?

\ifthenelse{\boolean{showSolutions}}{
    \vspace{2em}
    \begin{mdframed}
    \begin{enumerate}
    \item $\cos(x)$ : Oui, période $T = 2\pi$
    \item $\sin(2\pi x)$ : Oui, période $T = 1$ (car $\sin(2\pi(x+1)) = \sin(2\pi x + 2\pi) = \sin(2\pi x)$)
    \item $\cos(x/2)$ : Oui, période $T = 4\pi$ (car $\cos((x+4\pi)/2) = \cos(x/2 + 2\pi) = \cos(x/2)$)
    \item $\sin(2x) + \cos(3x)$ : Oui, période $T = 2\pi$ (le PPCM des périodes $\pi$ et $\frac{2\pi}{3}$)
    \item $\sin(nx)$ : Oui, période $T = \frac{2\pi}{n}$
    \item $\cos\left(\frac{3x}{2}-\frac{\pi}{4}\right)$ : Oui, période $T = \frac{4\pi}{3}$
    \item $x-\lfloor x\rfloor$ : Oui, période $T = 1$ (c'est la fonction partie fractionnaire)
\end{enumerate}
\end{mdframed}
}{}
\ifthenelse{\boolean{showSolutions}}{}
{\begin{multicols}{2}}
\begin{enumerate}
    \item $\displaystyle \cos(x)$
    
    \item $\displaystyle \sin(2\pi x)$
    \item $\displaystyle \cos(x/2)$
    \item $\displaystyle \sin(2x) + \cos(3x)$
    \item $\displaystyle \sin(nx)$, $n$ est un entier naturel non nul
    \item $\displaystyle \cos \left(\frac{3 x}{2}-\frac{\pi}{4}\right)$
    \item $\displaystyle x-\lfloor x\rfloor$
\end{enumerate}
\ifthenelse{\boolean{showSolutions}}{}{
\end{multicols}
}

Les fonctions complexes suivantes sont-elles périodiques et si oui, quelle est leur période ?

\ifthenelse{\boolean{showSolutions}}{
    \vspace{2em}
    \begin{mdframed}
    \begin{enumerate}
    \item $e^{ix}$ : Oui, période $T = 2\pi$ (car $e^{i(x+2\pi)} = e^{ix}e^{i2\pi} = e^{ix} \cdot 1 = e^{ix}$)
    \item $e^{2ix}$ : Oui, période $T = \pi$ (car $e^{2i(x+\pi)} = e^{2ix}e^{i2\pi} = e^{2ix}$)
    \item $e^{ix/2\pi}$ : Oui, période $T = 4\pi^2$ (car $e^{i(x+4\pi^2)/2\pi} = e^{ix/2\pi}e^{i2\pi} = e^{ix/2\pi}$)
    \item $e^{2i\pi x/T}$ : Oui, période $T$ (car $e^{2i\pi (x+T)/T} = e^{2i\pi x/T}e^{i2\pi} = e^{2i\pi x/T}$)
    \item $e^{inx} + e^{ipx}$ : Oui, période $T = \frac{2\pi}{\text{PGCD}(n,p)}$ (le PPCM des périodes $\frac{2\pi}{n}$ et $\frac{2\pi}{p}$)
\end{enumerate}
\end{mdframed}
}{}
\ifthenelse{\boolean{showSolutions}}{}
{\begin{multicols}{2}}
\begin{enumerate}
    \item $\displaystyle e^{ix}$
    \item $\displaystyle e^{2ix}$
    \item $\displaystyle e^{ix/2\pi}$
    \item $\displaystyle e^{2i\pi x/T}$, $T$ est un réel strictement positif
    \item $\displaystyle e^{inx} + e^{ipx}$
\end{enumerate}
\ifthenelse{\boolean{showSolutions}}{}{
\end{multicols}
}


\section*{Produit scalaire réel}

\subsection{Définition}
On appelle produit scalaire sur un espace vectoriel $E$ une application 
$$\langle \cdot, \cdot \rangle : E \times E \to \mathbb{R}$$
telle que :
\begin{multicols}{2}
\begin{itemize}
    \item[*] symétrie : $\langle u, v \rangle = \langle v, u \rangle$ 
    \item[*] linéarité à gauche : $\langle \lambda u + v, w \rangle = \lambda \langle u, w \rangle + \langle v, w \rangle$
    \item[*] positivité : $\langle u, u \rangle \geq 0$
    \item[*] définie positivité : $\langle u, u \rangle = 0 \iff u = 0$
\end{itemize}
\end{multicols}


\vspace{1em}

\subsection{Dans $\mathbb{R}^3$}

On se place dans $\mathbb{R}^3$, qu'on munit de la base 
$$e_1 = (1,2,1), \qquad e_2 = (2,1,-4), \qquad e_3 = (-3,2,-1)$$

\begin{enumerate}
\item La famille est-elle orthogonale ? 
\item Est-elle orthonormée ? Si non, définissez une base $(f_1,f_2,f_3)$ orthonormée à partir de la famille $(e_1,e_2,e_3)$. 
\end{enumerate}

Soit $u$ un vecteur de $\mathbb{R}^3$, on note $u_i$ ses coordonnées dans la base orthonormée $(f_1,f_2,f_3)$. Cela signifie que 
$$u = u_1 f_1 + u_2 f_2 + u_3 f_3$$

Déterminer les coordonnées de $u = (1,0,1)$ dans la base $(f_1,f_2,f_3)$.

\ifthenelse{\boolean{showSolutions}}{
    \vspace{2em}
    \begin{mdframed}
    \textbf{1.} Vérifions si la famille est orthogonale :
    
    $\langle e_1, e_2 \rangle = 1 \cdot 2 + 2 \cdot 1 + 1 \cdot (-4) = 2 + 2 - 4 = 0$ $\checkmark$
    
    $\langle e_1, e_3 \rangle = 1 \cdot (-3) + 2 \cdot 2 + 1 \cdot (-1) = -3 + 4 - 1 = 0$ $\checkmark$
    
    $\langle e_2, e_3 \rangle = 2 \cdot (-3) + 1 \cdot 2 + (-4) \cdot (-1) = -6 + 2 + 4 = 0$ $\checkmark$
    
    La famille est orthogonale.
    
    \textbf{2.} Vérifions si elle est orthonormée :
    
    $\|e_1\|^2 = 1^2 + 2^2 + 1^2 = 6$, donc $\|e_1\| = \sqrt{6}$
    
    $\|e_2\|^2 = 2^2 + 1^2 + (-4)^2 = 4 + 1 + 16 = 21$, donc $\|e_2\| = \sqrt{21}$
    
    $\|e_3\|^2 = (-3)^2 + 2^2 + (-1)^2 = 9 + 4 + 1 = 14$, donc $\|e_3\| = \sqrt{14}$
    
    La famille n'est pas orthonormée. Une base orthonormée est :
    $$f_1 = \frac{e_1}{\|e_1\|} = \frac{1}{\sqrt{6}}(1,2,1) = \left(\frac{1}{\sqrt{6}}, \frac{2}{\sqrt{6}}, \frac{1}{\sqrt{6}}\right)$$
    
    $$f_2 = \frac{e_2}{\|e_2\|} = \frac{1}{\sqrt{21}}(2,1,-4) = \left(\frac{2}{\sqrt{21}}, \frac{1}{\sqrt{21}}, \frac{-4}{\sqrt{21}}\right)$$
    
    $$f_3 = \frac{e_3}{\|e_3\|} = \frac{1}{\sqrt{14}}(-3,2,-1) = \left(\frac{-3}{\sqrt{14}}, \frac{2}{\sqrt{14}}, \frac{-1}{\sqrt{14}}\right)$$
    
    \textbf{3.} Coordonnées de $u = (1,0,1)$ dans la base $(f_1,f_2,f_3)$ :
    
    $u_1 = \langle u, f_1 \rangle = 1 \cdot \frac{1}{\sqrt{6}} + 0 \cdot \frac{2}{\sqrt{6}} + 1 \cdot \frac{1}{\sqrt{6}} = \frac{2}{\sqrt{6}} = \frac{\sqrt{6}}{3}$
    
    $u_2 = \langle u, f_2 \rangle = 1 \cdot \frac{2}{\sqrt{21}} + 0 \cdot \frac{1}{\sqrt{21}} + 1 \cdot \frac{-4}{\sqrt{21}} = \frac{-2}{\sqrt{21}}$
    
    $u_3 = \langle u, f_3 \rangle = 1 \cdot \frac{-3}{\sqrt{14}} + 0 \cdot \frac{2}{\sqrt{14}} + 1 \cdot \frac{-1}{\sqrt{14}} = \frac{-4}{\sqrt{14}}$
    
    Donc $u = \frac{\sqrt{6}}{3}f_1 - \frac{2}{\sqrt{21}}f_2 - \frac{4}{\sqrt{14}}f_3$
\end{mdframed}
}{}

\vspace{1em}

\subsection{Dans $\mathbb{R}[X]$}

\begin{itemize}
    \item Quelle est la dimension de $\mathbb{R}[X]$ ?
\end{itemize}
La famille $(1,X,X^2,X^3, \cdots )$ est appelée base hilbertienne de $\mathbb{R}[X]$ : tout élément de $\mathbb{R}[X]$ peut s'écrire comme une combinaison linéaire finie de vecteurs de cette famille.

On munit cet espace du produit scalaire : 
$$ \langle P, Q \rangle = \int_{0}^{1} P(x) Q(x) dx $$

\begin{itemize}
    \item Montrer que c'est bien un produit scalaire en vérifiant les propriétés ci-dessus.
    \item La famille $(1,X,X^2,X^3, \cdots )$ est-elle orthogonale ? Est-elle orthonormée ?
    \item Comment trouver $a, b, c$ tels que la famille $(1, X-a, X^2-bX-c)$ soit orthogonale ?
    \item Quelles sont les coordonnées de $P = 1+2X+3X^2$ dans la base $(1, X, X^2, \cdots)$ ?
    \item Peut-on retrouver ces coordonnées avec le produit scalaire comme dans l'exercice précédent ?
\end{itemize}
\vspace{1em}

\section*{Produit scalaire complexe}
\subsection{Définition}
On appelle produit scalaire sur un espace vectoriel $E$ une application 
$$\langle \cdot, \cdot \rangle : E \times E \to \mathbb{R}$$
telle que :
\begin{multicols}{2}
\begin{itemize}
    \item[*] symétrie conjuguée : $\langle u, v \rangle = \overline{\langle v, u \rangle}$
    \item[*] linéarité à gauche : $\langle \lambda u + v, w \rangle = \lambda \langle u, w \rangle + \langle v, w \rangle$
    \item[*] positivité : $\langle u, u \rangle \geq 0$
    \item[*] définie positivité : $\langle u, u \rangle = 0 \iff u = 0$
\end{itemize}
\end{multicols}


\subsection{Dans l'espace des fonctions complexes $2\pi$-périodiques}

On définit le produit scalaire :
$$
\langle f, g \rangle = \int_{0}^{2\pi} f(x) \overline{g(x)} dx
$$

Montrer que c'est un produit scalaire.

Montrer que la famille $(e^{inx})_{n \in \mathbb{Z}}$ est orthonormée.

Les coefficients de Fourier d'une fonction $f$ sont les coordonnées de $f$ dans la base $(e^{inx})_{n \in \mathbb{Z}}$.

Déterminer les coefficients de Fourier des fonctions suivantes :

\ifthenelse{\boolean{showSolutions}}{
    \vspace{2em}
    \begin{mdframed}
    \textbf{Démonstration que c'est un produit scalaire :}
    
    Les propriétés de symétrie conjuguée, linéarité et positivité découlent des propriétés de l'intégrale et du conjugué complexe.
    
    \textbf{Démonstration que $(e^{inx})_{n \in \mathbb{Z}}$ est orthonormée :}
    
    Pour $n = m$ : $\langle e^{inx}, e^{inx} \rangle = \int_0^{2\pi} e^{inx} \overline{e^{inx}} dx = \int_0^{2\pi} 1 dx = 2\pi$
    
    Pour $n \neq m$ : $\langle e^{inx}, e^{imx} \rangle = \int_0^{2\pi} e^{inx} \overline{e^{imx}} dx = \int_0^{2\pi} e^{i(n-m)x} dx = 0$
    
    Donc la famille $(e^{inx})_{n \in \mathbb{Z}}$ est orthogonale. Pour l'orthonormaliser, on divise par $\sqrt{2\pi}$.
    
    \textbf{Coefficients de Fourier :}
    \begin{enumerate}
    \item $\cos(x) = \frac{e^{ix} + e^{-ix}}{2}$, donc $c_1 = \frac{1}{2}$, $c_{-1} = \frac{1}{2}$, $c_n = 0$ sinon.
    
    \item $\sin(2\pi x)$ : Cette fonction n'est pas $2\pi$-périodique ! Elle est de période $1$.
    
    \item $\cos(x/2) = \frac{e^{ix/2} + e^{-ix/2}}{2}$, donc $c_{1/2} = \frac{1}{2}$, $c_{-1/2} = \frac{1}{2}$, $c_n = 0$ sinon.
    
    \item $\sin(2x) + \cos(3x) = \frac{e^{i2x} - e^{-i2x}}{2i} + \frac{e^{i3x} + e^{-i3x}}{2}$
    Donc $c_2 = \frac{1}{2i}$, $c_{-2} = -\frac{1}{2i}$, $c_3 = \frac{1}{2}$, $c_{-3} = \frac{1}{2}$, $c_n = 0$ sinon.
    
    \item $f(x) = e^{-x}$ sur $[0, 2\pi]$ :
    $$c_n = \frac{1}{2\pi}\int_0^{2\pi} e^{-x} e^{-inx} dx = \frac{1}{2\pi}\int_0^{2\pi} e^{-(1+in)x} dx$$
    $$= \frac{1}{2\pi}\left[\frac{e^{-(1+in)x}}{-(1+in)}\right]_0^{2\pi} = \frac{1}{2\pi} \cdot \frac{1-e^{-2\pi(1+in)}}{1+in} = \frac{1-e^{-2\pi}e^{-2\pi in}}{2\pi(1+in)}$$
\end{enumerate}
\end{mdframed}
}{}
\ifthenelse{\boolean{showSolutions}}{}
{\begin{multicols}{2}}
\begin{enumerate}
\item $\displaystyle \cos(x)$
\item $\displaystyle \sin(2\pi x)$
\item $\displaystyle \cos(x/2)$
\item $\displaystyle \sin(2x) + \cos(3x)$
\item $\displaystyle \exp^{-x}$ sur l'intervalle $[0, 2\pi]$
\end{enumerate}
\ifthenelse{\boolean{showSolutions}}{}{
\end{multicols}
}



% \newpage



% ====== Controle continu =======
% \chapter{Controle continu - Pivot de Gauss}
% Quelles sont les $3$ questions qu'on doit se poser pour montrer qu'un ensemble est un espace vectoriel ?

\vspace{2em}

Les ensembles suivants sont-ils des espaces vectoriels inclus dans $\mathbb{R}^n$ ? 
$$ \mathrm{F}=\left\{\left(x_1, \ldots, x_{\mathrm{n}}\right) \in \mathbb{R}^{\mathrm{n}} / x_1+\ldots+x_{\mathrm{n}}=0\right\} \qquad 
    \mathrm{F}=\left\{\left(x_1, \ldots, x_{\mathrm{n}}\right) \in \mathbb{R}^{\mathrm{n}} / x_1 \times x_2=0\right\}
    $$

% ================== Séquence 2 ==================
% \setcounter{chapter}{1}
% \chapter{Espaces vectoriels}

% \section*{Les fonctions périodiques}

\subsection{Les fonctions complexes périodiques}

Les fonctions réelles suivantes sont-elles périodiques et si oui, quelle est leur période ?

\ifthenelse{\boolean{showSolutions}}{
    \vspace{2em}
    \begin{mdframed}
    \begin{enumerate}
    \item $\cos(x)$ : Oui, période $T = 2\pi$
    \item $\sin(2\pi x)$ : Oui, période $T = 1$ (car $\sin(2\pi(x+1)) = \sin(2\pi x + 2\pi) = \sin(2\pi x)$)
    \item $\cos(x/2)$ : Oui, période $T = 4\pi$ (car $\cos((x+4\pi)/2) = \cos(x/2 + 2\pi) = \cos(x/2)$)
    \item $\sin(2x) + \cos(3x)$ : Oui, période $T = 2\pi$ (le PPCM des périodes $\pi$ et $\frac{2\pi}{3}$)
    \item $\sin(nx)$ : Oui, période $T = \frac{2\pi}{n}$
    \item $\cos\left(\frac{3x}{2}-\frac{\pi}{4}\right)$ : Oui, période $T = \frac{4\pi}{3}$
    \item $x-\lfloor x\rfloor$ : Oui, période $T = 1$ (c'est la fonction partie fractionnaire)
\end{enumerate}
\end{mdframed}
}{}
\ifthenelse{\boolean{showSolutions}}{}
{\begin{multicols}{2}}
\begin{enumerate}
    \item $\displaystyle \cos(x)$
    
    \item $\displaystyle \sin(2\pi x)$
    \item $\displaystyle \cos(x/2)$
    \item $\displaystyle \sin(2x) + \cos(3x)$
    \item $\displaystyle \sin(nx)$, $n$ est un entier naturel non nul
    \item $\displaystyle \cos \left(\frac{3 x}{2}-\frac{\pi}{4}\right)$
    \item $\displaystyle x-\lfloor x\rfloor$
\end{enumerate}
\ifthenelse{\boolean{showSolutions}}{}{
\end{multicols}
}

Les fonctions complexes suivantes sont-elles périodiques et si oui, quelle est leur période ?

\ifthenelse{\boolean{showSolutions}}{
    \vspace{2em}
    \begin{mdframed}
    \begin{enumerate}
    \item $e^{ix}$ : Oui, période $T = 2\pi$ (car $e^{i(x+2\pi)} = e^{ix}e^{i2\pi} = e^{ix} \cdot 1 = e^{ix}$)
    \item $e^{2ix}$ : Oui, période $T = \pi$ (car $e^{2i(x+\pi)} = e^{2ix}e^{i2\pi} = e^{2ix}$)
    \item $e^{ix/2\pi}$ : Oui, période $T = 4\pi^2$ (car $e^{i(x+4\pi^2)/2\pi} = e^{ix/2\pi}e^{i2\pi} = e^{ix/2\pi}$)
    \item $e^{2i\pi x/T}$ : Oui, période $T$ (car $e^{2i\pi (x+T)/T} = e^{2i\pi x/T}e^{i2\pi} = e^{2i\pi x/T}$)
    \item $e^{inx} + e^{ipx}$ : Oui, période $T = \frac{2\pi}{\text{PGCD}(n,p)}$ (le PPCM des périodes $\frac{2\pi}{n}$ et $\frac{2\pi}{p}$)
\end{enumerate}
\end{mdframed}
}{}
\ifthenelse{\boolean{showSolutions}}{}
{\begin{multicols}{2}}
\begin{enumerate}
    \item $\displaystyle e^{ix}$
    \item $\displaystyle e^{2ix}$
    \item $\displaystyle e^{ix/2\pi}$
    \item $\displaystyle e^{2i\pi x/T}$, $T$ est un réel strictement positif
    \item $\displaystyle e^{inx} + e^{ipx}$
\end{enumerate}
\ifthenelse{\boolean{showSolutions}}{}{
\end{multicols}
}


\section*{Produit scalaire réel}

\subsection{Définition}
On appelle produit scalaire sur un espace vectoriel $E$ une application 
$$\langle \cdot, \cdot \rangle : E \times E \to \mathbb{R}$$
telle que :
\begin{multicols}{2}
\begin{itemize}
    \item[*] symétrie : $\langle u, v \rangle = \langle v, u \rangle$ 
    \item[*] linéarité à gauche : $\langle \lambda u + v, w \rangle = \lambda \langle u, w \rangle + \langle v, w \rangle$
    \item[*] positivité : $\langle u, u \rangle \geq 0$
    \item[*] définie positivité : $\langle u, u \rangle = 0 \iff u = 0$
\end{itemize}
\end{multicols}


\vspace{1em}

\subsection{Dans $\mathbb{R}^3$}

On se place dans $\mathbb{R}^3$, qu'on munit de la base 
$$e_1 = (1,2,1), \qquad e_2 = (2,1,-4), \qquad e_3 = (-3,2,-1)$$

\begin{enumerate}
\item La famille est-elle orthogonale ? 
\item Est-elle orthonormée ? Si non, définissez une base $(f_1,f_2,f_3)$ orthonormée à partir de la famille $(e_1,e_2,e_3)$. 
\end{enumerate}

Soit $u$ un vecteur de $\mathbb{R}^3$, on note $u_i$ ses coordonnées dans la base orthonormée $(f_1,f_2,f_3)$. Cela signifie que 
$$u = u_1 f_1 + u_2 f_2 + u_3 f_3$$

Déterminer les coordonnées de $u = (1,0,1)$ dans la base $(f_1,f_2,f_3)$.

\ifthenelse{\boolean{showSolutions}}{
    \vspace{2em}
    \begin{mdframed}
    \textbf{1.} Vérifions si la famille est orthogonale :
    
    $\langle e_1, e_2 \rangle = 1 \cdot 2 + 2 \cdot 1 + 1 \cdot (-4) = 2 + 2 - 4 = 0$ $\checkmark$
    
    $\langle e_1, e_3 \rangle = 1 \cdot (-3) + 2 \cdot 2 + 1 \cdot (-1) = -3 + 4 - 1 = 0$ $\checkmark$
    
    $\langle e_2, e_3 \rangle = 2 \cdot (-3) + 1 \cdot 2 + (-4) \cdot (-1) = -6 + 2 + 4 = 0$ $\checkmark$
    
    La famille est orthogonale.
    
    \textbf{2.} Vérifions si elle est orthonormée :
    
    $\|e_1\|^2 = 1^2 + 2^2 + 1^2 = 6$, donc $\|e_1\| = \sqrt{6}$
    
    $\|e_2\|^2 = 2^2 + 1^2 + (-4)^2 = 4 + 1 + 16 = 21$, donc $\|e_2\| = \sqrt{21}$
    
    $\|e_3\|^2 = (-3)^2 + 2^2 + (-1)^2 = 9 + 4 + 1 = 14$, donc $\|e_3\| = \sqrt{14}$
    
    La famille n'est pas orthonormée. Une base orthonormée est :
    $$f_1 = \frac{e_1}{\|e_1\|} = \frac{1}{\sqrt{6}}(1,2,1) = \left(\frac{1}{\sqrt{6}}, \frac{2}{\sqrt{6}}, \frac{1}{\sqrt{6}}\right)$$
    
    $$f_2 = \frac{e_2}{\|e_2\|} = \frac{1}{\sqrt{21}}(2,1,-4) = \left(\frac{2}{\sqrt{21}}, \frac{1}{\sqrt{21}}, \frac{-4}{\sqrt{21}}\right)$$
    
    $$f_3 = \frac{e_3}{\|e_3\|} = \frac{1}{\sqrt{14}}(-3,2,-1) = \left(\frac{-3}{\sqrt{14}}, \frac{2}{\sqrt{14}}, \frac{-1}{\sqrt{14}}\right)$$
    
    \textbf{3.} Coordonnées de $u = (1,0,1)$ dans la base $(f_1,f_2,f_3)$ :
    
    $u_1 = \langle u, f_1 \rangle = 1 \cdot \frac{1}{\sqrt{6}} + 0 \cdot \frac{2}{\sqrt{6}} + 1 \cdot \frac{1}{\sqrt{6}} = \frac{2}{\sqrt{6}} = \frac{\sqrt{6}}{3}$
    
    $u_2 = \langle u, f_2 \rangle = 1 \cdot \frac{2}{\sqrt{21}} + 0 \cdot \frac{1}{\sqrt{21}} + 1 \cdot \frac{-4}{\sqrt{21}} = \frac{-2}{\sqrt{21}}$
    
    $u_3 = \langle u, f_3 \rangle = 1 \cdot \frac{-3}{\sqrt{14}} + 0 \cdot \frac{2}{\sqrt{14}} + 1 \cdot \frac{-1}{\sqrt{14}} = \frac{-4}{\sqrt{14}}$
    
    Donc $u = \frac{\sqrt{6}}{3}f_1 - \frac{2}{\sqrt{21}}f_2 - \frac{4}{\sqrt{14}}f_3$
\end{mdframed}
}{}

\vspace{1em}

\subsection{Dans $\mathbb{R}[X]$}

\begin{itemize}
    \item Quelle est la dimension de $\mathbb{R}[X]$ ?
\end{itemize}
La famille $(1,X,X^2,X^3, \cdots )$ est appelée base hilbertienne de $\mathbb{R}[X]$ : tout élément de $\mathbb{R}[X]$ peut s'écrire comme une combinaison linéaire finie de vecteurs de cette famille.

On munit cet espace du produit scalaire : 
$$ \langle P, Q \rangle = \int_{0}^{1} P(x) Q(x) dx $$

\begin{itemize}
    \item Montrer que c'est bien un produit scalaire en vérifiant les propriétés ci-dessus.
    \item La famille $(1,X,X^2,X^3, \cdots )$ est-elle orthogonale ? Est-elle orthonormée ?
    \item Comment trouver $a, b, c$ tels que la famille $(1, X-a, X^2-bX-c)$ soit orthogonale ?
    \item Quelles sont les coordonnées de $P = 1+2X+3X^2$ dans la base $(1, X, X^2, \cdots)$ ?
    \item Peut-on retrouver ces coordonnées avec le produit scalaire comme dans l'exercice précédent ?
\end{itemize}
\vspace{1em}

\section*{Produit scalaire complexe}
\subsection{Définition}
On appelle produit scalaire sur un espace vectoriel $E$ une application 
$$\langle \cdot, \cdot \rangle : E \times E \to \mathbb{R}$$
telle que :
\begin{multicols}{2}
\begin{itemize}
    \item[*] symétrie conjuguée : $\langle u, v \rangle = \overline{\langle v, u \rangle}$
    \item[*] linéarité à gauche : $\langle \lambda u + v, w \rangle = \lambda \langle u, w \rangle + \langle v, w \rangle$
    \item[*] positivité : $\langle u, u \rangle \geq 0$
    \item[*] définie positivité : $\langle u, u \rangle = 0 \iff u = 0$
\end{itemize}
\end{multicols}


\subsection{Dans l'espace des fonctions complexes $2\pi$-périodiques}

On définit le produit scalaire :
$$
\langle f, g \rangle = \int_{0}^{2\pi} f(x) \overline{g(x)} dx
$$

Montrer que c'est un produit scalaire.

Montrer que la famille $(e^{inx})_{n \in \mathbb{Z}}$ est orthonormée.

Les coefficients de Fourier d'une fonction $f$ sont les coordonnées de $f$ dans la base $(e^{inx})_{n \in \mathbb{Z}}$.

Déterminer les coefficients de Fourier des fonctions suivantes :

\ifthenelse{\boolean{showSolutions}}{
    \vspace{2em}
    \begin{mdframed}
    \textbf{Démonstration que c'est un produit scalaire :}
    
    Les propriétés de symétrie conjuguée, linéarité et positivité découlent des propriétés de l'intégrale et du conjugué complexe.
    
    \textbf{Démonstration que $(e^{inx})_{n \in \mathbb{Z}}$ est orthonormée :}
    
    Pour $n = m$ : $\langle e^{inx}, e^{inx} \rangle = \int_0^{2\pi} e^{inx} \overline{e^{inx}} dx = \int_0^{2\pi} 1 dx = 2\pi$
    
    Pour $n \neq m$ : $\langle e^{inx}, e^{imx} \rangle = \int_0^{2\pi} e^{inx} \overline{e^{imx}} dx = \int_0^{2\pi} e^{i(n-m)x} dx = 0$
    
    Donc la famille $(e^{inx})_{n \in \mathbb{Z}}$ est orthogonale. Pour l'orthonormaliser, on divise par $\sqrt{2\pi}$.
    
    \textbf{Coefficients de Fourier :}
    \begin{enumerate}
    \item $\cos(x) = \frac{e^{ix} + e^{-ix}}{2}$, donc $c_1 = \frac{1}{2}$, $c_{-1} = \frac{1}{2}$, $c_n = 0$ sinon.
    
    \item $\sin(2\pi x)$ : Cette fonction n'est pas $2\pi$-périodique ! Elle est de période $1$.
    
    \item $\cos(x/2) = \frac{e^{ix/2} + e^{-ix/2}}{2}$, donc $c_{1/2} = \frac{1}{2}$, $c_{-1/2} = \frac{1}{2}$, $c_n = 0$ sinon.
    
    \item $\sin(2x) + \cos(3x) = \frac{e^{i2x} - e^{-i2x}}{2i} + \frac{e^{i3x} + e^{-i3x}}{2}$
    Donc $c_2 = \frac{1}{2i}$, $c_{-2} = -\frac{1}{2i}$, $c_3 = \frac{1}{2}$, $c_{-3} = \frac{1}{2}$, $c_n = 0$ sinon.
    
    \item $f(x) = e^{-x}$ sur $[0, 2\pi]$ :
    $$c_n = \frac{1}{2\pi}\int_0^{2\pi} e^{-x} e^{-inx} dx = \frac{1}{2\pi}\int_0^{2\pi} e^{-(1+in)x} dx$$
    $$= \frac{1}{2\pi}\left[\frac{e^{-(1+in)x}}{-(1+in)}\right]_0^{2\pi} = \frac{1}{2\pi} \cdot \frac{1-e^{-2\pi(1+in)}}{1+in} = \frac{1-e^{-2\pi}e^{-2\pi in}}{2\pi(1+in)}$$
\end{enumerate}
\end{mdframed}
}{}
\ifthenelse{\boolean{showSolutions}}{}
{\begin{multicols}{2}}
\begin{enumerate}
\item $\displaystyle \cos(x)$
\item $\displaystyle \sin(2\pi x)$
\item $\displaystyle \cos(x/2)$
\item $\displaystyle \sin(2x) + \cos(3x)$
\item $\displaystyle \exp^{-x}$ sur l'intervalle $[0, 2\pi]$
\end{enumerate}
\ifthenelse{\boolean{showSolutions}}{}{
\end{multicols}
}


% \setcounter{chapter}{2}
% \chapter{Familles de vecteurs}
% \section*{Les fonctions périodiques}

\subsection{Les fonctions complexes périodiques}

Les fonctions réelles suivantes sont-elles périodiques et si oui, quelle est leur période ?

\ifthenelse{\boolean{showSolutions}}{
    \vspace{2em}
    \begin{mdframed}
    \begin{enumerate}
    \item $\cos(x)$ : Oui, période $T = 2\pi$
    \item $\sin(2\pi x)$ : Oui, période $T = 1$ (car $\sin(2\pi(x+1)) = \sin(2\pi x + 2\pi) = \sin(2\pi x)$)
    \item $\cos(x/2)$ : Oui, période $T = 4\pi$ (car $\cos((x+4\pi)/2) = \cos(x/2 + 2\pi) = \cos(x/2)$)
    \item $\sin(2x) + \cos(3x)$ : Oui, période $T = 2\pi$ (le PPCM des périodes $\pi$ et $\frac{2\pi}{3}$)
    \item $\sin(nx)$ : Oui, période $T = \frac{2\pi}{n}$
    \item $\cos\left(\frac{3x}{2}-\frac{\pi}{4}\right)$ : Oui, période $T = \frac{4\pi}{3}$
    \item $x-\lfloor x\rfloor$ : Oui, période $T = 1$ (c'est la fonction partie fractionnaire)
\end{enumerate}
\end{mdframed}
}{}
\ifthenelse{\boolean{showSolutions}}{}
{\begin{multicols}{2}}
\begin{enumerate}
    \item $\displaystyle \cos(x)$
    
    \item $\displaystyle \sin(2\pi x)$
    \item $\displaystyle \cos(x/2)$
    \item $\displaystyle \sin(2x) + \cos(3x)$
    \item $\displaystyle \sin(nx)$, $n$ est un entier naturel non nul
    \item $\displaystyle \cos \left(\frac{3 x}{2}-\frac{\pi}{4}\right)$
    \item $\displaystyle x-\lfloor x\rfloor$
\end{enumerate}
\ifthenelse{\boolean{showSolutions}}{}{
\end{multicols}
}

Les fonctions complexes suivantes sont-elles périodiques et si oui, quelle est leur période ?

\ifthenelse{\boolean{showSolutions}}{
    \vspace{2em}
    \begin{mdframed}
    \begin{enumerate}
    \item $e^{ix}$ : Oui, période $T = 2\pi$ (car $e^{i(x+2\pi)} = e^{ix}e^{i2\pi} = e^{ix} \cdot 1 = e^{ix}$)
    \item $e^{2ix}$ : Oui, période $T = \pi$ (car $e^{2i(x+\pi)} = e^{2ix}e^{i2\pi} = e^{2ix}$)
    \item $e^{ix/2\pi}$ : Oui, période $T = 4\pi^2$ (car $e^{i(x+4\pi^2)/2\pi} = e^{ix/2\pi}e^{i2\pi} = e^{ix/2\pi}$)
    \item $e^{2i\pi x/T}$ : Oui, période $T$ (car $e^{2i\pi (x+T)/T} = e^{2i\pi x/T}e^{i2\pi} = e^{2i\pi x/T}$)
    \item $e^{inx} + e^{ipx}$ : Oui, période $T = \frac{2\pi}{\text{PGCD}(n,p)}$ (le PPCM des périodes $\frac{2\pi}{n}$ et $\frac{2\pi}{p}$)
\end{enumerate}
\end{mdframed}
}{}
\ifthenelse{\boolean{showSolutions}}{}
{\begin{multicols}{2}}
\begin{enumerate}
    \item $\displaystyle e^{ix}$
    \item $\displaystyle e^{2ix}$
    \item $\displaystyle e^{ix/2\pi}$
    \item $\displaystyle e^{2i\pi x/T}$, $T$ est un réel strictement positif
    \item $\displaystyle e^{inx} + e^{ipx}$
\end{enumerate}
\ifthenelse{\boolean{showSolutions}}{}{
\end{multicols}
}


\section*{Produit scalaire réel}

\subsection{Définition}
On appelle produit scalaire sur un espace vectoriel $E$ une application 
$$\langle \cdot, \cdot \rangle : E \times E \to \mathbb{R}$$
telle que :
\begin{multicols}{2}
\begin{itemize}
    \item[*] symétrie : $\langle u, v \rangle = \langle v, u \rangle$ 
    \item[*] linéarité à gauche : $\langle \lambda u + v, w \rangle = \lambda \langle u, w \rangle + \langle v, w \rangle$
    \item[*] positivité : $\langle u, u \rangle \geq 0$
    \item[*] définie positivité : $\langle u, u \rangle = 0 \iff u = 0$
\end{itemize}
\end{multicols}


\vspace{1em}

\subsection{Dans $\mathbb{R}^3$}

On se place dans $\mathbb{R}^3$, qu'on munit de la base 
$$e_1 = (1,2,1), \qquad e_2 = (2,1,-4), \qquad e_3 = (-3,2,-1)$$

\begin{enumerate}
\item La famille est-elle orthogonale ? 
\item Est-elle orthonormée ? Si non, définissez une base $(f_1,f_2,f_3)$ orthonormée à partir de la famille $(e_1,e_2,e_3)$. 
\end{enumerate}

Soit $u$ un vecteur de $\mathbb{R}^3$, on note $u_i$ ses coordonnées dans la base orthonormée $(f_1,f_2,f_3)$. Cela signifie que 
$$u = u_1 f_1 + u_2 f_2 + u_3 f_3$$

Déterminer les coordonnées de $u = (1,0,1)$ dans la base $(f_1,f_2,f_3)$.

\ifthenelse{\boolean{showSolutions}}{
    \vspace{2em}
    \begin{mdframed}
    \textbf{1.} Vérifions si la famille est orthogonale :
    
    $\langle e_1, e_2 \rangle = 1 \cdot 2 + 2 \cdot 1 + 1 \cdot (-4) = 2 + 2 - 4 = 0$ $\checkmark$
    
    $\langle e_1, e_3 \rangle = 1 \cdot (-3) + 2 \cdot 2 + 1 \cdot (-1) = -3 + 4 - 1 = 0$ $\checkmark$
    
    $\langle e_2, e_3 \rangle = 2 \cdot (-3) + 1 \cdot 2 + (-4) \cdot (-1) = -6 + 2 + 4 = 0$ $\checkmark$
    
    La famille est orthogonale.
    
    \textbf{2.} Vérifions si elle est orthonormée :
    
    $\|e_1\|^2 = 1^2 + 2^2 + 1^2 = 6$, donc $\|e_1\| = \sqrt{6}$
    
    $\|e_2\|^2 = 2^2 + 1^2 + (-4)^2 = 4 + 1 + 16 = 21$, donc $\|e_2\| = \sqrt{21}$
    
    $\|e_3\|^2 = (-3)^2 + 2^2 + (-1)^2 = 9 + 4 + 1 = 14$, donc $\|e_3\| = \sqrt{14}$
    
    La famille n'est pas orthonormée. Une base orthonormée est :
    $$f_1 = \frac{e_1}{\|e_1\|} = \frac{1}{\sqrt{6}}(1,2,1) = \left(\frac{1}{\sqrt{6}}, \frac{2}{\sqrt{6}}, \frac{1}{\sqrt{6}}\right)$$
    
    $$f_2 = \frac{e_2}{\|e_2\|} = \frac{1}{\sqrt{21}}(2,1,-4) = \left(\frac{2}{\sqrt{21}}, \frac{1}{\sqrt{21}}, \frac{-4}{\sqrt{21}}\right)$$
    
    $$f_3 = \frac{e_3}{\|e_3\|} = \frac{1}{\sqrt{14}}(-3,2,-1) = \left(\frac{-3}{\sqrt{14}}, \frac{2}{\sqrt{14}}, \frac{-1}{\sqrt{14}}\right)$$
    
    \textbf{3.} Coordonnées de $u = (1,0,1)$ dans la base $(f_1,f_2,f_3)$ :
    
    $u_1 = \langle u, f_1 \rangle = 1 \cdot \frac{1}{\sqrt{6}} + 0 \cdot \frac{2}{\sqrt{6}} + 1 \cdot \frac{1}{\sqrt{6}} = \frac{2}{\sqrt{6}} = \frac{\sqrt{6}}{3}$
    
    $u_2 = \langle u, f_2 \rangle = 1 \cdot \frac{2}{\sqrt{21}} + 0 \cdot \frac{1}{\sqrt{21}} + 1 \cdot \frac{-4}{\sqrt{21}} = \frac{-2}{\sqrt{21}}$
    
    $u_3 = \langle u, f_3 \rangle = 1 \cdot \frac{-3}{\sqrt{14}} + 0 \cdot \frac{2}{\sqrt{14}} + 1 \cdot \frac{-1}{\sqrt{14}} = \frac{-4}{\sqrt{14}}$
    
    Donc $u = \frac{\sqrt{6}}{3}f_1 - \frac{2}{\sqrt{21}}f_2 - \frac{4}{\sqrt{14}}f_3$
\end{mdframed}
}{}

\vspace{1em}

\subsection{Dans $\mathbb{R}[X]$}

\begin{itemize}
    \item Quelle est la dimension de $\mathbb{R}[X]$ ?
\end{itemize}
La famille $(1,X,X^2,X^3, \cdots )$ est appelée base hilbertienne de $\mathbb{R}[X]$ : tout élément de $\mathbb{R}[X]$ peut s'écrire comme une combinaison linéaire finie de vecteurs de cette famille.

On munit cet espace du produit scalaire : 
$$ \langle P, Q \rangle = \int_{0}^{1} P(x) Q(x) dx $$

\begin{itemize}
    \item Montrer que c'est bien un produit scalaire en vérifiant les propriétés ci-dessus.
    \item La famille $(1,X,X^2,X^3, \cdots )$ est-elle orthogonale ? Est-elle orthonormée ?
    \item Comment trouver $a, b, c$ tels que la famille $(1, X-a, X^2-bX-c)$ soit orthogonale ?
    \item Quelles sont les coordonnées de $P = 1+2X+3X^2$ dans la base $(1, X, X^2, \cdots)$ ?
    \item Peut-on retrouver ces coordonnées avec le produit scalaire comme dans l'exercice précédent ?
\end{itemize}
\vspace{1em}

\section*{Produit scalaire complexe}
\subsection{Définition}
On appelle produit scalaire sur un espace vectoriel $E$ une application 
$$\langle \cdot, \cdot \rangle : E \times E \to \mathbb{R}$$
telle que :
\begin{multicols}{2}
\begin{itemize}
    \item[*] symétrie conjuguée : $\langle u, v \rangle = \overline{\langle v, u \rangle}$
    \item[*] linéarité à gauche : $\langle \lambda u + v, w \rangle = \lambda \langle u, w \rangle + \langle v, w \rangle$
    \item[*] positivité : $\langle u, u \rangle \geq 0$
    \item[*] définie positivité : $\langle u, u \rangle = 0 \iff u = 0$
\end{itemize}
\end{multicols}


\subsection{Dans l'espace des fonctions complexes $2\pi$-périodiques}

On définit le produit scalaire :
$$
\langle f, g \rangle = \int_{0}^{2\pi} f(x) \overline{g(x)} dx
$$

Montrer que c'est un produit scalaire.

Montrer que la famille $(e^{inx})_{n \in \mathbb{Z}}$ est orthonormée.

Les coefficients de Fourier d'une fonction $f$ sont les coordonnées de $f$ dans la base $(e^{inx})_{n \in \mathbb{Z}}$.

Déterminer les coefficients de Fourier des fonctions suivantes :

\ifthenelse{\boolean{showSolutions}}{
    \vspace{2em}
    \begin{mdframed}
    \textbf{Démonstration que c'est un produit scalaire :}
    
    Les propriétés de symétrie conjuguée, linéarité et positivité découlent des propriétés de l'intégrale et du conjugué complexe.
    
    \textbf{Démonstration que $(e^{inx})_{n \in \mathbb{Z}}$ est orthonormée :}
    
    Pour $n = m$ : $\langle e^{inx}, e^{inx} \rangle = \int_0^{2\pi} e^{inx} \overline{e^{inx}} dx = \int_0^{2\pi} 1 dx = 2\pi$
    
    Pour $n \neq m$ : $\langle e^{inx}, e^{imx} \rangle = \int_0^{2\pi} e^{inx} \overline{e^{imx}} dx = \int_0^{2\pi} e^{i(n-m)x} dx = 0$
    
    Donc la famille $(e^{inx})_{n \in \mathbb{Z}}$ est orthogonale. Pour l'orthonormaliser, on divise par $\sqrt{2\pi}$.
    
    \textbf{Coefficients de Fourier :}
    \begin{enumerate}
    \item $\cos(x) = \frac{e^{ix} + e^{-ix}}{2}$, donc $c_1 = \frac{1}{2}$, $c_{-1} = \frac{1}{2}$, $c_n = 0$ sinon.
    
    \item $\sin(2\pi x)$ : Cette fonction n'est pas $2\pi$-périodique ! Elle est de période $1$.
    
    \item $\cos(x/2) = \frac{e^{ix/2} + e^{-ix/2}}{2}$, donc $c_{1/2} = \frac{1}{2}$, $c_{-1/2} = \frac{1}{2}$, $c_n = 0$ sinon.
    
    \item $\sin(2x) + \cos(3x) = \frac{e^{i2x} - e^{-i2x}}{2i} + \frac{e^{i3x} + e^{-i3x}}{2}$
    Donc $c_2 = \frac{1}{2i}$, $c_{-2} = -\frac{1}{2i}$, $c_3 = \frac{1}{2}$, $c_{-3} = \frac{1}{2}$, $c_n = 0$ sinon.
    
    \item $f(x) = e^{-x}$ sur $[0, 2\pi]$ :
    $$c_n = \frac{1}{2\pi}\int_0^{2\pi} e^{-x} e^{-inx} dx = \frac{1}{2\pi}\int_0^{2\pi} e^{-(1+in)x} dx$$
    $$= \frac{1}{2\pi}\left[\frac{e^{-(1+in)x}}{-(1+in)}\right]_0^{2\pi} = \frac{1}{2\pi} \cdot \frac{1-e^{-2\pi(1+in)}}{1+in} = \frac{1-e^{-2\pi}e^{-2\pi in}}{2\pi(1+in)}$$
\end{enumerate}
\end{mdframed}
}{}
\ifthenelse{\boolean{showSolutions}}{}
{\begin{multicols}{2}}
\begin{enumerate}
\item $\displaystyle \cos(x)$
\item $\displaystyle \sin(2\pi x)$
\item $\displaystyle \cos(x/2)$
\item $\displaystyle \sin(2x) + \cos(3x)$
\item $\displaystyle \exp^{-x}$ sur l'intervalle $[0, 2\pi]$
\end{enumerate}
\ifthenelse{\boolean{showSolutions}}{}{
\end{multicols}
}


% \chapter{Controle continu - Espaces vectoriels}
% Quelles sont les $3$ questions qu'on doit se poser pour montrer qu'un ensemble est un espace vectoriel ?

\vspace{2em}

Les ensembles suivants sont-ils des espaces vectoriels inclus dans $\mathbb{R}^n$ ? 
$$ \mathrm{F}=\left\{\left(x_1, \ldots, x_{\mathrm{n}}\right) \in \mathbb{R}^{\mathrm{n}} / x_1+\ldots+x_{\mathrm{n}}=0\right\} \qquad 
    \mathrm{F}=\left\{\left(x_1, \ldots, x_{\mathrm{n}}\right) \in \mathbb{R}^{\mathrm{n}} / x_1 \times x_2=0\right\}
    $$

% \setcounter{chapter}{2}
% \chapter{Mise au point sur la rédaction}
% Les ensembles suivants sont-ils des espaces vectoriels inclus dans $\mathbb{R}^n$ ? 


\textbf{1.}
$$ \mathrm{F}=\left\{\left(x_1, x_2\right) \in \mathbb{R}^2 / x_1-x_2=0\right\}$$

\begin{itemize}[label=$\bullet$, itemsep=1em]
    \item Le vecteur nul est-il dans $\mathrm{F}$ ? 

    Le vecteur nul est le vecteur \dots\dots\dots 

    Ses coordonnées vérifient \dots\dots\dots
    donc \dots\dots\dots 

\item L'addition est-elle une loi interne dans $\mathrm{F}$ ? 

    Prenons deux vecteurs quelconques de $\mathrm{F}$ : \dots\dots\dots \newline
    Comme ces vecteurs sont dans $\mathrm{F}$, leurs coordonnées vérifient \dotfill

    Leur somme est le vecteur \dots\dots\dots 

    Ses coordonnées sont \dots\dots\dots et vérifient \dotfill

    Donc \dots\dots\dots 

\item La multiplication par un scalaire est-elle une loi interne dans $\mathrm{F}$ ? 

    Prenons un vecteur quelconque de $\mathrm{F}$ et un scalaire réel quelconque : \ldots\ldots\ldots 

    Le produit du scalaire par le vecteur est \dots\dots\dots 

    Ses coordonnées sont \dots\dots\dots et vérifient \dots\dots\dots 
    
    Donc \dots\dots\dots 
\end{itemize}

Finalement \dots\dots\dots 

\vspace{1em}

\newpage 

\textbf{2.}

$$    \mathrm{F}=\left\{\left(x_1, x_2\right) \in \mathbb{R}^2 / x_1 \, x_2=0\right\}
    $$

    \begin{itemize}[label=$\bullet$, itemsep=1em]
        \item Le vecteur nul est-il dans $\mathrm{F}$ ? 

    Le vecteur nul est le vecteur \dots\dots\dots 

    Ses coordonnées vérifient \dots\dots\dots 

    Donc \dots\dots\dots 

    \item L'addition est-elle une loi interne dans $\mathrm{F}$ ? 

    Prenons deux vecteurs quelconques de $\mathrm{F}$ : \dots\dots\dots 

    Leur somme est le vecteur \dots\dots\dots 

    Ses coordonnées vérifient : \dots\dots\dots 

    Or, \dots\dots\dots 

    Donc, en général, \dots\dots\dots 
\end{itemize}

\textbf{3.}

$$
\mathrm{F} = \left\{ (x_1, x_2, x_3) \in \mathbb{R}^3 \mid x_1 + 2x_2 - x_3 = 0 \right\}
$$

\begin{itemize}[label=$\bullet$, itemsep=1em]
    \item Le vecteur nul est-il dans $\mathrm{F}$ ?

    Le vecteur nul est le vecteur \dotfill

    Ses coordonnées vérifient \dotfill

    Donc \dotfill

    \item L'addition est-elle une loi interne dans $\mathrm{F}$ ?

    Prenons deux vecteurs quelconques de $\mathrm{F}$ : \dotfill

    Leur somme est le vecteur \dotfill

    Vérifions : \dotfill

    Donc \dotfill

    \item La multiplication par un scalaire est-elle une loi interne dans $\mathrm{F}$ ?

    Prenons un vecteur quelconque de $\mathrm{F}$ et un scalaire réel quelconque : \dotfill

    Le produit du scalaire par le vecteur est \dotfill

    Vérifions : \dotfill

    Donc \dotfill
\end{itemize}

Finalement \dotfill

\vspace{1em}




\setcounter{chapter}{3}
\chapter{Applications linéaires}
% \section*{Les fonctions périodiques}

\subsection{Les fonctions complexes périodiques}

Les fonctions réelles suivantes sont-elles périodiques et si oui, quelle est leur période ?

\ifthenelse{\boolean{showSolutions}}{
    \vspace{2em}
    \begin{mdframed}
    \begin{enumerate}
    \item $\cos(x)$ : Oui, période $T = 2\pi$
    \item $\sin(2\pi x)$ : Oui, période $T = 1$ (car $\sin(2\pi(x+1)) = \sin(2\pi x + 2\pi) = \sin(2\pi x)$)
    \item $\cos(x/2)$ : Oui, période $T = 4\pi$ (car $\cos((x+4\pi)/2) = \cos(x/2 + 2\pi) = \cos(x/2)$)
    \item $\sin(2x) + \cos(3x)$ : Oui, période $T = 2\pi$ (le PPCM des périodes $\pi$ et $\frac{2\pi}{3}$)
    \item $\sin(nx)$ : Oui, période $T = \frac{2\pi}{n}$
    \item $\cos\left(\frac{3x}{2}-\frac{\pi}{4}\right)$ : Oui, période $T = \frac{4\pi}{3}$
    \item $x-\lfloor x\rfloor$ : Oui, période $T = 1$ (c'est la fonction partie fractionnaire)
\end{enumerate}
\end{mdframed}
}{}
\ifthenelse{\boolean{showSolutions}}{}
{\begin{multicols}{2}}
\begin{enumerate}
    \item $\displaystyle \cos(x)$
    
    \item $\displaystyle \sin(2\pi x)$
    \item $\displaystyle \cos(x/2)$
    \item $\displaystyle \sin(2x) + \cos(3x)$
    \item $\displaystyle \sin(nx)$, $n$ est un entier naturel non nul
    \item $\displaystyle \cos \left(\frac{3 x}{2}-\frac{\pi}{4}\right)$
    \item $\displaystyle x-\lfloor x\rfloor$
\end{enumerate}
\ifthenelse{\boolean{showSolutions}}{}{
\end{multicols}
}

Les fonctions complexes suivantes sont-elles périodiques et si oui, quelle est leur période ?

\ifthenelse{\boolean{showSolutions}}{
    \vspace{2em}
    \begin{mdframed}
    \begin{enumerate}
    \item $e^{ix}$ : Oui, période $T = 2\pi$ (car $e^{i(x+2\pi)} = e^{ix}e^{i2\pi} = e^{ix} \cdot 1 = e^{ix}$)
    \item $e^{2ix}$ : Oui, période $T = \pi$ (car $e^{2i(x+\pi)} = e^{2ix}e^{i2\pi} = e^{2ix}$)
    \item $e^{ix/2\pi}$ : Oui, période $T = 4\pi^2$ (car $e^{i(x+4\pi^2)/2\pi} = e^{ix/2\pi}e^{i2\pi} = e^{ix/2\pi}$)
    \item $e^{2i\pi x/T}$ : Oui, période $T$ (car $e^{2i\pi (x+T)/T} = e^{2i\pi x/T}e^{i2\pi} = e^{2i\pi x/T}$)
    \item $e^{inx} + e^{ipx}$ : Oui, période $T = \frac{2\pi}{\text{PGCD}(n,p)}$ (le PPCM des périodes $\frac{2\pi}{n}$ et $\frac{2\pi}{p}$)
\end{enumerate}
\end{mdframed}
}{}
\ifthenelse{\boolean{showSolutions}}{}
{\begin{multicols}{2}}
\begin{enumerate}
    \item $\displaystyle e^{ix}$
    \item $\displaystyle e^{2ix}$
    \item $\displaystyle e^{ix/2\pi}$
    \item $\displaystyle e^{2i\pi x/T}$, $T$ est un réel strictement positif
    \item $\displaystyle e^{inx} + e^{ipx}$
\end{enumerate}
\ifthenelse{\boolean{showSolutions}}{}{
\end{multicols}
}


\section*{Produit scalaire réel}

\subsection{Définition}
On appelle produit scalaire sur un espace vectoriel $E$ une application 
$$\langle \cdot, \cdot \rangle : E \times E \to \mathbb{R}$$
telle que :
\begin{multicols}{2}
\begin{itemize}
    \item[*] symétrie : $\langle u, v \rangle = \langle v, u \rangle$ 
    \item[*] linéarité à gauche : $\langle \lambda u + v, w \rangle = \lambda \langle u, w \rangle + \langle v, w \rangle$
    \item[*] positivité : $\langle u, u \rangle \geq 0$
    \item[*] définie positivité : $\langle u, u \rangle = 0 \iff u = 0$
\end{itemize}
\end{multicols}


\vspace{1em}

\subsection{Dans $\mathbb{R}^3$}

On se place dans $\mathbb{R}^3$, qu'on munit de la base 
$$e_1 = (1,2,1), \qquad e_2 = (2,1,-4), \qquad e_3 = (-3,2,-1)$$

\begin{enumerate}
\item La famille est-elle orthogonale ? 
\item Est-elle orthonormée ? Si non, définissez une base $(f_1,f_2,f_3)$ orthonormée à partir de la famille $(e_1,e_2,e_3)$. 
\end{enumerate}

Soit $u$ un vecteur de $\mathbb{R}^3$, on note $u_i$ ses coordonnées dans la base orthonormée $(f_1,f_2,f_3)$. Cela signifie que 
$$u = u_1 f_1 + u_2 f_2 + u_3 f_3$$

Déterminer les coordonnées de $u = (1,0,1)$ dans la base $(f_1,f_2,f_3)$.

\ifthenelse{\boolean{showSolutions}}{
    \vspace{2em}
    \begin{mdframed}
    \textbf{1.} Vérifions si la famille est orthogonale :
    
    $\langle e_1, e_2 \rangle = 1 \cdot 2 + 2 \cdot 1 + 1 \cdot (-4) = 2 + 2 - 4 = 0$ $\checkmark$
    
    $\langle e_1, e_3 \rangle = 1 \cdot (-3) + 2 \cdot 2 + 1 \cdot (-1) = -3 + 4 - 1 = 0$ $\checkmark$
    
    $\langle e_2, e_3 \rangle = 2 \cdot (-3) + 1 \cdot 2 + (-4) \cdot (-1) = -6 + 2 + 4 = 0$ $\checkmark$
    
    La famille est orthogonale.
    
    \textbf{2.} Vérifions si elle est orthonormée :
    
    $\|e_1\|^2 = 1^2 + 2^2 + 1^2 = 6$, donc $\|e_1\| = \sqrt{6}$
    
    $\|e_2\|^2 = 2^2 + 1^2 + (-4)^2 = 4 + 1 + 16 = 21$, donc $\|e_2\| = \sqrt{21}$
    
    $\|e_3\|^2 = (-3)^2 + 2^2 + (-1)^2 = 9 + 4 + 1 = 14$, donc $\|e_3\| = \sqrt{14}$
    
    La famille n'est pas orthonormée. Une base orthonormée est :
    $$f_1 = \frac{e_1}{\|e_1\|} = \frac{1}{\sqrt{6}}(1,2,1) = \left(\frac{1}{\sqrt{6}}, \frac{2}{\sqrt{6}}, \frac{1}{\sqrt{6}}\right)$$
    
    $$f_2 = \frac{e_2}{\|e_2\|} = \frac{1}{\sqrt{21}}(2,1,-4) = \left(\frac{2}{\sqrt{21}}, \frac{1}{\sqrt{21}}, \frac{-4}{\sqrt{21}}\right)$$
    
    $$f_3 = \frac{e_3}{\|e_3\|} = \frac{1}{\sqrt{14}}(-3,2,-1) = \left(\frac{-3}{\sqrt{14}}, \frac{2}{\sqrt{14}}, \frac{-1}{\sqrt{14}}\right)$$
    
    \textbf{3.} Coordonnées de $u = (1,0,1)$ dans la base $(f_1,f_2,f_3)$ :
    
    $u_1 = \langle u, f_1 \rangle = 1 \cdot \frac{1}{\sqrt{6}} + 0 \cdot \frac{2}{\sqrt{6}} + 1 \cdot \frac{1}{\sqrt{6}} = \frac{2}{\sqrt{6}} = \frac{\sqrt{6}}{3}$
    
    $u_2 = \langle u, f_2 \rangle = 1 \cdot \frac{2}{\sqrt{21}} + 0 \cdot \frac{1}{\sqrt{21}} + 1 \cdot \frac{-4}{\sqrt{21}} = \frac{-2}{\sqrt{21}}$
    
    $u_3 = \langle u, f_3 \rangle = 1 \cdot \frac{-3}{\sqrt{14}} + 0 \cdot \frac{2}{\sqrt{14}} + 1 \cdot \frac{-1}{\sqrt{14}} = \frac{-4}{\sqrt{14}}$
    
    Donc $u = \frac{\sqrt{6}}{3}f_1 - \frac{2}{\sqrt{21}}f_2 - \frac{4}{\sqrt{14}}f_3$
\end{mdframed}
}{}

\vspace{1em}

\subsection{Dans $\mathbb{R}[X]$}

\begin{itemize}
    \item Quelle est la dimension de $\mathbb{R}[X]$ ?
\end{itemize}
La famille $(1,X,X^2,X^3, \cdots )$ est appelée base hilbertienne de $\mathbb{R}[X]$ : tout élément de $\mathbb{R}[X]$ peut s'écrire comme une combinaison linéaire finie de vecteurs de cette famille.

On munit cet espace du produit scalaire : 
$$ \langle P, Q \rangle = \int_{0}^{1} P(x) Q(x) dx $$

\begin{itemize}
    \item Montrer que c'est bien un produit scalaire en vérifiant les propriétés ci-dessus.
    \item La famille $(1,X,X^2,X^3, \cdots )$ est-elle orthogonale ? Est-elle orthonormée ?
    \item Comment trouver $a, b, c$ tels que la famille $(1, X-a, X^2-bX-c)$ soit orthogonale ?
    \item Quelles sont les coordonnées de $P = 1+2X+3X^2$ dans la base $(1, X, X^2, \cdots)$ ?
    \item Peut-on retrouver ces coordonnées avec le produit scalaire comme dans l'exercice précédent ?
\end{itemize}
\vspace{1em}

\section*{Produit scalaire complexe}
\subsection{Définition}
On appelle produit scalaire sur un espace vectoriel $E$ une application 
$$\langle \cdot, \cdot \rangle : E \times E \to \mathbb{R}$$
telle que :
\begin{multicols}{2}
\begin{itemize}
    \item[*] symétrie conjuguée : $\langle u, v \rangle = \overline{\langle v, u \rangle}$
    \item[*] linéarité à gauche : $\langle \lambda u + v, w \rangle = \lambda \langle u, w \rangle + \langle v, w \rangle$
    \item[*] positivité : $\langle u, u \rangle \geq 0$
    \item[*] définie positivité : $\langle u, u \rangle = 0 \iff u = 0$
\end{itemize}
\end{multicols}


\subsection{Dans l'espace des fonctions complexes $2\pi$-périodiques}

On définit le produit scalaire :
$$
\langle f, g \rangle = \int_{0}^{2\pi} f(x) \overline{g(x)} dx
$$

Montrer que c'est un produit scalaire.

Montrer que la famille $(e^{inx})_{n \in \mathbb{Z}}$ est orthonormée.

Les coefficients de Fourier d'une fonction $f$ sont les coordonnées de $f$ dans la base $(e^{inx})_{n \in \mathbb{Z}}$.

Déterminer les coefficients de Fourier des fonctions suivantes :

\ifthenelse{\boolean{showSolutions}}{
    \vspace{2em}
    \begin{mdframed}
    \textbf{Démonstration que c'est un produit scalaire :}
    
    Les propriétés de symétrie conjuguée, linéarité et positivité découlent des propriétés de l'intégrale et du conjugué complexe.
    
    \textbf{Démonstration que $(e^{inx})_{n \in \mathbb{Z}}$ est orthonormée :}
    
    Pour $n = m$ : $\langle e^{inx}, e^{inx} \rangle = \int_0^{2\pi} e^{inx} \overline{e^{inx}} dx = \int_0^{2\pi} 1 dx = 2\pi$
    
    Pour $n \neq m$ : $\langle e^{inx}, e^{imx} \rangle = \int_0^{2\pi} e^{inx} \overline{e^{imx}} dx = \int_0^{2\pi} e^{i(n-m)x} dx = 0$
    
    Donc la famille $(e^{inx})_{n \in \mathbb{Z}}$ est orthogonale. Pour l'orthonormaliser, on divise par $\sqrt{2\pi}$.
    
    \textbf{Coefficients de Fourier :}
    \begin{enumerate}
    \item $\cos(x) = \frac{e^{ix} + e^{-ix}}{2}$, donc $c_1 = \frac{1}{2}$, $c_{-1} = \frac{1}{2}$, $c_n = 0$ sinon.
    
    \item $\sin(2\pi x)$ : Cette fonction n'est pas $2\pi$-périodique ! Elle est de période $1$.
    
    \item $\cos(x/2) = \frac{e^{ix/2} + e^{-ix/2}}{2}$, donc $c_{1/2} = \frac{1}{2}$, $c_{-1/2} = \frac{1}{2}$, $c_n = 0$ sinon.
    
    \item $\sin(2x) + \cos(3x) = \frac{e^{i2x} - e^{-i2x}}{2i} + \frac{e^{i3x} + e^{-i3x}}{2}$
    Donc $c_2 = \frac{1}{2i}$, $c_{-2} = -\frac{1}{2i}$, $c_3 = \frac{1}{2}$, $c_{-3} = \frac{1}{2}$, $c_n = 0$ sinon.
    
    \item $f(x) = e^{-x}$ sur $[0, 2\pi]$ :
    $$c_n = \frac{1}{2\pi}\int_0^{2\pi} e^{-x} e^{-inx} dx = \frac{1}{2\pi}\int_0^{2\pi} e^{-(1+in)x} dx$$
    $$= \frac{1}{2\pi}\left[\frac{e^{-(1+in)x}}{-(1+in)}\right]_0^{2\pi} = \frac{1}{2\pi} \cdot \frac{1-e^{-2\pi(1+in)}}{1+in} = \frac{1-e^{-2\pi}e^{-2\pi in}}{2\pi(1+in)}$$
\end{enumerate}
\end{mdframed}
}{}
\ifthenelse{\boolean{showSolutions}}{}
{\begin{multicols}{2}}
\begin{enumerate}
\item $\displaystyle \cos(x)$
\item $\displaystyle \sin(2\pi x)$
\item $\displaystyle \cos(x/2)$
\item $\displaystyle \sin(2x) + \cos(3x)$
\item $\displaystyle \exp^{-x}$ sur l'intervalle $[0, 2\pi]$
\end{enumerate}
\ifthenelse{\boolean{showSolutions}}{}{
\end{multicols}
}


% \setcounter{section}{3}
% \section*{Les fonctions périodiques}

\subsection{Les fonctions complexes périodiques}

Les fonctions réelles suivantes sont-elles périodiques et si oui, quelle est leur période ?

\ifthenelse{\boolean{showSolutions}}{
    \vspace{2em}
    \begin{mdframed}
    \begin{enumerate}
    \item $\cos(x)$ : Oui, période $T = 2\pi$
    \item $\sin(2\pi x)$ : Oui, période $T = 1$ (car $\sin(2\pi(x+1)) = \sin(2\pi x + 2\pi) = \sin(2\pi x)$)
    \item $\cos(x/2)$ : Oui, période $T = 4\pi$ (car $\cos((x+4\pi)/2) = \cos(x/2 + 2\pi) = \cos(x/2)$)
    \item $\sin(2x) + \cos(3x)$ : Oui, période $T = 2\pi$ (le PPCM des périodes $\pi$ et $\frac{2\pi}{3}$)
    \item $\sin(nx)$ : Oui, période $T = \frac{2\pi}{n}$
    \item $\cos\left(\frac{3x}{2}-\frac{\pi}{4}\right)$ : Oui, période $T = \frac{4\pi}{3}$
    \item $x-\lfloor x\rfloor$ : Oui, période $T = 1$ (c'est la fonction partie fractionnaire)
\end{enumerate}
\end{mdframed}
}{}
\ifthenelse{\boolean{showSolutions}}{}
{\begin{multicols}{2}}
\begin{enumerate}
    \item $\displaystyle \cos(x)$
    
    \item $\displaystyle \sin(2\pi x)$
    \item $\displaystyle \cos(x/2)$
    \item $\displaystyle \sin(2x) + \cos(3x)$
    \item $\displaystyle \sin(nx)$, $n$ est un entier naturel non nul
    \item $\displaystyle \cos \left(\frac{3 x}{2}-\frac{\pi}{4}\right)$
    \item $\displaystyle x-\lfloor x\rfloor$
\end{enumerate}
\ifthenelse{\boolean{showSolutions}}{}{
\end{multicols}
}

Les fonctions complexes suivantes sont-elles périodiques et si oui, quelle est leur période ?

\ifthenelse{\boolean{showSolutions}}{
    \vspace{2em}
    \begin{mdframed}
    \begin{enumerate}
    \item $e^{ix}$ : Oui, période $T = 2\pi$ (car $e^{i(x+2\pi)} = e^{ix}e^{i2\pi} = e^{ix} \cdot 1 = e^{ix}$)
    \item $e^{2ix}$ : Oui, période $T = \pi$ (car $e^{2i(x+\pi)} = e^{2ix}e^{i2\pi} = e^{2ix}$)
    \item $e^{ix/2\pi}$ : Oui, période $T = 4\pi^2$ (car $e^{i(x+4\pi^2)/2\pi} = e^{ix/2\pi}e^{i2\pi} = e^{ix/2\pi}$)
    \item $e^{2i\pi x/T}$ : Oui, période $T$ (car $e^{2i\pi (x+T)/T} = e^{2i\pi x/T}e^{i2\pi} = e^{2i\pi x/T}$)
    \item $e^{inx} + e^{ipx}$ : Oui, période $T = \frac{2\pi}{\text{PGCD}(n,p)}$ (le PPCM des périodes $\frac{2\pi}{n}$ et $\frac{2\pi}{p}$)
\end{enumerate}
\end{mdframed}
}{}
\ifthenelse{\boolean{showSolutions}}{}
{\begin{multicols}{2}}
\begin{enumerate}
    \item $\displaystyle e^{ix}$
    \item $\displaystyle e^{2ix}$
    \item $\displaystyle e^{ix/2\pi}$
    \item $\displaystyle e^{2i\pi x/T}$, $T$ est un réel strictement positif
    \item $\displaystyle e^{inx} + e^{ipx}$
\end{enumerate}
\ifthenelse{\boolean{showSolutions}}{}{
\end{multicols}
}


\section*{Produit scalaire réel}

\subsection{Définition}
On appelle produit scalaire sur un espace vectoriel $E$ une application 
$$\langle \cdot, \cdot \rangle : E \times E \to \mathbb{R}$$
telle que :
\begin{multicols}{2}
\begin{itemize}
    \item[*] symétrie : $\langle u, v \rangle = \langle v, u \rangle$ 
    \item[*] linéarité à gauche : $\langle \lambda u + v, w \rangle = \lambda \langle u, w \rangle + \langle v, w \rangle$
    \item[*] positivité : $\langle u, u \rangle \geq 0$
    \item[*] définie positivité : $\langle u, u \rangle = 0 \iff u = 0$
\end{itemize}
\end{multicols}


\vspace{1em}

\subsection{Dans $\mathbb{R}^3$}

On se place dans $\mathbb{R}^3$, qu'on munit de la base 
$$e_1 = (1,2,1), \qquad e_2 = (2,1,-4), \qquad e_3 = (-3,2,-1)$$

\begin{enumerate}
\item La famille est-elle orthogonale ? 
\item Est-elle orthonormée ? Si non, définissez une base $(f_1,f_2,f_3)$ orthonormée à partir de la famille $(e_1,e_2,e_3)$. 
\end{enumerate}

Soit $u$ un vecteur de $\mathbb{R}^3$, on note $u_i$ ses coordonnées dans la base orthonormée $(f_1,f_2,f_3)$. Cela signifie que 
$$u = u_1 f_1 + u_2 f_2 + u_3 f_3$$

Déterminer les coordonnées de $u = (1,0,1)$ dans la base $(f_1,f_2,f_3)$.

\ifthenelse{\boolean{showSolutions}}{
    \vspace{2em}
    \begin{mdframed}
    \textbf{1.} Vérifions si la famille est orthogonale :
    
    $\langle e_1, e_2 \rangle = 1 \cdot 2 + 2 \cdot 1 + 1 \cdot (-4) = 2 + 2 - 4 = 0$ $\checkmark$
    
    $\langle e_1, e_3 \rangle = 1 \cdot (-3) + 2 \cdot 2 + 1 \cdot (-1) = -3 + 4 - 1 = 0$ $\checkmark$
    
    $\langle e_2, e_3 \rangle = 2 \cdot (-3) + 1 \cdot 2 + (-4) \cdot (-1) = -6 + 2 + 4 = 0$ $\checkmark$
    
    La famille est orthogonale.
    
    \textbf{2.} Vérifions si elle est orthonormée :
    
    $\|e_1\|^2 = 1^2 + 2^2 + 1^2 = 6$, donc $\|e_1\| = \sqrt{6}$
    
    $\|e_2\|^2 = 2^2 + 1^2 + (-4)^2 = 4 + 1 + 16 = 21$, donc $\|e_2\| = \sqrt{21}$
    
    $\|e_3\|^2 = (-3)^2 + 2^2 + (-1)^2 = 9 + 4 + 1 = 14$, donc $\|e_3\| = \sqrt{14}$
    
    La famille n'est pas orthonormée. Une base orthonormée est :
    $$f_1 = \frac{e_1}{\|e_1\|} = \frac{1}{\sqrt{6}}(1,2,1) = \left(\frac{1}{\sqrt{6}}, \frac{2}{\sqrt{6}}, \frac{1}{\sqrt{6}}\right)$$
    
    $$f_2 = \frac{e_2}{\|e_2\|} = \frac{1}{\sqrt{21}}(2,1,-4) = \left(\frac{2}{\sqrt{21}}, \frac{1}{\sqrt{21}}, \frac{-4}{\sqrt{21}}\right)$$
    
    $$f_3 = \frac{e_3}{\|e_3\|} = \frac{1}{\sqrt{14}}(-3,2,-1) = \left(\frac{-3}{\sqrt{14}}, \frac{2}{\sqrt{14}}, \frac{-1}{\sqrt{14}}\right)$$
    
    \textbf{3.} Coordonnées de $u = (1,0,1)$ dans la base $(f_1,f_2,f_3)$ :
    
    $u_1 = \langle u, f_1 \rangle = 1 \cdot \frac{1}{\sqrt{6}} + 0 \cdot \frac{2}{\sqrt{6}} + 1 \cdot \frac{1}{\sqrt{6}} = \frac{2}{\sqrt{6}} = \frac{\sqrt{6}}{3}$
    
    $u_2 = \langle u, f_2 \rangle = 1 \cdot \frac{2}{\sqrt{21}} + 0 \cdot \frac{1}{\sqrt{21}} + 1 \cdot \frac{-4}{\sqrt{21}} = \frac{-2}{\sqrt{21}}$
    
    $u_3 = \langle u, f_3 \rangle = 1 \cdot \frac{-3}{\sqrt{14}} + 0 \cdot \frac{2}{\sqrt{14}} + 1 \cdot \frac{-1}{\sqrt{14}} = \frac{-4}{\sqrt{14}}$
    
    Donc $u = \frac{\sqrt{6}}{3}f_1 - \frac{2}{\sqrt{21}}f_2 - \frac{4}{\sqrt{14}}f_3$
\end{mdframed}
}{}

\vspace{1em}

\subsection{Dans $\mathbb{R}[X]$}

\begin{itemize}
    \item Quelle est la dimension de $\mathbb{R}[X]$ ?
\end{itemize}
La famille $(1,X,X^2,X^3, \cdots )$ est appelée base hilbertienne de $\mathbb{R}[X]$ : tout élément de $\mathbb{R}[X]$ peut s'écrire comme une combinaison linéaire finie de vecteurs de cette famille.

On munit cet espace du produit scalaire : 
$$ \langle P, Q \rangle = \int_{0}^{1} P(x) Q(x) dx $$

\begin{itemize}
    \item Montrer que c'est bien un produit scalaire en vérifiant les propriétés ci-dessus.
    \item La famille $(1,X,X^2,X^3, \cdots )$ est-elle orthogonale ? Est-elle orthonormée ?
    \item Comment trouver $a, b, c$ tels que la famille $(1, X-a, X^2-bX-c)$ soit orthogonale ?
    \item Quelles sont les coordonnées de $P = 1+2X+3X^2$ dans la base $(1, X, X^2, \cdots)$ ?
    \item Peut-on retrouver ces coordonnées avec le produit scalaire comme dans l'exercice précédent ?
\end{itemize}
\vspace{1em}

\section*{Produit scalaire complexe}
\subsection{Définition}
On appelle produit scalaire sur un espace vectoriel $E$ une application 
$$\langle \cdot, \cdot \rangle : E \times E \to \mathbb{R}$$
telle que :
\begin{multicols}{2}
\begin{itemize}
    \item[*] symétrie conjuguée : $\langle u, v \rangle = \overline{\langle v, u \rangle}$
    \item[*] linéarité à gauche : $\langle \lambda u + v, w \rangle = \lambda \langle u, w \rangle + \langle v, w \rangle$
    \item[*] positivité : $\langle u, u \rangle \geq 0$
    \item[*] définie positivité : $\langle u, u \rangle = 0 \iff u = 0$
\end{itemize}
\end{multicols}


\subsection{Dans l'espace des fonctions complexes $2\pi$-périodiques}

On définit le produit scalaire :
$$
\langle f, g \rangle = \int_{0}^{2\pi} f(x) \overline{g(x)} dx
$$

Montrer que c'est un produit scalaire.

Montrer que la famille $(e^{inx})_{n \in \mathbb{Z}}$ est orthonormée.

Les coefficients de Fourier d'une fonction $f$ sont les coordonnées de $f$ dans la base $(e^{inx})_{n \in \mathbb{Z}}$.

Déterminer les coefficients de Fourier des fonctions suivantes :

\ifthenelse{\boolean{showSolutions}}{
    \vspace{2em}
    \begin{mdframed}
    \textbf{Démonstration que c'est un produit scalaire :}
    
    Les propriétés de symétrie conjuguée, linéarité et positivité découlent des propriétés de l'intégrale et du conjugué complexe.
    
    \textbf{Démonstration que $(e^{inx})_{n \in \mathbb{Z}}$ est orthonormée :}
    
    Pour $n = m$ : $\langle e^{inx}, e^{inx} \rangle = \int_0^{2\pi} e^{inx} \overline{e^{inx}} dx = \int_0^{2\pi} 1 dx = 2\pi$
    
    Pour $n \neq m$ : $\langle e^{inx}, e^{imx} \rangle = \int_0^{2\pi} e^{inx} \overline{e^{imx}} dx = \int_0^{2\pi} e^{i(n-m)x} dx = 0$
    
    Donc la famille $(e^{inx})_{n \in \mathbb{Z}}$ est orthogonale. Pour l'orthonormaliser, on divise par $\sqrt{2\pi}$.
    
    \textbf{Coefficients de Fourier :}
    \begin{enumerate}
    \item $\cos(x) = \frac{e^{ix} + e^{-ix}}{2}$, donc $c_1 = \frac{1}{2}$, $c_{-1} = \frac{1}{2}$, $c_n = 0$ sinon.
    
    \item $\sin(2\pi x)$ : Cette fonction n'est pas $2\pi$-périodique ! Elle est de période $1$.
    
    \item $\cos(x/2) = \frac{e^{ix/2} + e^{-ix/2}}{2}$, donc $c_{1/2} = \frac{1}{2}$, $c_{-1/2} = \frac{1}{2}$, $c_n = 0$ sinon.
    
    \item $\sin(2x) + \cos(3x) = \frac{e^{i2x} - e^{-i2x}}{2i} + \frac{e^{i3x} + e^{-i3x}}{2}$
    Donc $c_2 = \frac{1}{2i}$, $c_{-2} = -\frac{1}{2i}$, $c_3 = \frac{1}{2}$, $c_{-3} = \frac{1}{2}$, $c_n = 0$ sinon.
    
    \item $f(x) = e^{-x}$ sur $[0, 2\pi]$ :
    $$c_n = \frac{1}{2\pi}\int_0^{2\pi} e^{-x} e^{-inx} dx = \frac{1}{2\pi}\int_0^{2\pi} e^{-(1+in)x} dx$$
    $$= \frac{1}{2\pi}\left[\frac{e^{-(1+in)x}}{-(1+in)}\right]_0^{2\pi} = \frac{1}{2\pi} \cdot \frac{1-e^{-2\pi(1+in)}}{1+in} = \frac{1-e^{-2\pi}e^{-2\pi in}}{2\pi(1+in)}$$
\end{enumerate}
\end{mdframed}
}{}
\ifthenelse{\boolean{showSolutions}}{}
{\begin{multicols}{2}}
\begin{enumerate}
\item $\displaystyle \cos(x)$
\item $\displaystyle \sin(2\pi x)$
\item $\displaystyle \cos(x/2)$
\item $\displaystyle \sin(2x) + \cos(3x)$
\item $\displaystyle \exp^{-x}$ sur l'intervalle $[0, 2\pi]$
\end{enumerate}
\ifthenelse{\boolean{showSolutions}}{}{
\end{multicols}
}


% \setcounter{section}{5}
% \section*{Les fonctions périodiques}

\subsection{Les fonctions complexes périodiques}

Les fonctions réelles suivantes sont-elles périodiques et si oui, quelle est leur période ?

\ifthenelse{\boolean{showSolutions}}{
    \vspace{2em}
    \begin{mdframed}
    \begin{enumerate}
    \item $\cos(x)$ : Oui, période $T = 2\pi$
    \item $\sin(2\pi x)$ : Oui, période $T = 1$ (car $\sin(2\pi(x+1)) = \sin(2\pi x + 2\pi) = \sin(2\pi x)$)
    \item $\cos(x/2)$ : Oui, période $T = 4\pi$ (car $\cos((x+4\pi)/2) = \cos(x/2 + 2\pi) = \cos(x/2)$)
    \item $\sin(2x) + \cos(3x)$ : Oui, période $T = 2\pi$ (le PPCM des périodes $\pi$ et $\frac{2\pi}{3}$)
    \item $\sin(nx)$ : Oui, période $T = \frac{2\pi}{n}$
    \item $\cos\left(\frac{3x}{2}-\frac{\pi}{4}\right)$ : Oui, période $T = \frac{4\pi}{3}$
    \item $x-\lfloor x\rfloor$ : Oui, période $T = 1$ (c'est la fonction partie fractionnaire)
\end{enumerate}
\end{mdframed}
}{}
\ifthenelse{\boolean{showSolutions}}{}
{\begin{multicols}{2}}
\begin{enumerate}
    \item $\displaystyle \cos(x)$
    
    \item $\displaystyle \sin(2\pi x)$
    \item $\displaystyle \cos(x/2)$
    \item $\displaystyle \sin(2x) + \cos(3x)$
    \item $\displaystyle \sin(nx)$, $n$ est un entier naturel non nul
    \item $\displaystyle \cos \left(\frac{3 x}{2}-\frac{\pi}{4}\right)$
    \item $\displaystyle x-\lfloor x\rfloor$
\end{enumerate}
\ifthenelse{\boolean{showSolutions}}{}{
\end{multicols}
}

Les fonctions complexes suivantes sont-elles périodiques et si oui, quelle est leur période ?

\ifthenelse{\boolean{showSolutions}}{
    \vspace{2em}
    \begin{mdframed}
    \begin{enumerate}
    \item $e^{ix}$ : Oui, période $T = 2\pi$ (car $e^{i(x+2\pi)} = e^{ix}e^{i2\pi} = e^{ix} \cdot 1 = e^{ix}$)
    \item $e^{2ix}$ : Oui, période $T = \pi$ (car $e^{2i(x+\pi)} = e^{2ix}e^{i2\pi} = e^{2ix}$)
    \item $e^{ix/2\pi}$ : Oui, période $T = 4\pi^2$ (car $e^{i(x+4\pi^2)/2\pi} = e^{ix/2\pi}e^{i2\pi} = e^{ix/2\pi}$)
    \item $e^{2i\pi x/T}$ : Oui, période $T$ (car $e^{2i\pi (x+T)/T} = e^{2i\pi x/T}e^{i2\pi} = e^{2i\pi x/T}$)
    \item $e^{inx} + e^{ipx}$ : Oui, période $T = \frac{2\pi}{\text{PGCD}(n,p)}$ (le PPCM des périodes $\frac{2\pi}{n}$ et $\frac{2\pi}{p}$)
\end{enumerate}
\end{mdframed}
}{}
\ifthenelse{\boolean{showSolutions}}{}
{\begin{multicols}{2}}
\begin{enumerate}
    \item $\displaystyle e^{ix}$
    \item $\displaystyle e^{2ix}$
    \item $\displaystyle e^{ix/2\pi}$
    \item $\displaystyle e^{2i\pi x/T}$, $T$ est un réel strictement positif
    \item $\displaystyle e^{inx} + e^{ipx}$
\end{enumerate}
\ifthenelse{\boolean{showSolutions}}{}{
\end{multicols}
}


\section*{Produit scalaire réel}

\subsection{Définition}
On appelle produit scalaire sur un espace vectoriel $E$ une application 
$$\langle \cdot, \cdot \rangle : E \times E \to \mathbb{R}$$
telle que :
\begin{multicols}{2}
\begin{itemize}
    \item[*] symétrie : $\langle u, v \rangle = \langle v, u \rangle$ 
    \item[*] linéarité à gauche : $\langle \lambda u + v, w \rangle = \lambda \langle u, w \rangle + \langle v, w \rangle$
    \item[*] positivité : $\langle u, u \rangle \geq 0$
    \item[*] définie positivité : $\langle u, u \rangle = 0 \iff u = 0$
\end{itemize}
\end{multicols}


\vspace{1em}

\subsection{Dans $\mathbb{R}^3$}

On se place dans $\mathbb{R}^3$, qu'on munit de la base 
$$e_1 = (1,2,1), \qquad e_2 = (2,1,-4), \qquad e_3 = (-3,2,-1)$$

\begin{enumerate}
\item La famille est-elle orthogonale ? 
\item Est-elle orthonormée ? Si non, définissez une base $(f_1,f_2,f_3)$ orthonormée à partir de la famille $(e_1,e_2,e_3)$. 
\end{enumerate}

Soit $u$ un vecteur de $\mathbb{R}^3$, on note $u_i$ ses coordonnées dans la base orthonormée $(f_1,f_2,f_3)$. Cela signifie que 
$$u = u_1 f_1 + u_2 f_2 + u_3 f_3$$

Déterminer les coordonnées de $u = (1,0,1)$ dans la base $(f_1,f_2,f_3)$.

\ifthenelse{\boolean{showSolutions}}{
    \vspace{2em}
    \begin{mdframed}
    \textbf{1.} Vérifions si la famille est orthogonale :
    
    $\langle e_1, e_2 \rangle = 1 \cdot 2 + 2 \cdot 1 + 1 \cdot (-4) = 2 + 2 - 4 = 0$ $\checkmark$
    
    $\langle e_1, e_3 \rangle = 1 \cdot (-3) + 2 \cdot 2 + 1 \cdot (-1) = -3 + 4 - 1 = 0$ $\checkmark$
    
    $\langle e_2, e_3 \rangle = 2 \cdot (-3) + 1 \cdot 2 + (-4) \cdot (-1) = -6 + 2 + 4 = 0$ $\checkmark$
    
    La famille est orthogonale.
    
    \textbf{2.} Vérifions si elle est orthonormée :
    
    $\|e_1\|^2 = 1^2 + 2^2 + 1^2 = 6$, donc $\|e_1\| = \sqrt{6}$
    
    $\|e_2\|^2 = 2^2 + 1^2 + (-4)^2 = 4 + 1 + 16 = 21$, donc $\|e_2\| = \sqrt{21}$
    
    $\|e_3\|^2 = (-3)^2 + 2^2 + (-1)^2 = 9 + 4 + 1 = 14$, donc $\|e_3\| = \sqrt{14}$
    
    La famille n'est pas orthonormée. Une base orthonormée est :
    $$f_1 = \frac{e_1}{\|e_1\|} = \frac{1}{\sqrt{6}}(1,2,1) = \left(\frac{1}{\sqrt{6}}, \frac{2}{\sqrt{6}}, \frac{1}{\sqrt{6}}\right)$$
    
    $$f_2 = \frac{e_2}{\|e_2\|} = \frac{1}{\sqrt{21}}(2,1,-4) = \left(\frac{2}{\sqrt{21}}, \frac{1}{\sqrt{21}}, \frac{-4}{\sqrt{21}}\right)$$
    
    $$f_3 = \frac{e_3}{\|e_3\|} = \frac{1}{\sqrt{14}}(-3,2,-1) = \left(\frac{-3}{\sqrt{14}}, \frac{2}{\sqrt{14}}, \frac{-1}{\sqrt{14}}\right)$$
    
    \textbf{3.} Coordonnées de $u = (1,0,1)$ dans la base $(f_1,f_2,f_3)$ :
    
    $u_1 = \langle u, f_1 \rangle = 1 \cdot \frac{1}{\sqrt{6}} + 0 \cdot \frac{2}{\sqrt{6}} + 1 \cdot \frac{1}{\sqrt{6}} = \frac{2}{\sqrt{6}} = \frac{\sqrt{6}}{3}$
    
    $u_2 = \langle u, f_2 \rangle = 1 \cdot \frac{2}{\sqrt{21}} + 0 \cdot \frac{1}{\sqrt{21}} + 1 \cdot \frac{-4}{\sqrt{21}} = \frac{-2}{\sqrt{21}}$
    
    $u_3 = \langle u, f_3 \rangle = 1 \cdot \frac{-3}{\sqrt{14}} + 0 \cdot \frac{2}{\sqrt{14}} + 1 \cdot \frac{-1}{\sqrt{14}} = \frac{-4}{\sqrt{14}}$
    
    Donc $u = \frac{\sqrt{6}}{3}f_1 - \frac{2}{\sqrt{21}}f_2 - \frac{4}{\sqrt{14}}f_3$
\end{mdframed}
}{}

\vspace{1em}

\subsection{Dans $\mathbb{R}[X]$}

\begin{itemize}
    \item Quelle est la dimension de $\mathbb{R}[X]$ ?
\end{itemize}
La famille $(1,X,X^2,X^3, \cdots )$ est appelée base hilbertienne de $\mathbb{R}[X]$ : tout élément de $\mathbb{R}[X]$ peut s'écrire comme une combinaison linéaire finie de vecteurs de cette famille.

On munit cet espace du produit scalaire : 
$$ \langle P, Q \rangle = \int_{0}^{1} P(x) Q(x) dx $$

\begin{itemize}
    \item Montrer que c'est bien un produit scalaire en vérifiant les propriétés ci-dessus.
    \item La famille $(1,X,X^2,X^3, \cdots )$ est-elle orthogonale ? Est-elle orthonormée ?
    \item Comment trouver $a, b, c$ tels que la famille $(1, X-a, X^2-bX-c)$ soit orthogonale ?
    \item Quelles sont les coordonnées de $P = 1+2X+3X^2$ dans la base $(1, X, X^2, \cdots)$ ?
    \item Peut-on retrouver ces coordonnées avec le produit scalaire comme dans l'exercice précédent ?
\end{itemize}
\vspace{1em}

\section*{Produit scalaire complexe}
\subsection{Définition}
On appelle produit scalaire sur un espace vectoriel $E$ une application 
$$\langle \cdot, \cdot \rangle : E \times E \to \mathbb{R}$$
telle que :
\begin{multicols}{2}
\begin{itemize}
    \item[*] symétrie conjuguée : $\langle u, v \rangle = \overline{\langle v, u \rangle}$
    \item[*] linéarité à gauche : $\langle \lambda u + v, w \rangle = \lambda \langle u, w \rangle + \langle v, w \rangle$
    \item[*] positivité : $\langle u, u \rangle \geq 0$
    \item[*] définie positivité : $\langle u, u \rangle = 0 \iff u = 0$
\end{itemize}
\end{multicols}


\subsection{Dans l'espace des fonctions complexes $2\pi$-périodiques}

On définit le produit scalaire :
$$
\langle f, g \rangle = \int_{0}^{2\pi} f(x) \overline{g(x)} dx
$$

Montrer que c'est un produit scalaire.

Montrer que la famille $(e^{inx})_{n \in \mathbb{Z}}$ est orthonormée.

Les coefficients de Fourier d'une fonction $f$ sont les coordonnées de $f$ dans la base $(e^{inx})_{n \in \mathbb{Z}}$.

Déterminer les coefficients de Fourier des fonctions suivantes :

\ifthenelse{\boolean{showSolutions}}{
    \vspace{2em}
    \begin{mdframed}
    \textbf{Démonstration que c'est un produit scalaire :}
    
    Les propriétés de symétrie conjuguée, linéarité et positivité découlent des propriétés de l'intégrale et du conjugué complexe.
    
    \textbf{Démonstration que $(e^{inx})_{n \in \mathbb{Z}}$ est orthonormée :}
    
    Pour $n = m$ : $\langle e^{inx}, e^{inx} \rangle = \int_0^{2\pi} e^{inx} \overline{e^{inx}} dx = \int_0^{2\pi} 1 dx = 2\pi$
    
    Pour $n \neq m$ : $\langle e^{inx}, e^{imx} \rangle = \int_0^{2\pi} e^{inx} \overline{e^{imx}} dx = \int_0^{2\pi} e^{i(n-m)x} dx = 0$
    
    Donc la famille $(e^{inx})_{n \in \mathbb{Z}}$ est orthogonale. Pour l'orthonormaliser, on divise par $\sqrt{2\pi}$.
    
    \textbf{Coefficients de Fourier :}
    \begin{enumerate}
    \item $\cos(x) = \frac{e^{ix} + e^{-ix}}{2}$, donc $c_1 = \frac{1}{2}$, $c_{-1} = \frac{1}{2}$, $c_n = 0$ sinon.
    
    \item $\sin(2\pi x)$ : Cette fonction n'est pas $2\pi$-périodique ! Elle est de période $1$.
    
    \item $\cos(x/2) = \frac{e^{ix/2} + e^{-ix/2}}{2}$, donc $c_{1/2} = \frac{1}{2}$, $c_{-1/2} = \frac{1}{2}$, $c_n = 0$ sinon.
    
    \item $\sin(2x) + \cos(3x) = \frac{e^{i2x} - e^{-i2x}}{2i} + \frac{e^{i3x} + e^{-i3x}}{2}$
    Donc $c_2 = \frac{1}{2i}$, $c_{-2} = -\frac{1}{2i}$, $c_3 = \frac{1}{2}$, $c_{-3} = \frac{1}{2}$, $c_n = 0$ sinon.
    
    \item $f(x) = e^{-x}$ sur $[0, 2\pi]$ :
    $$c_n = \frac{1}{2\pi}\int_0^{2\pi} e^{-x} e^{-inx} dx = \frac{1}{2\pi}\int_0^{2\pi} e^{-(1+in)x} dx$$
    $$= \frac{1}{2\pi}\left[\frac{e^{-(1+in)x}}{-(1+in)}\right]_0^{2\pi} = \frac{1}{2\pi} \cdot \frac{1-e^{-2\pi(1+in)}}{1+in} = \frac{1-e^{-2\pi}e^{-2\pi in}}{2\pi(1+in)}$$
\end{enumerate}
\end{mdframed}
}{}
\ifthenelse{\boolean{showSolutions}}{}
{\begin{multicols}{2}}
\begin{enumerate}
\item $\displaystyle \cos(x)$
\item $\displaystyle \sin(2\pi x)$
\item $\displaystyle \cos(x/2)$
\item $\displaystyle \sin(2x) + \cos(3x)$
\item $\displaystyle \exp^{-x}$ sur l'intervalle $[0, 2\pi]$
\end{enumerate}
\ifthenelse{\boolean{showSolutions}}{}{
\end{multicols}
}


\setcounter{section}{7}
\section*{Les fonctions périodiques}

\subsection{Les fonctions complexes périodiques}

Les fonctions réelles suivantes sont-elles périodiques et si oui, quelle est leur période ?

\ifthenelse{\boolean{showSolutions}}{
    \vspace{2em}
    \begin{mdframed}
    \begin{enumerate}
    \item $\cos(x)$ : Oui, période $T = 2\pi$
    \item $\sin(2\pi x)$ : Oui, période $T = 1$ (car $\sin(2\pi(x+1)) = \sin(2\pi x + 2\pi) = \sin(2\pi x)$)
    \item $\cos(x/2)$ : Oui, période $T = 4\pi$ (car $\cos((x+4\pi)/2) = \cos(x/2 + 2\pi) = \cos(x/2)$)
    \item $\sin(2x) + \cos(3x)$ : Oui, période $T = 2\pi$ (le PPCM des périodes $\pi$ et $\frac{2\pi}{3}$)
    \item $\sin(nx)$ : Oui, période $T = \frac{2\pi}{n}$
    \item $\cos\left(\frac{3x}{2}-\frac{\pi}{4}\right)$ : Oui, période $T = \frac{4\pi}{3}$
    \item $x-\lfloor x\rfloor$ : Oui, période $T = 1$ (c'est la fonction partie fractionnaire)
\end{enumerate}
\end{mdframed}
}{}
\ifthenelse{\boolean{showSolutions}}{}
{\begin{multicols}{2}}
\begin{enumerate}
    \item $\displaystyle \cos(x)$
    
    \item $\displaystyle \sin(2\pi x)$
    \item $\displaystyle \cos(x/2)$
    \item $\displaystyle \sin(2x) + \cos(3x)$
    \item $\displaystyle \sin(nx)$, $n$ est un entier naturel non nul
    \item $\displaystyle \cos \left(\frac{3 x}{2}-\frac{\pi}{4}\right)$
    \item $\displaystyle x-\lfloor x\rfloor$
\end{enumerate}
\ifthenelse{\boolean{showSolutions}}{}{
\end{multicols}
}

Les fonctions complexes suivantes sont-elles périodiques et si oui, quelle est leur période ?

\ifthenelse{\boolean{showSolutions}}{
    \vspace{2em}
    \begin{mdframed}
    \begin{enumerate}
    \item $e^{ix}$ : Oui, période $T = 2\pi$ (car $e^{i(x+2\pi)} = e^{ix}e^{i2\pi} = e^{ix} \cdot 1 = e^{ix}$)
    \item $e^{2ix}$ : Oui, période $T = \pi$ (car $e^{2i(x+\pi)} = e^{2ix}e^{i2\pi} = e^{2ix}$)
    \item $e^{ix/2\pi}$ : Oui, période $T = 4\pi^2$ (car $e^{i(x+4\pi^2)/2\pi} = e^{ix/2\pi}e^{i2\pi} = e^{ix/2\pi}$)
    \item $e^{2i\pi x/T}$ : Oui, période $T$ (car $e^{2i\pi (x+T)/T} = e^{2i\pi x/T}e^{i2\pi} = e^{2i\pi x/T}$)
    \item $e^{inx} + e^{ipx}$ : Oui, période $T = \frac{2\pi}{\text{PGCD}(n,p)}$ (le PPCM des périodes $\frac{2\pi}{n}$ et $\frac{2\pi}{p}$)
\end{enumerate}
\end{mdframed}
}{}
\ifthenelse{\boolean{showSolutions}}{}
{\begin{multicols}{2}}
\begin{enumerate}
    \item $\displaystyle e^{ix}$
    \item $\displaystyle e^{2ix}$
    \item $\displaystyle e^{ix/2\pi}$
    \item $\displaystyle e^{2i\pi x/T}$, $T$ est un réel strictement positif
    \item $\displaystyle e^{inx} + e^{ipx}$
\end{enumerate}
\ifthenelse{\boolean{showSolutions}}{}{
\end{multicols}
}


\section*{Produit scalaire réel}

\subsection{Définition}
On appelle produit scalaire sur un espace vectoriel $E$ une application 
$$\langle \cdot, \cdot \rangle : E \times E \to \mathbb{R}$$
telle que :
\begin{multicols}{2}
\begin{itemize}
    \item[*] symétrie : $\langle u, v \rangle = \langle v, u \rangle$ 
    \item[*] linéarité à gauche : $\langle \lambda u + v, w \rangle = \lambda \langle u, w \rangle + \langle v, w \rangle$
    \item[*] positivité : $\langle u, u \rangle \geq 0$
    \item[*] définie positivité : $\langle u, u \rangle = 0 \iff u = 0$
\end{itemize}
\end{multicols}


\vspace{1em}

\subsection{Dans $\mathbb{R}^3$}

On se place dans $\mathbb{R}^3$, qu'on munit de la base 
$$e_1 = (1,2,1), \qquad e_2 = (2,1,-4), \qquad e_3 = (-3,2,-1)$$

\begin{enumerate}
\item La famille est-elle orthogonale ? 
\item Est-elle orthonormée ? Si non, définissez une base $(f_1,f_2,f_3)$ orthonormée à partir de la famille $(e_1,e_2,e_3)$. 
\end{enumerate}

Soit $u$ un vecteur de $\mathbb{R}^3$, on note $u_i$ ses coordonnées dans la base orthonormée $(f_1,f_2,f_3)$. Cela signifie que 
$$u = u_1 f_1 + u_2 f_2 + u_3 f_3$$

Déterminer les coordonnées de $u = (1,0,1)$ dans la base $(f_1,f_2,f_3)$.

\ifthenelse{\boolean{showSolutions}}{
    \vspace{2em}
    \begin{mdframed}
    \textbf{1.} Vérifions si la famille est orthogonale :
    
    $\langle e_1, e_2 \rangle = 1 \cdot 2 + 2 \cdot 1 + 1 \cdot (-4) = 2 + 2 - 4 = 0$ $\checkmark$
    
    $\langle e_1, e_3 \rangle = 1 \cdot (-3) + 2 \cdot 2 + 1 \cdot (-1) = -3 + 4 - 1 = 0$ $\checkmark$
    
    $\langle e_2, e_3 \rangle = 2 \cdot (-3) + 1 \cdot 2 + (-4) \cdot (-1) = -6 + 2 + 4 = 0$ $\checkmark$
    
    La famille est orthogonale.
    
    \textbf{2.} Vérifions si elle est orthonormée :
    
    $\|e_1\|^2 = 1^2 + 2^2 + 1^2 = 6$, donc $\|e_1\| = \sqrt{6}$
    
    $\|e_2\|^2 = 2^2 + 1^2 + (-4)^2 = 4 + 1 + 16 = 21$, donc $\|e_2\| = \sqrt{21}$
    
    $\|e_3\|^2 = (-3)^2 + 2^2 + (-1)^2 = 9 + 4 + 1 = 14$, donc $\|e_3\| = \sqrt{14}$
    
    La famille n'est pas orthonormée. Une base orthonormée est :
    $$f_1 = \frac{e_1}{\|e_1\|} = \frac{1}{\sqrt{6}}(1,2,1) = \left(\frac{1}{\sqrt{6}}, \frac{2}{\sqrt{6}}, \frac{1}{\sqrt{6}}\right)$$
    
    $$f_2 = \frac{e_2}{\|e_2\|} = \frac{1}{\sqrt{21}}(2,1,-4) = \left(\frac{2}{\sqrt{21}}, \frac{1}{\sqrt{21}}, \frac{-4}{\sqrt{21}}\right)$$
    
    $$f_3 = \frac{e_3}{\|e_3\|} = \frac{1}{\sqrt{14}}(-3,2,-1) = \left(\frac{-3}{\sqrt{14}}, \frac{2}{\sqrt{14}}, \frac{-1}{\sqrt{14}}\right)$$
    
    \textbf{3.} Coordonnées de $u = (1,0,1)$ dans la base $(f_1,f_2,f_3)$ :
    
    $u_1 = \langle u, f_1 \rangle = 1 \cdot \frac{1}{\sqrt{6}} + 0 \cdot \frac{2}{\sqrt{6}} + 1 \cdot \frac{1}{\sqrt{6}} = \frac{2}{\sqrt{6}} = \frac{\sqrt{6}}{3}$
    
    $u_2 = \langle u, f_2 \rangle = 1 \cdot \frac{2}{\sqrt{21}} + 0 \cdot \frac{1}{\sqrt{21}} + 1 \cdot \frac{-4}{\sqrt{21}} = \frac{-2}{\sqrt{21}}$
    
    $u_3 = \langle u, f_3 \rangle = 1 \cdot \frac{-3}{\sqrt{14}} + 0 \cdot \frac{2}{\sqrt{14}} + 1 \cdot \frac{-1}{\sqrt{14}} = \frac{-4}{\sqrt{14}}$
    
    Donc $u = \frac{\sqrt{6}}{3}f_1 - \frac{2}{\sqrt{21}}f_2 - \frac{4}{\sqrt{14}}f_3$
\end{mdframed}
}{}

\vspace{1em}

\subsection{Dans $\mathbb{R}[X]$}

\begin{itemize}
    \item Quelle est la dimension de $\mathbb{R}[X]$ ?
\end{itemize}
La famille $(1,X,X^2,X^3, \cdots )$ est appelée base hilbertienne de $\mathbb{R}[X]$ : tout élément de $\mathbb{R}[X]$ peut s'écrire comme une combinaison linéaire finie de vecteurs de cette famille.

On munit cet espace du produit scalaire : 
$$ \langle P, Q \rangle = \int_{0}^{1} P(x) Q(x) dx $$

\begin{itemize}
    \item Montrer que c'est bien un produit scalaire en vérifiant les propriétés ci-dessus.
    \item La famille $(1,X,X^2,X^3, \cdots )$ est-elle orthogonale ? Est-elle orthonormée ?
    \item Comment trouver $a, b, c$ tels que la famille $(1, X-a, X^2-bX-c)$ soit orthogonale ?
    \item Quelles sont les coordonnées de $P = 1+2X+3X^2$ dans la base $(1, X, X^2, \cdots)$ ?
    \item Peut-on retrouver ces coordonnées avec le produit scalaire comme dans l'exercice précédent ?
\end{itemize}
\vspace{1em}

\section*{Produit scalaire complexe}
\subsection{Définition}
On appelle produit scalaire sur un espace vectoriel $E$ une application 
$$\langle \cdot, \cdot \rangle : E \times E \to \mathbb{R}$$
telle que :
\begin{multicols}{2}
\begin{itemize}
    \item[*] symétrie conjuguée : $\langle u, v \rangle = \overline{\langle v, u \rangle}$
    \item[*] linéarité à gauche : $\langle \lambda u + v, w \rangle = \lambda \langle u, w \rangle + \langle v, w \rangle$
    \item[*] positivité : $\langle u, u \rangle \geq 0$
    \item[*] définie positivité : $\langle u, u \rangle = 0 \iff u = 0$
\end{itemize}
\end{multicols}


\subsection{Dans l'espace des fonctions complexes $2\pi$-périodiques}

On définit le produit scalaire :
$$
\langle f, g \rangle = \int_{0}^{2\pi} f(x) \overline{g(x)} dx
$$

Montrer que c'est un produit scalaire.

Montrer que la famille $(e^{inx})_{n \in \mathbb{Z}}$ est orthonormée.

Les coefficients de Fourier d'une fonction $f$ sont les coordonnées de $f$ dans la base $(e^{inx})_{n \in \mathbb{Z}}$.

Déterminer les coefficients de Fourier des fonctions suivantes :

\ifthenelse{\boolean{showSolutions}}{
    \vspace{2em}
    \begin{mdframed}
    \textbf{Démonstration que c'est un produit scalaire :}
    
    Les propriétés de symétrie conjuguée, linéarité et positivité découlent des propriétés de l'intégrale et du conjugué complexe.
    
    \textbf{Démonstration que $(e^{inx})_{n \in \mathbb{Z}}$ est orthonormée :}
    
    Pour $n = m$ : $\langle e^{inx}, e^{inx} \rangle = \int_0^{2\pi} e^{inx} \overline{e^{inx}} dx = \int_0^{2\pi} 1 dx = 2\pi$
    
    Pour $n \neq m$ : $\langle e^{inx}, e^{imx} \rangle = \int_0^{2\pi} e^{inx} \overline{e^{imx}} dx = \int_0^{2\pi} e^{i(n-m)x} dx = 0$
    
    Donc la famille $(e^{inx})_{n \in \mathbb{Z}}$ est orthogonale. Pour l'orthonormaliser, on divise par $\sqrt{2\pi}$.
    
    \textbf{Coefficients de Fourier :}
    \begin{enumerate}
    \item $\cos(x) = \frac{e^{ix} + e^{-ix}}{2}$, donc $c_1 = \frac{1}{2}$, $c_{-1} = \frac{1}{2}$, $c_n = 0$ sinon.
    
    \item $\sin(2\pi x)$ : Cette fonction n'est pas $2\pi$-périodique ! Elle est de période $1$.
    
    \item $\cos(x/2) = \frac{e^{ix/2} + e^{-ix/2}}{2}$, donc $c_{1/2} = \frac{1}{2}$, $c_{-1/2} = \frac{1}{2}$, $c_n = 0$ sinon.
    
    \item $\sin(2x) + \cos(3x) = \frac{e^{i2x} - e^{-i2x}}{2i} + \frac{e^{i3x} + e^{-i3x}}{2}$
    Donc $c_2 = \frac{1}{2i}$, $c_{-2} = -\frac{1}{2i}$, $c_3 = \frac{1}{2}$, $c_{-3} = \frac{1}{2}$, $c_n = 0$ sinon.
    
    \item $f(x) = e^{-x}$ sur $[0, 2\pi]$ :
    $$c_n = \frac{1}{2\pi}\int_0^{2\pi} e^{-x} e^{-inx} dx = \frac{1}{2\pi}\int_0^{2\pi} e^{-(1+in)x} dx$$
    $$= \frac{1}{2\pi}\left[\frac{e^{-(1+in)x}}{-(1+in)}\right]_0^{2\pi} = \frac{1}{2\pi} \cdot \frac{1-e^{-2\pi(1+in)}}{1+in} = \frac{1-e^{-2\pi}e^{-2\pi in}}{2\pi(1+in)}$$
\end{enumerate}
\end{mdframed}
}{}
\ifthenelse{\boolean{showSolutions}}{}
{\begin{multicols}{2}}
\begin{enumerate}
\item $\displaystyle \cos(x)$
\item $\displaystyle \sin(2\pi x)$
\item $\displaystyle \cos(x/2)$
\item $\displaystyle \sin(2x) + \cos(3x)$
\item $\displaystyle \exp^{-x}$ sur l'intervalle $[0, 2\pi]$
\end{enumerate}
\ifthenelse{\boolean{showSolutions}}{}{
\end{multicols}
}

\end{document}
