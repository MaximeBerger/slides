% =====================================================================
% Template LaTeX – Traces distribuées aux étudiants
% Auteur : (à compléter)
% Compilation : pdflatex/xelatex (pdflatex recommandé ici)
% =====================================================================
\documentclass[11pt,a4paper]{report}

% -------------------- Encodage & langue --------------------
\usepackage[T1]{fontenc}
\usepackage[utf8]{inputenc}
\usepackage[french]{babel}
\usepackage{lmodern}
\usepackage{microtype}
\usepackage{amsmath, amssymb}
\usepackage{multicol}
\usepackage{enumitem}


% -------------------- Mise en page --------------------------
\usepackage[a4paper,margin=2cm]{geometry}
\usepackage{fancyhdr}
\usepackage{parskip}      % espace entre paragraphes
\setlength{\parindent}{0pt}

% -------------------- Couleurs & liens ----------------------
\usepackage{xcolor}
\definecolor{Theme}{HTML}{0E7490} % teal-700
\definecolor{ThemeLight}{HTML}{E0F2F1}
\definecolor{Accent}{HTML}{F59E0B} % amber-500
\definecolor{Gray}{HTML}{374151}
\usepackage[colorlinks=true,linkcolor=Theme,urlcolor=Theme,citecolor=Theme]{hyperref}

% -------------------- Graphiques / décor --------------------
\usepackage{tikz}
\usetikzlibrary{patterns,positioning,calc}
\usepackage{graphicx}
\usepackage{tcolorbox}
\tcbuselibrary{skins,breakable,hooks,most}
\usepackage{fontawesome5}

% -------------------- Titres -------------------------------
\usepackage{titlesec}
\titleformat{\chapter}[display]
  {\Huge\bfseries\color{Theme}}
  {\filright\rule{0.75\linewidth}{1.2pt}\\[3pt]{Algèbre linéaire - Chapitre~\thechapter}}
  {0.2ex}
  {\filright}
  [\vspace{0.1ex}\rule{0.35\linewidth}{1.2pt}]

\titleformat{\section}
  {\Large\bfseries\color{Gray}}
  {\thesection}{0.6em}{}

% -------------------- En-têtes / pieds ---------------------
\pagestyle{fancy}
\fancyhf{}
\fancyhead[L]{\color{Gray}\leftmark}
\fancyhead[R]{\color{Gray}\textit{BMC2}}
\fancyfoot[L]{\color{Gray}\small Auteur~: \textit{M. Berger}}
\fancyfoot[R]{\color{Gray}\small p.\ \thepage}
\renewcommand{\headrulewidth}{0pt}
\renewcommand{\footrulewidth}{0pt}

% -------------------- Macros utilitaires -------------------


% Tcolorboxes stylisées
\tcbset{tracebox/.style={breakable,enhanced,sharp corners,boxrule=0pt,frame hidden,arc=2mm,
  colback=white,coltitle=black,fonttitle=\bfseries\large,
  borderline west={2mm}{0pt}{Theme},
  before skip=8pt,after skip=8pt,drop fuzzy shadow}}

\newtcolorbox{resumeBox}{tracebox,title={\faStickyNote\quad Résumé des idées}}
\newtcolorbox{rappelsBox}{tracebox,title={\faRedo\quad Ce que je dois savoir }}
\newtcolorbox{exempleBox}{tracebox,title={\faChalkboardTeacher\quad Exemple vu ensemble}}

% Encadré « Formules & illustrations »
\newtcolorbox{formulesBox}{tracebox,title={\faCalculator\quad Formules \& illustrations},colback=ThemeLight}

% Astuce : puces clean
\newenvironment{niceitemize}{\begin{itemize}\setlength{\itemsep}{0.25em}\color{Gray}}{\end{itemize}}

% Raccourci pour une « Trace » complète
% Usage : \TraceSection{Titre}{Objectif court}
\newcommand{\TraceSection}[2]{%
  
}

\makeatletter
\renewcommand{\thesubsection}{\arabic{subsection}}
\renewcommand{\p@subsection}{}
% -------------------- Page de titre ------------------------
\title{\textbf{Traces de cours}\\\large (résumés, formules, exemples, mini-exercices)}
\author{ BMC2 }
\date{\today}


\usepackage{mdframed}
\usepackage{ifthen}

% \usepackage[sf]{titlesec}
% Définition de la variable pour afficher les corrections
\newboolean{showSolutions}
% Décommentez la ligne suivante pour afficher les solutions
\input \jobname.adr

% -------------------- Document ----------------------------
\begin{document}




% ================== Séquence 1 ==================

% \chapter{Les vecteurs de $\mathbb{R}^n$}
% 
\newcommand{\mtn}{\mathbb{N}}
\newcommand{\mtns}{\mathbb{N}^*}
\newcommand{\mtz}{\mathbb{Z}}
\newcommand{\mtr}{\mathbb{R}}
\newcommand{\mtk}{\mathbb{K}}
\newcommand{\mtq}{\mathbb{Q}}
\newcommand{\mtc}{\mathbb{C}}
\newcommand{\mch}{\mathcal{H}}
\newcommand{\mcp}{\mathcal{P}}
\newcommand{\mcb}{\mathcal{B}}
\newcommand{\mcl}{\mathcal{L}}
\newcommand{\mcm}{\mathcal{M}}
\newcommand{\mcc}{\mathcal{C}}
\newcommand{\mcmn}{\mathcal{M}}
\newcommand{\mcmnr}{\mathcal{M}_n(\mtr)}
\newcommand{\mcmnk}{\mathcal{M}_n(\mtk)}
\newcommand{\mcsn}{\mathcal{S}_n}
\newcommand{\mcs}{\mathcal{S}}
\newcommand{\mcd}{\mathcal{D}}
\newcommand{\mcsns}{\mathcal{S}_n^{++}}
\newcommand{\glnk}{GL_n(\mtk)}
\newcommand{\mnr}{\mathcal{M}_n(\mtr)}
\newcommand{\veps}{\varepsilon}
\newcommand{\mcu}{\mathcal{U}}
\newcommand{\mcun}{\mcu_n}
\newcommand{\dis}{\displaystyle}
\newcommand{\croouv}{[\![}
\newcommand{\crofer}{]\!]}
\newcommand{\rab}{\mathcal{R}(a,b)}
\newcommand{\pss}[2]{\langle #1,#2\rangle}
 %Document 


\section*{Rappels sur les nombres complexes}

\subsection*{Définitions et formes usuelles}

Un \underline{nombre complexe} $z$ s'écrit sous la forme :
\[
z = a + ib \qquad (a, b \in \mathbb{R},\ i^2 = -1)
\]
où $a$ est la partie réelle $\Re(z)$ et $b$ la partie imaginaire $\Im(z)$.


\underline{Forme algébrique} : $z = a + ib$

\underline{Forme trigonométrique} : 
\[
z = r(\cos\theta + i\sin\theta)
\]
où $r = |z| = \sqrt{a^2 + b^2}$ est le \textbf{module} de $z$, et $\theta = \arg(z)$ est un \textbf{argument} de $z$ (défini à $2\pi$ près).

\underline{Forme exponentielle} (formule d'Euler) :
\[
z = r e^{i\theta}
\]
avec $e^{i\theta} = \cos\theta + i\sin\theta$.


\subsection*{Module, argument et conjugué}

\begin{itemize}
    \item \underline{Module} : $|z| = \sqrt{a^2 + b^2}$
    \item \underline{Argument} : $\theta = \arctan\left(\frac{b}{a}\right)$ (attention au quadrant)
    \item \underline{Conjugué} : $\overline{z} = a - ib$
\end{itemize}

\subsection*{Formule de Moivre}

Pour tout $n \in \mathbb{Z}$,
\[
\left(\cos\theta + i\sin\theta\right)^n = \cos(n\theta) + i\sin(n\theta)
\]
ou, sous forme exponentielle :
\[
\left(e^{i\theta}\right)^n = e^{in\theta}
\]

\subsection*{Racines $n$-ièmes de l'unité}

Les solutions de $z^n = 1$ sont :
\[
z_k = e^{i\frac{2\pi k}{n}},\quad k = 0, 1, \ldots, n-1
\]



\vspace{3em}

\subsection{Module et argument}

Écrire sous la forme $a+i b$, puis sous forme exponentielle les nombres complexes suivants :
\begin{enumerate}
\item Nombre de module 2 et d'argument $\pi / 3$.
\item Nombre de module 3 et d'argument $-\pi / 8$.
\item Nombre de module 1 et d'argument $\pi / 4$.
\item Nombre de module 2 et d'argument $-\pi / 6$.
\item Nombre de module 7 et d'argument $-\pi / 2$.
\end{enumerate}

\vspace{2em}

\subsection{Forme exponentielle $\rightarrow$ forme algébrique}

Écrire sous la forme $a+ib$ les nombres complexes suivants, donnés sous forme exponentielle :
\begin{multicols}{2}
\begin{enumerate}
    \item $z_1 = 5 e^{i \frac{\pi}{6}}$
    \item $z_2 = 2 e^{-i \frac{\pi}{4}}$
    \item $z_3 = 3 e^{i \frac{2\pi}{3}}$
    \item $z_4 = 7 e^{i \pi}$
    \item $z_5 = 4 e^{i 0}$
    \item $z_6 = 6 e^{-i \frac{\pi}{2}}$
\end{enumerate}
\end{multicols}

\vspace{2em}

\subsection{Forme exponentielle}
Mettre sous forme exponentielle les nombres complexes suivants : 
\begin{multicols}{3}
\begin{enumerate}
    \item $z_1=1+i \sqrt{3}$, 
    \item $z_2=1+i$, 
    \item $z_3=-2 \sqrt{3}+2 i$, 
    \item $z_4=i$, 
    \item $z_5=-2 i$, 
    \item $z_6=-3$,
    \item $z_7=1$
    \item $z_8=9 i$
    \item $z_9=0$
    \item $z_{10}=\frac{-i \sqrt{2}}{1+i}$
    \item $z_{11}=\frac{(1+i \sqrt{3})^3}{(1-i)^5}$
    \item $z_{12}=\sin x+i \cos x$.
\end{enumerate}
\end{multicols}



\vspace{2em}

\subsection{Exponentielle}
Résoudre l'équation $e^z=3 \sqrt{3}-3 i$.

\newpage

\subsection{Trigonométrique}
En utilisant les nombres complexes, calculer  $\cos 5 \theta$ et $\sin 5 \theta$ en fonction de $\cos \theta$ et $\sin \theta$.

\vspace{2em}


\subsection{Pour préparer les séries de fourier}
Calculer les intégrales suivantes, pour toute valeur de $n$ et $m$ dans les entiers relatifs:

\begin{enumerate}
    \item $$\int_0^\pi e^{i n x} e^{i m x} dx$$
    \item $$\int_0^\pi \cos(n x) \cos(m x) dx$$
    \item $$\int_0^\pi \sin(n x) \sin(m x) dx$$
    \item $$\int_0^\pi \cos(n x) \sin(m x) dx$$
\end{enumerate}



\vspace{2em}

\subsection{Exponentielle}
On pose 
$$z_1=4 e^{i \frac{\pi}{4}}, \qquad z_2=3 i e^{i \frac{\pi}{6}}, \qquad z_3=-2 e^{i \frac{2 \pi}{3}}$$
Écrire sous forme exponentielle les nombres complexes : 
$$z_1,\qquad z_2,\qquad z_3 , \qquad z_1 z_2, \qquad \frac{z_1 z_2}{z_2}$$



\vspace{2em}

\subsection{Racines carrées}
Calculer de deux façons les racines carrées de $1+i$ et en déduire les valeurs exactes de $\cos \left(\frac{\pi}{8}\right)$ et $\sin \left(\frac{\pi}{8}\right)$.


% \newpage



% ====== Controle continu =======
% \chapter{Controle continu - Pivot de Gauss}
% 
Appliquer le pivot de Gauss pour transformer ces systèmes en systèmes échelonnés. Indiquez pour chacun s'il possède aucune solution, une unique solution ou une infinité de solutions.

$$
\left\{\begin{array}{l}
    2 x+3 y-z=1 \\
    4 x+y+2 z=6 \\
    x-3 y+z=2
\end{array}\right.
$$

\ifthenelse{\boolean{showSolutions}}{
    \vspace{1em}

\begin{mdframed}
    On applique les opérations suivantes pour enlever les $x$ des lignes 2 et 3 :
    \begin{align*}
        L_2 &\leftarrow L_2 - 2L_1 \\
        L_3 &\leftarrow 2L_3 - L_1
    \end{align*}
    On obtient le système suivant :
    $$
    \left\{\begin{array}{l}
        2 x+3 y-z=1 \\
        \quad -5 y+4 z=4 \\
        \quad -9 y+3 z=3
    \end{array}\right.
    $$
    Pour retirer le $y$ de la ligne 3, on applique $L_3 \leftarrow 5L_3 + 9L_2$ :
    $$
    \left\{\begin{array}{l}
        2 x+3 y-z=1 \\
        \quad -5 y+4 z=4 \\
        \quad \quad -21z=-21
    \end{array}\right.
    $$
    Le système est parfaitement échelonné, il admet une unique solution. 
\end{mdframed}
}{}

$$
\left\{\begin{array}{l}
    2 x+y=1 \\
    x+y=0 \\
    3 x+4 y=-1
\end{array}\right.
$$
\ifthenelse{\boolean{showSolutions}}{
    \vspace{1em}

\begin{mdframed}
    On applique les opérations suivantes pour enlever les $x$ des lignes 2 et 3 :
    \begin{align*}
        L_2 &\leftarrow 2L_2 - L_1 \\
        L_3 &\leftarrow 2L_3 - 3L_1
    \end{align*}
    On obtient le système suivant :
    $$
    \left\{\begin{array}{l}
        2 x+y=1 \\
        \quad y=-1 \\
        \quad 5 y=-5
    \end{array}\right.
    $$
    Les deux dernières équations sont les mêmes, on se ramène donc au système :
    $$
    \left\{\begin{array}{l}
        2 x+y=1 \\
        \quad y=-1
    \end{array}\right.
    $$
    Le système est parfaitement échelonné, il admet une unique solution. 
\end{mdframed}
}{}
$$
\left\{\begin{array}{l}
    x+y+z+t=3 \\
    x+y+z-t=3 \\
    x-y-z-t=-1
\end{array}\right.
$$
\ifthenelse{\boolean{showSolutions}}{
    \vspace{1em}
\begin{mdframed}
    On applique les opérations suivantes pour enlever les $x$ des lignes 2 et 3 :
    \begin{align*}
        L_2 &\leftarrow L_2 - L_1 \\
        L_3 &\leftarrow L_3 - L_1
    \end{align*}
    On obtient le système suivant :
    $$
    \left\{\begin{array}{l}
        x+y+z+t=3 \\
        \quad \quad \quad -2 t=6 \\
        \quad -2y -2z -2 t=-4
    \end{array}\right.
    $$
    En échangeant les lignes 2 et 3, on obtient un système échelonné :
    $$
    \left\{\begin{array}{l}
        x+y+z+t=3 \\
        \quad -2y -2z -2 t=-4 \\
        \quad \quad \quad -2 t=6
    \end{array}\right.
    $$
    Le système échelonné possède plus d'inconnues que d'équations, on peut donc garder un paramètre libre 
\end{mdframed}
}{}


% ================== Séquence 2 ==================
% \setcounter{chapter}{1}
% \chapter{Espaces vectoriels}

% 
\newcommand{\mtn}{\mathbb{N}}
\newcommand{\mtns}{\mathbb{N}^*}
\newcommand{\mtz}{\mathbb{Z}}
\newcommand{\mtr}{\mathbb{R}}
\newcommand{\mtk}{\mathbb{K}}
\newcommand{\mtq}{\mathbb{Q}}
\newcommand{\mtc}{\mathbb{C}}
\newcommand{\mch}{\mathcal{H}}
\newcommand{\mcp}{\mathcal{P}}
\newcommand{\mcb}{\mathcal{B}}
\newcommand{\mcl}{\mathcal{L}}
\newcommand{\mcm}{\mathcal{M}}
\newcommand{\mcc}{\mathcal{C}}
\newcommand{\mcmn}{\mathcal{M}}
\newcommand{\mcmnr}{\mathcal{M}_n(\mtr)}
\newcommand{\mcmnk}{\mathcal{M}_n(\mtk)}
\newcommand{\mcsn}{\mathcal{S}_n}
\newcommand{\mcs}{\mathcal{S}}
\newcommand{\mcd}{\mathcal{D}}
\newcommand{\mcsns}{\mathcal{S}_n^{++}}
\newcommand{\glnk}{GL_n(\mtk)}
\newcommand{\mnr}{\mathcal{M}_n(\mtr)}
\newcommand{\veps}{\varepsilon}
\newcommand{\mcu}{\mathcal{U}}
\newcommand{\mcun}{\mcu_n}
\newcommand{\dis}{\displaystyle}
\newcommand{\croouv}{[\![}
\newcommand{\crofer}{]\!]}
\newcommand{\rab}{\mathcal{R}(a,b)}
\newcommand{\pss}[2]{\langle #1,#2\rangle}
 %Document 


\section*{Rappels sur les nombres complexes}

\subsection*{Définitions et formes usuelles}

Un \underline{nombre complexe} $z$ s'écrit sous la forme :
\[
z = a + ib \qquad (a, b \in \mathbb{R},\ i^2 = -1)
\]
où $a$ est la partie réelle $\Re(z)$ et $b$ la partie imaginaire $\Im(z)$.


\underline{Forme algébrique} : $z = a + ib$

\underline{Forme trigonométrique} : 
\[
z = r(\cos\theta + i\sin\theta)
\]
où $r = |z| = \sqrt{a^2 + b^2}$ est le \textbf{module} de $z$, et $\theta = \arg(z)$ est un \textbf{argument} de $z$ (défini à $2\pi$ près).

\underline{Forme exponentielle} (formule d'Euler) :
\[
z = r e^{i\theta}
\]
avec $e^{i\theta} = \cos\theta + i\sin\theta$.


\subsection*{Module, argument et conjugué}

\begin{itemize}
    \item \underline{Module} : $|z| = \sqrt{a^2 + b^2}$
    \item \underline{Argument} : $\theta = \arctan\left(\frac{b}{a}\right)$ (attention au quadrant)
    \item \underline{Conjugué} : $\overline{z} = a - ib$
\end{itemize}

\subsection*{Formule de Moivre}

Pour tout $n \in \mathbb{Z}$,
\[
\left(\cos\theta + i\sin\theta\right)^n = \cos(n\theta) + i\sin(n\theta)
\]
ou, sous forme exponentielle :
\[
\left(e^{i\theta}\right)^n = e^{in\theta}
\]

\subsection*{Racines $n$-ièmes de l'unité}

Les solutions de $z^n = 1$ sont :
\[
z_k = e^{i\frac{2\pi k}{n}},\quad k = 0, 1, \ldots, n-1
\]



\vspace{3em}

\subsection{Module et argument}

Écrire sous la forme $a+i b$, puis sous forme exponentielle les nombres complexes suivants :
\begin{enumerate}
\item Nombre de module 2 et d'argument $\pi / 3$.
\item Nombre de module 3 et d'argument $-\pi / 8$.
\item Nombre de module 1 et d'argument $\pi / 4$.
\item Nombre de module 2 et d'argument $-\pi / 6$.
\item Nombre de module 7 et d'argument $-\pi / 2$.
\end{enumerate}

\vspace{2em}

\subsection{Forme exponentielle $\rightarrow$ forme algébrique}

Écrire sous la forme $a+ib$ les nombres complexes suivants, donnés sous forme exponentielle :
\begin{multicols}{2}
\begin{enumerate}
    \item $z_1 = 5 e^{i \frac{\pi}{6}}$
    \item $z_2 = 2 e^{-i \frac{\pi}{4}}$
    \item $z_3 = 3 e^{i \frac{2\pi}{3}}$
    \item $z_4 = 7 e^{i \pi}$
    \item $z_5 = 4 e^{i 0}$
    \item $z_6 = 6 e^{-i \frac{\pi}{2}}$
\end{enumerate}
\end{multicols}

\vspace{2em}

\subsection{Forme exponentielle}
Mettre sous forme exponentielle les nombres complexes suivants : 
\begin{multicols}{3}
\begin{enumerate}
    \item $z_1=1+i \sqrt{3}$, 
    \item $z_2=1+i$, 
    \item $z_3=-2 \sqrt{3}+2 i$, 
    \item $z_4=i$, 
    \item $z_5=-2 i$, 
    \item $z_6=-3$,
    \item $z_7=1$
    \item $z_8=9 i$
    \item $z_9=0$
    \item $z_{10}=\frac{-i \sqrt{2}}{1+i}$
    \item $z_{11}=\frac{(1+i \sqrt{3})^3}{(1-i)^5}$
    \item $z_{12}=\sin x+i \cos x$.
\end{enumerate}
\end{multicols}



\vspace{2em}

\subsection{Exponentielle}
Résoudre l'équation $e^z=3 \sqrt{3}-3 i$.

\newpage

\subsection{Trigonométrique}
En utilisant les nombres complexes, calculer  $\cos 5 \theta$ et $\sin 5 \theta$ en fonction de $\cos \theta$ et $\sin \theta$.

\vspace{2em}


\subsection{Pour préparer les séries de fourier}
Calculer les intégrales suivantes, pour toute valeur de $n$ et $m$ dans les entiers relatifs:

\begin{enumerate}
    \item $$\int_0^\pi e^{i n x} e^{i m x} dx$$
    \item $$\int_0^\pi \cos(n x) \cos(m x) dx$$
    \item $$\int_0^\pi \sin(n x) \sin(m x) dx$$
    \item $$\int_0^\pi \cos(n x) \sin(m x) dx$$
\end{enumerate}



\vspace{2em}

\subsection{Exponentielle}
On pose 
$$z_1=4 e^{i \frac{\pi}{4}}, \qquad z_2=3 i e^{i \frac{\pi}{6}}, \qquad z_3=-2 e^{i \frac{2 \pi}{3}}$$
Écrire sous forme exponentielle les nombres complexes : 
$$z_1,\qquad z_2,\qquad z_3 , \qquad z_1 z_2, \qquad \frac{z_1 z_2}{z_2}$$



\vspace{2em}

\subsection{Racines carrées}
Calculer de deux façons les racines carrées de $1+i$ et en déduire les valeurs exactes de $\cos \left(\frac{\pi}{8}\right)$ et $\sin \left(\frac{\pi}{8}\right)$.

% \setcounter{chapter}{2}
% \chapter{Familles de vecteurs}
% 
\newcommand{\mtn}{\mathbb{N}}
\newcommand{\mtns}{\mathbb{N}^*}
\newcommand{\mtz}{\mathbb{Z}}
\newcommand{\mtr}{\mathbb{R}}
\newcommand{\mtk}{\mathbb{K}}
\newcommand{\mtq}{\mathbb{Q}}
\newcommand{\mtc}{\mathbb{C}}
\newcommand{\mch}{\mathcal{H}}
\newcommand{\mcp}{\mathcal{P}}
\newcommand{\mcb}{\mathcal{B}}
\newcommand{\mcl}{\mathcal{L}}
\newcommand{\mcm}{\mathcal{M}}
\newcommand{\mcc}{\mathcal{C}}
\newcommand{\mcmn}{\mathcal{M}}
\newcommand{\mcmnr}{\mathcal{M}_n(\mtr)}
\newcommand{\mcmnk}{\mathcal{M}_n(\mtk)}
\newcommand{\mcsn}{\mathcal{S}_n}
\newcommand{\mcs}{\mathcal{S}}
\newcommand{\mcd}{\mathcal{D}}
\newcommand{\mcsns}{\mathcal{S}_n^{++}}
\newcommand{\glnk}{GL_n(\mtk)}
\newcommand{\mnr}{\mathcal{M}_n(\mtr)}
\newcommand{\veps}{\varepsilon}
\newcommand{\mcu}{\mathcal{U}}
\newcommand{\mcun}{\mcu_n}
\newcommand{\dis}{\displaystyle}
\newcommand{\croouv}{[\![}
\newcommand{\crofer}{]\!]}
\newcommand{\rab}{\mathcal{R}(a,b)}
\newcommand{\pss}[2]{\langle #1,#2\rangle}
 %Document 


\section*{Rappels sur les nombres complexes}

\subsection*{Définitions et formes usuelles}

Un \underline{nombre complexe} $z$ s'écrit sous la forme :
\[
z = a + ib \qquad (a, b \in \mathbb{R},\ i^2 = -1)
\]
où $a$ est la partie réelle $\Re(z)$ et $b$ la partie imaginaire $\Im(z)$.


\underline{Forme algébrique} : $z = a + ib$

\underline{Forme trigonométrique} : 
\[
z = r(\cos\theta + i\sin\theta)
\]
où $r = |z| = \sqrt{a^2 + b^2}$ est le \textbf{module} de $z$, et $\theta = \arg(z)$ est un \textbf{argument} de $z$ (défini à $2\pi$ près).

\underline{Forme exponentielle} (formule d'Euler) :
\[
z = r e^{i\theta}
\]
avec $e^{i\theta} = \cos\theta + i\sin\theta$.


\subsection*{Module, argument et conjugué}

\begin{itemize}
    \item \underline{Module} : $|z| = \sqrt{a^2 + b^2}$
    \item \underline{Argument} : $\theta = \arctan\left(\frac{b}{a}\right)$ (attention au quadrant)
    \item \underline{Conjugué} : $\overline{z} = a - ib$
\end{itemize}

\subsection*{Formule de Moivre}

Pour tout $n \in \mathbb{Z}$,
\[
\left(\cos\theta + i\sin\theta\right)^n = \cos(n\theta) + i\sin(n\theta)
\]
ou, sous forme exponentielle :
\[
\left(e^{i\theta}\right)^n = e^{in\theta}
\]

\subsection*{Racines $n$-ièmes de l'unité}

Les solutions de $z^n = 1$ sont :
\[
z_k = e^{i\frac{2\pi k}{n}},\quad k = 0, 1, \ldots, n-1
\]



\vspace{3em}

\subsection{Module et argument}

Écrire sous la forme $a+i b$, puis sous forme exponentielle les nombres complexes suivants :
\begin{enumerate}
\item Nombre de module 2 et d'argument $\pi / 3$.
\item Nombre de module 3 et d'argument $-\pi / 8$.
\item Nombre de module 1 et d'argument $\pi / 4$.
\item Nombre de module 2 et d'argument $-\pi / 6$.
\item Nombre de module 7 et d'argument $-\pi / 2$.
\end{enumerate}

\vspace{2em}

\subsection{Forme exponentielle $\rightarrow$ forme algébrique}

Écrire sous la forme $a+ib$ les nombres complexes suivants, donnés sous forme exponentielle :
\begin{multicols}{2}
\begin{enumerate}
    \item $z_1 = 5 e^{i \frac{\pi}{6}}$
    \item $z_2 = 2 e^{-i \frac{\pi}{4}}$
    \item $z_3 = 3 e^{i \frac{2\pi}{3}}$
    \item $z_4 = 7 e^{i \pi}$
    \item $z_5 = 4 e^{i 0}$
    \item $z_6 = 6 e^{-i \frac{\pi}{2}}$
\end{enumerate}
\end{multicols}

\vspace{2em}

\subsection{Forme exponentielle}
Mettre sous forme exponentielle les nombres complexes suivants : 
\begin{multicols}{3}
\begin{enumerate}
    \item $z_1=1+i \sqrt{3}$, 
    \item $z_2=1+i$, 
    \item $z_3=-2 \sqrt{3}+2 i$, 
    \item $z_4=i$, 
    \item $z_5=-2 i$, 
    \item $z_6=-3$,
    \item $z_7=1$
    \item $z_8=9 i$
    \item $z_9=0$
    \item $z_{10}=\frac{-i \sqrt{2}}{1+i}$
    \item $z_{11}=\frac{(1+i \sqrt{3})^3}{(1-i)^5}$
    \item $z_{12}=\sin x+i \cos x$.
\end{enumerate}
\end{multicols}



\vspace{2em}

\subsection{Exponentielle}
Résoudre l'équation $e^z=3 \sqrt{3}-3 i$.

\newpage

\subsection{Trigonométrique}
En utilisant les nombres complexes, calculer  $\cos 5 \theta$ et $\sin 5 \theta$ en fonction de $\cos \theta$ et $\sin \theta$.

\vspace{2em}


\subsection{Pour préparer les séries de fourier}
Calculer les intégrales suivantes, pour toute valeur de $n$ et $m$ dans les entiers relatifs:

\begin{enumerate}
    \item $$\int_0^\pi e^{i n x} e^{i m x} dx$$
    \item $$\int_0^\pi \cos(n x) \cos(m x) dx$$
    \item $$\int_0^\pi \sin(n x) \sin(m x) dx$$
    \item $$\int_0^\pi \cos(n x) \sin(m x) dx$$
\end{enumerate}



\vspace{2em}

\subsection{Exponentielle}
On pose 
$$z_1=4 e^{i \frac{\pi}{4}}, \qquad z_2=3 i e^{i \frac{\pi}{6}}, \qquad z_3=-2 e^{i \frac{2 \pi}{3}}$$
Écrire sous forme exponentielle les nombres complexes : 
$$z_1,\qquad z_2,\qquad z_3 , \qquad z_1 z_2, \qquad \frac{z_1 z_2}{z_2}$$



\vspace{2em}

\subsection{Racines carrées}
Calculer de deux façons les racines carrées de $1+i$ et en déduire les valeurs exactes de $\cos \left(\frac{\pi}{8}\right)$ et $\sin \left(\frac{\pi}{8}\right)$.

\chapter{Controle continu - Espaces vectoriels}

Appliquer le pivot de Gauss pour transformer ces systèmes en systèmes échelonnés. Indiquez pour chacun s'il possède aucune solution, une unique solution ou une infinité de solutions.

$$
\left\{\begin{array}{l}
    2 x+3 y-z=1 \\
    4 x+y+2 z=6 \\
    x-3 y+z=2
\end{array}\right.
$$

\ifthenelse{\boolean{showSolutions}}{
    \vspace{1em}

\begin{mdframed}
    On applique les opérations suivantes pour enlever les $x$ des lignes 2 et 3 :
    \begin{align*}
        L_2 &\leftarrow L_2 - 2L_1 \\
        L_3 &\leftarrow 2L_3 - L_1
    \end{align*}
    On obtient le système suivant :
    $$
    \left\{\begin{array}{l}
        2 x+3 y-z=1 \\
        \quad -5 y+4 z=4 \\
        \quad -9 y+3 z=3
    \end{array}\right.
    $$
    Pour retirer le $y$ de la ligne 3, on applique $L_3 \leftarrow 5L_3 + 9L_2$ :
    $$
    \left\{\begin{array}{l}
        2 x+3 y-z=1 \\
        \quad -5 y+4 z=4 \\
        \quad \quad -21z=-21
    \end{array}\right.
    $$
    Le système est parfaitement échelonné, il admet une unique solution. 
\end{mdframed}
}{}

$$
\left\{\begin{array}{l}
    2 x+y=1 \\
    x+y=0 \\
    3 x+4 y=-1
\end{array}\right.
$$
\ifthenelse{\boolean{showSolutions}}{
    \vspace{1em}

\begin{mdframed}
    On applique les opérations suivantes pour enlever les $x$ des lignes 2 et 3 :
    \begin{align*}
        L_2 &\leftarrow 2L_2 - L_1 \\
        L_3 &\leftarrow 2L_3 - 3L_1
    \end{align*}
    On obtient le système suivant :
    $$
    \left\{\begin{array}{l}
        2 x+y=1 \\
        \quad y=-1 \\
        \quad 5 y=-5
    \end{array}\right.
    $$
    Les deux dernières équations sont les mêmes, on se ramène donc au système :
    $$
    \left\{\begin{array}{l}
        2 x+y=1 \\
        \quad y=-1
    \end{array}\right.
    $$
    Le système est parfaitement échelonné, il admet une unique solution. 
\end{mdframed}
}{}
$$
\left\{\begin{array}{l}
    x+y+z+t=3 \\
    x+y+z-t=3 \\
    x-y-z-t=-1
\end{array}\right.
$$
\ifthenelse{\boolean{showSolutions}}{
    \vspace{1em}
\begin{mdframed}
    On applique les opérations suivantes pour enlever les $x$ des lignes 2 et 3 :
    \begin{align*}
        L_2 &\leftarrow L_2 - L_1 \\
        L_3 &\leftarrow L_3 - L_1
    \end{align*}
    On obtient le système suivant :
    $$
    \left\{\begin{array}{l}
        x+y+z+t=3 \\
        \quad \quad \quad -2 t=6 \\
        \quad -2y -2z -2 t=-4
    \end{array}\right.
    $$
    En échangeant les lignes 2 et 3, on obtient un système échelonné :
    $$
    \left\{\begin{array}{l}
        x+y+z+t=3 \\
        \quad -2y -2z -2 t=-4 \\
        \quad \quad \quad -2 t=6
    \end{array}\right.
    $$
    Le système échelonné possède plus d'inconnues que d'équations, on peut donc garder un paramètre libre 
\end{mdframed}
}{}

\end{document}
