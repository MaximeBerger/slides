
\begin{resumeBox}
  \emph{À retenir dans une semaine :} 
  \begin{niceitemize}
    \item Un espace vectoriel est un ensemble avec deux lois : une addition interne et une multiplication avec des scalaire.
    \item Un vecteur est un élément d'un espace vectoriel.
    \item Plusieurs espaces vectoriels à identifier :
      \begin{itemize}
        \item[$\bullet$] $\mathbb{R}^n$
        \item[$\bullet$] L'ensemble des polynômes à coefficients réels
        \item[$\bullet$] L'ensemble des matrices carrées de taille $n$
        \item[$\bullet$] L'ensemble des fonctions de $\mathbb{R}$ dans $\mathbb{R}$
      \end{itemize}
  \end{niceitemize}
\end{resumeBox}

\begin{rappelsBox}
  \begin{niceitemize}
    \item Que signifie qu'une loi soit interne à un ensemble ?
    \item Quelles sont les règles de calculs pour les espaces vectoriels ?
    \item Comment montrer qu'un ensemble est un espace vectoriel ?
  \end{niceitemize}
\end{rappelsBox}

\section{Exercices}
  \subsection{Exemples d'espaces vectoriels}

  Les ensembles ci dessous sont munis des opérations classiques d'addition et de multiplication, possèdent-ils une structure d'espace vectoriel ? \newline 
  Si oui, préciser le corps de scalaires, donner leur dimension et expliciter une base.
  
  \begin{enumerate}[i]
      \item L'ensemble des points d'un demi-plan de \(\mathbb{R}^2\).
      \item L'ensemble des polynômes dont la dérivée vaut $1$ en $0$.
      \item L'ensemble des nombres complexes. ( de deux façons différentes )
      \item L'ensemble des polynômes dont le degré est inférieur ou égal à un certain entier $n$.
      \item L'ensemble des suites réelles bornées dont le premier terme est $1$.
  \end{enumerate}
  
  \subsection{parties de $\mathbb{R}^2$}

    Les parties suivantes sont-elles des sous-espaces vectoriels de $\mathbb{R}^2$ ?
    \begin{enumerate}[i]
      \item A=\left\{(x, y) \in \mathbb{R}^2 \mid x \leqslant y\right\} \\
      \item B=\left\{(x, y) \in \mathbb{R}^2 \mid x y=0\right\} \\
      \item C=\left\{(x, y) \in \mathbb{R}^2 \mid x=y\right\} \\
      \item D=\left\{(x, y) \in \mathbb{R}^2 \mid x+y=1\right\}
    \end{enumerate}
    
    \subsection{Sous-espaces vectoriels}
    Déterminer si les ensembles suivants sont ou ne sont pas des sous-espaces vectoriels:
    \begin{enumerate}[i]
      \item $E_1=\{P \in \mathbb{R}[X] ; P(0)=P(2)\}$;

      \ifthenelse{\boolean{showanswers}}{
      \begin{mdframed}
      Soit $f, g \in E$, et soit $M_1, M_2$ un majorant respectif de $|f|,|g|$. Alors, pour tout $\lambda \in \mathbb{R}$, et tout $x \in \mathbb{R},$

      $$  
      |f(x)+g(x)| \leq|f(x)|+|g(x)| \leq M_1+M_2,|\lambda f(x)| \leq|\lambda| \times M_1 .
      $$


      Ainsi, $f+g$ et $\lambda f$ sont elles aussi bornées, et $V$ est un sous-espace vectoriel de $E$.
      \end{mdframed}
      }{
      }

      \item $E_2=\left\{P \in \mathbb{R}[X] ; P^{\prime}(0)=2\right\}$;
      \ifthenelse{\boolean{showanswers}}{
      \begin{mdframed}
        Considérons la fonction $f$ définie pour $x \in \mathbb{R}$ par $f(x)=-|x|$. Alors $f$ est majorée (par $0$). Mais on a pour tout $x \in \mathbb{R}:-f(x)=|x|$. Ainsi, la fonction $-f$ n'est pas majorée. Donc $f \in V$ et $-f \notin V: V$ n'est pas un espace vectoriel de $E$.
      \end{mdframed}
      }{
      }

      \item Pour $A \in \mathbb{R}[X]$ non-nul fixé, $E_3=\{P \in \mathbb{R}[X] ; A \mid P\}$;

      \ifthenelse{\boolean{showanswers}}{
      \begin{mdframed}
        Prenons $f$ et $g$ deux fonctions paires et $\lambda \in \mathbb{R}$. Alors

        $$
        \begin{aligned}
        (f+g)(-x) & =f(-x)+g(-x) \\
        & =f(x)+g(x)=(f+g)(x), \\
        (\lambda f)(-x)= & \lambda f(-x)=\lambda f(x)=(\lambda f)(x) .
        \end{aligned}
        $$
  
        Ainsi, $f+g$ et $\lambda f$ sont paires et $V$ est un sous-espace vectoriel de $E$.     
      \end{mdframed}
      }{
      }

      \item D l'ensemble des fonctions de $\mathbb{R}$ dans $\mathbb{R}$ qui sont dérivables;
      \item $E_4$, l'ensemble des solutions de l'équation différentielle $y^{\prime}+a(x) y=0$, où $a \in \mathcal{D}$.
      \item $E_5$, l'ensemble des solutions de l'équation différentielle $y^{\prime}+a(x) y=x$, où $a \in \mathcal{D}$.
    \end{enumerate}

      \ifthenelse{\boolean{showanswers}}{
      \begin{mdframed}
        Prenons $f(x)=x^2$ et $g(x)=x$. Alors $f$ est paire et $g$ est impaire. Mais $(f+g)(1)=2$ et $(f+g)(-1)=0$. Ainsi, $f+g$ n'est ni paire, ni impaire et $V$ n'est pas un sous-espace vectoriel de $E$.
      \end{mdframed}
    }{}
  
\end{document}