
\begin{resumeBox}
  \emph{À retenir dans une semaine :} 
  \begin{niceitemize}
    \item Un espace vectoriel est un ensemble avec deux lois : une addition interne et une multiplication par un scalaire.
    \item Un vecteur est un élément d'un espace vectoriel.
    \item Pour vérifier qu'un ensemble est un espace vectoriel  :
      \begin{itemize}
        \item[$\bullet$] Possède-t-il un vecteur nul ?
        \item[$\bullet$] L'addition est-elle une loi interne ?
        \item[$\bullet$] La multiplication par un scalaire est-elle une loi externe ?
        \item[$\bullet$] Les règles de calcul de l'addition et de la multiplication par un scalaire sont-elles vérifiées ?
      \end{itemize}
  \end{niceitemize}
\end{resumeBox}

\begin{rappelsBox}
  \begin{niceitemize}
    \item Que signifie qu'une loi soit interne à un ensemble ?
    \item Quelles sont les règles de calculs pour les espaces vectoriels ?
    \item Qu'est-ce que le vecteur nul d'un espace vectoriel ?
  \end{niceitemize}
\end{rappelsBox}

\section{Exercices}
\subsection{Les espaces vectoriels dans $\mathbb{R}^n$}

\begin{enumerate}
  \item Quels sont les espaces vectoriels inclus dans $\mathbb{R}$ ?
  \item Quels sont les espaces vectoriels inclus dans $\mathbb{R}^2$ ?
  \item Quels sont les espaces vectoriels inclus dans $\mathbb{R}^3$ ?
\end{enumerate}

\ifthenelse{\boolean{showSolutions}}{}{\newpage }




\vspace{2em}
\subsection{parties de $\mathbb{R}^2$}

  Les parties suivantes sont-elles des sous-espaces vectoriels de $\mathbb{R}^2$ ?

  \ifthenelse{\boolean{showSolutions}}{}{\begin{multicols}{2} }
  \begin{enumerate}
    \item $A=\left\{(x, y) \in \mathbb{R}^2 \mid x \leqslant y\right\}$ 
    \item $B=\left\{(x, y) \in \mathbb{R}^2 \mid x y=0\right\}$ 
    \item $C=\left\{(x, y) \in \mathbb{R}^2 \mid x=y\right\}$ 
    \item $D=\left\{(x, y) \in \mathbb{R}^2 \mid x+y=1\right\}$
  \end{enumerate}
  \ifthenelse{\boolean{showSolutions}}{}{\end{multicols} }


\vspace{1em}

  \subsection{Dans un espace de fonctions}

  Soit $E$ le $\mathbb{R}$-espace vectoriel des applications de $[0,1]$ dans $\mathbb{R}$ muni des opérations usuelles. 
  Soit $F$ l'ensemble des applications de $[0,1]$ dans $\mathbb{R}$ vérifiant l'une des conditions suivantes :

  \ifthenelse{\boolean{showSolutions}}{}{\begin{multicols}{2} }
  \begin{enumerate}
    \item $f(0)+f(1)=0$
    \item $f(0)=0$
    \item $f\left(\frac{1}{2}\right)=\frac{1}{4}$
    \item $\forall x \in[0,1], f(x)+f(1-x)=0$
    \item $\forall x \in[0,1], f(x) \geqslant 0$
    \item $2 f(0)=f(1)+3$
  \end{enumerate}
  \ifthenelse{\boolean{showSolutions}}{}{\end{multicols} }

Dans quel cas $F$ est-il un espace vectoriel inclus dans $E$?

\vspace{1em}

\subsection{Dans $\mathbb{R}^n$}

On munit $\mathbb{R}^n$ des lois usuelles. Parmi les sous-ensembles suivants $F$ de $\mathbb{R}^n$, lesquels sont des espaces vectoriels?

\ifthenelse{\boolean{showSolutions}}{}{\begin{multicols}{2} }
\begin{enumerate}
  \item $\mathrm{F}=\left\{\left(\mathrm{x}_1, \ldots, \mathrm{x}_{\mathrm{n}}\right) \in \mathbb{R}^{\mathrm{n}} / \mathrm{x}_1=0\right\}$
  \item $\mathrm{F}=\left\{\left(\mathrm{x}_1, \ldots, \mathrm{x}_{\mathrm{n}}\right) \in \mathbb{R}^{\mathrm{n}} / \mathrm{x}_1=1\right\}$
  \item $F=\left\{\left(x_1, \ldots, x_n\right) \in \mathbb{R}^n / x_1=x_2\right\}$
  \item $\mathrm{F}=\left\{\left(x_1, \ldots, x_{\mathrm{n}}\right) \in \mathbb{R}^{\mathrm{n}} / x_1+\ldots+x_{\mathrm{n}}=0\right\}$
  \item $\mathrm{F}=\left\{\left(\mathrm{x}_1, \ldots, \mathrm{x}_{\mathrm{n}}\right) \in \mathbb{R}^{\mathrm{n}} / \mathrm{x}_1 \times \mathrm{x}_2=0\right\}$
\end{enumerate}
\ifthenelse{\boolean{showSolutions}}{}{\end{multicols} }


\vspace{2em}

\subsection{Dans $\mathbb{R}[X]$}


Les ensembles suivants sont-ils des espaces vectoriels?
\begin{enumerate}
  \item $A_1$ est l'ensemble des polynômes réels P vérifiant $\mathrm{P}(0)=1$.
  \item $A_2$ est l'ensemble des polynômes réels ayant $a$ comme racine. ( $a \in \mathbb{R}$ fixé).
  \item $A_3$ est l'ensemble des polynômes réels ayant au moins une racine réelle.
  \item $A_4$ est l'ensemble des polynômes réels de degré 3.
\end{enumerate}

\vspace{2em}

\subsection{Dans l'espace des suites réelles}

Les ensembles suivants sont-ils des espaces vectoriels?
\begin{enumerate}
  \item $A_1$ est l'ensemble des suites réelles convergentes vers 1.
  \item $A_2$ est l'ensemble des suites réelles négligeables devant $n^2$
  \item $A_3$ est l'ensemble des suites réelles équivalentes à $n^2$.
  \item $A_4$ est l'ensemble des suites réelles $\left(u_n\right)_{n \in \mathbb{N}}$ vérifiant :

$$
\forall n \in \mathbb{N}, u_{n+1}=5 u_n-3
$$

  \item $A_5$ est l'ensemble des suites réelles arithmétiques.
  \item $A_6$ est l'ensemble des suites réelles géométriques.
\end{enumerate}


  \vspace{2em}
    \subsection{Mélange}
    Déterminer si les ensembles suivants sont ou ne sont pas des espaces vectoriels:
    \begin{enumerate}
      \item $E_1=\{P \in \mathbb{R}[X] ; P(0)=P(2)\}$;
      \item $E_2=\left\{P \in \mathbb{R}[X] ; P^{\prime}(0)=2\right\}$;
      \item Pour $A \in \mathbb{R}[X]$ non-nul fixé, $E_3=\{P \in \mathbb{R}[X] ; A \mid P\}$;
      \item D l'ensemble des fonctions de $\mathbb{R}$ dans $\mathbb{R}$ qui sont dérivables;
      \item $E_4$, l'ensemble des solutions de l'équation différentielle $y^{\prime}+a(x) y=0$, où $a \in \mathcal{D}$.
      \item $E_5$, l'ensemble des solutions de l'équation différentielle $y^{\prime}+a(x) y=x$, où $a \in \mathcal{D}$.
    \end{enumerate}

\section*{Bases}
\vspace{2em}

\subsection{Dans $\mathbb{R}^2$}
\begin{enumerate}
\item  Montrer que les vecteurs $v_1=(0,1), v_2=(1,1)$ forment une base de $\mathbb{R}^2$. Trouver les composantes du vecteur $w=(1,2)$ dans cette base $\left(v_1, v_2\right)$.
\item Montrer que les vecteurs $v_1=(1,2), v_2=(3,4)$forment une base de $\mathbb{R}^2$. Trouver les composantes des vecteurs $e_1=(1,0)$ et $e_2=(0,1)$ dans cette base $\left(v_1, v_2, v_3\right)$.
\item Dans $\mathbb{R}^2$, donner un exemple de famille libre qui n'est pas génératrice.
\item Dans $\mathbb{R}^2$, donner un exemple de famille génératrice qui n'est pas libre.
\end{enumerate}

\vspace{2em}

\subsection{Dans $\mathbb{R}^3$}
\begin{enumerate}
\item  Montrer que les vecteurs $v_1=(0,1,1), v_2=(1,0,1)$ et $v_3=(1,1,0)$ forment une base de $\mathbb{R}^3$. \newline
 Trouver les composantes du vecteur $w=(1,1,1)$ dans cette base $\left(v_1, v_2, v_3\right)$.
\item Montrer que les vecteurs $v_1=(1,1,1), v_2=(-1,1,0)$ et $v_3=(1,0,-1)$ forment une base de $\mathbb{R}^3$. \newline 
Trouver les composantes du vecteur $e_1=(1,0,0), e_2=(0,1,0), e_3=(0,0,1)$ et $w=(1,2,-3)$ dans cette base $\left(v_1, v_2, v_3\right)$.
\item Dans $\mathbb{R}^3$, donner un exemple de famille libre qui n'est pas génératrice.
\item Dans $\mathbb{R}^3$, donner un exemple de famille génératrice qui n'est pas libre.
\end{enumerate}

\vspace{2em}

\subsection{avec un paramètre}

Déterminer pour quelles valeurs de $t \in \mathbb{R}$ les vecteurs

$$
\{(1,0, t),(1,1, t),(t, 0,1)\}
$$

forment une base de $\mathbb{R}^3$.

\vspace{2em}
Considérons la famille $B=\left(Q_1, Q_2, Q_3\right)$ où $$\left\{\begin{array}{l}Q_1=X^2+1 \\ Q_2=3 X^2-X+3 \\ Q_3=X^2-X-1\end{array}\right.$$
$B$ est-elle une base de $\mathbb{R}_2[X]$ ? Sioui, déterminer les coordonnées de $X$.
  
\end{document}