Les ensembles suivants sont-ils des espaces vectoriels inclus dans $\mathbb{R}^n$ ? 


\textbf{1.}
$$ \mathrm{F}=\left\{\left(x_1, x_2\right) \in \mathbb{R}^2 / x_1-x_2=0\right\}$$

\begin{itemize}[label=$\bullet$, itemsep=1em]
    \item Le vecteur nul est-il dans $\mathrm{F}$ ? 

    Le vecteur nul est le vecteur \dots\dots\dots 

    Ses coordonnées vérifient \dots\dots\dots
    donc \dots\dots\dots 

\item L'addition est-elle une loi interne dans $\mathrm{F}$ ? 

    Prenons deux vecteurs quelconques de $\mathrm{F}$ : \dots\dots\dots \newline
    Comme ces vecteurs sont dans $\mathrm{F}$, leurs coordonnées vérifient \dotfill

    Leur somme est le vecteur \dots\dots\dots 

    Ses coordonnées sont \dots\dots\dots et vérifient \dotfill

    Donc \dots\dots\dots 

\item La multiplication par un scalaire est-elle une loi interne dans $\mathrm{F}$ ? 

    Prenons un vecteur quelconque de $\mathrm{F}$ et un scalaire réel quelconque : \ldots\ldots\ldots 

    Le produit du scalaire par le vecteur est \dots\dots\dots 

    Ses coordonnées sont \dots\dots\dots et vérifient \dots\dots\dots 
    
    Donc \dots\dots\dots 
\end{itemize}

Finalement \dots\dots\dots 

\vspace{1em}

\newpage 

\textbf{2.}

$$    \mathrm{F}=\left\{\left(x_1, x_2\right) \in \mathbb{R}^2 / x_1 \, x_2=0\right\}
    $$

    \begin{itemize}[label=$\bullet$, itemsep=1em]
        \item Le vecteur nul est-il dans $\mathrm{F}$ ? 

    Le vecteur nul est le vecteur \dots\dots\dots 

    Ses coordonnées vérifient \dots\dots\dots 

    Donc \dots\dots\dots 

    \item L'addition est-elle une loi interne dans $\mathrm{F}$ ? 

    Prenons deux vecteurs quelconques de $\mathrm{F}$ : \dots\dots\dots 

    Leur somme est le vecteur \dots\dots\dots 

    Ses coordonnées vérifient : \dots\dots\dots 

    Or, \dots\dots\dots 

    Donc, en général, \dots\dots\dots 
\end{itemize}

\textbf{3.}

$$
\mathrm{F} = \left\{ (x_1, x_2, x_3) \in \mathbb{R}^3 \mid x_1 + 2x_2 - x_3 = 0 \right\}
$$

\begin{itemize}[label=$\bullet$, itemsep=1em]
    \item Le vecteur nul est-il dans $\mathrm{F}$ ?

    Le vecteur nul est le vecteur \dotfill

    Ses coordonnées vérifient \dotfill

    Donc \dotfill

    \item L'addition est-elle une loi interne dans $\mathrm{F}$ ?

    Prenons deux vecteurs quelconques de $\mathrm{F}$ : \dotfill

    Leur somme est le vecteur \dotfill

    Vérifions : \dotfill

    Donc \dotfill

    \item La multiplication par un scalaire est-elle une loi interne dans $\mathrm{F}$ ?

    Prenons un vecteur quelconque de $\mathrm{F}$ et un scalaire réel quelconque : \dotfill

    Le produit du scalaire par le vecteur est \dotfill

    Vérifions : \dotfill

    Donc \dotfill
\end{itemize}

Finalement \dotfill

\vspace{1em}


