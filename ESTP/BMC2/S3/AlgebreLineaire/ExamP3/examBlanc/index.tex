\documentclass[12pt]{article}

% Packages pour les marges
\usepackage[
    top=2cm,
    bottom=2cm,
    left=2cm,
    right=2cm
]{geometry}

% Police sans serif
% \usepackage{helvet}
% \renewcommand{\familydefault}{\sfdefault}

% Packages existants
\usepackage[french]{babel}
\usepackage[utf8]{inputenc}
\usepackage[T1]{fontenc}
\usepackage{amsmath}
\usepackage{amsfonts}
\usepackage{amssymb}
\usepackage{mathtools}
\usepackage{array}
\usepackage[version=4]{mhchem}
\usepackage{stmaryrd}
\usepackage{enumitem}
\usepackage{ifthen}
\usepackage{eurosym}
\usepackage{textcomp}
\usepackage{graphicx}
\usepackage{multicol}

\usepackage{xcolor}
\definecolor{Theme}{HTML}{0E7490} % teal-700
\definecolor{ThemeLight}{HTML}{E0F2F1}
\definecolor{Accent}{HTML}{F59E0B} % amber-500
\definecolor{Gray}{HTML}{374151}
\usepackage[colorlinks=true,linkcolor=Theme,urlcolor=Theme,citecolor=Theme]{hyperref}

\usepackage{mdframed}
\usepackage[sf]{titlesec}
\newboolean{showSolutions}
% Décommentez la ligne suivante pour afficher les solutions
\input \jobname.adr
\title{Mathématiques : Examen}

\author{}
\date{}

\newenvironment{solution}
{
    \vspace{0.5em}
    \begin{mdframed}[backgroundcolor=ThemeLight,leftmargin=0,rightmargin=0,skipabove=0.2em,skipbelow=0.2em]
    \textbf{Solution.}\\[0.5em]
}
{
    \end{mdframed}
    \vspace{0.5em}
}
\begin{document}
\sffamily

\begin{center}
    \renewcommand{\arraystretch}{1.5} % Ajuste l'espacement vertical des lignes
    \begin{tabular}{|>{\centering\arraybackslash}m{4cm}|>{\centering\arraybackslash}m{6cm}|>{\centering\arraybackslash}m{4cm}|}
        \hline 
        \vspace{5mm} \hspace{5mm}\raisebox{-0.2\height}{\includegraphics[width=3cm]{Logo-ESTP.png}} \vspace{5mm}  & 
        \textbf{Contrôle de connaissances et de compétences} & 
        \textbf{FO-002-VLA-XX-001} \\
        \hline
        \textbf{21/05/2025}  &  & \textbf{Page 1/3} \\
        \hline
    \end{tabular}
\end{center}
\vspace{1em}

\begin{center}
    \renewcommand{\arraystretch}{1.5}
    \begin{tabular}{|c|m{10cm}|}
        \hline 
        \multicolumn{2}{|c|}{\textbf{ANNÉE SCOLAIRE 2024-2025 -- Semestre 6}} \\
        \hline 
        \textbf{Nom de l'enseignant} & Maxime BERGER, Karine Serier \\
        \hline 
        \textbf{Promotion} & BMC2 - S3 \\
        \hline 
        \textbf{Matière} & Mathématiques  \\
        \hline 
        \textbf{Durée de l'examen} & 3h00 \\
        \hline 
        \textbf{Consignes} & 
        \vspace{0.5em}
        \begin{itemize}
            \item Calculatrice \textbf{NON} autorisée
            \item Aucun document n'est autorisé \vspace{1em}
        \end{itemize}\\
        \hline
    \end{tabular}
\end{center}



%==============================================================================
\section*{Exercice 1 : Convergence de séries (4 points)}
%==============================================================================

Étudier la nature (convergence ou divergence) des séries suivantes. Justifier soigneusement chaque réponse.
\ifthenelse{\boolean{showSolutions}}{}{
\begin{multicols}{2}
}

\begin{enumerate}
    \item $\displaystyle \sum_{n=0}^{+\infty} \left(\frac{2}{3}\right)^n$ \textit{(1 pt)}
    
    \ifthenelse{\boolean{showSolutions}}{
    \begin{solution}
    C'est une série géométrique de raison $r = \frac{2}{3}$.
    
    Comme $|r| = \frac{2}{3} < 1$, la série \textbf{converge}.
    
    Sa somme vaut : $S = \frac{1}{1 - \frac{2}{3}} = \frac{1}{\frac{1}{3}} = 3$
    \end{solution}
    }{}
    
    \item $\displaystyle \sum_{n=1}^{+\infty} \frac{(-1)^n}{\sqrt{n}}$ \textit{(1 pt)}
    
    \ifthenelse{\boolean{showSolutions}}{
    \begin{solution}
    C'est une série alternée de la forme $\sum (-1)^n a_n$ avec $a_n = \frac{1}{\sqrt{n}}$.
    
    Vérifions le critère des séries alternées (Leibniz) :
    \begin{itemize}
        \item $(a_n)$ est décroissante : $\frac{1}{\sqrt{n+1}} < \frac{1}{\sqrt{n}}$ \checkmark
        \item $\lim_{n \to +\infty} a_n = 0$ \checkmark
    \end{itemize}
    
    Par le critère de Leibniz, la série \textbf{converge}.
    
    \textit{Remarque : elle ne converge pas absolument car $\sum \frac{1}{\sqrt{n}}$ diverge (Riemann avec $\alpha = 1/2 < 1$).}
    \end{solution}
    }{}
    
    \item $\displaystyle \sum_{n=1}^{+\infty} \frac{n!}{3^n}$ \textit{(1 pt)}
    
    \ifthenelse{\boolean{showSolutions}}{
    \begin{solution}
    Utilisons le critère de D'Alembert. Posons $u_n = \frac{n!}{3^n}$.
    \[
    \frac{u_{n+1}}{u_n} = \frac{(n+1)!}{3^{n+1}} \cdot \frac{3^n}{n!} = \frac{(n+1) \cdot n!}{3 \cdot 3^n} \cdot \frac{3^n}{n!} = \frac{n+1}{3}
    \]
    
    Donc $\lim_{n \to +\infty} \frac{u_{n+1}}{u_n} = +\infty > 1$.
    
    Par le critère de D'Alembert, la série \textbf{diverge}.
    \end{solution}
    }{}
    
    \item $\displaystyle \sum_{n=2}^{+\infty} \frac{1}{n^2 - 1}$ \textit{(1.5 pts)}
    
    \ifthenelse{\boolean{showSolutions}}{
    \begin{solution}
    Cherchons un équivalent du terme général :
    \[
    \frac{1}{n^2 - 1} = \frac{1}{n^2(1 - \frac{1}{n^2})} \sim \frac{1}{n^2} \quad \text{quand } n \to +\infty
    \]
    
    Or $\sum \frac{1}{n^2}$ est une série de Riemann avec $\alpha = 2 > 1$, donc convergente.
    
    Par équivalence de séries à termes positifs, la série $\sum \frac{1}{n^2-1}$ \textbf{converge}.
    \end{solution}
    }{}
\end{enumerate}
\ifthenelse{\boolean{showSolutions}}{}{
\end{multicols}
}

\vspace{2em}

%==============================================================================
\section*{Exercice 2 : Série télescopique et comparaison (4 points)}
%==============================================================================

\begin{enumerate}
    \item Décomposer $\frac{1}{n(n+1)}$ en éléments simples. \textit{(0.5 pt)}
    
    \ifthenelse{\boolean{showSolutions}}{
    \begin{solution}
    \[
    \frac{1}{n(n+1)} = \frac{A}{n} + \frac{B}{n+1}
    \]
    En multipliant par $n(n+1)$ : $1 = A(n+1) + Bn$
    \begin{itemize}
        \item $n = 0$ : $A = 1$
        \item $n = -1$ : $B = -1$
    \end{itemize}
    Donc $\frac{1}{n(n+1)} = \frac{1}{n} - \frac{1}{n+1}$
    \end{solution}
    }{}
    
    \item En déduire la valeur de la somme partielle $S_N = \sum_{n=1}^{N} \frac{1}{n(n+1)}$. \textit{(1 pt)}
    
    \ifthenelse{\boolean{showSolutions}}{
    \begin{solution}
    \[
    S_N = \sum_{n=1}^{N} \left(\frac{1}{n} - \frac{1}{n+1}\right)
    \]
    C'est une série télescopique. En développant :
    \[
    S_N = \left(1 - \frac{1}{2}\right) + \left(\frac{1}{2} - \frac{1}{3}\right) + \cdots + \left(\frac{1}{N} - \frac{1}{N+1}\right) = 1 - \frac{1}{N+1}
    \]
    \end{solution}
    }{}
    
    \item Calculer $\sum_{n=1}^{+\infty} \frac{1}{n(n+1)}$. \textit{(0.5 pt)}
    
    \ifthenelse{\boolean{showSolutions}}{
    \begin{solution}
    \[
    \sum_{n=1}^{+\infty} \frac{1}{n(n+1)} = \lim_{N \to +\infty} S_N = \lim_{N \to +\infty} \left(1 - \frac{1}{N+1}\right) = 1
    \]
    \end{solution}
    }{}
    
    \item À l'aide d'une comparaison série/intégrale, montrer que $\sum_{n=1}^{+\infty} \frac{1}{n^3}$ converge. \textit{(2 pts)}
    
    \textit{Indication : comparer avec $\int_1^{+\infty} \frac{1}{x^3} dx$.}
    
    \ifthenelse{\boolean{showSolutions}}{
    \begin{solution}
    La fonction $f(x) = \frac{1}{x^3}$ est positive, continue et décroissante sur $[1, +\infty[$.

    On a, pour $x \in [n, n+1]$, $\dfrac{1}{(n+1)^3} \leq \dfrac{1}{x^3} \leq \dfrac{1}{n^3}$.
    
    En intégrant cette inégalité entre $n$ et $n+1$, on obtient
    \[
    \int_n^{n+1} \frac{1}{(n+1)^3} dx \leq \int_n^{n+1} \frac{1}{x^3} dx \leq \int_n^{n+1} \frac{1}{n^3} dx
    \]
    Ce qui revient, en calculant les intégrales à gauche et à droite, à
    \[
    \frac{1}{(n+1)^3} \leq \int_n^{n+1} \frac{1}{x^3} dx \leq \frac{1}{n^3}
    \]

    Gardons seulement la partie gauche de cette inégalité, et sommons pour $n$ allant de 1 à $N$, on obtient
    \[
    \sum_{n=1}^{N} \frac{1}{(n+1)^3} \leq \int_1^{N+1} \frac{1}{x^3} dx
    \]

    On peut calculer cette intégrale : 
    \[
    \int_1^{N+1} \frac{1}{x^3} dx = \left[-\frac{1}{2x^2}\right]_1^{N+1} = \frac{1}{2(N+1)^2} - \left(-\frac{1}{2}\right) = \frac{1}{2} - \frac{1}{2(N+1)^2}
    \]

    On peut aussi réindexer la somme pour obtenir la somme initiale : 
    \[
    \sum_{n=1}^{N} \frac{1}{(n+1)^3} = \sum_{n=2}^{N+1} \frac{1}{n^3}
    \]
    Si on veut commencer la somme à 1, on ajoute le terme $\frac{1}{1^3}$ qui vaut 1:
    \[
    \sum_{n=1}^{N} \frac{1}{(n+1)^3} = \sum_{n=1}^{N+1} \frac{1}{n^3} - 1
    \]
    L'inégalité devient : 
    \[
    \sum_{n=1}^{N+1} \frac{1}{n^3} - 1 \leq \frac{1}{2} - \frac{1}{2(N+1)^2}
    \]
    Par passage à la limite, on montre que la série ne tend pas vers l'infini, elle est bornée par $3/2$.
    On peut donc conclure que la série $\sum \frac{1}{n^3}$ \textbf{converge}.

    \end{solution}
    }{}
\end{enumerate}

\newpage

\begin{center}
    \renewcommand{\arraystretch}{1.5} 
    \begin{tabular}{|>{\centering\arraybackslash}m{4cm}|>{\centering\arraybackslash}m{6cm}|>{\centering\arraybackslash}m{4cm}|}
        \hline
            \hspace{4cm}&\hspace{6cm} & \textbf{Page 2/3}\\
            \hline
    \end{tabular}
\end{center}

\vspace{1em}

%==============================================================================
\section*{Exercice 3 : Algèbre linéaire (4 points)}
%==============================================================================

Soit $E = \{(x, y, z) \in \mathbb{R}^3 \mid x + y - z = 0\}$.

\begin{enumerate}
    \item Montrer que $E$ est un sous-espace vectoriel de $\mathbb{R}^3$. \textit{(1.5 pts)}
    
    \ifthenelse{\boolean{showSolutions}}{
    \begin{solution}
    Vérifions les trois axiomes d'un sous-espace vectoriel :
    
    \textbf{1) Non vide :} $(0, 0, 0) \in E$ car $0 + 0 - 0 = 0$. \checkmark
    
    \textbf{2) Stabilité par addition :} Soient $u = (x_1, y_1, z_1)$ et $v = (x_2, y_2, z_2)$ dans $E$.
    \begin{itemize}
        \item $x_1 + y_1 - z_1 = 0$ et $x_2 + y_2 - z_2 = 0$
        \item $u + v = (x_1 + x_2, y_1 + y_2, z_1 + z_2)$
        \item $(x_1 + x_2) + (y_1 + y_2) - (z_1 + z_2) = (x_1 + y_1 - z_1) + (x_2 + y_2 - z_2) = 0$
    \end{itemize}
    Donc $u + v \in E$. \checkmark
    
    \textbf{3) Stabilité par multiplication :} Soit $\lambda \in \mathbb{R}$ et $u = (x, y, z) \in E$.
    \begin{itemize}
        \item $\lambda u = (\lambda x, \lambda y, \lambda z)$
        \item $\lambda x + \lambda y - \lambda z = \lambda(x + y - z) = 0$
    \end{itemize}
    Donc $\lambda u \in E$. \checkmark
    
    $E$ est bien un sous-espace vectoriel de $\mathbb{R}^3$.
    \end{solution}
    }{}
    
    \item Ecrire $E$ avec le mot clé "Vect", donner une base et la dimension. \textit{(1 pt)}
    
    \ifthenelse{\boolean{showSolutions}}{
    \begin{solution}
    De $x + y - z = 0$, on tire $z = x + y$. Donc on peut choisir $x$ et $y$ comme paramètres libres :
    \[
    (x, y, z) = (x, y, x+y) = x(1, 0, 1) + y(0, 1, 1)
    \]

    Donc $E = \text{Vect}\{(1, 0, 1), (0, 1, 1)\}$.

    Les vecteurs $e_1 = (1, 0, 1)$ et $e_2 = (0, 1, 1)$ sont clairement linéairement indépendants.
    
    \textbf{Base de $E$ :} $\mathcal{B}_E = \{(1, 0, 1), (0, 1, 1)\}$
    
    \textbf{Dimension :} $\dim(E) = 2$
    \end{solution}
    }{}
    
    \item Soit $f : \mathbb{R}^2 \to \mathbb{R}^3$ définie par $f(x, y) = (x + y, x - y, 2x)$.
    
    Montrer que $f$ est une application linéaire. \textit{(1 pt)}
    
    \ifthenelse{\boolean{showSolutions}}{
    \begin{solution}
    Soient $(x_1, y_1), (x_2, y_2) \in \mathbb{R}^2$ et $\lambda \in \mathbb{R}$.
    
    \textbf{Additivité :}
    \begin{align*}
    f((x_1, y_1) + (x_2, y_2)) &= f(x_1 + x_2, y_1 + y_2) \\
    &= ((x_1+x_2) + (y_1+y_2), (x_1+x_2) - (y_1+y_2), 2(x_1+x_2)) \\
    &= (x_1+y_1, x_1-y_1, 2x_1) + (x_2+y_2, x_2-y_2, 2x_2) \\
    &= f(x_1, y_1) + f(x_2, y_2) \quad \checkmark
    \end{align*}
    
    \textbf{Homogénéité :}
    \begin{align*}
    f(\lambda(x, y)) &= f(\lambda x, \lambda y) \\
    &= (\lambda x + \lambda y, \lambda x - \lambda y, 2\lambda x) \\
    &= \lambda(x + y, x - y, 2x) = \lambda f(x, y) \quad \checkmark
    \end{align*}
    
    $f$ est bien une application linéaire.
    \end{solution}
    }{}
    
    \item Écrire la matrice $A$ de $f$ dans les bases canoniques de $\mathbb{R}^2$ et $\mathbb{R}^3$. \textit{(1 pt)}
    
    \ifthenelse{\boolean{showSolutions}}{
    \begin{solution}
    On calcule les images des vecteurs de la base canonique de $\mathbb{R}^2$ :
    \begin{itemize}
        \item $f(1, 0) = (1, 1, 2)$
        \item $f(0, 1) = (1, -1, 0)$
    \end{itemize}
    
    La matrice de $f$ est :
    \[
    A = \begin{pmatrix} 1 & 1 \\ 1 & -1 \\ 2 & 0 \end{pmatrix}
    \]
    \end{solution}
    }{}
\end{enumerate}

\vspace{2em}

%==============================================================================
\section*{Exercice 4 : Noyau et image (4 points)}
%==============================================================================

Soit $g : \mathbb{R}^3 \to \mathbb{R}^2$ l'application linéaire définie par :
\[
g(x, y, z) = (x + 2y - z, 2x + 4y - 2z)
\]

\begin{enumerate}
    \item Écrire la matrice $B$ de $g$ dans les bases canoniques. \textit{(0.5 pt)}
    
    \ifthenelse{\boolean{showSolutions}}{
    \begin{solution}
    \[
    B = \begin{pmatrix} 1 & 2 & -1 \\ 2 & 4 & -2 \end{pmatrix}
    \]
    \end{solution}
    }{}
    
    \item Déterminer le noyau $\ker(g)$. Donner une base et la dimension. \textit{(2 pts)}
    
    \ifthenelse{\boolean{showSolutions}}{
    \begin{solution}
    $(x, y, z) \in \ker(g) \Leftrightarrow g(x, y, z) = (0, 0)$
    
    On résout le système :
    \[
    \begin{cases}
    x + 2y - z = 0 \\
    2x + 4y - 2z = 0
    \end{cases}
    \]
    
    La deuxième équation est le double de la première, donc on a une seule contrainte. On choisira deux paramètres libres $y$ et $z$. 
    On exprime alors $x$ en fonction de $y$ et $z$ : $x = z - 2y$.
    
    Les solutions sont :
    \[
    (x, y, z) = (z - 2y, y, z) = y(-2, 1, 0) + z(1, 0, 1)
    \]
    
    \textbf{Base de $\ker(g)$ :} $\{(-2, 1, 0), (1, 0, 1)\}$
    
    \textbf{Dimension :} $\dim(\ker(g)) = 2$
    \end{solution}
    }{}
    
    \item Déterminer l'image $\text{Im}(g)$. Donner une base et la dimension. \textit{(1.5 pts)}
    
    \ifthenelse{\boolean{showSolutions}}{
    \begin{solution}
    L'image de $g$ est engendrée par les colonnes de $B$ :
    \[
    \text{Im}(g) = \text{Vect}\left\{\begin{pmatrix} 1 \\ 2 \end{pmatrix}, \begin{pmatrix} 2 \\ 4 \end{pmatrix}, \begin{pmatrix} -1 \\ -2 \end{pmatrix}\right\}
    \]
    
    On remarque que $(2, 4) = 2(1, 2)$ et $(-1, -2) = -(1, 2)$.
    
    Donc $\text{Im}(g) = \text{Vect}\{(1, 2)\}$.
    
    \textbf{Base de $\text{Im}(g)$ :} $\{(1, 2)\}$
    
    \textbf{Dimension :} $\dim(\text{Im}(g)) = 1$
    
    \textit{Vérification par le théorème du rang :} $\dim(\ker(g)) + \dim(\text{Im}(g)) = 2 + 1 = 3 = \dim(\mathbb{R}^3)$ \checkmark
    \end{solution}
    }{}
\end{enumerate}

\vspace{1em}

%==============================================================================
\section*{Exercice 5 : Calcul différentiel et intégrales curvilignes (4 points)}
%==============================================================================

\begin{enumerate}
    \item Soit $f(x, y) = x^2 y + e^{xy}$. Calculer la différentielle $df$. \textit{(1 pt)}
    
    \ifthenelse{\boolean{showSolutions}}{
    \begin{solution}
    \[
    \frac{\partial f}{\partial x} = 2xy + ye^{xy}, \qquad \frac{\partial f}{\partial y} = x^2 + xe^{xy}
    \]
    
    Donc :
    \[
    df = (2xy + ye^{xy})dx + (x^2 + xe^{xy})dy
    \]
    \end{solution}
    }{}
    
    \item Soit $g(x, y, z) = x^2 + y^2 z - z^3$. Calculer le gradient $\nabla g$. \textit{(0.5 pt)}
    
    \ifthenelse{\boolean{showSolutions}}{
    \begin{solution}
    \[
    \nabla g = \begin{pmatrix} 
    \dfrac{\partial g}{\partial x} \\[0.8em] 
    \dfrac{\partial g}{\partial y} \\[0.8em] 
    \dfrac{\partial g}{\partial z} 
    \end{pmatrix} = \begin{pmatrix} 2x \\[0.5em] 2yz \\[0.5em] y^2 - 3z^2 \end{pmatrix}
    \]
    \end{solution}
    }{}
    
    \item Soit $\vec{F}(x, y, z) = (x^2, xyz, z^2)$. Calculer la matrice jacobienne de $\vec{F}$ et la divergence $\text{div}(\vec{F})$. \textit{(1.5 pts)}
    
    \ifthenelse{\boolean{showSolutions}}{
    \begin{solution}
    La matrice jacobienne est :
    \[
    J_{\vec{F}} = \begin{pmatrix}
    \dfrac{\partial F_1}{\partial x} & \dfrac{\partial F_1}{\partial y} & \dfrac{\partial F_1}{\partial z} \\[0.8em]
    \dfrac{\partial F_2}{\partial x} & \dfrac{\partial F_2}{\partial y} & \dfrac{\partial F_2}{\partial z} \\[0.8em]
    \dfrac{\partial F_3}{\partial x} & \dfrac{\partial F_3}{\partial y} & \dfrac{\partial F_3}{\partial z}
    \end{pmatrix} = \begin{pmatrix}
    2x & 0 & 0 \\
    yz & xz & xy \\
    0 & 0 & 2z
    \end{pmatrix}
    \]
    
    La divergence est :
    \[
    \text{div}(\vec{F}) = \frac{\partial F_1}{\partial x} + \frac{\partial F_2}{\partial y} + \frac{\partial F_3}{\partial z} = 2x + xz + 2z
    \]
    \end{solution}
    }{}
    
    \item Soit $\vec{G}(x, y) = (2xy + y^2, x^2 + 2xy)$. 
    
    Montrer que $\vec{G}$ dérive d'un potentiel scalaire et déterminer ce potentiel. \textit{(1 pt)}
    
    \ifthenelse{\boolean{showSolutions}}{
    \begin{solution}
    Vérifions la condition d'irrotationnalité : $\frac{\partial G_1}{\partial y} = \frac{\partial G_2}{\partial x}$
    \[
    \frac{\partial G_1}{\partial y} = 2x + 2y, \qquad \frac{\partial G_2}{\partial x} = 2x + 2y \quad \checkmark
    \]
    
    Donc $\vec{G}$ dérive d'un potentiel $\varphi$ tel que $\nabla \varphi = \vec{G}$.
    
    On intègre :
    \[
    \frac{\partial \varphi}{\partial x} = 2xy + y^2 \implies \varphi(x, y) = x^2 y + xy^2 + h(y)
    \]
    
    On vérifie avec la deuxième composante :
    \[
    \frac{\partial \varphi}{\partial y} = x^2 + 2xy + h'(y) = x^2 + 2xy \implies h'(y) = 0 \implies h(y) = C
    \]
    
    \textbf{Potentiel :} $\varphi(x, y) = x^2 y + xy^2 + C$
    \end{solution}
    }{}
    
    \item Calculer l'intégrale curviligne $\displaystyle \int_{C^+} xy \, dx + (x + y) \, dy$
    
    où $C$ est le segment de droite allant de $(0, 0)$ à $(1, 2)$, parcouru dans ce sens. \textit{(1 pt)}
    
    \ifthenelse{\boolean{showSolutions}}{
    \begin{solution}
    Paramétrons le segment : $\gamma(t) = (t, 2t)$ pour $t \in [0, 1]$.
    
    On a : $x = t$, $y = 2t$, $dx = dt$, $dy = 2dt$.
    
    \begin{align*}
    \int_{C^+} xy \, dx + (x + y) \, dy &= \int_0^1 (t)(2t) \, dt + (t + 2t)(2 \, dt) \\
    &= \int_0^1 2t^2 \, dt + \int_0^1 6t \, dt \\
    &= \left[\frac{2t^3}{3}\right]_0^1 + \left[3t^2\right]_0^1 \\
    &= \frac{2}{3} + 3 = \frac{11}{3}
    \end{align*}
    \end{solution}
    }{}
\end{enumerate}


\end{document}