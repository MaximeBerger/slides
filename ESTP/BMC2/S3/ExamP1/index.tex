\documentclass[a4paper,11pt]{article}

%	Tous les packages pour francisation compl\' ete
\usepackage[utf8]{inputenc}

\usepackage[T1]{fontenc}
\usepackage{lmodern,textcomp}
\usepackage[frenchb]{babel}

\usepackage{amsfonts,amsmath,amssymb,mathrsfs}
\usepackage{geometry}
\usepackage{aeguill}
\usepackage{hyperref}
\usepackage{array}
\usepackage{multirow}
\usepackage{asymptote}
\usepackage{graphicx}
\usepackage{tabularx}
\usepackage{enumerate,tablists}
\usepackage{eurosym}
\usepackage{pst-solides3d}
\usepackage{multirow}
\usepackage{multicol}
\usepackage{extsizes}
\usepackage{calc}
\usepackage{boites}
\usepackage{color,colortbl}
\usepackage{fancyhdr}
\usepackage{stmaryrd}
\usepackage{pifont}
%\usepackage{pst-all,pstricks,pstricks-add,pst-plot,pst-eucl,pst-text,pst-tree,pst-math,pst-eps}

%\usepackage{auto-pst-pdf}
\makeatletter
\let\Test@pr@shipout\pr@shipout%% save the original definition
\let\Test@shipout\shipout
\makeatother
\usepackage{tikz,tkz-tab}
\makeatletter
\AtBeginDocument{%
  \let\pr@shipout\Test@pr@shipout%% restore it 
  \let\shipout\Test@shipout
}
\makeatother

\newtheorem{definition}{D\'efinition}
\newtheorem{theoreme}{Th\'eor\`eme}

\newcommand{\C} {\ensuremath{\mathbb{C}}}
\newcommand{\R} {\ensuremath{\mathbb{R}}}
\newcommand{\Z} {\ensuremath{\mathbb{Z}}}
\newcommand{\N} {\ensuremath{\mathbb{N}}}
\newcommand{\ds} {\displaystyle}
\newcommand{\p}{\par}
\newcommand{\abs}[1]{\left |#1\right |}
\newcommand{\eq}[1]{\underset{#1}{\sim}}
\newcommand{\tend}[1]{\underset{#1}{\longrightarrow}}
\newcommand{\norme}[1]{\left\Vert #1\right\Vert}
\newcommand{\V}[1]{\overrightarrow{#1}} %vecteur

\definecolor{gris1}{gray}{0.85}
\definecolor{gris2}{gray}{0.65}

\setlength{\textwidth} {19cm}
\setlength{\textheight} {26cm}
\addtolength{\topmargin} {-2.5cm}
\setlength{\oddsidemargin} {-1.5cm}
\setlength{\evensidemargin} {-1.5cm}

\newcommand{\encad}[1]{\begin{center}%
\fcolorbox{black}{gray!20}{\begin{minipage}[t]{0.8\linewidth}%
#1\end{minipage}}\end{center}}


\pagestyle{fancy}
\fancyfoot[R]{\today}\renewcommand\headrulewidth{1pt}
\fancyhead[L]{ESTP / Bachelor 2 / \the\year}
\fancyhead[R]{Composition de math\'ematiques}
\renewcommand\footrulewidth{1pt}
%\fancyfoot[L]{Math\' ematiques : les polyn\' emes}
%\fancyfoot[R]{\thepage}




\begin{document}
\begin{center}\bf {\Large\underline{Composition de MATH\'EMATIQUES  (3h)}}\end{center}
%\begin{center}\bf {{\small Rendre le sujet avec votre copie si le graphe de l'exercice 4 pr\' esente la courbe de la fonction $f$.}}\end{center}
\hfill\break




\underline {\ding{110}\, \bf{Exercice 1  : Limites (6 points)}}
\medskip

Calculer les limites suivantes :


$$ \lim_{x \to 0} \left(\frac{e^x - (1+x)}{x^2}\right)\qquad \quad
 \lim_{x \to 0} \left(\frac{x^2}{\cos(x) - 1}\right)\qquad \quad
 \lim_{x \to 0} \left(\frac{(e^x - 1)\sin(x)}{x^2}\right)$$


\medskip
{\color{blue}\underline{\bf Solution}\\

Au voisinage de $0$:\\
\[
e^x=1+x+\dfrac{x^2}{2}+\dfrac{x^3}{3!}+o(x^3) \qquad \cos(x)=1-\dfrac{x^2}{2}+o(x^3) \qquad \sin(x)=x-\dfrac{x^3}{3!}+o(x^3)
\]
Ainsi

$
\dfrac{e^x - (1+x)}{x^2}=\dfrac{\dfrac{x^2}{2}+\dfrac{x^3}{3!}+o(x^3)}{x^2} =\dfrac{1}{2}+\dfrac{x}{6}+o(x) \qquad
\text{  donc }\qquad
\underset{x \to 0}{\lim} \left(\dfrac{e^x - (1+x)}{x^2}\right)=\dfrac{1}{2}
$\\

$
\dfrac{x^2}{\cos(x)-1}=\frac{x^2}{-\dfrac{x^2}{2}+o(x^3)} =\dfrac{-2}{1+o(x)} \qquad
\text{  donc }\qquad
 \underset{x \to 0}{\lim} \left(\dfrac{x^2}{\cos(x) - 1}\right)=-2
$\\


\[
\frac{(e^x - 1)\sin(x)}{x^2}=\frac{\left(\dfrac{x^2}{2}+\dfrac{x^3}{3!}+o(x^3)\right)\times\left(x-\dfrac{x^3}{3!}+o(x^3) \right)}{x^2} =\frac{\dfrac{x^3}{2}+o(x^3) }{x^2}=\dfrac{x}{2}+o(x)
\text{  donc }
\lim_{x \to 0} \left(\frac{(e^x - 1)\sin(x)}{x^2}\right)=0
\]



}


 \hfill\break
\hrule
\hfill\break

\underline {\ding{110}\, \bf{Exercice 2 : Inversibilité et expression de l'inverse via un polynôme annulateur (8 points)}}\\

    Soit la matrice $P = \begin{pmatrix} 1 & 1 \\ 0 & 2 \end{pmatrix}$.
    \begin{enumerate}
        \item Calculer $P^2$.\\
        \item Vérifier que $P^2 - 3P + 2I_2 = 0_2$, où $I_2$ est la matrice identité d'ordre 2 et $0_2$ la matrice nulle d'ordre $2$.\\
        \item À partir de cette relation, déduisez que la matrice $P$ est inversible et exprimez son inverse $P^{-1}$ en fonction de $P$ et $I_2$.\\
        \item Utiliser cette expression pour calculer $P^{-1}$.\\
    \end{enumerate}




\hfill\break
\hrule
\hfill\break

\underline {\ding{110}\, \bf{Exercice 3 : Déterminant et condition d'inversibilité (3 points)}}\\

    Sans effectuer de calculs explicites, expliquer pourquoi les déterminants suivants sont nuls.\\ Préciser la propriété utilisée.\\
    
$$A = \begin{vmatrix} 1 & 2 & 3 \\ 4 & 5 & 6 \\ 1 & 2 & 3 \end{vmatrix}\qquad
        B = \begin{vmatrix} 1 & 2 & 4 \\ 2 & 4 & 8 \\ 3 & 5 & 7 \end{vmatrix}\qquad C = \begin{vmatrix} 1 & 2 & 3 \\ 0 & 0 & 0 \\ 4 & 5 & 6 \end{vmatrix}$$



\hfill\break
\hrule
\hfill\break

\underline {\ding{110}\, \bf{Exercice 4 : Déterminant et condition d'inversibilité (3 points)}}\\

    Pour quelle(s) valeur(s) du réel $m$ la matrice suivante est-elle inversible ?
    $$H = \begin{pmatrix} 1 & 1 & 1 \\ 1 & m & 1 \\ 1 & 1 & m \end{pmatrix}$$




\hfill\break
\hrule
\hfill\break
\newpage

\underline {\ding{110}\, \bf{Exercice 5 : Étude de la Conchoïde de Nicomède (20 points)}}\\



\subsection*{Partie 1 : Forme Paramétrique}

La conchoïde de Nicomède est définie par les équations paramétriques :
$$
\begin{cases}
x(t) = \frac{2}{\cos(t)} + \cos(t) \\
y(t) = 2 \tan(t) + \sin(t)
\end{cases}
$$

\begin{enumerate}
    \item      \begin{enumerate}
    \item Déterminer le domaine de définition de $t$;.\\
    \item \'Etudier les symétries de la courbe.\\
    \end{enumerate}
    \item 
        \begin{enumerate}
    \item Calculer les dérivées $x'(t)$ et $y'(t)$.\\
    \item En déduire les coordonnées d'un vecteur tangent $\vec{V}(t)$ à la courbe à un point régulier $M_t$ de coordonnées $\big(x(t),y(t)\big)$.\\
    \end{enumerate}
    \item         \begin{enumerate}
    \item
    Déterminer les points où la tangente est horizontale ou verticale.\\ 
    \item Donner les coordonnées cartésiennes de ces points.\\
     \end{enumerate}
    
     %\begin{enumerate}
     \item Effectuer un développement limité de $x(t)$ et $y(t)$ au voisinage de $t=0$ à l'ordre $3$.\\ 
     %\item En déduire la nature du point correspondant à $t=0$.\\
    % \end{enumerate}
    \item
    \begin{enumerate}
    \item Étudier les limites de $x(t)$ et $y(t)$ lorsque $t$ approche $\frac{\pi}{2}$ et $-\frac{\pi}{2}$.\\ 
  
    \item Que pouvez-vous en déduire?
   %  Donner l'équation des asymptotes à la courbe.\\
\end{enumerate}
\end{enumerate}

\subsection*{Partie 2 : Forme Polaire}

La conchoïde de Nicomède est définie en coordonnées polaires par l'équation :
$$r(\theta) = \frac{2}{\cos \theta} + 1$$

\begin{enumerate}
    \item Déterminer le domaine de définition de $\theta$ et étudier les symétries de la courbe.\\
    \item Convertir l'équation polaire en coordonnées cartésiennes $x(\theta)$ et $y(\theta)$.\\
    \item 
        \begin{enumerate}
    \item Étudier les limites de $x(\theta)$ et $y(\theta)$ lorsque $\theta$ approche $\frac{\pi}{2}$ et $-\frac{\pi}{2}$.\\
    \item  Donner l'équation des asymptotes.\\
    \end{enumerate}
    \item Calculer la dérivée $\frac{dr}{d\theta}$.\\
    \item Déterminer l'angle $\alpha$ de la tangente à la courbe en un point de coordonnées $(r, \theta)$.\\
    \item
            \begin{enumerate}
    \item Effectuer un développement limité de $r(\theta)$ au voisinage de $\theta=0$ à l'ordre $2$.\\
    \item  En déduire la nature du point correspondant à $\theta=0$.\\
\end{enumerate}
\end{enumerate}


{\color{blue}\underline{\bf Solution}

\subsection*{Partie 1 : Forme Paramétrique}

\begin{enumerate}
    \item
    \begin{enumerate}
        \item Domaine de définition de $x$ et $y$:\\
         $\cos(t) \neq 0$ si et seulement si $t \neq \frac{\pi}{2} + k\pi$, $k \in \mathbb{Z}$.\\
        Donc, le domaine de définition est $D_f=\mathbb{R} \setminus \left\{ \frac{\pi}{2} + k\pi \mid k \in \mathbb{Z} \right\}$.

        \item 
        L'ensemble de définition est symétrique par rapport à $0$.\\
        On a :
        \[
        \forall  t \in D_f,\quad
        x(-t) = \frac{2}{\cos(-t)} + \cos(-t) = \frac{2}{\cos t} + \cos t = x(t)
        \]
        donc $x$ est paire.\\
        De même :
        \[ \forall  t \in D_f,\quad
        y(-t) = 2 \tan(-t) + \sin(-t) = -2 \tan t - \sin t = -y(t)
        \]
        donc $y$ est impaire.\\
        Par conséquent, la courbe est symétrique par rapport à l’axe des abscisses.

    \end{enumerate}

    \item
    \begin{enumerate}
        \item Les fonctions $x$ et $y$ sont dérivables sur l'ensemble de définition comme somme de fonctions dérivables.:
        \[
        x'(t) = \frac{2 \sin t}{\cos^2 t} - \sin t = \sin t \left( \frac{2}{\cos^2 t} - 1 \right)
        \qquad
        y'(t) =  \frac{2}{\cos^2 t} + \cos t
        \]
%\begin{tikzpicture}
%  \tkzTabInit[lgt=2,espcl=3]
%  {$t$ /1, $x'(t)$ /1, $x(t)$ /2}
%  {$-\frac{\pi}{2}$, $-\frac{\pi}{4}$, $0$, $\frac{\pi}{4}$, $\frac{\pi}{2}$}
%  \tkzTabLine{, -, z, +, z, -}
%  \tkzTabVar{-/$-\infty$, +/, -/$3$, +/, -/$+\infty$}
%\end{tikzpicture}
%
%\begin{tikzpicture}
%  \tkzTabInit[lgt=2,espcl=3]
%  {$t$ /1, $y'(t)$ /1, $y(t)$ /2}
%  {$-\frac{\pi}{2}$, $-\frac{\pi}{4}$, $0$, $\frac{\pi}{4}$, $\frac{\pi}{2}$}
%  \tkzTabLine{, +, z, +, z, +}
%  \tkzTabVar{-/$-\infty$, +/, +/$0$, +/, +/$+\infty$}
%\end{tikzpicture}

        \item Un vecteur tangent à la courbe au point régulier $M_t = (x(t), y(t))$ est donné par :
        \[
        \vec{V}(t) = \left( x'(t), y'(t) \right) =\left(\sin t \left( \frac{2}{\cos^2 t} - 1 \right),   \frac{2}{\cos^2 t} + \cos t \right)
        \]

    \end{enumerate}

    \item
    \begin{enumerate}
        \item La tangente est horizontale lorsque $y'(t) = 0$ :
        \[
        y'(t) = \frac{2}{\cos^2 t} + \cos t = 0 \iff \cos t=-2^{\frac{1}{3}}
        \]
     Or $-2^{\frac{1}{3}}<-1$. Il n'y a pas de tangente horizontale.

        La tangente est verticale lorsque $x'(t) = 0$, soit :
        \[
        \sin t \left( \frac{2}{\cos^2 t} - 1 \right) = 0
        \]
        Deux cas sont possibles :
        \begin{itemize}
            \item $\sin t = 0$ si et seulement si $t = k\pi$ avec $k \in \mathbb{Z}$.
            \item $\frac{2}{\cos^2 t} - 1 = 0$ si et seulement si $\cos^2 t = 2$, ce qui est impossible.
        \end{itemize}
        Les tangentes sont donc verticales pour $t = k\pi$ avec $k \in \mathbb{Z}$.

        \item Pour $t = 0$ :
        \[
        x(0) = \frac{2}{1} + 1 = 3,\quad y(0) = 0 + 0 = 0
        \]
        Pour $t = \pi$ :
        \[
        x(\pi) = \frac{2}{-1} -1 = -3,\quad y(\pi) = 2 \cdot 0 + \sin(\pi) = 0
        \]
        Les coordonnées des points où la tangente est verticale sont $(3, 0)$ et $(-3, 0)$

    \end{enumerate}

    \item
   % \begin{enumerate}
      %  \item Développement limité à l’ordre $3$ en $t = 0$ :

        \[
        \cos t = 1 - \frac{t^2}{2} + \frac{t^4}{24} + o(t^4), \quad \frac{1}{\cos t} = 1 + \frac{t^2}{2} + \frac{5t^4}{24} + o(t^4)
        \]
        \[
        x(t) = 2 \left(1 + \frac{t^2}{2}\right) + \left(1 - \frac{t^2}{2}\right) + o(t^2) = 3 + t^2 + o(t^2)
        \]
        \[
        \tan t = t + \frac{t^3}{3} + o(t^3), \quad \sin t = t - \frac{t^3}{6} + o(t^3)
        \]
        \[
        y(t) = 2t + \frac{2t^3}{3} + t - \frac{t^3}{6} + o(t^3) = 3t + \frac{t^3}{2} + o(t^3)
        \]

        On obtient :
        \[
        x(t) = 3 + t^2 + o(t^2),\quad y(t) = 3t + \frac{t^3}{2} + o(t^3)
        \]
        %C’est une parabole horizontale issue du point $(3,0)$ avec tangente verticale. Le point est un point régulier, non stationnaire, avec tangente verticale.

   % \end{enumerate}

    \item
    \begin{enumerate}
        \item Étude des limites quand $t \to \pm \frac{\pi}{2}$ :

    \[
\lim_{t \to \frac{\pi}{2}^{-}} \cos t = 0^{+}, \quad \lim_{t \to \frac{\pi}{2}^{-}} \frac{1}{\cos t} = +\infty, \quad \lim_{t \to \frac{\pi}{2}^{-}} \tan t = +\infty
\]
\[
\lim_{t \to \frac{\pi}{2}^{-}} x(t) = \lim_{t \to \frac{\pi}{2}^{-}} \left( \frac{2}{\cos t} + \cos t \right) = +\infty
\]
\[
\lim_{t \to \frac{\pi}{2}^{-}} y(t) = \lim_{t \to \frac{\pi}{2}^{-}} \left( 2 \tan t + \sin t \right) = +\infty
\]

\[
\lim_{t \to -\frac{\pi}{2}^{+}} \cos t = 0^{+}, \quad \lim_{t \to -\frac{\pi}{2}^{+}} \frac{1}{\cos t} = +\infty, \quad \lim_{t \to -\frac{\pi}{2}^{+}} \tan t = -\infty
\]
\[
\lim_{t \to -\frac{\pi}{2}^{+}} x(t) = \lim_{t \to -\frac{\pi}{2}^{+}} \left( \frac{2}{\cos t} + \cos t \right) = +\infty
\]
\[
\lim_{t \to -\frac{\pi}{2}^{+}} y(t) = \lim_{t \to -\frac{\pi}{2}^{+}} \left( 2 \tan t + \sin t \right) = -\infty
\]


        \item Les deux coordonnées tendent vers l’infini donc la courbe admet des branches infinies.
    \end{enumerate}
\end{enumerate}

\subsection*{Partie 2 : Forme Polaire}


\begin{enumerate}
    \item Domaine : $\cos \theta \ne 0$ D'où $\theta \in \mathbb{R} \setminus \left\{ \frac{\pi}{2} + k\pi \mid k \in \mathbb{Z} \right\}$.\\
    Symétrie : $\cos(-\theta) = \cos \theta$ donc $r(-\theta) = r(\theta)$ : la courbe est symétrique par rapport à l’axe des abscisses.

    \item On utilise :
    \[
    x(\theta) = r(\theta) \cos \theta = \left( \frac{2}{\cos \theta} + 1 \right) \cos \theta = 2 + \cos \theta
    \]
    \[
    y(\theta) = r(\theta) \sin \theta = \left( \frac{2}{\cos \theta} + 1 \right) \sin \theta
    \]

    \item
    \begin{enumerate}
        \item \[
\lim_{\theta \to \frac{\pi}{2}^{-}} \cos \theta = 0^{+}, \quad \lim_{\theta \to \frac{\pi}{2}^{-}} r(\theta) = \lim_{\theta \to \frac{\pi}{2}^{-}} \left( \frac{2}{\cos \theta} + 1 \right) = +\infty
\]
\[
\lim_{\theta \to \frac{\pi}{2}^{-}} x(\theta) = \lim_{\theta \to \frac{\pi}{2}^{-}} \left( r(\theta) \cos \theta \right) = \lim_{\theta \to \frac{\pi}{2}^{-}} \left( \left( \frac{2}{\cos \theta} + 1 \right) \cos \theta \right) = 2 
\]
\[
\lim_{\theta \to \frac{\pi}{2}^{-}} y(\theta) = \lim_{\theta \to \frac{\pi}{2}^{-}} \left( r(\theta) \sin \theta \right) = \lim_{\theta \to \frac{\pi}{2}^{-}} \left( \left( \frac{2}{\cos \theta} + 1 \right) \sin \theta \right) = +\infty
\]


        \item Asymptote verticale d'équation $x = 2$.
    \end{enumerate}

    \item
    \[
    r'(\theta)  = \frac{2 \sin \theta}{\cos^2 \theta}
    \]

    \item
    En coordonnées polaires, l’angle $\alpha$ entre la tangente et le rayon vecteur est donné par :
    \[
    \tan \alpha = \frac{r'(\theta)}{r(\theta)}
    \]

    \item
    \begin{enumerate}
        \item Développement limité en $\theta = 0$ :
    \[
\cos \theta = 1 - \frac{\theta^2}{2} + o(\theta^2) \quad \text{(développement limité en 0 à l'ordre 2)}
\]
\[
 \frac{1}{1 - u}=1+u+u^2+o(u^2) \text{ si } u \to 0 \text{ avec }u=\dfrac{\theta ^2}{2}
 \]
\[
\frac{1}{\cos \theta} = \frac{1}{1 - \frac{\theta^2}{2} + o(\theta^2)} = 1 + \frac{\theta^2}{2} + o(\theta^2)
\]
\[
r(\theta) = \frac{2}{\cos \theta} + 1 = 2\left(1 + \frac{\theta^2}{2} + o(\theta^2)\right) + 1 
= 2 + \theta^2 + o(\theta^2) + 1 = 3 + \theta^2 + o(\theta^2)
\]

%\Rightarrow \text{Au voisinage de } \theta = 0,\quad r(\theta) = 3 + \theta^2 + o(\theta^2)


        \item Le point correspondant à $\theta = 0$ est à distance $r = 3$ sur l’axe des abscisses. La courbe présente ici un point régulier non singulier avec une tangente d’angle nul.
    \end{enumerate}
\end{enumerate}
%
%\begin{tikzpicture}[scale=0.25]
%    % Définition des axes
%    \draw[->] (-3,0) -- (7,0) node[right] {$x$};
%    \draw[->] (0,-7) -- (0,7) node[above] {$y$};
%
%    % Dessin de la conchoïde
%    % La conchoïde de Nicomède a des branches qui s'étendent à l'infini,
%    % il faut donc choisir un domaine pour 't' qui évite les asymptotes
%    % (quand cos(t) est proche de 0 ou tan(t) est indéfini).
%    % Nous allons tracer deux branches pour couvrir le comportement de la courbe.
%
%    % Première branche (pour t de -pi/2 + epsilon à pi/2 - epsilon)
%    \draw[blue, thick, variable=\t, domain=-1.5:1.5, samples=200]
%        plot ({2/cos(\t r) + cos(\t r)}, {2*tan(\t r) + sin(\t r)});
%
%    % Deuxième branche (pour t de pi/2 + epsilon à 3*pi/2 - epsilon)
%    % On décale le domaine pour éviter les singularités
%    \draw[blue, thick, variable=\t, domain=1.65:4.6, samples=200]
%        plot ({2/cos(\t r) + cos(\t r)}, {2*tan(\t r) + sin(\t r)});
%
%    % Asymptote (ligne verticale x=2)
%    \draw[red, dashed] (2,-7) -- (2,7) node[below right] {$x=2$};
%
%    % Légende (facultatif)
%    \node[below left, align=left] at (-2,-6) {
%        \textbf{Conchoïde de Nicomède} \\
%        $x(t) = \frac{2}{\cos(t)} + \cos(t)$ \\
%        $y(t) = 2 \tan(t) + \sin(t)$
%    };
%
%\end{tikzpicture}
}

\end{document}
