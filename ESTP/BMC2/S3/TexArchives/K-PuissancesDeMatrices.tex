








\section{La conjecture des puissances}

Soit $n \in \mathbb{N}$.\\
Pour toute matrice $A \in \mathcal{M}_n(\mathbb{R})$, on peut calculer les premières puissances de $A$, puis conjecturer la « forme » de la matrice $A^n$ en toute généralité sur $n$. Il faut ensuite démontrer cette conjecture par récurrence.

%\begin{Ex}
Une conjecture importante à connaître est celle associée à la matrice suivante :

\[
J = 
\begin{pmatrix}
1 & 1 & 1\\
1 & 1 & 1\\
1 & 1 & 1
\end{pmatrix}
\]

Avec cette matrice, on obtient :

\[
J^2 = 
\begin{pmatrix}
3 & 3 & 3\\
3 & 3 & 3\\
3 & 3 & 3
\end{pmatrix}
\quad \text{et} \quad
J^3 = 
\begin{pmatrix}
9 & 9 & 9\\
9 & 9 & 9\\
9 & 9 & 9
\end{pmatrix}
\]

La conjecture à faire est : 
\[
\forall k \in \mathbb{N}^*, J^k = 3^{k-1} J.
\]

\textbf{Démonstration par récurrence :}\\
Montrons par récurrence que $\forall k \in \mathbb{N}^*, J^k = 3^{k-1} J$.\\

\textbf{Initialisation :}\\
Pour $k = 1$, $J^1 = J$, donc la propriété est vraie au rang $k=1$.\\

\textbf{Hérédité :}\\
Supposons que $\forall k \geq 1$, $J^k = 3^{k-1} J$. Montrons que cela reste vrai pour $k+1$.\\

\[
J^{k+1} = J^k \cdot J = (3^{k-1} J) \cdot J \quad \text{(par hypothèse de récurrence)} 
\]
\[
= 3^{k-1}
\begin{pmatrix}
1 & 1 & 1\\
1 & 1 & 1\\
1 & 1 & 1
\end{pmatrix}
\cdot 
\begin{pmatrix}
1 & 1 & 1\\
1 & 1 & 1\\
1 & 1 & 1
\end{pmatrix}
= 3^{k-1}
\begin{pmatrix}
3 & 3 & 3\\
3 & 3 & 3\\
3 & 3 & 3
\end{pmatrix}
\]
\[
= 3^k 
\begin{pmatrix}
1 & 1 & 1\\
1 & 1 & 1\\
1 & 1 & 1
\end{pmatrix} = 3^k J.
\]

Ainsi, la propriété est vraie au rang $k+1$. Par le principe de récurrence, on conclut que la propriété est vraie $\forall k \in \mathbb{N}^*$.
%\end{Ex}

\section{Utilisation des relations entre certaines matrices}


\textbf{Exemple :} Soit $A = 
\begin{pmatrix}
4 & -3\\
3 & 2
\end{pmatrix}
$.\\

Posons $B = A - I_2$, alors $B = 
\begin{pmatrix}
3 & -3\\
3 & -3
\end{pmatrix}
$ et $B^2 = 
\begin{pmatrix}
0 & 0\\
0 & 0
\end{pmatrix}
$.

Par conséquent, 
\[
A^n = (B + I_2)^n.
\]

Développons cette expression en utilisant le binôme de Newton :
\[
A^n = \sum_{k=0}^n \binom{n}{k} B^k I_2^{n-k}.
\]

Puisque $B^2 = 0$, les termes pour $k \geq 2$ sont nuls. On en déduit :
\[
A^n = B + I_2.
\]

\section{Les matrices diagonales et nilpotentes}

Les matrices diagonales sont simples à manipuler. Si $A$ est diagonale :
\[
A = 
\begin{pmatrix}
a & 0 & 0\\
0 & b & 0\\
0 & 0 & c
\end{pmatrix}
\quad \Rightarrow \quad
A^n = 
\begin{pmatrix}
a^n & 0 & 0\\
0 & b^n & 0\\
0 & 0 & c^n
\end{pmatrix}.
\]

Les matrices nilpotentes $N$ sont telles qu'il existe un $k$ tel que $N^k = 0$. Pour $n \geq k$, $N^n = 0$.

\section{Exercices}

\exo{\textbf{Puissance $n$-ième, par récurrence}\\
\vskip0.2cm
On considère les matrices suivantes : 
$$A=\left(\begin{array}{cc}
1&-1\\
-1&1\\
\end{array}\right),\ B=\left(\begin{array}{cc}
1&1\\
0&2\\
\end{array}\right).$$
Calculer $A^2$, $A^3$. En déduire la valeur de $A^n$ pour tout $n\geq 1$. Répondre aux mêmes questions pour $B$.\\
}


% Exercice 914


\exo{\textbf{Puissance $n$-ième - avec la formule du binôme}\\
\vskip0.2cm
Soit $$A=\left(
\begin{array}{ccc}
1&1&0\\
0&1&1\\
0&0&1
\end{array}\right),\quad
I=\left(
\begin{array}{ccc}
1&0&0\\
0&1&0\\
0&0&1
\end{array}\right)\textrm{ et }
B=A-I.$$
Calculer $B^n$ pour tout $n\in\mathbb N$. En déduire $A^n$.\\
}

% Exercice 2387


\exo{\textbf{Puissance $k$-ième sans division euclidienne}\\
\vskip0.2cm
Soit $U$ la matrice $$U=\left(\begin{array}{cccc}
0&1&1&1\\
1&0&1&1\\
1&1&0&1\\
1&1&1&0
\end{array}\right).$$
\begin{enumerate}
\item Calculer $U^2$ et en déduire une relation simple liant $U^2$, $U$ et $I_4$.
\item Soit $(\alpha_k)$ et $(\beta_k)$ les suites définies par $\alpha_0=1$, $\beta_0=0$, $\alpha_{k+1}=3\beta_k$, $\beta_{k+1}=\alpha_k+2\beta_k$. Démontrer que, pour tout $k\in\mathbb N$, on a 
$$U^k=\left(
\begin{array}{cccc}
\alpha_k&\beta_k&\beta_k&\beta_k\\
\beta_k&\alpha_k&\beta_k&\beta_k\\
\beta_k&\beta_k&\alpha_k&\beta_k\\
\beta_k&\beta_k&\beta_k&\alpha_k
\end{array}\right).$$
\item Démontrer que, pour tout $k\in\mathbb N$, on a $\beta_{k+2}=2\beta_{k+1}+3\beta_k$.
\item En déduire que, pour tout $k\in\mathbb N$, 
$\beta_k=\frac{3^k-(-1)^k}{4}$ et $\alpha_k=\frac{3^k+3(-1)^k}{4}$.\\ 
\end{enumerate}
}


% Exercice 915


\exo{\textbf{Puissance $n$-ième - avec un polynôme annulateur}\\
\vskip0.2cm
\begin{enumerate}
\item Pour $n\geq 2$, déterminer le reste de la division euclidienne de $X^n$ par $X^2-3X+2$.
\item Soit  $A=\begin{pmatrix} 
0&1&-1\\
-1&2&-1\\
1&-1&2
\end{pmatrix}$. Déduire de la question précédente la valeur de $A^n$, pour $n\geq 2$.\\
\end{enumerate}

}
% Exercice 2388


\exo{\textbf{ Puissance $k$-ième, avec polynôme annulateur}\\
\vskip0.2cm
Soit $U$ la matrice $$U=\left(\begin{array}{cccc}
0&1&1&1\\
1&0&1&1\\
1&1&0&1\\
1&1&1&0
\end{array}\right).$$

\begin{enumerate}
\item Déterminer une relation simple liant $I_4,U$ et $U^2$.
\item En déduire, pour $k\geq 0$, la valeur de $U^k$.\\
\end{enumerate}
}
