\section*{Les fonctions périodiques}

\subsection{Les fonctions complexes périodiques}

Les fonctions réelles suivantes sont-elles périodiques et si oui, quelle est leur période ?

\ifthenelse{\boolean{showSolutions}}{}
{\begin{multicols}{2}}
\begin{enumerate}
    \item $\displaystyle \cos(x)$
    \item $\displaystyle \sin(2\pi x)$
    \item $\displaystyle \cos(x/2)$
    \item $\displaystyle \sin(2x) + \cos(3x)$
    \item $\displaystyle \sin(nx)$, $n$ est un entier naturel non nul
    \item $\displaystyle \cos \left(\frac{3 x}{2}-\frac{\pi}{4}\right)$
    \item $\displaystyle x-\lfloor x\rfloor$
\end{enumerate}
\ifthenelse{\boolean{showSolutions}}{}{
\end{multicols}
}

Les fonctions complexes suivantes sont-elles périodiques et si oui, quelle est leur période ?

\ifthenelse{\boolean{showSolutions}}{}
{\begin{multicols}{2}}
\begin{enumerate}
    \item $\displaystyle e^{ix}$
    \item $\displaystyle e^{2ix}$
    \item $\displaystyle e^{ix/2\pi}$
    \item $\displaystyle e^{2i\pi x/T}$, $T$ est un réel strictement positif
    \item $\displaystyle e^{inx} + e^{ipx}$
\end{enumerate}
\ifthenelse{\boolean{showSolutions}}{}{
\end{multicols}
}


\section*{Produit scalaire réel}

\subsection{Définition}
On appelle produit scalaire sur un espace vectoriel $E$ une application 
$$\langle \cdot, \cdot \rangle : E \times E \to \mathbb{R}$$
telle que :
\begin{multicols}{2}
\begin{itemize}
    \item[*] symétrie : $\langle u, v \rangle = \langle v, u \rangle$ 
    \item[*] linéarité à gauche : $\langle \lambda u + v, w \rangle = \lambda \langle u, w \rangle + \langle v, w \rangle$
    \item[*] positivité : $\langle u, u \rangle \geq 0$
    \item[*] définie positivité : $\langle u, u \rangle = 0 \iff u = 0$
\end{itemize}
\end{multicols}


\vspace{1em}

\subsection{Dans $\mathbb{R}^3$}

On se place dans $\mathbb{R}^3$, qu'on munit de la base 
$$e_1 = (1,2,1), \qquad e_2 = (2,1,-4), \qquad e_3 = (-3,2,-1)$$

\begin{enumerate}
\item La famille est-elle orthogonale ? 
\item Est-elle orthonormée ? Si non, définissez une base $(f_1,f_2,f_3)$ orthonormée à partir de la famille $(e_1,e_2,e_3)$. 
\end{enumerate}

Soit $u$ un vecteur de $\mathbb{R}^3$, on note $u_i$ ses coordonnées dans la base orthonormée $(f_1,f_2,f_3)$. Cela signifie que 
$$u = u_1 f_1 + u_2 f_2 + u_3 f_3$$

Déterminer les coordonnées de $u = (1,0,1)$ dans la base $(f_1,f_2,f_3)$.

\vspace{1em}

\subsection{Dans $\mathbb{R}[X]$}

\begin{itemize}
    \item Quelle est la dimension de $\mathbb{R}[X]$ ?
\end{itemize}
La famille $(1,X,X^2,X^3, \cdots )$ est appelée base hilbertienne de $\mathbb{R}[X]$ : tout élément de $\mathbb{R}[X]$ peut s'écrire comme une combinaison linéaire finie de vecteurs de cette famille.

On munit cet espace du produit scalaire : 
$$ \langle P, Q \rangle = \int_{0}^{1} P(x) Q(x) dx $$

\begin{itemize}
    \item Montrer que c'est bien un produit scalaire en vérifiant les propriétés ci-dessus.
    \item La famille $(1,X,X^2,X^3, \cdots )$ est-elle orthogonale ? Est-elle orthonormée ?
    \item Comment trouver $a, b, c$ tels que la famille $(1, X-a, X^2-bX-c)$ soit orthogonale ?
    \item Quelles sont les coordonnées de $P = 1+2X+3X^2$ dans la base $(1, X, X^2, \cdots)$ ?
    \item Peut-on retrouver ces coordonnées avec le produit scalaire comme dans l'exercice précédent ?
\end{itemize}
\vspace{1em}

\section*{Produit scalaire complexe}
\subsection{Définition}
On appelle produit scalaire sur un espace vectoriel $E$ une application 
$$\langle \cdot, \cdot \rangle : E \times E \to \mathbb{R}$$
telle que :
\begin{multicols}{2}
\begin{itemize}
    \item[*] symétrie conjuguée : $\langle u, v \rangle = \overline{\langle v, u \rangle}$
    \item[*] linéarité à gauche : $\langle \lambda u + v, w \rangle = \lambda \langle u, w \rangle + \langle v, w \rangle$
    \item[*] positivité : $\langle u, u \rangle \geq 0$
    \item[*] définie positivité : $\langle u, u \rangle = 0 \iff u = 0$
\end{itemize}
\end{multicols}


\subsection{Dans l'espace des fonctions complexes $2\pi$-périodiques}

On définit le produit scalaire :
$$
\langle f, g \rangle = \int_{0}^{2\pi} f(x) \overline{g(x)} dx
$$

Montrer que c'est un produit scalaire.

Montrer que la famille $(e^{inx})_{n \in \mathbb{Z}}$ est orthonormée.

Les coefficients de Fourier d'une fonction $f$ sont les coordonnées de $f$ dans la base $(e^{inx})_{n \in \mathbb{Z}}$.

Déterminer les coefficients de Fourier des fonctions suivantes :
\ifthenelse{\boolean{showSolutions}}{}
{\begin{multicols}{2}}
\begin{enumerate}
\item $\displaystyle \cos(x)$
\item $\displaystyle \sin(2\pi x)$
\item $\displaystyle \cos(x/2)$
\item $\displaystyle \sin(2x) + \cos(3x)$
\item $\displaystyle \exp^{-x}$ sur l'intervalle $[0, 2\pi]$
\end{enumerate}
\ifthenelse{\boolean{showSolutions}}{}{
\end{multicols}
}
