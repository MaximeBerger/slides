\section{Les fonctions périodiques}

\subsection{Les fonctions complexes périodiques}

Les fonctions réelles suivantes sont-elles périodiques et si oui, quelle est leur période ?

\begin{enumerate}
    \item $f(x) = \cos(x)$
    \item $f(x) = \sin(2\pi x)$
    \item $f(x) = \cos(x/2)$
    \item $f(x) = \sin(2x) + \cos(3x)$
    \item $f(x) = \sin(nx) + \cos(px)$
\end{enumerate}
Les fonctions complexes suivantes sont-elles périodiques et si oui, quelle est leur période ?

\begin{enumerate}
    \item $f(x) = e^{ix}$
    \item $f(x) = e^{2ix}$
    \item $f(x) = e^{ix/2\pi}$
    \item $f(x) = e^{inx} + e^{ipx}$
\end{enumerate}


\section{Produit scalaire}

\subsection{Définition}
On appelle produit scalaire sur un espace vectoriel $E$ une application 
$$\langle \cdot, \cdot \rangle : E \times E \to \mathbb{R}$$
telle que :
\begin{itemize}
    \item $\langle u, v \rangle = \langle v, u \rangle$
    \item $\langle u, v + \lambda w \rangle = \langle u, v \rangle + \lambda \langle u, w \rangle$
    \item $\langle u, u \rangle \geq 0$
    \item $\langle u, u \rangle = 0 \iff u = 0$
\end{itemize}



\subsection{Dans $\mathbb{R}^3$}

On se place dans $\mathbb{R}^3$, qu'on muni de la base 
$$e_1 = (1,2,1), \qquad e_2 = (2,1,-4), \qquad e_3 = (-3,2,-1)$$

\begin{enumerate}
\item La famille est-elle orthogonale ? 
\item Est-elle orthonormée ? Si non, définissez une base $(f_1,f_2,f_3)$ orthonormée à partir de la famille $(e_1,e_2,e_3)$. 
\end{enumerate}

Soit $u$ un vecteur de $\mathbb{R}^3$, on note $u_i$ ses coordonnées dans la base orthonormée $(f_1,f_2,f_3)$. Cela signifie que 
$$u = u_1 f_1 + u_2 f_2 + u_3 f_3$$

Déterminer les coordonnées de $u = (1,0,1)$ dans la base $(f_1,f_2,f_3)$.

\vspace{2em}

\subsection{Dans $\mathbb{R}_n[X]$}

On se place dans $\mathbb{R}_n[X]$, qu'on muni de la base 
$$e_1 = (1, -1, 0, \dots, 0), \qquad e_2 = (0, 1, -1, 0, \dots, 0), \qquad \dots, \qquad e_{n-1} = (0, \dots, 0, 1, -1), \qquad e_n = (1, \dots, 1, 1)$$
\subsection{Dans $\mathbb{R}[X]$}



\subsection{Dans l'espace des fonctions complexes $2\pi$-périodiques}

On définit le produit scalaire :
$$
\langle f, g \rangle = \int_{0}^{2\pi} f(x) \overline{g(x)} dx
$$

Montrer que c'est un produit scalaire.