
\newcommand{\mtn}{\mathbb{N}}
\newcommand{\mtns}{\mathbb{N}^*}
\newcommand{\mtz}{\mathbb{Z}}
\newcommand{\mtr}{\mathbb{R}}
\newcommand{\mtk}{\mathbb{K}}
\newcommand{\mtq}{\mathbb{Q}}
\newcommand{\mtc}{\mathbb{C}}
\newcommand{\mch}{\mathcal{H}}
\newcommand{\mcp}{\mathcal{P}}
\newcommand{\mcb}{\mathcal{B}}
\newcommand{\mcl}{\mathcal{L}}
\newcommand{\mcm}{\mathcal{M}}
\newcommand{\mcc}{\mathcal{C}}
\newcommand{\mcmn}{\mathcal{M}}
\newcommand{\mcmnr}{\mathcal{M}_n(\mtr)}
\newcommand{\mcmnk}{\mathcal{M}_n(\mtk)}
\newcommand{\mcsn}{\mathcal{S}_n}
\newcommand{\mcs}{\mathcal{S}}
\newcommand{\mcd}{\mathcal{D}}
\newcommand{\mcsns}{\mathcal{S}_n^{++}}
\newcommand{\glnk}{GL_n(\mtk)}
\newcommand{\mnr}{\mathcal{M}_n(\mtr)}
\newcommand{\veps}{\varepsilon}
\newcommand{\mcu}{\mathcal{U}}
\newcommand{\mcun}{\mcu_n}
\newcommand{\dis}{\displaystyle}
\newcommand{\croouv}{[\![}
\newcommand{\crofer}{]\!]}
\newcommand{\rab}{\mathcal{R}(a,b)}
\newcommand{\pss}[2]{\langle #1,#2\rangle}
 %Document 





% Exercice 3125

\vspace{1em}

\subsection{Primitives usuelles}

Déterminer toutes les primitives des fonctions suivantes, sur un intervalle bien choisi : 
$$\begin{array}{lll}
\displaystyle f_1(x)=5x^3-3x+7&\displaystyle f_2(x)=2\cos(x)-3\sin(x)&\displaystyle f_3(x)=10-3e^x+x\\
\displaystyle f_4(x)=\frac{5}{\sqrt x}+\frac 4x+\frac{2}{x^2}+\frac{2}{x^3}&\displaystyle f_5(x)=\frac{x+5}{x^2}&\displaystyle f_6(x)=\frac{x^2}{5}+\frac 1{6}\\
\end{array}$$



% Exercice 3126

\vspace{1em}
\subsection{Primitives usuelles}

Déterminer toutes les primitives des fonctions suivantes sur un intervalle bien choisi : 
$$\begin{array}{lll}
\displaystyle f_1(x)=e^{4x}&\displaystyle f_2(x)=e^{4x+3}& \displaystyle f_3(x)=\sin(2x)\\
\displaystyle f_4(x)=\cos\left(3x+\frac\pi 3\right)&
\displaystyle f_5(x)=(2x+1)^2&\displaystyle f_6(x)=\frac{3}{\sqrt{5x+1}}.
\end{array}$$



% Exercice 2600


\vspace{1em}
\subsection{Reconnaissance de formes}

Déterminer toutes les primitives des fonctions suivantes : 
$$
\begin{array}{lllll}
\displaystyle f(x)=\frac{x}{1+x^2}&\quad&\displaystyle g(x)=\frac{e^{3x}}{1+e^{3x}}&\quad&
\displaystyle h(x)=\frac{\ln x}{x}\\
\displaystyle k(x)=\cos(x)\sin^2(x)&\quad&l(x)=\frac{1}{x\ln x}&\quad&m(x)=3x\sqrt{1+x^2}.
\end{array}
$$


% Exercice 431


\vspace{1em}
\subsection{Reconnaissance de formes}

Déterminer une primitive des fonctions suivantes sur l'intervalle considéré :
\begin{multicols}{2}
    \begin{enumerate}[itemsep = 0.5em]
\item $f(x)=(3x-1)(3x^2-2x+3)^3,\ I=\mathbb{R}$
\item $f(x)=\frac{1-x^2}{(x^3-3x+2)^3},\ I=]-\infty,-2[$
\item $f(x)=\frac{(x-1)}{\sqrt{x(x-2)}},\ I=]-\infty,0[$
\item $f(x)=\frac{1}{x\ln(x^2)},\ I=]1,+\infty[.$
    \end{enumerate}
\end{multicols}


% Exercice 397


\vspace{1em}
\subsection{Intégration par parties - Niveau 1}

Calculer les intégrales suivantes :
$$\mathbf{1.}\quad I=\int_0^1 xe^xdx\quad\quad\mathbf{2.}\quad J=\int_1^e x^2\ln xdx$$


% Exercice 398


\vspace{1em}
\subsection{Intégration par parties - Niveau 2}

Déterminer une primitive des fonctions suivantes :
$$\mathbf{1.}\quad x\mapsto\arctan(x)\quad\quad\mathbf{2.}\quad x\mapsto (\ln x)^2\quad\quad\mathbf{3.} x\mapsto \sin(\ln x).$$


% Exercice 3492


\vspace{1em}
\subsection{Intégration par parties en boucle}

Calculer les intégrales suivantes :
$$1.\ \int_1^2 \frac{\ln(x)}{x}dx\quad\quad 2.\ \int_0^\pi e^x\sin(x)dx.$$


% Exercice 393


\vspace{1em}
\subsection{Changements de variables - Niveau 1}

En effectuant le changement de variables demandé, calculer les intégrales suivantes :
\begin{enumerate}
\item $\displaystyle \int_1^4\frac{1-\sqrt t}{\sqrt t}dt$ en posant $x=\sqrt t$;
\item $\displaystyle \int_0^{\pi}\frac{\sin t}{1+\cos^2 t}dt$ en posant $x=\cos t$;
\item $\displaystyle \int_1^e \frac{dt}{2t\ln (t)+t}$ en posant $x=\ln t$.
\end{enumerate}


% Exercice 394


\vspace{1em}
\subsection{Changements de variables - Niveau 2}

En effectuant le changement de variables indiqué, calculer les intégrales suivantes :
\begin{enumerate}
\item  $\displaystyle \int_0^1\frac{dt}{1+e^t}$ en posant $x=e^t$;
\item $\displaystyle \int_1^3\frac{\sqrt t}{t+1}dt$ en posant $x=\sqrt t$;
\item $\displaystyle \int_{-1}^1 \sqrt{1-t^2}dt$ en posant $t=\sin\theta$.
\end{enumerate}


% Exercice 3154


\vspace{1em}
\subsection{Changement de variables - Recherche de primitives - Niveau 1}

En effectuant le changement de variables indiqué, déterminer une primitive des fonctions suivantes :
\begin{enumerate}
\item $\displaystyle x\mapsto \frac{x}{\sqrt{1+x}}$, en posant $u=\sqrt{1+x}$;
\item $\displaystyle x\mapsto \frac{1}{e^x+1}$, en posant $u=e^x$;
\item $\displaystyle x\mapsto \frac{1}{x+x(\ln x)^2}$, en posant $u=\ln x$.
\end{enumerate}


% Exercice 395


\vspace{1em}
\subsection{Changements de variables - Recherche de primitives - Niveau 2}

En effectuant un changement de variables, déterminer une primitive des fonctions suivantes :
\begin{enumerate}
\item $\displaystyle x\mapsto \cos(2\ln x)$;
\item $\displaystyle x\mapsto\cos(\sqrt x)$;
\item $\displaystyle x\mapsto \frac{e^x}{(3+e^x)\sqrt{e^x-1}}$.
\end{enumerate}


% Exercice 433


\vspace{1em}
\subsection{Primitive de fractions rationnelles}

Déterminer une primitive des fractions rationnelles suivantes :
$$
\begin{array}{lll}
\mathbf 1.\ f(x)=\frac{2x^2-3x+4}{(x-1)^2}\textrm{ sur }]1,+\infty[&\quad&\mathbf 2. f(x)=\frac{2x-1}{(x+1)^2}\textrm{ sur }]-1,+\infty[ \\
\mathbf 3.\ f(x)=\frac{x}{(x^2-4)^2}\textrm{ sur }]2,+\infty[&&\mathbf 4. f(x)=\frac{24x^3+18x^2+10x-9}{(3x-1)(2x+1)^2}\textrm{ sur }]-1/2,1/3[
\end{array}
$$


% Exercice 3152


\vspace{1em}
\subsection{Exponentielle * Polynôme}

Calculer les intégrales suivantes  :
$$\begin{array}{lcl}
\displaystyle \mathbf{1.}\int_{0}^2 (x+6)e^{2x}dx &\quad&\displaystyle \mathbf{2.} \int_0^1 e^x(2x^3+3x^2-x+1)dx
\end{array}$$


% Exercice 3153


\vspace{1em}
\subsection{Exponentielle * trigonométrique}

Calculer les intégrales suivantes  :
$$\begin{array}{lcl}
\displaystyle \mathbf{1.} \int_0^\pi e^x\sin(2x)dx&\quad&\displaystyle \mathbf{2.} \int_0^{2\pi}e^{-x}\sin^2 xdx\\
\end{array}$$


% Exercice 437


\vspace{1em}
\subsection{Exponentielle * polynôme * trigonométrique}

Calculer l'intégrale :
$$\int_0^\pi x^2e^x \cos xdx.$$


% Exercice 2496


\vspace{1em}
\subsection{Quelques primitives à savoir calculer!}

Déterminer une primitive des fonctions suivantes :
$$
\begin{array}{lcl}
\displaystyle \mathbf{1.}\quad x\mapsto \frac{1}{x^2+4}&\quad\quad&\displaystyle \mathbf{2.}\quad x\mapsto\frac{1}{x^2+4x+5}\\
\displaystyle \mathbf{3.}\quad x\mapsto \frac{1}{1-x^2}&&\displaystyle \mathbf{4.}\quad x\mapsto e^x(2x^3+3x^2-x+1)\\
\displaystyle \mathbf{5.}\quad x\mapsto\sin^3(x)&&\displaystyle \mathbf{6.}\quad x\mapsto \arctan(x)
\end{array}$$



% Exercice 165


% \vspace{1em}
% \subsection{Quelques décompositions en séries de Fourier}

% Déterminer les séries de Fourier (termes en sinus et cosinus) des fonctions suivantes :
% \begin{enumerate}
% \item $f$ $2\pi-$périodique, définie par $f(x)=x$ si $-\pi\leq x<\pi$.
% \item la fonction créneau : $f$ est $2\pi$-périodique, définie par $f(x)=1$ si $x\in[0,\pi[$, et $f(x)=-1$ si $x\in[-\pi,0[$.
% \item la fonction $L-$périodique, où $L>0$, définie par $f(x)=|x|$ si $x\in[-L/2,L/2]$.
% \end{enumerate}

\vspace{1em}

Pour les corrigés : http://www.bibmath.net/ressources/justeunefeuille.php?id=46200 

%Vous avez accès aux corrigés de cette feuille par l'url : https://www.bibmath.net/ressources/justeunefeuille.php?id=46200
