
\newcommand{\mtn}{\mathbb{N}}
\newcommand{\mtns}{\mathbb{N}^*}
\newcommand{\mtz}{\mathbb{Z}}
\newcommand{\mtr}{\mathbb{R}}
\newcommand{\mtk}{\mathbb{K}}
\newcommand{\mtq}{\mathbb{Q}}
\newcommand{\mtc}{\mathbb{C}}
\newcommand{\mch}{\mathcal{H}}
\newcommand{\mcp}{\mathcal{P}}
\newcommand{\mcb}{\mathcal{B}}
\newcommand{\mcl}{\mathcal{L}}
\newcommand{\mcm}{\mathcal{M}}
\newcommand{\mcc}{\mathcal{C}}
\newcommand{\mcmn}{\mathcal{M}}
\newcommand{\mcmnr}{\mathcal{M}_n(\mtr)}
\newcommand{\mcmnk}{\mathcal{M}_n(\mtk)}
\newcommand{\mcsn}{\mathcal{S}_n}
\newcommand{\mcs}{\mathcal{S}}
\newcommand{\mcd}{\mathcal{D}}
\newcommand{\mcsns}{\mathcal{S}_n^{++}}
\newcommand{\glnk}{GL_n(\mtk)}
\newcommand{\mnr}{\mathcal{M}_n(\mtr)}
\newcommand{\veps}{\varepsilon}
\newcommand{\mcu}{\mathcal{U}}
\newcommand{\mcun}{\mcu_n}
\newcommand{\dis}{\displaystyle}
\newcommand{\croouv}{[\![}
\newcommand{\crofer}{]\!]}
\newcommand{\rab}{\mathcal{R}(a,b)}
\newcommand{\pss}[2]{\langle #1,#2\rangle}
 %Document 





% Exercice 3125

\vspace{1em}

\subsection{Primitives usuelles}

Déterminer toutes les primitives des fonctions suivantes, sur un intervalle bien choisi : 
$$\begin{array}{lll}
\displaystyle f_1(x)=5x^3-3x+7&\displaystyle f_2(x)=2\cos(x)-3\sin(x)&\displaystyle f_3(x)=10-3e^x+x\\
\displaystyle f_4(x)=\frac{5}{\sqrt x}+\frac 4x+\frac{2}{x^2}+\frac{2}{x^3}&\displaystyle f_5(x)=\frac{x+5}{x^2}&\displaystyle f_6(x)=\frac{x^2}{5}+\frac 1{6}\\
\end{array}$$

\ifthenelse{\boolean{showSolutions}}{
    \vspace{2em}
    \begin{mdframed}
    \begin{enumerate}
    \item $f_1(x)=5x^3-3x+7$ sur $\mathbb{R}$ :
    $$F_1(x) = 5\frac{x^4}{4} - 3\frac{x^2}{2} + 7x + C = \frac{5x^4}{4} - \frac{3x^2}{2} + 7x + C$$
    
    \item $f_2(x)=2\cos(x)-3\sin(x)$ sur $\mathbb{R}$ :
    $$F_2(x) = 2\sin(x) - 3(-\cos(x)) + C = 2\sin(x) + 3\cos(x) + C$$
    
    \item $f_3(x)=10-3e^x+x$ sur $\mathbb{R}$ :
    $$F_3(x) = 10x - 3e^x + \frac{x^2}{2} + C$$
    
    \item $f_4(x)=\frac{5}{\sqrt x}+\frac 4x+\frac{2}{x^2}+\frac{2}{x^3}$ sur $]0,+\infty[$ :
    $$f_4(x) = 5x^{-1/2} + 4x^{-1} + 2x^{-2} + 2x^{-3}$$
    $$F_4(x) = 5\frac{x^{1/2}}{1/2} + 4\ln|x| + 2\frac{x^{-1}}{-1} + 2\frac{x^{-2}}{-2} + C$$
    $$F_4(x) = 10\sqrt{x} + 4\ln(x) - \frac{2}{x} - \frac{1}{x^2} + C$$
    
    \item $f_5(x)=\frac{x+5}{x^2}$ sur $]0,+\infty[$ ou $]-\infty,0[$ :
    $$f_5(x) = \frac{x}{x^2} + \frac{5}{x^2} = \frac{1}{x} + 5x^{-2}$$
    $$F_5(x) = \ln|x| + 5\frac{x^{-1}}{-1} + C = \ln|x| - \frac{5}{x} + C$$
    
    \item $f_6(x)=\frac{x^2}{5}+\frac 1{6}$ sur $\mathbb{R}$ :
    $$F_6(x) = \frac{1}{5}\frac{x^3}{3} + \frac{1}{6}x + C = \frac{x^3}{15} + \frac{x}{6} + C$$
\end{enumerate}
\end{mdframed}
}{}



% Exercice 3126

\vspace{1em}
\subsection{Primitives usuelles}

Déterminer toutes les primitives des fonctions suivantes sur un intervalle bien choisi : 
$$\begin{array}{lll}
\displaystyle f_1(x)=e^{4x}&\displaystyle f_2(x)=e^{4x+3}& \displaystyle f_3(x)=\sin(2x)\\
\displaystyle f_4(x)=\cos\left(3x+\frac\pi 3\right)&
\displaystyle f_5(x)=(2x+1)^2&\displaystyle f_6(x)=\frac{3}{\sqrt{5x+1}}.
\end{array}$$

\ifthenelse{\boolean{showSolutions}}{
    \vspace{2em}
    \begin{mdframed}
    \begin{enumerate}
    \item $f_1(x)=e^{4x}$ sur $\mathbb{R}$ :
    $$F_1(x) = \frac{e^{4x}}{4} + C$$
    
    \item $f_2(x)=e^{4x+3}$ sur $\mathbb{R}$ :
    $$f_2(x) = e^{4x}e^3 = e^3 \cdot e^{4x}$$
    $$F_2(x) = e^3 \cdot \frac{e^{4x}}{4} + C = \frac{e^{4x+3}}{4} + C$$
    
    \item $f_3(x)=\sin(2x)$ sur $\mathbb{R}$ :
    $$F_3(x) = -\frac{\cos(2x)}{2} + C$$
    
    \item $f_4(x)=\cos\left(3x+\frac\pi 3\right)$ sur $\mathbb{R}$ :
    $$F_4(x) = \frac{\sin\left(3x+\frac\pi 3\right)}{3} + C$$
    
    \item $f_5(x)=(2x+1)^2$ sur $\mathbb{R}$ :
    $$f_5(x) = (2x+1)^2 = 4x^2 + 4x + 1$$
    $$F_5(x) = 4\frac{x^3}{3} + 4\frac{x^2}{2} + x + C = \frac{4x^3}{3} + 2x^2 + x + C$$
    Ou en utilisant la substitution $u = 2x+1$ :
    $$F_5(x) = \frac{(2x+1)^3}{3 \cdot 2} + C = \frac{(2x+1)^3}{6} + C$$
    
    \item $f_6(x)=\frac{3}{\sqrt{5x+1}}$ sur $]-\frac{1}{5},+\infty[$ :
    $$f_6(x) = 3(5x+1)^{-1/2}$$
    $$F_6(x) = 3\frac{(5x+1)^{1/2}}{1/2 \cdot 5} + C = \frac{6\sqrt{5x+1}}{5} + C$$
\end{enumerate}
\end{mdframed}
}{}



% Exercice 2600


\vspace{1em}
\subsection{Reconnaissance de formes}

Déterminer toutes les primitives des fonctions suivantes : 
$$
\begin{array}{lllll}
\displaystyle f(x)=\frac{x}{1+x^2}&\quad&\displaystyle g(x)=\frac{e^{3x}}{1+e^{3x}}&\quad&
\displaystyle h(x)=\frac{\ln x}{x}\\
\displaystyle k(x)=\cos(x)\sin^2(x)&\quad&l(x)=\frac{1}{x\ln x}&\quad&m(x)=3x\sqrt{1+x^2}.
\end{array}
$$

\ifthenelse{\boolean{showSolutions}}{
    \vspace{2em}
    \begin{mdframed}
    \begin{enumerate}
    \item $f(x)=\frac{x}{1+x^2}$ sur $\mathbb{R}$ :
    
    On reconnaît la forme $\frac{u'}{u}$ avec $u = 1+x^2$ et $u' = 2x$, donc :
    $$f(x) = \frac{1}{2} \cdot \frac{2x}{1+x^2}$$
    $$F(x) = \frac{1}{2}\ln|1+x^2| + C = \frac{1}{2}\ln(1+x^2) + C$$
    
    \item $g(x)=\frac{e^{3x}}{1+e^{3x}}$ sur $\mathbb{R}$ :
    
    On reconnaît la forme $\frac{u'}{u}$ avec $u = 1+e^{3x}$ et $u' = 3e^{3x}$, donc :
    $$g(x) = \frac{1}{3} \cdot \frac{3e^{3x}}{1+e^{3x}}$$
    $$G(x) = \frac{1}{3}\ln|1+e^{3x}| + C = \frac{1}{3}\ln(1+e^{3x}) + C$$
    
    \item $h(x)=\frac{\ln x}{x}$ sur $]0,+\infty[$ :
    
    On reconnaît la forme $u' \cdot u$ avec $u = \ln x$ et $u' = \frac{1}{x}$, donc :
    $$H(x) = \frac{(\ln x)^2}{2} + C$$
    
    \item $k(x)=\cos(x)\sin^2(x)$ sur $\mathbb{R}$ :
    
    On reconnaît la forme $u' \cdot u^n$ avec $u = \sin x$ et $u' = \cos x$, donc :
    $$K(x) = \frac{\sin^3(x)}{3} + C$$
    
    \item $l(x)=\frac{1}{x\ln x}$ sur $]0,1[$ ou $]1,+\infty[$ :
    
    On reconnaît la forme $\frac{u'}{u}$ avec $u = \ln x$ et $u' = \frac{1}{x}$, donc :
    $$L(x) = \ln|\ln x| + C$$
    
    \item $m(x)=3x\sqrt{1+x^2}$ sur $\mathbb{R}$ :
    
    On reconnaît la forme $u' \cdot u^{1/2}$ avec $u = 1+x^2$ et $u' = 2x$, donc :
    $$m(x) = \frac{3}{2} \cdot 2x \cdot (1+x^2)^{1/2}$$
    $$M(x) = \frac{3}{2} \cdot \frac{(1+x^2)^{3/2}}{3/2} + C = (1+x^2)^{3/2} + C$$
\end{enumerate}
\end{mdframed}
}{}


% Exercice 431


\vspace{1em}
\subsection{Reconnaissance de formes}

Déterminer une primitive des fonctions suivantes sur l'intervalle considéré :
\begin{multicols}{2}
    \begin{enumerate}[itemsep = 0.5em]
\item $f(x)=(3x-1)(3x^2-2x+3)^3,\ I=\mathbb{R}$
\item $f(x)=\frac{1-x^2}{(x^3-3x+2)^3},\ I=]-\infty,-2[$
\item $f(x)=\frac{(x-1)}{\sqrt{x(x-2)}},\ I=]-\infty,0[$
\item $f(x)=\frac{1}{x\ln(x^2)},\ I=]1,+\infty[.$
    \end{enumerate}
\end{multicols}


% Exercice 397


\vspace{1em}
\subsection{Intégration par parties - Niveau 1}

Calculer les intégrales suivantes :
$$\mathbf{1.}\quad I=\int_0^1 xe^xdx\quad\quad\mathbf{2.}\quad J=\int_1^e x^2\ln xdx$$

\ifthenelse{\boolean{showSolutions}}{
    \vspace{2em}
    \begin{mdframed}
    \textbf{1.} Calcul de $I=\int_0^1 xe^xdx$
    
    Intégration par parties : $u = x$, $dv = e^x dx$, donc $du = dx$, $v = e^x$
    
    $$I = [xe^x]_0^1 - \int_0^1 e^x dx = 1 \cdot e^1 - 0 \cdot e^0 - [e^x]_0^1 = e - (e^1 - e^0) = e - (e - 1) = 1$$
    
    \textbf{2.} Calcul de $J=\int_1^e x^2\ln xdx$
    
    Intégration par parties : $u = \ln x$, $dv = x^2 dx$, donc $du = \frac{1}{x} dx$, $v = \frac{x^3}{3}$
    
    $$J = \left[\frac{x^3}{3}\ln x\right]_1^e - \int_1^e \frac{x^3}{3} \cdot \frac{1}{x} dx = \frac{e^3}{3}\ln e - \frac{1^3}{3}\ln 1 - \frac{1}{3}\int_1^e x^2 dx$$
    
    $$= \frac{e^3}{3} - 0 - \frac{1}{3}\left[\frac{x^3}{3}\right]_1^e = \frac{e^3}{3} - \frac{1}{9}(e^3 - 1) = \frac{e^3}{3} - \frac{e^3}{9} + \frac{1}{9} = \frac{2e^3}{9} + \frac{1}{9} = \frac{2e^3 + 1}{9}$$
\end{mdframed}
}{}


% Exercice 398


\vspace{1em}
\subsection{Intégration par parties - Niveau 2}

Déterminer une primitive des fonctions suivantes :
$$\mathbf{1.}\quad x\mapsto\arctan(x)\quad\quad\mathbf{2.}\quad x\mapsto (\ln x)^2\quad\quad\mathbf{3.} x\mapsto \sin(\ln x).$$

\ifthenelse{\boolean{showSolutions}}{
    \vspace{2em}
    \begin{mdframed}
    \textbf{1.} Primitive de $x\mapsto\arctan(x)$
    
    Intégration par parties : $u = \arctan(x)$, $dv = dx$, donc $du = \frac{1}{1+x^2} dx$, $v = x$
    
    $$\int \arctan(x) dx = x\arctan(x) - \int x \cdot \frac{1}{1+x^2} dx$$
    
    Pour $\int \frac{x}{1+x^2} dx$, on reconnaît la forme $\frac{u'}{2u}$ avec $u = 1+x^2$ :
    $$\int \frac{x}{1+x^2} dx = \frac{1}{2}\ln|1+x^2| + C = \frac{1}{2}\ln(1+x^2) + C$$
    
    Donc : $\int \arctan(x) dx = x\arctan(x) - \frac{1}{2}\ln(1+x^2) + C$
    
    \textbf{2.} Primitive de $x\mapsto (\ln x)^2$
    
    Intégration par parties : $u = (\ln x)^2$, $dv = dx$, donc $du = \frac{2\ln x}{x} dx$, $v = x$
    
    $$\int (\ln x)^2 dx = x(\ln x)^2 - \int x \cdot \frac{2\ln x}{x} dx = x(\ln x)^2 - 2\int \ln x dx$$
    
    Pour $\int \ln x dx$, on utilise l'intégration par parties : $u = \ln x$, $dv = dx$, donc $du = \frac{1}{x} dx$, $v = x$
    $$\int \ln x dx = x\ln x - \int x \cdot \frac{1}{x} dx = x\ln x - x + C$$
    
    Donc : $\int (\ln x)^2 dx = x(\ln x)^2 - 2(x\ln x - x) + C = x(\ln x)^2 - 2x\ln x + 2x + C$
    
    \textbf{3.} Primitive de $x\mapsto \sin(\ln x)$
    
    Intégration par parties : $u = \sin(\ln x)$, $dv = dx$, donc $du = \frac{\cos(\ln x)}{x} dx$, $v = x$
    
    $$\int \sin(\ln x) dx = x\sin(\ln x) - \int x \cdot \frac{\cos(\ln x)}{x} dx = x\sin(\ln x) - \int \cos(\ln x) dx$$
    
    Pour $\int \cos(\ln x) dx$, on utilise l'intégration par parties : $u = \cos(\ln x)$, $dv = dx$, donc $du = -\frac{\sin(\ln x)}{x} dx$, $v = x$
    $$\int \cos(\ln x) dx = x\cos(\ln x) + \int \sin(\ln x) dx$$
    
    En substituant dans la première équation :
    $$\int \sin(\ln x) dx = x\sin(\ln x) - \left(x\cos(\ln x) + \int \sin(\ln x) dx\right)$$
    
    $$2\int \sin(\ln x) dx = x\sin(\ln x) - x\cos(\ln x) + C$$
    
    Donc : $\int \sin(\ln x) dx = \frac{x}{2}(\sin(\ln x) - \cos(\ln x)) + C$
\end{mdframed}
}{}


% Exercice 3492


\vspace{1em}
\subsection{Intégration par parties en boucle}

Calculer les intégrales suivantes :
$$1.\ \int_1^2 \frac{\ln(x)}{x}dx\quad\quad 2.\ \int_0^\pi e^x\sin(x)dx.$$


% Exercice 393


\vspace{1em}
\subsection{Changements de variables - Niveau 1}

En effectuant le changement de variables demandé, calculer les intégrales suivantes :
\begin{enumerate}
\item $\displaystyle \int_1^4\frac{1-\sqrt t}{\sqrt t}dt$ en posant $x=\sqrt t$;
\item $\displaystyle \int_0^{\pi}\frac{\sin t}{1+\cos^2 t}dt$ en posant $x=\cos t$;
\item $\displaystyle \int_1^e \frac{dt}{2t\ln (t)+t}$ en posant $x=\ln t$.
\end{enumerate}


% Exercice 394


\vspace{1em}
\subsection{Changements de variables - Niveau 2}

En effectuant le changement de variables indiqué, calculer les intégrales suivantes :
\begin{enumerate}
\item  $\displaystyle \int_0^1\frac{dt}{1+e^t}$ en posant $x=e^t$;
\item $\displaystyle \int_1^3\frac{\sqrt t}{t+1}dt$ en posant $x=\sqrt t$;
\item $\displaystyle \int_{-1}^1 \sqrt{1-t^2}dt$ en posant $t=\sin\theta$.
\end{enumerate}


% Exercice 3154


\vspace{1em}
\subsection{Changement de variables - Recherche de primitives - Niveau 1}

En effectuant le changement de variables indiqué, déterminer une primitive des fonctions suivantes :
\begin{enumerate}
\item $\displaystyle x\mapsto \frac{x}{\sqrt{1+x}}$, en posant $u=\sqrt{1+x}$;
\item $\displaystyle x\mapsto \frac{1}{e^x+1}$, en posant $u=e^x$;
\item $\displaystyle x\mapsto \frac{1}{x+x(\ln x)^2}$, en posant $u=\ln x$.
\end{enumerate}


% Exercice 395


\vspace{1em}
\subsection{Changements de variables - Recherche de primitives - Niveau 2}

En effectuant un changement de variables, déterminer une primitive des fonctions suivantes :
\begin{enumerate}
\item $\displaystyle x\mapsto \cos(2\ln x)$;
\item $\displaystyle x\mapsto\cos(\sqrt x)$;
\item $\displaystyle x\mapsto \frac{e^x}{(3+e^x)\sqrt{e^x-1}}$.
\end{enumerate}


% Exercice 433


\vspace{1em}
\subsection{Primitive de fractions rationnelles}

Déterminer une primitive des fractions rationnelles suivantes :
$$
\begin{array}{lll}
\mathbf 1.\ f(x)=\frac{2x^2-3x+4}{(x-1)^2}\textrm{ sur }]1,+\infty[&\quad&\mathbf 2. f(x)=\frac{2x-1}{(x+1)^2}\textrm{ sur }]-1,+\infty[ \\
\mathbf 3.\ f(x)=\frac{x}{(x^2-4)^2}\textrm{ sur }]2,+\infty[&&\mathbf 4. f(x)=\frac{24x^3+18x^2+10x-9}{(3x-1)(2x+1)^2}\textrm{ sur }]-1/2,1/3[
\end{array}
$$


% Exercice 3152


\vspace{1em}
\subsection{Exponentielle * Polynôme}

Calculer les intégrales suivantes  :
$$\begin{array}{lcl}
\displaystyle \mathbf{1.}\int_{0}^2 (x+6)e^{2x}dx &\quad&\displaystyle \mathbf{2.} \int_0^1 e^x(2x^3+3x^2-x+1)dx
\end{array}$$


% Exercice 3153


\vspace{1em}
\subsection{Exponentielle * trigonométrique}

Calculer les intégrales suivantes  :
$$\begin{array}{lcl}
\displaystyle \mathbf{1.} \int_0^\pi e^x\sin(2x)dx&\quad&\displaystyle \mathbf{2.} \int_0^{2\pi}e^{-x}\sin^2 xdx\\
\end{array}$$


% Exercice 437


\vspace{1em}
\subsection{Exponentielle * polynôme * trigonométrique}

Calculer l'intégrale :
$$\int_0^\pi x^2e^x \cos xdx.$$


% Exercice 2496


\vspace{1em}
\subsection{Quelques primitives à savoir calculer!}

Déterminer une primitive des fonctions suivantes :
$$
\begin{array}{lcl}
\displaystyle \mathbf{1.}\quad x\mapsto \frac{1}{x^2+4}&\quad\quad&\displaystyle \mathbf{2.}\quad x\mapsto\frac{1}{x^2+4x+5}\\
\displaystyle \mathbf{3.}\quad x\mapsto \frac{1}{1-x^2}&&\displaystyle \mathbf{4.}\quad x\mapsto e^x(2x^3+3x^2-x+1)\\
\displaystyle \mathbf{5.}\quad x\mapsto\sin^3(x)&&\displaystyle \mathbf{6.}\quad x\mapsto \arctan(x)
\end{array}$$



% Exercice 165


% \vspace{1em}
% \subsection{Quelques décompositions en séries de Fourier}

% Déterminer les séries de Fourier (termes en sinus et cosinus) des fonctions suivantes :
% \begin{enumerate}
% \item $f$ $2\pi-$périodique, définie par $f(x)=x$ si $-\pi\leq x<\pi$.
% \item la fonction créneau : $f$ est $2\pi$-périodique, définie par $f(x)=1$ si $x\in[0,\pi[$, et $f(x)=-1$ si $x\in[-\pi,0[$.
% \item la fonction $L-$périodique, où $L>0$, définie par $f(x)=|x|$ si $x\in[-L/2,L/2]$.
% \end{enumerate}

\vspace{1em}

Pour les corrigés : \url{http://www.bibmath.net/ressources/justeunefeuille.php?id=46200} 

%Vous avez accès aux corrigés de cette feuille par l'url : \url{https://www.bibmath.net/ressources/justeunefeuille.php?id=46200}