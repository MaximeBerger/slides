
\newcommand{\mtn}{\mathbb{N}}
\newcommand{\mtns}{\mathbb{N}^*}
\newcommand{\mtz}{\mathbb{Z}}
\newcommand{\mtr}{\mathbb{R}}
\newcommand{\mtk}{\mathbb{K}}
\newcommand{\mtq}{\mathbb{Q}}
\newcommand{\mtc}{\mathbb{C}}
\newcommand{\mch}{\mathcal{H}}
\newcommand{\mcp}{\mathcal{P}}
\newcommand{\mcb}{\mathcal{B}}
\newcommand{\mcl}{\mathcal{L}}
\newcommand{\mcm}{\mathcal{M}}
\newcommand{\mcc}{\mathcal{C}}
\newcommand{\mcmn}{\mathcal{M}}
\newcommand{\mcmnr}{\mathcal{M}_n(\mtr)}
\newcommand{\mcmnk}{\mathcal{M}_n(\mtk)}
\newcommand{\mcsn}{\mathcal{S}_n}
\newcommand{\mcs}{\mathcal{S}}
\newcommand{\mcd}{\mathcal{D}}
\newcommand{\mcsns}{\mathcal{S}_n^{++}}
\newcommand{\glnk}{GL_n(\mtk)}
\newcommand{\mnr}{\mathcal{M}_n(\mtr)}
\newcommand{\veps}{\varepsilon}
\newcommand{\mcu}{\mathcal{U}}
\newcommand{\mcun}{\mcu_n}
\newcommand{\dis}{\displaystyle}
\newcommand{\croouv}{[\![}
\newcommand{\crofer}{]\!]}
\newcommand{\rab}{\mathcal{R}(a,b)}
\newcommand{\pss}[2]{\langle #1,#2\rangle}
 %Document 


\section*{Rappels sur les nombres complexes}

\subsection*{Définitions et formes usuelles}

Un \underline{nombre complexe} $z$ s'écrit sous la forme :
\[
z = a + ib \qquad (a, b \in \mathbb{R},\ i^2 = -1)
\]
où $a$ est la partie réelle $\Re(z)$ et $b$ la partie imaginaire $\Im(z)$.


\underline{Forme algébrique} : $z = a + ib$

\underline{Forme trigonométrique} : 
\[
z = r(\cos\theta + i\sin\theta)
\]
où $r = |z| = \sqrt{a^2 + b^2}$ est le \textbf{module} de $z$, et $\theta = \arg(z)$ est un \textbf{argument} de $z$ (défini à $2\pi$ près).

\underline{Forme exponentielle} (formule d'Euler) :
\[
z = r e^{i\theta}
\]
avec $e^{i\theta} = \cos\theta + i\sin\theta$.


\subsection*{Module, argument et conjugué}

\begin{itemize}
    \item \underline{Module} : $|z| = \sqrt{a^2 + b^2}$
    \item \underline{Argument} : $\theta = \arctan\left(\frac{b}{a}\right)$ (attention au quadrant)
    \item \underline{Conjugué} : $\overline{z} = a - ib$
\end{itemize}

\subsection*{Formule de Moivre}

Pour tout $n \in \mathbb{Z}$,
\[
\left(\cos\theta + i\sin\theta\right)^n = \cos(n\theta) + i\sin(n\theta)
\]
ou, sous forme exponentielle :
\[
\left(e^{i\theta}\right)^n = e^{in\theta}
\]

\subsection*{Racines $n$-ièmes de l'unité}

Les solutions de $z^n = 1$ sont :
\[
z_k = e^{i\frac{2\pi k}{n}},\quad k = 0, 1, \ldots, n-1
\]



\vspace{3em}

\subsection{Module et argument}

Écrire sous la forme $a+i b$, puis sous forme exponentielle les nombres complexes suivants :
\begin{enumerate}
\item Nombre de module 2 et d'argument $\pi / 3$.
\item Nombre de module 3 et d'argument $-\pi / 8$.
\item Nombre de module 1 et d'argument $\pi / 4$.
\item Nombre de module 2 et d'argument $-\pi / 6$.
\item Nombre de module 7 et d'argument $-\pi / 2$.
\end{enumerate}

\ifthenelse{\boolean{showSolutions}}{
    \vspace{2em}

    \begin{mdframed}
\begin{enumerate}
    \item $1 + \sqrt{3}  i $ = $2e^{i\frac{\pi}{3}}$
    \item $3 \cos \left(-\frac{\pi}{8}\right) + i \sin \left(-\frac{\pi}{8}\right) $ = $3e^{-i\frac{\pi}{8}}$
    \item $1 + i $ = $e^{i\frac{\pi}{4}}$
    \item $\sqrt{3} - i $ = $2e^{-i\frac{\pi}{6}}$
    \item $-7 i $ = $7e^{-i\frac{\pi}{2}}$
\end{enumerate}
\end{mdframed}

}{}
\vspace{2em}

\subsection{Forme exponentielle $\rightarrow$ forme algébrique}

Écrire sous la forme $a+ib$ les nombres complexes suivants, donnés sous forme exponentielle :
\begin{multicols}{2}
\begin{enumerate}
    \item $z_1 = 5 e^{i \frac{\pi}{6}}$
    \item $z_2 = 2 e^{-i \frac{\pi}{4}}$
    \item $z_3 = 3 e^{i \frac{2\pi}{3}}$
    \item $z_4 = 7 e^{i \pi}$
    \item $z_5 = 4 e^{i 0}$
    \item $z_6 = 6 e^{-i \frac{\pi}{2}}$
\end{enumerate}
\end{multicols}
\ifthenelse{\boolean{showSolutions}}{
    \vspace{2em}
    \begin{mdframed}
    \begin{enumerate}
    \item $5 \frac{\sqrt{3}}{2} + i \frac{5}{2} $ 
    \item $\sqrt{2} - \sqrt{2} i  $ 
    \item $- \frac{3}{2} + i \frac{\sqrt{3}}{2} $ 
    \item $-7 $ 
    \item $4 $ 
    \item $-6 i $ 
\end{enumerate}
\end{mdframed}
}{}

\vspace{2em}

\subsection{Forme exponentielle}
Mettre sous forme exponentielle les nombres complexes suivants : 
\begin{multicols}{3}
\begin{enumerate}
    \item $z_1=1+i \sqrt{3}$, 
    \item $z_2=1+i$, 
    \item $z_3=-2 \sqrt{3}+2 i$, 
    \item $z_4=i$, 
    \item $z_5=-2 i$, 
    \item $z_6=-3$,
    \item $z_7=1$
    \item $z_8=9 i$
    \item $z_9=0$
    \item $z_{10}=\frac{-i \sqrt{2}}{1+i}$
    \item $z_{11}=\frac{(1+i \sqrt{3})^3}{(1-i)^5}$
    \item $z_{12}=\sin x+i \cos x$.
\end{enumerate}
\end{multicols}
\ifthenelse{\boolean{showSolutions}}{
    \vspace{2em}
    \begin{mdframed}
    \begin{enumerate}
    \item $1 + i \sqrt{3} $ = $2e^{i\frac{\pi}{3}}$
    \item $1 + i $ = $e^{i\frac{\pi}{4}}$
    \item $-2 \sqrt{3} + 2 i $ = $4e^{i\frac{4\pi}{6}}$
    \item $i $ = $e^{i\frac{\pi}{2}}$
    \item $-2 i $ = $2e^{-i\frac{\pi}{2}}$
    \item $-3 $ = $3e^{i\pi}$
    \item $1 $ = $e^{i0}$
    \item $9 i $ = $9e^{i\frac{\pi}{2}}$
    \item $0 $ = $0e^{i0}$
    \item On met sous forme exponentielle le numérateur et le dénominateur et on simplifie $-i \sqrt{2}= \sqrt{2}e^{-i\frac{\pi}{2}}$ et $1+i= \sqrt{2}e^{i\frac{\pi}{4}}$
    
    donc $\frac{-i \sqrt{2}}{1+i} $ = $e^{-i\frac{3\pi}{4}}$
    \item De même, $1+i \sqrt{3} = 2e^{i\frac{\pi}{3}}$, $(1-i) = \sqrt{2}e^{-i\frac{\pi}{4}}$
    donc
    $$\frac{(1+i \sqrt{3})^3}{(1-i)^5}  = \frac{2^3}{\sqrt{2}^5}e^{i(\pi + 5\frac{\pi}{4})}= \sqrt{2}e^{i\frac{5\pi}{4}}$$
    \item $\sin x + i \cos x = e^{i(\frac{\pi}{2}-x)}$
\end{enumerate}
\end{mdframed}
}{}



\vspace{2em}

\subsection{Exponentielle}
Résoudre l'équation $e^z=3 \sqrt{3}-3 i$.
\ifthenelse{\boolean{showSolutions}}{
    \vspace{2em}
    \begin{mdframed}
    \begin{enumerate}
    \item en mettant le terme de droite sous forme exponentielle, on obtient $3 \sqrt{3}-3 i = 6e^{-i\frac{\pi}{6}}$
    En cherchant $z=a+ib$, $e^z = e^{a+ib} = e^a e^{ib} = 6e^{-i\frac{\pi}{6}}$
    donc $a= \ln(6)$ et $b=-\frac{\pi}{6}$
    donc $z= \ln(6)-\frac{\pi}{6}i$
\end{enumerate}
\end{mdframed}
}{}


\newpage

\subsection{Trigonométrique}
En utilisant les nombres complexes, calculer  $\cos 5 \theta$ et $\sin 5 \theta$ en fonction de $\cos \theta$ et $\sin \theta$.
\ifthenelse{\boolean{showSolutions}}{
    \vspace{2em}
    \begin{mdframed}
    en utilisant la formule de Moivre, on obtient
    \[
    (\cos \theta + i \sin \theta)^5 = \cos(5\theta) + i \sin(5\theta)
    \]
    En développant à l'aide du binôme de Newton :
    \[
    (\cos \theta + i \sin \theta)^5 = \sum_{k=0}^5 \binom{5}{k} (\cos \theta)^{5-k} (i \sin \theta)^k
    \]
    \[
    = \sum_{k=0}^5 \binom{5}{k} i^k (\cos \theta)^{5-k} (\sin \theta)^k
    \]
    En séparant la partie réelle et imaginaire, on obtient :
    \[
    \cos(5\theta) = \Re\left[(\cos \theta + i \sin \theta)^5\right] = \sum_{k=0}^{5} \binom{5}{k} (\cos \theta)^{5-k} (\sin \theta)^k \Re(i^k)
    \]
    \[
    \sin(5\theta) = \Im\left[(\cos \theta + i \sin \theta)^5\right] = \sum_{k=0}^{5} \binom{5}{k} (\cos \theta)^{5-k} (\sin \theta)^k \Im(i^k)
    \]
    En explicitant, on trouve :
    \[
    \cos(5\theta) = \cos^5\theta - 10\cos^3\theta\sin^2\theta + 5\cos\theta\sin^4\theta
    \]
    \[
    \sin(5\theta) = 5\cos^4\theta\sin\theta - 10\cos^2\theta\sin^3\theta + \sin^5\theta
    \]
\end{mdframed}
}{}
\vspace{2em}


\subsection{Pour préparer les séries de fourier}
Calculer les intégrales suivantes, pour toute valeur de $n$ et $m$ dans les entiers relatifs:

\begin{enumerate}
    \item $$\int_0^{2\pi} e^{i n x} e^{i m x} dx$$
    \item $$\int_0^{2\pi} \cos(n x) \cos(m x) dx$$
    \item $$\int_0^{2\pi} \sin(n x) \sin(m x) dx$$
    \item $$\int_0^{2\pi} \cos(n x) \sin(m x) dx$$
\end{enumerate}

\ifthenelse{\boolean{showSolutions}}{
    \vspace{2em}
    \begin{mdframed}
    \begin{enumerate}
    \item $$\int_0^{2\pi} e^{i n x} e^{i m x} dx = \int_0^{2\pi} e^{i (n+m) x} dx$$
    
    Si $n+m \neq 0$ : 
    $$\int_0^{2\pi} e^{i (n+m) x} dx = \left[\frac{e^{i (n+m) x}}{i(n+m)}\right]_0^{2\pi} = \frac{e^{i (n+m) 2\pi} - e^{i (n+m) 0}}{i(n+m)} = \frac{1 - 1}{i(n+m)} = 0$$
    
    Si $n+m = 0$ (c'est-à-dire $m = -n$) :
    $$\int_0^{2\pi} e^{i 0} dx = \int_0^{2\pi} 1 dx = 2\pi$$
    
    Donc : $\int_0^{2\pi} e^{i n x} e^{i m x} dx = \begin{cases} 2\pi & \text{si } m = -n \\ 0 & \text{sinon} \end{cases}$
    
    \item $$\int_0^{2\pi} \cos(n x) \cos(m x) dx$$
    
    En utilisant $\cos(nx)\cos(mx) = \frac{1}{2}[\cos((n+m)x) + \cos((n-m)x)]$ :
    
    Si $n \neq \pm m$ :
    $$\int_0^{2\pi} \cos(n x) \cos(m x) dx = \frac{1}{2}\int_0^{2\pi} [\cos((n+m)x) + \cos((n-m)x)] dx = 0$$
    
    Si $n = m \neq 0$ :
    $$\int_0^{2\pi} \cos^2(nx) dx = \int_0^{2\pi} \frac{1+\cos(2nx)}{2} dx = \frac{1}{2}[x]_0^{2\pi} + 0 = \pi$$
    
    Si $n = m = 0$ :
    $$\int_0^{2\pi} \cos^2(0) dx = \int_0^{2\pi} 1 dx = 2\pi$$
    
    Donc : $\int_0^{2\pi} \cos(n x) \cos(m x) dx = \begin{cases} 2\pi & \text{si } n = m = 0 \\ \pi & \text{si } n = m \neq 0 \\ 0 & \text{sinon} \end{cases}$
    
    \item $$\int_0^{2\pi} \sin(n x) \sin(m x) dx$$
    
    En utilisant $\sin(nx)\sin(mx) = \frac{1}{2}[\cos((n-m)x) - \cos((n+m)x)]$ :
    
    Si $n \neq \pm m$ :
    $$\int_0^{2\pi} \sin(n x) \sin(m x) dx = \frac{1}{2}\int_0^{2\pi} [\cos((n-m)x) - \cos((n+m)x)] dx = 0$$
    
    Si $n = m \neq 0$ :
    $$\int_0^{2\pi} \sin^2(nx) dx = \int_0^{2\pi} \frac{1-\cos(2nx)}{2} dx = \frac{1}{2}[x]_0^{2\pi} - 0 = \pi$$
    
    Si $n = m = 0$ :
    $$\int_0^{2\pi} \sin^2(0) dx = \int_0^{2\pi} 0 dx = 0$$
    
    Donc : $\int_0^{2\pi} \sin(n x) \sin(m x) dx = \begin{cases} \pi & \text{si } n = m \neq 0 \\ 0 & \text{sinon} \end{cases}$
    
    \item $$\int_0^{2\pi} \cos(n x) \sin(m x) dx$$
    
    En utilisant $\cos(nx)\sin(mx) = \frac{1}{2}[\sin((n+m)x) + \sin((m-n)x)]$ :
    
    Pour toute valeur de $n$ et $m$ :
    $$\int_0^{2\pi} \cos(n x) \sin(m x) dx = \frac{1}{2}\int_0^{2\pi} [\sin((n+m)x) + \sin((m-n)x)] dx = 0$$
    
    Donc : $\int_0^{2\pi} \cos(n x) \sin(m x) dx = 0$ pour tous $n, m \in \mathbb{Z}$
\end{enumerate}
\end{mdframed}
}{}

\vspace{2em}

\subsection{Exponentielle}
On pose 
$$z_1=4 e^{i \frac{\pi}{4}}, \qquad z_2=3 i e^{i \frac{\pi}{6}}, \qquad z_3=-2 e^{i \frac{2 \pi}{3}}$$
Écrire sous forme exponentielle les nombres complexes : 
$$z_1,\qquad z_2,\qquad z_3 , \qquad z_1 z_2, \qquad \frac{z_1 z_2}{z_2}$$

\ifthenelse{\boolean{showSolutions}}{
    \vspace{2em}
    \begin{mdframed}
    \begin{enumerate}
    \item $z_1 = 4 e^{i \frac{\pi}{4}}$ (déjà sous forme exponentielle)
    \item $z_2 = 3i e^{i \frac{\pi}{6}} = 3 e^{i \frac{\pi}{2}} e^{i \frac{\pi}{6}} = 3 e^{i \frac{\pi}{2} + i \frac{\pi}{6}} = 3 e^{i \frac{2\pi}{3}}$
    \item $z_3 = -2 e^{i \frac{2\pi}{3}} = 2 e^{i\pi} e^{i \frac{2\pi}{3}} = 2 e^{i\pi + i \frac{2\pi}{3}} = 2 e^{i \frac{5\pi}{3}}$
    \item $z_1 z_2 = 4 e^{i \frac{\pi}{4}} \times 3 e^{i \frac{2\pi}{3}} = 12 e^{i \frac{\pi}{4} + i \frac{2\pi}{3}} = 12 e^{i \frac{3\pi + 8\pi}{12}} = 12 e^{i \frac{11\pi}{12}}$
    \item $\frac{z_1 z_2}{z_2} = \frac{12 e^{i \frac{11\pi}{12}}}{3 e^{i \frac{2\pi}{3}}} = 4 e^{i \frac{11\pi}{12} - i \frac{2\pi}{3}} = 4 e^{i \frac{11\pi - 8\pi}{12}} = 4 e^{i \frac{3\pi}{12}} = 4 e^{i \frac{\pi}{4}} = z_1$
\end{enumerate}
\end{mdframed}
}{}

\vspace{2em}

\subsection{Racines carrées}
Calculer de deux façons les racines carrées de $1+i$ et en déduire les valeurs exactes de $\cos \left(\frac{\pi}{8}\right)$ et $\sin \left(\frac{\pi}{8}\right)$.

\ifthenelse{\boolean{showSolutions}}{
    \vspace{2em}
    \begin{mdframed}
    \textbf{Méthode 1 : Forme algébrique}
    
    On cherche $z = a + ib$ tel que $z^2 = 1 + i$.
    
    $(a + ib)^2 = a^2 - b^2 + 2iab = 1 + i$
    
    En identifiant partie réelle et imaginaire :
    $$\begin{cases} a^2 - b^2 = 1 \\ 2ab = 1 \end{cases}$$
    
    De la deuxième équation : $b = \frac{1}{2a}$ (avec $a \neq 0$)
    
    En substituant dans la première : $a^2 - \frac{1}{4a^2} = 1$
    
    En multipliant par $4a^2$ : $4a^4 - 1 = 4a^2$, soit $4a^4 - 4a^2 - 1 = 0$
    
    Posons $X = a^2$ : $4X^2 - 4X - 1 = 0$
    
    $\Delta = 16 + 16 = 32$, donc $X = \frac{4 \pm \sqrt{32}}{8} = \frac{4 \pm 4\sqrt{2}}{8} = \frac{1 \pm \sqrt{2}}{2}$
    
    Comme $X = a^2 \geq 0$, on prend $X = \frac{1 + \sqrt{2}}{2}$
    
    Donc $a = \pm\sqrt{\frac{1 + \sqrt{2}}{2}}$ et $b = \frac{1}{2a}$
    
    \textbf{Méthode 2 : Forme exponentielle}
    
    $1 + i = \sqrt{2}e^{i\frac{\pi}{4}}$
    
    Les racines carrées sont : $\sqrt[2]{\sqrt{2}}e^{i\frac{\pi}{4}/2 + ik\pi}$ avec $k \in \{0,1\}$
    
    Donc : $z_1 = 2^{1/4}e^{i\frac{\pi}{8}} = 2^{1/4}(\cos\frac{\pi}{8} + i\sin\frac{\pi}{8})$
    
    et $z_2 = 2^{1/4}e^{i\frac{\pi}{8} + i\pi} = 2^{1/4}e^{i\frac{9\pi}{8}} = -2^{1/4}(\cos\frac{\pi}{8} + i\sin\frac{\pi}{8})$
    
    \textbf{Identification des valeurs}
    
    En comparant les deux méthodes, on a :
    $$2^{1/4}\cos\frac{\pi}{8} = \sqrt{\frac{1 + \sqrt{2}}{2}}$$
    $$2^{1/4}\sin\frac{\pi}{8} = \frac{1}{2\sqrt{\frac{1 + \sqrt{2}}{2}}} = \frac{1}{2^{3/4}\sqrt{1 + \sqrt{2}}}$$
    
    Comme $2^{1/4} = \sqrt{\sqrt{2}} = 2^{1/2}$, on a :
    $$\cos\frac{\pi}{8} = \frac{\sqrt{\frac{1 + \sqrt{2}}{2}}}{2^{1/4}} = \frac{\sqrt{\frac{1 + \sqrt{2}}{2}}}{\sqrt{\sqrt{2}}} = \frac{\sqrt{1 + \sqrt{2}}}{\sqrt{2\sqrt{2}}} = \frac{\sqrt{1 + \sqrt{2}}}{\sqrt{2}}$$
    
    $$\sin\frac{\pi}{8} = \frac{\sqrt{1 - \sqrt{2}}}{\sqrt{2}}$$
    
    En fait, en utilisant les identités trigonométriques :
    $$\cos\frac{\pi}{8} = \frac{\sqrt{2 + \sqrt{2}}}{2}$$
    $$\sin\frac{\pi}{8} = \frac{\sqrt{2 - \sqrt{2}}}{2}$$
\end{mdframed}
}{}