
\newcommand{\mtn}{\mathbb{N}}
\newcommand{\mtns}{\mathbb{N}^*}
\newcommand{\mtz}{\mathbb{Z}}
\newcommand{\mtr}{\mathbb{R}}
\newcommand{\mtk}{\mathbb{K}}
\newcommand{\mtq}{\mathbb{Q}}
\newcommand{\mtc}{\mathbb{C}}
\newcommand{\mch}{\mathcal{H}}
\newcommand{\mcp}{\mathcal{P}}
\newcommand{\mcb}{\mathcal{B}}
\newcommand{\mcl}{\mathcal{L}}
\newcommand{\mcm}{\mathcal{M}}
\newcommand{\mcc}{\mathcal{C}}
\newcommand{\mcmn}{\mathcal{M}}
\newcommand{\mcmnr}{\mathcal{M}_n(\mtr)}
\newcommand{\mcmnk}{\mathcal{M}_n(\mtk)}
\newcommand{\mcsn}{\mathcal{S}_n}
\newcommand{\mcs}{\mathcal{S}}
\newcommand{\mcd}{\mathcal{D}}
\newcommand{\mcsns}{\mathcal{S}_n^{++}}
\newcommand{\glnk}{GL_n(\mtk)}
\newcommand{\mnr}{\mathcal{M}_n(\mtr)}
\newcommand{\veps}{\varepsilon}
\newcommand{\mcu}{\mathcal{U}}
\newcommand{\mcun}{\mcu_n}
\newcommand{\dis}{\displaystyle}
\newcommand{\croouv}{[\![}
\newcommand{\crofer}{]\!]}
\newcommand{\rab}{\mathcal{R}(a,b)}
\newcommand{\pss}[2]{\langle #1,#2\rangle}
 %Document 


\section*{Rappels sur les nombres complexes}

\subsection*{Définitions et formes usuelles}

Un \underline{nombre complexe} $z$ s'écrit sous la forme :
\[
z = a + ib \qquad (a, b \in \mathbb{R},\ i^2 = -1)
\]
où $a$ est la partie réelle $\Re(z)$ et $b$ la partie imaginaire $\Im(z)$.


\underline{Forme algébrique} : $z = a + ib$

\underline{Forme trigonométrique} : 
\[
z = r(\cos\theta + i\sin\theta)
\]
où $r = |z| = \sqrt{a^2 + b^2}$ est le \textbf{module} de $z$, et $\theta = \arg(z)$ est un \textbf{argument} de $z$ (défini à $2\pi$ près).

\underline{Forme exponentielle} (formule d'Euler) :
\[
z = r e^{i\theta}
\]
avec $e^{i\theta} = \cos\theta + i\sin\theta$.


\subsection*{Module, argument et conjugué}

\begin{itemize}
    \item \underline{Module} : $|z| = \sqrt{a^2 + b^2}$
    \item \underline{Argument} : $\theta = \arctan\left(\frac{b}{a}\right)$ (attention au quadrant)
    \item \underline{Conjugué} : $\overline{z} = a - ib$
\end{itemize}

\subsection*{Formule de Moivre}

Pour tout $n \in \mathbb{Z}$,
\[
\left(\cos\theta + i\sin\theta\right)^n = \cos(n\theta) + i\sin(n\theta)
\]
ou, sous forme exponentielle :
\[
\left(e^{i\theta}\right)^n = e^{in\theta}
\]

\subsection*{Racines $n$-ièmes de l'unité}

Les solutions de $z^n = 1$ sont :
\[
z_k = e^{i\frac{2\pi k}{n}},\quad k = 0, 1, \ldots, n-1
\]



\vspace{3em}

\subsection{Module et argument}

Écrire sous la forme $a+i b$, puis sous forme exponentielle les nombres complexes suivants :
\begin{enumerate}
\item Nombre de module 2 et d'argument $\pi / 3$.
\item Nombre de module 3 et d'argument $-\pi / 8$.
\item Nombre de module 1 et d'argument $\pi / 4$.
\item Nombre de module 2 et d'argument $-\pi / 6$.
\item Nombre de module 7 et d'argument $-\pi / 2$.
\end{enumerate}

\vspace{2em}

\subsection{Forme exponentielle $\rightarrow$ forme algébrique}

Écrire sous la forme $a+ib$ les nombres complexes suivants, donnés sous forme exponentielle :
\begin{multicols}{2}
\begin{enumerate}
    \item $z_1 = 5 e^{i \frac{\pi}{6}}$
    \item $z_2 = 2 e^{-i \frac{\pi}{4}}$
    \item $z_3 = 3 e^{i \frac{2\pi}{3}}$
    \item $z_4 = 7 e^{i \pi}$
    \item $z_5 = 4 e^{i 0}$
    \item $z_6 = 6 e^{-i \frac{\pi}{2}}$
\end{enumerate}
\end{multicols}

\vspace{2em}

\subsection{Forme exponentielle}
Mettre sous forme exponentielle les nombres complexes suivants : 
\begin{multicols}{3}
\begin{enumerate}
    \item $z_1=1+i \sqrt{3}$, 
    \item $z_2=1+i$, 
    \item $z_3=-2 \sqrt{3}+2 i$, 
    \item $z_4=i$, 
    \item $z_5=-2 i$, 
    \item $z_6=-3$,
    \item $z_7=1$
    \item $z_8=9 i$
    \item $z_9=0$
    \item $z_{10}=\frac{-i \sqrt{2}}{1+i}$
    \item $z_{11}=\frac{(1+i \sqrt{3})^3}{(1-i)^5}$
    \item $z_{12}=\sin x+i \cos x$.
\end{enumerate}
\end{multicols}



\vspace{2em}

\subsection{Exponentielle}
Résoudre l'équation $e^z=3 \sqrt{3}-3 i$.

\newpage

\subsection{Trigonométrique}
En utilisant les nombres complexes, calculer  $\cos 5 \theta$ et $\sin 5 \theta$ en fonction de $\cos \theta$ et $\sin \theta$.

\vspace{2em}


\subsection{Pour préparer les séries de fourier}
Calculer les intégrales suivantes, pour toute valeur de $n$ et $m$ dans les entiers relatifs:

\begin{enumerate}
    \item $$\int_0^\pi e^{i n x} e^{i m x} dx$$
    \item $$\int_0^\pi \cos(n x) \cos(m x) dx$$
    \item $$\int_0^\pi \sin(n x) \sin(m x) dx$$
    \item $$\int_0^\pi \cos(n x) \sin(m x) dx$$
\end{enumerate}



\vspace{2em}

\subsection{Exponentielle}
On pose 
$$z_1=4 e^{i \frac{\pi}{4}}, \qquad z_2=3 i e^{i \frac{\pi}{6}}, \qquad z_3=-2 e^{i \frac{2 \pi}{3}}$$
Écrire sous forme exponentielle les nombres complexes : 
$$z_1,\qquad z_2,\qquad z_3 , \qquad z_1 z_2, \qquad \frac{z_1 z_2}{z_2}$$



\vspace{2em}

\subsection{Racines carrées}
Calculer de deux façons les racines carrées de $1+i$ et en déduire les valeurs exactes de $\cos \left(\frac{\pi}{8}\right)$ et $\sin \left(\frac{\pi}{8}\right)$.