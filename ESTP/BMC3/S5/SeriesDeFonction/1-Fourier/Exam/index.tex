
\newcommand{\mtn}{\mathbb{N}}
\newcommand{\mtns}{\mathbb{N}^*}
\newcommand{\mtz}{\mathbb{Z}}
\newcommand{\mtr}{\mathbb{R}}
\newcommand{\mtk}{\mathbb{K}}
\newcommand{\mtq}{\mathbb{Q}}
\newcommand{\mtc}{\mathbb{C}}
\newcommand{\mch}{\mathcal{H}}
\newcommand{\mcp}{\mathcal{P}}
\newcommand{\mcb}{\mathcal{B}}
\newcommand{\mcl}{\mathcal{L}}
\newcommand{\mcm}{\mathcal{M}}
\newcommand{\mcc}{\mathcal{C}}
\newcommand{\mcmn}{\mathcal{M}}
\newcommand{\mcmnr}{\mathcal{M}_n(\mtr)}
\newcommand{\mcmnk}{\mathcal{M}_n(\mtk)}
\newcommand{\mcsn}{\mathcal{S}_n}
\newcommand{\mcs}{\mathcal{S}}
\newcommand{\mcd}{\mathcal{D}}
\newcommand{\mcsns}{\mathcal{S}_n^{++}}
\newcommand{\glnk}{GL_n(\mtk)}
\newcommand{\mnr}{\mathcal{M}_n(\mtr)}
\newcommand{\veps}{\varepsilon}
\newcommand{\mcu}{\mathcal{U}}
\newcommand{\mcun}{\mcu_n}
\newcommand{\dis}{\displaystyle}
\newcommand{\croouv}{[\![}
\newcommand{\crofer}{]\!]}
\newcommand{\rab}{\mathcal{R}(a,b)}
\newcommand{\pss}[2]{\langle #1,#2\rangle}

\vspace{4em}

\section*{Exercice 1 - Nombres complexes (6 points)}

\begin{enumerate}
    \item Montrer que $z \overline{z} = |z|^2$ pour tout nombre complexe $z$.

    \item Mettre sous forme exponentielle les nombres complexes suivants :

    $$z_1 = -2\sqrt{3} - 2i, \qquad z_2 = \dfrac{\sqrt{2}(i-1)}{i}, \qquad z_3 = \frac{(1 - \sqrt{3}i)^4}{(1 - i)^5}$$

    \item Ecrire sous forme algébrique les conjugués de ces nombres complexes.

\end{enumerate}


\vspace{3em}

\section*{Exercice 2 - Intégration (8 points)}

Calculer les intégrales suivantes :
$$I = \int_0^1 (x^2 + 1)e^x dx, \qquad  J = \int_1^e x^2\ln (x^3) dx$$

Déterminer une primitive des fonctions suivantes :
$$f(x) = \arctan(x), \qquad g(x) = \cos(x)\sin^2(x)$$

\vspace{2em}



\newpage

\section*{Exercice 5 - Fonctions périodiques (5 points)}

Les fonctions suivantes sont-elles périodiques ? Si oui, déterminer leur période :
\begin{multicols}{3}
\begin{enumerate}
    \item $f_1(x) = \sin(5x)$
    \item $f_2(x) = \tan(x)$
    \item $f_3(x) = \sin^2(x)$
    \item $f_4(x) = \cos(x) + \sin(\sqrt{2}x)$
    \item $f_5(x) = e^{ix} + e^{2ix}$
\end{enumerate}
\end{multicols}


\vspace{2em}

\section*{Exercice 6 - Signal de température et analyse spectrale (5 points)}

Un capteur de température dans un bâtiment enregistre les variations de température.


    Le signal de température est modélisé par :
    $$T(t) = 20 + 5\cos\left(\frac{\pi t}{12}\right) + 2\sin\left(\frac{\pi t}{6}\right) + 0.5\cos\left(\frac{\pi t}{4}\right)$$
    
    où $t$ est en heures et $T$ en degrés Celsius. Le graphe est tracé à la fin de l'exercice.
   \begin{enumerate}
    \item Déterminer sur le graphe et avec le calcul la période de ce signal.
    \item Représenter sur le graphe la sinusoïdale la plus proche du signal.

    \item Calculer les coefficients de Fourier complexes de ce signal.

\begin{center}
    \includegraphics[width=0.9\textwidth]{1-Fourier/Exam/temperature_signal.png}
\end{center}
\end{enumerate}
