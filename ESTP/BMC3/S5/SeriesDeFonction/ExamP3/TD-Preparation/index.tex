\documentclass[12pt]{article}

% Packages pour les marges
\usepackage[
    top=1.5cm,
    bottom=1.5cm,
    left=1.5cm,
    right=1.5cm
]{geometry}

% Police sans serif
% \usepackage{helvet}
% \renewcommand{\familydefault}{\sfdefault}

% Packages existants
\usepackage[french]{babel}
\usepackage[utf8]{inputenc}
\usepackage[T1]{fontenc}
\usepackage{amsmath}
\usepackage{amsfonts}
\usepackage{amssymb}
\usepackage{mathtools}
\usepackage{array}
\usepackage[version=4]{mhchem}
\usepackage{stmaryrd}
\usepackage{enumitem}
\usepackage{ifthen}
\usepackage{eurosym}
\usepackage{textcomp}
\usepackage{graphicx}
\usepackage{xcolor}
\usepackage{multicol}
\definecolor{Theme}{HTML}{0E7490} % teal-700
\definecolor{ThemeLight}{HTML}{E0F2F1}
\definecolor{Accent}{HTML}{F59E0B} % amber-500
\definecolor{Gray}{HTML}{374151}
\usepackage[colorlinks=true,linkcolor=Theme,urlcolor=Theme,citecolor=Theme]{hyperref}

\usepackage{mdframed}
\usepackage[sf]{titlesec}
\usepackage{environ}

% Définition de la variable pour afficher les corrections
\newboolean{showSolutions}
% Décommentez la ligne suivante pour afficher les solutions
\input \jobname.adr

\title{TD de Préparation à l'Examen S5}
\author{}
\date{}

\newenvironment{solution}
    {\par\vspace{0.5em}\begin{mdframed}[backgroundcolor=ThemeLight,linewidth=0.5pt]\noindent\textbf{Solution :}\par}
    {\end{mdframed}\par\vspace{0.5em}}

\begin{document}
\sffamily

\begin{center}
    {\Large\textbf{Révisions -- Mathématiques S5}}
    
    \vspace{0.5em}
    {\textit{BMC3 -- Semestre 5}}
\end{center}

\vspace{1em}

%==============================================================================
\section*{Développements limités}
%==============================================================================

%------------------------------------------------------------------------------
\subsection*{Exercice 1 -- Produit et quotient de DL}
%------------------------------------------------------------------------------

\begin{enumerate}
    \item Calculer le DL de $\dfrac{\sin x}{1+x}$ à l'ordre 3 au voisinage de 0.
    
    \ifthenelse{\boolean{showSolutions}}{
    \begin{solution}
    On multiplie les DL en ne gardant que les termes d'ordre $\leq 3$ :
    \begin{align*}
    \dfrac{\sin x}{1+x} &= \left(x - \dfrac{x^3}{6}\right)\left(1 - x + x^2 - x^3\right) + o(x^3) \\
    &= x - x^2 + x^3 - \dfrac{x^3}{6} + o(x^3) \\
    &= x - x^2 + \dfrac{5x^3}{6} + o(x^3)
    \end{align*}
    \end{solution}
    }{}
    
    \item Calculer le DL de $\dfrac{e^x}{1+x}$ à l'ordre 3.
    
    \ifthenelse{\boolean{showSolutions}}{
    \begin{solution}
    On a $e^x = 1 + x + \dfrac{x^2}{2} + \dfrac{x^3}{6} + o(x^3)$ et $\dfrac{1}{1+x} = 1 - x + x^2 - x^3 + o(x^3)$.
    
    \begin{align*}
    \dfrac{e^x}{1+x} &= \left(1 + x + \dfrac{x^2}{2} + \dfrac{x^3}{6}\right)(1 - x + x^2 - x^3) + o(x^3) \\
    &= 1 - x + x^2 + x - x^2 + x^3 + \dfrac{x^2}{2} - \dfrac{x^3}{2} + \dfrac{x^3}{6} + o(x^3) \\
    &= 1 + \dfrac{x^2}{2} + \dfrac{2x^3}{3} + o(x^3)
    \end{align*}
    \end{solution}
    }{}
\end{enumerate}

\vspace{1em}
%------------------------------------------------------------------------------
\subsection*{Exercice 2 -- DL par composition}
%------------------------------------------------------------------------------

On cherche le DL de $\ln(\cos x)$ à l'ordre 4 au voisinage de 0.

\begin{enumerate}
    \item En posant $u = \cos x - 1$, montrer que $u = -\dfrac{x^2}{2} + \dfrac{x^4}{24} + o(x^4)$.
    
    \ifthenelse{\boolean{showSolutions}}{
    \begin{solution}
    On a $\cos x = 1 - \dfrac{x^2}{2} + \dfrac{x^4}{24} + o(x^4)$, donc :
    
    $u = \cos x - 1 = -\dfrac{x^2}{2} + \dfrac{x^4}{24} + o(x^4)$
    \end{solution}
    }{}
    
    \item En utilisant $\ln(1+u) = u - \dfrac{u^2}{2} + o(u^2)$, en déduire le DL de $\ln(\cos x)$ à l'ordre 4.
    
    \ifthenelse{\boolean{showSolutions}}{
    \begin{solution}
    On a $u^2 = \dfrac{x^4}{4} + o(x^4)$.
    \begin{align*}
    \ln(\cos x) &= \ln(1+u) = u - \dfrac{u^2}{2} + o(u^2) \\
    &= -\dfrac{x^2}{2} + \dfrac{x^4}{24} - \dfrac{1}{2} \cdot \dfrac{x^4}{4} + o(x^4) \\
    &= -\dfrac{x^2}{2} + \dfrac{x^4}{24} - \dfrac{x^4}{8} + o(x^4) \\
    &= -\dfrac{x^2}{2} - \dfrac{x^4}{12} + o(x^4)
    \end{align*}
    \end{solution}
    }{}
\end{enumerate}

\vspace{1em}
%------------------------------------------------------------------------------
\subsection*{Exercice 3 -- Calcul de limites par DL}
%------------------------------------------------------------------------------

Calculer les limites suivantes à l'aide de développements limités :

\begin{enumerate}
    \item $\displaystyle\lim_{x \to 0} \dfrac{\ln(1+x) - x + \frac{x^2}{2}}{x^3}$
    
    \ifthenelse{\boolean{showSolutions}}{
    \begin{solution}
    $\ln(1+x) = x - \dfrac{x^2}{2} + \dfrac{x^3}{3} + o(x^3)$, donc :
    
    $\ln(1+x) - x + \dfrac{x^2}{2} = \dfrac{x^3}{3} + o(x^3)$
    
    $\dfrac{\ln(1+x) - x + \frac{x^2}{2}}{x^3} = \dfrac{1}{3} + o(1) \xrightarrow[x \to 0]{} \boxed{\dfrac{1}{3}}$
    \end{solution}
    }{}
    
    \item $\displaystyle\lim_{x \to 0} \dfrac{\sin x - x\cos x}{x^3}$
    
    \ifthenelse{\boolean{showSolutions}}{
    \begin{solution}
    $\sin x = x - \dfrac{x^3}{6} + o(x^3)$ et $\cos x = 1 - \dfrac{x^2}{2} + o(x^2)$
    
    $x\cos x = x - \dfrac{x^3}{2} + o(x^3)$
    
    $\sin x - x\cos x = x - \dfrac{x^3}{6} - x + \dfrac{x^3}{2} + o(x^3) = \dfrac{x^3}{3} + o(x^3)$
    
    $\dfrac{\sin x - x\cos x}{x^3} = \dfrac{1}{3} + o(1) \xrightarrow[x \to 0]{} \boxed{\dfrac{1}{3}}$
    \end{solution}
    }{}
\end{enumerate}

\vspace{1em}
%------------------------------------------------------------------------------
\subsection*{Exercice 4 -- Prolongement et position par rapport à la tangente}
%------------------------------------------------------------------------------

Soit $f(x) = \dfrac{e^x - 1 - x}{x^2}$ pour $x \neq 0$.

\begin{enumerate}
    \item Montrer que $f$ admet un prolongement par continuité en 0 et déterminer $f(0)$.
    
    \ifthenelse{\boolean{showSolutions}}{
    \begin{solution}
    $e^x = 1 + x + \dfrac{x^2}{2} + \dfrac{x^3}{6} + o(x^3)$, donc $e^x - 1 - x = \dfrac{x^2}{2} + \dfrac{x^3}{6} + o(x^3)$.
    
    $f(x) = \dfrac{e^x - 1 - x}{x^2} = \dfrac{1}{2} + \dfrac{x}{6} + o(x) \xrightarrow[x \to 0]{} \dfrac{1}{2}$
    
    On peut prolonger $f$ par continuité en posant $f(0) = \dfrac{1}{2}$.
    \end{solution}
    }{}
    
    \item Donner le DL de $f$ à l'ordre 2 en 0, puis l'équation de la tangente à la courbe en 0.
    
    \ifthenelse{\boolean{showSolutions}}{
    \begin{solution}
    $e^x - 1 - x = \dfrac{x^2}{2} + \dfrac{x^3}{6} + \dfrac{x^4}{24} + o(x^4)$
    
    $f(x) = \dfrac{1}{2} + \dfrac{x}{6} + \dfrac{x^2}{24} + o(x^2)$
    
    La tangente en 0 a pour équation : $y = \dfrac{1}{2} + \dfrac{x}{6}$
    \end{solution}
    }{}
    
    \item En déduire la position de la courbe par rapport à sa tangente au voisinage de 0.
    
    \ifthenelse{\boolean{showSolutions}}{
    \begin{solution}
    $f(x) - \left(\dfrac{1}{2} + \dfrac{x}{6}\right) = \dfrac{x^2}{24} + o(x^2) \sim \dfrac{x^2}{24} > 0$ pour $x \neq 0$ petit.
    
    Donc la courbe est \textbf{au-dessus} de sa tangente au voisinage de 0.
    \end{solution}
    }{}
\end{enumerate}

\newpage
%==============================================================================
\section*{Équations différentielles exactes}
%==============================================================================

%------------------------------------------------------------------------------
\subsection*{Exercice 5 -- Équation exacte}
%------------------------------------------------------------------------------

Résoudre l'équation différentielle $(2xy + 1)\,dx + (x^2 + 2y)\,dy = 0$.

\begin{enumerate}
    \item Vérifier que l'équation est exacte.
    
    \ifthenelse{\boolean{showSolutions}}{
    \begin{solution}
    On pose $f(x,y) = 2xy + 1$ et $g(x,y) = x^2 + 2y$.
    
    $\dfrac{\partial f}{\partial y} = 2x$ et $\dfrac{\partial g}{\partial x} = 2x$
    
    Comme $\dfrac{\partial f}{\partial y} = \dfrac{\partial g}{\partial x}$, l'équation est exacte.
    \end{solution}
    }{}
    
    \item Trouver $F(x,y)$ telle que $dF = f\,dx + g\,dy$.
    
    \ifthenelse{\boolean{showSolutions}}{
    \begin{solution}
    On cherche $F$ telle que $\dfrac{\partial F}{\partial x} = 2xy + 1$.
    
    En intégrant par rapport à $x$ : $F(x,y) = x^2y + x + H(y)$
    
    On vérifie avec $\dfrac{\partial F}{\partial y} = x^2 + H'(y) = x^2 + 2y$.
    
    Donc $H'(y) = 2y$, soit $H(y) = y^2$.
    
    \textbf{Conclusion :} $F(x,y) = x^2y + x + y^2$
    \end{solution}
    }{}
    
    \item En déduire la solution générale.
    
    \ifthenelse{\boolean{showSolutions}}{
    \begin{solution}
    Les solutions sont données par $F(x,y) = K$ :
    \[\boxed{x^2y + x + y^2 = K}\]
    où $K$ est une constante.
    \end{solution}
    }{}
\end{enumerate}

\vspace{1em}

%==============================================================================
\section*{Équations aux dérivées partielles}
%==============================================================================

%------------------------------------------------------------------------------
\subsection*{Exercice 6 -- EDP}
%------------------------------------------------------------------------------

Résoudre l'équation $2\dfrac{\partial f}{\partial x} + \dfrac{\partial f}{\partial y} = 4x$ par changement de variables.

\ifthenelse{\boolean{showSolutions}}{
\begin{solution}
Avec $X = x - 2y$ et $Y = x$, on a $x = Y$ et $y = \frac{Y-X}{2}$.

On pose $F(X,Y) = f(x,y)$. Par la règle de la chaîne :

$\dfrac{\partial f}{\partial x} = \dfrac{\partial F}{\partial X} \cdot 1 + \dfrac{\partial F}{\partial Y} \cdot 1 = \dfrac{\partial F}{\partial X} + \dfrac{\partial F}{\partial Y}$

$\dfrac{\partial f}{\partial y} = \dfrac{\partial F}{\partial X} \cdot (-2) + \dfrac{\partial F}{\partial Y} \cdot 0 = -2\dfrac{\partial F}{\partial X}$

L'équation devient : $2\left(\dfrac{\partial F}{\partial X} + \dfrac{\partial F}{\partial Y}\right) - 2\dfrac{\partial F}{\partial X} = 4Y$

$2\dfrac{\partial F}{\partial Y} = 4Y$, soit $\dfrac{\partial F}{\partial Y} = 2Y$

En intégrant par rapport à $Y$ : $F(X,Y) = Y^2 + K(X)$

En revenant aux variables initiales : $\boxed{f(x,y) = x^2 + K(x-2y)}$

où $K$ est une fonction $\mathcal{C}^1$ quelconque.
\end{solution}
}{}
\vspace{1em}

Résoudre par séparation de variables : $\dfrac{\partial f}{\partial x} - 3\dfrac{\partial f}{\partial y} = 2f$.

\ifthenelse{\boolean{showSolutions}}{
\begin{solution}
On cherche $f(x,y) = X(x)Y(y)$.

$X'Y - 3XY' = 2XY$

En divisant par $XY$ : $\dfrac{X'}{X} - 3\dfrac{Y'}{Y} = 2$

Donc $\dfrac{X'}{X} = 2 + 3\dfrac{Y'}{Y}$. Le membre de gauche ne dépend que de $x$, le membre de droite que de $y$. Ils sont égaux à une constante $k$.

\textbf{Pour $X$ :} $\dfrac{X'}{X} = k \Rightarrow X(x) = C_1 e^{kx}$

\textbf{Pour $Y$ :} $3\dfrac{Y'}{Y} = k - 2 \Rightarrow Y' = \dfrac{k-2}{3}Y \Rightarrow Y(y) = C_2 e^{(k-2)y/3}$

\textbf{Solutions :} $\boxed{f(x,y) = C \, e^{kx} e^{(k-2)y/3}}$ pour $k \in \mathbb{R}$, $C \in \mathbb{R}$.
\end{solution}
}{}

\vspace{1em}

%==============================================================================
\section*{Modélisation}
%==============================================================================

%------------------------------------------------------------------------------
\subsection*{Exercice 7 -- Refroidissement de Newton}
%------------------------------------------------------------------------------

Un objet de température initiale $T_0 = 80$ °C est placé dans une pièce à température ambiante $T_a = 20$ °C. La loi de Newton stipule que :
\[
\frac{dT}{dt} = -k(T - T_a)
\]
avec $k = 0{,}1$ min$^{-1}$.

\begin{enumerate}
    \item En posant $\theta(t) = T(t) - T_a$, montrer que $\theta' = -k\theta$ et résoudre.
    
    \ifthenelse{\boolean{showSolutions}}{
    \begin{solution}
    $\theta' = T' = -k(T - T_a) = -k\theta$
    
    Donc $\theta(t) = \theta_0 e^{-kt}$ avec $\theta_0 = T_0 - T_a = 80 - 20 = 60$ °C.
    
    $\theta(t) = 60 e^{-0{,}1t}$, donc $T(t) = T_a + \theta(t) = 20 + 60e^{-0{,}1t}$ °C
    \end{solution}
    }{}
    
    \item Calculer la température après 10 minutes. \textit{(On donne $e^{-1} \approx 0{,}37$.)}
    
    \ifthenelse{\boolean{showSolutions}}{
    \begin{solution}
    $T(10) = 20 + 60e^{-1} \approx 20 + 60 \times 0{,}37 = 20 + 22{,}2 = \boxed{42{,}2 \text{ °C}}$
    \end{solution}
    }{}
    
    \item Au bout de combien de temps la température sera-t-elle de 30 °C ? \textit{(On donne $\ln 6 \approx 1{,}8$.)}
    
    \ifthenelse{\boolean{showSolutions}}{
    \begin{solution}
    On résout $T(t) = 30$ :
    
    $20 + 60e^{-0{,}1t} = 30 \Rightarrow 60e^{-0{,}1t} = 10 \Rightarrow e^{-0{,}1t} = \dfrac{1}{6}$
    
    $-0{,}1t = -\ln 6 \Rightarrow t = \dfrac{\ln 6}{0{,}1} = \dfrac{1{,}8}{0{,}1} = \boxed{18 \text{ min}}$
    \end{solution}
    }{}
\end{enumerate}

%------------------------------------------------------------------------------
\subsection*{Exercice 8 -- Croissance bactérienne avec limitation}
%------------------------------------------------------------------------------

Une population de bactéries $N(t)$ croît dans un milieu où les nutriments sont renouvelés à débit constant $D$, mais sont consommés proportionnellement à la population :
\[
\frac{dN}{dt} = D - kN
\]
avec $k = 0{,}5$ h$^{-1}$, $D = 1000$ bactéries/h, et $N(0) = 100$ bactéries.

\begin{enumerate}
    \item Déterminer la population d'équilibre $N_{eq}$.
    
    \ifthenelse{\boolean{showSolutions}}{
    \begin{solution}
    À l'équilibre, $\dfrac{dN}{dt} = 0$, donc $D - kN_{eq} = 0$.
    
    $N_{eq} = \dfrac{D}{k} = \dfrac{1000}{0{,}5} = 2000$ bactéries
    \end{solution}
    }{}
    
    \item Résoudre l'équation et donner $N(t)$.
    
    \ifthenelse{\boolean{showSolutions}}{
    \begin{solution}
    On pose $\theta = N - N_{eq}$. Alors $\theta' = N' = D - kN = D - k(N_{eq} + \theta) = -k\theta$.
    
    Donc $\theta(t) = \theta_0 e^{-kt}$ avec $\theta_0 = N(0) - N_{eq} = 100 - 2000 = -1900$.
    
    $N(t) = N_{eq} + \theta(t) = 2000 - 1900e^{-0{,}5t}$
    
    $\boxed{N(t) = 2000 - 1900e^{-0{,}5t}}$
    \end{solution}
    }{}
    
    \item Au bout de combien de temps la population atteint-elle 95\% de sa valeur d'équilibre ?
    
    \textit{(On donne $\ln 20 \approx 3$.)}
    
    \ifthenelse{\boolean{showSolutions}}{
    \begin{solution}
    On cherche $t$ tel que $N(t) = 0{,}95 \times 2000 = 1900$.
    
    $2000 - 1900e^{-0{,}5t} = 1900 \Rightarrow 1900e^{-0{,}5t} = 100 \Rightarrow e^{-0{,}5t} = \dfrac{1}{19} \approx \dfrac{1}{20}$
    
    $-0{,}5t = -\ln 20 \Rightarrow t = \dfrac{\ln 20}{0{,}5} = \dfrac{3}{0{,}5} = \boxed{6 \text{ h}}$
    \end{solution}
    }{}
\end{enumerate}

\end{document}
