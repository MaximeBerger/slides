% =====================================================================
% Template LaTeX – Traces distribuées aux étudiants
% Auteur : (à compléter)
% Compilation : pdflatex/xelatex (pdflatex recommandé ici)
% =====================================================================
\documentclass[11pt,a4paper]{report}

% -------------------- Encodage & langue --------------------
\usepackage[T1]{fontenc}
\usepackage[utf8]{inputenc}
\usepackage[french]{babel}
\usepackage{lmodern}
\usepackage{microtype}
\usepackage{amsmath, amssymb}
\usepackage{multicol}
\usepackage{enumitem}
\usepackage{ifthen}
\usepackage{url}

\DeclareMathOperator{\ch}{ch}
\DeclareMathOperator{\sh}{sh}
\DeclareMathOperator{\vect}{vect}
\DeclareMathOperator{\card}{card}
\DeclareMathOperator{\comat}{comat}
\DeclareMathOperator{\imv}{Im}
\DeclareMathOperator{\rang}{rg}
\DeclareMathOperator{\Fr}{Fr}
\DeclareMathOperator{\diam}{diam}
\DeclareMathOperator{\supp}{supp}

% -------------------- Mise en page --------------------------
\usepackage[a4paper,margin=2cm]{geometry}
\usepackage{fancyhdr}
\usepackage{parskip}      % espace entre paragraphes
\setlength{\parindent}{0pt}

% -------------------- Couleurs & liens ----------------------
\usepackage{xcolor}
\definecolor{Theme}{HTML}{0E7490} % teal-700
\definecolor{ThemeLight}{HTML}{E0F2F1}
\definecolor{Accent}{HTML}{F59E0B} % amber-500
\definecolor{Gray}{HTML}{374151}
\usepackage[colorlinks=true,linkcolor=Theme,urlcolor=Theme,citecolor=Theme]{hyperref}

% -------------------- Graphiques / décor --------------------
\usepackage{tikz}
\usetikzlibrary{patterns,positioning,calc}
\usepackage{graphicx}
\usepackage{tcolorbox}
\tcbuselibrary{skins,breakable,hooks,most}
\usepackage{fontawesome5}

% -------------------- Titres -------------------------------
\usepackage{titlesec}
\titleformat{\chapter}[display]
  {\huge\bfseries\color{Theme}}
  {\filright\rule{0.75\linewidth}{1.2pt}\\[1pt]{Séries de fonctions - Chapitre~\thechapter}}
  {0.1ex}
  {\filright}
  [\vspace{0ex}\rule{0.35\linewidth}{1.2pt}]

\titlespacing*{\chapter}{0pt}{0.5ex}{0.25ex}

\titleformat{\section}
  {\Large\bfseries\color{Gray}}
  {\thesection}{0.6em}{}

% -------------------- En-têtes / pieds ---------------------
\pagestyle{fancy}
\fancyhf{}
\fancyhead[L]{\color{Gray}\leftmark}
\fancyhead[R]{\color{Gray}\textit{BMC3}}
\fancyfoot[L]{\color{Gray}\small Auteur~: \textit{M. Berger}}
\fancyfoot[R]{\color{Gray}\small p.\ \thepage}
\renewcommand{\headrulewidth}{0pt}
\renewcommand{\footrulewidth}{0pt}

% -------------------- Macros utilitaires -------------------


% Tcolorboxes stylisées
\tcbset{tracebox/.style={breakable,enhanced,sharp corners,boxrule=0pt,frame hidden,arc=2mm,
  colback=white,coltitle=black,fonttitle=\bfseries\large,
  borderline west={2mm}{0pt}{Theme},
  before skip=8pt,after skip=8pt,drop fuzzy shadow}}

\newtcolorbox{resumeBox}{tracebox,title={\faStickyNote\quad Résumé des idées}}
\newtcolorbox{rappelsBox}{tracebox,title={\faRedo\quad Questions de cours }}
\newtcolorbox{exempleBox}{tracebox,title={\faChalkboardTeacher\quad Exemple vu ensemble}}

% Encadré « Formules & illustrations »
\newtcolorbox{formulesBox}{tracebox,title={\faCalculator\quad Formules \& illustrations},colback=ThemeLight}

% Astuce : puces clean
\newenvironment{niceitemize}{\begin{itemize}\setlength{\itemsep}{0.25em}}{\end{itemize}}

% Raccourci pour une « Trace » complète
% Usage : \TraceSection{Titre}{Objectif court}
\newcommand{\TraceSection}[2]{%
  
}

% -------------------- Page de titre ------------------------
\title{\textbf{Traces de cours}\\\large (résumés, formules, exemples, mini-exercices)}
\author{ BMC3 }
\date{\today}


\makeatletter
\renewcommand{\thesubsection}{\arabic{subsection}}
\renewcommand{\p@subsection}{}% supprime le préfixe section/chapter dans \ref
% Si vous voulez la même chose pour les sous-sous-sections :
% \renewcommand{\thesubsubsection}{\arabic{subsubsection}}
% \renewcommand{\p@subsubsection}{}
\makeatother

\usepackage{mdframed}
\usepackage{ifthen}

% \usepackage[sf]{titlesec}
% Définition de la variable pour afficher les corrections
\newboolean{showSolutions}
% Décommentez la ligne suivante pour afficher les solutions
\input \jobname.adr

% -------------------- Document ----------------------------
\begin{document}



% ================== Séquence 1 ==================

% \chapter{Suites et séries numériques}
% \section*{Les fonctions périodiques}

\subsection{Les fonctions complexes périodiques}

Les fonctions réelles suivantes sont-elles périodiques et si oui, quelle est leur période ?

\ifthenelse{\boolean{showSolutions}}{
    \vspace{2em}
    \begin{mdframed}
    \begin{enumerate}
    \item $\cos(x)$ : Oui, période $T = 2\pi$
    \item $\sin(2\pi x)$ : Oui, période $T = 1$ (car $\sin(2\pi(x+1)) = \sin(2\pi x + 2\pi) = \sin(2\pi x)$)
    \item $\cos(x/2)$ : Oui, période $T = 4\pi$ (car $\cos((x+4\pi)/2) = \cos(x/2 + 2\pi) = \cos(x/2)$)
    \item $\sin(2x) + \cos(3x)$ : Oui, période $T = 2\pi$ (le PPCM des périodes $\pi$ et $\frac{2\pi}{3}$)
    \item $\sin(nx)$ : Oui, période $T = \frac{2\pi}{n}$
    \item $\cos\left(\frac{3x}{2}-\frac{\pi}{4}\right)$ : Oui, période $T = \frac{4\pi}{3}$
    \item $x-\lfloor x\rfloor$ : Oui, période $T = 1$ (c'est la fonction partie fractionnaire)
\end{enumerate}
\end{mdframed}
}{}
\ifthenelse{\boolean{showSolutions}}{}
{\begin{multicols}{2}}
\begin{enumerate}
    \item $\displaystyle \cos(x)$
    
    \item $\displaystyle \sin(2\pi x)$
    \item $\displaystyle \cos(x/2)$
    \item $\displaystyle \sin(2x) + \cos(3x)$
    \item $\displaystyle \sin(nx)$, $n$ est un entier naturel non nul
    \item $\displaystyle \cos \left(\frac{3 x}{2}-\frac{\pi}{4}\right)$
    \item $\displaystyle x-\lfloor x\rfloor$
\end{enumerate}
\ifthenelse{\boolean{showSolutions}}{}{
\end{multicols}
}

Les fonctions complexes suivantes sont-elles périodiques et si oui, quelle est leur période ?

\ifthenelse{\boolean{showSolutions}}{
    \vspace{2em}
    \begin{mdframed}
    \begin{enumerate}
    \item $e^{ix}$ : Oui, période $T = 2\pi$ (car $e^{i(x+2\pi)} = e^{ix}e^{i2\pi} = e^{ix} \cdot 1 = e^{ix}$)
    \item $e^{2ix}$ : Oui, période $T = \pi$ (car $e^{2i(x+\pi)} = e^{2ix}e^{i2\pi} = e^{2ix}$)
    \item $e^{ix/2\pi}$ : Oui, période $T = 4\pi^2$ (car $e^{i(x+4\pi^2)/2\pi} = e^{ix/2\pi}e^{i2\pi} = e^{ix/2\pi}$)
    \item $e^{2i\pi x/T}$ : Oui, période $T$ (car $e^{2i\pi (x+T)/T} = e^{2i\pi x/T}e^{i2\pi} = e^{2i\pi x/T}$)
    \item $e^{inx} + e^{ipx}$ : Oui, période $T = \frac{2\pi}{\text{PGCD}(n,p)}$ (le PPCM des périodes $\frac{2\pi}{n}$ et $\frac{2\pi}{p}$)
\end{enumerate}
\end{mdframed}
}{}
\ifthenelse{\boolean{showSolutions}}{}
{\begin{multicols}{2}}
\begin{enumerate}
    \item $\displaystyle e^{ix}$
    \item $\displaystyle e^{2ix}$
    \item $\displaystyle e^{ix/2\pi}$
    \item $\displaystyle e^{2i\pi x/T}$, $T$ est un réel strictement positif
    \item $\displaystyle e^{inx} + e^{ipx}$
\end{enumerate}
\ifthenelse{\boolean{showSolutions}}{}{
\end{multicols}
}


\section*{Produit scalaire réel}

\subsection{Définition}
On appelle produit scalaire sur un espace vectoriel $E$ une application 
$$\langle \cdot, \cdot \rangle : E \times E \to \mathbb{R}$$
telle que :
\begin{multicols}{2}
\begin{itemize}
    \item[*] symétrie : $\langle u, v \rangle = \langle v, u \rangle$ 
    \item[*] linéarité à gauche : $\langle \lambda u + v, w \rangle = \lambda \langle u, w \rangle + \langle v, w \rangle$
    \item[*] positivité : $\langle u, u \rangle \geq 0$
    \item[*] définie positivité : $\langle u, u \rangle = 0 \iff u = 0$
\end{itemize}
\end{multicols}


\vspace{1em}

\subsection{Dans $\mathbb{R}^3$}

On se place dans $\mathbb{R}^3$, qu'on munit de la base 
$$e_1 = (1,2,1), \qquad e_2 = (2,1,-4), \qquad e_3 = (-3,2,-1)$$

\begin{enumerate}
\item La famille est-elle orthogonale ? 
\item Est-elle orthonormée ? Si non, définissez une base $(f_1,f_2,f_3)$ orthonormée à partir de la famille $(e_1,e_2,e_3)$. 
\end{enumerate}

Soit $u$ un vecteur de $\mathbb{R}^3$, on note $u_i$ ses coordonnées dans la base orthonormée $(f_1,f_2,f_3)$. Cela signifie que 
$$u = u_1 f_1 + u_2 f_2 + u_3 f_3$$

Déterminer les coordonnées de $u = (1,0,1)$ dans la base $(f_1,f_2,f_3)$.

\ifthenelse{\boolean{showSolutions}}{
    \vspace{2em}
    \begin{mdframed}
    \textbf{1.} Vérifions si la famille est orthogonale :
    
    $\langle e_1, e_2 \rangle = 1 \cdot 2 + 2 \cdot 1 + 1 \cdot (-4) = 2 + 2 - 4 = 0$ $\checkmark$
    
    $\langle e_1, e_3 \rangle = 1 \cdot (-3) + 2 \cdot 2 + 1 \cdot (-1) = -3 + 4 - 1 = 0$ $\checkmark$
    
    $\langle e_2, e_3 \rangle = 2 \cdot (-3) + 1 \cdot 2 + (-4) \cdot (-1) = -6 + 2 + 4 = 0$ $\checkmark$
    
    La famille est orthogonale.
    
    \textbf{2.} Vérifions si elle est orthonormée :
    
    $\|e_1\|^2 = 1^2 + 2^2 + 1^2 = 6$, donc $\|e_1\| = \sqrt{6}$
    
    $\|e_2\|^2 = 2^2 + 1^2 + (-4)^2 = 4 + 1 + 16 = 21$, donc $\|e_2\| = \sqrt{21}$
    
    $\|e_3\|^2 = (-3)^2 + 2^2 + (-1)^2 = 9 + 4 + 1 = 14$, donc $\|e_3\| = \sqrt{14}$
    
    La famille n'est pas orthonormée. Une base orthonormée est :
    $$f_1 = \frac{e_1}{\|e_1\|} = \frac{1}{\sqrt{6}}(1,2,1) = \left(\frac{1}{\sqrt{6}}, \frac{2}{\sqrt{6}}, \frac{1}{\sqrt{6}}\right)$$
    
    $$f_2 = \frac{e_2}{\|e_2\|} = \frac{1}{\sqrt{21}}(2,1,-4) = \left(\frac{2}{\sqrt{21}}, \frac{1}{\sqrt{21}}, \frac{-4}{\sqrt{21}}\right)$$
    
    $$f_3 = \frac{e_3}{\|e_3\|} = \frac{1}{\sqrt{14}}(-3,2,-1) = \left(\frac{-3}{\sqrt{14}}, \frac{2}{\sqrt{14}}, \frac{-1}{\sqrt{14}}\right)$$
    
    \textbf{3.} Coordonnées de $u = (1,0,1)$ dans la base $(f_1,f_2,f_3)$ :
    
    $u_1 = \langle u, f_1 \rangle = 1 \cdot \frac{1}{\sqrt{6}} + 0 \cdot \frac{2}{\sqrt{6}} + 1 \cdot \frac{1}{\sqrt{6}} = \frac{2}{\sqrt{6}} = \frac{\sqrt{6}}{3}$
    
    $u_2 = \langle u, f_2 \rangle = 1 \cdot \frac{2}{\sqrt{21}} + 0 \cdot \frac{1}{\sqrt{21}} + 1 \cdot \frac{-4}{\sqrt{21}} = \frac{-2}{\sqrt{21}}$
    
    $u_3 = \langle u, f_3 \rangle = 1 \cdot \frac{-3}{\sqrt{14}} + 0 \cdot \frac{2}{\sqrt{14}} + 1 \cdot \frac{-1}{\sqrt{14}} = \frac{-4}{\sqrt{14}}$
    
    Donc $u = \frac{\sqrt{6}}{3}f_1 - \frac{2}{\sqrt{21}}f_2 - \frac{4}{\sqrt{14}}f_3$
\end{mdframed}
}{}

\vspace{1em}

\subsection{Dans $\mathbb{R}[X]$}

\begin{itemize}
    \item Quelle est la dimension de $\mathbb{R}[X]$ ?
\end{itemize}
La famille $(1,X,X^2,X^3, \cdots )$ est appelée base hilbertienne de $\mathbb{R}[X]$ : tout élément de $\mathbb{R}[X]$ peut s'écrire comme une combinaison linéaire finie de vecteurs de cette famille.

On munit cet espace du produit scalaire : 
$$ \langle P, Q \rangle = \int_{0}^{1} P(x) Q(x) dx $$

\begin{itemize}
    \item Montrer que c'est bien un produit scalaire en vérifiant les propriétés ci-dessus.
    \item La famille $(1,X,X^2,X^3, \cdots )$ est-elle orthogonale ? Est-elle orthonormée ?
    \item Comment trouver $a, b, c$ tels que la famille $(1, X-a, X^2-bX-c)$ soit orthogonale ?
    \item Quelles sont les coordonnées de $P = 1+2X+3X^2$ dans la base $(1, X, X^2, \cdots)$ ?
    \item Peut-on retrouver ces coordonnées avec le produit scalaire comme dans l'exercice précédent ?
\end{itemize}
\vspace{1em}

\section*{Produit scalaire complexe}
\subsection{Définition}
On appelle produit scalaire sur un espace vectoriel $E$ une application 
$$\langle \cdot, \cdot \rangle : E \times E \to \mathbb{R}$$
telle que :
\begin{multicols}{2}
\begin{itemize}
    \item[*] symétrie conjuguée : $\langle u, v \rangle = \overline{\langle v, u \rangle}$
    \item[*] linéarité à gauche : $\langle \lambda u + v, w \rangle = \lambda \langle u, w \rangle + \langle v, w \rangle$
    \item[*] positivité : $\langle u, u \rangle \geq 0$
    \item[*] définie positivité : $\langle u, u \rangle = 0 \iff u = 0$
\end{itemize}
\end{multicols}


\subsection{Dans l'espace des fonctions complexes $2\pi$-périodiques}

On définit le produit scalaire :
$$
\langle f, g \rangle = \int_{0}^{2\pi} f(x) \overline{g(x)} dx
$$

Montrer que c'est un produit scalaire.

Montrer que la famille $(e^{inx})_{n \in \mathbb{Z}}$ est orthonormée.

Les coefficients de Fourier d'une fonction $f$ sont les coordonnées de $f$ dans la base $(e^{inx})_{n \in \mathbb{Z}}$.

Déterminer les coefficients de Fourier des fonctions suivantes :

\ifthenelse{\boolean{showSolutions}}{
    \vspace{2em}
    \begin{mdframed}
    \textbf{Démonstration que c'est un produit scalaire :}
    
    Les propriétés de symétrie conjuguée, linéarité et positivité découlent des propriétés de l'intégrale et du conjugué complexe.
    
    \textbf{Démonstration que $(e^{inx})_{n \in \mathbb{Z}}$ est orthonormée :}
    
    Pour $n = m$ : $\langle e^{inx}, e^{inx} \rangle = \int_0^{2\pi} e^{inx} \overline{e^{inx}} dx = \int_0^{2\pi} 1 dx = 2\pi$
    
    Pour $n \neq m$ : $\langle e^{inx}, e^{imx} \rangle = \int_0^{2\pi} e^{inx} \overline{e^{imx}} dx = \int_0^{2\pi} e^{i(n-m)x} dx = 0$
    
    Donc la famille $(e^{inx})_{n \in \mathbb{Z}}$ est orthogonale. Pour l'orthonormaliser, on divise par $\sqrt{2\pi}$.
    
    \textbf{Coefficients de Fourier :}
    \begin{enumerate}
    \item $\cos(x) = \frac{e^{ix} + e^{-ix}}{2}$, donc $c_1 = \frac{1}{2}$, $c_{-1} = \frac{1}{2}$, $c_n = 0$ sinon.
    
    \item $\sin(2\pi x)$ : Cette fonction n'est pas $2\pi$-périodique ! Elle est de période $1$.
    
    \item $\cos(x/2) = \frac{e^{ix/2} + e^{-ix/2}}{2}$, donc $c_{1/2} = \frac{1}{2}$, $c_{-1/2} = \frac{1}{2}$, $c_n = 0$ sinon.
    
    \item $\sin(2x) + \cos(3x) = \frac{e^{i2x} - e^{-i2x}}{2i} + \frac{e^{i3x} + e^{-i3x}}{2}$
    Donc $c_2 = \frac{1}{2i}$, $c_{-2} = -\frac{1}{2i}$, $c_3 = \frac{1}{2}$, $c_{-3} = \frac{1}{2}$, $c_n = 0$ sinon.
    
    \item $f(x) = e^{-x}$ sur $[0, 2\pi]$ :
    $$c_n = \frac{1}{2\pi}\int_0^{2\pi} e^{-x} e^{-inx} dx = \frac{1}{2\pi}\int_0^{2\pi} e^{-(1+in)x} dx$$
    $$= \frac{1}{2\pi}\left[\frac{e^{-(1+in)x}}{-(1+in)}\right]_0^{2\pi} = \frac{1}{2\pi} \cdot \frac{1-e^{-2\pi(1+in)}}{1+in} = \frac{1-e^{-2\pi}e^{-2\pi in}}{2\pi(1+in)}$$
\end{enumerate}
\end{mdframed}
}{}
\ifthenelse{\boolean{showSolutions}}{}
{\begin{multicols}{2}}
\begin{enumerate}
\item $\displaystyle \cos(x)$
\item $\displaystyle \sin(2\pi x)$
\item $\displaystyle \cos(x/2)$
\item $\displaystyle \sin(2x) + \cos(3x)$
\item $\displaystyle \exp^{-x}$ sur l'intervalle $[0, 2\pi]$
\end{enumerate}
\ifthenelse{\boolean{showSolutions}}{}{
\end{multicols}
}


% \setcounter{chapter}{0}
% \chapter{Contrôle continu}
% Quelles sont les $3$ questions qu'on doit se poser pour montrer qu'un ensemble est un espace vectoriel ?

\vspace{2em}

Les ensembles suivants sont-ils des espaces vectoriels inclus dans $\mathbb{R}^n$ ? 
$$ \mathrm{F}=\left\{\left(x_1, \ldots, x_{\mathrm{n}}\right) \in \mathbb{R}^{\mathrm{n}} / x_1+\ldots+x_{\mathrm{n}}=0\right\} \qquad 
    \mathrm{F}=\left\{\left(x_1, \ldots, x_{\mathrm{n}}\right) \in \mathbb{R}^{\mathrm{n}} / x_1 \times x_2=0\right\}
    $$

% \setcounter{chapter}{1}
% \chapter{Révisions sur les intégrales}
% \section*{Les fonctions périodiques}

\subsection{Les fonctions complexes périodiques}

Les fonctions réelles suivantes sont-elles périodiques et si oui, quelle est leur période ?

\ifthenelse{\boolean{showSolutions}}{
    \vspace{2em}
    \begin{mdframed}
    \begin{enumerate}
    \item $\cos(x)$ : Oui, période $T = 2\pi$
    \item $\sin(2\pi x)$ : Oui, période $T = 1$ (car $\sin(2\pi(x+1)) = \sin(2\pi x + 2\pi) = \sin(2\pi x)$)
    \item $\cos(x/2)$ : Oui, période $T = 4\pi$ (car $\cos((x+4\pi)/2) = \cos(x/2 + 2\pi) = \cos(x/2)$)
    \item $\sin(2x) + \cos(3x)$ : Oui, période $T = 2\pi$ (le PPCM des périodes $\pi$ et $\frac{2\pi}{3}$)
    \item $\sin(nx)$ : Oui, période $T = \frac{2\pi}{n}$
    \item $\cos\left(\frac{3x}{2}-\frac{\pi}{4}\right)$ : Oui, période $T = \frac{4\pi}{3}$
    \item $x-\lfloor x\rfloor$ : Oui, période $T = 1$ (c'est la fonction partie fractionnaire)
\end{enumerate}
\end{mdframed}
}{}
\ifthenelse{\boolean{showSolutions}}{}
{\begin{multicols}{2}}
\begin{enumerate}
    \item $\displaystyle \cos(x)$
    
    \item $\displaystyle \sin(2\pi x)$
    \item $\displaystyle \cos(x/2)$
    \item $\displaystyle \sin(2x) + \cos(3x)$
    \item $\displaystyle \sin(nx)$, $n$ est un entier naturel non nul
    \item $\displaystyle \cos \left(\frac{3 x}{2}-\frac{\pi}{4}\right)$
    \item $\displaystyle x-\lfloor x\rfloor$
\end{enumerate}
\ifthenelse{\boolean{showSolutions}}{}{
\end{multicols}
}

Les fonctions complexes suivantes sont-elles périodiques et si oui, quelle est leur période ?

\ifthenelse{\boolean{showSolutions}}{
    \vspace{2em}
    \begin{mdframed}
    \begin{enumerate}
    \item $e^{ix}$ : Oui, période $T = 2\pi$ (car $e^{i(x+2\pi)} = e^{ix}e^{i2\pi} = e^{ix} \cdot 1 = e^{ix}$)
    \item $e^{2ix}$ : Oui, période $T = \pi$ (car $e^{2i(x+\pi)} = e^{2ix}e^{i2\pi} = e^{2ix}$)
    \item $e^{ix/2\pi}$ : Oui, période $T = 4\pi^2$ (car $e^{i(x+4\pi^2)/2\pi} = e^{ix/2\pi}e^{i2\pi} = e^{ix/2\pi}$)
    \item $e^{2i\pi x/T}$ : Oui, période $T$ (car $e^{2i\pi (x+T)/T} = e^{2i\pi x/T}e^{i2\pi} = e^{2i\pi x/T}$)
    \item $e^{inx} + e^{ipx}$ : Oui, période $T = \frac{2\pi}{\text{PGCD}(n,p)}$ (le PPCM des périodes $\frac{2\pi}{n}$ et $\frac{2\pi}{p}$)
\end{enumerate}
\end{mdframed}
}{}
\ifthenelse{\boolean{showSolutions}}{}
{\begin{multicols}{2}}
\begin{enumerate}
    \item $\displaystyle e^{ix}$
    \item $\displaystyle e^{2ix}$
    \item $\displaystyle e^{ix/2\pi}$
    \item $\displaystyle e^{2i\pi x/T}$, $T$ est un réel strictement positif
    \item $\displaystyle e^{inx} + e^{ipx}$
\end{enumerate}
\ifthenelse{\boolean{showSolutions}}{}{
\end{multicols}
}


\section*{Produit scalaire réel}

\subsection{Définition}
On appelle produit scalaire sur un espace vectoriel $E$ une application 
$$\langle \cdot, \cdot \rangle : E \times E \to \mathbb{R}$$
telle que :
\begin{multicols}{2}
\begin{itemize}
    \item[*] symétrie : $\langle u, v \rangle = \langle v, u \rangle$ 
    \item[*] linéarité à gauche : $\langle \lambda u + v, w \rangle = \lambda \langle u, w \rangle + \langle v, w \rangle$
    \item[*] positivité : $\langle u, u \rangle \geq 0$
    \item[*] définie positivité : $\langle u, u \rangle = 0 \iff u = 0$
\end{itemize}
\end{multicols}


\vspace{1em}

\subsection{Dans $\mathbb{R}^3$}

On se place dans $\mathbb{R}^3$, qu'on munit de la base 
$$e_1 = (1,2,1), \qquad e_2 = (2,1,-4), \qquad e_3 = (-3,2,-1)$$

\begin{enumerate}
\item La famille est-elle orthogonale ? 
\item Est-elle orthonormée ? Si non, définissez une base $(f_1,f_2,f_3)$ orthonormée à partir de la famille $(e_1,e_2,e_3)$. 
\end{enumerate}

Soit $u$ un vecteur de $\mathbb{R}^3$, on note $u_i$ ses coordonnées dans la base orthonormée $(f_1,f_2,f_3)$. Cela signifie que 
$$u = u_1 f_1 + u_2 f_2 + u_3 f_3$$

Déterminer les coordonnées de $u = (1,0,1)$ dans la base $(f_1,f_2,f_3)$.

\ifthenelse{\boolean{showSolutions}}{
    \vspace{2em}
    \begin{mdframed}
    \textbf{1.} Vérifions si la famille est orthogonale :
    
    $\langle e_1, e_2 \rangle = 1 \cdot 2 + 2 \cdot 1 + 1 \cdot (-4) = 2 + 2 - 4 = 0$ $\checkmark$
    
    $\langle e_1, e_3 \rangle = 1 \cdot (-3) + 2 \cdot 2 + 1 \cdot (-1) = -3 + 4 - 1 = 0$ $\checkmark$
    
    $\langle e_2, e_3 \rangle = 2 \cdot (-3) + 1 \cdot 2 + (-4) \cdot (-1) = -6 + 2 + 4 = 0$ $\checkmark$
    
    La famille est orthogonale.
    
    \textbf{2.} Vérifions si elle est orthonormée :
    
    $\|e_1\|^2 = 1^2 + 2^2 + 1^2 = 6$, donc $\|e_1\| = \sqrt{6}$
    
    $\|e_2\|^2 = 2^2 + 1^2 + (-4)^2 = 4 + 1 + 16 = 21$, donc $\|e_2\| = \sqrt{21}$
    
    $\|e_3\|^2 = (-3)^2 + 2^2 + (-1)^2 = 9 + 4 + 1 = 14$, donc $\|e_3\| = \sqrt{14}$
    
    La famille n'est pas orthonormée. Une base orthonormée est :
    $$f_1 = \frac{e_1}{\|e_1\|} = \frac{1}{\sqrt{6}}(1,2,1) = \left(\frac{1}{\sqrt{6}}, \frac{2}{\sqrt{6}}, \frac{1}{\sqrt{6}}\right)$$
    
    $$f_2 = \frac{e_2}{\|e_2\|} = \frac{1}{\sqrt{21}}(2,1,-4) = \left(\frac{2}{\sqrt{21}}, \frac{1}{\sqrt{21}}, \frac{-4}{\sqrt{21}}\right)$$
    
    $$f_3 = \frac{e_3}{\|e_3\|} = \frac{1}{\sqrt{14}}(-3,2,-1) = \left(\frac{-3}{\sqrt{14}}, \frac{2}{\sqrt{14}}, \frac{-1}{\sqrt{14}}\right)$$
    
    \textbf{3.} Coordonnées de $u = (1,0,1)$ dans la base $(f_1,f_2,f_3)$ :
    
    $u_1 = \langle u, f_1 \rangle = 1 \cdot \frac{1}{\sqrt{6}} + 0 \cdot \frac{2}{\sqrt{6}} + 1 \cdot \frac{1}{\sqrt{6}} = \frac{2}{\sqrt{6}} = \frac{\sqrt{6}}{3}$
    
    $u_2 = \langle u, f_2 \rangle = 1 \cdot \frac{2}{\sqrt{21}} + 0 \cdot \frac{1}{\sqrt{21}} + 1 \cdot \frac{-4}{\sqrt{21}} = \frac{-2}{\sqrt{21}}$
    
    $u_3 = \langle u, f_3 \rangle = 1 \cdot \frac{-3}{\sqrt{14}} + 0 \cdot \frac{2}{\sqrt{14}} + 1 \cdot \frac{-1}{\sqrt{14}} = \frac{-4}{\sqrt{14}}$
    
    Donc $u = \frac{\sqrt{6}}{3}f_1 - \frac{2}{\sqrt{21}}f_2 - \frac{4}{\sqrt{14}}f_3$
\end{mdframed}
}{}

\vspace{1em}

\subsection{Dans $\mathbb{R}[X]$}

\begin{itemize}
    \item Quelle est la dimension de $\mathbb{R}[X]$ ?
\end{itemize}
La famille $(1,X,X^2,X^3, \cdots )$ est appelée base hilbertienne de $\mathbb{R}[X]$ : tout élément de $\mathbb{R}[X]$ peut s'écrire comme une combinaison linéaire finie de vecteurs de cette famille.

On munit cet espace du produit scalaire : 
$$ \langle P, Q \rangle = \int_{0}^{1} P(x) Q(x) dx $$

\begin{itemize}
    \item Montrer que c'est bien un produit scalaire en vérifiant les propriétés ci-dessus.
    \item La famille $(1,X,X^2,X^3, \cdots )$ est-elle orthogonale ? Est-elle orthonormée ?
    \item Comment trouver $a, b, c$ tels que la famille $(1, X-a, X^2-bX-c)$ soit orthogonale ?
    \item Quelles sont les coordonnées de $P = 1+2X+3X^2$ dans la base $(1, X, X^2, \cdots)$ ?
    \item Peut-on retrouver ces coordonnées avec le produit scalaire comme dans l'exercice précédent ?
\end{itemize}
\vspace{1em}

\section*{Produit scalaire complexe}
\subsection{Définition}
On appelle produit scalaire sur un espace vectoriel $E$ une application 
$$\langle \cdot, \cdot \rangle : E \times E \to \mathbb{R}$$
telle que :
\begin{multicols}{2}
\begin{itemize}
    \item[*] symétrie conjuguée : $\langle u, v \rangle = \overline{\langle v, u \rangle}$
    \item[*] linéarité à gauche : $\langle \lambda u + v, w \rangle = \lambda \langle u, w \rangle + \langle v, w \rangle$
    \item[*] positivité : $\langle u, u \rangle \geq 0$
    \item[*] définie positivité : $\langle u, u \rangle = 0 \iff u = 0$
\end{itemize}
\end{multicols}


\subsection{Dans l'espace des fonctions complexes $2\pi$-périodiques}

On définit le produit scalaire :
$$
\langle f, g \rangle = \int_{0}^{2\pi} f(x) \overline{g(x)} dx
$$

Montrer que c'est un produit scalaire.

Montrer que la famille $(e^{inx})_{n \in \mathbb{Z}}$ est orthonormée.

Les coefficients de Fourier d'une fonction $f$ sont les coordonnées de $f$ dans la base $(e^{inx})_{n \in \mathbb{Z}}$.

Déterminer les coefficients de Fourier des fonctions suivantes :

\ifthenelse{\boolean{showSolutions}}{
    \vspace{2em}
    \begin{mdframed}
    \textbf{Démonstration que c'est un produit scalaire :}
    
    Les propriétés de symétrie conjuguée, linéarité et positivité découlent des propriétés de l'intégrale et du conjugué complexe.
    
    \textbf{Démonstration que $(e^{inx})_{n \in \mathbb{Z}}$ est orthonormée :}
    
    Pour $n = m$ : $\langle e^{inx}, e^{inx} \rangle = \int_0^{2\pi} e^{inx} \overline{e^{inx}} dx = \int_0^{2\pi} 1 dx = 2\pi$
    
    Pour $n \neq m$ : $\langle e^{inx}, e^{imx} \rangle = \int_0^{2\pi} e^{inx} \overline{e^{imx}} dx = \int_0^{2\pi} e^{i(n-m)x} dx = 0$
    
    Donc la famille $(e^{inx})_{n \in \mathbb{Z}}$ est orthogonale. Pour l'orthonormaliser, on divise par $\sqrt{2\pi}$.
    
    \textbf{Coefficients de Fourier :}
    \begin{enumerate}
    \item $\cos(x) = \frac{e^{ix} + e^{-ix}}{2}$, donc $c_1 = \frac{1}{2}$, $c_{-1} = \frac{1}{2}$, $c_n = 0$ sinon.
    
    \item $\sin(2\pi x)$ : Cette fonction n'est pas $2\pi$-périodique ! Elle est de période $1$.
    
    \item $\cos(x/2) = \frac{e^{ix/2} + e^{-ix/2}}{2}$, donc $c_{1/2} = \frac{1}{2}$, $c_{-1/2} = \frac{1}{2}$, $c_n = 0$ sinon.
    
    \item $\sin(2x) + \cos(3x) = \frac{e^{i2x} - e^{-i2x}}{2i} + \frac{e^{i3x} + e^{-i3x}}{2}$
    Donc $c_2 = \frac{1}{2i}$, $c_{-2} = -\frac{1}{2i}$, $c_3 = \frac{1}{2}$, $c_{-3} = \frac{1}{2}$, $c_n = 0$ sinon.
    
    \item $f(x) = e^{-x}$ sur $[0, 2\pi]$ :
    $$c_n = \frac{1}{2\pi}\int_0^{2\pi} e^{-x} e^{-inx} dx = \frac{1}{2\pi}\int_0^{2\pi} e^{-(1+in)x} dx$$
    $$= \frac{1}{2\pi}\left[\frac{e^{-(1+in)x}}{-(1+in)}\right]_0^{2\pi} = \frac{1}{2\pi} \cdot \frac{1-e^{-2\pi(1+in)}}{1+in} = \frac{1-e^{-2\pi}e^{-2\pi in}}{2\pi(1+in)}$$
\end{enumerate}
\end{mdframed}
}{}
\ifthenelse{\boolean{showSolutions}}{}
{\begin{multicols}{2}}
\begin{enumerate}
\item $\displaystyle \cos(x)$
\item $\displaystyle \sin(2\pi x)$
\item $\displaystyle \cos(x/2)$
\item $\displaystyle \sin(2x) + \cos(3x)$
\item $\displaystyle \exp^{-x}$ sur l'intervalle $[0, 2\pi]$
\end{enumerate}
\ifthenelse{\boolean{showSolutions}}{}{
\end{multicols}
}


% \setcounter{chapter}{2}
% \chapter{Révisions sur les complexes}
% \section*{Les fonctions périodiques}

\subsection{Les fonctions complexes périodiques}

Les fonctions réelles suivantes sont-elles périodiques et si oui, quelle est leur période ?

\ifthenelse{\boolean{showSolutions}}{
    \vspace{2em}
    \begin{mdframed}
    \begin{enumerate}
    \item $\cos(x)$ : Oui, période $T = 2\pi$
    \item $\sin(2\pi x)$ : Oui, période $T = 1$ (car $\sin(2\pi(x+1)) = \sin(2\pi x + 2\pi) = \sin(2\pi x)$)
    \item $\cos(x/2)$ : Oui, période $T = 4\pi$ (car $\cos((x+4\pi)/2) = \cos(x/2 + 2\pi) = \cos(x/2)$)
    \item $\sin(2x) + \cos(3x)$ : Oui, période $T = 2\pi$ (le PPCM des périodes $\pi$ et $\frac{2\pi}{3}$)
    \item $\sin(nx)$ : Oui, période $T = \frac{2\pi}{n}$
    \item $\cos\left(\frac{3x}{2}-\frac{\pi}{4}\right)$ : Oui, période $T = \frac{4\pi}{3}$
    \item $x-\lfloor x\rfloor$ : Oui, période $T = 1$ (c'est la fonction partie fractionnaire)
\end{enumerate}
\end{mdframed}
}{}
\ifthenelse{\boolean{showSolutions}}{}
{\begin{multicols}{2}}
\begin{enumerate}
    \item $\displaystyle \cos(x)$
    
    \item $\displaystyle \sin(2\pi x)$
    \item $\displaystyle \cos(x/2)$
    \item $\displaystyle \sin(2x) + \cos(3x)$
    \item $\displaystyle \sin(nx)$, $n$ est un entier naturel non nul
    \item $\displaystyle \cos \left(\frac{3 x}{2}-\frac{\pi}{4}\right)$
    \item $\displaystyle x-\lfloor x\rfloor$
\end{enumerate}
\ifthenelse{\boolean{showSolutions}}{}{
\end{multicols}
}

Les fonctions complexes suivantes sont-elles périodiques et si oui, quelle est leur période ?

\ifthenelse{\boolean{showSolutions}}{
    \vspace{2em}
    \begin{mdframed}
    \begin{enumerate}
    \item $e^{ix}$ : Oui, période $T = 2\pi$ (car $e^{i(x+2\pi)} = e^{ix}e^{i2\pi} = e^{ix} \cdot 1 = e^{ix}$)
    \item $e^{2ix}$ : Oui, période $T = \pi$ (car $e^{2i(x+\pi)} = e^{2ix}e^{i2\pi} = e^{2ix}$)
    \item $e^{ix/2\pi}$ : Oui, période $T = 4\pi^2$ (car $e^{i(x+4\pi^2)/2\pi} = e^{ix/2\pi}e^{i2\pi} = e^{ix/2\pi}$)
    \item $e^{2i\pi x/T}$ : Oui, période $T$ (car $e^{2i\pi (x+T)/T} = e^{2i\pi x/T}e^{i2\pi} = e^{2i\pi x/T}$)
    \item $e^{inx} + e^{ipx}$ : Oui, période $T = \frac{2\pi}{\text{PGCD}(n,p)}$ (le PPCM des périodes $\frac{2\pi}{n}$ et $\frac{2\pi}{p}$)
\end{enumerate}
\end{mdframed}
}{}
\ifthenelse{\boolean{showSolutions}}{}
{\begin{multicols}{2}}
\begin{enumerate}
    \item $\displaystyle e^{ix}$
    \item $\displaystyle e^{2ix}$
    \item $\displaystyle e^{ix/2\pi}$
    \item $\displaystyle e^{2i\pi x/T}$, $T$ est un réel strictement positif
    \item $\displaystyle e^{inx} + e^{ipx}$
\end{enumerate}
\ifthenelse{\boolean{showSolutions}}{}{
\end{multicols}
}


\section*{Produit scalaire réel}

\subsection{Définition}
On appelle produit scalaire sur un espace vectoriel $E$ une application 
$$\langle \cdot, \cdot \rangle : E \times E \to \mathbb{R}$$
telle que :
\begin{multicols}{2}
\begin{itemize}
    \item[*] symétrie : $\langle u, v \rangle = \langle v, u \rangle$ 
    \item[*] linéarité à gauche : $\langle \lambda u + v, w \rangle = \lambda \langle u, w \rangle + \langle v, w \rangle$
    \item[*] positivité : $\langle u, u \rangle \geq 0$
    \item[*] définie positivité : $\langle u, u \rangle = 0 \iff u = 0$
\end{itemize}
\end{multicols}


\vspace{1em}

\subsection{Dans $\mathbb{R}^3$}

On se place dans $\mathbb{R}^3$, qu'on munit de la base 
$$e_1 = (1,2,1), \qquad e_2 = (2,1,-4), \qquad e_3 = (-3,2,-1)$$

\begin{enumerate}
\item La famille est-elle orthogonale ? 
\item Est-elle orthonormée ? Si non, définissez une base $(f_1,f_2,f_3)$ orthonormée à partir de la famille $(e_1,e_2,e_3)$. 
\end{enumerate}

Soit $u$ un vecteur de $\mathbb{R}^3$, on note $u_i$ ses coordonnées dans la base orthonormée $(f_1,f_2,f_3)$. Cela signifie que 
$$u = u_1 f_1 + u_2 f_2 + u_3 f_3$$

Déterminer les coordonnées de $u = (1,0,1)$ dans la base $(f_1,f_2,f_3)$.

\ifthenelse{\boolean{showSolutions}}{
    \vspace{2em}
    \begin{mdframed}
    \textbf{1.} Vérifions si la famille est orthogonale :
    
    $\langle e_1, e_2 \rangle = 1 \cdot 2 + 2 \cdot 1 + 1 \cdot (-4) = 2 + 2 - 4 = 0$ $\checkmark$
    
    $\langle e_1, e_3 \rangle = 1 \cdot (-3) + 2 \cdot 2 + 1 \cdot (-1) = -3 + 4 - 1 = 0$ $\checkmark$
    
    $\langle e_2, e_3 \rangle = 2 \cdot (-3) + 1 \cdot 2 + (-4) \cdot (-1) = -6 + 2 + 4 = 0$ $\checkmark$
    
    La famille est orthogonale.
    
    \textbf{2.} Vérifions si elle est orthonormée :
    
    $\|e_1\|^2 = 1^2 + 2^2 + 1^2 = 6$, donc $\|e_1\| = \sqrt{6}$
    
    $\|e_2\|^2 = 2^2 + 1^2 + (-4)^2 = 4 + 1 + 16 = 21$, donc $\|e_2\| = \sqrt{21}$
    
    $\|e_3\|^2 = (-3)^2 + 2^2 + (-1)^2 = 9 + 4 + 1 = 14$, donc $\|e_3\| = \sqrt{14}$
    
    La famille n'est pas orthonormée. Une base orthonormée est :
    $$f_1 = \frac{e_1}{\|e_1\|} = \frac{1}{\sqrt{6}}(1,2,1) = \left(\frac{1}{\sqrt{6}}, \frac{2}{\sqrt{6}}, \frac{1}{\sqrt{6}}\right)$$
    
    $$f_2 = \frac{e_2}{\|e_2\|} = \frac{1}{\sqrt{21}}(2,1,-4) = \left(\frac{2}{\sqrt{21}}, \frac{1}{\sqrt{21}}, \frac{-4}{\sqrt{21}}\right)$$
    
    $$f_3 = \frac{e_3}{\|e_3\|} = \frac{1}{\sqrt{14}}(-3,2,-1) = \left(\frac{-3}{\sqrt{14}}, \frac{2}{\sqrt{14}}, \frac{-1}{\sqrt{14}}\right)$$
    
    \textbf{3.} Coordonnées de $u = (1,0,1)$ dans la base $(f_1,f_2,f_3)$ :
    
    $u_1 = \langle u, f_1 \rangle = 1 \cdot \frac{1}{\sqrt{6}} + 0 \cdot \frac{2}{\sqrt{6}} + 1 \cdot \frac{1}{\sqrt{6}} = \frac{2}{\sqrt{6}} = \frac{\sqrt{6}}{3}$
    
    $u_2 = \langle u, f_2 \rangle = 1 \cdot \frac{2}{\sqrt{21}} + 0 \cdot \frac{1}{\sqrt{21}} + 1 \cdot \frac{-4}{\sqrt{21}} = \frac{-2}{\sqrt{21}}$
    
    $u_3 = \langle u, f_3 \rangle = 1 \cdot \frac{-3}{\sqrt{14}} + 0 \cdot \frac{2}{\sqrt{14}} + 1 \cdot \frac{-1}{\sqrt{14}} = \frac{-4}{\sqrt{14}}$
    
    Donc $u = \frac{\sqrt{6}}{3}f_1 - \frac{2}{\sqrt{21}}f_2 - \frac{4}{\sqrt{14}}f_3$
\end{mdframed}
}{}

\vspace{1em}

\subsection{Dans $\mathbb{R}[X]$}

\begin{itemize}
    \item Quelle est la dimension de $\mathbb{R}[X]$ ?
\end{itemize}
La famille $(1,X,X^2,X^3, \cdots )$ est appelée base hilbertienne de $\mathbb{R}[X]$ : tout élément de $\mathbb{R}[X]$ peut s'écrire comme une combinaison linéaire finie de vecteurs de cette famille.

On munit cet espace du produit scalaire : 
$$ \langle P, Q \rangle = \int_{0}^{1} P(x) Q(x) dx $$

\begin{itemize}
    \item Montrer que c'est bien un produit scalaire en vérifiant les propriétés ci-dessus.
    \item La famille $(1,X,X^2,X^3, \cdots )$ est-elle orthogonale ? Est-elle orthonormée ?
    \item Comment trouver $a, b, c$ tels que la famille $(1, X-a, X^2-bX-c)$ soit orthogonale ?
    \item Quelles sont les coordonnées de $P = 1+2X+3X^2$ dans la base $(1, X, X^2, \cdots)$ ?
    \item Peut-on retrouver ces coordonnées avec le produit scalaire comme dans l'exercice précédent ?
\end{itemize}
\vspace{1em}

\section*{Produit scalaire complexe}
\subsection{Définition}
On appelle produit scalaire sur un espace vectoriel $E$ une application 
$$\langle \cdot, \cdot \rangle : E \times E \to \mathbb{R}$$
telle que :
\begin{multicols}{2}
\begin{itemize}
    \item[*] symétrie conjuguée : $\langle u, v \rangle = \overline{\langle v, u \rangle}$
    \item[*] linéarité à gauche : $\langle \lambda u + v, w \rangle = \lambda \langle u, w \rangle + \langle v, w \rangle$
    \item[*] positivité : $\langle u, u \rangle \geq 0$
    \item[*] définie positivité : $\langle u, u \rangle = 0 \iff u = 0$
\end{itemize}
\end{multicols}


\subsection{Dans l'espace des fonctions complexes $2\pi$-périodiques}

On définit le produit scalaire :
$$
\langle f, g \rangle = \int_{0}^{2\pi} f(x) \overline{g(x)} dx
$$

Montrer que c'est un produit scalaire.

Montrer que la famille $(e^{inx})_{n \in \mathbb{Z}}$ est orthonormée.

Les coefficients de Fourier d'une fonction $f$ sont les coordonnées de $f$ dans la base $(e^{inx})_{n \in \mathbb{Z}}$.

Déterminer les coefficients de Fourier des fonctions suivantes :

\ifthenelse{\boolean{showSolutions}}{
    \vspace{2em}
    \begin{mdframed}
    \textbf{Démonstration que c'est un produit scalaire :}
    
    Les propriétés de symétrie conjuguée, linéarité et positivité découlent des propriétés de l'intégrale et du conjugué complexe.
    
    \textbf{Démonstration que $(e^{inx})_{n \in \mathbb{Z}}$ est orthonormée :}
    
    Pour $n = m$ : $\langle e^{inx}, e^{inx} \rangle = \int_0^{2\pi} e^{inx} \overline{e^{inx}} dx = \int_0^{2\pi} 1 dx = 2\pi$
    
    Pour $n \neq m$ : $\langle e^{inx}, e^{imx} \rangle = \int_0^{2\pi} e^{inx} \overline{e^{imx}} dx = \int_0^{2\pi} e^{i(n-m)x} dx = 0$
    
    Donc la famille $(e^{inx})_{n \in \mathbb{Z}}$ est orthogonale. Pour l'orthonormaliser, on divise par $\sqrt{2\pi}$.
    
    \textbf{Coefficients de Fourier :}
    \begin{enumerate}
    \item $\cos(x) = \frac{e^{ix} + e^{-ix}}{2}$, donc $c_1 = \frac{1}{2}$, $c_{-1} = \frac{1}{2}$, $c_n = 0$ sinon.
    
    \item $\sin(2\pi x)$ : Cette fonction n'est pas $2\pi$-périodique ! Elle est de période $1$.
    
    \item $\cos(x/2) = \frac{e^{ix/2} + e^{-ix/2}}{2}$, donc $c_{1/2} = \frac{1}{2}$, $c_{-1/2} = \frac{1}{2}$, $c_n = 0$ sinon.
    
    \item $\sin(2x) + \cos(3x) = \frac{e^{i2x} - e^{-i2x}}{2i} + \frac{e^{i3x} + e^{-i3x}}{2}$
    Donc $c_2 = \frac{1}{2i}$, $c_{-2} = -\frac{1}{2i}$, $c_3 = \frac{1}{2}$, $c_{-3} = \frac{1}{2}$, $c_n = 0$ sinon.
    
    \item $f(x) = e^{-x}$ sur $[0, 2\pi]$ :
    $$c_n = \frac{1}{2\pi}\int_0^{2\pi} e^{-x} e^{-inx} dx = \frac{1}{2\pi}\int_0^{2\pi} e^{-(1+in)x} dx$$
    $$= \frac{1}{2\pi}\left[\frac{e^{-(1+in)x}}{-(1+in)}\right]_0^{2\pi} = \frac{1}{2\pi} \cdot \frac{1-e^{-2\pi(1+in)}}{1+in} = \frac{1-e^{-2\pi}e^{-2\pi in}}{2\pi(1+in)}$$
\end{enumerate}
\end{mdframed}
}{}
\ifthenelse{\boolean{showSolutions}}{}
{\begin{multicols}{2}}
\begin{enumerate}
\item $\displaystyle \cos(x)$
\item $\displaystyle \sin(2\pi x)$
\item $\displaystyle \cos(x/2)$
\item $\displaystyle \sin(2x) + \cos(3x)$
\item $\displaystyle \exp^{-x}$ sur l'intervalle $[0, 2\pi]$
\end{enumerate}
\ifthenelse{\boolean{showSolutions}}{}{
\end{multicols}
}


% \setcounter{chapter}{3}
% \chapter{Des produits scalaires aux coefficients de Fourier}
% \section*{Les fonctions périodiques}

\subsection{Les fonctions complexes périodiques}

Les fonctions réelles suivantes sont-elles périodiques et si oui, quelle est leur période ?

\ifthenelse{\boolean{showSolutions}}{
    \vspace{2em}
    \begin{mdframed}
    \begin{enumerate}
    \item $\cos(x)$ : Oui, période $T = 2\pi$
    \item $\sin(2\pi x)$ : Oui, période $T = 1$ (car $\sin(2\pi(x+1)) = \sin(2\pi x + 2\pi) = \sin(2\pi x)$)
    \item $\cos(x/2)$ : Oui, période $T = 4\pi$ (car $\cos((x+4\pi)/2) = \cos(x/2 + 2\pi) = \cos(x/2)$)
    \item $\sin(2x) + \cos(3x)$ : Oui, période $T = 2\pi$ (le PPCM des périodes $\pi$ et $\frac{2\pi}{3}$)
    \item $\sin(nx)$ : Oui, période $T = \frac{2\pi}{n}$
    \item $\cos\left(\frac{3x}{2}-\frac{\pi}{4}\right)$ : Oui, période $T = \frac{4\pi}{3}$
    \item $x-\lfloor x\rfloor$ : Oui, période $T = 1$ (c'est la fonction partie fractionnaire)
\end{enumerate}
\end{mdframed}
}{}
\ifthenelse{\boolean{showSolutions}}{}
{\begin{multicols}{2}}
\begin{enumerate}
    \item $\displaystyle \cos(x)$
    
    \item $\displaystyle \sin(2\pi x)$
    \item $\displaystyle \cos(x/2)$
    \item $\displaystyle \sin(2x) + \cos(3x)$
    \item $\displaystyle \sin(nx)$, $n$ est un entier naturel non nul
    \item $\displaystyle \cos \left(\frac{3 x}{2}-\frac{\pi}{4}\right)$
    \item $\displaystyle x-\lfloor x\rfloor$
\end{enumerate}
\ifthenelse{\boolean{showSolutions}}{}{
\end{multicols}
}

Les fonctions complexes suivantes sont-elles périodiques et si oui, quelle est leur période ?

\ifthenelse{\boolean{showSolutions}}{
    \vspace{2em}
    \begin{mdframed}
    \begin{enumerate}
    \item $e^{ix}$ : Oui, période $T = 2\pi$ (car $e^{i(x+2\pi)} = e^{ix}e^{i2\pi} = e^{ix} \cdot 1 = e^{ix}$)
    \item $e^{2ix}$ : Oui, période $T = \pi$ (car $e^{2i(x+\pi)} = e^{2ix}e^{i2\pi} = e^{2ix}$)
    \item $e^{ix/2\pi}$ : Oui, période $T = 4\pi^2$ (car $e^{i(x+4\pi^2)/2\pi} = e^{ix/2\pi}e^{i2\pi} = e^{ix/2\pi}$)
    \item $e^{2i\pi x/T}$ : Oui, période $T$ (car $e^{2i\pi (x+T)/T} = e^{2i\pi x/T}e^{i2\pi} = e^{2i\pi x/T}$)
    \item $e^{inx} + e^{ipx}$ : Oui, période $T = \frac{2\pi}{\text{PGCD}(n,p)}$ (le PPCM des périodes $\frac{2\pi}{n}$ et $\frac{2\pi}{p}$)
\end{enumerate}
\end{mdframed}
}{}
\ifthenelse{\boolean{showSolutions}}{}
{\begin{multicols}{2}}
\begin{enumerate}
    \item $\displaystyle e^{ix}$
    \item $\displaystyle e^{2ix}$
    \item $\displaystyle e^{ix/2\pi}$
    \item $\displaystyle e^{2i\pi x/T}$, $T$ est un réel strictement positif
    \item $\displaystyle e^{inx} + e^{ipx}$
\end{enumerate}
\ifthenelse{\boolean{showSolutions}}{}{
\end{multicols}
}


\section*{Produit scalaire réel}

\subsection{Définition}
On appelle produit scalaire sur un espace vectoriel $E$ une application 
$$\langle \cdot, \cdot \rangle : E \times E \to \mathbb{R}$$
telle que :
\begin{multicols}{2}
\begin{itemize}
    \item[*] symétrie : $\langle u, v \rangle = \langle v, u \rangle$ 
    \item[*] linéarité à gauche : $\langle \lambda u + v, w \rangle = \lambda \langle u, w \rangle + \langle v, w \rangle$
    \item[*] positivité : $\langle u, u \rangle \geq 0$
    \item[*] définie positivité : $\langle u, u \rangle = 0 \iff u = 0$
\end{itemize}
\end{multicols}


\vspace{1em}

\subsection{Dans $\mathbb{R}^3$}

On se place dans $\mathbb{R}^3$, qu'on munit de la base 
$$e_1 = (1,2,1), \qquad e_2 = (2,1,-4), \qquad e_3 = (-3,2,-1)$$

\begin{enumerate}
\item La famille est-elle orthogonale ? 
\item Est-elle orthonormée ? Si non, définissez une base $(f_1,f_2,f_3)$ orthonormée à partir de la famille $(e_1,e_2,e_3)$. 
\end{enumerate}

Soit $u$ un vecteur de $\mathbb{R}^3$, on note $u_i$ ses coordonnées dans la base orthonormée $(f_1,f_2,f_3)$. Cela signifie que 
$$u = u_1 f_1 + u_2 f_2 + u_3 f_3$$

Déterminer les coordonnées de $u = (1,0,1)$ dans la base $(f_1,f_2,f_3)$.

\ifthenelse{\boolean{showSolutions}}{
    \vspace{2em}
    \begin{mdframed}
    \textbf{1.} Vérifions si la famille est orthogonale :
    
    $\langle e_1, e_2 \rangle = 1 \cdot 2 + 2 \cdot 1 + 1 \cdot (-4) = 2 + 2 - 4 = 0$ $\checkmark$
    
    $\langle e_1, e_3 \rangle = 1 \cdot (-3) + 2 \cdot 2 + 1 \cdot (-1) = -3 + 4 - 1 = 0$ $\checkmark$
    
    $\langle e_2, e_3 \rangle = 2 \cdot (-3) + 1 \cdot 2 + (-4) \cdot (-1) = -6 + 2 + 4 = 0$ $\checkmark$
    
    La famille est orthogonale.
    
    \textbf{2.} Vérifions si elle est orthonormée :
    
    $\|e_1\|^2 = 1^2 + 2^2 + 1^2 = 6$, donc $\|e_1\| = \sqrt{6}$
    
    $\|e_2\|^2 = 2^2 + 1^2 + (-4)^2 = 4 + 1 + 16 = 21$, donc $\|e_2\| = \sqrt{21}$
    
    $\|e_3\|^2 = (-3)^2 + 2^2 + (-1)^2 = 9 + 4 + 1 = 14$, donc $\|e_3\| = \sqrt{14}$
    
    La famille n'est pas orthonormée. Une base orthonormée est :
    $$f_1 = \frac{e_1}{\|e_1\|} = \frac{1}{\sqrt{6}}(1,2,1) = \left(\frac{1}{\sqrt{6}}, \frac{2}{\sqrt{6}}, \frac{1}{\sqrt{6}}\right)$$
    
    $$f_2 = \frac{e_2}{\|e_2\|} = \frac{1}{\sqrt{21}}(2,1,-4) = \left(\frac{2}{\sqrt{21}}, \frac{1}{\sqrt{21}}, \frac{-4}{\sqrt{21}}\right)$$
    
    $$f_3 = \frac{e_3}{\|e_3\|} = \frac{1}{\sqrt{14}}(-3,2,-1) = \left(\frac{-3}{\sqrt{14}}, \frac{2}{\sqrt{14}}, \frac{-1}{\sqrt{14}}\right)$$
    
    \textbf{3.} Coordonnées de $u = (1,0,1)$ dans la base $(f_1,f_2,f_3)$ :
    
    $u_1 = \langle u, f_1 \rangle = 1 \cdot \frac{1}{\sqrt{6}} + 0 \cdot \frac{2}{\sqrt{6}} + 1 \cdot \frac{1}{\sqrt{6}} = \frac{2}{\sqrt{6}} = \frac{\sqrt{6}}{3}$
    
    $u_2 = \langle u, f_2 \rangle = 1 \cdot \frac{2}{\sqrt{21}} + 0 \cdot \frac{1}{\sqrt{21}} + 1 \cdot \frac{-4}{\sqrt{21}} = \frac{-2}{\sqrt{21}}$
    
    $u_3 = \langle u, f_3 \rangle = 1 \cdot \frac{-3}{\sqrt{14}} + 0 \cdot \frac{2}{\sqrt{14}} + 1 \cdot \frac{-1}{\sqrt{14}} = \frac{-4}{\sqrt{14}}$
    
    Donc $u = \frac{\sqrt{6}}{3}f_1 - \frac{2}{\sqrt{21}}f_2 - \frac{4}{\sqrt{14}}f_3$
\end{mdframed}
}{}

\vspace{1em}

\subsection{Dans $\mathbb{R}[X]$}

\begin{itemize}
    \item Quelle est la dimension de $\mathbb{R}[X]$ ?
\end{itemize}
La famille $(1,X,X^2,X^3, \cdots )$ est appelée base hilbertienne de $\mathbb{R}[X]$ : tout élément de $\mathbb{R}[X]$ peut s'écrire comme une combinaison linéaire finie de vecteurs de cette famille.

On munit cet espace du produit scalaire : 
$$ \langle P, Q \rangle = \int_{0}^{1} P(x) Q(x) dx $$

\begin{itemize}
    \item Montrer que c'est bien un produit scalaire en vérifiant les propriétés ci-dessus.
    \item La famille $(1,X,X^2,X^3, \cdots )$ est-elle orthogonale ? Est-elle orthonormée ?
    \item Comment trouver $a, b, c$ tels que la famille $(1, X-a, X^2-bX-c)$ soit orthogonale ?
    \item Quelles sont les coordonnées de $P = 1+2X+3X^2$ dans la base $(1, X, X^2, \cdots)$ ?
    \item Peut-on retrouver ces coordonnées avec le produit scalaire comme dans l'exercice précédent ?
\end{itemize}
\vspace{1em}

\section*{Produit scalaire complexe}
\subsection{Définition}
On appelle produit scalaire sur un espace vectoriel $E$ une application 
$$\langle \cdot, \cdot \rangle : E \times E \to \mathbb{R}$$
telle que :
\begin{multicols}{2}
\begin{itemize}
    \item[*] symétrie conjuguée : $\langle u, v \rangle = \overline{\langle v, u \rangle}$
    \item[*] linéarité à gauche : $\langle \lambda u + v, w \rangle = \lambda \langle u, w \rangle + \langle v, w \rangle$
    \item[*] positivité : $\langle u, u \rangle \geq 0$
    \item[*] définie positivité : $\langle u, u \rangle = 0 \iff u = 0$
\end{itemize}
\end{multicols}


\subsection{Dans l'espace des fonctions complexes $2\pi$-périodiques}

On définit le produit scalaire :
$$
\langle f, g \rangle = \int_{0}^{2\pi} f(x) \overline{g(x)} dx
$$

Montrer que c'est un produit scalaire.

Montrer que la famille $(e^{inx})_{n \in \mathbb{Z}}$ est orthonormée.

Les coefficients de Fourier d'une fonction $f$ sont les coordonnées de $f$ dans la base $(e^{inx})_{n \in \mathbb{Z}}$.

Déterminer les coefficients de Fourier des fonctions suivantes :

\ifthenelse{\boolean{showSolutions}}{
    \vspace{2em}
    \begin{mdframed}
    \textbf{Démonstration que c'est un produit scalaire :}
    
    Les propriétés de symétrie conjuguée, linéarité et positivité découlent des propriétés de l'intégrale et du conjugué complexe.
    
    \textbf{Démonstration que $(e^{inx})_{n \in \mathbb{Z}}$ est orthonormée :}
    
    Pour $n = m$ : $\langle e^{inx}, e^{inx} \rangle = \int_0^{2\pi} e^{inx} \overline{e^{inx}} dx = \int_0^{2\pi} 1 dx = 2\pi$
    
    Pour $n \neq m$ : $\langle e^{inx}, e^{imx} \rangle = \int_0^{2\pi} e^{inx} \overline{e^{imx}} dx = \int_0^{2\pi} e^{i(n-m)x} dx = 0$
    
    Donc la famille $(e^{inx})_{n \in \mathbb{Z}}$ est orthogonale. Pour l'orthonormaliser, on divise par $\sqrt{2\pi}$.
    
    \textbf{Coefficients de Fourier :}
    \begin{enumerate}
    \item $\cos(x) = \frac{e^{ix} + e^{-ix}}{2}$, donc $c_1 = \frac{1}{2}$, $c_{-1} = \frac{1}{2}$, $c_n = 0$ sinon.
    
    \item $\sin(2\pi x)$ : Cette fonction n'est pas $2\pi$-périodique ! Elle est de période $1$.
    
    \item $\cos(x/2) = \frac{e^{ix/2} + e^{-ix/2}}{2}$, donc $c_{1/2} = \frac{1}{2}$, $c_{-1/2} = \frac{1}{2}$, $c_n = 0$ sinon.
    
    \item $\sin(2x) + \cos(3x) = \frac{e^{i2x} - e^{-i2x}}{2i} + \frac{e^{i3x} + e^{-i3x}}{2}$
    Donc $c_2 = \frac{1}{2i}$, $c_{-2} = -\frac{1}{2i}$, $c_3 = \frac{1}{2}$, $c_{-3} = \frac{1}{2}$, $c_n = 0$ sinon.
    
    \item $f(x) = e^{-x}$ sur $[0, 2\pi]$ :
    $$c_n = \frac{1}{2\pi}\int_0^{2\pi} e^{-x} e^{-inx} dx = \frac{1}{2\pi}\int_0^{2\pi} e^{-(1+in)x} dx$$
    $$= \frac{1}{2\pi}\left[\frac{e^{-(1+in)x}}{-(1+in)}\right]_0^{2\pi} = \frac{1}{2\pi} \cdot \frac{1-e^{-2\pi(1+in)}}{1+in} = \frac{1-e^{-2\pi}e^{-2\pi in}}{2\pi(1+in)}$$
\end{enumerate}
\end{mdframed}
}{}
\ifthenelse{\boolean{showSolutions}}{}
{\begin{multicols}{2}}
\begin{enumerate}
\item $\displaystyle \cos(x)$
\item $\displaystyle \sin(2\pi x)$
\item $\displaystyle \cos(x/2)$
\item $\displaystyle \sin(2x) + \cos(3x)$
\item $\displaystyle \exp^{-x}$ sur l'intervalle $[0, 2\pi]$
\end{enumerate}
\ifthenelse{\boolean{showSolutions}}{}{
\end{multicols}
}


% \setcounter{chapter}{4}
% \chapter{Produit scalaire - exercices intermédiaires}
% \section*{Les fonctions périodiques}

\subsection{Les fonctions complexes périodiques}

Les fonctions réelles suivantes sont-elles périodiques et si oui, quelle est leur période ?

\ifthenelse{\boolean{showSolutions}}{
    \vspace{2em}
    \begin{mdframed}
    \begin{enumerate}
    \item $\cos(x)$ : Oui, période $T = 2\pi$
    \item $\sin(2\pi x)$ : Oui, période $T = 1$ (car $\sin(2\pi(x+1)) = \sin(2\pi x + 2\pi) = \sin(2\pi x)$)
    \item $\cos(x/2)$ : Oui, période $T = 4\pi$ (car $\cos((x+4\pi)/2) = \cos(x/2 + 2\pi) = \cos(x/2)$)
    \item $\sin(2x) + \cos(3x)$ : Oui, période $T = 2\pi$ (le PPCM des périodes $\pi$ et $\frac{2\pi}{3}$)
    \item $\sin(nx)$ : Oui, période $T = \frac{2\pi}{n}$
    \item $\cos\left(\frac{3x}{2}-\frac{\pi}{4}\right)$ : Oui, période $T = \frac{4\pi}{3}$
    \item $x-\lfloor x\rfloor$ : Oui, période $T = 1$ (c'est la fonction partie fractionnaire)
\end{enumerate}
\end{mdframed}
}{}
\ifthenelse{\boolean{showSolutions}}{}
{\begin{multicols}{2}}
\begin{enumerate}
    \item $\displaystyle \cos(x)$
    
    \item $\displaystyle \sin(2\pi x)$
    \item $\displaystyle \cos(x/2)$
    \item $\displaystyle \sin(2x) + \cos(3x)$
    \item $\displaystyle \sin(nx)$, $n$ est un entier naturel non nul
    \item $\displaystyle \cos \left(\frac{3 x}{2}-\frac{\pi}{4}\right)$
    \item $\displaystyle x-\lfloor x\rfloor$
\end{enumerate}
\ifthenelse{\boolean{showSolutions}}{}{
\end{multicols}
}

Les fonctions complexes suivantes sont-elles périodiques et si oui, quelle est leur période ?

\ifthenelse{\boolean{showSolutions}}{
    \vspace{2em}
    \begin{mdframed}
    \begin{enumerate}
    \item $e^{ix}$ : Oui, période $T = 2\pi$ (car $e^{i(x+2\pi)} = e^{ix}e^{i2\pi} = e^{ix} \cdot 1 = e^{ix}$)
    \item $e^{2ix}$ : Oui, période $T = \pi$ (car $e^{2i(x+\pi)} = e^{2ix}e^{i2\pi} = e^{2ix}$)
    \item $e^{ix/2\pi}$ : Oui, période $T = 4\pi^2$ (car $e^{i(x+4\pi^2)/2\pi} = e^{ix/2\pi}e^{i2\pi} = e^{ix/2\pi}$)
    \item $e^{2i\pi x/T}$ : Oui, période $T$ (car $e^{2i\pi (x+T)/T} = e^{2i\pi x/T}e^{i2\pi} = e^{2i\pi x/T}$)
    \item $e^{inx} + e^{ipx}$ : Oui, période $T = \frac{2\pi}{\text{PGCD}(n,p)}$ (le PPCM des périodes $\frac{2\pi}{n}$ et $\frac{2\pi}{p}$)
\end{enumerate}
\end{mdframed}
}{}
\ifthenelse{\boolean{showSolutions}}{}
{\begin{multicols}{2}}
\begin{enumerate}
    \item $\displaystyle e^{ix}$
    \item $\displaystyle e^{2ix}$
    \item $\displaystyle e^{ix/2\pi}$
    \item $\displaystyle e^{2i\pi x/T}$, $T$ est un réel strictement positif
    \item $\displaystyle e^{inx} + e^{ipx}$
\end{enumerate}
\ifthenelse{\boolean{showSolutions}}{}{
\end{multicols}
}


\section*{Produit scalaire réel}

\subsection{Définition}
On appelle produit scalaire sur un espace vectoriel $E$ une application 
$$\langle \cdot, \cdot \rangle : E \times E \to \mathbb{R}$$
telle que :
\begin{multicols}{2}
\begin{itemize}
    \item[*] symétrie : $\langle u, v \rangle = \langle v, u \rangle$ 
    \item[*] linéarité à gauche : $\langle \lambda u + v, w \rangle = \lambda \langle u, w \rangle + \langle v, w \rangle$
    \item[*] positivité : $\langle u, u \rangle \geq 0$
    \item[*] définie positivité : $\langle u, u \rangle = 0 \iff u = 0$
\end{itemize}
\end{multicols}


\vspace{1em}

\subsection{Dans $\mathbb{R}^3$}

On se place dans $\mathbb{R}^3$, qu'on munit de la base 
$$e_1 = (1,2,1), \qquad e_2 = (2,1,-4), \qquad e_3 = (-3,2,-1)$$

\begin{enumerate}
\item La famille est-elle orthogonale ? 
\item Est-elle orthonormée ? Si non, définissez une base $(f_1,f_2,f_3)$ orthonormée à partir de la famille $(e_1,e_2,e_3)$. 
\end{enumerate}

Soit $u$ un vecteur de $\mathbb{R}^3$, on note $u_i$ ses coordonnées dans la base orthonormée $(f_1,f_2,f_3)$. Cela signifie que 
$$u = u_1 f_1 + u_2 f_2 + u_3 f_3$$

Déterminer les coordonnées de $u = (1,0,1)$ dans la base $(f_1,f_2,f_3)$.

\ifthenelse{\boolean{showSolutions}}{
    \vspace{2em}
    \begin{mdframed}
    \textbf{1.} Vérifions si la famille est orthogonale :
    
    $\langle e_1, e_2 \rangle = 1 \cdot 2 + 2 \cdot 1 + 1 \cdot (-4) = 2 + 2 - 4 = 0$ $\checkmark$
    
    $\langle e_1, e_3 \rangle = 1 \cdot (-3) + 2 \cdot 2 + 1 \cdot (-1) = -3 + 4 - 1 = 0$ $\checkmark$
    
    $\langle e_2, e_3 \rangle = 2 \cdot (-3) + 1 \cdot 2 + (-4) \cdot (-1) = -6 + 2 + 4 = 0$ $\checkmark$
    
    La famille est orthogonale.
    
    \textbf{2.} Vérifions si elle est orthonormée :
    
    $\|e_1\|^2 = 1^2 + 2^2 + 1^2 = 6$, donc $\|e_1\| = \sqrt{6}$
    
    $\|e_2\|^2 = 2^2 + 1^2 + (-4)^2 = 4 + 1 + 16 = 21$, donc $\|e_2\| = \sqrt{21}$
    
    $\|e_3\|^2 = (-3)^2 + 2^2 + (-1)^2 = 9 + 4 + 1 = 14$, donc $\|e_3\| = \sqrt{14}$
    
    La famille n'est pas orthonormée. Une base orthonormée est :
    $$f_1 = \frac{e_1}{\|e_1\|} = \frac{1}{\sqrt{6}}(1,2,1) = \left(\frac{1}{\sqrt{6}}, \frac{2}{\sqrt{6}}, \frac{1}{\sqrt{6}}\right)$$
    
    $$f_2 = \frac{e_2}{\|e_2\|} = \frac{1}{\sqrt{21}}(2,1,-4) = \left(\frac{2}{\sqrt{21}}, \frac{1}{\sqrt{21}}, \frac{-4}{\sqrt{21}}\right)$$
    
    $$f_3 = \frac{e_3}{\|e_3\|} = \frac{1}{\sqrt{14}}(-3,2,-1) = \left(\frac{-3}{\sqrt{14}}, \frac{2}{\sqrt{14}}, \frac{-1}{\sqrt{14}}\right)$$
    
    \textbf{3.} Coordonnées de $u = (1,0,1)$ dans la base $(f_1,f_2,f_3)$ :
    
    $u_1 = \langle u, f_1 \rangle = 1 \cdot \frac{1}{\sqrt{6}} + 0 \cdot \frac{2}{\sqrt{6}} + 1 \cdot \frac{1}{\sqrt{6}} = \frac{2}{\sqrt{6}} = \frac{\sqrt{6}}{3}$
    
    $u_2 = \langle u, f_2 \rangle = 1 \cdot \frac{2}{\sqrt{21}} + 0 \cdot \frac{1}{\sqrt{21}} + 1 \cdot \frac{-4}{\sqrt{21}} = \frac{-2}{\sqrt{21}}$
    
    $u_3 = \langle u, f_3 \rangle = 1 \cdot \frac{-3}{\sqrt{14}} + 0 \cdot \frac{2}{\sqrt{14}} + 1 \cdot \frac{-1}{\sqrt{14}} = \frac{-4}{\sqrt{14}}$
    
    Donc $u = \frac{\sqrt{6}}{3}f_1 - \frac{2}{\sqrt{21}}f_2 - \frac{4}{\sqrt{14}}f_3$
\end{mdframed}
}{}

\vspace{1em}

\subsection{Dans $\mathbb{R}[X]$}

\begin{itemize}
    \item Quelle est la dimension de $\mathbb{R}[X]$ ?
\end{itemize}
La famille $(1,X,X^2,X^3, \cdots )$ est appelée base hilbertienne de $\mathbb{R}[X]$ : tout élément de $\mathbb{R}[X]$ peut s'écrire comme une combinaison linéaire finie de vecteurs de cette famille.

On munit cet espace du produit scalaire : 
$$ \langle P, Q \rangle = \int_{0}^{1} P(x) Q(x) dx $$

\begin{itemize}
    \item Montrer que c'est bien un produit scalaire en vérifiant les propriétés ci-dessus.
    \item La famille $(1,X,X^2,X^3, \cdots )$ est-elle orthogonale ? Est-elle orthonormée ?
    \item Comment trouver $a, b, c$ tels que la famille $(1, X-a, X^2-bX-c)$ soit orthogonale ?
    \item Quelles sont les coordonnées de $P = 1+2X+3X^2$ dans la base $(1, X, X^2, \cdots)$ ?
    \item Peut-on retrouver ces coordonnées avec le produit scalaire comme dans l'exercice précédent ?
\end{itemize}
\vspace{1em}

\section*{Produit scalaire complexe}
\subsection{Définition}
On appelle produit scalaire sur un espace vectoriel $E$ une application 
$$\langle \cdot, \cdot \rangle : E \times E \to \mathbb{R}$$
telle que :
\begin{multicols}{2}
\begin{itemize}
    \item[*] symétrie conjuguée : $\langle u, v \rangle = \overline{\langle v, u \rangle}$
    \item[*] linéarité à gauche : $\langle \lambda u + v, w \rangle = \lambda \langle u, w \rangle + \langle v, w \rangle$
    \item[*] positivité : $\langle u, u \rangle \geq 0$
    \item[*] définie positivité : $\langle u, u \rangle = 0 \iff u = 0$
\end{itemize}
\end{multicols}


\subsection{Dans l'espace des fonctions complexes $2\pi$-périodiques}

On définit le produit scalaire :
$$
\langle f, g \rangle = \int_{0}^{2\pi} f(x) \overline{g(x)} dx
$$

Montrer que c'est un produit scalaire.

Montrer que la famille $(e^{inx})_{n \in \mathbb{Z}}$ est orthonormée.

Les coefficients de Fourier d'une fonction $f$ sont les coordonnées de $f$ dans la base $(e^{inx})_{n \in \mathbb{Z}}$.

Déterminer les coefficients de Fourier des fonctions suivantes :

\ifthenelse{\boolean{showSolutions}}{
    \vspace{2em}
    \begin{mdframed}
    \textbf{Démonstration que c'est un produit scalaire :}
    
    Les propriétés de symétrie conjuguée, linéarité et positivité découlent des propriétés de l'intégrale et du conjugué complexe.
    
    \textbf{Démonstration que $(e^{inx})_{n \in \mathbb{Z}}$ est orthonormée :}
    
    Pour $n = m$ : $\langle e^{inx}, e^{inx} \rangle = \int_0^{2\pi} e^{inx} \overline{e^{inx}} dx = \int_0^{2\pi} 1 dx = 2\pi$
    
    Pour $n \neq m$ : $\langle e^{inx}, e^{imx} \rangle = \int_0^{2\pi} e^{inx} \overline{e^{imx}} dx = \int_0^{2\pi} e^{i(n-m)x} dx = 0$
    
    Donc la famille $(e^{inx})_{n \in \mathbb{Z}}$ est orthogonale. Pour l'orthonormaliser, on divise par $\sqrt{2\pi}$.
    
    \textbf{Coefficients de Fourier :}
    \begin{enumerate}
    \item $\cos(x) = \frac{e^{ix} + e^{-ix}}{2}$, donc $c_1 = \frac{1}{2}$, $c_{-1} = \frac{1}{2}$, $c_n = 0$ sinon.
    
    \item $\sin(2\pi x)$ : Cette fonction n'est pas $2\pi$-périodique ! Elle est de période $1$.
    
    \item $\cos(x/2) = \frac{e^{ix/2} + e^{-ix/2}}{2}$, donc $c_{1/2} = \frac{1}{2}$, $c_{-1/2} = \frac{1}{2}$, $c_n = 0$ sinon.
    
    \item $\sin(2x) + \cos(3x) = \frac{e^{i2x} - e^{-i2x}}{2i} + \frac{e^{i3x} + e^{-i3x}}{2}$
    Donc $c_2 = \frac{1}{2i}$, $c_{-2} = -\frac{1}{2i}$, $c_3 = \frac{1}{2}$, $c_{-3} = \frac{1}{2}$, $c_n = 0$ sinon.
    
    \item $f(x) = e^{-x}$ sur $[0, 2\pi]$ :
    $$c_n = \frac{1}{2\pi}\int_0^{2\pi} e^{-x} e^{-inx} dx = \frac{1}{2\pi}\int_0^{2\pi} e^{-(1+in)x} dx$$
    $$= \frac{1}{2\pi}\left[\frac{e^{-(1+in)x}}{-(1+in)}\right]_0^{2\pi} = \frac{1}{2\pi} \cdot \frac{1-e^{-2\pi(1+in)}}{1+in} = \frac{1-e^{-2\pi}e^{-2\pi in}}{2\pi(1+in)}$$
\end{enumerate}
\end{mdframed}
}{}
\ifthenelse{\boolean{showSolutions}}{}
{\begin{multicols}{2}}
\begin{enumerate}
\item $\displaystyle \cos(x)$
\item $\displaystyle \sin(2\pi x)$
\item $\displaystyle \cos(x/2)$
\item $\displaystyle \sin(2x) + \cos(3x)$
\item $\displaystyle \exp^{-x}$ sur l'intervalle $[0, 2\pi]$
\end{enumerate}
\ifthenelse{\boolean{showSolutions}}{}{
\end{multicols}
}


% \setcounter{chapter}{5}
% \chapter{Contrôle continu 2}
% \documentclass[12pt]{article}

% Packages pour les marges
\usepackage[
    top=1.5cm,
    bottom=1.5cm,
    left=1.5cm,
    right=1.5cm
]{geometry}

% Police sans serif
% \usepackage{helvet}
% \renewcommand{\familydefault}{\sfdefault}

% Packages existants
\usepackage[french]{babel}
\usepackage[utf8]{inputenc}
\usepackage[T1]{fontenc}
\usepackage{amsmath}
\usepackage{amsfonts}
\usepackage{amssymb}
\usepackage{mathtools}
\usepackage{array}
\usepackage[version=4]{mhchem}
\usepackage{stmaryrd}
\usepackage{enumitem}
\usepackage{ifthen}
\usepackage{eurosym}
\usepackage{textcomp}
\usepackage{graphicx}
\usepackage{xcolor}
\usepackage{multicol}
\definecolor{Theme}{HTML}{0E7490} % teal-700
\definecolor{ThemeLight}{HTML}{E0F2F1}
\definecolor{Accent}{HTML}{F59E0B} % amber-500
\definecolor{Gray}{HTML}{374151}
\usepackage[colorlinks=true,linkcolor=Theme,urlcolor=Theme,citecolor=Theme]{hyperref}

\usepackage{mdframed}
\usepackage[sf]{titlesec}
\usepackage{environ}

% Définition de la variable pour afficher les corrections
\newboolean{showSolutions}
% Décommentez la ligne suivante pour afficher les solutions
\input \jobname.adr

\title{TD de Préparation à l'Examen S5}
\author{}
\date{}

\newenvironment{solution}
    {\par\vspace{0.5em}\begin{mdframed}[backgroundcolor=ThemeLight,linewidth=0.5pt]\noindent\textbf{Solution :}\par}
    {\end{mdframed}\par\vspace{0.5em}}

\begin{document}
\sffamily

\begin{center}
    {\Large\textbf{Révisions -- Mathématiques S5}}
    
    \vspace{0.5em}
    {\textit{BMC3 -- Semestre 5}}
\end{center}

\vspace{1em}

%==============================================================================
\section*{Développements limités}
%==============================================================================

%------------------------------------------------------------------------------
\subsection*{Exercice 1 -- Produit et quotient de DL}
%------------------------------------------------------------------------------

\begin{enumerate}
    \item Calculer le DL de $\dfrac{\sin x}{1+x}$ à l'ordre 3 au voisinage de 0.
    
    \ifthenelse{\boolean{showSolutions}}{
    \begin{solution}
    On multiplie les DL en ne gardant que les termes d'ordre $\leq 3$ :
    \begin{align*}
    \dfrac{\sin x}{1+x} &= \left(x - \dfrac{x^3}{6}\right)\left(1 - x + x^2 - x^3\right) + o(x^3) \\
    &= x - x^2 + x^3 - \dfrac{x^3}{6} + o(x^3) \\
    &= x - x^2 + \dfrac{5x^3}{6} + o(x^3)
    \end{align*}
    \end{solution}
    }{}
    
    \item Calculer le DL de $\dfrac{e^x}{1+x}$ à l'ordre 3.
    
    \ifthenelse{\boolean{showSolutions}}{
    \begin{solution}
    On a $e^x = 1 + x + \dfrac{x^2}{2} + \dfrac{x^3}{6} + o(x^3)$ et $\dfrac{1}{1+x} = 1 - x + x^2 - x^3 + o(x^3)$.
    
    \begin{align*}
    \dfrac{e^x}{1+x} &= \left(1 + x + \dfrac{x^2}{2} + \dfrac{x^3}{6}\right)(1 - x + x^2 - x^3) + o(x^3) \\
    &= 1 - x + x^2 + x - x^2 + x^3 + \dfrac{x^2}{2} - \dfrac{x^3}{2} + \dfrac{x^3}{6} + o(x^3) \\
    &= 1 + \dfrac{x^2}{2} + \dfrac{2x^3}{3} + o(x^3)
    \end{align*}
    \end{solution}
    }{}
\end{enumerate}

\vspace{1em}
%------------------------------------------------------------------------------
\subsection*{Exercice 2 -- DL par composition}
%------------------------------------------------------------------------------

On cherche le DL de $\ln(\cos x)$ à l'ordre 4 au voisinage de 0.

\begin{enumerate}
    \item En posant $u = \cos x - 1$, montrer que $u = -\dfrac{x^2}{2} + \dfrac{x^4}{24} + o(x^4)$.
    
    \ifthenelse{\boolean{showSolutions}}{
    \begin{solution}
    On a $\cos x = 1 - \dfrac{x^2}{2} + \dfrac{x^4}{24} + o(x^4)$, donc :
    
    $u = \cos x - 1 = -\dfrac{x^2}{2} + \dfrac{x^4}{24} + o(x^4)$
    \end{solution}
    }{}
    
    \item En utilisant $\ln(1+u) = u - \dfrac{u^2}{2} + o(u^2)$, en déduire le DL de $\ln(\cos x)$ à l'ordre 4.
    
    \ifthenelse{\boolean{showSolutions}}{
    \begin{solution}
    On a $u^2 = \dfrac{x^4}{4} + o(x^4)$.
    \begin{align*}
    \ln(\cos x) &= \ln(1+u) = u - \dfrac{u^2}{2} + o(u^2) \\
    &= -\dfrac{x^2}{2} + \dfrac{x^4}{24} - \dfrac{1}{2} \cdot \dfrac{x^4}{4} + o(x^4) \\
    &= -\dfrac{x^2}{2} + \dfrac{x^4}{24} - \dfrac{x^4}{8} + o(x^4) \\
    &= -\dfrac{x^2}{2} - \dfrac{x^4}{12} + o(x^4)
    \end{align*}
    \end{solution}
    }{}
\end{enumerate}

\vspace{1em}
%------------------------------------------------------------------------------
\subsection*{Exercice 3 -- Calcul de limites par DL}
%------------------------------------------------------------------------------

Calculer les limites suivantes à l'aide de développements limités :

\begin{enumerate}
    \item $\displaystyle\lim_{x \to 0} \dfrac{\ln(1+x) - x + \frac{x^2}{2}}{x^3}$
    
    \ifthenelse{\boolean{showSolutions}}{
    \begin{solution}
    $\ln(1+x) = x - \dfrac{x^2}{2} + \dfrac{x^3}{3} + o(x^3)$, donc :
    
    $\ln(1+x) - x + \dfrac{x^2}{2} = \dfrac{x^3}{3} + o(x^3)$
    
    $\dfrac{\ln(1+x) - x + \frac{x^2}{2}}{x^3} = \dfrac{1}{3} + o(1) \xrightarrow[x \to 0]{} \boxed{\dfrac{1}{3}}$
    \end{solution}
    }{}
    
    \item $\displaystyle\lim_{x \to 0} \dfrac{\sin x - x\cos x}{x^3}$
    
    \ifthenelse{\boolean{showSolutions}}{
    \begin{solution}
    $\sin x = x - \dfrac{x^3}{6} + o(x^3)$ et $\cos x = 1 - \dfrac{x^2}{2} + o(x^2)$
    
    $x\cos x = x - \dfrac{x^3}{2} + o(x^3)$
    
    $\sin x - x\cos x = x - \dfrac{x^3}{6} - x + \dfrac{x^3}{2} + o(x^3) = \dfrac{x^3}{3} + o(x^3)$
    
    $\dfrac{\sin x - x\cos x}{x^3} = \dfrac{1}{3} + o(1) \xrightarrow[x \to 0]{} \boxed{\dfrac{1}{3}}$
    \end{solution}
    }{}
\end{enumerate}

\vspace{1em}
%------------------------------------------------------------------------------
\subsection*{Exercice 4 -- Prolongement et position par rapport à la tangente}
%------------------------------------------------------------------------------

Soit $f(x) = \dfrac{e^x - 1 - x}{x^2}$ pour $x \neq 0$.

\begin{enumerate}
    \item Montrer que $f$ admet un prolongement par continuité en 0 et déterminer $f(0)$.
    
    \ifthenelse{\boolean{showSolutions}}{
    \begin{solution}
    $e^x = 1 + x + \dfrac{x^2}{2} + \dfrac{x^3}{6} + o(x^3)$, donc $e^x - 1 - x = \dfrac{x^2}{2} + \dfrac{x^3}{6} + o(x^3)$.
    
    $f(x) = \dfrac{e^x - 1 - x}{x^2} = \dfrac{1}{2} + \dfrac{x}{6} + o(x) \xrightarrow[x \to 0]{} \dfrac{1}{2}$
    
    On peut prolonger $f$ par continuité en posant $f(0) = \dfrac{1}{2}$.
    \end{solution}
    }{}
    
    \item Donner le DL de $f$ à l'ordre 2 en 0, puis l'équation de la tangente à la courbe en 0.
    
    \ifthenelse{\boolean{showSolutions}}{
    \begin{solution}
    $e^x - 1 - x = \dfrac{x^2}{2} + \dfrac{x^3}{6} + \dfrac{x^4}{24} + o(x^4)$
    
    $f(x) = \dfrac{1}{2} + \dfrac{x}{6} + \dfrac{x^2}{24} + o(x^2)$
    
    La tangente en 0 a pour équation : $y = \dfrac{1}{2} + \dfrac{x}{6}$
    \end{solution}
    }{}
    
    \item En déduire la position de la courbe par rapport à sa tangente au voisinage de 0.
    
    \ifthenelse{\boolean{showSolutions}}{
    \begin{solution}
    $f(x) - \left(\dfrac{1}{2} + \dfrac{x}{6}\right) = \dfrac{x^2}{24} + o(x^2) \sim \dfrac{x^2}{24} > 0$ pour $x \neq 0$ petit.
    
    Donc la courbe est \textbf{au-dessus} de sa tangente au voisinage de 0.
    \end{solution}
    }{}
\end{enumerate}

\newpage
%==============================================================================
\section*{Équations différentielles exactes}
%==============================================================================

%------------------------------------------------------------------------------
\subsection*{Exercice 5 -- Équation exacte}
%------------------------------------------------------------------------------

Résoudre l'équation différentielle $(2xy + 1)\,dx + (x^2 + 2y)\,dy = 0$.

\begin{enumerate}
    \item Vérifier que l'équation est exacte.
    
    \ifthenelse{\boolean{showSolutions}}{
    \begin{solution}
    On pose $f(x,y) = 2xy + 1$ et $g(x,y) = x^2 + 2y$.
    
    $\dfrac{\partial f}{\partial y} = 2x$ et $\dfrac{\partial g}{\partial x} = 2x$
    
    Comme $\dfrac{\partial f}{\partial y} = \dfrac{\partial g}{\partial x}$, l'équation est exacte.
    \end{solution}
    }{}
    
    \item Trouver $F(x,y)$ telle que $dF = f\,dx + g\,dy$.
    
    \ifthenelse{\boolean{showSolutions}}{
    \begin{solution}
    On cherche $F$ telle que $\dfrac{\partial F}{\partial x} = 2xy + 1$.
    
    En intégrant par rapport à $x$ : $F(x,y) = x^2y + x + H(y)$
    
    On vérifie avec $\dfrac{\partial F}{\partial y} = x^2 + H'(y) = x^2 + 2y$.
    
    Donc $H'(y) = 2y$, soit $H(y) = y^2$.
    
    \textbf{Conclusion :} $F(x,y) = x^2y + x + y^2$
    \end{solution}
    }{}
    
    \item En déduire la solution générale.
    
    \ifthenelse{\boolean{showSolutions}}{
    \begin{solution}
    Les solutions sont données par $F(x,y) = K$ :
    \[\boxed{x^2y + x + y^2 = K}\]
    où $K$ est une constante.
    \end{solution}
    }{}
\end{enumerate}

\vspace{1em}

%==============================================================================
\section*{Équations aux dérivées partielles}
%==============================================================================

%------------------------------------------------------------------------------
\subsection*{Exercice 6 -- EDP}
%------------------------------------------------------------------------------

Résoudre l'équation $2\dfrac{\partial f}{\partial x} + \dfrac{\partial f}{\partial y} = 4x$ par changement de variables.

\ifthenelse{\boolean{showSolutions}}{
\begin{solution}
Avec $X = x - 2y$ et $Y = x$, on a $x = Y$ et $y = \frac{Y-X}{2}$.

On pose $F(X,Y) = f(x,y)$. Par la règle de la chaîne :

$\dfrac{\partial f}{\partial x} = \dfrac{\partial F}{\partial X} \cdot 1 + \dfrac{\partial F}{\partial Y} \cdot 1 = \dfrac{\partial F}{\partial X} + \dfrac{\partial F}{\partial Y}$

$\dfrac{\partial f}{\partial y} = \dfrac{\partial F}{\partial X} \cdot (-2) + \dfrac{\partial F}{\partial Y} \cdot 0 = -2\dfrac{\partial F}{\partial X}$

L'équation devient : $2\left(\dfrac{\partial F}{\partial X} + \dfrac{\partial F}{\partial Y}\right) - 2\dfrac{\partial F}{\partial X} = 4Y$

$2\dfrac{\partial F}{\partial Y} = 4Y$, soit $\dfrac{\partial F}{\partial Y} = 2Y$

En intégrant par rapport à $Y$ : $F(X,Y) = Y^2 + K(X)$

En revenant aux variables initiales : $\boxed{f(x,y) = x^2 + K(x-2y)}$

où $K$ est une fonction $\mathcal{C}^1$ quelconque.
\end{solution}
}{}
\vspace{1em}

Résoudre par séparation de variables : $\dfrac{\partial f}{\partial x} - 3\dfrac{\partial f}{\partial y} = 2f$.

\ifthenelse{\boolean{showSolutions}}{
\begin{solution}
On cherche $f(x,y) = X(x)Y(y)$.

$X'Y - 3XY' = 2XY$

En divisant par $XY$ : $\dfrac{X'}{X} - 3\dfrac{Y'}{Y} = 2$

Donc $\dfrac{X'}{X} = 2 + 3\dfrac{Y'}{Y}$. Le membre de gauche ne dépend que de $x$, le membre de droite que de $y$. Ils sont égaux à une constante $k$.

\textbf{Pour $X$ :} $\dfrac{X'}{X} = k \Rightarrow X(x) = C_1 e^{kx}$

\textbf{Pour $Y$ :} $3\dfrac{Y'}{Y} = k - 2 \Rightarrow Y' = \dfrac{k-2}{3}Y \Rightarrow Y(y) = C_2 e^{(k-2)y/3}$

\textbf{Solutions :} $\boxed{f(x,y) = C \, e^{kx} e^{(k-2)y/3}}$ pour $k \in \mathbb{R}$, $C \in \mathbb{R}$.
\end{solution}
}{}

\vspace{1em}

%==============================================================================
\section*{Modélisation}
%==============================================================================

%------------------------------------------------------------------------------
\subsection*{Exercice 7 -- Refroidissement de Newton}
%------------------------------------------------------------------------------

Un objet de température initiale $T_0 = 80$ °C est placé dans une pièce à température ambiante $T_a = 20$ °C. La loi de Newton stipule que :
\[
\frac{dT}{dt} = -k(T - T_a)
\]
avec $k = 0{,}1$ min$^{-1}$.

\begin{enumerate}
    \item En posant $\theta(t) = T(t) - T_a$, montrer que $\theta' = -k\theta$ et résoudre.
    
    \ifthenelse{\boolean{showSolutions}}{
    \begin{solution}
    $\theta' = T' = -k(T - T_a) = -k\theta$
    
    Donc $\theta(t) = \theta_0 e^{-kt}$ avec $\theta_0 = T_0 - T_a = 80 - 20 = 60$ °C.
    
    $\theta(t) = 60 e^{-0{,}1t}$, donc $T(t) = T_a + \theta(t) = 20 + 60e^{-0{,}1t}$ °C
    \end{solution}
    }{}
    
    \item Calculer la température après 10 minutes. \textit{(On donne $e^{-1} \approx 0{,}37$.)}
    
    \ifthenelse{\boolean{showSolutions}}{
    \begin{solution}
    $T(10) = 20 + 60e^{-1} \approx 20 + 60 \times 0{,}37 = 20 + 22{,}2 = \boxed{42{,}2 \text{ °C}}$
    \end{solution}
    }{}
    
    \item Au bout de combien de temps la température sera-t-elle de 30 °C ? \textit{(On donne $\ln 6 \approx 1{,}8$.)}
    
    \ifthenelse{\boolean{showSolutions}}{
    \begin{solution}
    On résout $T(t) = 30$ :
    
    $20 + 60e^{-0{,}1t} = 30 \Rightarrow 60e^{-0{,}1t} = 10 \Rightarrow e^{-0{,}1t} = \dfrac{1}{6}$
    
    $-0{,}1t = -\ln 6 \Rightarrow t = \dfrac{\ln 6}{0{,}1} = \dfrac{1{,}8}{0{,}1} = \boxed{18 \text{ min}}$
    \end{solution}
    }{}
\end{enumerate}

%------------------------------------------------------------------------------
\subsection*{Exercice 8 -- Croissance bactérienne avec limitation}
%------------------------------------------------------------------------------

Une population de bactéries $N(t)$ croît dans un milieu où les nutriments sont renouvelés à débit constant $D$, mais sont consommés proportionnellement à la population :
\[
\frac{dN}{dt} = D - kN
\]
avec $k = 0{,}5$ h$^{-1}$, $D = 1000$ bactéries/h, et $N(0) = 100$ bactéries.

\begin{enumerate}
    \item Déterminer la population d'équilibre $N_{eq}$.
    
    \ifthenelse{\boolean{showSolutions}}{
    \begin{solution}
    À l'équilibre, $\dfrac{dN}{dt} = 0$, donc $D - kN_{eq} = 0$.
    
    $N_{eq} = \dfrac{D}{k} = \dfrac{1000}{0{,}5} = 2000$ bactéries
    \end{solution}
    }{}
    
    \item Résoudre l'équation et donner $N(t)$.
    
    \ifthenelse{\boolean{showSolutions}}{
    \begin{solution}
    On pose $\theta = N - N_{eq}$. Alors $\theta' = N' = D - kN = D - k(N_{eq} + \theta) = -k\theta$.
    
    Donc $\theta(t) = \theta_0 e^{-kt}$ avec $\theta_0 = N(0) - N_{eq} = 100 - 2000 = -1900$.
    
    $N(t) = N_{eq} + \theta(t) = 2000 - 1900e^{-0{,}5t}$
    
    $\boxed{N(t) = 2000 - 1900e^{-0{,}5t}}$
    \end{solution}
    }{}
    
    \item Au bout de combien de temps la population atteint-elle 95\% de sa valeur d'équilibre ?
    
    \textit{(On donne $\ln 20 \approx 3$.)}
    
    \ifthenelse{\boolean{showSolutions}}{
    \begin{solution}
    On cherche $t$ tel que $N(t) = 0{,}95 \times 2000 = 1900$.
    
    $2000 - 1900e^{-0{,}5t} = 1900 \Rightarrow 1900e^{-0{,}5t} = 100 \Rightarrow e^{-0{,}5t} = \dfrac{1}{19} \approx \dfrac{1}{20}$
    
    $-0{,}5t = -\ln 20 \Rightarrow t = \dfrac{\ln 20}{0{,}5} = \dfrac{3}{0{,}5} = \boxed{6 \text{ h}}$
    \end{solution}
    }{}
\end{enumerate}

\end{document}


% \chapter*{Contrôle sur les séries de Fourier}
% \documentclass[12pt]{article}

% Packages pour les marges
\usepackage[
    top=1.5cm,
    bottom=1.5cm,
    left=1.5cm,
    right=1.5cm
]{geometry}

% Police sans serif
% \usepackage{helvet}
% \renewcommand{\familydefault}{\sfdefault}

% Packages existants
\usepackage[french]{babel}
\usepackage[utf8]{inputenc}
\usepackage[T1]{fontenc}
\usepackage{amsmath}
\usepackage{amsfonts}
\usepackage{amssymb}
\usepackage{mathtools}
\usepackage{array}
\usepackage[version=4]{mhchem}
\usepackage{stmaryrd}
\usepackage{enumitem}
\usepackage{ifthen}
\usepackage{eurosym}
\usepackage{textcomp}
\usepackage{graphicx}
\usepackage{xcolor}
\usepackage{multicol}
\definecolor{Theme}{HTML}{0E7490} % teal-700
\definecolor{ThemeLight}{HTML}{E0F2F1}
\definecolor{Accent}{HTML}{F59E0B} % amber-500
\definecolor{Gray}{HTML}{374151}
\usepackage[colorlinks=true,linkcolor=Theme,urlcolor=Theme,citecolor=Theme]{hyperref}

\usepackage{mdframed}
\usepackage[sf]{titlesec}
\usepackage{environ}

% Définition de la variable pour afficher les corrections
\newboolean{showSolutions}
% Décommentez la ligne suivante pour afficher les solutions
\input \jobname.adr

\title{TD de Préparation à l'Examen S5}
\author{}
\date{}

\newenvironment{solution}
    {\par\vspace{0.5em}\begin{mdframed}[backgroundcolor=ThemeLight,linewidth=0.5pt]\noindent\textbf{Solution :}\par}
    {\end{mdframed}\par\vspace{0.5em}}

\begin{document}
\sffamily

\begin{center}
    {\Large\textbf{Révisions -- Mathématiques S5}}
    
    \vspace{0.5em}
    {\textit{BMC3 -- Semestre 5}}
\end{center}

\vspace{1em}

%==============================================================================
\section*{Développements limités}
%==============================================================================

%------------------------------------------------------------------------------
\subsection*{Exercice 1 -- Produit et quotient de DL}
%------------------------------------------------------------------------------

\begin{enumerate}
    \item Calculer le DL de $\dfrac{\sin x}{1+x}$ à l'ordre 3 au voisinage de 0.
    
    \ifthenelse{\boolean{showSolutions}}{
    \begin{solution}
    On multiplie les DL en ne gardant que les termes d'ordre $\leq 3$ :
    \begin{align*}
    \dfrac{\sin x}{1+x} &= \left(x - \dfrac{x^3}{6}\right)\left(1 - x + x^2 - x^3\right) + o(x^3) \\
    &= x - x^2 + x^3 - \dfrac{x^3}{6} + o(x^3) \\
    &= x - x^2 + \dfrac{5x^3}{6} + o(x^3)
    \end{align*}
    \end{solution}
    }{}
    
    \item Calculer le DL de $\dfrac{e^x}{1+x}$ à l'ordre 3.
    
    \ifthenelse{\boolean{showSolutions}}{
    \begin{solution}
    On a $e^x = 1 + x + \dfrac{x^2}{2} + \dfrac{x^3}{6} + o(x^3)$ et $\dfrac{1}{1+x} = 1 - x + x^2 - x^3 + o(x^3)$.
    
    \begin{align*}
    \dfrac{e^x}{1+x} &= \left(1 + x + \dfrac{x^2}{2} + \dfrac{x^3}{6}\right)(1 - x + x^2 - x^3) + o(x^3) \\
    &= 1 - x + x^2 + x - x^2 + x^3 + \dfrac{x^2}{2} - \dfrac{x^3}{2} + \dfrac{x^3}{6} + o(x^3) \\
    &= 1 + \dfrac{x^2}{2} + \dfrac{2x^3}{3} + o(x^3)
    \end{align*}
    \end{solution}
    }{}
\end{enumerate}

\vspace{1em}
%------------------------------------------------------------------------------
\subsection*{Exercice 2 -- DL par composition}
%------------------------------------------------------------------------------

On cherche le DL de $\ln(\cos x)$ à l'ordre 4 au voisinage de 0.

\begin{enumerate}
    \item En posant $u = \cos x - 1$, montrer que $u = -\dfrac{x^2}{2} + \dfrac{x^4}{24} + o(x^4)$.
    
    \ifthenelse{\boolean{showSolutions}}{
    \begin{solution}
    On a $\cos x = 1 - \dfrac{x^2}{2} + \dfrac{x^4}{24} + o(x^4)$, donc :
    
    $u = \cos x - 1 = -\dfrac{x^2}{2} + \dfrac{x^4}{24} + o(x^4)$
    \end{solution}
    }{}
    
    \item En utilisant $\ln(1+u) = u - \dfrac{u^2}{2} + o(u^2)$, en déduire le DL de $\ln(\cos x)$ à l'ordre 4.
    
    \ifthenelse{\boolean{showSolutions}}{
    \begin{solution}
    On a $u^2 = \dfrac{x^4}{4} + o(x^4)$.
    \begin{align*}
    \ln(\cos x) &= \ln(1+u) = u - \dfrac{u^2}{2} + o(u^2) \\
    &= -\dfrac{x^2}{2} + \dfrac{x^4}{24} - \dfrac{1}{2} \cdot \dfrac{x^4}{4} + o(x^4) \\
    &= -\dfrac{x^2}{2} + \dfrac{x^4}{24} - \dfrac{x^4}{8} + o(x^4) \\
    &= -\dfrac{x^2}{2} - \dfrac{x^4}{12} + o(x^4)
    \end{align*}
    \end{solution}
    }{}
\end{enumerate}

\vspace{1em}
%------------------------------------------------------------------------------
\subsection*{Exercice 3 -- Calcul de limites par DL}
%------------------------------------------------------------------------------

Calculer les limites suivantes à l'aide de développements limités :

\begin{enumerate}
    \item $\displaystyle\lim_{x \to 0} \dfrac{\ln(1+x) - x + \frac{x^2}{2}}{x^3}$
    
    \ifthenelse{\boolean{showSolutions}}{
    \begin{solution}
    $\ln(1+x) = x - \dfrac{x^2}{2} + \dfrac{x^3}{3} + o(x^3)$, donc :
    
    $\ln(1+x) - x + \dfrac{x^2}{2} = \dfrac{x^3}{3} + o(x^3)$
    
    $\dfrac{\ln(1+x) - x + \frac{x^2}{2}}{x^3} = \dfrac{1}{3} + o(1) \xrightarrow[x \to 0]{} \boxed{\dfrac{1}{3}}$
    \end{solution}
    }{}
    
    \item $\displaystyle\lim_{x \to 0} \dfrac{\sin x - x\cos x}{x^3}$
    
    \ifthenelse{\boolean{showSolutions}}{
    \begin{solution}
    $\sin x = x - \dfrac{x^3}{6} + o(x^3)$ et $\cos x = 1 - \dfrac{x^2}{2} + o(x^2)$
    
    $x\cos x = x - \dfrac{x^3}{2} + o(x^3)$
    
    $\sin x - x\cos x = x - \dfrac{x^3}{6} - x + \dfrac{x^3}{2} + o(x^3) = \dfrac{x^3}{3} + o(x^3)$
    
    $\dfrac{\sin x - x\cos x}{x^3} = \dfrac{1}{3} + o(1) \xrightarrow[x \to 0]{} \boxed{\dfrac{1}{3}}$
    \end{solution}
    }{}
\end{enumerate}

\vspace{1em}
%------------------------------------------------------------------------------
\subsection*{Exercice 4 -- Prolongement et position par rapport à la tangente}
%------------------------------------------------------------------------------

Soit $f(x) = \dfrac{e^x - 1 - x}{x^2}$ pour $x \neq 0$.

\begin{enumerate}
    \item Montrer que $f$ admet un prolongement par continuité en 0 et déterminer $f(0)$.
    
    \ifthenelse{\boolean{showSolutions}}{
    \begin{solution}
    $e^x = 1 + x + \dfrac{x^2}{2} + \dfrac{x^3}{6} + o(x^3)$, donc $e^x - 1 - x = \dfrac{x^2}{2} + \dfrac{x^3}{6} + o(x^3)$.
    
    $f(x) = \dfrac{e^x - 1 - x}{x^2} = \dfrac{1}{2} + \dfrac{x}{6} + o(x) \xrightarrow[x \to 0]{} \dfrac{1}{2}$
    
    On peut prolonger $f$ par continuité en posant $f(0) = \dfrac{1}{2}$.
    \end{solution}
    }{}
    
    \item Donner le DL de $f$ à l'ordre 2 en 0, puis l'équation de la tangente à la courbe en 0.
    
    \ifthenelse{\boolean{showSolutions}}{
    \begin{solution}
    $e^x - 1 - x = \dfrac{x^2}{2} + \dfrac{x^3}{6} + \dfrac{x^4}{24} + o(x^4)$
    
    $f(x) = \dfrac{1}{2} + \dfrac{x}{6} + \dfrac{x^2}{24} + o(x^2)$
    
    La tangente en 0 a pour équation : $y = \dfrac{1}{2} + \dfrac{x}{6}$
    \end{solution}
    }{}
    
    \item En déduire la position de la courbe par rapport à sa tangente au voisinage de 0.
    
    \ifthenelse{\boolean{showSolutions}}{
    \begin{solution}
    $f(x) - \left(\dfrac{1}{2} + \dfrac{x}{6}\right) = \dfrac{x^2}{24} + o(x^2) \sim \dfrac{x^2}{24} > 0$ pour $x \neq 0$ petit.
    
    Donc la courbe est \textbf{au-dessus} de sa tangente au voisinage de 0.
    \end{solution}
    }{}
\end{enumerate}

\newpage
%==============================================================================
\section*{Équations différentielles exactes}
%==============================================================================

%------------------------------------------------------------------------------
\subsection*{Exercice 5 -- Équation exacte}
%------------------------------------------------------------------------------

Résoudre l'équation différentielle $(2xy + 1)\,dx + (x^2 + 2y)\,dy = 0$.

\begin{enumerate}
    \item Vérifier que l'équation est exacte.
    
    \ifthenelse{\boolean{showSolutions}}{
    \begin{solution}
    On pose $f(x,y) = 2xy + 1$ et $g(x,y) = x^2 + 2y$.
    
    $\dfrac{\partial f}{\partial y} = 2x$ et $\dfrac{\partial g}{\partial x} = 2x$
    
    Comme $\dfrac{\partial f}{\partial y} = \dfrac{\partial g}{\partial x}$, l'équation est exacte.
    \end{solution}
    }{}
    
    \item Trouver $F(x,y)$ telle que $dF = f\,dx + g\,dy$.
    
    \ifthenelse{\boolean{showSolutions}}{
    \begin{solution}
    On cherche $F$ telle que $\dfrac{\partial F}{\partial x} = 2xy + 1$.
    
    En intégrant par rapport à $x$ : $F(x,y) = x^2y + x + H(y)$
    
    On vérifie avec $\dfrac{\partial F}{\partial y} = x^2 + H'(y) = x^2 + 2y$.
    
    Donc $H'(y) = 2y$, soit $H(y) = y^2$.
    
    \textbf{Conclusion :} $F(x,y) = x^2y + x + y^2$
    \end{solution}
    }{}
    
    \item En déduire la solution générale.
    
    \ifthenelse{\boolean{showSolutions}}{
    \begin{solution}
    Les solutions sont données par $F(x,y) = K$ :
    \[\boxed{x^2y + x + y^2 = K}\]
    où $K$ est une constante.
    \end{solution}
    }{}
\end{enumerate}

\vspace{1em}

%==============================================================================
\section*{Équations aux dérivées partielles}
%==============================================================================

%------------------------------------------------------------------------------
\subsection*{Exercice 6 -- EDP}
%------------------------------------------------------------------------------

Résoudre l'équation $2\dfrac{\partial f}{\partial x} + \dfrac{\partial f}{\partial y} = 4x$ par changement de variables.

\ifthenelse{\boolean{showSolutions}}{
\begin{solution}
Avec $X = x - 2y$ et $Y = x$, on a $x = Y$ et $y = \frac{Y-X}{2}$.

On pose $F(X,Y) = f(x,y)$. Par la règle de la chaîne :

$\dfrac{\partial f}{\partial x} = \dfrac{\partial F}{\partial X} \cdot 1 + \dfrac{\partial F}{\partial Y} \cdot 1 = \dfrac{\partial F}{\partial X} + \dfrac{\partial F}{\partial Y}$

$\dfrac{\partial f}{\partial y} = \dfrac{\partial F}{\partial X} \cdot (-2) + \dfrac{\partial F}{\partial Y} \cdot 0 = -2\dfrac{\partial F}{\partial X}$

L'équation devient : $2\left(\dfrac{\partial F}{\partial X} + \dfrac{\partial F}{\partial Y}\right) - 2\dfrac{\partial F}{\partial X} = 4Y$

$2\dfrac{\partial F}{\partial Y} = 4Y$, soit $\dfrac{\partial F}{\partial Y} = 2Y$

En intégrant par rapport à $Y$ : $F(X,Y) = Y^2 + K(X)$

En revenant aux variables initiales : $\boxed{f(x,y) = x^2 + K(x-2y)}$

où $K$ est une fonction $\mathcal{C}^1$ quelconque.
\end{solution}
}{}
\vspace{1em}

Résoudre par séparation de variables : $\dfrac{\partial f}{\partial x} - 3\dfrac{\partial f}{\partial y} = 2f$.

\ifthenelse{\boolean{showSolutions}}{
\begin{solution}
On cherche $f(x,y) = X(x)Y(y)$.

$X'Y - 3XY' = 2XY$

En divisant par $XY$ : $\dfrac{X'}{X} - 3\dfrac{Y'}{Y} = 2$

Donc $\dfrac{X'}{X} = 2 + 3\dfrac{Y'}{Y}$. Le membre de gauche ne dépend que de $x$, le membre de droite que de $y$. Ils sont égaux à une constante $k$.

\textbf{Pour $X$ :} $\dfrac{X'}{X} = k \Rightarrow X(x) = C_1 e^{kx}$

\textbf{Pour $Y$ :} $3\dfrac{Y'}{Y} = k - 2 \Rightarrow Y' = \dfrac{k-2}{3}Y \Rightarrow Y(y) = C_2 e^{(k-2)y/3}$

\textbf{Solutions :} $\boxed{f(x,y) = C \, e^{kx} e^{(k-2)y/3}}$ pour $k \in \mathbb{R}$, $C \in \mathbb{R}$.
\end{solution}
}{}

\vspace{1em}

%==============================================================================
\section*{Modélisation}
%==============================================================================

%------------------------------------------------------------------------------
\subsection*{Exercice 7 -- Refroidissement de Newton}
%------------------------------------------------------------------------------

Un objet de température initiale $T_0 = 80$ °C est placé dans une pièce à température ambiante $T_a = 20$ °C. La loi de Newton stipule que :
\[
\frac{dT}{dt} = -k(T - T_a)
\]
avec $k = 0{,}1$ min$^{-1}$.

\begin{enumerate}
    \item En posant $\theta(t) = T(t) - T_a$, montrer que $\theta' = -k\theta$ et résoudre.
    
    \ifthenelse{\boolean{showSolutions}}{
    \begin{solution}
    $\theta' = T' = -k(T - T_a) = -k\theta$
    
    Donc $\theta(t) = \theta_0 e^{-kt}$ avec $\theta_0 = T_0 - T_a = 80 - 20 = 60$ °C.
    
    $\theta(t) = 60 e^{-0{,}1t}$, donc $T(t) = T_a + \theta(t) = 20 + 60e^{-0{,}1t}$ °C
    \end{solution}
    }{}
    
    \item Calculer la température après 10 minutes. \textit{(On donne $e^{-1} \approx 0{,}37$.)}
    
    \ifthenelse{\boolean{showSolutions}}{
    \begin{solution}
    $T(10) = 20 + 60e^{-1} \approx 20 + 60 \times 0{,}37 = 20 + 22{,}2 = \boxed{42{,}2 \text{ °C}}$
    \end{solution}
    }{}
    
    \item Au bout de combien de temps la température sera-t-elle de 30 °C ? \textit{(On donne $\ln 6 \approx 1{,}8$.)}
    
    \ifthenelse{\boolean{showSolutions}}{
    \begin{solution}
    On résout $T(t) = 30$ :
    
    $20 + 60e^{-0{,}1t} = 30 \Rightarrow 60e^{-0{,}1t} = 10 \Rightarrow e^{-0{,}1t} = \dfrac{1}{6}$
    
    $-0{,}1t = -\ln 6 \Rightarrow t = \dfrac{\ln 6}{0{,}1} = \dfrac{1{,}8}{0{,}1} = \boxed{18 \text{ min}}$
    \end{solution}
    }{}
\end{enumerate}

%------------------------------------------------------------------------------
\subsection*{Exercice 8 -- Croissance bactérienne avec limitation}
%------------------------------------------------------------------------------

Une population de bactéries $N(t)$ croît dans un milieu où les nutriments sont renouvelés à débit constant $D$, mais sont consommés proportionnellement à la population :
\[
\frac{dN}{dt} = D - kN
\]
avec $k = 0{,}5$ h$^{-1}$, $D = 1000$ bactéries/h, et $N(0) = 100$ bactéries.

\begin{enumerate}
    \item Déterminer la population d'équilibre $N_{eq}$.
    
    \ifthenelse{\boolean{showSolutions}}{
    \begin{solution}
    À l'équilibre, $\dfrac{dN}{dt} = 0$, donc $D - kN_{eq} = 0$.
    
    $N_{eq} = \dfrac{D}{k} = \dfrac{1000}{0{,}5} = 2000$ bactéries
    \end{solution}
    }{}
    
    \item Résoudre l'équation et donner $N(t)$.
    
    \ifthenelse{\boolean{showSolutions}}{
    \begin{solution}
    On pose $\theta = N - N_{eq}$. Alors $\theta' = N' = D - kN = D - k(N_{eq} + \theta) = -k\theta$.
    
    Donc $\theta(t) = \theta_0 e^{-kt}$ avec $\theta_0 = N(0) - N_{eq} = 100 - 2000 = -1900$.
    
    $N(t) = N_{eq} + \theta(t) = 2000 - 1900e^{-0{,}5t}$
    
    $\boxed{N(t) = 2000 - 1900e^{-0{,}5t}}$
    \end{solution}
    }{}
    
    \item Au bout de combien de temps la population atteint-elle 95\% de sa valeur d'équilibre ?
    
    \textit{(On donne $\ln 20 \approx 3$.)}
    
    \ifthenelse{\boolean{showSolutions}}{
    \begin{solution}
    On cherche $t$ tel que $N(t) = 0{,}95 \times 2000 = 1900$.
    
    $2000 - 1900e^{-0{,}5t} = 1900 \Rightarrow 1900e^{-0{,}5t} = 100 \Rightarrow e^{-0{,}5t} = \dfrac{1}{19} \approx \dfrac{1}{20}$
    
    $-0{,}5t = -\ln 20 \Rightarrow t = \dfrac{\ln 20}{0{,}5} = \dfrac{3}{0{,}5} = \boxed{6 \text{ h}}$
    \end{solution}
    }{}
\end{enumerate}

\end{document}


\setcounter{chapter}{6}
\chapter{Equations différentielles}
\section*{Les fonctions périodiques}

\subsection{Les fonctions complexes périodiques}

Les fonctions réelles suivantes sont-elles périodiques et si oui, quelle est leur période ?

\ifthenelse{\boolean{showSolutions}}{
    \vspace{2em}
    \begin{mdframed}
    \begin{enumerate}
    \item $\cos(x)$ : Oui, période $T = 2\pi$
    \item $\sin(2\pi x)$ : Oui, période $T = 1$ (car $\sin(2\pi(x+1)) = \sin(2\pi x + 2\pi) = \sin(2\pi x)$)
    \item $\cos(x/2)$ : Oui, période $T = 4\pi$ (car $\cos((x+4\pi)/2) = \cos(x/2 + 2\pi) = \cos(x/2)$)
    \item $\sin(2x) + \cos(3x)$ : Oui, période $T = 2\pi$ (le PPCM des périodes $\pi$ et $\frac{2\pi}{3}$)
    \item $\sin(nx)$ : Oui, période $T = \frac{2\pi}{n}$
    \item $\cos\left(\frac{3x}{2}-\frac{\pi}{4}\right)$ : Oui, période $T = \frac{4\pi}{3}$
    \item $x-\lfloor x\rfloor$ : Oui, période $T = 1$ (c'est la fonction partie fractionnaire)
\end{enumerate}
\end{mdframed}
}{}
\ifthenelse{\boolean{showSolutions}}{}
{\begin{multicols}{2}}
\begin{enumerate}
    \item $\displaystyle \cos(x)$
    
    \item $\displaystyle \sin(2\pi x)$
    \item $\displaystyle \cos(x/2)$
    \item $\displaystyle \sin(2x) + \cos(3x)$
    \item $\displaystyle \sin(nx)$, $n$ est un entier naturel non nul
    \item $\displaystyle \cos \left(\frac{3 x}{2}-\frac{\pi}{4}\right)$
    \item $\displaystyle x-\lfloor x\rfloor$
\end{enumerate}
\ifthenelse{\boolean{showSolutions}}{}{
\end{multicols}
}

Les fonctions complexes suivantes sont-elles périodiques et si oui, quelle est leur période ?

\ifthenelse{\boolean{showSolutions}}{
    \vspace{2em}
    \begin{mdframed}
    \begin{enumerate}
    \item $e^{ix}$ : Oui, période $T = 2\pi$ (car $e^{i(x+2\pi)} = e^{ix}e^{i2\pi} = e^{ix} \cdot 1 = e^{ix}$)
    \item $e^{2ix}$ : Oui, période $T = \pi$ (car $e^{2i(x+\pi)} = e^{2ix}e^{i2\pi} = e^{2ix}$)
    \item $e^{ix/2\pi}$ : Oui, période $T = 4\pi^2$ (car $e^{i(x+4\pi^2)/2\pi} = e^{ix/2\pi}e^{i2\pi} = e^{ix/2\pi}$)
    \item $e^{2i\pi x/T}$ : Oui, période $T$ (car $e^{2i\pi (x+T)/T} = e^{2i\pi x/T}e^{i2\pi} = e^{2i\pi x/T}$)
    \item $e^{inx} + e^{ipx}$ : Oui, période $T = \frac{2\pi}{\text{PGCD}(n,p)}$ (le PPCM des périodes $\frac{2\pi}{n}$ et $\frac{2\pi}{p}$)
\end{enumerate}
\end{mdframed}
}{}
\ifthenelse{\boolean{showSolutions}}{}
{\begin{multicols}{2}}
\begin{enumerate}
    \item $\displaystyle e^{ix}$
    \item $\displaystyle e^{2ix}$
    \item $\displaystyle e^{ix/2\pi}$
    \item $\displaystyle e^{2i\pi x/T}$, $T$ est un réel strictement positif
    \item $\displaystyle e^{inx} + e^{ipx}$
\end{enumerate}
\ifthenelse{\boolean{showSolutions}}{}{
\end{multicols}
}


\section*{Produit scalaire réel}

\subsection{Définition}
On appelle produit scalaire sur un espace vectoriel $E$ une application 
$$\langle \cdot, \cdot \rangle : E \times E \to \mathbb{R}$$
telle que :
\begin{multicols}{2}
\begin{itemize}
    \item[*] symétrie : $\langle u, v \rangle = \langle v, u \rangle$ 
    \item[*] linéarité à gauche : $\langle \lambda u + v, w \rangle = \lambda \langle u, w \rangle + \langle v, w \rangle$
    \item[*] positivité : $\langle u, u \rangle \geq 0$
    \item[*] définie positivité : $\langle u, u \rangle = 0 \iff u = 0$
\end{itemize}
\end{multicols}


\vspace{1em}

\subsection{Dans $\mathbb{R}^3$}

On se place dans $\mathbb{R}^3$, qu'on munit de la base 
$$e_1 = (1,2,1), \qquad e_2 = (2,1,-4), \qquad e_3 = (-3,2,-1)$$

\begin{enumerate}
\item La famille est-elle orthogonale ? 
\item Est-elle orthonormée ? Si non, définissez une base $(f_1,f_2,f_3)$ orthonormée à partir de la famille $(e_1,e_2,e_3)$. 
\end{enumerate}

Soit $u$ un vecteur de $\mathbb{R}^3$, on note $u_i$ ses coordonnées dans la base orthonormée $(f_1,f_2,f_3)$. Cela signifie que 
$$u = u_1 f_1 + u_2 f_2 + u_3 f_3$$

Déterminer les coordonnées de $u = (1,0,1)$ dans la base $(f_1,f_2,f_3)$.

\ifthenelse{\boolean{showSolutions}}{
    \vspace{2em}
    \begin{mdframed}
    \textbf{1.} Vérifions si la famille est orthogonale :
    
    $\langle e_1, e_2 \rangle = 1 \cdot 2 + 2 \cdot 1 + 1 \cdot (-4) = 2 + 2 - 4 = 0$ $\checkmark$
    
    $\langle e_1, e_3 \rangle = 1 \cdot (-3) + 2 \cdot 2 + 1 \cdot (-1) = -3 + 4 - 1 = 0$ $\checkmark$
    
    $\langle e_2, e_3 \rangle = 2 \cdot (-3) + 1 \cdot 2 + (-4) \cdot (-1) = -6 + 2 + 4 = 0$ $\checkmark$
    
    La famille est orthogonale.
    
    \textbf{2.} Vérifions si elle est orthonormée :
    
    $\|e_1\|^2 = 1^2 + 2^2 + 1^2 = 6$, donc $\|e_1\| = \sqrt{6}$
    
    $\|e_2\|^2 = 2^2 + 1^2 + (-4)^2 = 4 + 1 + 16 = 21$, donc $\|e_2\| = \sqrt{21}$
    
    $\|e_3\|^2 = (-3)^2 + 2^2 + (-1)^2 = 9 + 4 + 1 = 14$, donc $\|e_3\| = \sqrt{14}$
    
    La famille n'est pas orthonormée. Une base orthonormée est :
    $$f_1 = \frac{e_1}{\|e_1\|} = \frac{1}{\sqrt{6}}(1,2,1) = \left(\frac{1}{\sqrt{6}}, \frac{2}{\sqrt{6}}, \frac{1}{\sqrt{6}}\right)$$
    
    $$f_2 = \frac{e_2}{\|e_2\|} = \frac{1}{\sqrt{21}}(2,1,-4) = \left(\frac{2}{\sqrt{21}}, \frac{1}{\sqrt{21}}, \frac{-4}{\sqrt{21}}\right)$$
    
    $$f_3 = \frac{e_3}{\|e_3\|} = \frac{1}{\sqrt{14}}(-3,2,-1) = \left(\frac{-3}{\sqrt{14}}, \frac{2}{\sqrt{14}}, \frac{-1}{\sqrt{14}}\right)$$
    
    \textbf{3.} Coordonnées de $u = (1,0,1)$ dans la base $(f_1,f_2,f_3)$ :
    
    $u_1 = \langle u, f_1 \rangle = 1 \cdot \frac{1}{\sqrt{6}} + 0 \cdot \frac{2}{\sqrt{6}} + 1 \cdot \frac{1}{\sqrt{6}} = \frac{2}{\sqrt{6}} = \frac{\sqrt{6}}{3}$
    
    $u_2 = \langle u, f_2 \rangle = 1 \cdot \frac{2}{\sqrt{21}} + 0 \cdot \frac{1}{\sqrt{21}} + 1 \cdot \frac{-4}{\sqrt{21}} = \frac{-2}{\sqrt{21}}$
    
    $u_3 = \langle u, f_3 \rangle = 1 \cdot \frac{-3}{\sqrt{14}} + 0 \cdot \frac{2}{\sqrt{14}} + 1 \cdot \frac{-1}{\sqrt{14}} = \frac{-4}{\sqrt{14}}$
    
    Donc $u = \frac{\sqrt{6}}{3}f_1 - \frac{2}{\sqrt{21}}f_2 - \frac{4}{\sqrt{14}}f_3$
\end{mdframed}
}{}

\vspace{1em}

\subsection{Dans $\mathbb{R}[X]$}

\begin{itemize}
    \item Quelle est la dimension de $\mathbb{R}[X]$ ?
\end{itemize}
La famille $(1,X,X^2,X^3, \cdots )$ est appelée base hilbertienne de $\mathbb{R}[X]$ : tout élément de $\mathbb{R}[X]$ peut s'écrire comme une combinaison linéaire finie de vecteurs de cette famille.

On munit cet espace du produit scalaire : 
$$ \langle P, Q \rangle = \int_{0}^{1} P(x) Q(x) dx $$

\begin{itemize}
    \item Montrer que c'est bien un produit scalaire en vérifiant les propriétés ci-dessus.
    \item La famille $(1,X,X^2,X^3, \cdots )$ est-elle orthogonale ? Est-elle orthonormée ?
    \item Comment trouver $a, b, c$ tels que la famille $(1, X-a, X^2-bX-c)$ soit orthogonale ?
    \item Quelles sont les coordonnées de $P = 1+2X+3X^2$ dans la base $(1, X, X^2, \cdots)$ ?
    \item Peut-on retrouver ces coordonnées avec le produit scalaire comme dans l'exercice précédent ?
\end{itemize}
\vspace{1em}

\section*{Produit scalaire complexe}
\subsection{Définition}
On appelle produit scalaire sur un espace vectoriel $E$ une application 
$$\langle \cdot, \cdot \rangle : E \times E \to \mathbb{R}$$
telle que :
\begin{multicols}{2}
\begin{itemize}
    \item[*] symétrie conjuguée : $\langle u, v \rangle = \overline{\langle v, u \rangle}$
    \item[*] linéarité à gauche : $\langle \lambda u + v, w \rangle = \lambda \langle u, w \rangle + \langle v, w \rangle$
    \item[*] positivité : $\langle u, u \rangle \geq 0$
    \item[*] définie positivité : $\langle u, u \rangle = 0 \iff u = 0$
\end{itemize}
\end{multicols}


\subsection{Dans l'espace des fonctions complexes $2\pi$-périodiques}

On définit le produit scalaire :
$$
\langle f, g \rangle = \int_{0}^{2\pi} f(x) \overline{g(x)} dx
$$

Montrer que c'est un produit scalaire.

Montrer que la famille $(e^{inx})_{n \in \mathbb{Z}}$ est orthonormée.

Les coefficients de Fourier d'une fonction $f$ sont les coordonnées de $f$ dans la base $(e^{inx})_{n \in \mathbb{Z}}$.

Déterminer les coefficients de Fourier des fonctions suivantes :

\ifthenelse{\boolean{showSolutions}}{
    \vspace{2em}
    \begin{mdframed}
    \textbf{Démonstration que c'est un produit scalaire :}
    
    Les propriétés de symétrie conjuguée, linéarité et positivité découlent des propriétés de l'intégrale et du conjugué complexe.
    
    \textbf{Démonstration que $(e^{inx})_{n \in \mathbb{Z}}$ est orthonormée :}
    
    Pour $n = m$ : $\langle e^{inx}, e^{inx} \rangle = \int_0^{2\pi} e^{inx} \overline{e^{inx}} dx = \int_0^{2\pi} 1 dx = 2\pi$
    
    Pour $n \neq m$ : $\langle e^{inx}, e^{imx} \rangle = \int_0^{2\pi} e^{inx} \overline{e^{imx}} dx = \int_0^{2\pi} e^{i(n-m)x} dx = 0$
    
    Donc la famille $(e^{inx})_{n \in \mathbb{Z}}$ est orthogonale. Pour l'orthonormaliser, on divise par $\sqrt{2\pi}$.
    
    \textbf{Coefficients de Fourier :}
    \begin{enumerate}
    \item $\cos(x) = \frac{e^{ix} + e^{-ix}}{2}$, donc $c_1 = \frac{1}{2}$, $c_{-1} = \frac{1}{2}$, $c_n = 0$ sinon.
    
    \item $\sin(2\pi x)$ : Cette fonction n'est pas $2\pi$-périodique ! Elle est de période $1$.
    
    \item $\cos(x/2) = \frac{e^{ix/2} + e^{-ix/2}}{2}$, donc $c_{1/2} = \frac{1}{2}$, $c_{-1/2} = \frac{1}{2}$, $c_n = 0$ sinon.
    
    \item $\sin(2x) + \cos(3x) = \frac{e^{i2x} - e^{-i2x}}{2i} + \frac{e^{i3x} + e^{-i3x}}{2}$
    Donc $c_2 = \frac{1}{2i}$, $c_{-2} = -\frac{1}{2i}$, $c_3 = \frac{1}{2}$, $c_{-3} = \frac{1}{2}$, $c_n = 0$ sinon.
    
    \item $f(x) = e^{-x}$ sur $[0, 2\pi]$ :
    $$c_n = \frac{1}{2\pi}\int_0^{2\pi} e^{-x} e^{-inx} dx = \frac{1}{2\pi}\int_0^{2\pi} e^{-(1+in)x} dx$$
    $$= \frac{1}{2\pi}\left[\frac{e^{-(1+in)x}}{-(1+in)}\right]_0^{2\pi} = \frac{1}{2\pi} \cdot \frac{1-e^{-2\pi(1+in)}}{1+in} = \frac{1-e^{-2\pi}e^{-2\pi in}}{2\pi(1+in)}$$
\end{enumerate}
\end{mdframed}
}{}
\ifthenelse{\boolean{showSolutions}}{}
{\begin{multicols}{2}}
\begin{enumerate}
\item $\displaystyle \cos(x)$
\item $\displaystyle \sin(2\pi x)$
\item $\displaystyle \cos(x/2)$
\item $\displaystyle \sin(2x) + \cos(3x)$
\item $\displaystyle \exp^{-x}$ sur l'intervalle $[0, 2\pi]$
\end{enumerate}
\ifthenelse{\boolean{showSolutions}}{}{
\end{multicols}
}

\end{document}
