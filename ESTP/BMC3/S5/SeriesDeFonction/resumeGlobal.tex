% =====================================================================
% Template LaTeX – Traces distribuées aux étudiants
% Auteur : (à compléter)
% Compilation : pdflatex/xelatex (pdflatex recommandé ici)
% =====================================================================
\documentclass[11pt,a4paper]{report}

% -------------------- Encodage & langue --------------------
\usepackage[T1]{fontenc}
\usepackage[utf8]{inputenc}
\usepackage[french]{babel}
\usepackage{lmodern}
\usepackage{microtype}
\usepackage{amsmath, amssymb}
\usepackage{multicol}
\usepackage{enumitem}
\usepackage{ifthen}


\DeclareMathOperator{\ch}{ch}
\DeclareMathOperator{\sh}{sh}
\DeclareMathOperator{\vect}{vect}
\DeclareMathOperator{\card}{card}
\DeclareMathOperator{\comat}{comat}
\DeclareMathOperator{\imv}{Im}
\DeclareMathOperator{\rang}{rg}
\DeclareMathOperator{\Fr}{Fr}
\DeclareMathOperator{\diam}{diam}
\DeclareMathOperator{\supp}{supp}

% -------------------- Mise en page --------------------------
\usepackage[a4paper,margin=2cm]{geometry}
\usepackage{fancyhdr}
\usepackage{parskip}      % espace entre paragraphes
\setlength{\parindent}{0pt}

% -------------------- Couleurs & liens ----------------------
\usepackage{xcolor}
\definecolor{Theme}{HTML}{0E7490} % teal-700
\definecolor{ThemeLight}{HTML}{E0F2F1}
\definecolor{Accent}{HTML}{F59E0B} % amber-500
\definecolor{Gray}{HTML}{374151}
\usepackage[colorlinks=true,linkcolor=Theme,urlcolor=Theme,citecolor=Theme]{hyperref}

% -------------------- Graphiques / décor --------------------
\usepackage{tikz}
\usetikzlibrary{patterns,positioning,calc}
\usepackage{graphicx}
\usepackage{tcolorbox}
\tcbuselibrary{skins,breakable,hooks,most}
\usepackage{fontawesome5}

% -------------------- Titres -------------------------------
\usepackage{titlesec}
\titleformat{\chapter}[display]
  {\huge\bfseries\color{Theme}}
  {\filright\rule{0.75\linewidth}{1.2pt}\\[1pt]{Séries de fonctions - Chapitre~\thechapter}}
  {0.1ex}
  {\filright}
  [\vspace{0ex}\rule{0.35\linewidth}{1.2pt}]

\titlespacing*{\chapter}{0pt}{0.5ex}{0.25ex}

\titleformat{\section}
  {\Large\bfseries\color{Gray}}
  {\thesection}{0.6em}{}

% -------------------- En-têtes / pieds ---------------------
\pagestyle{fancy}
\fancyhf{}
\fancyhead[L]{\color{Gray}\leftmark}
\fancyhead[R]{\color{Gray}\textit{BMC3}}
\fancyfoot[L]{\color{Gray}\small Auteur~: \textit{M. Berger}}
\fancyfoot[R]{\color{Gray}\small p.\ \thepage}
\renewcommand{\headrulewidth}{0pt}
\renewcommand{\footrulewidth}{0pt}

% -------------------- Macros utilitaires -------------------


% Tcolorboxes stylisées
\tcbset{tracebox/.style={breakable,enhanced,sharp corners,boxrule=0pt,frame hidden,arc=2mm,
  colback=white,coltitle=black,fonttitle=\bfseries\large,
  borderline west={2mm}{0pt}{Theme},
  before skip=8pt,after skip=8pt,drop fuzzy shadow}}

\newtcolorbox{resumeBox}{tracebox,title={\faStickyNote\quad Résumé des idées}}
\newtcolorbox{rappelsBox}{tracebox,title={\faRedo\quad Questions de cours }}
\newtcolorbox{exempleBox}{tracebox,title={\faChalkboardTeacher\quad Exemple vu ensemble}}

% Encadré « Formules & illustrations »
\newtcolorbox{formulesBox}{tracebox,title={\faCalculator\quad Formules \& illustrations},colback=ThemeLight}

% Astuce : puces clean
\newenvironment{niceitemize}{\begin{itemize}\setlength{\itemsep}{0.25em}}{\end{itemize}}

% Raccourci pour une « Trace » complète
% Usage : \TraceSection{Titre}{Objectif court}
\newcommand{\TraceSection}[2]{%
  
}

% -------------------- Page de titre ------------------------
\title{\textbf{Traces de cours}\\\large (résumés, formules, exemples, mini-exercices)}
\author{ BMC3 }
\date{\today}


\makeatletter
\renewcommand{\thesubsection}{\arabic{subsection}}
\renewcommand{\p@subsection}{}% supprime le préfixe section/chapter dans \ref
% Si vous voulez la même chose pour les sous-sous-sections :
% \renewcommand{\thesubsubsection}{\arabic{subsubsection}}
% \renewcommand{\p@subsubsection}{}
\makeatother

\usepackage{mdframed}
\usepackage{ifthen}

% \usepackage[sf]{titlesec}
% Définition de la variable pour afficher les corrections
\newboolean{showSolutions}
% Décommentez la ligne suivante pour afficher les solutions
\input \jobname.adr

% -------------------- Document ----------------------------
\begin{document}



% ================== Séquence 1 ==================

% \chapter{Suites et séries numériques}
% 
\newcommand{\mtn}{\mathbb{N}}
\newcommand{\mtns}{\mathbb{N}^*}
\newcommand{\mtz}{\mathbb{Z}}
\newcommand{\mtr}{\mathbb{R}}
\newcommand{\mtk}{\mathbb{K}}
\newcommand{\mtq}{\mathbb{Q}}
\newcommand{\mtc}{\mathbb{C}}
\newcommand{\mch}{\mathcal{H}}
\newcommand{\mcp}{\mathcal{P}}
\newcommand{\mcb}{\mathcal{B}}
\newcommand{\mcl}{\mathcal{L}}
\newcommand{\mcm}{\mathcal{M}}
\newcommand{\mcc}{\mathcal{C}}
\newcommand{\mcmn}{\mathcal{M}}
\newcommand{\mcmnr}{\mathcal{M}_n(\mtr)}
\newcommand{\mcmnk}{\mathcal{M}_n(\mtk)}
\newcommand{\mcsn}{\mathcal{S}_n}
\newcommand{\mcs}{\mathcal{S}}
\newcommand{\mcd}{\mathcal{D}}
\newcommand{\mcsns}{\mathcal{S}_n^{++}}
\newcommand{\glnk}{GL_n(\mtk)}
\newcommand{\mnr}{\mathcal{M}_n(\mtr)}
\newcommand{\veps}{\varepsilon}
\newcommand{\mcu}{\mathcal{U}}
\newcommand{\mcun}{\mcu_n}
\newcommand{\dis}{\displaystyle}
\newcommand{\croouv}{[\![}
\newcommand{\crofer}{]\!]}
\newcommand{\rab}{\mathcal{R}(a,b)}
\newcommand{\pss}[2]{\langle #1,#2\rangle}
 %Document 


\section*{Rappels sur les nombres complexes}

\subsection*{Définitions et formes usuelles}

Un \underline{nombre complexe} $z$ s'écrit sous la forme :
\[
z = a + ib \qquad (a, b \in \mathbb{R},\ i^2 = -1)
\]
où $a$ est la partie réelle $\Re(z)$ et $b$ la partie imaginaire $\Im(z)$.


\underline{Forme algébrique} : $z = a + ib$

\underline{Forme trigonométrique} : 
\[
z = r(\cos\theta + i\sin\theta)
\]
où $r = |z| = \sqrt{a^2 + b^2}$ est le \textbf{module} de $z$, et $\theta = \arg(z)$ est un \textbf{argument} de $z$ (défini à $2\pi$ près).

\underline{Forme exponentielle} (formule d'Euler) :
\[
z = r e^{i\theta}
\]
avec $e^{i\theta} = \cos\theta + i\sin\theta$.


\subsection*{Module, argument et conjugué}

\begin{itemize}
    \item \underline{Module} : $|z| = \sqrt{a^2 + b^2}$
    \item \underline{Argument} : $\theta = \arctan\left(\frac{b}{a}\right)$ (attention au quadrant)
    \item \underline{Conjugué} : $\overline{z} = a - ib$
\end{itemize}

\subsection*{Formule de Moivre}

Pour tout $n \in \mathbb{Z}$,
\[
\left(\cos\theta + i\sin\theta\right)^n = \cos(n\theta) + i\sin(n\theta)
\]
ou, sous forme exponentielle :
\[
\left(e^{i\theta}\right)^n = e^{in\theta}
\]

\subsection*{Racines $n$-ièmes de l'unité}

Les solutions de $z^n = 1$ sont :
\[
z_k = e^{i\frac{2\pi k}{n}},\quad k = 0, 1, \ldots, n-1
\]



\vspace{3em}

\subsection{Module et argument}

Écrire sous la forme $a+i b$, puis sous forme exponentielle les nombres complexes suivants :
\begin{enumerate}
\item Nombre de module 2 et d'argument $\pi / 3$.
\item Nombre de module 3 et d'argument $-\pi / 8$.
\item Nombre de module 1 et d'argument $\pi / 4$.
\item Nombre de module 2 et d'argument $-\pi / 6$.
\item Nombre de module 7 et d'argument $-\pi / 2$.
\end{enumerate}

\vspace{2em}

\subsection{Forme exponentielle $\rightarrow$ forme algébrique}

Écrire sous la forme $a+ib$ les nombres complexes suivants, donnés sous forme exponentielle :
\begin{multicols}{2}
\begin{enumerate}
    \item $z_1 = 5 e^{i \frac{\pi}{6}}$
    \item $z_2 = 2 e^{-i \frac{\pi}{4}}$
    \item $z_3 = 3 e^{i \frac{2\pi}{3}}$
    \item $z_4 = 7 e^{i \pi}$
    \item $z_5 = 4 e^{i 0}$
    \item $z_6 = 6 e^{-i \frac{\pi}{2}}$
\end{enumerate}
\end{multicols}

\vspace{2em}

\subsection{Forme exponentielle}
Mettre sous forme exponentielle les nombres complexes suivants : 
\begin{multicols}{3}
\begin{enumerate}
    \item $z_1=1+i \sqrt{3}$, 
    \item $z_2=1+i$, 
    \item $z_3=-2 \sqrt{3}+2 i$, 
    \item $z_4=i$, 
    \item $z_5=-2 i$, 
    \item $z_6=-3$,
    \item $z_7=1$
    \item $z_8=9 i$
    \item $z_9=0$
    \item $z_{10}=\frac{-i \sqrt{2}}{1+i}$
    \item $z_{11}=\frac{(1+i \sqrt{3})^3}{(1-i)^5}$
    \item $z_{12}=\sin x+i \cos x$.
\end{enumerate}
\end{multicols}



\vspace{2em}

\subsection{Exponentielle}
Résoudre l'équation $e^z=3 \sqrt{3}-3 i$.

\newpage

\subsection{Trigonométrique}
En utilisant les nombres complexes, calculer  $\cos 5 \theta$ et $\sin 5 \theta$ en fonction de $\cos \theta$ et $\sin \theta$.

\vspace{2em}


\subsection{Pour préparer les séries de fourier}
Calculer les intégrales suivantes, pour toute valeur de $n$ et $m$ dans les entiers relatifs:

\begin{enumerate}
    \item $$\int_0^\pi e^{i n x} e^{i m x} dx$$
    \item $$\int_0^\pi \cos(n x) \cos(m x) dx$$
    \item $$\int_0^\pi \sin(n x) \sin(m x) dx$$
    \item $$\int_0^\pi \cos(n x) \sin(m x) dx$$
\end{enumerate}



\vspace{2em}

\subsection{Exponentielle}
On pose 
$$z_1=4 e^{i \frac{\pi}{4}}, \qquad z_2=3 i e^{i \frac{\pi}{6}}, \qquad z_3=-2 e^{i \frac{2 \pi}{3}}$$
Écrire sous forme exponentielle les nombres complexes : 
$$z_1,\qquad z_2,\qquad z_3 , \qquad z_1 z_2, \qquad \frac{z_1 z_2}{z_2}$$



\vspace{2em}

\subsection{Racines carrées}
Calculer de deux façons les racines carrées de $1+i$ et en déduire les valeurs exactes de $\cos \left(\frac{\pi}{8}\right)$ et $\sin \left(\frac{\pi}{8}\right)$.

% \setcounter{chapter}{0}
% \chapter{Contrôle continu}
% 
Appliquer le pivot de Gauss pour transformer ces systèmes en systèmes échelonnés. Indiquez pour chacun s'il possède aucune solution, une unique solution ou une infinité de solutions.

$$
\left\{\begin{array}{l}
    2 x+3 y-z=1 \\
    4 x+y+2 z=6 \\
    x-3 y+z=2
\end{array}\right.
$$

\ifthenelse{\boolean{showSolutions}}{
    \vspace{1em}

\begin{mdframed}
    On applique les opérations suivantes pour enlever les $x$ des lignes 2 et 3 :
    \begin{align*}
        L_2 &\leftarrow L_2 - 2L_1 \\
        L_3 &\leftarrow 2L_3 - L_1
    \end{align*}
    On obtient le système suivant :
    $$
    \left\{\begin{array}{l}
        2 x+3 y-z=1 \\
        \quad -5 y+4 z=4 \\
        \quad -9 y+3 z=3
    \end{array}\right.
    $$
    Pour retirer le $y$ de la ligne 3, on applique $L_3 \leftarrow 5L_3 + 9L_2$ :
    $$
    \left\{\begin{array}{l}
        2 x+3 y-z=1 \\
        \quad -5 y+4 z=4 \\
        \quad \quad -21z=-21
    \end{array}\right.
    $$
    Le système est parfaitement échelonné, il admet une unique solution. 
\end{mdframed}
}{}

$$
\left\{\begin{array}{l}
    2 x+y=1 \\
    x+y=0 \\
    3 x+4 y=-1
\end{array}\right.
$$
\ifthenelse{\boolean{showSolutions}}{
    \vspace{1em}

\begin{mdframed}
    On applique les opérations suivantes pour enlever les $x$ des lignes 2 et 3 :
    \begin{align*}
        L_2 &\leftarrow 2L_2 - L_1 \\
        L_3 &\leftarrow 2L_3 - 3L_1
    \end{align*}
    On obtient le système suivant :
    $$
    \left\{\begin{array}{l}
        2 x+y=1 \\
        \quad y=-1 \\
        \quad 5 y=-5
    \end{array}\right.
    $$
    Les deux dernières équations sont les mêmes, on se ramène donc au système :
    $$
    \left\{\begin{array}{l}
        2 x+y=1 \\
        \quad y=-1
    \end{array}\right.
    $$
    Le système est parfaitement échelonné, il admet une unique solution. 
\end{mdframed}
}{}
$$
\left\{\begin{array}{l}
    x+y+z+t=3 \\
    x+y+z-t=3 \\
    x-y-z-t=-1
\end{array}\right.
$$
\ifthenelse{\boolean{showSolutions}}{
    \vspace{1em}
\begin{mdframed}
    On applique les opérations suivantes pour enlever les $x$ des lignes 2 et 3 :
    \begin{align*}
        L_2 &\leftarrow L_2 - L_1 \\
        L_3 &\leftarrow L_3 - L_1
    \end{align*}
    On obtient le système suivant :
    $$
    \left\{\begin{array}{l}
        x+y+z+t=3 \\
        \quad \quad \quad -2 t=6 \\
        \quad -2y -2z -2 t=-4
    \end{array}\right.
    $$
    En échangeant les lignes 2 et 3, on obtient un système échelonné :
    $$
    \left\{\begin{array}{l}
        x+y+z+t=3 \\
        \quad -2y -2z -2 t=-4 \\
        \quad \quad \quad -2 t=6
    \end{array}\right.
    $$
    Le système échelonné possède plus d'inconnues que d'équations, on peut donc garder un paramètre libre 
\end{mdframed}
}{}


% \setcounter{chapter}{1}
% \chapter{Révisions sur les intégrales}
% 
\newcommand{\mtn}{\mathbb{N}}
\newcommand{\mtns}{\mathbb{N}^*}
\newcommand{\mtz}{\mathbb{Z}}
\newcommand{\mtr}{\mathbb{R}}
\newcommand{\mtk}{\mathbb{K}}
\newcommand{\mtq}{\mathbb{Q}}
\newcommand{\mtc}{\mathbb{C}}
\newcommand{\mch}{\mathcal{H}}
\newcommand{\mcp}{\mathcal{P}}
\newcommand{\mcb}{\mathcal{B}}
\newcommand{\mcl}{\mathcal{L}}
\newcommand{\mcm}{\mathcal{M}}
\newcommand{\mcc}{\mathcal{C}}
\newcommand{\mcmn}{\mathcal{M}}
\newcommand{\mcmnr}{\mathcal{M}_n(\mtr)}
\newcommand{\mcmnk}{\mathcal{M}_n(\mtk)}
\newcommand{\mcsn}{\mathcal{S}_n}
\newcommand{\mcs}{\mathcal{S}}
\newcommand{\mcd}{\mathcal{D}}
\newcommand{\mcsns}{\mathcal{S}_n^{++}}
\newcommand{\glnk}{GL_n(\mtk)}
\newcommand{\mnr}{\mathcal{M}_n(\mtr)}
\newcommand{\veps}{\varepsilon}
\newcommand{\mcu}{\mathcal{U}}
\newcommand{\mcun}{\mcu_n}
\newcommand{\dis}{\displaystyle}
\newcommand{\croouv}{[\![}
\newcommand{\crofer}{]\!]}
\newcommand{\rab}{\mathcal{R}(a,b)}
\newcommand{\pss}[2]{\langle #1,#2\rangle}
 %Document 


\section*{Rappels sur les nombres complexes}

\subsection*{Définitions et formes usuelles}

Un \underline{nombre complexe} $z$ s'écrit sous la forme :
\[
z = a + ib \qquad (a, b \in \mathbb{R},\ i^2 = -1)
\]
où $a$ est la partie réelle $\Re(z)$ et $b$ la partie imaginaire $\Im(z)$.


\underline{Forme algébrique} : $z = a + ib$

\underline{Forme trigonométrique} : 
\[
z = r(\cos\theta + i\sin\theta)
\]
où $r = |z| = \sqrt{a^2 + b^2}$ est le \textbf{module} de $z$, et $\theta = \arg(z)$ est un \textbf{argument} de $z$ (défini à $2\pi$ près).

\underline{Forme exponentielle} (formule d'Euler) :
\[
z = r e^{i\theta}
\]
avec $e^{i\theta} = \cos\theta + i\sin\theta$.


\subsection*{Module, argument et conjugué}

\begin{itemize}
    \item \underline{Module} : $|z| = \sqrt{a^2 + b^2}$
    \item \underline{Argument} : $\theta = \arctan\left(\frac{b}{a}\right)$ (attention au quadrant)
    \item \underline{Conjugué} : $\overline{z} = a - ib$
\end{itemize}

\subsection*{Formule de Moivre}

Pour tout $n \in \mathbb{Z}$,
\[
\left(\cos\theta + i\sin\theta\right)^n = \cos(n\theta) + i\sin(n\theta)
\]
ou, sous forme exponentielle :
\[
\left(e^{i\theta}\right)^n = e^{in\theta}
\]

\subsection*{Racines $n$-ièmes de l'unité}

Les solutions de $z^n = 1$ sont :
\[
z_k = e^{i\frac{2\pi k}{n}},\quad k = 0, 1, \ldots, n-1
\]



\vspace{3em}

\subsection{Module et argument}

Écrire sous la forme $a+i b$, puis sous forme exponentielle les nombres complexes suivants :
\begin{enumerate}
\item Nombre de module 2 et d'argument $\pi / 3$.
\item Nombre de module 3 et d'argument $-\pi / 8$.
\item Nombre de module 1 et d'argument $\pi / 4$.
\item Nombre de module 2 et d'argument $-\pi / 6$.
\item Nombre de module 7 et d'argument $-\pi / 2$.
\end{enumerate}

\vspace{2em}

\subsection{Forme exponentielle $\rightarrow$ forme algébrique}

Écrire sous la forme $a+ib$ les nombres complexes suivants, donnés sous forme exponentielle :
\begin{multicols}{2}
\begin{enumerate}
    \item $z_1 = 5 e^{i \frac{\pi}{6}}$
    \item $z_2 = 2 e^{-i \frac{\pi}{4}}$
    \item $z_3 = 3 e^{i \frac{2\pi}{3}}$
    \item $z_4 = 7 e^{i \pi}$
    \item $z_5 = 4 e^{i 0}$
    \item $z_6 = 6 e^{-i \frac{\pi}{2}}$
\end{enumerate}
\end{multicols}

\vspace{2em}

\subsection{Forme exponentielle}
Mettre sous forme exponentielle les nombres complexes suivants : 
\begin{multicols}{3}
\begin{enumerate}
    \item $z_1=1+i \sqrt{3}$, 
    \item $z_2=1+i$, 
    \item $z_3=-2 \sqrt{3}+2 i$, 
    \item $z_4=i$, 
    \item $z_5=-2 i$, 
    \item $z_6=-3$,
    \item $z_7=1$
    \item $z_8=9 i$
    \item $z_9=0$
    \item $z_{10}=\frac{-i \sqrt{2}}{1+i}$
    \item $z_{11}=\frac{(1+i \sqrt{3})^3}{(1-i)^5}$
    \item $z_{12}=\sin x+i \cos x$.
\end{enumerate}
\end{multicols}



\vspace{2em}

\subsection{Exponentielle}
Résoudre l'équation $e^z=3 \sqrt{3}-3 i$.

\newpage

\subsection{Trigonométrique}
En utilisant les nombres complexes, calculer  $\cos 5 \theta$ et $\sin 5 \theta$ en fonction de $\cos \theta$ et $\sin \theta$.

\vspace{2em}


\subsection{Pour préparer les séries de fourier}
Calculer les intégrales suivantes, pour toute valeur de $n$ et $m$ dans les entiers relatifs:

\begin{enumerate}
    \item $$\int_0^\pi e^{i n x} e^{i m x} dx$$
    \item $$\int_0^\pi \cos(n x) \cos(m x) dx$$
    \item $$\int_0^\pi \sin(n x) \sin(m x) dx$$
    \item $$\int_0^\pi \cos(n x) \sin(m x) dx$$
\end{enumerate}



\vspace{2em}

\subsection{Exponentielle}
On pose 
$$z_1=4 e^{i \frac{\pi}{4}}, \qquad z_2=3 i e^{i \frac{\pi}{6}}, \qquad z_3=-2 e^{i \frac{2 \pi}{3}}$$
Écrire sous forme exponentielle les nombres complexes : 
$$z_1,\qquad z_2,\qquad z_3 , \qquad z_1 z_2, \qquad \frac{z_1 z_2}{z_2}$$



\vspace{2em}

\subsection{Racines carrées}
Calculer de deux façons les racines carrées de $1+i$ et en déduire les valeurs exactes de $\cos \left(\frac{\pi}{8}\right)$ et $\sin \left(\frac{\pi}{8}\right)$.

% \setcounter{chapter}{2}
% \chapter{Révisions sur les complexes}
% 
\newcommand{\mtn}{\mathbb{N}}
\newcommand{\mtns}{\mathbb{N}^*}
\newcommand{\mtz}{\mathbb{Z}}
\newcommand{\mtr}{\mathbb{R}}
\newcommand{\mtk}{\mathbb{K}}
\newcommand{\mtq}{\mathbb{Q}}
\newcommand{\mtc}{\mathbb{C}}
\newcommand{\mch}{\mathcal{H}}
\newcommand{\mcp}{\mathcal{P}}
\newcommand{\mcb}{\mathcal{B}}
\newcommand{\mcl}{\mathcal{L}}
\newcommand{\mcm}{\mathcal{M}}
\newcommand{\mcc}{\mathcal{C}}
\newcommand{\mcmn}{\mathcal{M}}
\newcommand{\mcmnr}{\mathcal{M}_n(\mtr)}
\newcommand{\mcmnk}{\mathcal{M}_n(\mtk)}
\newcommand{\mcsn}{\mathcal{S}_n}
\newcommand{\mcs}{\mathcal{S}}
\newcommand{\mcd}{\mathcal{D}}
\newcommand{\mcsns}{\mathcal{S}_n^{++}}
\newcommand{\glnk}{GL_n(\mtk)}
\newcommand{\mnr}{\mathcal{M}_n(\mtr)}
\newcommand{\veps}{\varepsilon}
\newcommand{\mcu}{\mathcal{U}}
\newcommand{\mcun}{\mcu_n}
\newcommand{\dis}{\displaystyle}
\newcommand{\croouv}{[\![}
\newcommand{\crofer}{]\!]}
\newcommand{\rab}{\mathcal{R}(a,b)}
\newcommand{\pss}[2]{\langle #1,#2\rangle}
 %Document 


\section*{Rappels sur les nombres complexes}

\subsection*{Définitions et formes usuelles}

Un \underline{nombre complexe} $z$ s'écrit sous la forme :
\[
z = a + ib \qquad (a, b \in \mathbb{R},\ i^2 = -1)
\]
où $a$ est la partie réelle $\Re(z)$ et $b$ la partie imaginaire $\Im(z)$.


\underline{Forme algébrique} : $z = a + ib$

\underline{Forme trigonométrique} : 
\[
z = r(\cos\theta + i\sin\theta)
\]
où $r = |z| = \sqrt{a^2 + b^2}$ est le \textbf{module} de $z$, et $\theta = \arg(z)$ est un \textbf{argument} de $z$ (défini à $2\pi$ près).

\underline{Forme exponentielle} (formule d'Euler) :
\[
z = r e^{i\theta}
\]
avec $e^{i\theta} = \cos\theta + i\sin\theta$.


\subsection*{Module, argument et conjugué}

\begin{itemize}
    \item \underline{Module} : $|z| = \sqrt{a^2 + b^2}$
    \item \underline{Argument} : $\theta = \arctan\left(\frac{b}{a}\right)$ (attention au quadrant)
    \item \underline{Conjugué} : $\overline{z} = a - ib$
\end{itemize}

\subsection*{Formule de Moivre}

Pour tout $n \in \mathbb{Z}$,
\[
\left(\cos\theta + i\sin\theta\right)^n = \cos(n\theta) + i\sin(n\theta)
\]
ou, sous forme exponentielle :
\[
\left(e^{i\theta}\right)^n = e^{in\theta}
\]

\subsection*{Racines $n$-ièmes de l'unité}

Les solutions de $z^n = 1$ sont :
\[
z_k = e^{i\frac{2\pi k}{n}},\quad k = 0, 1, \ldots, n-1
\]



\vspace{3em}

\subsection{Module et argument}

Écrire sous la forme $a+i b$, puis sous forme exponentielle les nombres complexes suivants :
\begin{enumerate}
\item Nombre de module 2 et d'argument $\pi / 3$.
\item Nombre de module 3 et d'argument $-\pi / 8$.
\item Nombre de module 1 et d'argument $\pi / 4$.
\item Nombre de module 2 et d'argument $-\pi / 6$.
\item Nombre de module 7 et d'argument $-\pi / 2$.
\end{enumerate}

\vspace{2em}

\subsection{Forme exponentielle $\rightarrow$ forme algébrique}

Écrire sous la forme $a+ib$ les nombres complexes suivants, donnés sous forme exponentielle :
\begin{multicols}{2}
\begin{enumerate}
    \item $z_1 = 5 e^{i \frac{\pi}{6}}$
    \item $z_2 = 2 e^{-i \frac{\pi}{4}}$
    \item $z_3 = 3 e^{i \frac{2\pi}{3}}$
    \item $z_4 = 7 e^{i \pi}$
    \item $z_5 = 4 e^{i 0}$
    \item $z_6 = 6 e^{-i \frac{\pi}{2}}$
\end{enumerate}
\end{multicols}

\vspace{2em}

\subsection{Forme exponentielle}
Mettre sous forme exponentielle les nombres complexes suivants : 
\begin{multicols}{3}
\begin{enumerate}
    \item $z_1=1+i \sqrt{3}$, 
    \item $z_2=1+i$, 
    \item $z_3=-2 \sqrt{3}+2 i$, 
    \item $z_4=i$, 
    \item $z_5=-2 i$, 
    \item $z_6=-3$,
    \item $z_7=1$
    \item $z_8=9 i$
    \item $z_9=0$
    \item $z_{10}=\frac{-i \sqrt{2}}{1+i}$
    \item $z_{11}=\frac{(1+i \sqrt{3})^3}{(1-i)^5}$
    \item $z_{12}=\sin x+i \cos x$.
\end{enumerate}
\end{multicols}



\vspace{2em}

\subsection{Exponentielle}
Résoudre l'équation $e^z=3 \sqrt{3}-3 i$.

\newpage

\subsection{Trigonométrique}
En utilisant les nombres complexes, calculer  $\cos 5 \theta$ et $\sin 5 \theta$ en fonction de $\cos \theta$ et $\sin \theta$.

\vspace{2em}


\subsection{Pour préparer les séries de fourier}
Calculer les intégrales suivantes, pour toute valeur de $n$ et $m$ dans les entiers relatifs:

\begin{enumerate}
    \item $$\int_0^\pi e^{i n x} e^{i m x} dx$$
    \item $$\int_0^\pi \cos(n x) \cos(m x) dx$$
    \item $$\int_0^\pi \sin(n x) \sin(m x) dx$$
    \item $$\int_0^\pi \cos(n x) \sin(m x) dx$$
\end{enumerate}



\vspace{2em}

\subsection{Exponentielle}
On pose 
$$z_1=4 e^{i \frac{\pi}{4}}, \qquad z_2=3 i e^{i \frac{\pi}{6}}, \qquad z_3=-2 e^{i \frac{2 \pi}{3}}$$
Écrire sous forme exponentielle les nombres complexes : 
$$z_1,\qquad z_2,\qquad z_3 , \qquad z_1 z_2, \qquad \frac{z_1 z_2}{z_2}$$



\vspace{2em}

\subsection{Racines carrées}
Calculer de deux façons les racines carrées de $1+i$ et en déduire les valeurs exactes de $\cos \left(\frac{\pi}{8}\right)$ et $\sin \left(\frac{\pi}{8}\right)$.

% \setcounter{chapter}{3}
% \chapter{Des produits scalaires aux coefficients de Fourier}
% 
\newcommand{\mtn}{\mathbb{N}}
\newcommand{\mtns}{\mathbb{N}^*}
\newcommand{\mtz}{\mathbb{Z}}
\newcommand{\mtr}{\mathbb{R}}
\newcommand{\mtk}{\mathbb{K}}
\newcommand{\mtq}{\mathbb{Q}}
\newcommand{\mtc}{\mathbb{C}}
\newcommand{\mch}{\mathcal{H}}
\newcommand{\mcp}{\mathcal{P}}
\newcommand{\mcb}{\mathcal{B}}
\newcommand{\mcl}{\mathcal{L}}
\newcommand{\mcm}{\mathcal{M}}
\newcommand{\mcc}{\mathcal{C}}
\newcommand{\mcmn}{\mathcal{M}}
\newcommand{\mcmnr}{\mathcal{M}_n(\mtr)}
\newcommand{\mcmnk}{\mathcal{M}_n(\mtk)}
\newcommand{\mcsn}{\mathcal{S}_n}
\newcommand{\mcs}{\mathcal{S}}
\newcommand{\mcd}{\mathcal{D}}
\newcommand{\mcsns}{\mathcal{S}_n^{++}}
\newcommand{\glnk}{GL_n(\mtk)}
\newcommand{\mnr}{\mathcal{M}_n(\mtr)}
\newcommand{\veps}{\varepsilon}
\newcommand{\mcu}{\mathcal{U}}
\newcommand{\mcun}{\mcu_n}
\newcommand{\dis}{\displaystyle}
\newcommand{\croouv}{[\![}
\newcommand{\crofer}{]\!]}
\newcommand{\rab}{\mathcal{R}(a,b)}
\newcommand{\pss}[2]{\langle #1,#2\rangle}
 %Document 


\section*{Rappels sur les nombres complexes}

\subsection*{Définitions et formes usuelles}

Un \underline{nombre complexe} $z$ s'écrit sous la forme :
\[
z = a + ib \qquad (a, b \in \mathbb{R},\ i^2 = -1)
\]
où $a$ est la partie réelle $\Re(z)$ et $b$ la partie imaginaire $\Im(z)$.


\underline{Forme algébrique} : $z = a + ib$

\underline{Forme trigonométrique} : 
\[
z = r(\cos\theta + i\sin\theta)
\]
où $r = |z| = \sqrt{a^2 + b^2}$ est le \textbf{module} de $z$, et $\theta = \arg(z)$ est un \textbf{argument} de $z$ (défini à $2\pi$ près).

\underline{Forme exponentielle} (formule d'Euler) :
\[
z = r e^{i\theta}
\]
avec $e^{i\theta} = \cos\theta + i\sin\theta$.


\subsection*{Module, argument et conjugué}

\begin{itemize}
    \item \underline{Module} : $|z| = \sqrt{a^2 + b^2}$
    \item \underline{Argument} : $\theta = \arctan\left(\frac{b}{a}\right)$ (attention au quadrant)
    \item \underline{Conjugué} : $\overline{z} = a - ib$
\end{itemize}

\subsection*{Formule de Moivre}

Pour tout $n \in \mathbb{Z}$,
\[
\left(\cos\theta + i\sin\theta\right)^n = \cos(n\theta) + i\sin(n\theta)
\]
ou, sous forme exponentielle :
\[
\left(e^{i\theta}\right)^n = e^{in\theta}
\]

\subsection*{Racines $n$-ièmes de l'unité}

Les solutions de $z^n = 1$ sont :
\[
z_k = e^{i\frac{2\pi k}{n}},\quad k = 0, 1, \ldots, n-1
\]



\vspace{3em}

\subsection{Module et argument}

Écrire sous la forme $a+i b$, puis sous forme exponentielle les nombres complexes suivants :
\begin{enumerate}
\item Nombre de module 2 et d'argument $\pi / 3$.
\item Nombre de module 3 et d'argument $-\pi / 8$.
\item Nombre de module 1 et d'argument $\pi / 4$.
\item Nombre de module 2 et d'argument $-\pi / 6$.
\item Nombre de module 7 et d'argument $-\pi / 2$.
\end{enumerate}

\vspace{2em}

\subsection{Forme exponentielle $\rightarrow$ forme algébrique}

Écrire sous la forme $a+ib$ les nombres complexes suivants, donnés sous forme exponentielle :
\begin{multicols}{2}
\begin{enumerate}
    \item $z_1 = 5 e^{i \frac{\pi}{6}}$
    \item $z_2 = 2 e^{-i \frac{\pi}{4}}$
    \item $z_3 = 3 e^{i \frac{2\pi}{3}}$
    \item $z_4 = 7 e^{i \pi}$
    \item $z_5 = 4 e^{i 0}$
    \item $z_6 = 6 e^{-i \frac{\pi}{2}}$
\end{enumerate}
\end{multicols}

\vspace{2em}

\subsection{Forme exponentielle}
Mettre sous forme exponentielle les nombres complexes suivants : 
\begin{multicols}{3}
\begin{enumerate}
    \item $z_1=1+i \sqrt{3}$, 
    \item $z_2=1+i$, 
    \item $z_3=-2 \sqrt{3}+2 i$, 
    \item $z_4=i$, 
    \item $z_5=-2 i$, 
    \item $z_6=-3$,
    \item $z_7=1$
    \item $z_8=9 i$
    \item $z_9=0$
    \item $z_{10}=\frac{-i \sqrt{2}}{1+i}$
    \item $z_{11}=\frac{(1+i \sqrt{3})^3}{(1-i)^5}$
    \item $z_{12}=\sin x+i \cos x$.
\end{enumerate}
\end{multicols}



\vspace{2em}

\subsection{Exponentielle}
Résoudre l'équation $e^z=3 \sqrt{3}-3 i$.

\newpage

\subsection{Trigonométrique}
En utilisant les nombres complexes, calculer  $\cos 5 \theta$ et $\sin 5 \theta$ en fonction de $\cos \theta$ et $\sin \theta$.

\vspace{2em}


\subsection{Pour préparer les séries de fourier}
Calculer les intégrales suivantes, pour toute valeur de $n$ et $m$ dans les entiers relatifs:

\begin{enumerate}
    \item $$\int_0^\pi e^{i n x} e^{i m x} dx$$
    \item $$\int_0^\pi \cos(n x) \cos(m x) dx$$
    \item $$\int_0^\pi \sin(n x) \sin(m x) dx$$
    \item $$\int_0^\pi \cos(n x) \sin(m x) dx$$
\end{enumerate}



\vspace{2em}

\subsection{Exponentielle}
On pose 
$$z_1=4 e^{i \frac{\pi}{4}}, \qquad z_2=3 i e^{i \frac{\pi}{6}}, \qquad z_3=-2 e^{i \frac{2 \pi}{3}}$$
Écrire sous forme exponentielle les nombres complexes : 
$$z_1,\qquad z_2,\qquad z_3 , \qquad z_1 z_2, \qquad \frac{z_1 z_2}{z_2}$$



\vspace{2em}

\subsection{Racines carrées}
Calculer de deux façons les racines carrées de $1+i$ et en déduire les valeurs exactes de $\cos \left(\frac{\pi}{8}\right)$ et $\sin \left(\frac{\pi}{8}\right)$.

% \setcounter{chapter}{4}
% \chapter{Produit scalaire - exercices intermédiaires}
% 
\newcommand{\mtn}{\mathbb{N}}
\newcommand{\mtns}{\mathbb{N}^*}
\newcommand{\mtz}{\mathbb{Z}}
\newcommand{\mtr}{\mathbb{R}}
\newcommand{\mtk}{\mathbb{K}}
\newcommand{\mtq}{\mathbb{Q}}
\newcommand{\mtc}{\mathbb{C}}
\newcommand{\mch}{\mathcal{H}}
\newcommand{\mcp}{\mathcal{P}}
\newcommand{\mcb}{\mathcal{B}}
\newcommand{\mcl}{\mathcal{L}}
\newcommand{\mcm}{\mathcal{M}}
\newcommand{\mcc}{\mathcal{C}}
\newcommand{\mcmn}{\mathcal{M}}
\newcommand{\mcmnr}{\mathcal{M}_n(\mtr)}
\newcommand{\mcmnk}{\mathcal{M}_n(\mtk)}
\newcommand{\mcsn}{\mathcal{S}_n}
\newcommand{\mcs}{\mathcal{S}}
\newcommand{\mcd}{\mathcal{D}}
\newcommand{\mcsns}{\mathcal{S}_n^{++}}
\newcommand{\glnk}{GL_n(\mtk)}
\newcommand{\mnr}{\mathcal{M}_n(\mtr)}
\newcommand{\veps}{\varepsilon}
\newcommand{\mcu}{\mathcal{U}}
\newcommand{\mcun}{\mcu_n}
\newcommand{\dis}{\displaystyle}
\newcommand{\croouv}{[\![}
\newcommand{\crofer}{]\!]}
\newcommand{\rab}{\mathcal{R}(a,b)}
\newcommand{\pss}[2]{\langle #1,#2\rangle}
 %Document 


\section*{Rappels sur les nombres complexes}

\subsection*{Définitions et formes usuelles}

Un \underline{nombre complexe} $z$ s'écrit sous la forme :
\[
z = a + ib \qquad (a, b \in \mathbb{R},\ i^2 = -1)
\]
où $a$ est la partie réelle $\Re(z)$ et $b$ la partie imaginaire $\Im(z)$.


\underline{Forme algébrique} : $z = a + ib$

\underline{Forme trigonométrique} : 
\[
z = r(\cos\theta + i\sin\theta)
\]
où $r = |z| = \sqrt{a^2 + b^2}$ est le \textbf{module} de $z$, et $\theta = \arg(z)$ est un \textbf{argument} de $z$ (défini à $2\pi$ près).

\underline{Forme exponentielle} (formule d'Euler) :
\[
z = r e^{i\theta}
\]
avec $e^{i\theta} = \cos\theta + i\sin\theta$.


\subsection*{Module, argument et conjugué}

\begin{itemize}
    \item \underline{Module} : $|z| = \sqrt{a^2 + b^2}$
    \item \underline{Argument} : $\theta = \arctan\left(\frac{b}{a}\right)$ (attention au quadrant)
    \item \underline{Conjugué} : $\overline{z} = a - ib$
\end{itemize}

\subsection*{Formule de Moivre}

Pour tout $n \in \mathbb{Z}$,
\[
\left(\cos\theta + i\sin\theta\right)^n = \cos(n\theta) + i\sin(n\theta)
\]
ou, sous forme exponentielle :
\[
\left(e^{i\theta}\right)^n = e^{in\theta}
\]

\subsection*{Racines $n$-ièmes de l'unité}

Les solutions de $z^n = 1$ sont :
\[
z_k = e^{i\frac{2\pi k}{n}},\quad k = 0, 1, \ldots, n-1
\]



\vspace{3em}

\subsection{Module et argument}

Écrire sous la forme $a+i b$, puis sous forme exponentielle les nombres complexes suivants :
\begin{enumerate}
\item Nombre de module 2 et d'argument $\pi / 3$.
\item Nombre de module 3 et d'argument $-\pi / 8$.
\item Nombre de module 1 et d'argument $\pi / 4$.
\item Nombre de module 2 et d'argument $-\pi / 6$.
\item Nombre de module 7 et d'argument $-\pi / 2$.
\end{enumerate}

\vspace{2em}

\subsection{Forme exponentielle $\rightarrow$ forme algébrique}

Écrire sous la forme $a+ib$ les nombres complexes suivants, donnés sous forme exponentielle :
\begin{multicols}{2}
\begin{enumerate}
    \item $z_1 = 5 e^{i \frac{\pi}{6}}$
    \item $z_2 = 2 e^{-i \frac{\pi}{4}}$
    \item $z_3 = 3 e^{i \frac{2\pi}{3}}$
    \item $z_4 = 7 e^{i \pi}$
    \item $z_5 = 4 e^{i 0}$
    \item $z_6 = 6 e^{-i \frac{\pi}{2}}$
\end{enumerate}
\end{multicols}

\vspace{2em}

\subsection{Forme exponentielle}
Mettre sous forme exponentielle les nombres complexes suivants : 
\begin{multicols}{3}
\begin{enumerate}
    \item $z_1=1+i \sqrt{3}$, 
    \item $z_2=1+i$, 
    \item $z_3=-2 \sqrt{3}+2 i$, 
    \item $z_4=i$, 
    \item $z_5=-2 i$, 
    \item $z_6=-3$,
    \item $z_7=1$
    \item $z_8=9 i$
    \item $z_9=0$
    \item $z_{10}=\frac{-i \sqrt{2}}{1+i}$
    \item $z_{11}=\frac{(1+i \sqrt{3})^3}{(1-i)^5}$
    \item $z_{12}=\sin x+i \cos x$.
\end{enumerate}
\end{multicols}



\vspace{2em}

\subsection{Exponentielle}
Résoudre l'équation $e^z=3 \sqrt{3}-3 i$.

\newpage

\subsection{Trigonométrique}
En utilisant les nombres complexes, calculer  $\cos 5 \theta$ et $\sin 5 \theta$ en fonction de $\cos \theta$ et $\sin \theta$.

\vspace{2em}


\subsection{Pour préparer les séries de fourier}
Calculer les intégrales suivantes, pour toute valeur de $n$ et $m$ dans les entiers relatifs:

\begin{enumerate}
    \item $$\int_0^\pi e^{i n x} e^{i m x} dx$$
    \item $$\int_0^\pi \cos(n x) \cos(m x) dx$$
    \item $$\int_0^\pi \sin(n x) \sin(m x) dx$$
    \item $$\int_0^\pi \cos(n x) \sin(m x) dx$$
\end{enumerate}



\vspace{2em}

\subsection{Exponentielle}
On pose 
$$z_1=4 e^{i \frac{\pi}{4}}, \qquad z_2=3 i e^{i \frac{\pi}{6}}, \qquad z_3=-2 e^{i \frac{2 \pi}{3}}$$
Écrire sous forme exponentielle les nombres complexes : 
$$z_1,\qquad z_2,\qquad z_3 , \qquad z_1 z_2, \qquad \frac{z_1 z_2}{z_2}$$



\vspace{2em}

\subsection{Racines carrées}
Calculer de deux façons les racines carrées de $1+i$ et en déduire les valeurs exactes de $\cos \left(\frac{\pi}{8}\right)$ et $\sin \left(\frac{\pi}{8}\right)$.

\setcounter{chapter}{5}
\chapter{Contrôle continu 2}
\documentclass[11pt,a4paper]{report}

% -------------------- Encodage & langue --------------------
\usepackage[T1]{fontenc}
\usepackage[utf8]{inputenc}
\usepackage[french]{babel}
\usepackage{lmodern}
\usepackage{microtype}
\usepackage{amsmath, amssymb}
\usepackage{multicol}
\usepackage{enumitem}

\usepackage{amsfonts}
\usepackage[version=4]{mhchem}
\usepackage{stmaryrd}
\usepackage{graphicx}
\usepackage[export]{adjustbox}
\graphicspath{ {./images/} }
\usepackage{caption}
\usepackage{multirow}

% -------------------- Mise en page --------------------------
\usepackage[a4paper,margin=2cm]{geometry}
\usepackage{fancyhdr}
\usepackage{parskip}      % espace entre paragraphes
\setlength{\parindent}{0pt}

% -------------------- Couleurs & liens ----------------------
\usepackage{xcolor}
\definecolor{Theme}{HTML}{0E7490} % teal-700
\definecolor{ThemeLight}{HTML}{E0F2F1}
\definecolor{Accent}{HTML}{F59E0B} % amber-500
\definecolor{Gray}{HTML}{374151}
\usepackage[colorlinks=true,linkcolor=Theme,urlcolor=Theme,citecolor=Theme]{hyperref}

% -------------------- Graphiques / décor --------------------
\usepackage{tikz}
\usetikzlibrary{patterns,positioning,calc}
\usepackage{graphicx}
\usepackage{tcolorbox}
\tcbuselibrary{skins,breakable,hooks,most}
\usepackage{fontawesome5}

% -------------------- Titres -------------------------------
\usepackage{titlesec}
\titleformat{\chapter}[display]
  {\Huge\bfseries\color{Theme}}
  {\filright\rule{0.75\linewidth}{1.2pt}\\[3pt]{Algèbre linéaire - Chapitre~\thechapter}}
  {0.2ex}
  {\filright}
  [\vspace{0.1ex}\rule{0.35\linewidth}{1.2pt}]

\titleformat{\section}
  {\Large\bfseries\color{Gray}}
  {\thesection}{0.6em}{}

% -------------------- En-têtes / pieds ---------------------
\pagestyle{fancy}
\fancyhf{}
\fancyhead[L]{\color{Gray}\leftmark}
\fancyhead[R]{\color{Gray}\textit{MEF - 2025/2026}}
\fancyfoot[R]{\color{Gray}\small p.\ \thepage}
\renewcommand{\headrulewidth}{0pt}
\renewcommand{\footrulewidth}{0pt}

% -------------------- Macros utilitaires -------------------
\newenvironment{solution}
{
    \vspace{0.5em}
    \begin{mdframed}[backgroundcolor=ThemeLight,leftmargin=0,rightmargin=0,skipabove=0.2em,skipbelow=0.2em]
    \textbf{Solution.}\\[0.5em]
}
{
    \end{mdframed}
    \vspace{0.5em}
}


% Tcolorboxes stylisées
\tcbset{tracebox/.style={breakable,enhanced,sharp corners,boxrule=0pt,frame hidden,arc=2mm,
  colback=white,coltitle=black,fonttitle=\bfseries\large,
  borderline west={2mm}{0pt}{Theme},
  before skip=8pt,after skip=8pt,drop fuzzy shadow}}

\newtcolorbox{resumeBox}{tracebox,title={\faStickyNote\quad Résumé des idées}}
\newtcolorbox{rappelsBox}{tracebox,title={\faRedo\quad Ce que je dois savoir }}
\newtcolorbox{exempleBox}{tracebox,title={\faChalkboardTeacher\quad Exemple}}

% Encadré « Formules & illustrations »
\newtcolorbox{formulesBox}{tracebox,title={\faCalculator\quad Formules \& illustrations},colback=ThemeLight}

% Astuce : puces clean
\newenvironment{niceitemize}{\begin{itemize}\setlength{\itemsep}{0.25em}\color{Gray}}{\end{itemize}}

% Raccourci pour une « Trace » complète
% Usage : \TraceSection{Titre}{Objectif court}
\newcommand{\TraceSection}[2]{%
  
}

% -------------------- Page de titre ------------------------
\title{\textbf{Traces de cours}\\\large (résumés, formules, exemples, mini-exercices)}
\author{ MEF - 2025/2026 }
\date{\today}


\makeatletter
\renewcommand{\thesubsection}{\arabic{subsection}}
\renewcommand{\p@subsection}{}% supprime le préfixe section/chapter dans \ref
% Si vous voulez la même chose pour les sous-sous-sections :
% \renewcommand{\thesubsubsection}{\arabic{subsubsection}}
% \renewcommand{\p@subsubsection}{}
\makeatother

\usepackage{mdframed}
\usepackage{ifthen}

% \usepackage[sf]{titlesec}
% Définition de la variable pour afficher les corrections
\newboolean{showSolutions}
% Décommentez la ligne suivante pour afficher les solutions
\input \jobname.adr
% -------------------- Document ----------------------------
\begin{document}

\begin{center}
    {\LARGE \textbf{Méthode des éléments finis -- TD2}}\\[1em]
    {\large \textit{Avec des équations différentielles partielles}}
\end{center}
\section*{Contexte}
La méthode des éléments finis est une technique numérique puissante utilisée pour résoudre des équations différentielles partielles. Nous allons étudier un problème aux limites elliptique en dimension 1 : le problème de Dirichlet. 

Voici les étapes de la méthode quand on part d'une équation différentielle partielle :
\begin{enumerate}
    \item Découper le domaine en éléments simples
    \item Établir la formulation variationnelle de l'EDP
    \item Ecrire la matrice élémentaire sur chaque élément
    \item Assembler ces matrices pour obtenir la matrice de rigidité globale
    \item Résoudre le système linéaire
\end{enumerate}


Considérons l'équation différentielle suivante en dimension $1$ :
\[
- \frac{d}{dx} \left( p(x) \frac{du}{dx} \right) + q(x) u = f(x), \quad x \in (0, 1)
\]
où \( p(x) > 0 \), \( q(x) \geq 0 \), et \( f(x) \) est une fonction donnée.

\textbf{Conditions aux limites de Dirichlet: la valeur de l'inconnue est fixée au bord}
\[
u(0) = \alpha, \quad u(1) = \beta
\]

\subsection*{Exemple concret d'application : Isolation thermique d'un mur}

\begin{exempleBox}
\textbf{Contexte :} Imaginons un mur de bâtiment de 1 mètre d'épaisseur, composé de plusieurs matériaux (béton, isolant, plâtre). On souhaite déterminer la distribution de température $u(x)$ à travers le mur en régime stationnaire.

\vspace{0.3cm}

\textbf{Interprétation physique de l'équation :}

\begin{itemize}
    \item \textbf{$u(x)$} : température (en degré C) à la position $x$ dans le mur
    \item \textbf{$p(x)$} : conductivité thermique du matériau (en $W/(m \cdot K)$)
    \begin{itemize}
        \item Béton : $p \approx 1.5$ $W/(m \cdot K)$
        \item Isolant (laine de verre) : $p \approx 0.04$ $W/(m \cdot K)$
        \item Plâtre : $p \approx 0.5$ $W/(m \cdot K)$
    \end{itemize}
    \item \textbf{$q(x) u$} : pertes thermiques volumiques (par exemple dues à la convection interne) (en $W/(m^3 \cdot K)$)
    \item \textbf{$f(x)$} : sources de chaleur internes (par exemple, câbles électriques chauffants intégrés)
\end{itemize}

\vspace{0.3cm}


\textbf{Conditions aux limites :}
\begin{itemize}
    \item $u(0) = \alpha = 20$ degré C (température intérieure imposée par le chauffage)
    \item $u(1) = \beta = 5$ degré C (température extérieure en hiver)
\end{itemize}

\vspace{0.3cm}

\textbf{Cas simple :} Si $p$ et $q$ sont constants et $f = 0$ (pas de source interne), on retrouve l'équation classique de diffusion :
\[
-p u'' + q u = 0
\]

\textbf{Résultat attendu :} La méthode des éléments finis permet de calculer la température en tout point du mur, et notamment :
\begin{itemize}
    \item Localiser les zones de forte variation de température (ponts thermiques)
    \item Calculer le flux thermique total : $\Phi = -p \frac{du}{dx}$ (pour dimensionner le chauffage)
    \item Vérifier les risques de condensation (température du point de rosée)
\end{itemize}
\end{exempleBox}

\subsection*{Autres exemples d'applications}

L'équation $- \frac{d}{dx} \left( p(x) \frac{du}{dx} \right) + q(x) u = f(x)$ peut modéliser d'autres phénomènes physiques :

\begin{multicols}{2}
\begin{itemize}[label=]
    \item \textbf{Mécanique :} Déformation d'une barre élastique
    \begin{itemize}[label=$\bullet$]
        \item $u$ : déplacement
        \item $p$ : module d'Young
        \item $f$ : force appliquée
    \end{itemize}
    
    \item \textbf{Électrostatique :} Potentiel électrique
    \begin{itemize}[label=$\bullet$]
        \item $u$ : potentiel
        \item $p$ : permittivité
        \item $f$ : densité de charge
    \end{itemize}
    
    \item \textbf{Hydraulique :} Écoulement en milieu poreux
    \begin{itemize}[label=$\bullet$]
        \item $u$ : pression
        \item $p$ : perméabilité
        \item $f$ : source/puits
    \end{itemize}
    
    \item \textbf{Chimie :} Diffusion d'un polluant
    \begin{itemize}[label=$\bullet$]
        \item $u$ : concentration
        \item $p$ : coefficient de diffusion
        \item $q$ : taux de dégradation
    \end{itemize}
\end{itemize}
\end{multicols}

\section*{La formulation variationnelle}

\begin{rappelsBox}
La formulation variationnelle (ou formulation faible) est une reformulation d'une équation différentielle qui permet de :
\begin{itemize}
    \item Affaiblir les hypothèses de régularité (on passe de dérivées classiques à des dérivées au sens faible)
    \item Obtenir une forme adaptée à l'approximation numérique
    \item Garantir l'existence et l'unicité de la solution (théorème de Lax-Milgram)
\end{itemize}
\end{rappelsBox}

(Cette partie servira pour la question 2.)

Considérons une équation différentielle sous la forme :
\[
\mathcal{L}(u) = f \quad \text{dans } \Omega
\]
avec des conditions aux limites appropriées, où $\mathcal{L}$ est un opérateur différentiel.

\textbf{Étape 1 : Multiplier par une fonction test}

On multiplie l'équation par une fonction test $v$ (appelée aussi fonction poids) et on intègre sur le domaine :
\[
\int_\Omega \mathcal{L}(u) \, v \, dx = \int_\Omega f \, v \, dx
\]

\textbf{Étape 2 : Intégration par parties}

On utilise l'intégration par parties pour réduire l'ordre de dérivation. Cela permet de :
\begin{itemize}
    \item Répartir les dérivées entre $u$ et $v$ (formulation symétrique)
    \item Faire apparaître les conditions aux limites naturellement
    \item Réduire les exigences de régularité sur $u$
\end{itemize}

\textbf{Étape 3 : Formulation finale}

On obtient une formulation du type :
\[
a(u, v) = L(v) \quad \forall v \in V_0
\]
où :
\begin{itemize}
    \item $a(\cdot, \cdot)$ est une forme bilinéaire (linéaire en chaque argument)
    \item $L(\cdot)$ est une forme linéaire
    \item $V_0$ est l'espace des fonctions test (qui satisfont les conditions aux limites homogènes)
\end{itemize}

\subsection*{Exemple avec notre problème}

Pour l'équation sur un segment $[0,h]$:
\[
- \frac{d}{dx} \left( p(x) \frac{du}{dx} \right) + q(x) u = f(x)
\]

\textbf{Étape 1 :} Multipliez par $v$ et intégrez sur $[0,h]$ :
\ifthenelse{\boolean{showSolutions}}{
    \begin{solution}
    \[
    \int_0^h \left[ - \frac{d}{dx} \left( p \frac{du}{dx} \right) + q u \right] v \, dx = \int_0^h f v \, dx
    \]
    \end{solution}
    }{}

\textbf{Étape 2 :} Intégrez par parties sur le terme dérivé, on choisit la fonction test $v$ pour être nulle au bord ($v(0) = v(h) = 0$) :
\ifthenelse{\boolean{showSolutions}}{
    \begin{solution}
    \[
    \int_0^h \left( - \frac{d}{dx} \left( p \frac{du}{dx} \right) \right) v \, dx = \left[ -p \frac{du}{dx} v \right]_0^h + \int_0^h p \frac{du}{dx} \frac{dv}{dx} \, dx
    \]
    

    Si $v(0) = v(h) = 0$ (conditions homogènes), le terme de bord s'annule :

    \[
    \int_0^h p \frac{du}{dx} \frac{dv}{dx} \, dx + \int_0^h q u v \, dx = \int_0^h f v \, dx
    \]
\end{solution}
    }{}

\textbf{Étape 3 :} Formulation variationnelle finale :

\ifthenelse{\boolean{showSolutions}}{
    \begin{solution}
        L'équation différentielle devient :
    \[
    a(u, v) = L(v)
    \]
    avec $a(u, v) = \int_0^h p \frac{du}{dx} \frac{dv}{dx} \, dx + \int_0^h q u v \, dx$ et $L(v) = \int_0^h f v \, dx$.
    \end{solution}
    }{}

Résoudre l'équation différentielle revient à trouver une fonction $u$ telle que
\[
a(u, v) = L(v) \quad \forall v \in V_0
\]
avec $a(u, v)$ une fonction bilinéaire symétrique et $L(v)$ une fonction linéaire.

C'est cette équation qui donnera le système linéaire $Ku = F$ à résoudre pour trouver $u$.


\subsection*{Pourquoi cette formulation ?}

\begin{itemize}
    \item \textbf{Régularité} : On n'a besoin que de $u' $ et $v'$ (pas de $u''$)
    \item \textbf{Symétrie} : $a(u,v)$ est symétrique : $a(u,v) = a(v,u)$
    \item \textbf{Coercivité} : Sous certaines conditions, $a(u,u) \geq \alpha \|u\|^2$ garantit l'existence et l'unicité (Théorème de Lax-Milgram)
    \item \textbf{Approximation} : Cette forme se prête bien à la discrétisation par éléments finis
\end{itemize}

\section*{Questions}

\begin{enumerate}[itemsep=0.5em]
\item \textbf{Maillage}
\begin{enumerate}
    \item Définissez un maillage avec $4$ éléments $(0=x_0,\,x_1,\,x_2,\,x_3,\,x_4=1)$, chacun de longueur $h$.
    
    \ifthenelse{\boolean{showSolutions}}{
        
    \begin{solution}
    
    Divisons l'intervalle \([0,1]\) en 4 éléments égaux. La longueur de chaque élément est \( h = \frac{1}{4}\).

    Les noeuds sont :
    \[
    x_0 = 0, \quad x_1 = h, \quad x_2 = 2h, \quad x_3 = 3h, \quad x_4 = 4h = 1
    \]
    \end{solution}
    }{}
\end{enumerate}

     \item \textbf{Formulation Variationnelle}
\begin{enumerate}
    \item Sur le premier élément $[0,h]$, formulez le problème de Dirichlet sous sa forme variationnelle. On pourra choisir $V_0 = \{ v \in H^1(0,h) \ | \ v(0) = v(h) = 0 \}$ comme espace de fonctions test.
    
    \ifthenelse{\boolean{showSolutions}}{
    \begin{solution}

        En suivant les 3 étapes décrites ci-dessus, on obtient :
    \[
    a(u, v) = L(v) \quad \forall v \in V_0
    \]
    avec 
    \[
    a(u, v) = \int_0^h p \frac{du}{dx} \frac{dv}{dx} \, + \int_0^h q u v 
    \]
    \[
    L(v) = \int_0^h f v
    \]
    \end{solution}
    }{}

    \item Sur le second élément du maillage, comment la formulation variationnelle est-elle modifiée ? 
 \ifthenelse{\boolean{showSolutions}}{
    \begin{solution}
    Il faut changer les bornes d'intégration, ce sont les mêmes intégrales, mais on intègre sur $[h,2h]$.
    \end{solution}
    }{}
\end{enumerate}


\item \textbf{Approximation dans un Espace de Dimension Finie}
\begin{enumerate}
    \item La méthode des éléments finis nous donnera une solution seulement sur les noeuds du maillage. Nous relierons ensuite ces points par des segments de droite. 

    Déterminez les expressions des fonctions \( \varphi_i \), affines par morceaux et telles que \( \varphi_i(x_j) = \delta_{ij} \). 
    
    \vspace{2em}
    Pour obtenir une solution approchée de notre équation différentielle, nous chercherons alors une fonction $u_h$ telle que
    \[
    u_h = \sum_{i=0}^{4} \alpha_i \varphi_i
    \]
    où les $ \alpha_i $ sont des coefficients inconnus à déterminer. $u_h$ vérifiera les conditions au bord.
    
\end{enumerate}

    \ifthenelse{\boolean{showSolutions}}{
    \begin{solution}

    \begin{enumerate}
        \item Les fonctions \( \varphi_i \) sont les fonctions affines par morceaux définies comme suit :
        \[
        \varphi_i(x_j) = \delta_{ij}, \quad i,j = 0,1,2,3,4
        \]
        Chaque \( \varphi_i \) est égale à 1 au noeud \( x_i \) et à 0 aux autres noeuds.
            % Start Generation Here
            \[
            \varphi_0(x) =
            \begin{cases}
                1 - \dfrac{x}{h} & \text{si } 0 \leq x \leq x_1, \\
                0 & \text{sinon}.
            \end{cases}
            \]
            \[
            \varphi_1(x) =
            \begin{cases}
                \dfrac{x}{h} & \text{si } x_0 \leq x \leq x_1, \\
                1 - \dfrac{x - x_1}{h} & \text{si } x_1 \leq x \leq x_2, \\
                0 & \text{sinon}.
            \end{cases}
            \]
            \[
            \varphi_2(x) =
            \begin{cases}
                \dfrac{x - x_1}{h} & \text{si } x_1 \leq x \leq x_2, \\
                1 - \dfrac{x - x_2}{h} & \text{si } x_2 \leq x \leq x_3, \\
                0 & \text{sinon}.
            \end{cases}
            \]
            \[
            \varphi_3(x) =
            \begin{cases}
                \dfrac{x - x_2}{h} & \text{si } x_2 \leq x \leq x_3, \\
                1 - \dfrac{x - x_3}{h} & \text{si } x_3 \leq x \leq x_4, \\
                0 & \text{sinon}.
            \end{cases}
            \]
            \[
            \varphi_4(x) =
            \begin{cases}
                \dfrac{x - x_3}{h} & \text{si } x_3 \leq x \leq x_4, \\
                0 & \text{sinon}.
            \end{cases}
            \]
    \end{enumerate}
    \end{solution}
    }{}
\item \textbf{Obtention du Système Linéaire}
La matrice de rigidité élémentaire $K_e$ associée au problème de Dirichlet sur l'élément $[x_i,x_{i+1}]$ est
    
\[
K_e = \begin{pmatrix}
    a(\varphi_i, \varphi_i) & a(\varphi_i, \varphi_{i+1}) \\
    a(\varphi_{i+1}, \varphi_i) & a(\varphi_{i+1}, \varphi_{i+1})
\end{pmatrix}
\]
On supposera que $p(x)=p_0$ et $q(x)=q_0$ sont constants sur cet élément.\newline

\begin{enumerate}

    \item Montrer que si $u = \sum_{i=0}^{4} \alpha_i \varphi_i$, alors $a(u, v) = \sum_{i=0}^{4} \alpha_i a(\varphi_i, v)$. En déduire que
    \[
        u \text{ vérifie } a(u, v) = L(v) sur [x_i,x_{i+1}] \Leftrightarrow \, K_e \begin{pmatrix} \alpha_i \\ \alpha_{i+1} \end{pmatrix} = \begin{pmatrix} L(\varphi_i) \\ L(\varphi_{i+1}) \end{pmatrix}
    \]
    On pourra le vérifier pour $v = \varphi_i$ et $v = \varphi_{i+1}$.
    
    \item Ecrivez la matrice de rigidité du premier élément $[0,h]$.
    
    \item Sans vous lancer dans de longs calculs, explicitez la matrice de rigidité sur chacun des autres éléments. Nous continuerons à supposer que les fonctions $p$ et $q$ sont constantes.

    \item Etablissez la matrice de rigidité globale \( K \) et décrivez ses propriétés (par exemple: forme particulière, symétrie, définie positive). 

    \item En supposant maintenant que $f$ est constante sur chaque élément, explicitez le second membre du système linéaire.

\end{enumerate}

    \ifthenelse{\boolean{showSolutions}}{
    \begin{solution}

    \begin{enumerate}
            % Start of Selection


            \item L'équation se déduit par linéarité de $a$ et $L$, en effectuant la multiplication matricielle et en prenant pour fonction $v$ les fonctions   $\varphi_i$ et $\varphi_{i+1}$. 

            \item La matrice de rigidité locale \( K_e \) sur le premier élément est définie par :
            \[
            K = \left(
                \begin{array}{cc}
                    a(\varphi_0, \varphi_0) & a(\varphi_0, \varphi_1) \\
                    a(\varphi_1, \varphi_0) & a(\varphi_1, \varphi_1)
                \end{array}
            \right)
            \]
            Il faut donc calculer ces $4$ termes. En fait seulement $3$ puisque $a(\varphi_0, \varphi_1) = a(\varphi_1, \varphi_0)$, $a$ est symétrique.

            On peut commencer par calculer les dérivées des fonctions de base : sur l'intervalle $[0,h]$,
            \[
            \frac{d\varphi_0}{dx} = -\frac{1}{h}, \quad \frac{d\varphi_1}{dx} = \frac{1}{h}
            \]
            et les autres dérivées sont nulles.

            \[
            K_{11} = a\big(\varphi_0, \varphi_0\big) = \int_{0}^{h} p_0 \frac{d\varphi_0}{dx} \frac{d\varphi_0}{dx} \, dx + \int_{0}^{h} q_0 \varphi_0 \varphi_0 \, dx
            \]
            En remplaçant les fonctions de base par leur expression, on obtient :
            \[
            K_{11} = \int_{0}^{h} p_0 \left( -\frac{1}{h} \right)^2 dx + \int_{0}^{h} q_0 \varphi_0^2 dx
            \]
            C'est à dire :
            \[
            K_{11} = \frac{p_0}{h} + q_0 \int_{0}^{h} \Big(1 - \frac{x}{h}\Big)^2 dx 
            \]  
            Or 
            \[\int_{0}^{h} \Big(1 - \frac{x}{h}\Big)^2 dx = \Big[\frac{-h}{3}\Big(1 - \frac{x}{h}\Big)^3\Big]_{0}^{h} = \frac{h}{3}.\]

            Finalement :
            \[
            K_{11} = \frac{p_0}{h} + q_0 \frac{h}{3}
            \]
            
            Pour le second terme, on a :

            \[
            K_{12} = K_{21} = \int_{0}^{h} p_0 \left( -\frac{1}{h} \right) \left( \frac{1}{h} \right) dx + q_0 \int_{0}^{h} \Big(1 - \frac{x}{h}\Big) \Big(\frac{x}{h}\Big) dx
            \]
            Ainsi, 
            \[
            K_{12} = K_{21} = -\frac{p_0}{h} + q_0 \int_{0}^{h} \frac{x}{h} \Big(1 - \frac{x}{h}\Big)  dx
            \]
            Or 
            \[\int_{0}^{h} \frac{x}{h}\Big(1 - \frac{x}{h}\Big)  dx = \frac{1}{h}\Big[\frac{x^2}{2} - \frac{x^3}{3h}\Big]_{0}^{h} = \frac{h}{6}.\]

            Finalement :
            \[
            K_{12} = K_{21} = -\frac{p_0}{h} + q_0 \frac{h}{6}
            \]
            
            Pour le dernier coefficient de la matrice, on a :
            \[
            K_{22} = \int_{0}^{h} p_0 \left( \frac{1}{h} \right)^2 dx  + q_0 \int_{0}^{h} \Big(\frac{x}{h}\Big)^2 dx = \frac{p_0}{h} + q_0 \frac{h}{3}.
            \]
            
            En rassemblant les résultats, la matrice de rigidité sur le premier élément est:
            \[
            K = \frac{p_0}{h}
            \begin{pmatrix}
                1 & -1 \\
                -1 & 1
            \end{pmatrix}
            + q_0 \frac{h}{3}
            \begin{pmatrix}
                1 & 1/2 \\
                1/2 & 1
            \end{pmatrix}
            \]

            \item Les autres matrices de rigidité sont identiques aux constantes $p_0$ et $q_0$ près. Cela découle de la symétrie des fonctions $\varphi_i$, par exemple un changement de variable sur les intégrales $u=x+h$ conduit à :
            \[
            \int_0^h \left( \frac{d\varphi_0}{dx} \right)^2 dx = \int_h^{2h} \left( \frac{d\varphi_1}{dx} \right)^2 dx 
            \]
            On le voit bien aussi en traçant les graphes des fonctions $\varphi_i$.
            % Start of Selection
            \item En assemblant les matrices locales, on construit le système global $K \mathbf{u} = \mathbf{f}$. Ce processus implique de placer chaque matrice locale dans la matrice globale $K$ en fonction de la connectivité des nœuds de chaque élément. 
            Une fois tous les éléments assemblés, la matrice $K$ sera de taille $5 \times 5$ (pour 5 nœuds).

            Si les fonctions $p$ et $q$ étaient constantes sur tout le domaine égales à $p(x)=p_0$ et $q(x)=q_0$, on aurait

            $$
            K = \frac{p_0}{h}
            \begin{pmatrix}
                1 & -1 & 0 & 0 & 0\\
                -1 & 2 & -1 & 0 & 0\\
                0 & -1 & 2 & -1 & 0\\
                0 & 0 & -1 & 2 & -1\\
                0 & 0 & 0 & -1 & 1
            \end{pmatrix} 
            + q_0 \frac{h}{3}
            \begin{pmatrix}
                1 & 1/2 & 0 & 0 & 0\\
                1/2 & 2 & 1/2 & 0 & 0\\
                0 & 1/2 & 2  & 1/2 & 0\\
                0 & 0 & 1/2 & 2 & 1/2\\
                0 & 0 & 0 & 1/2 & 2
            \end{pmatrix}
            $$

        La matrice $K$ est symétrique donc diagonalisable d'après le théorème spectral. 
    
    \item Pour le second membre du système linéaire il faut également calculer les intégrales des $\varphi_i$:
    Pour le premier élément 
    \[
    \int_{0}^{h} f(x) \varphi_0(x) dx = f_0 \int_{0}^{h} \Big(1 - \frac{x}{h}\Big) dx = f_0\Big[\frac{-h}{2}\Big(1 - \frac{x}{h}\Big)^2\Big]_{0}^{h} = \frac{f_0h}{2}
    \]
    Par symétrie, on a de même
    \[
    \int_{0}^{h} f(x) \varphi_1(x) dx = \frac{f_0h}{2}
    \]

    On peut alors assembler un second membre global pour le système linéaire.
    \[
    \mathbf{f} = \frac{f_0h}{2}\begin{pmatrix}
        1 \\
        2 \\
        2 \\
        2 \\
        1
    \end{pmatrix}
    \]

    
    \end{enumerate}
    \end{solution}
    }{}
\item \textbf{Résolution Manuelle dans un Cas Simple}
\begin{enumerate}
    \item Considérez le cas où \( p(x) = 1 \), \( q(x) = 0 \), \( f(x) = 1 \) et dites à quelle équation différentielle du second ordre ces conditions correspondent.
    
    \item Les équations du système linéaire trouvées précédemment ne prennent pas du tout en compte les conditions aux limites. Ecrivez les $3$ équations sur $\alpha_1, \, \alpha_2, \, \alpha_3$ puis explicitez la solution approchée $u_h$. 
    
    \item Calculez la solution exacte de l'équation différentielle avec conditions aux limites.
    
    \item Comparez votre solution approchée avec la solution exacte et discutez de la précision de l'approximation à la fois aux noeuds du maillage et entre les noeuds. 
    
    \item Quelles méthodes proposez-vous pour améliorer l'approximation?
\end{enumerate}
\ifthenelse{\boolean{showSolutions}}{
    \begin{solution}

    \begin{enumerate}
        \item Avec \( p(x) = 1 \), \( q(x) = 0 \), \( f(x) = 1 \), l'équation devient :
        \[
        -u'' = 1
        \]
        C'est l'équation de Laplace. 

        \item Comme $\alpha_0 = u(0)$ et $\alpha_4 = u(1)$ sont connus, on doit retirer les équations correspondantes du système, cela correspond à la première ligne et la dernière ligne de la matrice $K$.
        
        On résout le système linéaire $K \mathbf{u} = \mathbf{f}$ pour obtenir les coefficients $\alpha_1, \alpha_2, \alpha_3$.

        \[
        \left\{
        \begin{array}{cccccc}
             \frac{p_0}{h}\Big(- \alpha & +  2 \alpha_1 &- \alpha_2 & & \Big)  &=  f_0 h \\
             \frac{p_0}{h}\Big( & - \alpha_1 &+  2 \alpha_2& - \alpha_3 &\Big)  &=  f_0 h \\
             \frac{p_0}{h}\Big(  & & - \alpha_2 &+  2 \alpha_3 &- \beta\Big)  &=  f_0 h \\
        \end{array}
        \right.
        \]
        La quantité $f_0 h^2/p_O$ apparaîtra souvent, nous la notons $\lambda$ dans la suite. Le système se réécrit:
        \[
            \left\{
                \begin{array}{cccccc}
                     - \alpha &+ 2 \alpha_1 &- \alpha_2 & & &=  \lambda \\
                     & - \alpha_1 &+ 2 \alpha_2 &- \alpha_3  & &=  \lambda \\
                     & & - \alpha_2 &+ 2 \alpha_3 &- \beta &= \lambda \\
                \end{array}
                \right.
        \]
         Pour obtenir $\alpha_3$, nous effectuons $L_1+2L_2+3L_3$ et nous avons
         \[
         4 \alpha_3 = 6 \lambda + \alpha + 3 \beta\,.
         \]
         Ainsi, 
         \[
         \alpha_3 = \frac{3}{2}\lambda + \frac{\alpha +3\beta}{4}\,.
         \]

         Il suffit alors de réinjecter dans $L_3$ pour obtenir $\alpha_2$ 
         \[
         \alpha_2 = 2\lambda + \frac{\alpha + \beta}{2}
         \]
         et de même pour $\alpha_1$
         \[
         \alpha_1 = \frac{3}{2}\lambda + \frac{3\alpha + \beta}{4}\,.
         \]
        
        \item \textbf{Solution Exacte :}
        
        L'équation est réduite à 
        \[
        u'' = - f_0
        \]
        Intégrant deux fois, la solution exacte s'écrit sous la forme :
        \[
        u(x) = -\frac{f_0}{2}x^2 + Cx + D
        \]
        En appliquant les conditions aux limites \( u(0) = \alpha \) donc $D=\alpha$ et \( u(1) = \beta \), donc $C= \beta-\alpha+\frac{f_0}{2}.$
        Finalement, 
        \[
        u(x) = -\frac{f_0}{2}x^2 + \left(\beta - \alpha + \frac{f_0}{2}\right)x + \alpha
        \]

        \item \textbf{Comparaison :}

        Regardons sur les noeuds du maillage, au point $x=h$ par exemple. D'un côté la solution approchée donne
        \[u_h(h) = \alpha_1 = \frac{3}{2}f_0 h^2 + \frac{\alpha +3\beta}{4} \]
        et de l'autre, pour la solution exacte: 
        \[u(h) = -\frac{f_0}{2}h^2 + \left(\beta - \alpha + \frac{f_0}{2}\right)h + \alpha = \frac{3}{2}f_0 h^2 + \frac{\alpha +3\beta}{4} \]

        ainsi, les deux fonctions coïncident. On aurait de même sur $x=2h$ et $x=3h$.
        En revanche, entre les noeuds, la différence est significative. Notre fonction approchée relie les valeurs aux noeud par des segments de droite. 
        La solution réelle est beaucoup plus lisse.

        \item \textbf{Amélioration de l'Approximation :}

        \begin{itemize}
            \item Affiner le maillage en augmentant le nombre d'éléments.
            \item Utiliser des fonctions de base de degré supérieur.
        \end{itemize}
    \end{enumerate}
    \end{solution}
    }{}
    

\end{enumerate}

\vspace{1em}

\subsection*{D'autres fonctions \( p \), \( q \), et \( f \)}

Reprenez le problème avec \( p(x) = e^x \), \( q(x) = \sin(x) \), et \( f(x) = x^2 \).

\end{document}

\end{document}
