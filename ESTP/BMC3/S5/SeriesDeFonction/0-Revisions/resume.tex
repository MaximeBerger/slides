
\begin{resumeBox}
  \emph{Ce qu'il faut savoir :} 
  \begin{niceitemize}
    \item Les formules pour calculer les sommes de séries arithmétiques et géométriques.
    \item Le critère de Riemann pour les séries à termes positifs.
    \item Les contre-exemples de suites de fonctions qui convergent simplement mais pas uniformément. 
  \end{niceitemize}
\end{resumeBox}
% \begin{formulesBox}
%   % Placez ici vos formules, figures TikZ ou images.
%   % Exemple : $F(\omega)=\int_{-\infty}^{+\infty} f(t)\,e^{-i\omega t}\,dt$
%   % \begin{center}\includegraphics[width=.8\linewidth]{exemple.jpg}\end{center}
% \end{formulesBox}
\begin{rappelsBox}
  \begin{niceitemize}
    \item Quelle est la définition d'une suite convergente ?
    \item Quels sont les outils pour montrer la convergence d'une série numérique ? 
    \item Quels sont les deux modes de convergence d'une suite de fonctions ?
  \end{niceitemize}
\end{rappelsBox}

\section*{Exercices - Suites Numériques}

\vspace{1em}

\subsection{Des exemples}
Donner deux exemples différents dans chacune des situations suivantes :
\begin{enumerate}[label = $\square$]
  \item une suite décroissante positive dont le terme général ne tend pas vers 0 .
  \item une suite bornée non convergente.
  \item une suite positive non bornée ne tendant pas vers $+\infty$.
  \item une suite non monotone qui tend vers 0.
  \item une suite positive qui tend vers 0 et qui n'est pas décroissante.
\end{enumerate}

\vspace{1em}

\newpage
  \subsection{Vrai ou Faux ?}
  Dire si les assertions suivantes sont vraies ou fausses. On justifiera les réponses avec une démonstration ou un contre-exemple.
  \begin{enumerate}[label = $\square$]
    \item Toute suite non-majorée tend vers $+\infty$.
    \item Soit $\left(u_n\right)_{n \geq 0}$ une suite à termes positifs convergeant vers $0$. Alors, $(u_n)$ est décroissante à partir d'un certain rang.
    \item Si $(u_n)$ est une suite géométrique de raison $q \neq 0$, alors $\left(\frac{1}{u_n}\right)$ est une suite géométrique de raison $\frac{1}{q}$.
    \item Soit $(u_n)$ une suite croissante et $\ell \in \mathbb{R}$. Si pour tout $N \in \mathbb{N}$, il existe $n_0 \geq N$ tel que $u_{n_0} > \ell$, alors $(u_n)$ ne converge pas vers $\ell$.
    \item Si $f: \mathbb{R} \rightarrow \mathbb{R}$ est croissante et que $(u_n)$ vérifie $u_{n+1}=f\left(u_n\right)$ pour tout entier $n$,  alors $(u_n)$ est croissante.
    \item Si $u$ est divergente, alors $u$ est non bornée.
    \item Si $u_n\to \ell$ et $f$ continue, alors $f(u_n)\to f(\ell)$
  \end{enumerate}

  \vspace{1em}


  \subsection{Étude de suites}
  Étudier la nature des suites suivantes, et déterminer un équivalent simple:
  \begin{multicols}{3}
  \begin{enumerate}[label = \alph*), itemsep = 0.5em]
    \item $\displaystyle u_n=\frac{\sin (n)+3 \cos \left(n^2\right)}{\sqrt{n}}$
    \item $\displaystyle u_n=\frac{2 n+(-1)^n}{5 n+(-1)^{n+1}}$
    \item $\displaystyle u_n=\frac{n^3+5 n}{4 n^2+\sin (n)+\ln (n)}$
    \item $\displaystyle u_n=\sqrt{2 n+1}-\sqrt{2 n-1}$
    \item $\displaystyle u_n=3^n e^{-3 n}$.
    \item $\displaystyle u_n=\frac{n}{2^n}$
    \item $\displaystyle u_n=\frac{n!}{45^n}$
    \item $\displaystyle u_n=\frac{n!}{n^n}$
    \item $\displaystyle u_n=\frac{n^3+2^n}{n^2+3^n}$.
  \end{enumerate}
  \end{multicols}

  \vspace{1em}


  \subsection{Plus difficile}
  Étudier la nature des suites suivantes, et déterminer un équivalent simple:
  \begin{multicols}{2}
  \begin{enumerate}[label = \alph*)]
    \item $u_n=\ln \left(2 n^2-n\right)-\ln (3 n+1)$
    \item $u_n=\sqrt{n^2+n+1}-\sqrt{n^2-n+1}$
    \item $_n=\frac{a^n-b^n}{a^n+b^n}, a, b \in] 0,+\infty[$
    \item $u_n=\frac{\ln \left(n+e^n\right)}{n}$
    \item $u_n=\frac{\ln (1+\sqrt{n})}{\ln \left(1+n^2\right)}$.
  \end{enumerate}
  \end{multicols}

  \vspace{1em}

  \subsection{Formule de Stirling}
  \begin{enumerate}[label = \alph*)]
    \item Soit $\left(x_n\right)$ une suite de réels et soit $\left(y_n\right)$ définie par $y_n=x_{n+1}-x_n$. Démontrer que la série $\sum_n y_n$ et la suite $\left(x_n\right)$ sont de même nature.
    \item On pose ( $u_n$ ) la suite définie par $u_n=\frac{n^n e^{-n} \sqrt{n}}{n!}$. Donner la nature de la série de terme général $v_n=\ln \left(\frac{u_{n+1}}{u_n}\right)$.
    \item En déduire l'existence d'une constante $C>0$ telle que :
  \end{enumerate}
  $$
  n!\sim_{+\infty} C \sqrt{n} n^n e^{-n}
  $$

  \vspace{1em}


  \subsection{Télescopiques}
  \begin{enumerate}[label = \alph*)]
    \item Déterminer deux réels $a$ et $b$ tels que $\displaystyle
\frac{1}{k^2-1}=\frac{a}{k-1}+\frac{b}{k+1} .
$

    \item En déduire la limite de la suite $\displaystyle
u_n=\sum_{k=2}^n \frac{1}{k^2-1} .
$

    \item Sur le même modèle, déterminer la limite de la suite$
v_n=\sum_{k=0}^n \frac{1}{k^2+3 k+2}
$


    \item Montrer que, pour tout $n \in \mathbb{N}^*$, on a $
\sqrt{n+1}-\sqrt{n} \leq \frac{1}{2 \sqrt{n}}
$


    \item En déduire le comportement de la suite ( $u_n$ ) définie par $
u_n=1+\frac{1}{\sqrt{2}}+\cdots+\frac{1}{\sqrt{n}} .
$
  \end{enumerate}


\vspace{1em}

\section*{Séries numériques}
\subsection{Paramètres}
Soit $a, b \in \mathbb{R}$. Pour $n \geq 1$, on pose $u_n=\ln (n)+a \ln (n+1)+b \ln (n+2)$.
\begin{enumerate}[label = \alph*)]
  \item Pour quelle(s) valeur(s) de $(a, b)$ la série $\sum u_n$ est-elle convergente?
  \item Dans le(s) cas où la série converge, déterminer $\sum_{n=1}^{+\infty} u_n$.
\end{enumerate}

\vspace{1em}

\subsection{Avec l'exponentielle}
Sachant que $e=\sum_{n \geq 0} \frac{1}{n!}$, déterminer la valeur des sommes suivantes :
\begin{enumerate}[label = \alph*)]
  \item $\sum_{n \geq 0} \frac{n+1}{n!}$
  \item $\sum_{n \geq 0} \frac{n^2-2}{n!}$
  \item $\sum_{n \geq 0} \frac{n^3}{n!}$.
\end{enumerate}


\vspace{1em}
\section*{Suites de fonctions}

\subsection{Suites de fonctions}
Soit $\left(f_n\right)$ une suite de fonctions qui converge simplement vers une fonction $f$ sur un intervalle $I$. Dire si les assertions suivantes sont vraies ou fausses :
\begin{enumerate}[label = \alph*)]
  \item Si les $f_n$ sont croissantes, alors $f$ aussi.
  \item Si les $f_n$ sont strictement croissantes, alors $f$ aussi.
  \item Si les $f_n$ sont périodiques de période $T$, alors $f$ aussi.
  \item Si les $f_n$ sont continues en $a$, alors $f$ aussi.
\end{enumerate}

Reprendre l'exercice en remplaçant la convergence simple par la convergence uniforme.

\vspace{1em}

\subsection{Suites de fonctions}
On pose, pour $n \geq 1$ et $x \in] 0,1], f_n(x)=n x^n \ln (x)$ et $f_n(0)=0$.
\begin{enumerate}[label = \alph*)]
  \item Démontrer que $\left(f_n\right)$ converge simplement sur $[0,1]$ vers une fonction $f$ que l'on précisera. On note ensuite $g=f-f_n$.
  \item Étudier les variations de g.
  \item En déduire que la convergence de $\left(f_n\right)$ vers $f$ n'est pas uniforme sur $[0,1]$.
  \item Soit $a \in[0,1]$. En remarquant qu'il existe $n_0 \in \mathbb{N}$ tel que $e^{-1 / n} \geq a$ pour tout $n \geq n_0$, démontrer que la suite $\left(f_n\right)$ converge uniformément vers $f$ sur $[0, a]$.
\end{enumerate}