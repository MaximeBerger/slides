\documentclass[12pt]{article}
\usepackage[french]{babel}
\usepackage[utf8]{inputenc}
\usepackage[T1]{fontenc}
\usepackage{lmodern}           % Police Latin Modern (plus nette)
\usepackage{charter}           % Police Charter (très lisible)
\usepackage[scaled=0.95]{inconsolata} % Police mono lisible
\usepackage{amsmath}
\usepackage{amsfonts}
\usepackage{amssymb}
\usepackage{amsthm}
\usepackage[version=4]{mhchem}
\usepackage{stmaryrd}
\usepackage[most]{tcolorbox}
\usepackage{xcolor}
\usepackage{geometry}
\geometry{margin=1.5cm}

\usepackage{xcolor}
\definecolor{Theme}{HTML}{0E7490} % teal-700
\definecolor{ThemeLight}{HTML}{E0F2F1}
\definecolor{Accent}{HTML}{F59E0B} % amber-500
\definecolor{Gray}{HTML}{374151}
\usepackage[colorlinks=true,linkcolor=Theme,urlcolor=Theme,citecolor=Theme]{hyperref}

\usepackage{mdframed}
\usepackage[sf]{titlesec}
\usepackage{array}
\usepackage{ifthen}
\usepackage{enumitem}

\newboolean{showSolutions}
% Décommentez la ligne suivante pour afficher les solutions
\input \jobname.adr

\title{Examen S5 - Mathématiques }
\author{}
\date{}

\newenvironment{solution}
    {\par\vspace{0.5em}\begin{mdframed}[backgroundcolor=ThemeLight,linewidth=0.5pt]\noindent\textbf{Solution :}\par}
    {\end{mdframed}\par\vspace{0.5em}}

\begin{document}
\sffamily

\begin{center}
    \renewcommand{\arraystretch}{1.5} % Ajuste l'espacement vertical des lignes
    \begin{tabular}{|>{\centering\arraybackslash}m{4cm}|>{\centering\arraybackslash}m{6cm}|>{\centering\arraybackslash}m{4cm}|}
        \hline 
        \vspace{5mm} \hspace{5mm}\raisebox{-0.2\height}{\includegraphics[width=3cm]{Logoestp.png}} \vspace{5mm}  & 
        \textbf{Contrôle de connaissances et de compétences} & 
        \textbf{FO-002-VLA-XX-001} \\
        \hline
        \textbf{26/01/2026}  &  & \textbf{Page 1/2} \\
        \hline
    \end{tabular}
\end{center}
\vspace{1em}

\begin{center}
    \renewcommand{\arraystretch}{1.5}
    \begin{tabular}{|c|m{10cm}|}
        \hline 
        \multicolumn{2}{|c|}{\textbf{ANNÉE SCOLAIRE 2025-2026 -- Semestre 1}} \\
        \hline 
        \textbf{Nom de l'enseignant} & Maxime Berger \& Antoine Perney \\
        \hline 
        \textbf{Promotion} & BMC3 - S5 \\
        \hline 
        \textbf{Matière} & Mathématiques \\
        \hline 
        \textbf{Durée de l'examen} & 3h00 \\
        \hline 
        \textbf{Consignes} & 
        \vspace{0.5em}
        \begin{itemize}
            \item Calculatrice \textbf{NON} autorisée
            \item Aucun document n'est autorisé \vspace{1em}
        \end{itemize}\\
        
        \hline
    \end{tabular}
\end{center}

\vspace{2em}

%==============================================================================
\section*{Exercice 1 : Développements limités (5 points)}
%==============================================================================

\begin{enumerate}
    \item \textbf{Calculs de développements limités.}
    \begin{enumerate}
        \item Donner le développement limité de $e^x$ à l'ordre 4 au voisinage de 0. \textit{(0.5 pt)}
        
        \ifthenelse{\boolean{showSolutions}}{
        \begin{solution}
        \[
        e^x = 1 + x + \frac{x^2}{2} + \frac{x^3}{6} + \frac{x^4}{24} + o(x^4)
        \]
        \end{solution}
        }{}
        
        \item Donner le développement limité de $\ln(1+x)$ à l'ordre 4 au voisinage de 0. \textit{(0.5 pt)}
        
        \ifthenelse{\boolean{showSolutions}}{
        \begin{solution}
        \[
        \ln(1+x) = x - \frac{x^2}{2} + \frac{x^3}{3} - \frac{x^4}{4} + o(x^4)
        \]
        \end{solution}
        }{}
        
        \item En déduire le développement limité de $f(x) = e^x \ln(1+x)$ à l'ordre 3 au voisinage de 0. \textit{(1 pt)}
        
        \ifthenelse{\boolean{showSolutions}}{
        \begin{solution}
        On multiplie les DL en ne gardant que les termes d'ordre $\leq 3$ :
        \begin{align*}
        e^x \ln(1+x) &= \left(1 + x + \frac{x^2}{2} + \frac{x^3}{6}\right)\left(x - \frac{x^2}{2} + \frac{x^3}{3}\right) + o(x^3) \\
        &= x - \frac{x^2}{2} + \frac{x^3}{3} + x^2 - \frac{x^3}{2} + \frac{x^3}{2} + o(x^3) \\
        &= x + \frac{x^2}{2} + \frac{x^3}{3} + o(x^3)
        \end{align*}
        \end{solution}
        }{}
    \end{enumerate}
    
    \item \textbf{Calcul de limite.} Calculer la limite suivante à l'aide d'un développement limité : \textit{(1.5 pts)}
    \[
    \lim_{x \to 0} \frac{\sin x - x + \frac{x^3}{6}}{x^5}
    \]
    
    \ifthenelse{\boolean{showSolutions}}{
    \begin{solution}
    On utilise le DL de $\sin x$ à l'ordre 5 :
    \[
    \sin x = x - \frac{x^3}{6} + \frac{x^5}{120} + o(x^5)
    \]
    Donc :
    \[
    \sin x - x + \frac{x^3}{6} = \frac{x^5}{120} + o(x^5)
    \]
    Et :
    \[
    \frac{\sin x - x + \frac{x^3}{6}}{x^5} = \frac{1}{120} + o(1) \xrightarrow[x \to 0]{} \boxed{\frac{1}{120}}
    \]
    \end{solution}
    }{}
    
    \item \textbf{Étude d'une fonction.} Soit $g(x) = \dfrac{1 - \cos x}{x^2}$ pour $x \neq 0$.
    \begin{enumerate}
        \item À l'aide d'un développement limité, montrer que $g$ admet un prolongement par continuité en 0 et déterminer sa valeur. \textit{(0.75 pt)}
        
        \ifthenelse{\boolean{showSolutions}}{
        \begin{solution}
        On a $\cos x = 1 - \frac{x^2}{2} + \frac{x^4}{24} + o(x^4)$, donc :
        \[
        1 - \cos x = \frac{x^2}{2} - \frac{x^4}{24} + o(x^4)
        \]
        Ainsi :
        \[
        g(x) = \frac{1 - \cos x}{x^2} = \frac{1}{2} - \frac{x^2}{24} + o(x^2) \xrightarrow[x \to 0]{} \frac{1}{2}
        \]
        On peut prolonger $g$ par continuité en posant $g(0) = \frac{1}{2}$.
        \end{solution}
        }{}
        
        \item En déduire la position de la courbe de $g$ par rapport à sa tangente horizontale en 0. \textit{(0.75 pt)}
        
        \ifthenelse{\boolean{showSolutions}}{
        \begin{solution}
        Le DL de $g$ en 0 est :
        \[
        g(x) = \frac{1}{2} - \frac{x^2}{24} + o(x^2)
        \]
        La tangente en 0 est $y = \frac{1}{2}$ (horizontale car le terme en $x$ est nul).
        
        On a $g(x) - \frac{1}{2} = -\frac{x^2}{24} + o(x^2) \sim -\frac{x^2}{24} < 0$ pour $x \neq 0$ petit.
        
        Donc la courbe est \textbf{en dessous} de sa tangente au voisinage de 0.
        \end{solution}
        }{}
    \end{enumerate}
\end{enumerate}

\newpage

\begin{center}
    \renewcommand{\arraystretch}{1.5} 
    \begin{tabular}{|>{\centering\arraybackslash}m{4cm}|>{\centering\arraybackslash}m{6cm}|>{\centering\arraybackslash}m{4cm}|}
        \hline
            \hspace{4cm}&\hspace{6cm} & \textbf{Page 2/2}\\
            \hline
    \end{tabular}
\end{center}

%==============================================================================
\section*{Exercice 2 : Équations aux dérivées partielles (5 points)}
%==============================================================================

\begin{enumerate}
    \item \textbf{Équation exacte.} On considère l'équation différentielle :
    \[
    (2xy + 3) \, dx + (x^2 + 4y) \, dy = 0
    \]
    \begin{enumerate}
        \item Vérifier que cette équation est exacte, c'est-à-dire que l'expression $f(x,y)\,dx + g(x,y)\,dy$ est une différentielle totale. \textit{(0.75 pt)}
        
        \ifthenelse{\boolean{showSolutions}}{
        \begin{solution}
        On pose $f(x,y) = 2xy + 3$ et $g(x,y) = x^2 + 4y$.
        \[
        \frac{\partial f}{\partial y} = 2x \qquad \text{et} \qquad \frac{\partial g}{\partial x} = 2x
        \]
        Comme $\frac{\partial f}{\partial y} = \frac{\partial g}{\partial x}$, l'équation est exacte.
        \end{solution}
        }{}
        
        \item Trouver une fonction $F(x,y)$ telle que $dF = f\,dx + g\,dy$. \textit{(1 pt)}
        
        \ifthenelse{\boolean{showSolutions}}{
        \begin{solution}
        On cherche $F$ telle que $\frac{\partial F}{\partial x} = 2xy + 3$.
        
        En intégrant par rapport à $x$ : $F(x,y) = x^2 y + 3x + H(y)$
        
        On vérifie avec $\frac{\partial F}{\partial y} = x^2 + H'(y) = x^2 + 4y$.
        
        Donc $H'(y) = 4y$, soit $H(y) = 2y^2 + C$.
        
        \textbf{Conclusion :} $F(x,y) = x^2 y + 3x + 2y^2$
        \end{solution}
        }{}
        
        \item En déduire la solution générale de l'équation différentielle. \textit{(0.5 pt)}
        
        \ifthenelse{\boolean{showSolutions}}{
        \begin{solution}
        Les solutions sont données par $F(x,y) = K$ où $K$ est une constante :
        \[
        \boxed{x^2 y + 3x + 2y^2 = K}
        \]
        \end{solution}
        }{}
    \end{enumerate}
    
    \item \textbf{EDP linéaire d'ordre 1.} Résoudre l'équation aux dérivées partielles :
    \[
    \frac{\partial f}{\partial x} + 2\frac{\partial f}{\partial y} = 0
    \]
    en utilisant un changement de variables linéaire. \textit{(1.5 pts)}
    
    \ifthenelse{\boolean{showSolutions}}{
    \begin{solution}
    On pose $\begin{cases} X = ax + by \\ Y = cx + dy \end{cases}$ et $F(X,Y) = f(x,y)$.
    
    L'équation devient $(a + 2b)\frac{\partial F}{\partial X} + (c + 2d)\frac{\partial F}{\partial Y} = 0$.
    
    On choisit $a = 2$, $b = -1$ (donc $a + 2b = 0$) et $c = 1$, $d = 0$ (donc $c + 2d = 1$).
    
    L'équation devient $\frac{\partial F}{\partial Y} = 0$, dont les solutions sont $F(X,Y) = K(X)$.
    
    En revenant aux variables initiales avec $X = 2x - y$ :
    \[
    \boxed{f(x,y) = K(2x - y)}
    \]
    où $K$ est une fonction de classe $\mathcal{C}^1$ quelconque.
    \end{solution}
    }{}
    
    \item \textbf{Méthode de séparation de variables.} On considère l'équation :
    \[
    \frac{\partial f}{\partial x} - 3\frac{\partial f}{\partial y} = 2f
    \]
    En cherchant des solutions sous la forme $f(x,y) = X(x)Y(y)$, déterminer les solutions de cette équation. \textit{(1.25 pts)}
    
    \ifthenelse{\boolean{showSolutions}}{
    \begin{solution}
    En posant $f(x,y) = X(x)Y(y)$, on obtient :
    \[
    X'Y - 3XY' = 2XY
    \]
    En divisant par $XY$ :
    \[
    \frac{X'}{X} - 3\frac{Y'}{Y} = 2 \quad \Rightarrow \quad \frac{X' - 2X}{X} = 3\frac{Y'}{Y}
    \]
    Le membre de gauche ne dépend que de $x$, le membre de droite que de $y$. Ils sont donc tous deux égaux à une constante $k$.
    
    \textbf{Pour $X$ :} $\frac{X' - 2X}{X} = k \Rightarrow X' = (k+2)X \Rightarrow X(x) = C_1 e^{(k+2)x}$
    
    \textbf{Pour $Y$ :} $3\frac{Y'}{Y} = k \Rightarrow Y' = \frac{k}{3}Y \Rightarrow Y(y) = C_2 e^{ky/3}$
    
    \textbf{Solutions :} $\boxed{f(x,y) = C \, e^{(k+2)x} e^{ky/3}}$ pour $k \in \mathbb{R}$, $C \in \mathbb{R}$.
    \end{solution}
    }{}
\end{enumerate}


\vspace{1em}

%==============================================================================
\section*{Exercice 3 : Modélisation -- Refroidissement d'une pièce métallique (5 points)}
%==============================================================================

Une pièce métallique de masse $m = 2$ kg et de capacité thermique massique $c = 500$ J/(kg$\cdot$K) est initialement à la température $T_0 = 400$ K. Elle est plongée dans un bain thermostaté à la température constante $T_{\infty} = 300$ K.

Le transfert thermique entre la pièce et le bain suit la loi de Newton :
\[
\frac{dQ}{dt} = -hS(T - T_{\infty})
\]
où $Q$ est l'énergie thermique de la pièce, $h = 25$ W/(m$^2\cdot$K) est le coefficient d'échange et $S = 0{,}04$ m$^2$ est la surface d'échange.

On rappelle que $Q = mcT$ (à constante près).

\begin{enumerate}
    \item Montrer que la température $T(t)$ de la pièce vérifie l'équation différentielle : \textit{(1 pt)}
    \[
    \frac{dT}{dt} = -\frac{hS}{mc}(T - T_{\infty})
    \]
    
    \ifthenelse{\boolean{showSolutions}}{
    \begin{solution}
    On a $Q = mcT + \text{cste}$, donc $\frac{dQ}{dt} = mc\frac{dT}{dt}$.
    
    En substituant dans la loi de Newton :
    \[
    mc\frac{dT}{dt} = -hS(T - T_{\infty})
    \]
    
    Soit :
    \[
    \frac{dT}{dt} = -\frac{hS}{mc}(T - T_{\infty})
    \]
    \end{solution}
    }{}
    
    \item On pose $\tau = \frac{mc}{hS}$ (constante de temps). Calculer la valeur numérique de $\tau$. \textit{(0.5 pt)}
    
    \ifthenelse{\boolean{showSolutions}}{
    \begin{solution}
    \[
    \tau = \frac{mc}{hS} = \frac{2 \times 500}{25 \times 0{,}04} = \frac{1000}{1} = 1000 \text{ s}
    \]
    \end{solution}
    }{}
    
    \item Résoudre l'équation différentielle avec la condition initiale $T(0) = T_0$. \textit{(1.5 pts)}
    
    \ifthenelse{\boolean{showSolutions}}{
    \begin{solution}
    On pose $\theta(t) = T(t) - T_{\infty}$. L'équation devient :
    \[
    \frac{d\theta}{dt} = -\frac{1}{\tau}\theta
    \]
    
    C'est une équation linéaire du premier ordre. La solution est :
    \[
    \theta(t) = \theta_0 e^{-t/\tau}
    \]
    
    Avec $\theta_0 = T_0 - T_{\infty} = 400 - 300 = 100$ K.
    
    Donc :
    \[
    \boxed{T(t) = T_{\infty} + (T_0 - T_{\infty})e^{-t/\tau} = 300 + 100\,e^{-t/1000}}
    \]
    \end{solution}
    }{}
    
    \item Au bout de combien de temps la pièce atteint-elle la température de 320 K ? \textit{(1 pt)}
    
    \textit{On donne $\ln(5) \approx 1{,}6$.}
    
    \ifthenelse{\boolean{showSolutions}}{
    \begin{solution}
    On résout $T(t) = 320$ :
    \[
    300 + 100\,e^{-t/1000} = 320
    \]
    \[
    100\,e^{-t/1000} = 20 \quad \Rightarrow \quad e^{-t/1000} = 0{,}2 = \frac{1}{5}
    \]
    \[
    -\frac{t}{1000} = \ln\left(\frac{1}{5}\right) = -\ln(5)
    \]
    \[
    t = 1000 \ln(5) \approx 1000 \times 1{,}6 = \boxed{1600 \text{ s} \approx 27 \text{ min}}
    \]
    \end{solution}
    }{}
    
    \item Vers quelle valeur tend $T(t)$ quand $t \to +\infty$ ? Interpréter physiquement. \textit{(1 pt)}
    
    \ifthenelse{\boolean{showSolutions}}{
    \begin{solution}
    \[
    \lim_{t \to +\infty} T(t) = \lim_{t \to +\infty} \left(300 + 100\,e^{-t/1000}\right) = 300 \text{ K}
    \]
    
    \textbf{Interprétation :} La pièce métallique tend vers l'équilibre thermique avec le bain thermostaté. Elle atteint asymptotiquement la température $T_{\infty} = 300$ K du bain.
    \end{solution}
    }{}
\end{enumerate}

\end{document}