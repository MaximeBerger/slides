\documentclass[12pt]{article}
\usepackage[utf8]{inputenc}
\usepackage[T1]{fontenc}
\usepackage[french]{babel}
\usepackage{amsmath, amssymb, amsfonts, amsthm}
\usepackage{graphicx}
\usepackage{hyperref}
\usepackage{geometry}
\usepackage{enumitem}
\usepackage{float}
\usepackage{tikz}
\usepackage{ifthen}
\usepackage{mdframed}
\usepackage[sf]{titlesec}

\newmdenv[linewidth=1pt]{solutionbox}
\geometry{a4paper, margin=2.5cm}

% Définition de la variable pour afficher les corrections
\newboolean{showSolutions}
% Décommentez la ligne suivante pour afficher les solutions
\setboolean{showSolutions}{true}
% \setboolean{showSolutions}{false}

\title{Méthode des Éléments Finis : Examen}

\newtheorem{definition}{Définition}[section]
\newtheorem{theorem}{Théorème}[section]
\newtheorem{lemma}{Lemme}[section]
\newtheorem{proposition}{Proposition}[section]
\newtheorem{corollary}{Corollaire}[section]
\theoremstyle{remark}
\newtheorem*{remark}{Remarque}

\begin{document}
\sffamily

\date{}

\begin{center}
    \begin{tabular}{|p{4cm}|c|p{4cm}|}
        \hline 
        \raisebox{-0.5\height}{\includegraphics[width=3cm]{Logo-ESTP.png}} & 
        \textbf{Contrôle de connaissances et de compétences} & FO-002-VLA-XX-001 \\
        \hline
        & & 21/01/2025 \\
        \hline
        & & Page 1/2 \\
        \hline
    \end{tabular}

\vspace{1em}

\begin{tabular}{|c|p{10cm}|}
    \hline & \rule{0pt}{3ex} {\textbf{ANNÉE SCOLAIRE 2024-2025 Semestre 7}} \\
    \hline \textbf{Nom de l'enseignant} & Maxime BERGER \\
    \hline \textbf{Matière} & Méthode des Éléments Finis \\
    \hline \textbf{Durée de l'examen} & 2h00 \\
    \hline \textbf{Consignes} & 
    \begin{tabular}{l}
    Calculatrice NON autorisée \\
    Aucun document n'est autorisé
    \end{tabular} \\
    \hline \textbf{NOM - PRENOM de l'étudiant} : & \\
    & \\
    \hline
\end{tabular}
\end{center}

\vspace{5em}
\section*{Exercice 1}
On considère une barre de longueur $1$. On la découpe en $n$ éléments finis de longueur $h$, d'ordre $1$. 

On cherche les polynômes de Hermite qui permettront d'interpoler la valeur de la fonction solution et de sa dérivée en chaque noeud. 
Considérons l'élément $[0, h]$. 
\begin{enumerate}
    \item Rappelez les conditions que doivent vérifier les polynômes de Hermite $\varphi_0$, $\varphi_1$, $\psi_0$ et $\psi_1$.
    \item Voici les polynômes, déterminer les éléments manquants $a_1, a_2, a_3, a_4$ dans ces formules. Justifier votre choix :
    \begin{align*}
    \varphi_0(x) &= a_1 - 3\Big(\frac{x}{h}\Big)^2 + 2 \Big(\frac{x}{h}\Big)^3 \\
    \varphi_1(x) &= 3\Big(\frac{x}{h}\Big)^2 + a_2 \Big(\frac{x}{h}\Big)^3 \\
    \psi_0(x) &= a_3x\Big(1-2\frac{x}{h} + \big(\frac{x}{h}\big)^2\Big) \\
    \psi_1(x) &= x\Big(-\frac{x}{h} + a_4\big(\frac{x}{h}\big)^2\Big)
\end{align*}
\end{enumerate}
\newpage 
\section*{Exercice 2}
Voici une pièce d'acier de forme trapézoïdale, trouée en son centre avec un trou carré de $0.5$ mètre de côté.
\begin{center}
\begin{tikzpicture}[scale=1.5]
    % Contour extérieur
    \draw[thick] (0,0) -- (4,0) -- (4,1) -- (3,2) -- (1,2) -- (0,1) -- cycle;
    
    % Trou carré
    \draw[thick] (1.75,0.75) rectangle (2.25,1.25);
    
    %% Cotations
    \draw[<->] (0,-0.3) -- (4,-0.3) node[midway,below] {4 m};
    \draw[<->] (-0.3,0) -- (-0.3,2) node[midway,left] {2 m};
    \draw[<->] (1,2.3) -- (3,2.3) node[midway,above] {2 m};
    \draw[<->] (4.3,0) -- (4.3,1) node[midway,right] {1 m};
% Ajouter une grille en pointillés
\draw[help lines,gray!30] (0,0) grid (4,2);
    
% Ajouter des suggestions de points pour le maillage
\foreach \x in {0,1,2,3,4} {
    \foreach \y in {0,1,2} {
        \fill[gray] (\x,\y) circle (0.02);
    }
}
\end{tikzpicture}
\end{center}
\begin{enumerate}
    \item Créez un maillage de cette pièce en triangles puis établissez les tableaux de coordonnées et de connectivités.

\textit{Vous représenterez sur votre copie le dessin soigné en donnant une notation claire aux éléments et aux noeuds.}

    \item Rappelez la méthode pour obtenir les polynômes de Lagrange associés à un maillage triangulaire en dimension $2$.
    \item Expliquez comment est construite la matrice de rigidité globale avec un tel maillage.
\end{enumerate}
\section*{Problème}
Considérons l'équation différentielle suivante en dimension $1$ :
\[
- \Delta u + q(x) u = 0, \quad x \in ]0, 1[
\]
avec \(q\) une fonction positive.

Avec les conditions aux limites de Dirichlet: \( u(0) = u(1) = 0 \).
\begin{enumerate}
    \item Etablir la formulation variationnelle du problème.

    \item Définissez un maillage avec $3$ éléments finis de longueur $h$ et d'ordre $1$.
    
    \item Donner les polynômes de Lagrange associés à chaque élément de ce maillage.

    \item A partir de cette question, on prendra \(q(x) = 0\). Calculez la matrice de rigidité associée au problème de Dirichlet sur le premier élément $[0,h]$.
        
    \item Etablissez la matrice de rigidité globale \( K \), puis appliquez les conditions de bord.

    \item Ecrivez le système linéaire puis donner la solution approchée du problème. 

    \item Comment le système linéaire est-il modifié si le second membre de l'équation différentielle est maintenant \(f(x) = \cos(\pi x)\) ?

    \item Si maintenant $q(x) = 1$, ecrire la nouvelle matrice de rigidité du système, sans vous préoccuper de sa résolution ou du second membre.
\end{enumerate}

\end{document}
