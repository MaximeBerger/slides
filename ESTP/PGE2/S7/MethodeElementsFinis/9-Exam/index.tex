\documentclass[12pt]{article}
\usepackage[french]{babel}
\usepackage[utf8]{inputenc}
\usepackage[T1]{fontenc}
\usepackage{lmodern}           % Police Latin Modern (plus nette)
\usepackage{charter}           % Police Charter (très lisible)
\usepackage[scaled=0.95]{inconsolata} % Police mono lisible
\usepackage{amsmath}
\usepackage{amsfonts}
\usepackage{amssymb}
\usepackage{amsthm}
\usepackage[version=4]{mhchem}
\usepackage{stmaryrd}
\usepackage[most]{tcolorbox}
\usepackage{xcolor}
\usepackage{geometry}
\geometry{margin=1.5cm}

\usepackage{mdframed}
\usepackage[sf]{titlesec}
\usepackage{array}
\usepackage{ifthen}
% Définition de la variable pour afficher les corrections
\newboolean{showSolutions}
% Décommentez la ligne suivante pour afficher les solutions
\input \jobname.adr

\title{Examen S7 - Méthode des éléments finis }

\author{}
\date{}


\newenvironment{solution}
    {\par\vspace{0.5em}\begin{mdframed}[linewidth=0.5pt]\noindent\textbf{Solution :}\par}
    {\end{mdframed}\par\vspace{0.5em}}

\begin{document}
\sffamily

\begin{center}
    \renewcommand{\arraystretch}{1.5} % Ajuste l'espacement vertical des lignes
    \begin{tabular}{|>{\centering\arraybackslash}m{4cm}|>{\centering\arraybackslash}m{6cm}|>{\centering\arraybackslash}m{4cm}|}
        \hline 
        \vspace{5mm} \hspace{5mm}\raisebox{-0.2\height}{\includegraphics[width=3cm]{Logo-ESTP.png}} \vspace{5mm}  & 
        \textbf{Contrôle de connaissances et de compétences} & 
        \textbf{FO-002-VLA-XX-001} \\
        \hline
        \textbf{22/01/2026}  &  & \textbf{Page 1/2} \\
        \hline
    \end{tabular}
\end{center}
\vspace{1em}

\begin{center}
    \renewcommand{\arraystretch}{1.5}
    \begin{tabular}{|c|m{10cm}|}
        \hline 
        \multicolumn{2}{|c|}{\textbf{ANNÉE SCOLAIRE 2025-2026 -- Semestre 1}} \\
        \hline 

        \textbf{Nom de l'enseignant} & Maxime Berger \\
        \hline 
        \textbf{Matière} & Méthode des éléments finis \\
        \hline 
        \textbf{Promotion} & PGE2 - S7 \\
        \hline 
        \textbf{Durée de l'examen} & 1h30 \\
        \hline 
        \textbf{Consignes} & 
        \vspace{0.5em}
        \begin{itemize}
            \item Calculatrice \textbf{NON} autorisée
            \item Aucun document n'est autorisé \vspace{1em}
        \end{itemize}\\
        
        \hline
    \end{tabular}
\end{center}


\vspace{1em}

\section*{Problème}
Considérons l'équation différentielle suivante en dimension $1$ :
\[
- u''(x) + u(x) = f(x), \quad x \in ]0, 1[
\]
avec les conditions aux limites de Dirichlet : $u(0) = 0$ et $u(1) = 0$.

\begin{enumerate}
    \item Établissez la formulation variationnelle du problème. On introduira une forme bilinéaire $a(u,v)$ et une forme linéaire $L(v)$.
    
    \ifthenelse{\boolean{showSolutions}}{
    \begin{solution}
        On multiplie par une fonction test $v$ nulle aux bords et on intègre sur $[0,1]$ :
        $$\int_0^1 (-u'' + u) v \, dx = \int_0^1 f v \, dx$$
        
        Intégration par parties sur le terme $-u''v$ :
        $$\int_0^1 u' v' \, dx - [u'v]_0^1 + \int_0^1 uv \, dx = \int_0^1 fv \, dx$$
        
        Comme $v(0) = v(1) = 0$, le terme de bord s'annule.
        
        On obtient : $a(u,v) = L(v)$ avec :
        $$a(u,v) = \int_0^1 u'v' \, dx + \int_0^1 uv \, dx$$
        $$L(v) = \int_0^1 fv \, dx$$
    \end{solution}
    }{}
    
    \item Définissez un maillage avec $2$ éléments finis de longueur $h = 1/2$ et d'ordre $1$. Donnez les noeuds $x_0, x_1, x_2$.
    
    \ifthenelse{\boolean{showSolutions}}{
    \begin{solution}
        Les noeuds sont : $x_0 = 0$, $x_1 = 1/2$, $x_2 = 1$.
        
        Élément 1 : $[0, 1/2]$, Élément 2 : $[1/2, 1]$.
    \end{solution}
    }{}
    
    \item Donnez les polynômes de Lagrange $\varphi_0$, $\varphi_1$ et $\varphi_2$ associés à ce maillage.
    
    \ifthenelse{\boolean{showSolutions}}{
    \begin{solution}
        Sur l'élément $[0, h]$ : $\varphi_0(x) = 1 - x/h$, $\varphi_1(x) = x/h$
        
        Sur l'élément $[h, 2h]$ : $\varphi_1(x) = 1 - (x-h)/h$, $\varphi_2(x) = (x-h)/h$
        
        Avec $h = 1/2$ :
        \begin{itemize}
            \item $\varphi_0(x) = 1 - 2x$ sur $[0, 1/2]$, $0$ ailleurs
            \item $\varphi_1(x) = 2x$ sur $[0, 1/2]$ et $\varphi_1(x) = 2 - 2x$ sur $[1/2, 1]$
            \item $\varphi_2(x) = 2x - 1$ sur $[1/2, 1]$, $0$ ailleurs
        \end{itemize}
    \end{solution}
    }{}
    
    \item Calculez la matrice de rigidité élémentaire $K_e$ sur le premier élément $[0, h]$.
    On donne : $\int_0^h (1-x/h)^2 dx = h/3$ et $\int_0^h (x/h)(1-x/h) dx = h/6$.
    

    \ifthenelse{\boolean{showSolutions}}{
    \begin{solution}
        La matrice de rigidité élémentaire est :
        $$K_e = \begin{pmatrix} a(\varphi_0, \varphi_0) & a(\varphi_0, \varphi_1) \\ a(\varphi_1, \varphi_0) & a(\varphi_1, \varphi_1) \end{pmatrix}$$
        
        Avec $a(u,v) = \int_0^h u'v' dx + \int_0^h uv dx$.
        
        Calcul des dérivées : $\varphi_0' = -1/h = -2$, $\varphi_1' = 1/h = 2$.
        
        Terme de rigidité (dérivées) : $\int_0^h \varphi_0' \varphi_0' dx = 4h = 2$, etc.
        
        Terme de masse : $\int_0^h \varphi_0^2 dx = h/3 = 1/6$, $\int_0^h \varphi_0\varphi_1 dx = h/6 = 1/12$.
        
        $$K_e = \frac{1}{h}\begin{pmatrix} 1 & -1 \\ -1 & 1 \end{pmatrix} + \frac{h}{6}\begin{pmatrix} 2 & 1 \\ 1 & 2 \end{pmatrix} = \begin{pmatrix} 2 + 1/12 & -2 + 1/24 \\ -2 + 1/24 & 2 + 1/12 \end{pmatrix}$$
        
        $$K_e = \begin{pmatrix} 25/12 & -47/24 \\ -47/24 & 25/12 \end{pmatrix}$$
    \end{solution}
    }{}
    
    \item Après l'avoir brièvement expliqué, on pourra utiliser le fait que les deux matrices de rigidité locales sont identiques.\newline 
    Établissez la matrice de rigidité globale $K$, puis appliquez les conditions aux limites pour obtenir le système réduit.
    
    \ifthenelse{\boolean{showSolutions}}{
    \begin{solution}
        On peut passer des intégrales sur le premier élément aux intégrales sur le second élément en utilisant la symétrie des fonctions $\varphi_i$. Le changement de variable est $u=1-x$.

        Par assemblage, avec les deux éléments identiques :
        $$K = \begin{pmatrix} 25/12 & -47/24 & 0 \\ -47/24 & 25/6 & -47/24 \\ 0 & -47/24 & 25/12 \end{pmatrix}$$
        
        Avec $u(0) = u(1) = 0$, donc $\alpha_0 = \alpha_2 = 0$, le système réduit est :
        $$\frac{25}{6} \alpha_1 = L(\varphi_1)$$
    \end{solution}
    }{}
    
    \item En prenant $f(x) = 1$, calculez le second membre $L(\varphi_1)$ et donnez la valeur de la solution approchée au noeud $x_1$.
    
    \ifthenelse{\boolean{showSolutions}}{
    \begin{solution}
        $$L(\varphi_1) = \int_0^1 f \varphi_1 dx = \int_0^{1/2} 2x \, dx + \int_{1/2}^1 (2-2x) dx$$
        $$= [x^2]_0^{1/2} + [2x - x^2]_{1/2}^1 = 1/4 + (2-1) - (1 - 1/4) = 1/4 + 1 - 3/4 = 1/2$$
        
        Donc : $\frac{25}{6} \alpha_1 = \frac{1}{2}$
        
        $$\alpha_1 = \frac{3}{25} = 0.12$$
    \end{solution}
    }{}
    
    \item La méthode des éléments finis a-t-elle plutôt tendance à sur-estimer ou à sous-estimer les contraintes appliquées sur une pièce ? Illustrez votre réponse avec un exemple.
    
    \ifthenelse{\boolean{showSolutions}}{
    \begin{solution}
        La méthode des éléments finis a tendance à sous-estimer les contraintes appliquées sur une pièce. Voir l'exemple du TD1.
    \end{solution}
    }{}
\end{enumerate}
\newpage

\begin{center}
    \renewcommand{\arraystretch}{1.5} 
    \begin{tabular}{|>{\centering\arraybackslash}m{4cm}|>{\centering\arraybackslash}m{6cm}|>{\centering\arraybackslash}m{4cm}|}
        \hline
            \hspace{4cm}&\hspace{6cm} & \textbf{Page 2/2}\\
            \hline
    \end{tabular}
\end{center}

\section*{Exercice 1 : Polynômes de Lagrange d'ordre 2}
On considère une barre de longueur $L$, découpée en $n$ éléments finis de longueur $h = L/n$.

On souhaite améliorer la précision de la méthode des éléments finis en utilisant des éléments d'ordre $2$ : on ajoute un noeud intermédiaire au milieu de chaque élément.

Considérons le premier élément $[0, h]$ avec trois noeuds : $x_0$, $x_{1/2}$ et $x_1$.


\begin{enumerate}
    \item Rappelez les conditions que doivent vérifier les polynômes de Lagrange $\varphi_0$, $\varphi_{1/2}$ et $\varphi_1$ pour interpoler correctement la solution aux noeuds.
    
    \ifthenelse{\boolean{showSolutions}}{
    \begin{solution}
        Les polynômes de Lagrange doivent vérifier la propriété d'interpolation :
        $$\varphi_i(x_j) = \delta_{ij}$$
        C'est-à-dire que $\varphi_i$ vaut $1$ au noeud $x_i$ et $0$ aux autres noeuds.
        
        Plus précisément :
        \begin{itemize}
            \item $\varphi_0(0) = 1$, $\varphi_0(h/2) = 0$, $\varphi_0(h) = 0$
            \item $\varphi_{1/2}(0) = 0$, $\varphi_{1/2}(h/2) = 1$, $\varphi_{1/2}(h) = 0$
            \item $\varphi_1(0) = 0$, $\varphi_1(h/2) = 0$, $\varphi_1(h) = 1$
        \end{itemize}
    \end{solution}
    }{}
    
    \item Voici les polynômes. Déterminez les éléments manquants $a_1, a_2, a_3, a_4, a_5, a_6$ en justifiant votre choix :
    \begin{align*}
    \varphi_0(x) &= a_1 \frac{(x-h/2)(x-h)}{h^2} \\
    \varphi_{1/2}(x) &= a_2 \frac{x(x-a_3)}{h^2} \\
    \varphi_1(x) &= a_4 \frac{x(x-a_5)}{a_6}
    \end{align*}

    \ifthenelse{\boolean{showSolutions}}{
    \begin{solution}
        En utilisant la formule de Lagrange et les conditions d'interpolation :
        
        \textbf{Pour $\varphi_0$ :} On doit avoir $\varphi_0(0) = 1$.
        $$\varphi_0(0) = a_1 \frac{(0-h/2)(0-h)}{h^2} = a_1 \frac{h^2/2}{h^2} = \frac{a_1}{2} = 1$$
        Donc $a_1 = 2$.
        
        \textbf{Pour $\varphi_{1/2}$ :} On doit avoir $\varphi_{1/2}(h/2) = 1$ et $\varphi_{1/2}(h) = 0$.
        La condition $\varphi_{1/2}(h) = 0$ impose $a_3 = h$.
        $$\varphi_{1/2}(h/2) = a_2 \frac{(h/2)(h/2-h)}{h^2} = a_2 \frac{(h/2)(-h/2)}{h^2} = -\frac{a_2}{4} = 1$$
        Donc $a_2 = -4$.
        
        \textbf{Pour $\varphi_1$ :} On doit avoir $\varphi_1(h) = 1$ et $\varphi_1(h/2) = 0$.
        La condition $\varphi_1(h/2) = 0$ impose $a_5 = h/2$.
        $$\varphi_1(h) = a_4 \frac{h(h-h/2)}{a_6} = a_4 \frac{h^2/2}{a_6} = 1$$
        En normalisant avec $a_6 = h^2$, on obtient $a_4 = 2$.
        
        \textbf{Réponses :} $a_1 = 2$, $a_2 = -4$, $a_3 = h$, $a_4 = 2$, $a_5 = h/2$, $a_6 = h^2$.
    \end{solution}
    }{}
    
    \item Quelles sont les dimensions de la matrice de rigidité élémentaire pour cet élément d'ordre $2$ ?\newline
    Quelles sont les dimensions de la matrice de rigidité globale ?
    
    \ifthenelse{\boolean{showSolutions}}{
    \begin{solution}
        La matrice de rigidité élémentaire est de taille $3 \times 3$.

        La matrice de rigidité globale est de taille $2n +1 \times 2n +1$.
    \end{solution}
    }{}
    
    \item Donnez l'expression générale de la matrice de rigidité élémentaire pour l'équation $-u'' = f$.
    
    \ifthenelse{\boolean{showSolutions}}{
    \begin{solution}
        La matrice de rigidité élémentaire est :
        $$K_e = \begin{pmatrix}
            a(\varphi_0, \varphi_0) & a(\varphi_0, \varphi_{1/2}) & a(\varphi_0, \varphi_1) \\
            a(\varphi_{1/2}, \varphi_0) & a(\varphi_{1/2}, \varphi_{1/2}) & a(\varphi_{1/2}, \varphi_1) \\
            a(\varphi_1, \varphi_0) & a(\varphi_1, \varphi_{1/2}) & a(\varphi_1, \varphi_1)
        \end{pmatrix}$$
        où $a(u,v) = \int_0^h u'(x) v'(x) \, dx$.
    \end{solution}
    }{}
\end{enumerate}

\vspace{1em}

\section*{Exercice 2 : Maillage 2D}
Voici une pièce en forme de \og L \fg{} qu'on souhaite mailler pour une analyse par éléments finis.

\begin{center}
\begin{tikzpicture}[scale=1.2]
    % Contour de la pièce en L
    \draw[thick] (0,0) -- (3,0) -- (3,1) -- (1,1) -- (1,3) -- (0,3) -- cycle;
    
    % Cotations
    \draw[<->] (0,-0.4) -- (3,-0.4) node[midway,below] {3 m};
    \draw[<->] (-0.4,0) -- (-0.4,3) node[midway,left] {3 m};
    \draw[<->] (1,3.3) -- (0,3.3) node[midway,above] {1 m};
    \draw[<->] (3.3,0) -- (3.3,1) node[midway,right] {1 m};
    
    % Grille en pointillés
    \draw[help lines,gray!40] (0,0) grid (3,3);
    
    % Points suggérés
    \foreach \x in {0,1,2,3} {
        \foreach \y in {0,1,2,3} {
            \fill[gray] (\x,\y) circle (0.03);
        }
    }
\end{tikzpicture}
\end{center}

\begin{enumerate}
    \item Proposez un maillage de cette pièce en utilisant des triangles. Numérotez clairement les noeuds et les éléments sur votre dessin.
    

    \ifthenelse{\boolean{showSolutions}}{
    \begin{solution}
        Une solution possible avec 8 triangles et 7 noeuds :
        
        Noeuds : 1=(0,0), 2=(1,0), 3=(2,0), 4=(3,0), 5=(1,1), 6=(2,1), 7=(3,1), 8=(0,1), 9=(1,2), 10=(0,2), 11=(1,3), 12=(0,3)
        
        Éléments triangulaires reliant les noeuds appropriés.
    \end{solution}
    }{}
    
    \item Établissez le tableau de coordonnées des noeuds et le tableau de connectivité des éléments.
    
    \ifthenelse{\boolean{showSolutions}}{
    \begin{solution}
        Tableaux dépendant du maillage choisi par l'étudiant.
    \end{solution}
    }{}
    
    \item Pour un triangle de sommets $A=(x_A, y_A)$, $B=(x_B, y_B)$, $C=(x_C, y_C)$, rappelez la méthode pour obtenir les polynômes de Lagrange $\varphi_A(x,y)$, $\varphi_B(x,y)$, $\varphi_C(x,y)$.
    
    \ifthenelse{\boolean{showSolutions}}{
    \begin{solution}
        On utilise l'élément de référence (triangle $A'=(0,0)$, $B'=(1,0)$, $C'=(0,1)$) avec les polynômes :
        $$\varphi'_A(r,s) = 1-r-s, \quad \varphi'_B(r,s) = r, \quad \varphi'_C(r,s) = s$$
        
        Puis on effectue un changement de variables affine pour passer au triangle réel :
        $$\begin{pmatrix} x \\ y \end{pmatrix} = \begin{pmatrix} x_A \\ y_A \end{pmatrix} + (x_B-x_A) r + (x_C-x_A) s$$
        
        On inverse ce changement pour exprimer $(r,s)$ en fonction de $(x,y)$, puis on substitue dans les polynômes de référence.
    \end{solution}
    }{}
    
    \item Expliquez brièvement comment est construite la matrice de rigidité globale à partir des matrices élémentaires.
    
    \ifthenelse{\boolean{showSolutions}}{
    \begin{solution}
        La matrice globale est construite par assemblage :
        \begin{enumerate}
            \item Initialiser une matrice $N \times N$ (N = nombre de noeuds) remplie de zéros
            \item Pour chaque élément, calculer sa matrice élémentaire $K_e$
            \item Ajouter les coefficients de $K_e$ dans la matrice globale aux positions correspondant aux indices globaux des noeuds de l'élément
            \item Les contributions des éléments partageant des noeuds communs s'additionnent
        \end{enumerate}
    \end{solution}
    }{}
\end{enumerate}




\end{document}