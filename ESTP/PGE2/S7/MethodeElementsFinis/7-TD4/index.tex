\documentclass[11pt,a4paper]{report}

% -------------------- Encodage & langue --------------------
\usepackage[T1]{fontenc}
\usepackage[utf8]{inputenc}
\usepackage[french]{babel}
\usepackage{lmodern}
\usepackage{microtype}
\usepackage{amsmath, amssymb}
\usepackage{multicol}
\usepackage{enumitem}

\usepackage{amsfonts}
\usepackage[version=4]{mhchem}
\usepackage{stmaryrd}
\usepackage{graphicx}
\usepackage[export]{adjustbox}
\usepackage{caption}
\usepackage{multirow}
\usepackage{tikz}
% -------------------- Mise en page --------------------------
\usepackage[a4paper,margin=2cm]{geometry}
\usepackage{fancyhdr}
\usepackage{parskip}      % espace entre paragraphes
\setlength{\parindent}{0pt}

% -------------------- Couleurs & liens ----------------------
\usepackage{xcolor}
\definecolor{Theme}{HTML}{0E7490} % teal-700
\definecolor{ThemeLight}{HTML}{E0F2F1}
\definecolor{Accent}{HTML}{F59E0B} % amber-500
\definecolor{Gray}{HTML}{374151}
\usepackage[colorlinks=true,linkcolor=Theme,urlcolor=Theme,citecolor=Theme]{hyperref}

% -------------------- Graphiques / décor --------------------
\usepackage{tikz}
\usetikzlibrary{patterns,positioning,calc}
\usepackage{graphicx}
\usepackage{tcolorbox}
\tcbuselibrary{skins,breakable,hooks,most}
\usepackage{fontawesome5}

% -------------------- Titres -------------------------------
\usepackage{titlesec}
\titleformat{\chapter}[display]
  {\Huge\bfseries\color{Theme}}
  {\filright\rule{0.75\linewidth}{1.2pt}\\[3pt]{Algèbre linéaire - Chapitre~\thechapter}}
  {0.2ex}
  {\filright}
  [\vspace{0.1ex}\rule{0.35\linewidth}{1.2pt}]

\titleformat{\section}
  {\Large\bfseries\color{Gray}}
  {\thesection}{0.6em}{}

% -------------------- En-têtes / pieds ---------------------
\pagestyle{fancy}
\fancyhf{}
\fancyhead[L]{\color{Gray}\leftmark}
\fancyhead[R]{\color{Gray}\textit{MEF - 2025/2026}}
\fancyfoot[R]{\color{Gray}\small p.\ \thepage}
\renewcommand{\headrulewidth}{0pt}
\renewcommand{\footrulewidth}{0pt}

% -------------------- Macros utilitaires -------------------
\newenvironment{solution}
{
    \vspace{0.5em}
    \begin{mdframed}[backgroundcolor=ThemeLight,leftmargin=0,rightmargin=0,skipabove=0.2em,skipbelow=0.2em]
    \textbf{Solution.}\\[0.5em]
}
{
    \end{mdframed}
    \vspace{0.5em}
}



% -------------------- Page de titre ------------------------
\title{\textbf{Traces de cours}\\\large (résumés, formules, exemples, mini-exercices)}
\author{ MEF - 2025/2026 }
\date{\today}


\makeatletter
\renewcommand{\thesubsection}{\arabic{subsection}}
\renewcommand{\p@subsection}{}% supprime le préfixe section/chapter dans \ref
% Si vous voulez la même chose pour les sous-sous-sections :
% \renewcommand{\thesubsubsection}{\arabic{subsubsection}}
% \renewcommand{\p@subsubsection}{}
\makeatother

\usepackage{mdframed}
\usepackage{ifthen}

% \usepackage[sf]{titlesec}
% Définition de la variable pour afficher les corrections
\newboolean{showSolutions}
% Décommentez la ligne suivante pour afficher les solutions
\input \jobname.adr
% -------------------- Document ----------------------------
\begin{document}

\begin{center}
    {\LARGE \textbf{Méthode des éléments finis -- TD4}}\\[1em]
    {\large \textit{Des éléments 2D}}
\end{center}


En dimension 2, les éléments finis de Lagrange sont des triangles et les fonctions de forme sont des polynômes à deux variables.
Pour des éléments d'ordre $1$ (les seuls points du maillage sont les sommets du triangle), il faut $3$ polynômes par éléments.

La première section vise à obtenir des formules pour calculer les expressions des polynômes sur chaque élément. 

\section*{Préliminaires : élément de réference}

Dans un repère $(O, r, s)$, prenons un triangle rectangle isocèle de côté $1$ et de sommets $A = (0,0)$, $B = (1,0)$, $C = (0,1)$, c'est notre élément de référence. 
\begin{enumerate}[label=\arabic*.]
\item Pour ce triangle, rappelez ou retrouvez les polynômes $\varphi_A(r,s)$, $\varphi_B(r,s)$, $\varphi_C(r,s)$ tels que
\[
\varphi_A(r_A,s_A) = 1, \quad \varphi_A(r_B,s_B) = 0, \quad \varphi_A(r_C,s_C) = 0.
\]
\[
\varphi_B(r_A,s_A) = 0, \quad \varphi_B(r_B,s_B) = 1, \quad \varphi_B(r_C,s_C) = 0.
\]
\[
\varphi_C(r_A,s_A) = 0, \quad \varphi_C(r_B,s_B) = 0, \quad \varphi_C(r_C,s_C) = 1.
\]

\ifthenelse{\boolean{showSolutions}}{
\begin{solution}
   \textbf{Rappel de cours :} Pour un élément fini de Lagrange d'ordre 1 en dimension 2, les fonctions de forme $\varphi_i$ sont des polynômes de degré 1 qui vérifient la propriété d'interpolation de Lagrange : $\varphi_i$ vaut 1 au noeud $i$ et 0 aux autres noeuds. Pour un triangle avec 3 sommets, on a besoin de 3 polynômes linéaires indépendants.
   
   \textbf{Stratégie :} On cherche des polynômes de la forme $\varphi(r,s) = \alpha + \beta r + \gamma s$ qui vérifient les conditions d'interpolation. Pour $\varphi_A$, on doit avoir $\varphi_A(0,0) = 1$, $\varphi_A(1,0) = 0$ et $\varphi_A(0,1) = 0$.
   
   \textbf{Calcul :} Pour $\varphi_A(r,s) = \alpha + \beta r + \gamma s$, les conditions donnent :
   \begin{itemize}
       \item $\varphi_A(0,0) = \alpha = 1$
       \item $\varphi_A(1,0) = \alpha + \beta = 1 + \beta = 0$ donc $\beta = -1$
       \item $\varphi_A(0,1) = \alpha + \gamma = 1 + \gamma = 0$ donc $\gamma = -1$
   \end{itemize}
   Ainsi $\varphi_A(r,s) = 1 - r - s$.
   
   De même, pour $\varphi_B$ : $\varphi_B(0,0) = 0$, $\varphi_B(1,0) = 1$, $\varphi_B(0,1) = 0$ donne $\varphi_B(r,s) = r$.
   
   Pour $\varphi_C$ : $\varphi_C(0,0) = 0$, $\varphi_C(1,0) = 0$, $\varphi_C(0,1) = 1$ donne $\varphi_C(r,s) = s$.
   
   Les polynômes sont donc :
   \[
   \varphi_A(r,s) = 1 - r - s, \quad \varphi_B(r,s) = r, \quad \varphi_C(r,s) = s.
   \]
\end{solution}
}{} 
\vspace{1em}

Plaçons-nous maintenant dans un autre repère orthonormé $(O, x, y)$ et considérons un triangle $T$ quelconque, de sommets $A=(x_A,y_A)$, $B=(x_B,y_B)$, $C=(x_C,y_C)$.
\item Faites un dessin.
\ifthenelse{\boolean{showSolutions}}{
    \begin{solution}
        \begin{tikzpicture}[scale=1.5]
        Par exemple :
    % Axes
    \draw[->] (-1,0) -- (3,0) node[right] {$x$};
    \draw[->] (0,-1) -- (0,3) node[above] {$y$};
    
    % Triangle
    \coordinate (A) at (1,1);
    \coordinate (B) at (2.5,0.5);
    \coordinate (C) at (0.5,2);
    \draw[thick] (A) -- (B) -- (C) -- cycle;
    
    % Points labels
    \node[below left] at (A) {$A(x_A,y_A)$};
    \node[below right] at (B) {$B(x_B,y_B)$};
    \node[above] at (C) {$C(x_C,y_C)$};
    
    % Points
    \fill (A) circle (1.5pt);
    \fill (B) circle (1.5pt);
    \fill (C) circle (1.5pt);
\end{tikzpicture}
\end{solution}
}{} 
Nous allons définir un nouveau repère $(A, r, s)$ lié au triangle $T$ de la façon suivante :
\begin{itemize}
    \item $A$ est l'origine,
    \item $r$ est le vecteur $\overrightarrow{AB}$,
    \item $s$ est le vecteur $\overrightarrow{AC}$.
\end{itemize}
\item Donner les coordonnées de $A$, $B$ et $C$ dans ce repère. et déduisez-en les expressions des polynômes $\varphi_A(r,s)$, $\varphi_B(r,s)$, $\varphi_C(r,s)$ dans ce repère.

\ifthenelse{\boolean{showSolutions}}{
\begin{solution}
   \textbf{Rappel de cours :} L'idée fondamentale de la méthode des éléments finis est de travailler sur un élément de référence (géométrie simple) puis d'utiliser un changement de variables pour passer à un élément quelconque. Cela permet de réutiliser les mêmes fonctions de forme pour tous les éléments du maillage.
   
   \textbf{Stratégie :} Le repère $(A, r, s)$ est défini de telle sorte que :
   \begin{itemize}
       \item $A$ est l'origine, donc ses coordonnées sont $(0,0)$
       \item $r$ est le vecteur $\overrightarrow{AB}$, donc $B$ a pour coordonnées $(1,0)$ dans ce repère
       \item $s$ est le vecteur $\overrightarrow{AC}$, donc $C$ a pour coordonnées $(0,1)$ dans ce repère
   \end{itemize}
   
   \textbf{Calcul :} Dans ce repère, les coordonnées de $A$, $B$ et $C$ sont respectivement $(0,0)$, $(1,0)$ et $(0,1)$. 

   On retrouve exactement notre élément de référence (triangle rectangle isocèle de côté 1) ! Cela signifie que les polynômes de forme sont identiques à ceux calculés précédemment.

   Les polynômes sont donc :
   \[
   \varphi_A(r,s) = 1 - r - s, \quad \varphi_B(r,s) = r, \quad \varphi_C(r,s) = s.
   \]
   
   \textbf{Remarque importante :} Cette construction montre que tout triangle peut être ramené à l'élément de référence par un simple changement de repère affine. C'est la clé de la méthode des éléments finis.
\end{solution}
}{} 
Notre but est maintenant de déterminer les polynômes $\varphi_A(x,y)$, $\varphi_B(x,y)$, $\varphi_C(x,y)$ dans le repère $(O, x, y)$. 
Pour cela, nous cherchons le changement de base sous la forme
\[
\begin{pmatrix} x \\ y \end{pmatrix} = \begin{pmatrix} a_1 \\ b_1 \end{pmatrix} + \begin{pmatrix} a_2 \\ b_2 \end{pmatrix} r + \begin{pmatrix} a_3 \\ b_3 \end{pmatrix} s.
\]
où $(a_1, b_1)$, $(a_2, b_2)$ et $(a_3, b_3)$ sont des constantes que nous allons déterminer.

\item Déterminez les constantes $(a_1, b_1)$, $(a_2, b_2)$ et $(a_3, b_3)$ en fonction des coordonnées de $A$, $B$ et $C$ dans les deux repères.

\ifthenelse{\boolean{showSolutions}}{
\begin{solution}
   \textbf{Rappel de cours :} Pour passer de l'élément de référence au triangle réel, on utilise un changement de variables affine. Ce changement de variables permet d'exprimer les coordonnées $(x,y)$ dans le repère global en fonction des coordonnées $(r,s)$ dans le repère de l'élément de référence.
   
   \textbf{Stratégie :} On cherche à déterminer les constantes $(a_i, b_i)$ en utilisant le fait que l'on connaît les coordonnées des trois sommets dans les deux repères :
   \begin{itemize}
       \item Dans le repère $(A, r, s)$ : $A = (0,0)$, $B = (1,0)$, $C = (0,1)$
       \item Dans le repère $(O, x, y)$ : $A = (x_A, y_A)$, $B = (x_B, y_B)$, $C = (x_C, y_C)$
   \end{itemize}
   On substitue ces valeurs dans la formule de changement de base pour obtenir un système d'équations.
   
   \textbf{Calcul :} Pour le sommet $A$ : $(r,s) = (0,0)$, donc :
   \[
   x_A = a_1 + a_2 \cdot 0 + a_3 \cdot 0 = a_1, \quad y_A = b_1 + b_2 \cdot 0 + b_3 \cdot 0 = b_1.
   \]
   
   Pour le sommet $B$ : $(r,s) = (1,0)$, donc :
   \[
   x_B = a_1 + a_2 \cdot 1 + a_3 \cdot 0 = a_1 + a_2, \quad y_B = b_1 + b_2 \cdot 1 + b_3 \cdot 0 = b_1 + b_2.
   \]
   
   Pour le sommet $C$ : $(r,s) = (0,1)$, donc :
   \[
   x_C = a_1 + a_2 \cdot 0 + a_3 \cdot 1 = a_1 + a_3, \quad y_C = b_1 + b_2 \cdot 0 + b_3 \cdot 1 = b_1 + b_3.
   \]
   
   On en déduit :
   \[
   a_1 = x_A, \quad a_2 = x_B - x_A, \quad a_3 = x_C - x_A.
   \]
   De même pour les ordonnées :
   \[
   b_1 = y_A, \quad b_2 = y_B - y_A, \quad b_3 = y_C - y_A.
   \]
   
   \textbf{Interprétation :} Les coefficients $a_2, b_2$ et $a_3, b_3$ représentent respectivement les composantes des vecteurs $\overrightarrow{AB}$ et $\overrightarrow{AC}$ dans le repère global, ce qui est cohérent avec la définition du repère $(A, r, s)$.

\end{solution}
}{} 
\item Déterminez les expressions des polynômes $\varphi_A(x,y)$, $\varphi_B(x,y)$, $\varphi_C(x,y)$ dans le cas où $A = (1,0)$, $B = (2,0)$, $C = (1,1)$.

\ifthenelse{\boolean{showSolutions}}{
\begin{solution}
   \textbf{Rappel de cours :} Une fois le changement de variables déterminé, on obtient les fonctions de forme dans le repère global en composant les fonctions de forme de l'élément de référence avec le changement de variables inverse. Si $(r,s) = T^{-1}(x,y)$, alors $\varphi_i(x,y) = \varphi'_i(r,s) = \varphi'_i(T^{-1}(x,y))$.
   
   \textbf{Stratégie :} 
   \begin{enumerate}
       \item D'abord, déterminer le changement de variables direct $(x,y) = T(r,s)$ en utilisant les formules précédentes
       \item Ensuite, inverser ce changement pour obtenir $(r,s) = T^{-1}(x,y)$
       \item Enfin, substituer dans les polynômes de référence $\varphi'_i(r,s)$
   \end{enumerate}
   
   \textbf{Calcul :} Avec $A = (1,0)$, $B = (2,0)$, $C = (1,1)$, on applique les formules :
   \begin{align*}
   a_1 &= x_A = 1, \quad a_2 = x_B - x_A = 2 - 1 = 1, \quad a_3 = x_C - x_A = 1 - 1 = 0 \\
   b_1 &= y_A = 0, \quad b_2 = y_B - y_A = 0 - 0 = 0, \quad b_3 = y_C - y_A = 1 - 0 = 1
   \end{align*}
   
   Le changement de variables s'écrit donc :
   \[
   \left\{
   \begin{aligned}
   x &= 1 + 1 \cdot r + 0 \cdot s = 1+r \\
   y &= 0 + 0 \cdot r + 1 \cdot s = s
   \end{aligned}
   \right.
   \] 
   
   Pour obtenir le changement inverse, on résout ce système :
   \[
   \left\{
   \begin{aligned}
   r &= x-1 \\
   s &= y
   \end{aligned}
   \right.
   \]
   
   On en déduit, en notant $\varphi'_A(r,s)$, $\varphi'_B(r,s)$ et $\varphi'_C(r,s)$ les polynômes de référence :
   \begin{align*}
   \varphi_A(x,y) &= \varphi'_A(r,s) = \varphi'_A(x-1,y) = 1 - (x-1) - y = 2-x-y, \\
   \varphi_B(x,y) &= \varphi'_B(r,s) = \varphi'_B(x-1,y) = x-1, \\
   \varphi_C(x,y) &= \varphi'_C(r,s) = \varphi'_C(x-1,y) = y.
   \end{align*}
   
   \textbf{Vérification :} On peut vérifier que $\varphi_A(1,0) = 1$, $\varphi_A(2,0) = 0$, $\varphi_A(1,1) = 0$, ce qui confirme que les conditions d'interpolation sont bien satisfaites.
\end{solution}
}{} 
\item Faites de même dans le cas où $A = (2,2)$, $B = (1,2)$, $C = (2,1)$.

\ifthenelse{\boolean{showSolutions}}{
\begin{solution}
   \textbf{Stratégie :} On applique la même méthode que précédemment. Notons que dans ce cas, les vecteurs $\overrightarrow{AB}$ et $\overrightarrow{AC}$ pointent dans des directions opposées aux axes, ce qui donnera des signes négatifs.
   
   \textbf{Calcul :} Avec $A = (2,2)$, $B = (1,2)$, $C = (2,1)$, on a :
   \begin{align*}
   a_1 &= x_A = 2, \quad a_2 = x_B - x_A = 1 - 2 = -1, \quad a_3 = x_C - x_A = 2 - 2 = 0 \\
   b_1 &= y_A = 2, \quad b_2 = y_B - y_A = 2 - 2 = 0, \quad b_3 = y_C - y_A = 1 - 2 = -1
   \end{align*}
   
   Le changement de variables s'écrit :
   \[
   \left\{
   \begin{aligned}
   x &= 2 - r \\
   y &= 2 - s
   \end{aligned}
   \right.
   \]
   
   Le changement inverse est donc :
   \[
   \left\{
   \begin{aligned}
   r &= 2-x \\
   s &= 2-y
   \end{aligned}
   \right.
   \]
   
   On en déduit les polynômes dans le repère global :
   \begin{align*}
   \varphi_A(x,y) &= \varphi'_A(r,s) = \varphi'_A(2-x,2-y) = 1 - (2-x) - (2-y) = x+y-3, \\
   \varphi_B(x,y) &= \varphi'_B(r,s) = \varphi'_B(2-x,2-y) = 2-x, \\
   \varphi_C(x,y) &= \varphi'_C(r,s) = \varphi'_C(2-x,2-y) = 2-y.
   \end{align*}
   
   \textbf{Remarque :} Ici, le triangle est orienté différemment (les vecteurs $\overrightarrow{AB}$ et $\overrightarrow{AC}$ pointent vers les valeurs décroissantes), mais la méthode reste identique. On peut vérifier que $\varphi_A(2,2) = 1$, $\varphi_A(1,2) = 0$, $\varphi_A(2,1) = 0$.
\end{solution}
}{} 
\end{enumerate}

\section*{Une équation différentielle}

Considérons l'équation différentielle suivante sur le domaine carré $\Omega = [0,1] \times [0,1]$ :
\[
-\Delta u = f \quad \text{dans } \Omega,
\]
avec les conditions aux limites de Dirichlet :
\[
u = 0 \quad \text{sur } \partial\Omega.
\]


\section*{Exercice 1 : des triangles rectangle isocèle}


\begin{enumerate}[label=\arabic*.]
    \item \textbf{Formulation Variationnelle} : Donnez la formulation variationnelle du problème. On pourra utiliser la formule de Green, généralisation de l'intégration par parties en dimension 2 :
    \[
        \int_{\Omega} \nabla u \cdot \nabla v \, d \Omega=-\int_{\Omega} v \Delta u \, d \Omega+\int_{\partial \Omega} v \frac{\partial u}{\partial n} \, d S,
    \] 
    où $\partial \Omega$ est la frontière du domaine $\Omega$.

    \ifthenelse{\boolean{showSolutions}}{
    
    
        \begin{solution}
           \textbf{Rappel de cours :} La méthode des éléments finis consiste à résoudre une formulation variationnelle (ou faible) du problème plutôt que l'équation différentielle directement. Cette approche permet de travailler avec des fonctions moins régulières et de construire une approximation par éléments finis.
           
           \textbf{Stratégie :} Pour obtenir la formulation variationnelle :
           \begin{enumerate}
               \item Multiplier l'équation différentielle par une fonction test $v$ appartenant à un espace de fonctions adéquat
               \item Intégrer sur le domaine $\Omega$
               \item Utiliser la formule de Green pour faire apparaître des conditions aux limites
               \item Choisir l'espace des fonctions tests pour satisfaire les conditions aux limites
           \end{enumerate}
           
           \textbf{Calcul :} On multiplie l'équation $-\Delta u = f$ par une fonction test $v$ et on intègre sur $\Omega$ :
           \[
           -\int_\Omega v \Delta u \, d \Omega = \int_\Omega f v \, d \Omega.
           \]
           
           \textbf{Application de la formule de Green :} La formule de Green (généralisation de l'intégration par parties en dimension 2) s'écrit :
           \[
           \int_{\Omega} \nabla u \cdot \nabla v \, d \Omega=-\int_{\Omega} v \Delta u \, d \Omega+\int_{\partial \Omega} v \frac{\partial u}{\partial n} \, d S,
           \]
           où $\frac{\partial u}{\partial n}$ est la dérivée normale de $u$ sur la frontière.
           
           En réarrangeant, on obtient :
           \[
           \int_\Omega \nabla u \cdot \nabla v \, d \Omega = -\int_{\Omega} v \Delta u \, d \Omega+\int_{\partial \Omega} v \frac{\partial u}{\partial n} \, d S.
           \]
           
           \textbf{Conditions aux limites :} Comme on a des conditions de Dirichlet homogènes $u = 0$ sur $\partial\Omega$, on choisit l'espace des fonctions tests tel que $v = 0$ sur $\partial \Omega$ également. Cela fait disparaître le terme de bord :
           \[
           \int_{\partial \Omega} v \frac{\partial u}{\partial n} \, d S = 0.
           \]
           
           On obtient donc la formulation variationnelle :
           \[
           \int_\Omega \nabla u \cdot \nabla v \, d \Omega = \int_\Omega f v \, d \Omega, \quad \forall v \text{ tel que } v = 0 \text{ sur } \partial\Omega.
           \]
           
           \textbf{Notation :} On pose traditionnellement :
           \[
           a(u,v) = \int_\Omega \nabla u \cdot \nabla v \, d \Omega, \quad L(v) = \int_\Omega f v \, d \Omega.
           \]
           La formulation variationnelle s'écrit alors : trouver $u$ tel que $a(u,v) = L(v)$ pour toute fonction test $v$.
           
           \textbf{Remarque :} La forme bilinéaire $a(\cdot,\cdot)$ est symétrique et définie positive, ce qui garantit l'existence et l'unicité de la solution.
        \end{solution}
        }{} 
    \item \textbf{Maillage en triangles rectangle isocèle} :
    \begin{enumerate}[label=\alph*.]
        \item Découpez le domaines en $4$ carrés, puis en $8$ triangles rectangle isocèles. Nommez les noeuds et les éléments correspondant.

        \ifthenelse{\boolean{showSolutions}}{
        \begin{solution}
            Une possibilité :
        \begin{center}
        \includegraphics[width=0.6\textwidth]{maillage1.jpeg}
        \end{center}

        \end{solution}
        }{} 
        \item Etablissez le tableau de coordonnées des noeuds et le tableau de connectivité des éléments.
    \end{enumerate}
    
\ifthenelse{\boolean{showSolutions}}{
    \begin{solution}
    Ce maillage est associé aux tableaux suivants : 

        Tableau de coordonnées des noeuds :
        \begin{tabular}{|c|c|c|}
            \hline
            Noeud & Coordonnées  \\
            \hline
            1 & (0,1) \\
            2 & (1/2,1) \\
            3 & (1,1) \\
            4 & (0,1/2) \\
            5 & (1/2,1/2) \\
            6 & (1,1/2) \\
            7 & (0,0) \\
            8 & (1/2,0) \\
            9 & (1,0) \\
            \hline
        \end{tabular}

        Tableau de connectivité des éléments :
        \begin{tabular}{|c|c|c|}
            \hline
            Elément & noeuds  \\
            \hline
            1 & 1, 4, 5 \\
            2 & 1, 5, 2 \\
            3 & 2, 5, 6 \\
            4 & 2, 6, 3 \\
            5 & 4, 7, 8 \\
            6 & 4, 8, 5 \\
            7 & 5, 8, 9 \\
            8 & 5, 9, 6 \\
            \hline
        \end{tabular}

    \end{solution}
    }{} 
    \item \textbf{Calcul des Coefficients du Système Linéaire} :
    \begin{enumerate}[label=\alph*.]
        \item Placez-vous dans l'élément de votre choix, et déterminer les $3$ polynômes pour cet élément. \textit{On utilisera la première partie}

        \ifthenelse{\boolean{showSolutions}}{
        \begin{solution}
           \textbf{Rappel de cours :} Pour chaque élément du maillage, on doit déterminer les fonctions de forme dans le repère global $(x,y)$. On utilise la méthode développée dans les préliminaires : on détermine d'abord le changement de variables de l'élément de référence vers l'élément réel, puis on compose les fonctions de forme de référence avec ce changement.
           
           \textbf{Stratégie :} 
           \begin{enumerate}
               \item Identifier les coordonnées des trois sommets de l'élément dans le repère global
               \item Calculer les coefficients du changement de variables $(a_i, b_i)$
               \item Exprimer le changement de variables direct puis son inverse
               \item Substituer dans les polynômes de référence
           \end{enumerate}
           
           \textbf{Calcul :} On se place sur l'élément 1. D'après le tableau de connectivité, cet élément relie les noeuds 1, 4 et 5. Les coordonnées des noeuds sont :
           \begin{itemize}
               \item $A = (0,1/2)$ est le sommet 4
               \item $B = (1/2,1/2)$ est le sommet 5
               \item $C = (0,1)$ est le sommet 1
           \end{itemize}
           
           \textbf{Détermination du changement de variables :} On applique les formules de la section préliminaire :
           \begin{align*}
           a_1 &= x_A = 0, \quad a_2 = x_B - x_A = \frac{1}{2} - 0 = \frac{1}{2}, \quad a_3 = x_C - x_A = 0 - 0 = 0 \\
           b_1 &= y_A = \frac{1}{2}, \quad b_2 = y_B - y_A = \frac{1}{2} - \frac{1}{2} = 0, \quad b_3 = y_C - y_A = 1 - \frac{1}{2} = \frac{1}{2}
           \end{align*}
           
           Le changement de repère s'écrit donc :
           \[
           \left\{
           \begin{aligned}
           x &= 0 + \frac{1}{2} \cdot r + 0 \cdot s = \frac{r}{2} \\
           y &= \frac{1}{2} + 0 \cdot r + \frac{1}{2} \cdot s = \frac{1}{2} + \frac{s}{2}
           \end{aligned}
           \right.
           \]
           
           Pour obtenir le changement inverse, on résout ce système :
           \[
           \left\{
           \begin{aligned}
           r &= 2x \\
           s &= 2y-1
           \end{aligned}
           \right.
           \]
           
           \textbf{Expression des polynômes :} En notant $\varphi'_A$, $\varphi'_B$, $\varphi'_C$ les polynômes de référence, on a :
           \begin{align*}
           \varphi_A(x, y) &= \varphi'_A(r,s) = \varphi'_A(2x, 2y-1) = 1 - 2x - (2y-1) = 2-2x-2y, \\
           \varphi_B(x, y) &= \varphi'_B(r,s) = \varphi'_B(2x, 2y-1) = 2x, \\
           \varphi_C(x, y) &= \varphi'_C(r,s) = \varphi'_C(2x, 2y-1) = 2y-1 = 2(y-1/2).
           \end{align*}
           
           \textbf{Vérification :} On peut vérifier que $\varphi_A(0,1/2) = 1$, $\varphi_A(1/2,1/2) = 0$, $\varphi_A(0,1) = 0$, ce qui confirme les conditions d'interpolation.

        \end{solution}
        }{} 
        \item Exprimez puis calculer la matrice de rigidité $K_e$ pour cet élément.

        \ifthenelse{\boolean{showSolutions}}{
        \begin{solution}
           \textbf{Rappel de cours :} La matrice de rigidité élémentaire $K_e$ est la matrice qui exprime la forme bilinéaire $a(\cdot,\cdot)$ dans la base des fonctions de forme de l'élément. Chaque coefficient $K_e[i,j]$ correspond à $a(\varphi_i, \varphi_j)$, où $\varphi_i$ et $\varphi_j$ sont les fonctions de forme associées aux noeuds de l'élément.
           
           \textbf{Stratégie :} Pour calculer la matrice de rigidité :
           \begin{enumerate}
               \item Calculer les gradients $\nabla \varphi_i$ de chaque fonction de forme
               \item Pour chaque paire $(i,j)$, calculer l'intégrale $a(\varphi_i, \varphi_j) = \int_{T} \nabla \varphi_i \cdot \nabla \varphi_j \, dx \, dy$ où $T$ est le triangle
               \item Organiser ces coefficients dans une matrice $3 \times 3$
           \end{enumerate}
           
           \textbf{Calcul des gradients :} On dérive les polynômes obtenus précédemment :
           \begin{align*}
           \varphi_A(x,y) &= 2-2x-2y \quad \Rightarrow \quad \nabla \varphi_A = \begin{pmatrix} -2 \\ -2 \end{pmatrix}, \\
           \varphi_B(x,y) &= 2x \quad \Rightarrow \quad \nabla \varphi_B = \begin{pmatrix} 2 \\ 0 \end{pmatrix}, \\
           \varphi_C(x,y) &= 2y-1 \quad \Rightarrow \quad \nabla \varphi_C = \begin{pmatrix} 0 \\ 2 \end{pmatrix}.
           \end{align*}
           
           \textbf{Calcul des coefficients :} La matrice de rigidité s'exprime :
           \[
           K_e = 
           \begin{pmatrix}
               a(\varphi_A, \varphi_A) & a(\varphi_A, \varphi_B) & a(\varphi_A, \varphi_C) \\
               a(\varphi_B, \varphi_A) & a(\varphi_B, \varphi_B) & a(\varphi_B, \varphi_C) \\
               a(\varphi_C, \varphi_A) & a(\varphi_C, \varphi_B) & a(\varphi_C, \varphi_C)
           \end{pmatrix}
           \]
           où 
           \[
           a(\varphi_i, \varphi_j) = \int_T \nabla \varphi_i \cdot \nabla \varphi_j \, dx \, dy,
           \]
           et $T$ est le triangle de sommets $A$, $B$, $C$.
           
           \textbf{Calcul détaillé :} Pour $a(\varphi_A, \varphi_A)$ :
           \begin{align*}
           a(\varphi_A, \varphi_A) &= \int_T \begin{pmatrix} -2 \\ -2 \end{pmatrix} \cdot \begin{pmatrix} -2 \\ -2 \end{pmatrix} \, dx \, dy \\
           &= \int_T (4 + 4) \, dx \, dy = 8 \int_T dx \, dy = 8 \times \text{Aire}(T).
           \end{align*}
           
           L'aire du triangle $T$ de sommets $(0,1/2)$, $(1/2,1/2)$, $(0,1)$ est :
           \[
           \text{Aire}(T) = \frac{1}{2} \left| \det \begin{pmatrix} 1/2 & 0 \\ 0 & 1/2 \end{pmatrix} \right| = \frac{1}{2} \times \frac{1}{4} = \frac{1}{8}.
           \]
           
           Donc $a(\varphi_A, \varphi_A) = 8 \times \frac{1}{8} = 1$.
           
           Pour $a(\varphi_A, \varphi_B)$ :
           \begin{align*}
           a(\varphi_A, \varphi_B) &= \int_T \begin{pmatrix} -2 \\ -2 \end{pmatrix} \cdot \begin{pmatrix} 2 \\ 0 \end{pmatrix} \, dx \, dy \\
           &= \int_T (-4 + 0) \, dx \, dy = -4 \times \frac{1}{8} = -\frac{1}{2}.
           \end{align*}
           
           De même :
           \[
           a(\varphi_A, \varphi_C) = \int_T \begin{pmatrix} -2 \\ -2 \end{pmatrix} \cdot \begin{pmatrix} 0 \\ 2 \end{pmatrix} \, dx \, dy = -4 \times \frac{1}{8} = -\frac{1}{2}.
           \]
           
           Par symétrie de la forme bilinéaire : $a(\varphi_B, \varphi_A) = a(\varphi_A, \varphi_B) = -\frac{1}{2}$ et $a(\varphi_C, \varphi_A) = a(\varphi_A, \varphi_C) = -\frac{1}{2}$.
           
           Pour $a(\varphi_B, \varphi_B)$ :
           \[
           a(\varphi_B, \varphi_B) = \int_T \begin{pmatrix} 2 \\ 0 \end{pmatrix} \cdot \begin{pmatrix} 2 \\ 0 \end{pmatrix} \, dx \, dy = 4 \times \frac{1}{8} = \frac{1}{2}.
           \]
           
           Pour $a(\varphi_C, \varphi_C)$ :
           \[
           a(\varphi_C, \varphi_C) = \int_T \begin{pmatrix} 0 \\ 2 \end{pmatrix} \cdot \begin{pmatrix} 0 \\ 2 \end{pmatrix} \, dx \, dy = 4 \times \frac{1}{8} = \frac{1}{2}.
           \]
           
           Pour $a(\varphi_B, \varphi_C)$ :
           \[
           a(\varphi_B, \varphi_C) = \int_T \begin{pmatrix} 2 \\ 0 \end{pmatrix} \cdot \begin{pmatrix} 0 \\ 2 \end{pmatrix} \, dx \, dy = 0 \times \frac{1}{8} = 0.
           \]
           
           \textbf{Matrice de rigidité :} On obtient finalement :
           \[
           K_e = \frac{1}{2} \begin{pmatrix} 2 & -1 & -1 \\ -1 & 1 & 0 \\ -1 & 0 & 1 \end{pmatrix}.
           \]
           
           \textbf{Remarque :} Cette matrice est symétrique (car $a(\varphi_i, \varphi_j) = a(\varphi_j, \varphi_i)$) et définie positive. Elle conduira au système linéaire :
           \[
           K_e \begin{pmatrix} \alpha_4 \\ \alpha_5 \\ \alpha_1 \end{pmatrix} = \begin{pmatrix} L(\varphi_A) \\ L(\varphi_B) \\ L(\varphi_C) \end{pmatrix},
           \]
           où les $\alpha_i$ sont les valeurs de la solution approchée aux noeuds et $L(\varphi_i) = \int_T f \varphi_i \, dx \, dy$.
        \end{solution}
        }{} 
        \item Choisissez un triangle qui n'est pas dans le même sens que votre premier choix, exprimez la matrice de rigidité pour ce triangle. 

        \ifthenelse{\boolean{showSolutions}}{
        \begin{solution}
           \textbf{Stratégie :} On choisit un triangle orienté différemment pour vérifier que la méthode fonctionne indépendamment de l'orientation. Le triangle 8 est un bon choix car il est dans le quadrant opposé et a une orientation différente.
           
           \textbf{Calcul :} Choisissons le triangle 8. D'après le tableau de connectivité, cet élément relie les noeuds 5, 9 et 6. Les coordonnées des noeuds sont :
           \begin{itemize}[label=$\bullet$]
            \item $A = (1,1/2)$ est le sommet 6
            \item $B = (1/2,1/2)$ est le sommet 5
            \item $C = (1,0)$ est le sommet 9
           \end{itemize}
           
           \textbf{Détermination du changement de variables :} On applique les formules :
           \begin{align*}
           a_1 &= x_A = 1, \quad a_2 = x_B - x_A = \frac{1}{2} - 1 = -\frac{1}{2}, \quad a_3 = x_C - x_A = 1 - 1 = 0 \\
           b_1 &= y_A = \frac{1}{2}, \quad b_2 = y_B - y_A = \frac{1}{2} - \frac{1}{2} = 0, \quad b_3 = y_C - y_A = 0 - \frac{1}{2} = -\frac{1}{2}
           \end{align*}
           
           Le changement de repère s'écrit donc :
           \[
           \left\{
           \begin{aligned}
           x &= 1 - \frac{1}{2} \cdot r + 0 \cdot s = 1-\frac{r}{2} \\
           y &= \frac{1}{2} + 0 \cdot r - \frac{1}{2} \cdot s = \frac{1}{2}-\frac{s}{2}
           \end{aligned}
           \right.
           \]
           
           Le changement inverse est :
           \[
           \left\{
           \begin{aligned}
           r &= 2-2x \\
           s &= 1-2y
           \end{aligned}
           \right.
           \]
           
           \textbf{Expression des polynômes :} En substituant dans les polynômes de référence :
           \begin{align*}
           \varphi_A(x, y) &= \varphi'_A(r,s) = \varphi'_A(2-2x, 1-2y) = 1 - (2-2x) - (1-2y) = 2x+2y-2, \\
           \varphi_B(x, y) &= \varphi'_B(r,s) = \varphi'_B(2-2x, 1-2y) = 2-2x, \\
           \varphi_C(x, y) &= \varphi'_C(r,s) = \varphi'_C(2-2x, 1-2y) = 1-2y.
           \end{align*}
           
           \textbf{Calcul des gradients :} On dérive :
           \[
            \nabla \varphi_A = \begin{pmatrix} 2 \\ 2 \end{pmatrix}, \quad \nabla \varphi_B = \begin{pmatrix} -2 \\ 0 \end{pmatrix}, \quad \nabla \varphi_C = \begin{pmatrix} 0 \\ -2 \end{pmatrix}.
           \]
           
           \textbf{Calcul de la matrice de rigidité :} En calculant les intégrales comme précédemment (l'aire du triangle est toujours $1/8$), on trouve :
           \[
           K_e = \frac{1}{2} \begin{pmatrix} 2 & -1 & -1 \\ -1 & 1 & 0 \\ -1 & 0 & 1 \end{pmatrix}.
           \]
           
           \textbf{Observation importante :} C'est exactement la même matrice que pour l'élément 1 ! Cela montre que la matrice de rigidité élémentaire ne dépend que de la géométrie de l'élément (aire, forme) et non de sa position ou orientation dans le maillage. Pour des triangles rectangles isocèles de même taille, la matrice est identique.
           
           Le système linéaire associé est :
           \[
           K_e \begin{pmatrix} \alpha_6 \\ \alpha_5 \\ \alpha_9 \end{pmatrix} = \begin{pmatrix} L(\varphi_A) \\ L(\varphi_B) \\ L(\varphi_C) \end{pmatrix}.
           \]
        \end{solution}
        }{} 
        \item Ecrivez le système linéaire associé à votre maillage. 

        \ifthenelse{\boolean{showSolutions}}{
            \begin{solution}
               \textbf{Rappel de cours :} La matrice de rigidité globale $K$ est obtenue en assemblant les matrices de rigidité élémentaires $K_e$ de tous les éléments du maillage. Chaque coefficient $K[i,j]$ de la matrice globale correspond à la contribution de tous les éléments qui partagent les noeuds $i$ et $j$.
               
               \textbf{Stratégie :} Pour construire la matrice globale :
               \begin{enumerate}
                   \item Initialiser une matrice $N \times N$ remplie de zéros, où $N$ est le nombre de noeuds
                   \item Pour chaque élément $e$ :
                   \begin{itemize}
                       \item Identifier les indices globaux des noeuds de l'élément
                       \item Ajouter les coefficients de $K_e$ aux positions correspondantes dans la matrice globale
                   \end{itemize}
                   \item Les coefficients se cumulent : si deux éléments partagent un même côté, leurs contributions s'additionnent
               \end{enumerate}
               
               \textbf{Calcul :} Pour obtenir la matrice de rigidité globale, il faut suivre les étapes décrites dans le cours : 
               \begin{itemize}[label = $\bullet$]
                \item On écrit une matrice pleine de 0, de la bonne taille : ici $9 \times 9$ (9 noeuds).
                \item On place les matrices de rigidité élémentaires dans les endroits adéquats en fonction de la numérotation des noeuds.
               \end{itemize}
                Par exemple, la matrice de rigidité qui prend en compte uniquement l'élément 1 est :
                \[
                K = \frac{1}{2} \begin{pmatrix} 
                 1 & 0 & 0 & 0 & -1 & 0 & 0 & 0 & 0 \\ 
                 0 & 0 & 0 & 0 & 0 & 0 & 0 & 0 & 0 \\ 
                 0 & 0 & 0 & 0 & 0 & 0 & 0 & 0 & 0 \\ 
                 0 & 0 & 0 & 2 & -1 & 0 & 0 & 0 & 0 \\ 
                 -1 & 0 & 0 & -1 & 1 & 0 & 0 & 0 & 0 \\ 
                 0 & 0 & 0 & 0 & 0 & 0 & 0 & 0 & 0 \\ 
                 0 & 0 & 0 & 0 & 0 & 0 & 0 & 0 & 0 \\ 
                 0 & 0 & 0 & 0 & 0 & 0 & 0 & 0 & 0 \\ 
                 0 & 0 & 0 & 0 & 0 & 0 & 0 & 0 & 0 
                \end{pmatrix}.
                \]
                En ajoutant celle pour l'élément 8, on obtient :
                \[
                K = \frac{1}{2} \begin{pmatrix} 
                 1 & 0 & 0 & -1 & 0 & 0 & 0 & 0 & 0 \\ 
                 0 & 0 & 0 & 0 & 0 & 0 & 0 & 0 & 0 \\ 
                 0 & 0 & 0 & 0 & 0 & 0 & 0 & 0 & 0 \\ 
                 -1 & 0 & 0 & 2 & -1 & 0 & 0 & 0 & 0 \\ 
                 0 & 0 & 0 & -1 & 2 & -1 & 0 & 0 & 0 \\ 
                 0 & 0 & 0 & 0 & -1 & 2 & 0 & 0 & -1 \\ 
                 0 & 0 & 0 & 0 & 0 & 0 & 0 & 0 & 0 \\ 
                 0 & 0 & 0 & 0 & 0 & 0 & 0 & 0 & 0 \\ 
                 0 & 0 & 0 & 0 & 0 & -1 & 0 & 0 & 1 
                \end{pmatrix}.
                \]
                Finalement, en ajoutant tous les éléments, on obtient : 
                \[
                K = \frac{1}{2} \begin{pmatrix} 
                 2 & -1 & . & -1 & . & . & . & . & . \\ 
                 -1 & 4 & -1 & . & -2 & . & . & . & . \\ 
                 . & -1 & 2 & . & . & -1 & . & . & . \\ 
                 -1 & . & . & 4 & -2 & . & -1 & . & . \\ 
                 . & -2 & . & -2 & 8 & -2 & . & -2 & . \\ 
                 . & . & -1 & . & -2 & 4 & . & . & -1 \\ 
                 . & . & . & -1 & . & . & 2 & -1 & . \\ 
                 . & . & . & . & -2 & . & -1 & 4 & -1 \\ 
                 . & . & . & . & . & -1 & . & -1 & 2 
                \end{pmatrix}.
                \]
                \textbf{Explication de l'assemblage :} 
                \begin{itemize}
                    \item L'élément 1 relie les noeuds 1, 4, 5. Les coefficients de $K_e$ sont placés aux positions correspondantes : lignes et colonnes 1, 4, 5.
                    \item L'élément 8 relie les noeuds 5, 9, 6. Ses coefficients s'ajoutent aux positions correspondantes.
                    \item Le noeud 5 apparaît dans plusieurs éléments, donc ses contributions se cumulent (par exemple, $K[5,5]$ reçoit des contributions de plusieurs éléments).
                \end{itemize}
                
                \textbf{Astuce pour vérifier rapidement :} 
                \begin{itemize}[label = $\bullet$]
                 \item Pour les coefficients diagonaux, on peut dire qu'un sommet marque 2 points s'il est sur un angle droit, 1 point sinon (car il appartient à 2 ou 4 triangles).
                 \item Pour les coefficients non diagonaux, on peut dire qu'une paire de sommets marque 1 point si c'est un côté d'un triangle, mais pas l'hypothénuse (car l'hypothénuse n'est partagée que par un triangle).
                \end{itemize}
             
             \textbf{Système linéaire final :} Le système linéaire associé est :
             \[
             K \begin{pmatrix} \alpha_1 \\ \alpha_2 \\ \alpha_3 \\ \alpha_4 \\ \alpha_5 \\ \alpha_6 \\ \alpha_7 \\ \alpha_8 \\ \alpha_9 \end{pmatrix} = \begin{pmatrix} f_1 \\ f_2 \\ f_3 \\ f_4 \\ f_5 \\ f_6 \\ f_7 \\ f_8 \\ f_9 \end{pmatrix},
             \]
             où les $\alpha_i$ sont les valeurs de la solution approchée aux noeuds et $f_i = L(\varphi_i) = \int_\Omega f \varphi_i \, dx \, dy$ sont les composantes du second membre, calculées en assemblant les contributions de chaque élément de manière analogue.

             \textbf{Remarque :} Pour les noeuds sur la frontière $\partial\Omega$, les conditions de Dirichlet $u = 0$ imposent $\alpha_i = 0$. Ces équations peuvent être supprimées du système ou imposées directement.

             \end{solution}
             }{} 

    \end{enumerate}
\end{enumerate}


\section*{Exercice 2 : des carrés}


\begin{enumerate}[label=\alph*.]
    \item \textbf{Préliminaires : élément de référence} :
    \begin{enumerate}[label=\arabic*.]
        Dans un repère $(O, r, s)$, prenons un carré de côté $2$ et de sommets $A = (-1,-1)$, $B = (1,-1)$, $C = (1,1)$, $D = (-1,1)$, c'est notre élément de référence. 

        \item Pour ce carré, rappelez ou retrouvez les polynômes $\varphi_A(r,s)$, $\varphi_B(r,s)$, $\varphi_C(r,s)$, $\varphi_D(r,s)$.
        \ifthenelse{\boolean{showSolutions}}{
\begin{solution}
   \textbf{Rappel de cours :} Pour un élément fini quadrilatéral d'ordre 1, on a 4 noeuds (les 4 sommets) et donc 4 fonctions de forme. Ces fonctions sont des polynômes bilinéaires (produits de polynômes de degré 1 en $r$ et en $s$) qui vérifient les conditions d'interpolation de Lagrange.
   
   \textbf{Stratégie :} On cherche des polynômes de la forme $\varphi(r,s) = (\alpha + \beta r)(\gamma + \delta s)$ qui valent 1 en un sommet et 0 aux trois autres. La forme factorisée est naturelle car les conditions d'interpolation se factorisent selon les directions $r$ et $s$.
   
   \textbf{Calcul :} Pour $\varphi_A$ qui doit valoir 1 en $A=(-1,-1)$ et 0 en $B=(1,-1)$, $C=(1,1)$, $D=(-1,1)$ :
   \begin{itemize}
       \item En $B=(1,-1)$ : $\varphi_A(1,-1) = 0$ implique que le facteur en $r$ doit s'annuler en $r=1$, donc contient $(1-r)$
       \item En $D=(-1,1)$ : $\varphi_A(-1,1) = 0$ implique que le facteur en $s$ doit s'annuler en $s=1$, donc contient $(1-s)$
       \item En $A=(-1,-1)$ : $\varphi_A(-1,-1) = 1$ détermine la constante de normalisation
   \end{itemize}
   
   On obtient donc $\varphi_A(r,s) = \frac{1}{4}(1-r)(1-s)$.
   
   De même :
   \begin{align*}
   \varphi_B(r,s) &= \frac{1}{4}(1+r)(1-s) \quad \text{(1 en $B$, 0 ailleurs)} \\
   \varphi_C(r,s) &= \frac{1}{4}(1+r)(1+s) \quad \text{(1 en $C$, 0 ailleurs)} \\
   \varphi_D(r,s) &= \frac{1}{4}(1-r)(1+s) \quad \text{(1 en $D$, 0 ailleurs)}
   \end{align*}
   
   \textbf{Vérification :} On peut vérifier que ces polynômes forment une partition de l'unité : $\varphi_A + \varphi_B + \varphi_C + \varphi_D = 1$ pour tout $(r,s)$ dans le carré.
\end{solution}
}{} 
        \vspace{1em}
        
        Plaçons-nous maintenant dans un autre repère orthonormé $(O, x, y)$ et considérons un quadrilatère $Q$ quelconque, de sommets $A=(x_A,y_A)$, $B=(x_B,y_B)$, $C=(x_C,y_C)$, $D=(x_D,y_D)$.
        \item Faites un dessin.
        \ifthenelse{\boolean{showSolutions}}{
            \begin{solution}
\begin{center}
\begin{tikzpicture}[scale=1.5]
    % Repère
    \draw[->] (-1,0) -- (3,0) node[right] {$x$};
    \draw[->] (0,-1) -- (0,3) node[above] {$y$};
    
    % Quadrilatère
    \coordinate (A) at (0.5,0.8);
    \coordinate (B) at (2.2,1);
    \coordinate (C) at (1.8,2.3);
    \coordinate (D) at (0.3,2);
    
    \draw (A) -- (B) -- (C) -- (D) -- cycle;
    
    % Points et labels
    \fill (A) circle (1.5pt) node[below left] {$A(x_A,y_A)$};
    \fill (B) circle (1.5pt) node[right] {$B(x_B,y_B)$};
    \fill (C) circle (1.5pt) node[above] {$C(x_C,y_C)$};
    \fill (D) circle (1.5pt) node[left] {$D(x_D,y_D)$};
\end{tikzpicture}
\end{center}
            \end{solution}
        }{} 
        Nous allons définir un nouveau repère $(M, r, s)$ lié au quadrilatère $Q$ de la façon suivante :
        \begin{itemize}
            \item $M$ est l'origine, le centre du quadrilatère,
            \item $r$ parallèle à $(AB)$,
            \item $s$ parallèle à $(AD)$.
        \end{itemize}
        \item Donner les coordonnées de $A$, $B$, $C$, $D$ dans ce repère et déduisez-en les expressions des polynômes $\varphi_A(r,s)$, $\varphi_B(r,s)$, $\varphi_C(r,s)$, $\varphi_D(r,s)$ dans ce repère.
        
        \ifthenelse{\boolean{showSolutions}}{
\begin{solution}
   Dans ce repère, on retrouve notre élément de référence, les polynômes sont les mêmes qu'à la première question.
\end{solution}
}{} 
        Notre but est maintenant de déterminer les polynômes $\varphi_A(x,y)$, $\varphi_B(x,y)$, $\varphi_C(x,y)$, $\varphi_D(x,y)$ dans le repère $(O, x, y)$. 
        Pour cela, nous cherchons le changement de base sous la forme
        \[
        \begin{pmatrix} x \\ y \end{pmatrix} = \begin{pmatrix} a_1 \\ b_1 \end{pmatrix} + \begin{pmatrix} a_2 \\ b_2 \end{pmatrix} r + \begin{pmatrix} a_3 \\ b_3 \end{pmatrix} s + \begin{pmatrix} a_4 \\ b_4 \end{pmatrix} rs.
        \]
        où $(a_1, b_1)$, $(a_2, b_2)$, $(a_3, b_3)$ et $(a_4, b_4)$ sont des constantes que nous allons déterminer.
        
        \item Déterminez les constantes $(a_1, b_1)$, $(a_2, b_2)$, $(a_3, b_3)$ et $(a_4, b_4)$ en fonction des coordonnées de $A$, $B$, $C$ et $D$ dans les deux repères.
        \ifthenelse{\boolean{showSolutions}}{
        \begin{solution}
           \textbf{Rappel de cours :} Pour les quadrilatères, le changement de variables n'est plus affine mais bilinéaire (il contient un terme en $rs$). Cela permet de transformer un carré de référence en un quadrilatère quelconque (pas nécessairement un parallélogramme).
           
           \textbf{Stratégie :} On substitue les coordonnées des 4 sommets dans les deux repères pour obtenir un système de 4 équations à 4 inconnues. Les sommets de référence sont $A=(-1,-1)$, $B=(1,-1)$, $C=(1,1)$, $D=(-1,1)$.
           
           \textbf{Calcul :} Pour les abscisses, on substitue dans $x = a_1 + a_2 r + a_3 s + a_4 rs$ :
           \begin{align*}
           x_A &= a_1 + a_2(-1) + a_3(-1) + a_4(-1)(-1) = a_1 - a_2 - a_3 + a_4 \\
           x_B &= a_1 + a_2(1) + a_3(-1) + a_4(1)(-1) = a_1 + a_2 - a_3 - a_4 \\
           x_C &= a_1 + a_2(1) + a_3(1) + a_4(1)(1) = a_1 + a_2 + a_3 + a_4 \\
           x_D &= a_1 + a_2(-1) + a_3(1) + a_4(-1)(1) = a_1 - a_2 + a_3 - a_4
           \end{align*}

           Cela donne un système de $4$ équations à $4$ inconnues, qu'on peut écrire sous la forme matricielle : 
           \[
           \begin{pmatrix} 1 & -1 & -1 & 1 \\ 1 & 1 & -1 & -1 \\ 1 & 1 & 1 & 1 \\ 1 & -1 & 1 & -1 \end{pmatrix} \begin{pmatrix} a_1 \\ a_2 \\ a_3 \\ a_4 \end{pmatrix} = \begin{pmatrix} x_A \\ x_B \\ x_C \\ x_D \end{pmatrix}.
           \]
           
           \textbf{Résolution du système :} L'inverse de la matrice du système est :
           \[
           \frac{1}{4}\begin{pmatrix} 1 & 1 & 1 & 1 \\ -1 & 1 & 1 & -1 \\ -1 & -1 & 1 & 1 \\ 1 & -1 & 1 & -1 \end{pmatrix}
           \]
           
           On trouve donc :
           \[
           \begin{pmatrix} a_1 \\ a_2 \\ a_3 \\ a_4 \end{pmatrix} = \frac{1}{4} \begin{pmatrix} x_A + x_B + x_C + x_D \\ -x_A + x_B + x_C - x_D \\ -x_A - x_B + x_C + x_D \\ x_A - x_B + x_C - x_D \end{pmatrix}.
           \]
           
           \textbf{Simplification :} En réarrangeant, on peut aussi écrire :
           \[
           \begin{pmatrix} a_1 \\ a_2 \\ a_3 \\ a_4 \end{pmatrix} = \frac{1}{2} \begin{pmatrix} \frac{x_A + x_C}{2} + \frac{x_B + x_D}{2}  \\ \frac{x_C - x_D}{2} + \frac{x_B - x_A}{2} \\ \frac{x_C - x_B}{2} + \frac{x_D - x_A}{2} \\ \frac{x_A - x_B + x_C - x_D}{2} \end{pmatrix}.
           \]
           
           Les ordonnées se déduisent de manière analogue :
           \[
           \begin{pmatrix} b_1 \\ b_2 \\ b_3 \\ b_4 \end{pmatrix} = \frac{1}{4} \begin{pmatrix} y_A + y_B + y_C + y_D \\ -y_A + y_B + y_C - y_D \\ -y_A - y_B + y_C + y_D \\ y_A - y_B + y_C - y_D \end{pmatrix}.
           \]
           
           \textbf{Interprétation :} Le terme $a_4 rs$ (et $b_4 rs$) permet de déformer le carré en un quadrilatère général. Si $a_4 = b_4 = 0$, on retrouve un changement affine et le quadrilatère est un parallélogramme.
\end{solution}
}{} 
        \item Déterminez les expressions des polynômes $\varphi_A(x,y)$, $\varphi_B(x,y)$, $\varphi_C(x,y)$, $\varphi_D(x,y)$ dans le cas où $A = (0,0)$, $B = (1,0)$, $C = (1,1)$, $D = (0,1)$.
        \ifthenelse{\boolean{showSolutions}}{
            \begin{solution}
            Avec ces coordonnées, le système devient : 
            \[
            a_1 = \frac{1}{2}, \qquad a_2 = \frac{1}{2}, \qquad a_3=a_4=0
            \]
            et
            \[
            b_1 = \frac{1}{2}, \qquad b_3 = \frac{1}{2}, \qquad b_2=b_4=0.
            \]
            Le changement de repère s'écrit donc 
            \[
            \left\{
            \begin{aligned}
            x = \frac{1}{2} + \frac{1}{2}r \\
            y = \frac{1}{2} + \frac{1}{2}s
            \end{aligned}
            \right.
            \]
            Donc 
            \[
            \left\{
            \begin{aligned}
            r = 2x-1 \\
            s = 2y-1
            \end{aligned}
            \right.
            \]
            En notant, $\varphi'_A, \varphi'_B, \varphi'_C, \varphi'_D$ les polynômes associés à l'élément de référence, on a :
            \begin{align*}
            \varphi_A(x,y) &= \varphi'_A(r, s) = \varphi'_A(2x-1, 2y-1) \\
            &= \frac{1}{4}(1-(2x-1))(1-(2y-1)) = (1-x)(1-y)
            \end{align*}    
            \begin{align*}
            \varphi_B(x,y) &= \varphi'_B(r, s) = \varphi'_B(2x-1, 2y-1) \\
            &= \frac{1}{4}(1+(2x-1))(1-(2y-1)) = x(1-y)
            \end{align*}
            \begin{align*}
            \varphi_C(x,y) &= \varphi'_C(r, s) = \varphi'_C(2x-1, 2y-1) \\
            &= \frac{1}{4}(1+(2x-1))(1+(2y-1)) = xy
            \end{align*}
            \begin{align*}
            \varphi_D(x,y) &= \varphi'_D(r, s) = \varphi'_D(2x-1, 2y-1) \\
            &= \frac{1}{4}(1-(2x-1))(1+(2y-1)) = (1-x)y
            \end{align*}   
\end{solution}
}{} 
        \item Faites de même dans le cas où $A = (1,1)$, $B = (2,1)$, $C = (2,2)$, $D = (1,2)$.
        \ifthenelse{\boolean{showSolutions}}{
            \begin{solution}

            Avec ces coordonnées, on obtient 
            \[
            a_1 = \frac{3}{2}, \quad a_2 = \frac{1}{2}, \quad a_3 = a_4 = 0,
            \]
            \[
            b_1 = \frac{3}{2}, \quad b_2 = 0, \quad b_3 = \frac{1}{2}, \quad b_4 = 0,
            \]
            le changement de repère s'écrit :
            \[
            \left\{
            \begin{aligned}
            x &= \frac{3}{2} + \frac{1}{2}r \\
            y &= \frac{3}{2} + \frac{1}{2}s
            \end{aligned}
            \right.
            \]
            Donc 
            \[
            \left\{
            \begin{aligned}
            r &= 2x-3 \\
            s &= 2y-3
            \end{aligned}
            \right.
            \]
            En notant $\varphi'_A, \varphi'_B, \varphi'_C, \varphi'_D$ les polynômes associés à l'élément de référence, on a :
            \begin{align*}
            \varphi_A(x,y) &= \varphi'_A(r, s) = \varphi'_A(2x-3, 2y-3) \\
            &= \frac{1}{4}(1-(2x-3))(1-(2y-3)) = (2-x)(2-y)
            \end{align*}    
            \begin{align*}
            \varphi_B(x,y) &= \varphi'_B(r, s) = \varphi'_B(2x-3, 2y-3) \\
            &= \frac{1}{4}(1+(2x-3))(1-(2y-3)) = (x-1)(2-y)
            \end{align*}  
            \begin{align*}
            \varphi_C(x,y) &= \varphi'_C(r, s) = \varphi'_C(2x-3, 2y-3) \\
            &= \frac{1}{4}(1+(2x-3))(1+(2y-3)) = (x-1)(y-1)
            \end{align*}  
            \begin{align*}
            \varphi_D(x,y) &= \varphi'_D(r, s) = \varphi'_D(2x-3, 2y-3) \\
            &= \frac{1}{4}(1-(2x-3))(1+(2y-3)) = (2-x)(y-1)
            \end{align*}  
   
\end{solution}
}{} 

    \end{enumerate}

    \item \textbf{définir le maillage} :
    \begin{enumerate}[label=\arabic*.]
        \item Découpez le domaine $\Omega$ en $4$ carrés. Nommez les noeuds et les éléments correspondant.
        \ifthenelse{\boolean{showSolutions}}{
        \begin{solution}
            Par exemple : 
        \begin{center}
            \includegraphics[width=0.6\textwidth]{maillage2.jpeg}
        \end{center}
        
        \end{solution}
        }{} 
        \item Etablissez le nouveau tableau de coordonnées des noeuds et le tableau de connectivité des éléments.
    \end{enumerate}
    
    \ifthenelse{\boolean{showSolutions}}{
    \begin{solution}
        
        Pour ce maillage, le tableau de coordonnées des noeuds est le même que pour les triangles : ce sont les mêmes noeuds:

        \begin{tabular}{|c|c|c|}
            \hline
            Noeud & Coordonnées  \\
            \hline
            1 & (0,1) \\
            2 & (1/2,1) \\
            3 & (1,1) \\
            4 & (0,1/2) \\
            5 & (1/2,1/2) \\
            6 & (1,1/2) \\
            7 & (0,0) \\
            8 & (1/2,0) \\
            9 & (1,0) \\
            \hline
        \end{tabular}

        Tableau de connectivité des éléments :
        \begin{tabular}{|c|c|c|}
            \hline
            Elément & noeuds  \\
            \hline
            1 & 4, 5, 2, 1 \\
            2 & 5, 6, 3, 2 \\
            3 & 7, 8, 5, 4 \\
            4 & 8, 9, 6, 5 \\
            \hline
        \end{tabular}
\end{solution}
}{} 
    \item \textbf{Obtention des fonctions de forme} :
    \begin{enumerate}[label=\arabic*.]
        \item Choisissez un élément fini, et exprimez les fonctions de forme pour cet élément.
        \ifthenelse{\boolean{showSolutions}}{
        \begin{solution}
            Prenons par exemple l'élément 1. Les coordonnées des points sont :
                \begin{itemize}
                    \item $A = (0, 1/2)$ pour le sommet $4$
                    \item $B = (1/2, 1/2)$ pour le sommet $5$
                    \item $C = (1/2, 1)$ pour le sommet $2$
                    \item $D = (0, 1)$ pour le sommet $1$
                \end{itemize}
                Le changement de repère s'écrit :
                \[
                \left\{
                    \begin{aligned}
                        x &= \frac{1}{4} + \frac{1}{4}r  \\
                        y &= \frac{3}{4} + \frac{1}{4}s
                    \end{aligned}
                \right.
                \]
                ce qui donne :
                \[
                \left\{
                    \begin{aligned}
                        r &= 4x-1 \\
                        s &= 4y-3
                    \end{aligned}
                \right.
                \]
                En notant $\varphi'_A(r,s)$, $\varphi'_B(r,s)$, $\varphi'_C(r,s)$, $\varphi'_D(r,s)$ les polynômes associés à l'élément de référence, les polynômes sont :
                \begin{align*}
                \varphi_A(x,y) &= \varphi'_A(r,s) = \varphi'_A(4x-1,4y-3) \\
                &= \frac{1}{4}(1-(4x-1))(1-(4y-3)) = \frac{1}{4}(2-4x)(4-4y) = 2(1-2x)(1-y)
                \end{align*}
                \begin{align*}
                \varphi_B(x,y) &= \varphi'_B(r,s) = \varphi'_B(4x-1,4y-3) \\
                &= \frac{1}{4}(1+(4x-1))(1-(4y-3)) = \frac{1}{4}(4x)(4-4y) = 4x(1-y)
                \end{align*}
                \begin{align*}
                \varphi_C(x,y) &= \varphi'_C(r,s) = \varphi'_C(4x-1,4y-3) \\
                &= \frac{1}{4}(1+(4x-1))(1+(4y-3)) = \frac{1}{4}(4x)(4y-2) = 4x(2y-1)
                \end{align*}
                \begin{align*}
                \varphi_D(x,y) &= \varphi'_D(r,s) = \varphi'_D(4x-1,4y-3) \\
                &= \frac{1}{4}(1-(4x-1))(1+(4y-3)) = \frac{1}{4}(2-4x)(4y-2) = (1-2x)(2y-1)
                \end{align*}
        \end{solution}
        }{} 
        \item Décrivez puis calculez la matrice élémentaire $K_e$ pour cet élément.
        \ifthenelse{\boolean{showSolutions}}{
        \begin{solution}
            La matrice de rigidité s'écrit :
            \[
            K_e = \begin{pmatrix}
            a(\varphi_A, \varphi_A) & a(\varphi_A, \varphi_B) & a(\varphi_A, \varphi_C) & a(\varphi_A, \varphi_D) \\
            a(\varphi_B, \varphi_A) & a(\varphi_B, \varphi_B) & a(\varphi_B, \varphi_C) & a(\varphi_B, \varphi_D) \\
            a(\varphi_C, \varphi_A) & a(\varphi_C, \varphi_B) & a(\varphi_C, \varphi_C) & a(\varphi_C, \varphi_D) \\
            a(\varphi_D, \varphi_A) & a(\varphi_D, \varphi_B) & a(\varphi_D, \varphi_C) & a(\varphi_D, \varphi_D)
            \end{pmatrix}
            \] 
            avec 
            \[
            a(u,v) = \int_0^1 \int_0^1 \nabla u\cdot \nabla v\,  dx dy.
            \]

            Commençons par calculer les vecteurs gradients : 
            \[
            \nabla \varphi_A = \begin{pmatrix}
            -4(1-y) \\
            -2(1-2x)
            \end{pmatrix}, \qquad 
            \nabla \varphi_B = \begin{pmatrix}
            4(1-y) \\
            -4x
            \end{pmatrix}, \qquad 
            \]
            \[
            \nabla \varphi_C = \begin{pmatrix}
            4(2y-1) \\
            8x
            \end{pmatrix}, \qquad 
            \nabla \varphi_D = \begin{pmatrix}
            -2(2y-1) \\
            2(1-2x)
            \end{pmatrix}.
            \]

            Pour le premier coefficient : 
            \begin{align*}
            a(\varphi_A, \varphi_A) &= \int_0^1 \int_0^1 \nabla \varphi_A \cdot \nabla \varphi_A dx dy \\
            &= \int_0^1 \int_0^1 16(1-y)^2 + 4(1-2x)^2  dx dy \\
            &= \int_{y=0}^{y=1} 16(1-y)^2 + \left[ -\frac{2}{3}(1-2x)^3 \right]_0^1 dy \\
            &= \left[ -\frac{16}{3}(1-y)^3\right]_0^1 + \frac{2}{3} + \frac{2}{3} \\
            &= \frac{16}{3} + \frac{4}{3}\\
            &= \frac{20}{3}
            \end{align*}

        Il faudrait calculer les $9$ autres intégrales pour pouvoir obtenir une matrice. 

        \end{solution}
        }{} 
        \item Comparez les résultats de cette méthode avec ceux obtenus dans l'Exercice 1. Les noeuds sont identiques, le système linéaire est-il le même ?
        \ifthenelse{\boolean{showSolutions}}{
        \begin{solution}
        Les deux méthodes utilisent les mêmes noeuds, mais le système linéaire obtenu n'est pas le même : avec les triangles, chaque matrice élémentaire relie seulement trois noeuds, alors qu'avec les carrés, elle relie quatre noeuds à la fois. Ainsi, la structure de la matrice globale diffère selon le type de maillage, même si les inconnues sont les mêmes.

        \end{solution}
        }{} 
    
    \end{enumerate}

    \item \textbf{maillage avec des triangles équilatéraux} :
    \begin{enumerate}[label=\arabic*.]
        \item Modifiez le domaine initial $\Omega$ en adaptant un maillage composé de $6$ triangles équilatéraux. 
        \textit{Vous ne pourrez pas paver le carré, il restera des trous.}
        \item Numérotez les noeuds et les éléments correspondants.
        \ifthenelse{\boolean{showSolutions}}{
        \begin{solution}
            On ne peut pas paver un carré avec des triangles équilatéraux. Un exemple de maillage qu'on puisse obtenir est le suivant : 
           \begin{center}
            \includegraphics[width=0.6\textwidth]{maillage3.jpeg}
           \end{center}
           On pourrait utiliser des triangles équilatéraux plus petits, pour couvrir une meilleure surface, mais il restera toujours des trous. 

        \end{solution}
        }{} 
        \item Établissez le tableau de coordonnées des noeuds et le tableau de connectivité des éléments.
        \item Exprimez les fonctions de forme pour cet élément.
        \ifthenelse{\boolean{showSolutions}}{
        \begin{solution}
            Plaçons-nous dans le triangle numéro 4, ses sommets sont 
            \begin{itemize}
             \item $A = (0,0)$ pour le sommet $6$
             \item $B = (1/2,0)$ pour le sommet $7$
             \item $C = (1/4, \sqrt{3}/4)$ pour le sommet $4$
            \end{itemize}
 
            En reprenant la partie préliminaire, le changement de repère s'écrit : 
            \[
            \left\{
            \begin{aligned}
            x &= \frac{1}{2}\, r + \frac{1}{4}\, s \\
            y &= \frac{\sqrt{3}}{4}\, s
            \end{aligned}
            \right.
            \]
            ce qui donne : 
            \[
            \left\{
            \begin{aligned}
            r &= 2x-\frac{2}{\sqrt{3}}\, y \\
            s &= \frac{4}{\sqrt{3}}\, y
            \end{aligned}
            \right.
            \]
            Si $\varphi'_A(r,s)$, $\varphi'_B(r,s)$, $\varphi'_C(r,s)$ sont les polynômes associés à l'élément de référence, alors les polynômes sont : 
            \begin{align*}
            \varphi_A(x,y) &= \varphi'_A(r,s) = \varphi'_A(2x-\frac{2}{\sqrt{3}}\, y, \frac{4}{\sqrt{3}}\, y) \\
            &= 1-(2x-\frac{2}{\sqrt{3}}\, y)-\frac{4}{\sqrt{3}}\, y = 1-2x-\frac{6}{\sqrt{3}}\, y
            \end{align*}
 
            \begin{align*}
            \varphi_B(x,y) &= \varphi'_B(r,s) = \varphi'_B(2x-\frac{2}{\sqrt{3}}\, y, \frac{4}{\sqrt{3}}\, y) \\
            &= 2x-\frac{2}{\sqrt{3}}\, y
            \end{align*}
  
            \begin{align*}
            \varphi_C(x,y) &= \varphi'_C(r,s) = \varphi'_C(2x-\frac{2}{\sqrt{3}}\, y, \frac{4}{\sqrt{3}}\, y) \\
            &= \frac{4}{\sqrt{3}}\, y
            \end{align*}
 
            Les gradients : 
            \[
            \nabla \varphi_A = \begin{pmatrix}
            -2 \\
            -\frac{6}{\sqrt{3}}
            \end{pmatrix}, \qquad 
            \nabla \varphi_B = \begin{pmatrix}
            2 \\
            -\frac{2}{\sqrt{3}}
            \end{pmatrix}, \qquad 
            \nabla \varphi_C = \begin{pmatrix}
            0 \\
            \frac{4}{\sqrt{3}}
            \end{pmatrix}.
            \]
            Cette fois, aucun coefficient de la matrice de rigidité n'est nul. 
        \end{solution}
        }{} 
    \end{enumerate}

    \item Quelle méthode préférez-vous ? Pourquoi ?


\end{enumerate}
 \ifthenelse{\boolean{showSolutions}}{
        \begin{solution}
            Si on doit faire les calculs à la main, la première méthode est clairement plus simple pour cette forme.
            Avec des triangles équilatéraux on arrivera rarement à couvrir exactement la pièce. 

            Pour des pièces complexes, les quadrilatères engendrent moins d'éléments donc moins d'erreurs d'arrondis.
            Si on arrive à paver une pièce avec des quadrilatères, on pourra toujours le faire avec des triangles ( simplement en traçant une diagonale), mais pas l'inverse. 
            
            Les triangles sont donc plus flexibles mais demandent plus de calculs car ils ont plus d'éléments, ils servent pour ajuster le maillage à des formes complexes.


        \end{solution}
        }{} 

\end{document}