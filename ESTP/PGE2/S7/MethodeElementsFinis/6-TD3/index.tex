\documentclass[11pt,a4paper]{report}

% -------------------- Encodage & langue --------------------
\usepackage[T1]{fontenc}
\usepackage[utf8]{inputenc}
\usepackage[french]{babel}
\usepackage{lmodern}
\usepackage{microtype}
\usepackage{amsmath, amssymb}
\usepackage{multicol}
\usepackage{enumitem}

\usepackage{amsfonts}
\usepackage[version=4]{mhchem}
\usepackage{stmaryrd}
\usepackage{graphicx}
\usepackage[export]{adjustbox}
\usepackage{caption}
\usepackage{multirow}
\usepackage{tikz}
% -------------------- Mise en page --------------------------
\usepackage[a4paper,margin=2cm]{geometry}
\usepackage{fancyhdr}
\usepackage{parskip}      % espace entre paragraphes
\setlength{\parindent}{0pt}

% -------------------- Couleurs & liens ----------------------
\usepackage{xcolor}
\definecolor{Theme}{HTML}{0E7490} % teal-700
\definecolor{ThemeLight}{HTML}{E0F2F1}
\definecolor{Accent}{HTML}{F59E0B} % amber-500
\definecolor{Gray}{HTML}{374151}
\usepackage[colorlinks=true,linkcolor=Theme,urlcolor=Theme,citecolor=Theme]{hyperref}

% -------------------- Graphiques / décor --------------------
\usepackage{tikz}
\usetikzlibrary{patterns,positioning,calc}
\usepackage{graphicx}
\usepackage{tcolorbox}
\tcbuselibrary{skins,breakable,hooks,most}
\usepackage{fontawesome5}

% -------------------- Titres -------------------------------
\usepackage{titlesec}
\titleformat{\chapter}[display]
  {\Huge\bfseries\color{Theme}}
  {\filright\rule{0.75\linewidth}{1.2pt}\\[3pt]{Algèbre linéaire - Chapitre~\thechapter}}
  {0.2ex}
  {\filright}
  [\vspace{0.1ex}\rule{0.35\linewidth}{1.2pt}]

\titleformat{\section}
  {\Large\bfseries\color{Gray}}
  {\thesection}{0.6em}{}

% -------------------- En-têtes / pieds ---------------------
\pagestyle{fancy}
\fancyhf{}
\fancyhead[L]{\color{Gray}\leftmark}
\fancyhead[R]{\color{Gray}\textit{MEF - 2025/2026}}
\fancyfoot[R]{\color{Gray}\small p.\ \thepage}
\renewcommand{\headrulewidth}{0pt}
\renewcommand{\footrulewidth}{0pt}

% -------------------- Macros utilitaires -------------------
\newenvironment{solution}
{
    \vspace{0.5em}
    \begin{mdframed}[backgroundcolor=ThemeLight,leftmargin=0,rightmargin=0,skipabove=0.2em,skipbelow=0.2em]
    \textbf{Solution.}\\[0.5em]
}
{
    \end{mdframed}
    \vspace{0.5em}
}



% -------------------- Page de titre ------------------------
\title{\textbf{Traces de cours}\\\large (résumés, formules, exemples, mini-exercices)}
\author{ MEF - 2025/2026 }
\date{\today}


\makeatletter
\renewcommand{\thesubsection}{\arabic{subsection}}
\renewcommand{\p@subsection}{}% supprime le préfixe section/chapter dans \ref
% Si vous voulez la même chose pour les sous-sous-sections :
% \renewcommand{\thesubsubsection}{\arabic{subsubsection}}
% \renewcommand{\p@subsubsection}{}
\makeatother

\usepackage{mdframed}
\usepackage{ifthen}

% \usepackage[sf]{titlesec}
% Définition de la variable pour afficher les corrections
\newboolean{showSolutions}
% Décommentez la ligne suivante pour afficher les solutions
\input \jobname.adr
% -------------------- Document ----------------------------
\begin{document}

\begin{center}
    {\LARGE \textbf{Méthode des éléments finis -- TD3}}\\[1em]
    {\large \textit{Des éléments un peu plus complexes}}
\end{center}

\section*{Exercice 1 : Éléments de barre, ordre supérieur}
On s'intéresse à l'équation de Laplace en 1D sur une barre de longueur $L$ de section $A$ et de module d'Young $E$. On considère une charge $f(x)$ appliquée à la barre.
$$ - E A \,\Delta u = f(x)$$
avec $u(0) = 0$ et $u(L) = 1$.
\begin{enumerate}
    \item Etablir la formulation variationnelle de l'équation différentielle pour l'écrire sous la forme 
    $$a(u,v) = \ell(v).$$
    avec $a$ une forme bilinéaire symétrique et $\ell$ une forme linéaire.

    \ifthenelse{\boolean{showSolutions}}{
        \begin{solution}
            Voir la méthode dans le TD2, on multiplie par une fonction test $v(x)$ nulle aux bordset on intégre sur le domaine $\Omega = [0, L]$ :
            \[
            \int_0^L - E A \,\frac{\partial^2 u}{\partial x^2} v(x) \, dx = \int_0^L f(x) v(x) \, dx
            \]
            Une intégration par parties donne :
            \[
            \int_0^L E A \, u'(x) v'(x) \, dx = \int_0^L f(x) v(x) \, dx
            \]
            On pose donc :
            \[
            a(u, v) = \int_0^L E A \, u'(x) v'(x) \, dx
            \]
            et
            \[
            \ell(v) = \int_0^L f(x) v(x) \, dx
            \]
        \end{solution}
    }{\vspace{1em}}
    

On découpe la barre en $n$ éléments finis de longueur $h = \frac{L}{n}$:
\[x_0 = 0,\, x_1 = h,\, x_2 = 2h, \ldots, x_n = L\]

\textbf{Eléments d'ordre 1 :}
La solution approchée sera cherchée sous la forme 
\[u_h(x) = \alpha_0 \varphi_0(x) + \alpha_1 \varphi_1(x) + \cdots + \alpha_n \varphi_n(x),\]
où $\varphi_i(x)$ sont les fonctions d'interpolation et $\alpha_i$ sont les inconnues.

    \item Rappelez l'expression de la matrice de rigidité élémentaire pour chaque élément fini. 
    \ifthenelse{\boolean{showSolutions}}{
        \begin{solution}

            Voir dans le TD2, l'équation différentielle sur le premier élément donne par exemple 
            \[a( u_h, \varphi_0) = \ell(\varphi_0), a( u_h, \varphi_1) = \ell(\varphi_1)\]
            En utilisant la linéarité de $a$, on obtient le système sous forme matricielle : 
            \[
            \begin{pmatrix}
                a(\varphi_0, \varphi_0) & a(\varphi_0, \varphi_1) \\
                a(\varphi_1, \varphi_0) & a(\varphi_1, \varphi_1)
            \end{pmatrix}
            \begin{pmatrix}
                \alpha_0 \\
                \alpha_1
            \end{pmatrix} = \begin{pmatrix} \ell(\varphi_0) \\ \ell(\varphi_1) \end{pmatrix}
            \]
            Rappel : nous avions obtenu la matrice de rigidité élémentaire :
            \[
            K_e = \frac{EA}{L} \begin{pmatrix} 1 & -1 \\ -1 & 1 \end{pmatrix}
            \]
            Pour les autres éléments, on obtient la même matrice de rigidité élémentaire.
        \end{solution}
    }{}

    \item Pour obtenir toutes les matrices de rigidité, en admettant qu'elles soient différentes sur tous les éléments, combien d'intégrales au total avez-vous besoin de calculer ?\newline 
    Comparez ce nombre avec le nombre de noeuds du maillage.
    \ifthenelse{\boolean{showSolutions}}{
        \begin{solution}
            Pour chaque élément fini, on a eu besoin de calculer 5 intégrales (par symétrie de $a$): 3 pour la matrice de rigidité, 2 pour le second membre. 
            Donc pour $n$ éléments finis, on aura besoin de calculer $5n$ intégrales.

            La solution approchée est construite sur $n+1$ points.
            Il faut donc calculer environ $5$ fois plus d'intégrales qu'il y a de noeuds dans le maillage. 
        \end{solution}
    }{}

    
    \item Si on voulait doubler le nombre de points de discrétisation, en dédoublant le maillage, combien d'intégrales au total avez-vous besoin de calculer ?
    \ifthenelse{\boolean{showSolutions}}{
        \begin{solution}
            Pour chaque élément fini, on a encore besoin de calculer 5 intégrales (par symétrie de $a$). 
            Donc pour $2n$ éléments finis, on aura besoin de calculer $10n$ intégrales. Deux fois plus d'intégrales que précédemment.
        \end{solution}
    }{\vspace{1em}}

    Essayons maintenant d'améliorer la solution en calculant moins d'intégrales :

On rappelle la formule générale pour un polynôme de Lagrange tel que $\varphi_i(x_j) = \delta_{ij}$ :
$$
\varphi_i(X)=\prod_{j \neq i} \frac{X-x_j}{x_i-x_j}\,.
$$



\textbf{Element d'ordre 2 :}
On ajoute maintenant un noeud intermédiaire à chaque élément. 
\[x_0 = 0, \, x_{1/2} = h/2, \, x_1 = h, \, x_{3/2} = 3h/2, \, x_2 = 2h, \ldots, \, x_{n-1/2} = (n-1)h/2, \, x_n = L\]


\item Comment est formée la solution approchée à partir des $\varphi_*$ ? 

    \ifthenelse{\boolean{showSolutions}}{
        \begin{solution}
            La solution approchée est formée par les $\varphi_i$ interpolés aux points du maillage :
            \[u_h(x) = \alpha_0 \varphi_0(x) + \alpha_{1/2} \varphi_{1/2}(x) + \alpha_1 \varphi_1(x) + \alpha_{3/2} \varphi_{3/2}(x) + \cdots + \alpha_n \varphi_n(x),\]
        \end{solution}
    }{}

\item Donner les expressions de $\varphi_0$, $\varphi_{1/2}$ et $\varphi_1$ entre $0$ et $h$.

    \ifthenelse{\boolean{showSolutions}}{
        \begin{solution}
            Sur chaque élément, il y a maintenant 3 noeuds. Cela conduira à des polynômes interpolateurs de degré 2. 
            Par exemple sur le premier élément, soit pour $x$ entre $0$ et $h$, on aura :
            \[
            \varphi_0(x) = \frac{(x-x_{1/2})(x-x_1)}{(x_0-x_{1/2})(x_0-x_1)}
            \]
            donc :
            \[
            \varphi_0(x) = 2 \frac{(x-h/2)(x-h)}{h^2}\]
            Ensuite, 
            \[
            \varphi_{1/2}(x) = \frac{(x-x_0)(x-x_1)}{(x_{1/2}-x_0)(x_{1/2}-x_1)}
            \]
            donc :
            \[
            \varphi_{1/2}(x) = -4 \frac{x(x-h)}{h^2}\]
            Enfin, 
            \[
            \varphi_1(x) = \frac{(x-x_0)(x-x_{1/2})}{(x_1-x_0)(x_1-x_{1/2})}
            \]
            donc :
            \[
            \varphi_1(x) = 2 \frac{x(x-h/2)}{h^2}\]
        \end{solution}
    }{}

    L'équation différentielle est toujours approchée par un système linéaire 
    \[K_e \begin{pmatrix} \alpha_0 \\ \alpha_{1/2} \\ \alpha_1 \end{pmatrix} = \begin{pmatrix} \ell(\varphi_0) \\ \ell(\varphi_{1/2}) \\ \ell(\varphi_1) \end{pmatrix}\]
\item Donner la taille et l'expression de la matrice de rigidité élémentaire $K_e$.

    \ifthenelse{\boolean{showSolutions}}{
        \begin{solution}
            La matrice de rigidité élémentaire est de taille $3 \times 3$ :
            \[
            K_e = 
            \begin{pmatrix}
                a(\varphi_0, \varphi_0) & a(\varphi_0, \varphi_{1/2}) & a(\varphi_0, \varphi_1) \\
                a(\varphi_{1/2}, \varphi_0) & a(\varphi_{1/2}, \varphi_{1/2}) & a(\varphi_{1/2}, \varphi_1) \\
                a(\varphi_1, \varphi_0) & a(\varphi_1, \varphi_{1/2}) & a(\varphi_1, \varphi_1)
            \end{pmatrix}
            \]
        \end{solution}
    }{}

    \item Combien de calculs d'intégrales au total avez-vous besoin de calculer par élément ?
    \ifthenelse{\boolean{showSolutions}}{
        \begin{solution}
            Pour chaque élément fini, on a eu besoin de calculer 9 intégrales (par symétrie de $a$): 6 pour la matrice de rigidité, 3 pour le second membre.  
            Donc pour $n$ éléments finis, on aura besoin de calculer $9n$ intégrales.
        \end{solution}
    }{}

    \item Calculer le coefficient première ligne, première colonne de la matrice de rigidité élémentaire $K_e$.
    \ifthenelse{\boolean{showSolutions}}{
        \begin{solution}
            \[
            a(\varphi_0, \varphi_0) = \int_0^h \varphi_0'(x) \varphi_0'(x) \, dx
            \]
            Sous forme développée, $\varphi_0$ s'écrit : 
            \[
            \varphi_0(x) = 2 \frac{x^2}{h^2} - 3 \frac{x}{h} + 1
            \]
            donc :
            \[
            \varphi_0'(x) = 4 \frac{x}{h^2} - \frac{3}{h}
            \]
            Ainsi :
            \[
            a(\varphi_0, \varphi_0) = \int_0^h \Big(4 \frac{x}{h^2} - \frac{3}{h}\Big)^2 \, dx
            \]
            Une primitive de $\Big(4 \frac{x}{h^2} - \frac{3}{h}\Big)^2$ est :
            \[
            \frac{h^2}{12} \Big(4\frac{x}{h^2} - \frac{3}{h}\Big)^3 = \frac{1}{12h} \Big(4\frac{x}{h} - 3\Big)^3
            \]
            donc :
            \[
            a(\varphi_0, \varphi_0) 
            = \frac{1}{12h} \Big(4 - 3\Big)^3 - \frac{1}{12h} \Big(- 3\Big)^3 
            = \frac{1 + 27}{12h}  = \frac{28}{12h} = \frac{7}{3h}
            \]
        \end{solution}
    }{}

    \item Comparer les deux méthodes de raffinement du maillage. 
    \ifthenelse{\boolean{showSolutions}}{
        \begin{solution}
            Pour un même nombre de points du maillage, on calcule (un peu) moins d'intégrale avec la deuxième méthode. 
            Les intégrales sont en revanche plus compliquées à calculer exactement mais l'ordinateur lui ne verra pas la différence. 
        \end{solution}
    }{\vspace{1em}}

    \item On attribue une lettre pour chaque élément fini, $a$ pour le premier, $b$ pour le second, etc. 
    Pour illustrer la construction de la matrice de rigidité globale, on utilise ces lettres pour les coefficients des matrices locales.
    Par exemple, pour $3$ éléments d'ordre $1$, la matrice de rigidité globale est assemblée comme suit : 
    \[
    \begin{pmatrix}
        a & a \,\, & 0 & 0 \\
        a & a+b & b & 0 \\
        0 &\, \,b & b+c & c \\
        0 & 0 & \,\,c & c 
    \end{pmatrix}
    \]
    (tous les coefficients notés $a$ ne sont pas identiques mais on les note de la même façon pour simplifier l'écriture).

    Montrer comment s'assemble la matrice de rigidité globale pour $3$ éléments d'ordre $2$.\newline

    \ifthenelse{\boolean{showSolutions}}{
        \begin{solution}
            Cette fois, les matrices locales sont de taille $3 \times 3$ :
            La matrice de rigidité globale est assemblée comme suit :
            \[
            \begin{pmatrix}
                a & a & a\,\, & 0 & 0 & 0 & 0\\
                a & a & a\,\, & 0 & 0 & 0 & 0\\
                a & a & a+b & b & b \,\,& 0 & 0\\
                0 & 0 & \,\,b & b & b\,\, & 0 & 0\\
                0 & 0 & \,\,b & b & b+c & c & c\\
                0 & 0 & 0 & 0 & \,\,c & c & c\\
                0 & 0 & 0 & 0 & \,\,c & c & c
            \end{pmatrix}
            \]
        \end{solution}
    }{}

\textbf{Element d'ordre 3 :}
On ajoute maintenant deux noeuds intermédiaires aux éléments finis.

\item Décrire les noeuds du maillage.

    \ifthenelse{\boolean{showSolutions}}{
        \begin{solution}
            Les noeuds du maillage sont :
            \[
            x_0 = 0,\,  x_{1/3} = h/3, \, x_{2/3} = 2h/3,\,  x_1 = h, \ldots,  x_{n-1/3} = (n-1)h/3, \, x_n = L
            \]
        \end{solution}
    }{}

\item Comment est formée la solution approchée $u_h(x)$ à partir des $\varphi_*$ ?

    \ifthenelse{\boolean{showSolutions}}{
        \begin{solution}
            Toujours par la même expression :
            \[u_h(x) = \alpha_0 \varphi_0(x) + \alpha_{1/3} \varphi_{1/3}(x) + \alpha_{2/3} \varphi_{2/3}(x) + \alpha_1 \varphi_1(x)  + \cdots + \alpha_n \varphi_n(x),\]
        \end{solution}
    }{}

\item Déterminer les expressions formelles des polynômes de Lagrange pour le premier élément fini.

    \ifthenelse{\boolean{showSolutions}}{
        \begin{solution}
            Les polynômes de Lagrange sont :
            \[
            \varphi_0(x) = \frac{(x-x_{1/3})(x-x_{2/3})(x-x_1)}{(x_0-x_{1/3})(x_0-x_{2/3})(x_0-x_1)}
            \]
            Ensuite, 
            \[
            \varphi_{1/3}(x) = \frac{(x-x_0)(x-x_{2/3})(x-x_1)}{(x_{1/3}-x_0)(x_{1/3}-x_{2/3})(x_{1/3}-x_1)}
            \] 
            \[
            \varphi_{2/3}(x) = \frac{(x-x_0)(x-x_{1/3})(x-x_1)}{(x_{2/3}-x_0)(x_{2/3}-x_{1/3})(x_{2/3}-x_1)}
            \]
            et
            \[
            \varphi_1(x) = \frac{(x-x_0)(x-x_{1/3})(x-x_{2/3})}{(x_1-x_0)(x_1-x_{1/3})(x_1-x_{2/3})}
            \]
           
        \end{solution}
    }{}

\item Donner la taille et l'expression de la matrice de rigidité élémentaire $K_e$.

    \ifthenelse{\boolean{showSolutions}}{
        \begin{solution}
            La matrice de rigidité élémentaire est de taille $4 \times 4$ :
            \[
            K_e = 
            \begin{pmatrix}
                a(\varphi_0, \varphi_0) & a(\varphi_0, \varphi_{1/3}) & a(\varphi_0, \varphi_{2/3}) & a(\varphi_0, \varphi_1)  \\
                a(\varphi_{1/3}, \varphi_0) & a(\varphi_{1/3}, \varphi_{1/3}) & a(\varphi_{1/3}, \varphi_{2/3}) & a(\varphi_{1/3}, \varphi_1) \\
                a(\varphi_{2/3}, \varphi_0) & a(\varphi_{2/3}, \varphi_{1/3}) & a(\varphi_{2/3}, \varphi_{2/3}) & a(\varphi_{2/3}, \varphi_1) \\
                a(\varphi_1, \varphi_0) & a(\varphi_1, \varphi_{1/3}) & a(\varphi_1, \varphi_{2/3}) & a(\varphi_1, \varphi_1) \\
             \end{pmatrix}
            \]
        \end{solution}
    }{}

    \item Combien d'intégrales au total avez-vous besoin de calculer par élément ?
    \ifthenelse{\boolean{showSolutions}}{
        \begin{solution}
            Pour chaque élément fini, on a eu besoin de calculer 14 intégrales (par symétrie de $a$). 
            Donc pour $n$ éléments finis, on aura besoin de calculer $14n$ intégrales.
        \end{solution}
    }{}

    \item Montrer comment s'assemble la matrice de rigidité globale pour $3$ éléments d'ordre $3$.

    \ifthenelse{\boolean{showSolutions}}{
        \begin{solution}
            Les matrices locales sont de taille $4 \times 4$ :
            La matrice de rigidité globale est assemblée comme suit :
            \[
            \begin{pmatrix}
                a & a & a & a \,\,  & 0 & 0 & 0   & 0 & 0 & 0 \\
                a & a & a & a \,\,  & 0 & 0 & 0   & 0 & 0 & 0 \\
                a & a & a & a \,\,  & 0 & 0 & 0   & 0 & 0 & 0 \\
                a & a & a & a+b & b & b & b   & 0 & 0 & 0 \\
                0 & 0 & 0 & \,\,b   & b & b & b \,\,  & 0 & 0 & 0 \\
                0 & 0 & 0 & \,\,b   & b & b & b  \,\, & 0 & 0 & 0 \\
                0 & 0 & 0 & \,\,b   & b & b & b+c & c & c & c \\
                0 & 0 & 0 & 0   & 0 & 0 & \,\,c   & c & c & c \\
                0 & 0 & 0 & 0   & 0 & 0 & \,\,c   & c & c & c \\
                0 & 0 & 0 & 0   & 0 & 0 &\,\, c   & c & c & c 
            \end{pmatrix}
            \]
        \end{solution}
    }{}
    \end{enumerate}

\vspace{2em}
\section*{Exercice 2 : Éléments de poutre}
On considère une poutre biencastrée soumise à une charge répartie uniformément.
\begin{center}
    \includegraphics[width=0.5\textwidth]{flexion.png}
\end{center}

\textbf{Modélisation Mécanique} : Une poutre en flexion subit des contraintes dues à une charge appliquée perpendiculairement à son axe longitudinal. 
    Les principaux paramètres sont le module d'élasticité $E$, le moment d'inertie de la section $I$ qu'on supposera constant, la longueur de la poutre $L$, et la distribution de la charge $q(x)$.
    
    L'équation différentielle régissant le déplacement transversal $w(x)$ est :
    \[
    EI \frac{d^4 w(x)}{dx^4} = q(x),
    \]
    où $w(x)$ est le déplacement transversal. Dans notre cas, la force $q$ est répartie uniformément sur la poutre.

\begin{enumerate}
    \item \textbf{Formulation Variationnelle} : Formulez le problème de flexion de la poutre en termes de formulation variationnelle.

    \ifthenelse{\boolean{showSolutions}}{
        \begin{solution}
    La formulation variationnelle est obtenue en multipliant l'équation différentielle par une fonction de test $v(x)$ et en intégrant sur le domaine $\Omega = [0, L]$. Après deux intégrations par parties pour obtenir le même degré de dérivation sur $w$ et $v$, on obtient :
    \[
    \int_{0}^{L} EI \frac{d^2 w(x)}{dx^2} \frac{d^2 v(x)}{dx^2} \, dx = \int_{0}^{L} q(x) v(x) \, dx.
    \]
    On posera donc 
    \[ a (w, v) = \int_{0}^{L} EI \frac{d^2 w(x)}{dx^2} \frac{d^2 v(x)}{dx^2} \, dx \]
    et
    \[ \ell (v) = \int_{0}^{L} q(x) v(x) \, dx. \]
    \end{solution}
    }{\newpage}
    
    
    Il y a cette fois deux inconnues par noeud : 
    \begin{itemize}
        \item le déplacement transversal $w(x)$ 
        \item  l'angle $\theta(x)$ qui est la dérivée première du déplacement transversal. 
    \end{itemize}
    Pour écrire une solution approchée du problème $w_h$, il y aura donc deux types de polynômes interpolateurs : 
    \begin{itemize}
        \item $\varphi_i(x)$ pour le déplacement transversal
        \item $\psi_i(x)$ pour l'angle.
    \end{itemize}

    Sur le premier élément fini, on cherchera alors $\alpha_0, \alpha_1, \beta_0, \beta_1$ tels que $$w(x) = \alpha_0 \varphi_0(x) + \beta_0 \psi_0(x) + \alpha_1\varphi_1(x) + \beta_1 \psi_1(x).$$

    \item Que doivent vérifier les polynômes $\varphi_i(x)$ et $\psi_i(x)$ pour que $\alpha_i$ soient bien les coefficients du déplacement transversal et $\beta_i$ les coefficients de l'angle ?
    \ifthenelse{\boolean{showSolutions}}{
        \begin{solution}
            il faut retrouver 
            $$w(0) = \alpha_0, w(x_1)= \alpha_1, \qquad \text{et} \qquad w'(0) = \beta_1, w'(x_1) = \beta_1$$

            Les polynômes $\varphi_i(x)$ et $\psi_i(x)$ doivent donc vérifier 
            \[
            \varphi_i(x_j) = \delta_{ij}, \quad \varphi_i'(x_j) = 0 \quad \forall i,j \in \{0,1\}.
            \]
            \[
            \psi_i(x_j) = 0, \quad \psi_i'(x_j) = \delta_{ij} \quad \forall i,j \in \{0,1\}.
            \]
        \end{solution}
    }{\vspace{1em}}
    
    Ces polynômes sont appelés polynômes de Hermite et les expressions générales de ces polynômes sont les suivantes :
    $$
        \varphi_i(X) =q_i(X)\left(1-q_i^{\prime}\left(x_i\right)\left(X-x_i\right)\right),  \qquad 
        \psi_i(X) =q_i(X) \left(X-x_i\right)
    $$
    avec $q_i(X)=\left(L_i(X)\right)^2$, où $L_i(X) = \prod_{j \neq i} \frac{X-x_j}{x_i-x_j}$ est le polynôme de Lagrange associé à $x_i$.
    
    Remarquez que $q_i' = 2 L_i' L_i$, puis vérifiez que ces polynômes satisfont les bonnes conditions

    \ifthenelse{\boolean{showSolutions}}{
        \begin{solution}

            \textbf{Pour $\varphi_i$ :}
            \[
                \varphi_i(X) = q_i(X)\big(1-q_i'(x_i)(X-x_i)\big)
            \]
            Calculons aux noeuds $x_j$ :
            \begin{itemize}
                \item Si $j=i$, $q_i(x_i) = 1$, $X-x_i = x_i-x_i=0$ donc $\varphi_i(x_i)=1$.
                \item Sinon, $q_i(x_j)=0$ et donc $\varphi(x_j)$ aussi. 
            \end{itemize}
            \textbf{Pour $\varphi_i'$ :}
            Commencons par calculer $\varphi_i'(X)$ :
            \[
                \varphi_i'(X) = q_i'(X)\big(1-q_i'(x_i)(X-x_i)\big) - q_i(X) q_i'(x_i)
            \]
            \begin{itemize}
                \item $X=x_j$ avec $j=i$ :
                \[
                    \varphi_i'(x_i) = q_i'(x_i) -  q_i'(x_i) = 0
                \]
                \item $X=x_j$ avec $j\neq i$ :
                \[
                    \varphi_i'(x_j) = q_i'(x_j)\cdot 0 - 0\cdot q_i'(x_j) = 0
                \]
            \end{itemize}

            \textbf{Pour $\psi_i(X)$ :}
            \[
                \psi_i(X) = q_i(X)(X-x_i)
            \]
            \begin{itemize}
                \item Pour $X=x_j$ ; si $j=i$, $X-x_i=0$, donc $\psi_i(x_i)=0$ ; 
                \item si $j\neq i$, $q_i(x_j)=0$, donc $\psi_i(x_j)=0$.
            \end{itemize}

            \textbf{Pour $\psi_i'$ :}
            Commencons par calculer $\psi_i'(X)$ :
            \[
                \psi_i'(X) = q_i'(X)(X-x_i) + q_i(X)
            \]
            \begin{itemize}
                \item $X=x_j$ avec $j=i$ :
                \[
                    \psi_i'(x_i) = q_i'(x_i) -  q_i'(x_i) = 0
                \]
                \item $X=x_j$ avec $j\neq i$ :
                \[
                    \psi_i'(x_j) = q_i'(x_j) -  q_i'(x_j) = 0
                \]
            \end{itemize}
            Ainsi, ces polynômes satisfont bien :
            \[
                \varphi_i(x_j) = \delta_{ij},\qquad \varphi_i'(x_j) = 0 
            \]
            \[
                \psi_i(x_j) = 0,\qquad \psi_i'(x_j) = \delta_{ij}
            \]
            pour tous les $i, j$.

            \vspace{1em}
        \end{solution}
    }{\vspace{1em}}

    \item Exprimer les polynômes de Hermite pour le premier élément du maillage. 
    \ifthenelse{\boolean{showSolutions}}{
        \begin{solution}
            Les fonctions de forme cubiques d'Hermite pour un élément fini en dimension 1 sont définies comme suit :
            \[
            \varphi_0(x) = 1 - 3\left(\frac{x}{h}\right)^2 + 2\left(\frac{x}{h}\right)^3,
            \]
            \[
            \psi_0(x) = x \left(1 - 2\frac{x}{h} + \left(\frac{x}{h}\right)^2\right),
            \]
            \[
            \varphi_1(x) = 3\left(\frac{x}{h}\right)^2 - 2\left(\frac{x}{h}\right)^3,
            \]
            \[
            \psi_1(x) = x \left(-\frac{x}{h} + \left(\frac{x}{h}\right)^2\right).
            \]
        \end{solution}
    }{\vspace{1em}}


    \item Le vecteur des inconnus est formé par $\begin{pmatrix} \alpha_0 \\ \beta_0 \\ \alpha_1 \\ \beta_1 \end{pmatrix}$. Exprimez la matrice de rigidité élémentaire $K_e$.
    \ifthenelse{\boolean{showSolutions}}{
        \begin{solution}
            La matrice de rigidité élémentaire $K_e$ est une matrice $4 \times 4$ intégrant les produits des dérivées secondes des fonctions de forme :
            Pour l'élément fini cubique d'Hermite, la matrice élémentaire $K_e$ est une matrice $4 \times 4$ intégrant les produits des dérivées secondes des fonctions de forme :
                \[
                K_e= 
                \begin{pmatrix}
                    a(\varphi_0,\varphi_0) & a(\varphi_0,\psi_0) & a(\varphi_0,\varphi_1) & a(\varphi_0,\psi_1) \\
                    a(\psi_0,\varphi_0) & a(\psi_0,\psi_0) & a(\psi_0,\varphi_1) & a(\psi_0,\psi_1) \\
                    a(\varphi_1,\varphi_0) & a(\varphi_1,\psi_0) & a(\varphi_1,\varphi_1) & a(\varphi_1,\psi_1) \\
                    a(\psi_1,\varphi_0) & a(\psi_1,\psi_0) & a(\psi_1,\varphi_1) & a(\psi_1,\psi_1) \\
                \end{pmatrix}
                \]
        \end{solution}
    }{\vspace{1em}}

    \item Choisisser un coefficient dans la matrice et calculer l'intégrale. vous devriez trouver un coefficient en accord avec la forme suivante :
    \[
    K_e = \frac{EI}{h^3}
    \begin{bmatrix}
    12 & 6h & -12 & 6h \\
    6h & 4h^2 & -6h & 2h^2 \\
    -12 & -6h & 12 & -6h \\
    6h & 2h^2 & -6h & 4h^2
    \end{bmatrix}
    \]

        \ifthenelse{\boolean{showSolutions}}{
            \begin{solution}
                
    
    En supposant $EI$ constant sur l'élément, les intégrales peuvent être calculées: pour la première ligne par exemple:
    \[
    a(\varphi_0,\varphi_0) = EI \int_0^h \frac{d^2 \varphi_1}{dx^2} \frac{d^2 \varphi_1}{dx^2} \, dx 
    \]
    \[
    a(\varphi_0,\varphi_0) = EI \int_0^h \frac{36}{h^4} \left( \frac{2x}{h} - 1 \right)^2 \, dx
    \]
    \[
    = EI \frac{36}{h^4} \int_0^h \left( \frac{4x^2}{h^2} - \frac{4x}{h} + 1 \right) \, dx
    \]
    \[
    = EI \frac{36}{h^4} \left[\frac{4x^3}{3h^2} - \frac{4x^2}{2h} + x\right]_0^h
    \]
    \[
    = EI \frac{12}{h^3}
    \]


    On trouve alors:
    \[
    a(\varphi_0,\varphi_0) = EI \frac{12}{h^3}
    \]

Il faut ensuite calculer une intégrale mélangeant $\varphi$ et $\psi$ : 

    \[
    a(\varphi_0,\psi_0) = EI \int_0^h \frac{d^2 \varphi_0}{dx^2} \frac{d^2 \psi_0}{dx^2} \, dx
    \]
    \[
    = EI \int_0^h \frac{6}{h^2} \left( \frac{2x}{h} - 1 \right) \frac{2}{h} \left( \frac{3x}{h} -2 \right) \, dx
    \]
    \[
    = EI \frac{12}{h^3} \int_0^h \left( \frac{6x^2}{h^2} - \frac{7x}{h} + 2 \right) \, dx
    \]
    \[
    = EI \frac{12}{h^3} \left[\frac{6x^3}{3h^2} - \frac{7x^2}{2h} + 2x\right]_0^h
    \]
    \[
    = EI \frac{6}{h^3}
    \]
Pour pouvoir déduire tous les coefficients, il faut maintenant calculer les intégrales avec $\psi$ :
\[
a(\psi_0,\psi_0) = EI \int_0^h \frac{d^2 \psi_0}{dx^2} \frac{d^2 \psi_0}{dx^2} \, dx
\]
\[
= EI \int_0^h \frac{4}{h^2} \left( \frac{3x}{h} -2 \right)  \left( \frac{3x}{h} -2 \right) \, dx
\]
\[
= EI \frac{4}{h^2} \int_0^h \left( \frac{9x^2}{h^2} - \frac{12x}{h} + 4 \right) \, dx
\]
\[
= EI \frac{4}{h^2} \left[\frac{9x^3}{3h^2} - \frac{12x^2}{2h} + 4x\right]_0^h
\]
\[
= EI \frac{4}{h}
\]

Pour $a(\psi_1,\psi_1)$, c'est presque la même chose :

\[
a(\psi_1,\psi_1) =  EI \frac{4}{h^2} \left[\frac{9x^3}{3h^2} - \frac{6x^2}{2h} + 1x\right]_0^h
\]
\[
= EI \frac{4}{h}
\]

On obtient finalement : 
\[
    K_e = \frac{EI}{h^3}
    \begin{bmatrix}
    12 & 6h & -12 & 6h \\
    6h & 4h^2 & -6h & 2h^2 \\
    -12 & -6h & 12 & -6h \\
    6h & 2h^2 & -6h & 4h^2
    \end{bmatrix}
\]
            \end{solution}
        }{\vspace{1em}}

    \item Ecrivez la matrice globale après assemblage en considérant trois éléments.
    \ifthenelse{\boolean{showSolutions}}{
        \begin{solution}
            La matrice globale après assemblage en considérant trois éléments est :
            \[
                \mathbf{K} = \frac{EI}{h^3}
                \begin{bmatrix}
                12 & 6h & -12 & 6h & 0 & 0 & 0 & 0 \\
                6h & 4h^2 & -6h & 2h^2 & 0 & 0 & 0 & 0 \\
                -12 & -6h & 24 & 0 & -12 & 6h & 0 & 0 \\
                6h & 2h^2 & 0 & 8h^2 & -6h & 2h^2 & 0 & 0 \\
                0 & 0 & -12 & -6h & 24 & 0 & -12 & 6h \\
                0 & 0 & 6h & 2h^2 & 0 & 8h^2 & -6h & 2h^2 \\
                0 & 0 & 0 & 0 & -12 & -6h & 12 & -6h \\
                0 & 0 & 0 & 0 & 6h & 2h^2 & -6h & 4h^2
                \end{bmatrix}
            \]
        \end{solution}
    }{}

\end{enumerate}

\end{document}