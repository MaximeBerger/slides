\documentclass[11pt,a4paper]{report}

% -------------------- Encodage & langue --------------------
\usepackage[T1]{fontenc}
\usepackage[utf8]{inputenc}
\usepackage[french]{babel}
\usepackage{lmodern}
\usepackage{microtype}
\usepackage{amsmath, amssymb}
\usepackage{multicol}
\usepackage{enumitem}

\usepackage{amsfonts}
\usepackage[version=4]{mhchem}
\usepackage{stmaryrd}
\usepackage{graphicx}
\usepackage[export]{adjustbox}
\graphicspath{ {./images/} }
\usepackage{caption}
\usepackage{multirow}
\usepackage{tikz}
% -------------------- Mise en page --------------------------
\usepackage[a4paper,margin=2cm]{geometry}
\usepackage{fancyhdr}
\usepackage{parskip}      % espace entre paragraphes
\setlength{\parindent}{0pt}

% -------------------- Couleurs & liens ----------------------
\usepackage{xcolor}
\definecolor{Theme}{HTML}{0E7490} % teal-700
\definecolor{ThemeLight}{HTML}{E0F2F1}
\definecolor{Accent}{HTML}{F59E0B} % amber-500
\definecolor{Gray}{HTML}{374151}
\usepackage[colorlinks=true,linkcolor=Theme,urlcolor=Theme,citecolor=Theme]{hyperref}

% -------------------- Graphiques / décor --------------------
\usepackage{tikz}
\usetikzlibrary{patterns,positioning,calc}
\usepackage{graphicx}
\usepackage{tcolorbox}
\tcbuselibrary{skins,breakable,hooks,most}
\usepackage{fontawesome5}

% -------------------- Titres -------------------------------
\usepackage{titlesec}
\titleformat{\chapter}[display]
  {\Huge\bfseries\color{Theme}}
  {\filright\rule{0.75\linewidth}{1.2pt}\\[3pt]{Algèbre linéaire - Chapitre~\thechapter}}
  {0.2ex}
  {\filright}
  [\vspace{0.1ex}\rule{0.35\linewidth}{1.2pt}]

\titleformat{\section}
  {\Large\bfseries\color{Gray}}
  {\thesection}{0.6em}{}

% -------------------- En-têtes / pieds ---------------------
\pagestyle{fancy}
\fancyhf{}
\fancyhead[L]{\color{Gray}\leftmark}
\fancyhead[R]{\color{Gray}\textit{MEF - 2025/2026}}
\fancyfoot[R]{\color{Gray}\small p.\ \thepage}
\renewcommand{\headrulewidth}{0pt}
\renewcommand{\footrulewidth}{0pt}

% -------------------- Macros utilitaires -------------------
\newenvironment{solution}
{
    \vspace{0.5em}
    \begin{mdframed}[backgroundcolor=ThemeLight,leftmargin=0,rightmargin=0,skipabove=0.2em,skipbelow=0.2em]
    \textbf{Solution.}\\[0.5em]
}
{
    \end{mdframed}
    \vspace{0.5em}
}


% Tcolorboxes stylisées
\tcbset{tracebox/.style={breakable,enhanced,sharp corners,boxrule=0pt,frame hidden,arc=2mm,
  colback=white,coltitle=black,fonttitle=\bfseries\large,
  borderline west={2mm}{0pt}{Theme},
  before skip=8pt,after skip=4pt,drop fuzzy shadow}}

\newtcolorbox{resumeBox}{tracebox,title={\faStickyNote\quad Résumé des idées}}
\newtcolorbox{rappelsBox}{tracebox,title={\faRedo\quad Ce que je dois savoir }}
\newtcolorbox{exempleBox}{tracebox,title={\faChalkboardTeacher\quad Exemple}}

% Encadré « Formules & illustrations »
\newtcolorbox{formulesBox}{tracebox,title={\faCalculator\quad Formules \& illustrations},colback=ThemeLight}

% Astuce : puces clean
\newenvironment{niceitemize}{\begin{itemize}\setlength{\itemsep}{0.25em}\color{Gray}}{\end{itemize}}

% Raccourci pour une « Trace » complète
% Usage : \TraceSection{Titre}{Objectif court}
\newcommand{\TraceSection}[2]{%
  
}

% -------------------- Page de titre ------------------------
\title{\textbf{Traces de cours}\\\large (résumés, formules, exemples, mini-exercices)}
\author{ MEF - 2025/2026 }
\date{\today}


\makeatletter
\renewcommand{\thesubsection}{\arabic{subsection}}
\renewcommand{\p@subsection}{}% supprime le préfixe section/chapter dans \ref
% Si vous voulez la même chose pour les sous-sous-sections :
% \renewcommand{\thesubsubsection}{\arabic{subsubsection}}
% \renewcommand{\p@subsubsection}{}
\makeatother

\usepackage{mdframed}
\usepackage{ifthen}

% \usepackage[sf]{titlesec}
% Définition de la variable pour afficher les corrections
\newboolean{showSolutions}
% Décommentez la ligne suivante pour afficher les solutions
\input \jobname.adr
% -------------------- Document ----------------------------
\begin{document}

\begin{center}
    {\LARGE \textbf{Méthode des éléments finis -- TD3}}\\[1em]
    {\large \textit{Des éléments un peu plus complexes}}
\end{center}

\section*{Exercice 1 : Éléments de barre, ordre supérieur}
On s'intéresse à l'équation de Laplace en 1D : 
$$\frac{\partial^2 u}{\partial x^2} = f(x)$$
avec $u(0) = 0$ et $u(L) = 1$.
Etablir la formulation variationnelle de l'équation différentielle. 

On considère un unique élément fini barre de longueur $L$, de section $A$ et de module d'Young $E$. 

On rappelle la formule générale pour un polynôme de Lagrange tel que $\varphi_i(x_j) = \delta_{ij}$ :
$$
\varphi_i(X)=\prod_{j \neq i} \frac{X-x_j}{x_i-x_j}
$$
\textbf{Element d'ordre 1 :}
Rappeler les polynômes de Lagrange d'ordre 1 et les calculs nécessaires pour retrouver la matrice de rigidité élémentaire : 
$$K = \frac{EA}{L} \begin{pmatrix} 1 & -1 \\ -1 & 1 \end{pmatrix}
$$

\textbf{Element d'ordre 2 :}
On ajoute maintenant un nœud intermédiaire à l'élément. 
Donner les polynômes de Lagrange nécessaires pour interpoler les 3 inconnues du problème.
Donner la taille de la matrice de rigidité élémentaire.
Donner la matrice de rigidité élémentaire.

\textbf{Element d'ordre 3 :}
On ajoute maintenant deux nœuds intermédiaires à l'élément fini.
Donner les polynômes de Lagrange nécessaires pour interpoler les 4 inconnues du problème.
Donner la taille de la matrice de rigidité élémentaire.
Donner la matrice de rigidité élémentaire.

\textbf{assemblage :}
Imaginons que le domaine d'étude soit composé de 3 éléments finis de barre, l'un à la suite de l'autre. 
En utilisant des symboles, montrer comment s'assemble la matrice de rigidité globale pour des éléments d'ordre 1, 2 et 3.


\section*{Exercice 3 : Une matrice de rigidité en 2D}
Dans un plan en 2D, on considère le déplacement horizontal et le déplacement vertical d'un point. On a donc 2 inconnues par nœud.
Quelle taille font les matrices de rigidité élémentaires pour les éléments finis d'ordre 1 ?
En utilisant des symboles, montrer comment s'assemble la matrice de rigidité globale pour des éléments d'ordre 1.


\includegraphics[width=0.5\textwidth]{triangle.png}


\section*{Exercice 2 : Éléments de poutre}
\begin{center}
    \includegraphics[width=0.5\textwidth]{flexion.png}
\end{center}

\begin{enumerate}[label=\alph*.]
    \item \textbf{Modélisation Mécanique} : Une poutre en flexion subit des contraintes de flexion dues à une charge appliquée perpendiculairement à son axe longitudinal. Les principaux paramètres sont le module d'élasticité $E$, le moment d'inertie de la section $I$ qu'on supposera constant, la longueur de la poutre $L$, et la distribution de la charge $q(x)$.
    
    \[
    EI \frac{d^4 w(x)}{dx^4} = q(x),
    \]
où $w(x)$ est le déplacement transversal. Dans notre cas, la force $q$ est répartie uniformément sur la poutre.
    
    \item \textbf{Formulation Variationnelle} : Formulez le problème de flexion de la poutre en termes de formulation variationnelle.
    
    \item \textbf{Utilisation de l'Élément Fini Cubique d'Hermite} : Expliquez pourquoi l'élément fini cubique d'Hermite est approprié pour l'analyse de flexion. Définissez pour un élément fini, les $4$ fonctions de forme correspondantes. On pourra soit chercher une expression générale dans le cours ou alors écrire le polynômes avec des coefficients inconnus puis les déterminer pour respecter les contraintes.
    
    On rappelle que ces polynômes doivent vérifier : 
    \[
    \varphi_i(x_j) = \delta_{ij}, \quad \frac{d\varphi_i}{dx}(x_j) = 0 \quad \forall i,j \in \{1,2,3,4\}.
    \]
    \[
    \psi_i(x_j) = 0, \quad \frac{d\psi_i}{dx}(x_j) = \delta_{ij} \quad \forall i,j \in \{1,2,3,4\}.
    \]
    On donne la formule générale pour les construire :
    $$
        \varphi_i =q_i(X)\left(1-q_i^{\prime}\left(x_i\right)\left(X-x_i\right)\right),  \qquad 
        \psi_i =q_i(X) \left(X-x_i\right)
    $$
    avec $q_i(X)=\prod_{j \neq i}\left(\frac{X-x_j}{x_i-x_j}\right)^2$,
    
    Vérifiez que ces polynômes vérifient les bonnes conditions
    \item \textbf{Calcul de la Matrice Élémentaire} : Calculez la matrice élémentaire $K_e$ pour l'élément fini cubique d'Hermite utilisé dans la flexion.
    
    \item \textbf{Assemblage et Résolution} : Décrivez le processus d'assemblage du système global et résolvez numériquement le déplacement pour une poutre encastrée soumise à une charge répartie uniformément.

    \item \textbf{Application Numérique} : Considérez une poutre encastrée de longueur $L = 10\, \text{m}$, avec un module d'élasticité $E = 210\, \text{GPa}$ et un moment d'inertie $I = 8.333 \times 10^{-6}\, \text{m}^4$. Supposons une charge répartie uniformément $q = 1000\, \text{N/m}$.
\end{enumerate}

\ifthenelse{\boolean{showSolutions}}{
\begin{mdframed}[linewidth=1pt]
\section*{Correction}

\begin{enumerate}[label=\alph*.]
   
    
    \item \textbf{Formulation Variationnelle} :
    
    La formulation variationnelle pour une poutre en flexion est obtenue en multipliant l'équation différentielle par une fonction de test $v(x)$ et en intégrant sur le domaine $\Omega = [0, L]$ :
    \[
    \int_{0}^{L} EI(x) \frac{d^2 w}{dx^2} \frac{d^2 v}{dx^2} \, dx = \int_{0}^{L} q(x) v(x) \, dx.
    \]
    
    Cette formulation nécessite que les fonctions de forme soient suffisamment régulières, c'est-à-dire qu'elles soient $C^1$ pour garantir la continuité des dérivées premières.
    
    \item \textbf{Utilisation de l'Élément Fini Cubique d'Hermite} :
    
    L'élément fini cubique d'Hermite est particulièrement adapté pour les problèmes de flexion car il permet de définir à la fois les déplacements et les rotations (dérivées premières) aux nœuds, assurant ainsi la continuité $C^1$ des fonctions de forme nécessaires pour une représentation précise des courbures.
    
    De cette manière, on n'obtient pas de "coins" dans la solution approchée comme on a pu le voir dans le TD1. 
    
    Les fonctions de forme cubiques d'Hermite pour un élément fini en dimension 1 sont définies comme suit :
    \[
    \varphi_1(x) = 1 - 3\left(\frac{x}{h}\right)^2 + 2\left(\frac{x}{h}\right)^3,
    \]
    \[
    \psi_1(x) = x \left(1 - 2\frac{x}{h} + \left(\frac{x}{h}\right)^2\right),
    \]
    \[
    \varphi_2(x) = 3\left(\frac{x}{h}\right)^2 - 2\left(\frac{x}{h}\right)^3,
    \]
    \[
    \psi_2(x) = x \left(-\frac{x}{h} + \left(\frac{x}{h}\right)^2\right).
    \]

    On cherche alors $\alpha_1, \alpha_2, \alpha_3, \alpha_4$ tels que $w(x) = \alpha_1 \varphi_1(x) + \alpha_2 \psi_1(x) + \alpha_3 \varphi_2(x) + \alpha_4 \psi_2(x)$.

    Pour les calculs des intégrales, nous aurons besoin des dérivées secondes, elles donnent:
    \[
    \frac{d^2 \varphi_1}{dx^2} = \frac{6}{h^2} \left( \frac{2x}{h} - 1 \right)
    \]
    \[
    \frac{d^2 \psi_1}{dx^2} = \frac{2}{h} \left( \frac{3x}{h} -2 \right)
    \]
    \[
    \frac{d^2 \varphi_2}{dx^2} = - \frac{6}{h^2} \left( \frac{2x}{h} - 1 \right)
    \]
    \[
    \frac{d^2 \psi_2}{dx^2} = \frac{2}{h} \left( \frac{3x}{h} - 1 \right)
    \]

    
    \item \textbf{Calcul de la Matrice Élémentaire} :
    
    Pour l'élément fini cubique d'Hermite, la matrice élémentaire $K_e$ est une matrice $4 \times 4$ intégrant les produits des dérivées secondes des fonctions de forme :
    \[
    K_e= 
    \begin{pmatrix}
        a(\varphi_1,\varphi_1) & a(\varphi_1,\psi_1) & a(\varphi_1,\varphi_2) & a(\varphi_1,\psi_2) \\
        a(\psi_1,\varphi_1) & a(\psi_1,\psi_1) & a(\psi_1,\varphi_2) & a(\psi_1,\psi_2) \\
        a(\varphi_2,\varphi_1) & a(\varphi_2,\psi_1) & a(\varphi_2,\varphi_2) & a(\varphi_2,\psi_2) \\
        a(\psi_2,\varphi_1) & a(\psi_2,\psi_1) & a(\psi_2,\varphi_2) & a(\psi_2,\psi_2)
    \end{pmatrix}
    \]
    
    En supposant $EI$ constant sur l'élément, les intégrales peuvent être calculées: pour la première ligne par exemple:
    \[
    a(\varphi_1,\varphi_1) = EI \int_0^h \frac{d^2 \varphi_1}{dx^2} \frac{d^2 \varphi_1}{dx^2} \, dx 
    \]
    \[
    a(\varphi_1,\varphi_1) = EI \int_0^h \frac{36}{h^4} \left( \frac{2x}{h} - 1 \right)^2 \, dx
    \]
    \[
    = EI \frac{36}{h^4} \int_0^h \left( \frac{4x^2}{h^2} - \frac{4x}{h} + 1 \right) \, dx
    \]
    \[
    = EI \frac{36}{h^4} \left[\frac{4x^3}{3h^2} - \frac{4x^2}{2h} + x\right]_0^h
    \]
    \[
    = EI \frac{12}{h^3}
    \]


    On trouve alors:
    \[
    a(\varphi_1,\varphi_1) = EI \frac{12}{h^3}
    \]

Il faut ensuite calculer une intégrale mélangeant $\varphi$ et $\psi$ : 

    \[
    a(\varphi_1,\psi_1) = EI \int_0^h \frac{d^2 \varphi_1}{dx^2} \frac{d^2 \psi_1}{dx^2} \, dx
    \]
    \[
    = EI \int_0^h \frac{6}{h^2} \left( \frac{2x}{h} - 1 \right) \frac{2}{h} \left( \frac{3x}{h} -2 \right) \, dx
    \]
    \[
    = EI \frac{12}{h^3} \int_0^h \left( \frac{6x^2}{h^2} - \frac{7x}{h} + 2 \right) \, dx
    \]
    \[
    = EI \frac{12}{h^3} \left[\frac{6x^3}{3h^2} - \frac{7x^2}{2h} + 2x\right]_0^h
    \]
    \[
    = EI \frac{6}{h^3}
    \]
Pour pouvoir déduire tous les coefficients, il faut maintenant calculer les intégrales avec $\psi$ :
\[
a(\psi_1,\psi_1) = EI \int_0^h \frac{d^2 \psi_1}{dx^2} \frac{d^2 \psi_1}{dx^2} \, dx
\]
\[
= EI \int_0^h \frac{4}{h^2} \left( \frac{3x}{h} -2 \right)  \left( \frac{3x}{h} -2 \right) \, dx
\]
\[
= EI \frac{4}{h^2} \int_0^h \left( \frac{9x^2}{h^2} - \frac{12x}{h} + 4 \right) \, dx
\]
\[
= EI \frac{4}{h^2} \left[\frac{9x^3}{3h^2} - \frac{12x^2}{2h} + 4x\right]_0^h
\]
\[
= EI \frac{4}{h}
\]

Pour $a(\psi_2,\psi_2)$, c'est presque la même chose :

\[
a(\psi_2,\psi_2) =  EI \frac{4}{h^2} \left[\frac{9x^3}{3h^2} - \frac{6x^2}{2h} + 1x\right]_0^h
\]
\[
= EI \frac{4}{h}
\]

On obtient finalement : 
\[
    K_e = \frac{EI}{h^3}
    \begin{bmatrix}
    12 & 6h & -12 & 6h \\
    6h & 4h^2 & -6h & 2h^2 \\
    -12 & -6h & 12 & -6h \\
    6h & 2h^2 & -6h & 4h^2
    \end{bmatrix}
\]
    
    \item \textbf{Assemblage et Résolution} :

        
        Assemblez la matrice globale de rigidité $\mathbf{K}$ en combinant les matrices élémentaires $K_e$ de chaque élément. Pour une poutre encastrée, imposez les conditions aux limites appropriées (par exemple, déplacement nul et rotation nulle à l'extrémité encastrée).
        La matrice globale de rigidité $\mathbf{K}$ pour une poutre discrétisée en $n$ éléments est une matrice de taille $2(n+1) \times 2(n+1)$, car chaque nœud a deux degrés de liberté (déplacement et rotation). Pour trois éléments ($n=3$), la matrice globale est de taille $8 \times 8$.

        L'assemblage se fait en additionnant les contributions de chaque élément aux positions appropriées dans la matrice globale. Pour un élément $e$ reliant les nœuds $e$ et $e+1$, les indices dans la matrice globale sont :
        \[
        \begin{array}{l}
        \text{Pour le nœud } e: \quad \text{indices } 2e-1 \text{ et } 2e \\
        \text{Pour le nœud } e+1: \quad \text{indices } 2e+1 \text{ et } 2e+2
        \end{array}
        \]

        Pour une poutre encastrée à gauche avec trois éléments, la matrice globale avant application des conditions aux limites est :
        \[
        \mathbf{K} = \frac{EI}{h^3}
        \begin{bmatrix}
        12 & 6h & -12 & 6h & 0 & 0 & 0 & 0 \\
        6h & 4h^2 & -6h & 2h^2 & 0 & 0 & 0 & 0 \\
        -12 & -6h & 24 & 0 & -12 & 6h & 0 & 0 \\
        6h & 2h^2 & 0 & 8h^2 & -6h & 2h^2 & 0 & 0 \\
        0 & 0 & -12 & -6h & 24 & 0 & -12 & 6h \\
        0 & 0 & 6h & 2h^2 & 0 & 8h^2 & -6h & 2h^2 \\
        0 & 0 & 0 & 0 & -12 & -6h & 12 & -6h \\
        0 & 0 & 0 & 0 & 6h & 2h^2 & -6h & 4h^2
        \end{bmatrix}
        \]

        Pour imposer les conditions d'encastrement à gauche (déplacement et rotation nuls au premier nœud), on supprime les deux premières lignes et colonnes de la matrice, obtenant ainsi une matrice $6 \times 6$ :
        \[
        \mathbf{K}_{\text{réduite}} = \frac{EI}{h^3}
        \begin{bmatrix}
        24 & 0 & -12 & 6h & 0 & 0 \\
        0 & 8h^2 & -6h & 2h^2 & 0 & 0 \\
        -12 & -6h & 24 & 0 & -12 & 6h \\
        6h & 2h^2 & 0 & 8h^2 & -6h & 2h^2 \\
        0 & 0 & -12 & -6h & 12 & -6h \\
        0 & 0 & 6h & 2h^2 & -6h & 4h^2
        \end{bmatrix}
        \]
        \item \textbf{Résolution} :
        
        On résout alors le système linéaire $\mathbf{K} \mathbf{W} = \mathbf{F}$ pour obtenir le vecteur des déplacements nodaux $\mathbf{W}$.
        Pour le cas de trois éléments avec une charge répartie $q$, le vecteur des forces nodales équivalentes est :
        \[
        \mathbf{F} = \frac{qh}{2}
        \begin{bmatrix}
        1 \\
        h/6 \\
        2 \\
        0 \\
        2 \\
        0 \\
        1 \\
        -h/6
        \end{bmatrix}
        \]
        
        Après application des conditions aux limites (suppression des deux premières composantes), on obtient :
        \[
        \mathbf{F}_{\text{réduit}} = \frac{qh}{2}
        \begin{bmatrix}
        2 \\
        0 \\
        2 \\
        0 \\
        1 \\
        -h/6
        \end{bmatrix}
        \]
        
    \item \textbf{Résultats} :
    
    Considérons une poutre encastrée de longueur $L = 10\, \text{m}$, avec un module d'élasticité $E = 210\, \text{GPa}$ et un moment d'inertie $I = 8.333 \times 10^{-6}\, \text{m}^4$. Supposons une charge répartie uniformément $q = 1000\, \text{N/m}$.
    
    En utilisant des éléments finis Hermite cubiques avec trois éléments, le système linéaire est résolu pour obtenir les déplacements nodaux. Les résultats montrent que le déplacement maximal se situe à l'extrémité libre et correspond étroitement à la solution analytique :
    \[
    w(L) = \frac{q L^4}{8 EI}.
    \]
    
    Pour les paramètres donnés :
    \[
    w(10) = \frac{1000 \times 10^4}{8 \times 210 \times 10^9 \times 8.333 \times 10^{-6}} \approx 7.2 \times 10^{-3}\, \text{m}.
    \]
    
    Les déplacements calculés par la méthode des éléments finis confirment cette valeur, démontrant la précision de l'élément fini cubique d'Hermite pour les problèmes de flexion.

\end{enumerate}

\end{mdframed}
}{} % Fin de la condition


\end{document}