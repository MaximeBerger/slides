% Classe et options générales
\documentclass[12pt,a4paper]{report}

% Langue et encodage
\usepackage[T1]{fontenc}

\usepackage{lmodern,textcomp}
\usepackage[french]{babel}

% Mise en page
\usepackage{geometry}
\geometry{top=2.5cm,bottom=2cm,left=2cm,right=2cm, headheight=50pt}
\usepackage{fancyhdr}
\pagestyle{fancy}
\fancyhf{}
\lhead{BMC Dijon}
% \includegraphics[width=4em]{images/logotype estp couleur.png}}
\fancyhead[R]{\leftmark} % Affiche le nom du chapitre à gauche
\renewcommand{\chaptermark}[1]{\markboth{\thechapter\quad #1}{}}

\cfoot{\thepage}

\renewcommand{\headrulewidth}{2pt}
\renewcommand{\headrule}{\hbox to\headwidth{\color{orange}\leaders\hrule height \headrulewidth\hfill}}

% Mathématiques et symboles
\usepackage{amsmath,amssymb,amsfonts,mathrsfs,bm}
\usepackage{array,multirow,tabularx}
\usepackage{enumitem}
\usepackage{tablists}
\usepackage{pifont,centernot,eurosym}
\usepackage{multicol}
\usepackage{enumitem}
\usepackage{eurosym}
\usepackage{tabvar}

% Graphiques
\usepackage{graphicx}
\usepackage{pdfpages}
\usepackage{tikz,pgfplots,tkz-tab}
\pgfplotsset{compat=1.18}

\usepackage[tikz]{bclogo}
\usetikzlibrary{angles, quotes, calc, decorations.markings}

% Théorèmes
\usepackage[standard,framed]{ntheorem}

\newcommand{\bclogoboxmaster}[4]{%
  \theoremprework{%
    \vspace{1em}
    \begin{lrbox}{\boitprop}

\begin{minipage}{\dimexpr\textwidth - 4.5em\relax}
        \vspace{0.5em}
  }
  \theorempostwork{%
        \vspace{.3em}
      \end{minipage}
    \end{lrbox}
    \begin{bclogo}[couleur=#1, arrondi=0.1, ombre=true,
      couleurOmbre=black!30, logo=#2]{\hspace{1em}\usebox{\boitprop}}
    \end{bclogo}
  }
  \newtheorem{#3}{#4}[section]%
}

\newcommand{\bclogoboxslave}[4]{%
  \theoremprework{%
    \vspace{1em}
    \begin{lrbox}{\boitprop}
\begin{minipage}{\dimexpr\textwidth - 4.5em\relax}
        \vspace{0.5em}
  }
  \theorempostwork{%
        \vspace{.3em}
      \end{minipage}
    \end{lrbox}
    \begin{bclogo}[couleur=#1, arrondi=0.1, ombre=true,
      couleurOmbre=black!30, logo=#2]{\hspace{1em}\usebox{\boitprop}}
    \end{bclogo}
  }
  \newtheorem{#3}[Thm]{#4}
}

\newcommand{\bclogoboxnonum}[4]{%
  \theoremprework{%
    \vspace{1em}
    \begin{lrbox}{\boitprop}
\begin{minipage}{\dimexpr\textwidth - 4.5em\relax}
        \vspace{0.5em}
  }
  \theorempostwork{%
        \vspace{.3em}
      \end{minipage}
    \end{lrbox}
    \begin{bclogo}[couleur=#1, arrondi=0.1, ombre=true,
      couleurOmbre=black!30, logo=#2]{\hspace{1em}\usebox{\boitprop}}
    \end{bclogo}
  }
  \newtheorem*{#3}{#4}%
}

\newtheorem*{Meth}{\underline{M\'ethode}}

%NOTATION
\theoremprework{\vspace{.5cm} 
    \begin{lrbox}{\boitprop}
     \begin{minipage}{.9\textwidth}
}
\theorempostwork{
     \end{minipage}
    \end{lrbox}
%        \begin{center}
    \begin{tikzpicture}
    \node [draw=violet!10,very thick,fill=blue!10,rectangle% , rounded corners
    , inner sep=10pt, inner ysep=8pt] (box){\usebox{\boitprop}};
    \end{tikzpicture}
 %   \end{center}
}

%NOTATION
\theoremprework{\vspace{.5cm} 
    \begin{lrbox}{\boitprop}
     \begin{minipage}{.9\textwidth}
}
\theorempostwork{
     \end{minipage}
    \end{lrbox}
%        \begin{center}
    \begin{tikzpicture}
    \node [draw=pink!10,very thick,fill=blue!10,rectangle% , rounded corners
    , inner sep=10pt, inner ysep=8pt] (box){\usebox{\boitprop}};
    \end{tikzpicture}
 %   \end{center}
}

% Autres utilitaires
\usepackage{verbatim,lastpage,afterpage,sectsty,color,colortbl,boites,calc,listings, caption}

% Styles de caption
% \AtBeginDocument{
%   \captionsetup[figure]{font=footnotesize,labelsep=period}
%   \captionsetup[table]{font=footnotesize,labelsep=newline,justification=centering}
% }
% Définition de blocs théorèmes avec bclogo
\newsavebox{\boitprop}


\bclogoboxmaster{orange!20}{\bcbook}{Thm}{Théorème}
\bclogoboxslave{teal!15}{\bcfleur}{Def}{Définition}
\bclogoboxslave{cyan!15}{\bcetoile}{Prop}{Proposition}
\bclogoboxnonum{red!10}{\bcfeuvert}{Rmq}{Remarque}

% Environnements personnalisés (exemples, vocabulaire, etc.)
% Exemple :
% \newenvironment{Ex}
% {\vspace{.5cm}\begin{tabular}{c|r}
%     \arrayrulecolor{black} \textbf{Exemple} &  \begin{minipage}[l]{\textwidth}}{
% \end{minipage}\end{tabular}\vspace{.2cm}}

% \newenvironment{Ex}{%
%   \vspace{.5cm}%
%   \begin{tabularx}{\textwidth}{c|X}
%     \arrayrulecolor{black} \textbf{Exemple} & 
% }{%
%   \\ % pour finir la ligne du tableau
%   \end{tabularx}%
%   \vspace{.2cm}%
% }
\usepackage[most]{tcolorbox}

\newtcolorbox{Ex}[1][]{
  colback=white,
  colframe=black,
  boxrule=1pt,
  left=5pt,
  rounded corners,
  before skip=10pt,
  after skip=10pt,
  title={\textbf{Exemple}},
  fonttitle=\bfseries,
  #1
}
% \newenvironment{Ex}{
% \vspace{.5cm}
% \noindent\textbf{Exemple.}\par
% \vspace{.3cm}
% }{\vspace{.3cm}}
% % Préambule
% \usepackage{xcolor}


% Définition du compteur
\newcounter{exocounter}

% Logo de difficulté pour les exercices (niveaux numériques 1/2/3)
\newcommand{\exologo}[1]{%
  \ifcase#1\relax
  \or \includegraphics[height=1em]{fourmi.png}% 1
  \or \includegraphics[height=1.2em]{abeille.png}% 2
  \or \includegraphics[height=1.5em]{castor.png}% 3
  \fi
}

% Commande \exo avec argument optionnel pour le niveau de difficulté
\newcommand{\exo}[2][]{%
  \refstepcounter{exocounter}%
  \vspace{1em}
  \noindent
  \textcolor{orange}{\rule{1ex}{1ex}}%
  \hspace{0.5em}%
  \textbf{\textcolor{orange}{Exercice \theexocounter}}%
  \hspace{0.5em}%
  % Affiche le logo uniquement si un niveau est donné
  \if\relax\detokenize{#1}\relax\else
    \exologo{#1}%
    \hspace{0.5em}%
  \fi
  \textbf{#2}%
  \par\vspace{0.5em}
}
\makeatletter
\@addtoreset{exocounter}{chapter}
\makeatother
\renewcommand{\theexocounter}{\thechapter.\arabic{exocounter}}

% Raccourcis mathématiques
\newcommand{\C}{\mathbb{C}} \newcommand{\R}{\mathbb{R}} \newcommand{\N}{\mathbb{N}} \newcommand{\Z}{\mathbb{Z}} \newcommand{\K}{\mathbb{K}}
\newcommand{\ds}{\displaystyle} \newcommand{\eps}{\epsilon} \renewcommand{\phi}{\varphi}
\newcommand{\abs}[1]{\left|#1\right|} \newcommand{\p}[1]{\left(#1\right)} \newcommand{\e}[1]{\left\{#1\right\}}

% Commandes logiques et limites
\newcommand{\limni}{\ds\lim_{n\ra+\infty}} \newcommand{\limxz}{\ds\lim_{x\ra 0}} \newcommand{\tend}[1]{\underset{#1}{\longrightarrow}}
\newcommand{\dx}{\, \mathrm{d}x}\newcommand{\dt}{\, \mathrm{d}t}\newcommand{\du}{\, \mathrm{d}u}

% Listings Python (extrait simplifié)
\lstset{language=Python,basicstyle=\ttfamily,commentstyle=\color{green!50!black},keywordstyle=\color{blue},stringstyle=\color{olive},frame=single,rulecolor=\color{blue},numbers=left,stepnumber=1,numbersep=8pt,showstringspaces=false}



\usepackage[sf]{titlesec}
\titleformat{\section}{\sffamily\large\bfseries}{\thesection}{1em}{}
\titleformat{\subsection}
  {\sffamily\normalsize\bfseries}
  {\thesubsection}
  {1em}
  {}

\renewcommand{\familydefault}{\sfdefault}

\titleformat{\chapter}[display]
  {\normalfont\bfseries\centering\Huge}
  {\chaptername\ \thechapter}
  {1ex}
  {}

\usepackage[colorlinks=true, linkcolor=black]{hyperref}
\usepackage{bookmark}
\usepackage{subfiles}
