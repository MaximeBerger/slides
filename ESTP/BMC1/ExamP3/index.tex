\documentclass[10pt]{article}

% Packages pour les marges
\usepackage[
    top=2cm,
    bottom=2cm,
    left=2cm,
    right=2cm
]{geometry}

% Police sans serif
% \usepackage{helvet}
% \renewcommand{\familydefault}{\sfdefault}

% Packages existants
\usepackage[french]{babel}
\usepackage[utf8]{inputenc}
\usepackage[T1]{fontenc}
\usepackage{amsmath}
\usepackage{amsfonts}
\usepackage{amssymb}
\usepackage{mathtools}
\usepackage{array}
\usepackage[version=4]{mhchem}
\usepackage{stmaryrd}
\usepackage{enumitem}
\usepackage{ifthen}
\usepackage{eurosym}
\usepackage{textcomp}
\usepackage{graphicx}
\usepackage{xcolor}
\definecolor{Theme}{HTML}{0E7490} % teal-700
\definecolor{ThemeLight}{HTML}{E0F2F1}
\definecolor{Accent}{HTML}{F59E0B} % amber-500
\definecolor{Gray}{HTML}{374151}
\usepackage[colorlinks=true,linkcolor=Theme,urlcolor=Theme,citecolor=Theme]{hyperref}

\usepackage{mdframed}
\usepackage[sf]{titlesec}
% Définition de la variable pour afficher les corrections
\newboolean{showSolutions}
% Décommentez la ligne suivante pour afficher les solutions
\setboolean{showSolutions}{true}
% \setboolean{showSolutions}{false}

\title{Probabilités : Examen}

\author{}
\date{}


\newenvironment{solution}
{
    \vspace{0.5em}
    \begin{mdframed}[backgroundcolor=ThemeLight,leftmargin=0,rightmargin=0,skipabove=0.2em,skipbelow=0.2em]
    \textbf{Solution.}\\[0.5em]
}
{
    \end{mdframed}
    \vspace{0.5em}
}
\begin{document}
\sffamily

\begin{center}
    \renewcommand{\arraystretch}{1.5} % Ajuste l'espacement vertical des lignes
    \begin{tabular}{|>{\centering\arraybackslash}m{4cm}|>{\centering\arraybackslash}m{6cm}|>{\centering\arraybackslash}m{4cm}|}
        \hline 
        \vspace{5mm} \hspace{5mm}\raisebox{-0.2\height}{\includegraphics[width=3cm]{Logo-ESTP.png}} \vspace{5mm}  & 
        \textbf{Contrôle de connaissances et de compétences} & 
        \textbf{FO-002-VLA-XX-001} \\
        \hline
        \textbf{21/05/2025}  &  & \textbf{Page 1/2} \\
        \hline
    \end{tabular}
\end{center}
\vspace{1em}

\begin{center}
    \renewcommand{\arraystretch}{1.5}
    \begin{tabular}{|c|m{10cm}|}
        \hline 
        \multicolumn{2}{|c|}{\textbf{ANNÉE SCOLAIRE 2024-2025 -- Semestre 6}} \\
        \hline 
        \textbf{Nom de l'enseignant} & Maxime BERGER \\
        \hline 
        \textbf{Promotion} & BMC1 - S1 \\
        \hline 
        \textbf{Matière} & Probabilités et statistiques  \\
        \hline 
        \textbf{Durée de l'examen} & 2h00 \\
        \hline 
        \textbf{Consignes} & 
        \begin{itemize}
            \item Calculatrice \textbf{NON} autorisée
            \item Aucun document n'est autorisé
        \end{itemize} \\
        \hline
    \end{tabular}
\end{center}



\section{Exercice 1 : Équations}
Résoudre :
\begin{enumerate}
    \item $$z-2i=iz+1$$
    \item $$z^4-2z^3-z+2=0$$ 
    \item $$2z^2-(7+3i)z+(2+4i)=0$$
    \item $$z^4=-1$$
    \item $$z^6=\frac{3}{1-i \sqrt{3}}$$
\end{enumerate}

\section{Exercice 2 : Lieux géométriques}
Dans le plan complexe, déterminer l'ensemble des points $M$ dont l'affixe $z$ vérifie la relation demandée :
\begin{enumerate}
    \item $$\arg \left( \frac{z}{1+i} \right )=\frac{\pi}{4}[2\pi]$$
    \item $$\left|\frac{z+1}{z-2}\right|=1$$
    \item $$\frac{2z-i}{z-2i} \in \mathbb{R}$$
    
\end{enumerate}

\section{Exercice 3 : Polynômes}
\begin{enumerate}
    \item Trouver $a,b,c$ réels tel que $X^2+X+1$ divise $X^4+aX^2+bX+c$.
    \item Déterminer tous les polynômes $P$ qui vérifient :
    $$P(X^2)=(X^2+1)P(X)$$
    \item Décomposer en éléments simples la fraction rationnelle :
    $$F(X)=\frac{X^4}{(X^2-1)(X+3)}$$
\end{enumerate}

\section{Exercice 4: Suites}
Donner une expression en fonction de $n$ des suites suivantes puis calculer leurs limites :
\begin{enumerate}
    \item 
   $$ \begin{cases}
    u_0=2\\
    u_{n+1}=2u_n-6
    \end{cases}$$
    \item $$\begin{cases}
    u_0=1\\
    u_1=0\\
    u_{n+2}=3u_{n+1}-2u_n
    \end{cases}$$
    \item $$\begin{cases}
    u_0=1\\
    u_{n+1}=\frac{4u_n+3}{2+u_n}
    \end{cases}$$
   On montera dans un premier temps, que la suite $(v_n)$ définie par :
    $$v_n=\frac{u_n-3}{u_n+1}$$
    est géométrique.
\end{enumerate}

\section{Exercice 5 : Barycentre}
Sur la figure ci-dessous, $ABCD$ représente une plaque métallique homogène carrée de centre $O$. on retire la partie triangulaire $OAB$ pour obtenir la plaque pentagonale $ADCBO$.\\
\includegraphics[]{barycentres.jpg}
\\
On appelle $H$ le centre d'inertie de la plaque $OAB$ et $G$ celui de la plaque $ADCBO$ que l'on cherche.\\
Justifier dans un premier temps que $O$ barycentre de $(H,1), (G,3)$ et en déduire que $G$ barycentre de $(O,4), (H,-1)$. 

\end{document}