% Classe et options générales
\documentclass[12pt,a4paper]{report}

% Langue et encodage
\usepackage[T1]{fontenc}

\usepackage{lmodern,textcomp}
\usepackage[french]{babel}

% Mise en page
\usepackage{geometry}
\geometry{top=2.5cm,bottom=2cm,left=2cm,right=2cm, headheight=50pt}
\usepackage{fancyhdr}
\pagestyle{fancy}
\fancyhf{}
\lhead{BMC Dijon}
% \includegraphics[width=4em]{images/logotype estp couleur.png}}
\fancyhead[R]{\leftmark} % Affiche le nom du chapitre à gauche
\renewcommand{\chaptermark}[1]{\markboth{\thechapter\quad #1}{}}

\cfoot{\thepage}

\renewcommand{\headrulewidth}{2pt}
\renewcommand{\headrule}{\hbox to\headwidth{\color{orange}\leaders\hrule height \headrulewidth\hfill}}

% Mathématiques et symboles
\usepackage{amsmath,amssymb,amsfonts,mathrsfs,bm}
\usepackage{array,multirow,tabularx}
\usepackage{enumitem}
\usepackage{tablists}
\usepackage{pifont,centernot,eurosym}
\usepackage{multicol}
\usepackage{enumitem}
\usepackage{eurosym}
\usepackage{tabvar}

% Graphiques
\usepackage{graphicx}
\usepackage{pdfpages}
\usepackage{tikz,pgfplots,tkz-tab}
\pgfplotsset{compat=1.18}

\usepackage[tikz]{bclogo}
\usetikzlibrary{angles, quotes, calc, decorations.markings}

% Théorèmes
\usepackage[standard,framed]{ntheorem}

\newcommand{\bclogoboxmaster}[4]{%
  \theoremprework{%
    \vspace{1em}
    \begin{lrbox}{\boitprop}

\begin{minipage}{\dimexpr\textwidth - 4.5em\relax}
        \vspace{0.5em}
  }
  \theorempostwork{%
        \vspace{.3em}
      \end{minipage}
    \end{lrbox}
    \begin{bclogo}[couleur=#1, arrondi=0.1, ombre=true,
      couleurOmbre=black!30, logo=#2]{\hspace{1em}\usebox{\boitprop}}
    \end{bclogo}
  }
  \newtheorem{#3}{#4}[section]%
}

\newcommand{\bclogoboxslave}[4]{%
  \theoremprework{%
    \vspace{1em}
    \begin{lrbox}{\boitprop}
\begin{minipage}{\dimexpr\textwidth - 4.5em\relax}
        \vspace{0.5em}
  }
  \theorempostwork{%
        \vspace{.3em}
      \end{minipage}
    \end{lrbox}
    \begin{bclogo}[couleur=#1, arrondi=0.1, ombre=true,
      couleurOmbre=black!30, logo=#2]{\hspace{1em}\usebox{\boitprop}}
    \end{bclogo}
  }
  \newtheorem{#3}[Thm]{#4}
}

\newcommand{\bclogoboxnonum}[4]{%
  \theoremprework{%
    \vspace{1em}
    \begin{lrbox}{\boitprop}
\begin{minipage}{\dimexpr\textwidth - 4.5em\relax}
        \vspace{0.5em}
  }
  \theorempostwork{%
        \vspace{.3em}
      \end{minipage}
    \end{lrbox}
    \begin{bclogo}[couleur=#1, arrondi=0.1, ombre=true,
      couleurOmbre=black!30, logo=#2]{\hspace{1em}\usebox{\boitprop}}
    \end{bclogo}
  }
  \newtheorem*{#3}{#4}%
}

\newtheorem*{Meth}{\underline{M\'ethode}}

%NOTATION
\theoremprework{\vspace{.5cm} 
    \begin{lrbox}{\boitprop}
     \begin{minipage}{.9\textwidth}
}
\theorempostwork{
     \end{minipage}
    \end{lrbox}
%        \begin{center}
    \begin{tikzpicture}
    \node [draw=violet!10,very thick,fill=blue!10,rectangle% , rounded corners
    , inner sep=10pt, inner ysep=8pt] (box){\usebox{\boitprop}};
    \end{tikzpicture}
 %   \end{center}
}

%NOTATION
\theoremprework{\vspace{.5cm} 
    \begin{lrbox}{\boitprop}
     \begin{minipage}{.9\textwidth}
}
\theorempostwork{
     \end{minipage}
    \end{lrbox}
%        \begin{center}
    \begin{tikzpicture}
    \node [draw=pink!10,very thick,fill=blue!10,rectangle% , rounded corners
    , inner sep=10pt, inner ysep=8pt] (box){\usebox{\boitprop}};
    \end{tikzpicture}
 %   \end{center}
}

% Autres utilitaires
\usepackage{verbatim,lastpage,afterpage,sectsty,color,colortbl,boites,calc,listings, caption}

% Styles de caption
% \AtBeginDocument{
%   \captionsetup[figure]{font=footnotesize,labelsep=period}
%   \captionsetup[table]{font=footnotesize,labelsep=newline,justification=centering}
% }
% Définition de blocs théorèmes avec bclogo
\newsavebox{\boitprop}


\bclogoboxmaster{orange!20}{\bcbook}{Thm}{Théorème}
\bclogoboxslave{teal!15}{\bcfleur}{Def}{Définition}
\bclogoboxslave{cyan!15}{\bcetoile}{Prop}{Proposition}
\bclogoboxnonum{red!10}{\bcfeuvert}{Rmq}{Remarque}

% Environnements personnalisés (exemples, vocabulaire, etc.)
% Exemple :
% \newenvironment{Ex}
% {\vspace{.5cm}\begin{tabular}{c|r}
%     \arrayrulecolor{black} \textbf{Exemple} &  \begin{minipage}[l]{\textwidth}}{
% \end{minipage}\end{tabular}\vspace{.2cm}}

% \newenvironment{Ex}{%
%   \vspace{.5cm}%
%   \begin{tabularx}{\textwidth}{c|X}
%     \arrayrulecolor{black} \textbf{Exemple} & 
% }{%
%   \\ % pour finir la ligne du tableau
%   \end{tabularx}%
%   \vspace{.2cm}%
% }
\usepackage[most]{tcolorbox}

\newtcolorbox{Ex}[1][]{
  colback=white,
  colframe=black,
  boxrule=1pt,
  left=5pt,
  rounded corners,
  before skip=10pt,
  after skip=10pt,
  title={\textbf{Exemple}},
  fonttitle=\bfseries,
  #1
}
% \newenvironment{Ex}{
% \vspace{.5cm}
% \noindent\textbf{Exemple.}\par
% \vspace{.3cm}
% }{\vspace{.3cm}}
% % Préambule
% \usepackage{xcolor}


% Définition du compteur
\newcounter{exocounter}

% Logo de difficulté pour les exercices (niveaux numériques 1/2/3)
\newcommand{\exologo}[1]{%
  \ifcase#1\relax
  \or \includegraphics[height=1em]{fourmi.png}% 1
  \or \includegraphics[height=1.2em]{abeille.png}% 2
  \or \includegraphics[height=1.5em]{castor.png}% 3
  \fi
}

% Commande \exo avec argument optionnel pour le niveau de difficulté
\newcommand{\exo}[2][]{%
  \refstepcounter{exocounter}%
  \vspace{1em}
  \noindent
  \textcolor{orange}{\rule{1ex}{1ex}}%
  \hspace{0.5em}%
  \textbf{\textcolor{orange}{Exercice \theexocounter}}%
  \hspace{0.5em}%
  % Affiche le logo uniquement si un niveau est donné
  \if\relax\detokenize{#1}\relax\else
    \exologo{#1}%
    \hspace{0.5em}%
  \fi
  \textbf{#2}%
  \par\vspace{0.5em}
}
\makeatletter
\@addtoreset{exocounter}{chapter}
\makeatother
\renewcommand{\theexocounter}{\thechapter.\arabic{exocounter}}

% Raccourcis mathématiques
\newcommand{\C}{\mathbb{C}} \newcommand{\R}{\mathbb{R}} \newcommand{\N}{\mathbb{N}} \newcommand{\Z}{\mathbb{Z}} \newcommand{\K}{\mathbb{K}}
\newcommand{\ds}{\displaystyle} \newcommand{\eps}{\epsilon} \renewcommand{\phi}{\varphi}
\newcommand{\abs}[1]{\left|#1\right|} \newcommand{\p}[1]{\left(#1\right)} \newcommand{\e}[1]{\left\{#1\right\}}

% Commandes logiques et limites
\newcommand{\limni}{\ds\lim_{n\ra+\infty}} \newcommand{\limxz}{\ds\lim_{x\ra 0}} \newcommand{\tend}[1]{\underset{#1}{\longrightarrow}}
\newcommand{\dx}{\, \mathrm{d}x}\newcommand{\dt}{\, \mathrm{d}t}\newcommand{\du}{\, \mathrm{d}u}

% Listings Python (extrait simplifié)
\lstset{language=Python,basicstyle=\ttfamily,commentstyle=\color{green!50!black},keywordstyle=\color{blue},stringstyle=\color{olive},frame=single,rulecolor=\color{blue},numbers=left,stepnumber=1,numbersep=8pt,showstringspaces=false}



\usepackage[sf]{titlesec}
\titleformat{\section}{\sffamily\large\bfseries}{\thesection}{1em}{}
\titleformat{\subsection}
  {\sffamily\normalsize\bfseries}
  {\thesubsection}
  {1em}
  {}

\renewcommand{\familydefault}{\sfdefault}

\titleformat{\chapter}[display]
  {\normalfont\bfseries\centering\Huge}
  {\chaptername\ \thechapter}
  {1ex}
  {}

\usepackage[colorlinks=true, linkcolor=black]{hyperref}
\usepackage{bookmark}
\usepackage{subfiles}


\lhead{BMC1 Dijon}


\begin{document}



\includepdf[pages=-,pagecommand={\thispagestyle{empty}}]{images/front.pdf}

\pagenumbering{roman}  % Numérotation romaine pour la table des matières
\tableofcontents

\newpage
\pagenumbering{arabic}  % Recommence à 1 en chiffres arabes


\chapter{Introduction}
Ce polycopié regroupe l'ensemble des contenus mathématiques qui vous seront présentés lors de ce premier semestre dans l'enseignement supérieur.

Vous y trouverez des concepts que certains d'entre vous connaissent déjà comme les vecteurs, la fonction exponentielle, ou les polynômes. Les premiers chapitres de ce poly servent de révision, vous ne pourrez pas acquérir les compétences de cette formation si vous n'êtes pas à l'aise avec la \textbf{manipulation des fractions} ou des puissances.  

Cependant, la manière de voir les choses sera différente du lycée. Le but est ici de renforcer un socle sur lequel vous bâtirez les compétences techniques des autres modules. Quelques exemples: 

\begin{itemize}
\item Dans le cours de mécanique des sols, vous manipulerez des échelles logarithmiques, vous aurez pour cela besoin d'être familier avec les \textbf{fonctions exponentielle et logarithme}.
Dans la partie Hydrogéologie, vous manipulerez des cartes pour visualiser les écoulements d'eau. Le calcul différentiel pour comprendre la théorie derrière ces lignes d'écoulement vous sera présenté au deuxième semestre avec les fonctions de plusieurs variables. 
\item  Durant le cours de stabilité des ouvrages, vous linéariserez des équations complexes pour vous ramener à l'étude de \textbf{systèmes linéaires}. Vous calculerez des descentes de charges, être à l'aise avec les \textbf{barycentres} vous sera d'une grande utilité
\item Les \textbf{vecteurs} seront l'objet central du cours de Mécanique générale, vecteur position, vecteur vitesse, champs de vitesse, forces de frottement. 
\end{itemize}

D'autres parties de ce cours vous prépareront à la manipulation d'outils mathématiques plus poussés. Les \textbf{suites réelles} par exemple, pourront vous permettre dans un premier temps de modéliser certains phénomènes physiques (concentration des gaz à effet de serre au cours du temps par exemple) mais elles seront aussi une base pour l'étude des séries numériques en deuxième année, puis des séries de fonctions en troisième année. Un des buts ultimes de ces séries de fonctions sera de comprendre les séries de Fourier, objet central de la manipulation de signaux (calculer l'effet d'un séisme sur un immeuble, compresser des images en JPEG, réduire le bruit de fond sur les appels Teams, ...)


Pour certains chapitres, vous ne verrez pas leur utilité immédiate, ils vous présenteront en fait des concepts abstraits qui font le lien entre plusieurs autres chapitres. Creusez ces liens, ce sont eux qui vous permettront de comprendre les choses en profondeur. Les \textbf{nombres complexes} vous serviront plus tard, notamment pour comprendre ce que les nombres réels ne peuvent pas décrire. Ils font aussi le lien entre la trigonométrie, la géométrie des vecteurs, les polynômes. 


% Chapitres de révisions : 

% Ajouter un doc avec des QCM sur la manipulation de fractions, de puissances 
\chapter{Manipulation de fractions, puissances}
\subfile{chapitres/K1-fractions}

\chapter{Exponentielle et Logarithme népérien}
\subfile{chapitres/K2-expoln}

\chapter{Systèmes linéaires simples}
\subfile{chapitres/K3-systLin}


% Karine - Trigo
\chapter{Trigonométrie}
\subfile{chapitres/K4-trigo}

% Remi - Logique, sommes, produits
\chapter{Logique, sommes, produits}
\subfile{chapitres/R1-logique}
\subfile{chapitres/R1-sommes}

% Remi - Vecteurs
\chapter{Calcul vectoriel}
\subfile{chapitres/R2-vecteurs}

% % Karine - complexes
\chapter{Nombres complexes}
\subfile{chapitres/K5-complexe}

% Remi - Suites réelles
\chapter{Suites réelles}
\subfile{chapitres/R3-suites}

% Karine - Barycentre
\chapter{Barycentres}
\subfile{chapitres/K6-barycentre}


% ------ Exam P1 et P2 ------

% % Remi - Polynomes
\chapter{Polynômes}
\subfile{chapitres/R4-polynomes}

\chapter{Annales 2024-2025}
Voici les sujets qui ont été posés lors des contrôles continus de l'année 2024-2025.

\section{Continuité}


On définit une fonction de cette manière :
\[f(x)=
\left\{
\begin{array}{cc}
(x-2)^2 &\text{ si } x<5\\
a&\text{ si } x=5\\
(2x-b)^2&\text{ si } x>5 \\
\end{array}
\right.
\]


\begin{enumerate}
     \item 
        Déterminer $a$ et $b$ pour que $f$ soit continue sur $\mathbb{R}$.\\
        
\vspace{1em}

        \textbf{Correction:}\\

        $ \underset{{x\overset{x<5}{\to} 5}}{\lim}((x-2)^2)=9$\\
        Pour que $f$ soit continue en $5$ à  gauche, il faut et il suffit que $a=9$.\\
        $ \underset{{x\overset{x>5}{\to} 5}}{\lim}((2x-b)^2)=(10-b)^2$\\
        Pour que $f$ soit continue en $5$ à  droite, il faut et il suffit que $(10-b)^2=9$.\\
        On obtient $b=7$ ou $b=13$.
        
        La fonction $f$ est prolongeable par continuité en $0$ en une focntion $\tilde{f}$ avec $\tilde{f}(5)=9$.
        
    \item La fonction $f$ est-elle dérivable au point $x=5$ ?\\
    
    Limite du taux d'accroissement à gauche:\\
     $ \underset{{x\overset{x<5}{\to} 5}}{\lim}\left(\dfrac{(x-2)^2-9}{x-5}\right)= \underset{{x\overset{x<5}{\to} 5}}{\lim}\left(\dfrac{(x+1)(x-5)}{x-5}\right)=6$\\
     
     Limite du taux d'accroissement à droite:\\
     $ \underset{{x\overset{x<5}{\to} 5}}{\lim}\left(\dfrac{(2x-b)^2-9}{x-5}\right)= \underset{{x\overset{x<5}{\to} 5}}{\lim}\left(\dfrac{(2x-b-3)(2x-b+3)}{x-5}\right)$\\
     
     Si $b=7$, $\underset{{x\overset{x<5}{\to} 5}}{\lim}\left(\dfrac{(2x-7-3)(2x-7+3)}{x-5}\right)=2\times 6=12$\\
          Si $b=13$, $\underset{{x\overset{x<5}{\to} 5}}{\lim}\left(\dfrac{(2x-13-3)(2x-13+3)}{x-5}\right)=-6\times 2=-12$\\
          
          
          $\tilde{f}$ n'est pas dérivable en $5$, comme les limites du taux d'accroisssement à gauche et à droite ne sont pas égales.\\
\end{enumerate}


\noindent\textbf{Exercice  : 2 points}\\



On considère les fonctions $f$ et $g$ définies par les expresssions ci-dessous:\\
$f(x)=\dfrac{1}{x-4}$ et $g(x)=\ln(2x-1)$

\begin{enumerate}
     \item 
        Déterminer les ensembles de définition de $f$ et $g$.\\
        
         La fonction $f$ est définie sur $\mathbb{R}- \{4\}$.  La fonction $f$ est définie sur $\left ] \dfrac{1}{2};+\infty\right[$
    \item Déterminer la fonction $f\circ g$ et son ensemble de définition.\\
    
    La fonction $f\circ g$ est définie par $f\circ g(x) =\dfrac{1}{\ln(2x-1)-4}$ sur l'ensemble des réels tels que $\ln(2x-1)-4\neq0$ et $2x-1\geq 0$.
\end{enumerate}

\section{Continuité, 2eme version}



On définit une fonction de cette manière :
\[f(x)=
\left\{
\begin{array}{cc}
\dfrac{x^2-4}{x-2} &\text{ si } x<2\\

x+2&\text{ si } x>2 \\
\end{array}
\right.
\]


\begin{enumerate}
     \item 
        Démontrer que $f$ est continue en $2$.
        
       
        
       % La fonction $f$ est prolongeable par continuité en $0$ en une focntion $\tilde{f}$ avec $\tilde{f}(5)=9$.
        
    \item La fonction $f$ est-elle dérivable au point $x=2$ ?\\
    
 
\end{enumerate}


\noindent\textbf{Exercice  : 2 points}\\



On considère les fonctions $f$ et $g$ définies par les expresssions ci-dessous:\\
$f(x)=\ln(3x-3)$ et $g(x)=\dfrac{1}{2x+5}$

\begin{enumerate}
     \item 
        Déterminer les ensembles de définition de $f$ et $g$.\\
        
 
    \item Déterminer la fonction $f\circ g$ et son ensemble de définition.\\
    
\end{enumerate}

\section{Equations Différentielles}


Résoudre les équations différentielles suivantes


\begin{enumerate}
    \item 
    $y''+y=0$.
    \item
    $z''-6z+9=0$.
    \item
    $3y''-4y'=0$.
\end{enumerate}

\section{Intégrale double}


Calculer l'intégrale suivante après avoir tracé le domaine

\begin{equation}
I = \int\int_D e^{-x-y} \,dx\,dy, \quad \text{où } D = \{(x, y) \in \mathbb{R}^2 \mid 0 \leq x \leq 1, \ 1 \leq y \leq x+3\}.
\end{equation}




%\section*{Correction}
%1 point tracé
%1 point ordre d'intégration
%1 point primitive
% 2 points fin de calcul
%\begin{center}
%\begin{tikzpicture}[scale=1.5]
%    % Axes
%    \draw[->] (-0.5,0) -- (2,0) node[right] {$x$};
%    \draw[->] (0,-0.5) -- (0,4.5) node[above] {$y$};
%    
%    % Domaine
%    \fill[blue!20, opacity=0.5] (0,1) -- (1,4) -- (1,1) -- cycle;
%    
%    % Bords du domaine
%    \draw[thick, blue] (0,1) -- (1,4);
%    \draw[thick, blue] (1,1) -- (1,4);
%    \draw[thick, blue] (0,1) -- (1,1);
%    
%    % Labels
%    \node[left] at (0,1) {1};
%    \node[below] at (1,0) {1};
%    %\node[right] at (1,4) {$(1,4)$};
%    
%    % Equation de la droite
%    \draw[dashed] (0,3) -- (1,4);
%    \node[right] at (0.5,3.5) {\small $y = x+3$};
%\end{tikzpicture}
%\end{center}

%\section*{Calcul de l'intégrale}
%
%L'intégrale double est donnée par :
%\begin{align*}
%I &= \int_0^1 \int_1^{x+3} e^{-x-y} \, dy \, dx.
%\end{align*}
%
%Calculons l'intégrale intérieure :
%\begin{align*}
%\int_1^{x+3} e^{-x-y} \, dy &= e^{-x} \int_1^{x+3} e^{-y} \, dy \\
%&= e^{-x} \left[ -e^{-y} \right]_1^{x+3} \\
%&= e^{-x} \left( -e^{-(x+3)} + e^{-1} \right) \\
%&= e^{-x} e^{-1} - e^{-x} e^{-(x+3)} \\
%&= e^{-x-1} - e^{-2x-3}.
%\end{align*}
%
%Intégrons maintenant par rapport à \( x \) :
%\begin{align*}
%I &= \int_0^1 \left( e^{-x-1} - e^{-2x-3} \right) dx \\
%&= e^{-1} \int_0^1 e^{-x} \,dx - e^{-3} \int_0^1 e^{-2x} \,dx.
%\end{align*}
%
%Les intégrales élémentaires donnent :
%\begin{align*}
%\int_0^1 e^{-x} \,dx &= \left[ -e^{-x} \right]_0^1 = 1 - e^{-1}, \\
%\int_0^1 e^{-2x} \,dx &= \left[ -\frac{e^{-2x}}{2} \right]_0^1 = \frac{1 - e^{-2}}{2}.
%\end{align*}
%
%Ainsi, on obtient :
%\begin{align*}
%I &= e^{-1} (1 - e^{-1}) - e^{-3} \frac{1 - e^{-2}}{2} \\
%&= e^{-1} - e^{-2} - \frac{e^{-3} (1 - e^{-2})}{2} \\
%&= e^{-1} - e^{-2} - \frac{e^{-3} - e^{-5}}{2}.
%\end{align*}

\section{}


% Liens vers les autres projets : 
% S1 : https://fr.overleaf.com/4534764743nhkfvmwwrbpy#55af2b
% S2 : https://fr.overleaf.com/4128697822vdfqstywdjmx#4c6293

% S3 : https://fr.overleaf.com/4578387271pywhzmryfgtw#642977
% S4 : https://fr.overleaf.com/1782189891wyfhvccrfxsm#b018ec

% S5 : https://fr.overleaf.com/2583585623qjvnxpgbsrdc#d6d95a
\end{document}