\documentclass[../PolyS1.tex]{subfiles}
\begin{document}

 Les fonctions \textbf{exponentielles} peuvent être utilisées dans de nombreuses applications, par exemple pour le calcul des intérêts en économie, la croissance d'une population, la dilatation thermique des matériaux de construction, etc. 


Avant l'invention des calculatrices scientifiques, les \textbf{logarithmes} étaient utilisés sous forme de tables logarithmiques pour effectuer de nombreux calculs.
Le mérite de l'invention des logarithmes revient à John Napier (d'où le qualificatif de \textbf{népérien}), bien que de nombreux scientifiques et mathématiciens aient contribué à la forme finale que nous utilisons aujourd'hui.


\section {La fonction Exponentielle}



\subsection{Définition de la fonction exponentielle}

Malgré son importance, cette fonction n'est pas facile à définir, nous admettons son existence pour le moment. 


\begin{Thm}\textbf{Existence de la fonction exponentielle}
  \vspace{1em}

\indent Il existe une unique fonction f dérivable sur $\R$ telle que :
$$
f (0) = 1\quad \text{ et,    pour tout nombre réel }x\quad  f^{\prime}(x) = f (x)\, .
$$

Cette fonction est appelée la fonction exponentielle et notée $\exp$.
$$
\exp:
\begin{array}{ccl}
\mathbb{R}&\longrightarrow&\mathbb{R}_+\\
x&\longmapsto &e^x\\
\end{array}
$$
On peut noter indifféremment $\exp(x)$ ou $e^x$
\end{Thm}

\newpage 

Le graphe de la fonction exponentielle :
\begin{center}
\includegraphics[]{images/pic.pdf}
\end{center}
% \begin{center}
% \includegraphics[scale=0.4]{expo.png}
% \end{center}

%
\begin{Rmq}$\,$
  
  La fonction exponentielle:
\begin{itemize}
\item est définie sur $\R$, continue, dérivable et strictement croissante.
\item est strictement positive sur $\R$ : $\forall x\in \R\quad e^x>0$
\item vaut $1$ au point $0$ : $\exp(0) =e^0= 1$
\item tend vers $0$ en moins l'infini : $\displaystyle\lim_{x\to -\infty } e^x = 0 $
\item tend vers l'infini en plus l'infini : $\displaystyle\lim_{x\to +\infty } e^x = +\infty$ 
\end{itemize}
\end{Rmq}
 
\subsection{Propriétés de la fonction exponentielle}

La fonction exponentielle se comporte comme une fonction puissance : 
\begin{Prop}$\,$
  \vspace{1em}

Pour tous $(a, b)$ deux nombres réels, pour tout $n\in \mathbb{N}$ :
\begin{multicols}{2}
\begin{itemize}
  \item[] $e^{a+b}=e^a\times e^b$
  \item[] $e^{-a}=1/e^a$
  \item[] $e^{a-b}=e^a/e^b$
  \item[] $(e^{a})^n=e^{na}$
\end{itemize}
\end{multicols}
\end{Prop}



 \begin{Prop}\textbf{Croissances comparées }
  \vspace{1em}

 La fonction exponentielle dicte son comportement lorsqu'elle est comparée à \newline 
 des fonctions polynomiales: pour tout entier $n\in \mathbb{N}$
$$\lim_{x \to -\infty}  x^n e^x  = 0, \qquad \qquad
\lim_{x \to +\infty} \frac{e^x}{x^n} = +\infty\qquad \qquad$$
\end{Prop}

%  \begin{Prop}\textbf{Taux d'accroissement}\\
% $$\lim_{x\to 0 }\left( \dfrac{e^x-1}{x-0}\right) = 1 $$
% \end{Prop}
\newpage

\subsection{QCM}
\begin{enumerate}
\item Que vaut le produit $e^3\times e^4$ ?
\begin{multicols}{4}
\begin{enumerate}[label=\alph*.]
\item   $e^{12}$
\item   $e^7$
\item   $e^{34}$
\item   $e^{3^4}$
\end{enumerate}
\end{multicols}

\item
La dérivée de la fonction $f$ définie par $f(x)=e^x$ est :
\begin{multicols}{2}
\begin{enumerate}[label=\alph*.]
\item
    $f'(x)=e^x$
\item
    
    $f'(x)=x$
\item
    
    $f'(x)=xe^x$
    \item
    
    $f'(x)=\ln(x)$
\end{enumerate}    
\end{multicols}

\item L'équation $e^x=3$
\begin{multicols}{2}

\begin{enumerate}[label=\alph*.]
\item  n'admet pas de solutions dans $\R$
\item  a une unique solution: $x=3$
\item  a une unique solution: $x=\ln(3)$
\item  a une unique solution:  $x=e^3$
\end{enumerate}   
\end{multicols}

\item Que vaut $\displaystyle \lim_{x\to -\infty}x^4e^{x}$ ?
\begin{multicols}{2}

\begin{enumerate}[label=\alph*.]
\item  $0$
\item  $+\infty$
\item  $-\infty$
\item  $4$
\end{enumerate}   
\end{multicols}

\item Que vaut $\displaystyle \lim_{x\to -\infty}\dfrac{1-e^x}{x}$ ?
\begin{multicols}{2}

\begin{enumerate}[label=\alph*.]
\item  $0$
\item  $+\infty$
\item  $-\infty$
\item  $1$
\end{enumerate}   
\end{multicols}

\end{enumerate}   
 
\section{La fonction Logarithme népérien}
\subsection{Définition de la fonction logarithme népérien}
 \begin{Thm}\textbf{La fonction logarithme népérien}
  \vspace{1em}

La fonction exponentielle réalise une bijection strictement croissante de $\R$ sur \newline 
l'intervalle $] 0 ;+\infty [$, c'est-à-dire que tout nombre réel strictement positif $y$ peut s'écrire
comme une exponentielle. 

$$\forall y >0 \quad \exists x\in\R\quad y=e^x $$

La fonction Logarithme népérien est la réciproque de la fonction exponentielle,\newline 
on la note $\ln$
$$
\ln :
\begin{array}{ccl}
\mathbb{R}_+^*&\longrightarrow&\mathbb{R}\\
x&\longmapsto &\ln(x)\\
\end{array}
$$
à tout nombre réel strictement positif $x$, elle associe le nombre dont l'exponentielle \newline vaut $x$, ce nombre est noté $\ln x$
\end{Thm}

\newpage 

Le graphe de la fonction logarithme népérien :
\begin{center}
\includegraphics[]{images/log.pdf}
\end{center}
Comme les deux fonctions exponentielle et logarithme népérien sont réciproques l'une de l'autre, leurs graphes sont symétriques par rapport à la droite d'équation $y=x$.

% \begin{center}
% \includegraphics[scale=0.6]{expoLn.png}
% \end{center}


\begin{Rmq}$\,$

La fonction Logarithme népérien : 
\begin{itemize}
\item est définie sur $\R_+^\star$, continue, dérivable et strictement croissante.
\item n'est pas définie au point $0$ ni pour les nombres réels négatifs.
\item vaut $0$ au point $1$ : $\ln(1) = 0 $
\item tend vers moins l'infini en $0$ : $\displaystyle \lim_{x\to 0 } \ln x = -\infty $
\item tend vers l'infini en plus l'infini : $\displaystyle \lim_{x\to +\infty } \ln x = +\infty$ 
\end{itemize}
\vspace{1em}

Pour tout nombre réel $x$ et tout nombre réel strictement positif $y$, \newline 
on a les propriétés suivantes :
\begin{multicols}{2}
\begin{itemize}
  \item[] $e^x=y \Longleftrightarrow x=\ln(y)$
  \item[] $e^x>y \Longleftrightarrow x>\ln(y)$
  \item[] $e^{\ln(y)}=y$
  \item[] $\ln(e^x)=x$
\end{itemize}
\end{multicols}

\end{Rmq}

Ces fonctions permettent également d'étendre la définition des puissances :
$$\forall a >0 \quad \forall b \in \R \quad a^b = \exp(b\ln a)$$

\newpage 

\subsection{Propriétés de la fonction logarithme népérien}

\begin{Rmq}$\,$

La fonction logarithme népérien est dérivable sur $\left] 0;+\infty \right[$ et sa dérivée est la fonction inverse :
\[\left(\ln{x}\right)'=\dfrac{1}{x}
\]

\end{Rmq}

Le logarithme transforme les produits en somme, il était utilisé par les astronomes du 16e siècle pour accélérer les multiplications de nombres à 10 chiffres.
 \begin{Prop}\textbf{Propriétés algébriques}
  \vspace{1em}

Pour tous $(a, b)$ deux nombres réels strictement positifs, pour tout $n\in \mathbb{N}$ :
\begin{multicols}{2}
    \begin{itemize}
  \item[] $\ln(ab)=\ln(a)+\ln(b)$
  \item[] $\ln\left(1/a\right)=-\ln(a)$.    
    \item[] $\ln\left(a/b\right)=\ln(a)-\ln(b)$. 
    \item[] $\ln(\sqrt{a}\, )=\dfrac{1}{2}\ln(a)$ 
    \item[] $\,$ 
        \item[] $\ln(a^n)=n\ln(a)$ 
  \end{itemize}
\end{multicols}



\end{Prop}
Il existe d'autres fonctions logarithmes, par exemple la fonction logarithme en base $10$, $\log_{10}$ qui vérifie les mêmes propriétés de calcul mais qui est définie par : 
$$
\log_{10} (10) = 1\,. 
$$



\vspace{1em}
\hrule
\vspace{1em}

\exo[1]{Application directe}

\begin{enumerate}
\item Résoudre les équations: 
\begin{multicols}{2}
    \begin{enumerate}
    \item[] $(E_1)\quad e^x=5$
    \item[] $(E_2)\quad \ln(x)=-5$
    \item[] $(E_3)\quad \ln(2x-1)=-2$
    \item[] $(E_4)\quad \ln(1+x)=100$
\end{enumerate}
\end{multicols}



\item Résoudre les syst\`emes:  

\[
\mathcal{S}_1
\left\{
\begin{array}{rcl}
 -\ln x + 2\ln y & = & 1 \\
  3\ln x - 5\ln y & = & -1
\end{array}
\right.
\qquad\qquad
\mathcal{S}_2
\left\{
\begin{array}{rcl}
 -2\ln x + 3\ln y & = & -1 \\
 -7\ln x - 8\ln y & = & 1
\end{array}
\right.
\]
\end{enumerate}


\vspace{1em}
\hrule
\vspace{1em}

 \begin{Prop}\textbf{Croissances comparées }
  \vspace{1em}

 Les fonctions puissances dictent leur comportement lorsqu'elles sont\newline  comparées à 
 la fonction logarithme: 
 pour tout entier $n\in \mathbb{N}$
$$\lim_{x\to 0 } x^n\ln(x) = 0 \qquad \qquad 
\lim_{x\to +\infty } \dfrac{\ln(x)}{x^n} = 0\qquad \qquad$$

\end{Prop}

%   \begin{Prop}\textbf{Taux d'accroissement}
% $$\lim_{x\to 1 }\left( \dfrac{\ln(x)}{x-1}\right) = 1 $$
% $$\lim\limits_{h \rightarrow 0} \left(\dfrac{\ln(1+h)}{h}\right)=1$$
% \end{Prop}


\subsection{QCM}
\begin{enumerate}[itemsep=2ex]
\item Quelle est la valeur de la somme $\ln(4)+\ln(10)$ ?
\begin{multicols}{2}
\begin{enumerate}[label=\alph*.]
\item   $\ln(14)$
\item   $\ln(40)$
\item   $\ln(4/10)$
\item   $\ln(10^4)$
\end{enumerate}
\end{multicols}

\item
La dérivée de la fonction $f$ définie par $f(x)=\ln(x)$ est :
\begin{multicols}{2}
    \begin{enumerate}[label=\alph*.]
\item $f'(x)=x$
\item $f'(x)=\dfrac{1}{x}$
\item $f'(x)=e^x$
    \item $f'(x)=\ln(x)$
\end{enumerate}  
\end{multicols}
  

\item L'équation $\ln(x)=-8$ 
\begin{multicols}{2}

\begin{enumerate}[label=\alph*.]
\item  n'admet pas de solutions dans $\R_+^\star$
\item  a une unique solution $x=-8$
\item  a une unique solution $x=-\ln(8)$
\item  a une unique solution $x=e^{-8}$
\end{enumerate}   
\end{multicols}

\item Que vaut $\displaystyle \lim_{x\to 0}x^2\ln(x^3)$ ? 
\begin{multicols}{2}

\begin{enumerate}[label=\alph*.]
\item  $0$
\item  $+\infty$
\item  $-\infty$
\item  $x^6$
\end{enumerate}   
\end{multicols}

\item Que vaut $\displaystyle \lim_{h\to 0}\dfrac{\ln(1+h)}{h}$ ?
\begin{multicols}{2}
\begin{enumerate}[label=\alph*.]
\item  $0$
\item  $+\infty$
\item  $-\infty$
\item  $1$
\end{enumerate}  
\end{multicols}

\end{enumerate}   

%

\newpage 

\section {Exercices}

\vspace{1em}
\hrule
\vspace{1em}


\exo[1]{Positionnement}
\begin{enumerate}
\item Déterminer, en fonction de $x$, le signe des fonctions suivantes :
\begin{multicols}{2}
\begin{enumerate}[label=\alph*)]
\item $f$ définie sur $\R$ par $$f(x)=(x^2+4)e^x$$ 
\item $g$ définie sur $\R$ par $$g(x)=\dfrac{e^{-4x}}{-x^4-7}$$ 
\item $h$ définie sur $\R$ par $$h(x)=(1+e^{2x})(e^{-3x}+4)$$ 
\item $i$ définie sur $\R$ par $$i(x)=(x^2-x-6)e^{x}$$
\end{enumerate}
\end{multicols}
\item Dans chacun des cas les fonctions sont  dérivables sur $\R$, déterminer leur dérivée :
\begin{multicols}{2}
\begin{enumerate}[label=\alph*)]
\item $f$ définie par $f(x)=e^{2x}$ 
\item $g$ définie par $g(x)=e^{4x}$ 
\item $h$ définie par $h(x)=e^{3x+4}$ 
\item $i$ définie par $i(x)=e^{5x-2}$ 
\item $j$ définie par $j(x)=e^{-7x+1}$ 
\item $k$ définie par $k(x)=e^{-6x-3}$ 
\end{enumerate}
\end{multicols}
\item Dans chacun des cas les fonctions sont  dérivables sur $\R$, déterminer leur dérivée :
\begin{multicols}{2}
\begin{enumerate}[label=\alph*), itemsep=1.2ex]
\item $f$ définie par $f(x)=e^x+x^2$ 
\item $g$ définie par $g(x)=3e^{x}+4x+5$ 
\item $h$ définie par $h(x)=3x^2-4e^x$ 
\item $i$ définie par $i(x)=xe^{x}$ 
\item $j$ définie par $j(x)=(1+x)e^{x}$ 
\item $k$ définie par $k(x)=3x^2-4e^x$ 
\item $l$ définie par $l(x)=\dfrac{x^2+5}{e^{x}}$ 
\end{enumerate}
\end{multicols}
\item Résoudre dans $\R$ les équations suivantes :
\begin{multicols}{3}
\begin{enumerate}[label=\alph*)]
\item $e^3=e^x$ 
\item $e^x-e^{-4}=0$ 
\item $e^x=1$ 
\item $e^x-e=0$ 
\item $e^{2x+4}=e^2$ 
\item $e^x=3$
\item $e^x+5=0$ 
\item $e^{-3x+5}=1$ 
\item $e^x=0$ 
\end{enumerate}
\end{multicols}
\item Résoudre dans $\R$ les inéquations suivantes :
\begin{multicols}{3}
\begin{enumerate}[label=\alph*), itemsep=1.2ex]
\item $e^x>e^4$ 
\item $e^{-2x}>1$ 
\item $e^{2x}<6$ 
\item $e^{3x+4}\geq e^{13}$ 
\item $e^{-2x+5}\leq e^9$ 
\item $e^{3x+5}\leq e^{6x-1}$ 
\item $e^{x}\leq 4$ 
\item $e^{2x+3}>4$
\end{enumerate}
\end{multicols}
\end{enumerate}

\vspace{1em}
\hrule
\vspace{1em}

\newpage

\exo[1]{QCM}

\begin{enumerate}[itemsep=2ex]
\item
Pour tout nombre réel $x$, $
A(x)=1-\dfrac{e^{-x}-1}{e^{-x}+1}$ s'écrit également :

\begin{multicols}{3}
\begin{enumerate}[label=\alph*)]
\item
    $\dfrac{2}{e^{-x}+1}$
\item
    $\dfrac{
    2e^{-x}}{e^{-x}+1}$
\item
    $2$
\end{enumerate}    
\end{multicols}

\item La limite $\displaystyle \lim_{x\to +\infty }\left(\dfrac{e^{-x}+3}{1+2e^{-x}}\right)$ est égale \`a :
\begin{multicols}{3}
\begin{enumerate}[label=\alph*)]
\item  $3$
\item      $+\infty$
\item $ \dfrac{3}{2}$

%
\end{enumerate}  
\end{multicols}

\item $f$ est la fonction définie et dérivable sur $]-1;+\infty[$ par :
$f(x)=\dfrac{-2e^x}{x+1}$ alors sa dérivée
$f'$  s'écrit :
\begin{multicols}{3}
\begin{enumerate}[label=\alph*)]
\item   $f'(x)=\dfrac{-2e^x}{(1+x)^2}$
\item   $f'(x)=\dfrac{-2xe^x}{(1+x)^2}$
\item   $f'(x)=\dfrac{2e^x}{(1+x)^2}$
\end{enumerate}
\end{multicols}

\item Le nombre $e^3(e^{-2})^5$ est égal \`a :
\begin{multicols}{3}
\begin{enumerate}[label=\alph*)]
\item   $ e^6$
\item $e^{-30}$
\item $ e^{-7}$
\end{enumerate}
\end{multicols}

\item Pour tout réel $x$  différent de $-\ln(2)$, la fraction $\dfrac{3-e^{-x-2}}{2e^{-x}-1}$ peut s'écrire aussi :
\begin{multicols}{2}
\begin{enumerate}[label=\alph*), itemsep=1.3ex]
\item    $\dfrac{1-5e^x}{1+2e^x}$
\item    $\dfrac{1+5e^x}{1-2e^x}$
\item    $\dfrac{1-5e^x}{1-2e^x}$
\item    $\dfrac{1-5e^x}{-1+2e^x}$
\end{enumerate}   
\end{multicols} 
    \end{enumerate}
    
\vspace{1em}
\hrule
\vspace{1em}

\exo[1]{Puissances}

Soit $n$ un nombre entier relatif. Simplifier les écritures suivantes: 
\begin{multicols}{3}
\begin{enumerate}[label=\alph*), itemsep=1.5ex]
\item $2^{2n}\times 2$ 
\item $ \dfrac{2^{3n+1}}{2^{2n+1}}$
\item\ $\left(2^{n+1}\right)^3\times 2^{-1}$
\item $ \dfrac{4^{n+2}}{2^{2n}}\times\dfrac{1}{8}$
\item $\dfrac{2^{n+3}}{4^{-n}}\times 2^{-n}$
\end{enumerate}
\end{multicols}



\vspace{1em}
\hrule
\vspace{1em}

\exo[1]{Parité et fonction exponentielle}

Étudier la parité des fonctions suivantes :
$$f_1(x)=e^x-e^{-x},\qquad f_2(x)=\frac{e^{2x}-1}{e^{2x}+1},\qquad f_3(x)=\frac{e^x}{(e^x+1)^2}.$$


\vspace{1em}
\hrule
\vspace{1em}

\exo[2]{Dérivées et étude de fonctions avec $\ln$}

\begin{enumerate}
\item Calculer la dérivée des fonctions suivantes sur leur ensemble de définition :
\begin{multicols}{3}
\begin{enumerate}[label=]
\item $f(x)=\ln(2x+1)$
\item $g(x)=\ln(x^2+1)$
\item $h(x)=x\ln(x)$
\item $k(x)=\dfrac{\ln(x)}{x}$
\item $\ell(x)=\ln(\ln(x))$
\item $m(x)=\ln\left(\dfrac{x+1}{x-1}\right)$
\end{enumerate}
\end{multicols}
\item Étudier les variations de la fonction $f$ définie sur $]0,+\infty[$ par $f(x)=x-\ln(x)$.
\item Étudier les variations de la fonction $g$ définie sur $]0,+\infty[$ par $g(x)=\dfrac{\ln(x)}{x^2}$.
\end{enumerate}

\vspace{1em}
\hrule
\vspace{1em}

\exo[2]{Limites et inéquations avec $\ln$}

\begin{enumerate}
\item Déterminer les limites suivantes :
\begin{multicols}{3}
\begin{enumerate}[label=]
\item $\displaystyle\lim_{x\to +\infty}\frac{\ln(x)}{x}$
\item $\displaystyle\lim_{x\to +\infty}\frac{\ln(x)}{\sqrt{x}}$
\item $\displaystyle\lim_{x\to 0^+}x\ln(x)$
\item $\displaystyle\lim_{x\to +\infty}\left(x^2-\ln(x)\right)$
\item $\displaystyle\lim_{x\to +\infty}\frac{\ln(x+1)}{\ln(x)}$
\item $\displaystyle\lim_{x\to 1}\frac{\ln(x)}{x-1}$
\end{enumerate}
\end{multicols}
\item Résoudre les inéquations suivantes sur leur ensemble de définition :
\begin{multicols}{2}
\begin{enumerate}[label=]
\item $(I_1) \quad \ln(x) \geq 1$
\item $(I_2) \quad \ln(2x-1) < 0$
\item $(I_3) \quad \ln(x^2-4) \leq \ln(3)$
\item $(I_4) \quad \ln(x) + \ln(x-2) > \ln(3)$
\end{enumerate}
\end{multicols}
\end{enumerate}

\vspace{1em}
\hrule
\vspace{1em}

\exo[2]{Équations}

Résoudre sur $\mathbb R$ les équations suivantes : 
$$
\ e^{2x}-e^x-6=0\qquad\qquad\ 3e^x-7e^{-x}-20=0.
$$

\vspace{1em}
\hrule
\vspace{1em}

\exo[2]{Syst\`emes}

Résoudre les syst\`emes d'équations suivantes : 
$$\begin{array}{lll}
\mathbf{1.}\ \left\{
\begin{array}{rcl}
e^xe^y&=&10\\
e^{x-y}&=&\frac 25
\end{array}
\right.&\quad\quad&\mathbf{2.}\ 
\left\{
\begin{array}{rcl}
e^x-2e^y&=&-5\\
3e^x+e^y&=&13
\end{array}\right.\quad\quad
\mathbf{3.}\ \left\{
\begin{array}{rcl}
5e^x-e^y&=&19\\
e^{x+y}&=&30
\end{array}
\right.
\end{array}$$

\vspace{1em}
\hrule
\vspace{1em}

\exo[2]{Exponentielle et valeur absolue}

\vspace{1em}
Démontrer que pour tout réel $x$, on a 
$$\frac{e^x+e^{-x}}{2}\leq e^{|x|}.$$


\vspace{1em}
\hrule
\vspace{1em}


\exo[3]{Un encadrement de $e$}

\vspace{1em}
Démontrer que, pour tout $n\geq 2$, on a 
$$\left(1+\frac 1n\right)^n \leq e\leq \left(1-\frac 1n\right)^{-n}.$$

\vspace{1em}
\hrule
\vspace{1em}

\exo[1]{Limites et exponentielle}

\vspace{1em}

Déterminer les limites suivantes : 
$$\lim_{x\to+\infty}\frac{e^x}{2e^x+e^{-2x}}\qquad \textrm{ et }\qquad\lim_{x\to-\infty}\frac{e^x}{2e^x+e^{-2x}}.$$


\vspace{1em}
\hrule
\vspace{1em}

\exo[2]{Positivité}

Soit $g:\mathbb R_+\to\mathbb R$ définie par 
$$g(x)=(x-2)e^{x}+(x+2).$$

Démontrer que $g\geq 0$ sur $\mathbb R_+$.


\vspace{1em}
\hrule
\vspace{1em}

\exo[3]{Médecin légiste}

Un inspecteur qui arrive sur le lieu d'un crime demande au médecin
légiste de prendre la température de la victime. Elle est de $32^{\circ} C$. Il prend la température de la pi\`ece, qui est de $20^{\circ} C$.

La loi de Newton sur le refroidissement d'un objet en milieu ambiant permet de modéliser la température de la victime en posant $T(t)=Ae^{-ct}+20$ o\`u $t>0$ représente le temps, exprimé en heures, depuis la mort de la victime et $T(t)$ la température de la victime \`a  l'instant $t$, en degrés Celsius.

Sachant qu'une demi-heure plus tard, la température de la victime est de $31^{\circ} C$, déterminer l'heure du crime (on prendra comme hypoth\`ese qu'au moment de sa mort, la température de la victime était de $37^{\circ} C$).


\vspace{1em}
\hrule
\vspace{1em}


\exo[3]{Pharmacocinétique}

\vspace{1em}
On injecte un médicament \`a  un patient en intraveineuse.
Dans de nombreux cas, la concentration dans le sang de la substance active, en $\textrm{mg.L}^{-1}$, vérifie la relation 
$$C(t)=C_0e^{-\lambda t}$$
o\`u $C_0$ est la concentration initiale, $t$ est le temps, exprimé en heures, apr\`es l'injection, et $\lambda$ est un coefficient spécifique au médicament, 
\begin{enumerate}
\item On appelle demi-vie du médicament le temps nécessaire pour que, apr\`es administration du médicament, sa concentration diminue de moitié. Calculer (en fonction de $\lambda$) le temps de demi-vie $T_{1/2}$ d'un médicament dont la concentration dans le sang satisfait la relation précédente. Quelle est la concentration apr\`es $2T_{1/2}$? Apr\`es $nT_{1/2}$?
\item L'aztréonam est un antibiotique qui est notamment utilisé chez les patients atteints de mucoviscidose pour soigner des infections bronchiques. Il n'est efficace que si sa concentration dans 
le sang dépasse $40\textrm{mg.L}^{-1}$. On dispose de doses de $2\textrm{g}$ et on souhaite connaitre le temps maximal entre deux injections pour maintenir cette concentration supérieure \`a  $40\textrm{mg.L}^{-1}$ chez un patient pesant $60\textrm{kg}$. 
Sachant que le volume sanguin d'un adulte est d'environ $70\textrm{mL.kg}^{-1}$ et que le temps de demi-vie de l'aztréonam, tel qu'indiqué par le fabricant, est de $1,\!7\textrm{h}$,  calculer 
\begin{enumerate}[label=\alph*)]
\item le temps maximal séparant la premi\`ere injection et la deuxi\`eme;
\item le temps maximal séparant les injections suivantes.
\end{enumerate}
\end{enumerate}


\vspace{1em}
\hrule
\vspace{1em}


\exo[3]{Point le plus proche de l'origine}

\vspace{1em}
On consid\`ere la courbe de la fonction exponentielle dans un rep\`ere orthonormé $(O,\vec i,\vec j)$. 
\begin{enumerate}[label=\alph*)]
\item Pour $x\in\mathbb R$, on pose $g(x)=x+e^{2x}$. \newline 
Démontrer qu'il existe un réel $c$ tel que $g(x)< 0$ si $x< c$ et $g(x)> 0$ si $x> c$.
\item En déduire qu'il y a un unique point sur la courbe de la fonction exponentielle qui minimise la distance \`a l'origine. On le note $M_0$.
\item Démontrer que la tangente \`a   la courbe en $M_0$ est perpendiculaire \`a  la droite $(OM_0)$.
\end{enumerate}



\vspace{1em}
\hrule
\vspace{1em}

\exo[2]{Unique solution}

\vspace{1em}
Soit $h$ la fonction définie sur $\mathbb R$ par $h(x)=x\exp(1-x)$.
\begin{enumerate}[label=\alph*)]
\item Dresser le tableau de variations de $h$.
\item Démontrer qu'il existe un unique $\rho\in\mathbb R$ tel que $h(\rho)=-1$.
\end{enumerate}

\vspace{1em}
\hrule
\vspace{1em}

\exo[1]{\'Equation différentielle}

\vspace{1em}
Soit $f$ une fonction définie sur $\R$ telle que  pour tout $x$ réel, $f'(x)=0$. 

\begin{enumerate}
\item
Que peut-on dire de $f$ ? 
\item On suppose de plus que $f(0)=2$. Que peut-on alors dire de $f$ ?
\end{enumerate}


\vspace{1em}
\hrule
\vspace{1em}

\exo[1]{Tangente}

Déterminer les équations des tangentes \`a la courbe représentative de
la fonction exponentielle aux points d'abscisse $0$, $1$ et $2$. 


\vspace{1em}
\hrule
\vspace{1em}

\exo[1]{Simplification}

\vspace{1em}
Simplifier les expressions: 
\begin{multicols}{3}
\begin{enumerate}[label=\alph*), itemsep=1.5ex]
\item $ (e^x)^5e^{-2x}$
\item $ \dfrac{e^{2x+3}}{e^{2x-1}}$
\item $\dfrac{e^x+e^{-x}}{e^{-x}}$
\end{enumerate}
\end{multicols}
\vspace{1em}
\hrule
\vspace{1em}


\exo[1]{Factorisation}

\vspace{1em}
Démontrer que pour tout réel $x$, 
\begin{multicols}{2}
\begin{enumerate}[label=\alph*), itemsep=1.5ex]
\item $\dfrac{e^{2x}-1}{e^x+1}=e^x\dfrac{1-e^{-2x}}{1+e^{-x}}$.
\quad
\item $(e^x+e^{-x})^2-(e^x-e^{-x})^2=4$.
\item $ \dfrac{e^x-1}{e^x+1}=\dfrac{1-e^{-x}}{1+e^{-x}}$
\item $ e^{-x}-e^{-2x}=\dfrac{e^x-1}{e^{2x}}$
\end{enumerate}
\end{multicols}
\vspace{1em}
\hrule
\vspace{1em}

\newpage

\exo[2]{Équations et inéquations}

Résoudre les équations suivantes
\begin{multicols}{3}
\begin{enumerate}
\item[] $(E_1) \quad e^x=1$
\item[] $(E_2) \quad e^{2x}=e$
\item[] $(E_3) \quad e^{x}=e^{-x}$
\item[] $(E_4) \quad e^{x^2}=(e^{-x})^2e^3$ 
\item[] $(E_5) \quad e^{2x+1}=e^{\frac{6}{x}}$
\end{enumerate}
\end{multicols}
Résoudre les inéquations suivantes
\begin{multicols}{3}
\begin{enumerate}
\item[] $(I_1) \quad e^x>e$
\item[] $(I_2) \quad e^{2x}\leq 1$
\item[] $(I_3) \quad (e^x)^3\leq \frac{1}{e}$
\item[] $(I_4) \quad e^x-\frac{1}{e^x}>0$
\item[] $(I_5) \quad e^{x^2}\geq e^{-x-1}$
\end{enumerate}
\end{multicols}
\vspace{1em}
\hrule
\vspace{1em}


\exo[2]{Étude}

\vspace{1em}
On note $\mathcal{C}$ la courbe représentative de la fonction
exponentielle dans un rep\`ere orthonormal $(O;\vec{i},\vec{j})$. 
\begin{enumerate}[label=\alph*)]
\item Soit $a$ un réel.\newline
Déterminer l'équation de la tangente $T_a$ \`a  la courbe $\mathcal{C}$ au point d'abscisse $a$. 
\item Etudier la position relative de la courbe $\mathcal{C}$ par
  rapport \`a la droite $T$ .\vspace{1em}
%  
  \emph{Indication: on pourra étudier les variations de la fonction
    $\varphi$ définie par $\varphi(x)=e^x-y$, o\`u $y$ désigne l'équation de la
    tangente $T_a$}. 
\end{enumerate}

\vspace{1em}
\hrule
\vspace{1em}


\exo[2]{Sens de variation}

Etudier le sens de variation des fonctions suivantes: 
\begin{multicols}{3}
\begin{enumerate}[label=\alph*)]
\item $f(x)=\dfrac{e^x}x$ \quad 
\item $f(x)=e^{2x}-2x$ \quad 
\item $f(x)=e^{x^2}-x^2$ \quad 
\item $f(x)=\dfrac{e^x+e^{-x}}2$ \quad 
\item $f(x)=(2x+3)e^{-2x}$
\end{enumerate}
\end{multicols}
\vspace{1em}
\hrule
\vspace{1em}


\exo[1]{Limites}

Déterminer les limites en $-\infty$ et $+\infty$ des fonctions suivantes: 
\begin{multicols}{3}
\begin{enumerate}[label=]
\item $f(x)=e^{-3x}$ 
\item $g(x)=e^{x}+e^{-x}$
\item $h(x)=x+e^x$
\item $k(x)=e^{2x}+e^x+1$

\item $ \ell(x)=e^{3x}-e^{x}$ 
\item $ m(x)=\frac{e^x+1}{e^x+2}$
\item $ n(x)=\dfrac{-2e^x}{1+e^x}$
\end{enumerate}
\end{multicols}
\vspace{1em}
\hrule
\vspace{1em}
\newpage

\exo[2]{Limites}

 Déterminer la limite en $+\infty$ des fonctions suivantes: 

$$f(x)=x^2+2-e^x \qquad
g(x)=\frac{2e^x-x}{x^2} \qquad 
h(x)=\frac{e^x}{x^2+1}$$

\vspace{0.2cm}
$$\ell(x)=e^{2x}-(x+1)e^x \qquad
k(x)=\frac{\sqrt{e^x+2}}{x} \qquad 
t(x)=\frac{e^{2x}+x^2}{x^2+x-3}$$

\vspace{1em}
\hrule
\vspace{1em}


\exo[2]{Limites}

 A l'aide d'un changement de variable, étudier la limite en
$+\infty$ des fonctions: 

$$f(x)=\frac{e^{2x+1}}{x}, \qquad 
g(x)=\frac{2e^{x^2-1}}{x^2}, \qquad 
h(x)=x^3e^{-\sqrt{x}}$$ 

\vspace{1em}
\hrule
\vspace{1em}

\exo[2]{Études}

Etudier sur $\R$ les fonctions suivantes (sens de variations et limites): 

\begin{multicols}{3}
\begin{enumerate}[label=]
\item $f(x)=e^{-x}$ 
\item $g(x)=e^x+e^{-x}$
\item $h(x)=x+e^x$
\item $k(x)=e^{3x}-3e^x$
\item $\ell(x)=e^{-x^2}$
\item $m(x)=(x-2)e^{-0.1x}$
\item $n(x)=(x+1)^2e^{-x}$
\end{enumerate}
\end{multicols}
\vspace{1em}
\hrule
\vspace{1em}

\exo[3]{Position relative}

Les fonctions $f$ et $g$ sont définies sur $\R$ par: 
$f(x)=-x^2e^x$ et $g(x)=\left( x^2-x-1\right) e^x$. 

\begin{enumerate}[label=\alph*)]
\item Déterminer les coordonnées des points d'intersection des courbes
  $\mathcal{C}_f$ et $\mathcal{C}_g$ représentatives des fonctions $f$
  et $g$. 
\item Déterminer la position relative de $\mathcal{C}_f$ et
  $\mathcal{C}_g$. 
\item Déterminer les limites de $f$ et $g$ en $-\infty$ et $+\infty$. 
\item Dresser les tableaux de variations de $f$ et $g$.
\end{enumerate}
\vspace{1em}
\hrule
\vspace{1em}

\newpage

\exo[2]{Capteur solaire}

Un capteur solaire récupère de la chaleur par le biais d'un fluide. On s'intéresse \`a l'évolution de la température du fluide dans un capteur de $1$m de longueur. 

Cette température est modélisée par: 
$T(x)=170-150e^{-0,6x}$, 
où $x\in[0;1]$ est la distance parcourue par le fluide en mètres
depuis son entrée dans le capteur, et $T(x)$ est la température en
$^\circ$C. 
\begin{enumerate}
\item Déterminer la température \`a l'entrée du capteur. 
\item 
\begin{enumerate}[label=\alph*)]
  \item Etudier les variations de la température $T$ sur $[0;1]$. 
  \item En déduire la température maximale, au degré prés, atteinte
    par le fluide. 
  \item Tracer dans un repére la courbe représentant la température
    $T$. 
\end{enumerate}
\end{enumerate}
\vspace{1em}
\hrule
\vspace{1em}



\exo[3]{D'apr\`es sujet bac Amérique du Sud 2018}

 \medskip 


$f$ est la fonction définie sur $[0 ; 12]$ par $f(x) = 2x e^{-x}$.

\medskip

 \textsf{\textbf{\textsc{Partie A}}}

\medskip

Un logiciel de calcul formel donne les résultats suivants : 


\begin{center}
\begin{tabularx}{0.75\linewidth}{|c|X|}\hline
\multirow{2}*{1} & \rule[-1ex]{0pt}{3ex} Dériver $\left(2*x*\mathrm{exp}(-x)\right)$\\
	& \multicolumn{1}{r@{\hspace{2em}}|}{$-2*x*\mathrm{exp}(-x)+2*\exp(-x)$}\\ \hline
\multirow{2}*{2} & \rule[-1ex]{0pt}{3ex} Factoriser $\left(-2*x*\mathrm{exp}(-x)+ 2*\mathrm{exp}(-x)\right)$\\
	&\multicolumn{1}{r@{\hspace{2em}}|}{$2*(1-x)*\mathrm{exp}(-x)$}\\ \hline
	\multirow{2}*{3} & \rule[-1ex]{0pt}{3ex} Dériver $\left(2*(1-x)*\mathrm{exp}(-x)\right)$\\
	&\multicolumn{1}{r@{\hspace{2em}}|}{$2*x*\mathrm{exp}(- x) - 4*\mathrm{exp}(- x)$}\\ \hline
	\multirow{2}*{4} & \rule[-1ex]{0pt}{3ex} Factoriser $\left(2 *x*\mathrm{exp}(-x) - 4*\mathrm{exp}(-x)\right)$\\
	&\multicolumn{1}{r@{\hspace{2em}}|}{$2 *(x - 2)*\mathrm{exp}(- x)$}\\ \hline
\end{tabularx}
\end{center}


\begin{enumerate}
\item Vérifier le résultat de la ligne 1 donné par le logiciel de calcul formel. 
\end{enumerate}

\medskip

\emph{Dans la suite, on pourra utiliser les résultats donnés par le logiciel de calcul formel sans les justifier.}

\medskip 
\begin{enumerate}
\setcounter{enumi}{1}
\item 
	\begin{enumerate}
		\item Dresser le tableau des variations de la fonction $f$ sur l'intervalle $[0 ; 12]$ en le justifiant. 
		\item Démontrer que l'équation $f(x) = 0,5$ admet deux solutions dans $[0 ; 12]$. 

Donner \`a l'aide de la calculatrice une valeur approchée au centi\`eme de chacune de ces solutions. 
	\end{enumerate}
\item étudier la convexité de la fonction $f$ sur l'intervalle $[0 ; 12]$. 
\end{enumerate}

\medskip


\newpage

 \textsf{\textbf{\textsc{Partie B}}}

\medskip

Le taux d'alcoolémie d'une personne pendant les $12$ heures suivant la consommation d'une certaine quantité d'alcool est modélisé par la fonction $f$ : 


\begin{itemize}
\item $x$ représente le temps (exprimé en heure) écoulé depuis la consommation d'alcool; 
\item $f(x)$ représente le taux d'alcoolémie (exprimé en g/L) de cette personne. 
\end{itemize}


\medskip

\begin{enumerate}
\item 
	\begin{enumerate}
		\item Décrire les variations du taux d'alcoolémie de cette personne pendant les 12 heures suivant la consommation d'alcool. 
		\item \`A quel instant le taux d'alcoolémie de cette personne est-il maximal ? 

Quelle est alors sa valeur ? Arrondir au centi\`eme. 
	\end{enumerate}
\item  Le Code de la route interdit toute conduite d'un véhicule lorsque le taux d'alcoolémie est supérieur ou égal \`a 0,5 g/L. 

Une fois l'alcool consommé, au bout de combien de temps le taux d'alcoolémie de l'automobiliste reprend-il une valeur conforme \`a la législation ?
\end{enumerate}


\vspace{1em}
\hrule
\vspace{1em}



\exo[2]{D'apr\`es sujet bac Amérique du Nord 2005}

 \medskip 


\emph{Les deux questions sont indépendantes. Les résultats seront arrondis \`a} $10^{-2}$.

\medskip

Le gouvernement d'un pays envisage de baisser un imp\^ot de 30\,\% en cinq ans.

\begin{enumerate}
\item On suppose que le pourcentage de baisse est le m\^eme chaque année.

Vérifier que ce pourcentage de baisse annuel est alors égal \`a environ 6,89\,\%.
\item  La premi\`ere année cet imp\^ot  baisse de 5\,\%, la deuxi\`eme année la baisse est de 1\,\% et la troisi\`eme année de 3\,\%.
	\begin{enumerate}
		\item Quelle est la baisse, en pourcentage, de cet imp\^ot  au terme de ces trois premi\`eres années ?
		\item  Pour atteindre son objectif quel pourcentage annuel de baisse doit décider ce gouvernement, en supposant que ce pourcentage est le m\^eme sur les deux derni\`eres années ?
	\end{enumerate}
\end{enumerate}


\vspace{1em}
\hrule
\vspace{1em}

\exo[3]{D'apr\`es sujet bac France Métropolitaine Septembre 2010}



\medskip

\textsf {\small{\textbf{\textsc{Partie A}}}:  étude d'une fonction}

\medskip
 
On consid\`ere les fonctions $f$, $g$ et $h$ définies et dérivables pour tout nombre réel $x$ de l'intervalle $[4~;~ 6]$ par $f(x) = 100\left(e^x - 45\right)$, $g(x) = 10^6 e^{-x}$ et $h(x) = g(x) - f(x)$.
 
On note $h'$ la fonction dérivée de la fonction $h$ sur l'intervalle $[4~;~6]$.

\medskip
 
Résolution de l'équation $h(x) = 0$. 

\medskip
 
\begin{enumerate}
\item 
	\begin{enumerate}
		\item Démontrer que la fonction $h$ est strictement décroissante sur l'intervalle $[4~;~6]$. 
		\item Dresser le tableau de variations de la fonction $h$. 
		\item Justifier que l'équation $h(x) = 0$ admet une solution unique $\alpha$ sur l'intervalle $[4~;~6]$.
	\end{enumerate} 


\end{enumerate}

\medskip

\emph{Dans la suite de l'exercice, on admet que la valeur exacte du nombre réel $\alpha$ est égale \`a $3\ln 5$ o\`u $\ln$ désigne la fonction logarithme népérien.}
 
\medskip

\textsf {\small{\textbf{\textsc{Partie B}}}:  Application économique}

\medskip
 
Les fonctions $f$ et $g$ définies dans la partie A modélisent respectivement l'offre et la demande d'un produit de prix unitaire $x$, compris entre 4 et 6 euros :


\begin{itemize}
\item $f(x)$ est la quantité, exprimée en kilogrammes, que les producteurs sont pr\^ets \`a vendre au prix unitaire $x$ ; 
\item $g(x)$ la quantité, exprimée en kilogrammes, que les consommateurs sont pr\^ets \`a acheter au prix unitaire $x$.
\end{itemize}

\medskip
 
On appelle prix unitaire d'équilibre du marché la valeur de $x$ pour laquelle l'offre est égale \`a la demande.
 
\begin{enumerate}
\item Quel est, exprimé au centime d'euro pr\`es, le prix unitaire d'équilibre du marché ? Justifier. 
\item  Quelle quantité de produit, exprimée en kilogrammes, correspond \`a ce prix unitaire d'équilibre ? 
\end{enumerate}

%\psset{unit=1cm}
%\begin{pspicture}(0,-9)(12,7)
%\rput(6,6.5){\textsf {Graphique \`a compléter}}
%\newrgbcolor{bleu}{0.1 0.05 .5}
%\def\pshlabel#1{\footnotesize #1}
%\def\psvlabel#1{\footnotesize #1}
%\psgrid[gridwidth=0.1pt,gridcolor=gray,subgriddiv=0,gridlabels=0](0,0)(0,-9)(12,6)
%\psset{xunit=5cm,yunit=.25cm}
%\psaxes[labelsep=.75mm,ticksize=-2pt 2pt,linewidth=0.75pt,Ox=4,Dx=.2,Dy=4, ylabelFactor={\,000}, comma]{->}(0,0)(-.02,-36)(2.4,24)
%\uput[l](0,0){\footnotesize{$0$}}
%\uput[dr](0,0){\footnotesize{$4$}}
%\uput[d](2.4,0){\footnotesize{$x$}}
%\uput[l](0,24){\footnotesize{$y$}}
%\end{pspicture}
%\end{center}
\vspace{1em}
\hrule
\vspace{1em}


\exo[3]{Étude de fonctions}

La fonction $f$ est définie par : \quad $f(x)=\dfrac{2x}{1-x^2}+\ln\left(\dfrac{1+x}{1-x}\right)$.
\begin{enumerate}
\item Justifier que $D_f=]-1;1[$.
\item étudier la parité de $f$.
\item Préciser les asymptotes \`a la courbe ${\cal C}_f$.
\item étudier les variations de $f$.
\item Donner l'équation de la tangente \`a ${\cal C}_f$ au point d'abscisse 0.
\item Faire une représentation graphique.
\end{enumerate}
\vspace{1em}
\hrule
\vspace{1em}

\exo[3]{Fonction auxiliaire}

En vous aidant des variations de la fonction $f:x\mapsto \dfrac{1}{x\ln x}$, démontrer que $$\forall x\in]0;1[, \quad \dfrac{1}{x\ln x}\leq -e$$

\vspace{1em}
\hrule
\vspace{1em}


\exo[2]{Inéquation et équation}

\begin{enumerate}
\item Résoudre dans $\R$:\quad $e^x-4=5e^{-x}$.

\item Résoudre dans $\R$:\quad $1\leq e^{2x-3}\leq e$.
\end{enumerate}

\vspace{1em}
\hrule
\vspace{1em}

\newpage

\exo[2]{Radon}

Le radon est un gaz radioactif qui se désintègre avec le temps. Si $m$ est la masse de radon un certain jour (origine), $x$ jours plus tard la masse $f(x)$ est donnée par :
$$f(x)=me^{-0,18x}$$
\begin{enumerate}
\item Exprimer $f(x+1)$ en fonction de $f(x)$. En déduire quelle est l'évolution de la masse en pourcentage en un jour.
\item Représenter qualitativement ${\cal C}_f$ sur une durée de 8 jours.
\item On appelle période le temps au bout duquel la masse a diminué de moitié. Calculer la période du radon en jours puis en heures. Contr\^oler graphiquement la réponse obtenue.
\end{enumerate}

\vspace{1em}
\hrule
\vspace{1em}

\exo[2]{\'Etude}

Soit la fonction $f$ définie sur $\R$ par $f(x)=3x+\dfrac{20e^x}{e^x+1}.$
\begin{enumerate}
\item Déterminer les limites de $f$ en $-\infty$ et $+\infty$.
\item Soit $g$ la fonction définie sur $\R$ par $g(x)=f(x)-3x$.\vspace{1em}
Déterminer les limites de $g$ en $-\infty$ et $+\infty$.
\item Dans un repère orthogonal, on note $\cal C$ la courbe représentative de $f$, $\Delta$ la droite d'équation $y=3x$ et $\Delta^{\prime}$ la droite d'équation $y=3x+20$.
	\begin{enumerate}%[(a)]
	\item Soit $a$ un réel ; $M$ le point d'abscisse $a$ de $\cal C$ et $N$ le point d'abscisse $a$ de $\Delta$. Quelles sont les ordonnées de $M$ et $N$ ?
	\item Exprimer la distance $MN$ en fonction de $e^a$. Justifier que $\displaystyle\lim_{a\to-\infty}MN=0$.
	\item Soit $P$ le point d'abscisse $a$ de $\Delta^{\prime}$. Quelle est l'ordonnée de $P$ ?
	\item Exprimer la distance $MP$ en fonction de $e^a$. Justifier que $\displaystyle\lim_{a\to+\infty}MP=0$.
	\item Que représentent $\Delta$ et $\Delta^{\prime}$ pour la courbe $\cal C$ ?
	\item Représenter graphiquement $\cal C$,  $\Delta$ et $\Delta^{\prime}$ dans un repère orthogonal (unités graphiques : 1 cm sur l'axe des abscisses, 0.5cm sur l'axe des ordonnées).
	\end{enumerate}
\end{enumerate}

\vspace{1em}
\hrule
\vspace{1em}

\exo[3]{Difficile}
Trouver la plus grande valeur de $n^{1/n}$ pour $n\in\N^*$.
\medskip

\textit{On étudiera (sous son écriture exponentielle) la fonction $f$ définie sur $]0,+\infty[$ par $f(x)=x^{1/x}$ pour conduire la recherche.}

\end{document}

