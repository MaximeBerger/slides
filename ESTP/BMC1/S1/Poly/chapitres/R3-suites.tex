\documentclass[../PolyS1.tex]{subfiles}
\begin{document}

\section{Suites de référence}

\subsection{Suites arithmétiques}

\begin{Def}\textbf{Suite arithmétique}
    \vspace{1em}

Une suite arithmétique est une suite dont le passage d'un terme au suivant s'obtient en \textbf{ajoutant} toujours la même quantité $r$ appelée la raison.
\end{Def}

De cette définition, découle la forme récurrente de la suite :
$$\begin{cases}
    u_0 \\
    u_{n+1}=u_n+r
\end{cases}$$

\begin{Thm}\textbf{Expression générale d'une suite arithmétique}
    \vspace{1em}

Avec un peu de réflexion, on en déduit l'expression en fonction de $n$ :
$$u_n=u_p+(n-p)r$$
Où $u_p$ représente le premier terme de la suite.
\end{Thm}


On rappelle la formule de sommation :
$$\sum_{k=0}^n u_k=\frac{(n+1)(u_0+u_n)}{2}$$


\exo[1]{Exemple 1 :}
$(u_n)$ est une suite arithmétique de raison $r$.
\begin{enumerate}
    \item $u_2=3$ et $r=0,2$. Calculer $u_{50}$.
    \item $u_6=4$ et $u_8=12$. Calculer $r$ et $u_5$.
    \item $u_2=7$ et $u_10=1$. Calculer $r$ et $u_0$.
\end{enumerate}

\subsection{Suites géométriques}

\begin{Def}\textbf{Suite géométrique}
    \vspace{1em}

Une suite géométrique est une suite dont le passage d'un terme au suivant s'obtient en \textbf{multipliant} toujours la même quantité $q$ appelée la raison.
\end{Def}


De cette définition, découle la forme récurrente de la suite: 
$$\begin{cases}
    u_0 \\
    u_{n+1}=q \times u_n
\end{cases}$$


\begin{Thm}\textbf{Expression générale d'une suite géométrique}
    \vspace{1em}

Avec un peu de réflexion, on en déduit l'expression en fonction de $n$ :
$$u_n=u_p\times q^{n-p}$$
Où $u_p$ représente le premier terme de la suite.
\end{Thm}


On rappelle la formule de sommation :
$$\sum_{k=0}^n u_k=\frac{1-q^{n+1}}{1-q}$$


\exo[1]{Exemple 2 :}
$(u_n)$ est une suite géométrique de raison $q$.
\begin{enumerate}
    \item $u_0=16$ et $q=0,5$. Calculer $u_8$.
    \item $u_6=9$ et $q=-3$. Calculer $u_0$.
    \item $u_10=18$ et $u_9=-6$. Calculer $q$.
\end{enumerate}

\subsection{Les suites arithmético-géométriques}

\begin{Def}\textbf{Suites arithmético-géométriques}
    \vspace{1em}

Ces suites sont de la forme :
$$\begin{cases}
    u_0 \\
    u_{n+1}=au_n+b
\end{cases}$$
\end{Def}


Si $a=1$, on retombe sur une suite arithmétique et si $b=0$ sur une suite géométrique. 

L'objectif, comme souvent avec les suites, est de trouver une expression en fonction de $n$. 
On se place donc dans le cas où $a\neq 1$ et $b \neq 0$.

\vspace{2em}

\textbf{Méthode de résolution des suites arithmético-géométriques}

\vspace{1em}

La réalisation de cet objectif se déroule en trois étapes :
\begin{enumerate}
    \item On cherche un point fixe, c'est à dire un réel $c$ tel que :
    $$c=ac+b$$
    \item On définit une suite auxiliaire $v$ définie par :
    $$v_n=u_n-c$$
    Et on montre que $v_n$ est géométrique (de raison $a$).
    \item On exprime $v$ en fonction de $n$ et on en déduit une expression de $u$ en fonction de $n$.    
\end{enumerate}


\exo[1]{Exemple 3 :}
Pour chacune des suites suivantes; déterminer son expression en fonction de $n$.
$$\begin{cases}
        u_0=-2 \\
        u_{n+1}=-2u_n+3
    \end{cases}, \qquad \begin{cases}
        u_0=150 \\
        u_{n+1}=0,8\, u_n+45
    \end{cases}, \qquad \begin{cases}
        u_0=5 \\
        u_{n+1}=\frac{1}{2}u_n-\frac{3}{2}
    \end{cases}$$

\subsection{Suites récurrentes linéaires d'ordre 2}


Ces suites sont de la forme :
$$\begin{cases}
    u_0 \\
    u_1 \\
    u_{n+2}=au_{n+1}+bu_n
\end{cases}$$


\vspace{2em}

\textbf{Méthode de résolution des suites récurrentes linéaires d'ordre 2}

\vspace{1em}

Pour trouver une expression en fonction de $n$, on considère ce que l'on appelle l'équation caractéristique :
$$r^2-ar-b=0$$
On calcule le discriminant $\Delta$ et on distingue les trois cas habituels : 
\begin{enumerate}
    \item $\Delta>0$

    L'équation admet deux solutions distinctes $r_1$ et $r_2$ et dans ce cas, il existe $\lambda$ et $\mu$ uniques tels que :
    $$u_n=\lambda r_1^n +\mu r_2^n$$
    
    \item $\Delta=0$

    L'équation admet une unique solution $r$ et dans ce cas, il existe $\lambda$ et $\mu$ uniques tels que :
    $$u_n=(\lambda+\mu n)r^n $$

    \item $\Delta<0$
    L'équation admet deux solutions complexes $r_1=re^{i \theta}$ et $r_2=re^{-i \theta}$ et dans ce cas, il existe $\lambda$ et $\mu$ uniques tels que :
    $$u_n=r^n\big(\lambda\cos({n \theta})+\mu \sin{(n\theta)}\big)$$
\end{enumerate}


\begin{Rmq}$\,$

Dans chacun des cas, le couple $(\lambda, \mu )$ est déterminé en appliquant les formules de $u_n$ pour $n=0$ et $n=1$.
\end{Rmq}

\exo[2]{Exemple 4 :}
Pour chacune des suites suivantes, déterminer son expression en fonction de $n$.
$$\begin{cases}
    u_0=2 \\
    u_1=3 \\
    u_{n+2}=5u_{n+1}-6u_n
\end{cases}, \qquad \begin{cases}
    u_0=1 \\
    u_1=0 \\
    u_{n+2}=-4u_{n+1}-4u_n
\end{cases}$$
$$\begin{cases}
    u_0=1 \\
    u_1=2 \\
    u_{n+2}=2u_{n+1}-2u_n
\end{cases}, \qquad \begin{cases}
    u_0=2 \\
    u_1=3 \\
    u_{n+2}=6u_{n+1}-9u_n
\end{cases}$$

\section{Variations et limites}

\subsection{Variations}

\begin{Def}\textbf{Variations}
    \vspace{1em}

Une suite $(u_n)$ est \textbf{croissante} si pour tout entier $n$, on a :
$$u_{n+1}\geq u_n$$

Une suite $(u_n)$ est \textbf{décroissante} si pour tout entier $n$, on a :
$$u_{n+1}\leq u_n$$
\end{Def}


Dans la pratique, on dispose de quatre méthodes pour étudier les variations d'une suite :
\begin{enumerate}
    \item \'Etudier le signe de $u_{n+1}-u_n$.
    \item Comparer le rapport $\frac{u_{n+1}}{u_n}$ à 1.
    \item Si $u_n=f(n)$, on peut étudier les variations de $f$ sur $[0;+\infty[$.
    \item Utiliser une démonstration par récurrence.
\end{enumerate}


\begin{Rmq}$\,$

Attention, une suite n'est pas nécessairement croissante ou décroissante... Les suites géométriques à raisons négatives ne sont ni l'une ni l'autre.
\end{Rmq}

\subsection{Majoration et minoration}

\begin{Def}\textbf{Majoration et minoration}
    \vspace{1em}

Une suite $(u_n)$ est \textbf{majorée} s'il existe $M$ réel tel que $u_n\leq M$.
Une suite $(u_n)$ est \textbf{minorée} s'il existe $m$ réel tel que $u_n\geq m$.
\end{Def}


Une suite croissante est minorée par son premier terme.

Une suite décroissante est majorée par son premier terme.

Une suite majorée et minorée est dite bornée.


\exo[1]{Exemple 5 :}
Trouver un majorant et un minorant pour chacune des suites suivantes :
\begin{enumerate}
    \item $\forall n \in \mathbb{N}, u_n=\sin{(\frac{\pi}{3}+\frac{n \pi}{4})}$  
     \item $\forall n \in \mathbb{N^*}, u_n=1-\frac{1}{n}$
      \item $\forall n \in \mathbb{N^*}, u_n=1+\frac{1}{n}$
       \item $\forall n \in \mathbb{N}, u_n=e^{-n}$
\end{enumerate}

\subsection{Limites}


Contrairement aux fonctions, la limite d'une suite n'a de sens que pour $n$ qui tend vers $+\infty$.

\begin{Def}\textbf{Limites}
    \vspace{1em}

On dit que $(u_n)$ converge vers une limite finie $\ell\in \mathbb{R}$ si :
$$\forall \epsilon >0\quad \exists N \in \mathbb{N}\quad \text{tel que}\quad n>N \Rightarrow |u_n-\ell|<\epsilon$$
On note : 
$$\lim\limits_{n \rightarrow +\infty} u_n=\ell$$

On dit que $(u_n)$ diverge vers $\pm \infty$ si :
$$\forall M >0\quad \exists N \in \mathbb{N}\quad \text{tel que}\quad n>N \Rightarrow |u_n|>M$$
On note : 
$$\lim\limits_{n \rightarrow +\infty} u_n=\pm \infty$$
\end{Def}

Une suite peut aussi ne pas avoir de limite, comme les fonctions. On parle parfois de divergence de seconde espèce.


\vspace{2em}

\textbf{Techniques de calcul de limites}

\vspace{1em}

On dispose de techniques très pratiques pour justifier de la convergence ou non d'une suite et parfois même de trouver sa limite.

\begin{enumerate}
    \item Toute suite croissante (resp. décroissante) majorée (resp. minorée) est convergente.
    \item si $u_{n+1}=f(u_n)$ et $\lim\limits_{n \rightarrow +\infty} u_n=\ell$, alors nécessairement $\ell$ est un point fixe de $f$ : $f(\ell)=\ell$.
    \item \textbf{Inégalité des accroissements finis}

    Si $f$ est dérivable sur un intervalle $I$ et si $\exists M>0$ tel que $|f'(x)|\leq M$, alors pour tout couple $a,b$ dans $I$, on a :
    $$|f(b)-f(a)|\leq M|b-a|$$ 
    \item \textbf{Équivalents}

    Deux suites $(u_n)$ et $(v_n)$ sont dites \textbf{équivalentes} si :
  $$\lim\limits_{n \rightarrow +\infty} \frac{u_n}{v_n}= 1$$  
  On note $u_n \sim v_n$.

  Deux suites équivalentes ont même limite.Il faut connaître les équivalents usuels pour $u$ convergente vers 0.
  $$e^{u_n}-1\sim u_n \quad \ln{(1+u_n)}\sim u_n \quad \sin u_n \sim u_n \quad \tan u_n \sim u_n$$
  $$(1-u_n)^\alpha \sim \alpha u_n \quad 1-\cos u_n \sim \frac{u_n^2}{2}$$
  
  \newpage

  \item \textbf{Sur les suites géométriques}
  \begin{itemize}
    \item Si $|q|<1$, alors $q^n$ tend vers $0$.
    \item Si $q>1$, alors $q^n$ tend vers $+\infty$.
    \item Si $q=1$, alors $q^n=1$, la suite est constante.
    \item Si $q\leq -1$, alors $q^n$ n'a pas de limite.
  \end{itemize}
\end{enumerate}


\exo[2]{Exemple 6 :}
\begin{enumerate}
    \item 
On considère la suite $u$ définie par : 
    $\begin{cases}
        u_0=1 \\
        u_{n+1}=\frac{5u_n+3}{3+u_n}
    \end{cases}$
 \begin{enumerate} 
     \item Montrer par récurrence que $u_n \geq 0$ puis que $u_n-3\leq 0$.
     \item  Montrer que $u_n$ est croissante.
      \item En déduire qu'elle converge et déterminer sa limite.
      \end{enumerate}
\item On considère la suite $u$ définie par : 
    $\begin{cases}
        u_0=1 \\
        u_{n+1}=\frac{1}{1+u_n}
    \end{cases}$
    On pose $f(x)=\frac{1}{1+x}$
    \begin{enumerate}
      \item  Montrer que l'équation $f(x)=x$ admet une seule solution $l$ sur $]0,1[$.
    \item Montrer que pour tout $n\geq 0, u_n \in [\frac{1}{2}, 1]$.
        \item  Montrer que pour tout $x\in [\frac{1}{2}, 1], |f'(x)|\leq \frac{4}{9}$.
     \item Montrer que $|u_{n+1}-l|\leq \frac{4}{9}|u_n-l|$ et en déduire que $\lim\limits_{n \rightarrow +\infty} u_n= l$.
     \end{enumerate}
\item Calculer les limites des suites suivantes :  
\begin{enumerate}
    \item $$\forall n \in \mathbb{N^*}, u_n=\frac{3^n-5^n}{3^n+5^n}$$
    \item $$\forall n \in \mathbb{N^*}, u_n=\frac{n-\sqrt n}{n^2-n}  $$
    \item $$\forall n \in \mathbb{N^*}, u_n=\sqrt{n^2+n+1}- \sqrt{n^2+1}  $$
    \item $$\forall n \in \mathbb{N^*}, u_n=\frac{n^2+\cos n}{2^n+\sin n}  $$
    \item $$\forall n \in \mathbb{N^*}, u_n=\frac{\ln (1+2\tan (1/n))}{\sin (1/n)}  $$
\end{enumerate}
\end{enumerate}

\newpage

\section{Suites extraites et suites adjacentes}

\subsection{Suites extraites}

\begin{Def}\textbf{Suite extraite}
    \vspace{1em}

Une \textbf{suite extraite} de la suite $(u_n)$ est une suite $(u_{\phi(n)})$ où $\phi(n)$ est une suite strictement croissante d'indices.
\end{Def}

En particulier, les suites $(u_{2n})$ et $(u_{2n+1})$ sont extraites de $(u_n)$.

Le résultat le plus utile sur les suites extraites est le suivant :

\begin{Thm}\textbf{Propriété fondamentale des suites extraites}
    \vspace{1em}

Si une suite $(u_n)$ converge vers $\ell$, alors toute suite extraite de $(u_n)$ converge vers $\ell$.
\end{Thm}

\begin{Rmq}$\,$

On se sert surtout de ce résultat pour montrer qu'une suite n'a pas de limite. \vspace{1em}

Par exemple, considérons la suite $u_n=(-1)^n$:
\begin{itemize}
\item La suite extraite $(u_{2n})$ tend vers 1 
\item alors que la suite extraite $(u_{2n+1})$ tend vers -1.
\end{itemize}
donc la suite $(u_n)$ n'a pas de limite. -1 et 1 sont appelés \textbf{valeurs d'adhérence} dans ce cas, ces deux valeurs "attirent" une infinité de terme de la suite $(u_n)$.

\vspace{1em}

La réciproque de ce résultat est moins restrictive, il suffit que $(u_{2n})$ et $(u_{2n+1})$ convergent vers la même limite et $(u_n)$ converge. 
Cela fonctionne avec n'importe quelle partition des entiers naturels, c'est une méthode de raisonnement très utilisée en arithmétique, notamment avec des tableaux de congruences. \vspace{1em}
\end{Rmq}

\exo[2]{Exemple 7 :}
Soit $(u_n)$ une suite réelle définie sur les entiers strictement positifs vérifiant :
$$0\leq u_{m+n} \leq \frac{m+n}{mn}$$
Pour tout $m,n\in \mathbb{N^*}$.
Montrer que $(u_n)$ tend vers 0 en utilisant les suites extraites de termes pairs et impairs.

\newpage

\subsection{Suites adjacentes}

\begin{Def}\textbf{Suites adjacentes}
    \vspace{1em}

Deux suites $(u_n)$ et $(v_n)$ sont dites \textbf{adjacentes} si :
\begin{enumerate}
    \item $(u_n)$ est croissante
    \item $(v_n)$ est décroissante
    \item $\lim\limits_{n \rightarrow +\infty} |v_n-u_n|=0$
\end{enumerate}
\end{Def}

\begin{Thm}\textbf{Propriété des suites adjacentes}
    \vspace{1em}

Si $(u_n)$ et $(v_n)$ sont deux suites adjacentes, alors :

Pour tout entier $n$, $u_n \leq v_n$ et $(u_n)$ et $(v_n)$ sont convergentes et ont la même limite.
\end{Thm}

\exo[2]{Exemple 8 :}
Montrer que les suites suivantes sont adjacentes :
\begin{enumerate}
    \item $\forall n \in \mathbb{N^*}, \quad u_n=3-\frac{1}{n}$ et $v_n=3+\frac{1}{n}$
    \item $\forall n \in \mathbb{N^*}, \quad u_n= \sum_{k=0}^n \frac{1}{k!}$ et $v_n=u_n+\frac{1}{n\times n!}$
\end{enumerate}

\section{Exercices}

\vspace{1em}
\hrule
\vspace{1em}


\exo[1]{Sommes}

Soit la suite $u$ définie par : 
$$u_n=(n+1)^2-n^2$$
\begin{enumerate}
    \item Calculer les trois premiers termes.
    \item Montrer que cette suite est arithmétique et préciser sa raison.
    \item Calculer la somme des 100 premiers nombres impairs.
\end{enumerate}

\vspace{1em}
\hrule
\vspace{1em}

\newpage

\exo[2]{Sommes}

On considère les suites $u$ et $v$ telles que :
$$\forall n \in \mathbb{N}, \quad u_n=\frac{3\times 2^n-4n+3}{2}; \quad u_n=\frac{3\times 2^n+4n-3}{2}$$
\begin{enumerate}
    \item Soit $w$ la suite telle que $w_n=u_n+v_n$. Montrer que $w$ est géométrique et calculer la somme des $(n+1)$ premiers termes de la suite $w$.
    \item Soit $t$ la suite telle que $t_n=u_n+v_n$. Montrer que $t$ est arithmétique et calculer la somme des $(n+1)$ premiers termes de la suite $t$.
    \item Calculer $S_n=\sum_{k=0}^n u_k$
\end{enumerate}

\vspace{1em}
\hrule
\vspace{1em}

\exo[2]{Suites extraites}

On considère les suites $u$ et $v$ telles que :
$$\begin{cases}
    u_0=0 \\
    u_{n+1}=\frac{u_n+v_n}{2}
\end{cases}$$
$$\begin{cases}
    v_0=4 \\
    v_{n+1}=\frac{u_{n+1}+v_n}{2}
\end{cases}$$
\begin{enumerate}
    \item Soit w la suite définie par :
    $$w_n=v_n-u_n$$
    Montrer que $w$ est géométrique et déterminer sa limite.
    \item Montrer que $u$ est croissante et $v$ décroissante.
    \item On considère la suite $t$ définie par :
    $$t_n=\frac{u_n+2v_n}{3}$$
    Montrer que $t$ est constante et en déduire la limite de $u$ et $v$.
\end{enumerate}

\vspace{1em}
\hrule
\vspace{1em}

\exo[2]{Suites adjacentes}

on considère la suite $u$ définie par :
$$\begin{cases}
    u_0=3\\
    u_{n+1}=\frac{3u_n+1}{4+u_n}
\end{cases}$$
On considère la suite $v$ définie par :
$$v_n=\frac{u_n-1}{u_n+2}$$
\begin{enumerate}
    \item Montrer que $v$ est géométrique et montrer qu'elle converge.
    \item En déduire l'expression de $u_n$ en fonction de $n$.
    \item trouver la limite de $u$.
\end{enumerate}

\vspace{1em}
\hrule
\vspace{1em}

\newpage

\exo[2]{Équivalents}

Quels sont les équivalents corrects en l'infini parmi les proposition suivantes ?
\begin{enumerate}
    \item $n \sim n+1$
    \item $n^2 \sim n^2+n$
    \item $\ln n \sim \ln (10^6n)$
    \item $e^n\sim e^{n+10^6}$
    \item $e^n \sim e^{2n}$
    \item $\ln n \sim \ln{(n+1)}$
\end{enumerate}

\vspace{1em}
\hrule
\vspace{1em}

\exo[2]{Équation}

Déterminer un équivalent le plus simple possible de chacune des suites suivantes en l'infini.
\begin{enumerate}
    \item $u_n =\frac{1}{n-1}-\frac{1}{n+1}$
    \item $u_n=\sqrt{n+1}-\sqrt{n-1}$
    \item $u_n=\frac{n^3-\sqrt{1-n^2}}{\ln n-2n^2}$
    \item $u_n= \sin{(\frac{1}{\sqrt{n+1}})}$
\end{enumerate}

\vspace{1em}
\hrule
\vspace{1em}

\exo[2]{Suite}

On définit la suite $u$ par :
$$\begin{cases}
    u_0=\frac{7}{2}\\
    u_{n+1}=f(u_n)
\end{cases}$$
où $f$ est la fonction définie sur l'intervalle $[-1;+infty[$ par : 
$$f(x)=4-\frac{1}{4} \ln x$$
\begin{enumerate}
    \item Dresser le tableau de variations de $f$ sur son ensemble de définition.
    \item Montrer que : 
    $$3 \leq u_n \leq 4$$
    \item Montrer que pout tout réel $x$ de l'intervalle $[3,4]$, on : 
    $$|f'(x)|\leq \frac{1}{12}$$
    \item Montrer l'existence d'un unique $\alpha \in [3,4]$ tel que $f(\alpha)=\alpha$.
    \item Montrer que : 
    $$\forall n \in \mathbb{N}, \quad |u_{n+1}-\alpha| \leq \frac{1}{12}|u_n-\alpha$$
    Puis que :
   $$\forall n \in \mathbb{N}, \quad |u_{n}-\alpha| \leq (\frac{1}{12})^{n+1}$$  
   \item En déduire la limite de $u$.
\end{enumerate}

\vspace{1em}
\hrule
\vspace{1em}

\newpage

\exo[2]{Suite}

Un jardinier décide de verser un seau d'eau au pied de chacun des peupliers qui bordent son champ. Il y a 95 peupliers, en ligne droite, espacés de 3 mètres et la prise d'eau est au pied du premier. Le jardinier ne porte qu'un seul seau à la fois. 
On suppose qu'il arrose les peupliers dans l'ordre d'alignement du plus proche au plus éloigné des arbres. On pose $u_n$ la distance aller-retour pour arroser le $n$-ième arbre et revenir au point d'eau. 
\begin{enumerate}
    \item Calculer les distances qui correspondent aux 4 premiers aller-retours.
\item  Quelle est la nature de $u$? préciser sa raison et son premier terme.
\item Exprimer $u_n$ en fonction de $n$.
\item Calculer la distance que doit parcourir le jardinier pour arroser tous ses peupliers.
\item Il commence sa lourde tâche ; mais épuisé, il décide de s'arrêter après avoir parcouru exactement 10 266 m. combien a-t-il arrosé d'arbres ?
\end{enumerate}

\vspace{1em}
\hrule
\vspace{1em}

\exo[3]{Suite}

Pour tout entier $n \geq 1$, on définit la fonction $f_n$ par :
$$\forall x \in \mathbb{R^+}, \quad f_n(x)=x^n+9x^2-4$$
\begin{enumerate}
    \item Montrer que l'équation $f_n(x)=0$ n'a qu'une seule solution strictement positive, notée $u_n$.
    \item Vérifier que $\forall x \in \mathbb{N^*}, \quad u_n \in ]0, \frac{2}{3}[$
    \item Montrer que, pour tout $x$ élément de $]0,1[$, on a $f_{n+1}(x)<f_n(x)$.
    \item En déduire le signe de $f_n(u_{n+1}$, puis les variations de la suite $(u_n)$.
    \item Montrer que $u$ converge.
    \item Déterminer la limite de $(u_n)^n$ en $+ \infty$.
    \item Donner enfin la valeur de $l$.
\end{enumerate}

\end{document}
