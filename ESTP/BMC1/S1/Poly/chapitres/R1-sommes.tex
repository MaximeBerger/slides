\documentclass[../PolyS1.tex]{subfiles}
\begin{document}

\section{Sommes}
\subsection{Définition}

\begin{Def}\textbf{Somme}
    \vspace{1em}

On considère les $n$ nombres réels (ou complexes) $a_{1}, ...,a_{n}$, l'écriture : 

$$\sum_{k=1}^{n} a_{k}=a_{1}+...+a_{n}$$

Représente la somme des $a_{1}, ...,a_{n},$ et $k$ est l'indice muet de sommation.
\end{Def}


On a les propriétés suivantes :

$$\sum_{k=1}^{n} \lambda a_{k}= \lambda \sum_{k=1}^{n}  a_{k}$$

$$\sum_{k=1}^{n} (a_{k}+b_{k})=\sum_{k=1}^{n} a_{k}+\sum_{k=1}^{n} b_{k}$$


\subsection{Techniques de sommation}
Pour le calcul des sommes, nous allons nous concentrer sur trois techniques de sommations.

\begin{enumerate}
    \item \underline{\textbf{Les sommes de référence.}}

    Il s'agit de formules vues au lycée dans le cadre de l'étude des suites numériques.

    \begin{Thm}\textbf{Somme d'une suite arithmétique}
        \vspace{1em}

    La somme de terme d'une suite arithmétique :
    $$\sum_{k=0}^{n} a_{k}=\frac{(a_{0}+a_{n}) (n+1)}{2}$$
    $(n+1)$ représente le nombre de terme qu'il y a entre $0$ et $n$, \newline 
    il peut bien sûr en être autrement en fonction de la somme que l'on regarde.
    \end{Thm}

    \begin{Thm}\textbf{Somme d'une suite géométrique}
        \vspace{1em}

    La somme de terme d'une suite géométrique :
    $$\sum_{k=0}^{n} a_{k}=a_{0}\times \frac{1-q^{n+1}}{1-q}$$

    $q$ étant la raison de la suite géométrique.\newline 
    $(n+1)$ représente le nombre de terme qu'il y a entre $0$ et $n$, \newline 
    il peut bien sûr en être autrement en fonction de la somme que l'on regarde.
    \end{Thm}

    \exo[1]{Exemple 1 :}

    Calculer :
    $$\sum_{k=0}^{n} (2^k+k)$$

    \item \underline{\textbf{Le téléscopage}}

    \begin{Thm}\textbf{Technique du téléscopage}
        \vspace{1em}

    L'objectif est de jouer sur un changement d'indice pour simplifier la somme.

     \begin{align*}
     \sum_{k=1}^{n} a_{k+1}-a_{k}
     &= \sum_{k=1}^{n} a_{k+1}-\sum_{k=1}^{n} a_{k} \\
     &= \sum_{k=2}^{n+1} a_{k}-\sum_{k=1}^{n} a_{k} \\
     &= a_{n+1}+\sum_{k=2}^{n} a_{k}-\sum_{k=2}^{n} a_{k}-a_{1} \\
     &= a_{n+1}-a_{1}
     \end{align*}
    \end{Thm}

     \exo[1]{Exemple 2 :}

     Calculer :
     $$\sum_{k=2}^{n} \frac{1}{k^2-k}$$
     En remarquant que : $\frac{1}{k^2-k}=\frac{1}{k-1}-\frac{1}{k}$
     \vspace{1em}
     \hrule
     \vspace{1em}

     \item \underline{\textbf{Le binôme de Newton}}
     \nopagebreak[4]
 \begin{Thm}\textbf{Formule du binôme de Newton}
    \vspace{1em}

 La célèbre formule pour développer une somme à une certaine puissance.
 $$(a+b)^n=\sum_{k=0}^{n} \begin{pmatrix}
     n\\
     k
 \end{pmatrix}
 a^k\times b^{n-k} $$
 Où $\begin{pmatrix}
     n\\
     k
 \end{pmatrix}$ est le coefficient binomial "$k$ parmi $n$" :
 $$\begin{pmatrix}
     n\\
     k
 \end{pmatrix}= \frac{n!}{(n-k)!k!}$$
 Et $n!=n\times (n-1)\times ... \times 2 \times1$
 \end{Thm}

 \exo[1]{Exemple 3 :}
 Calculer :
 $$\sum_{k=0}^{n} \begin{pmatrix}
     n\\
     k
 \end{pmatrix}
 3^k $$    
\end{enumerate}

\section{Produits}

\begin{Def}\textbf{Produit}
    \vspace{1em}

On considère les $n$ nombres réels (ou complexes) $a_{1}, ...,a_{n}$, l'écriture :
$$\prod_{k=1}^{n} a_{k}=a_{1}\times...\times a_{n}$$
On peut écrire par exemple :
$$n!=\prod_{k=1}^{n}k$$
\end{Def}


On a les propriétés suivantes :
$$\prod_{k=1}^{n} \lambda a_{k}= \lambda^n \prod_{k=1}^{n}  a_{k}$$

$$\prod_{k=1}^{n} (a_{k} \times b_{k})=\prod_{k=1}^{n} a_{k} \times \prod_{k=1}^{n} b_{k}$$


Le téléscopage est la technique que l'on utilise la plupart du temps pour calculer une somme.\\

\vspace{1em}
\hrule
\vspace{1em}

\exo[2]{Exemple 4 :}
Calculer :
$$\prod_{k=1}^{n} \Big(1-\frac{1}{k^2}\Big)$$
en remarquant que $1-\frac{1}{k^2}=(1-\frac{1}{k})(1+\frac{1}{k})$

\section{Exercices}

\vspace{1em}
\hrule
\vspace{1em}

\exo[2]{Propriétés}
Parmi les formules suivantes, lesquelles sont vraies ? 
\begin{multicols}{2}
\begin{enumerate}
    \item $\displaystyle \sum_{k=1}^{n} (a+b_{k})=a+\sum_{k=1}^{n} b_{k}$
    \item $\displaystyle \sum_{k=1}^{n} (a_{k}+b_k)=\sum_{k=1}^{n} a_{k}+\sum_{k=1}^{n} b_{k}$
    \item $\displaystyle \sum_{k=1}^{n} a_{k}b_k=\sum_{k=1}^{n} a_{k} \times \sum_{k=1}^{n} b_{k}$
    \item $\displaystyle \sum_{k=1}^{n} a_{k}^p=(\sum_{k=1}^{n} a_{k})^p$
    \item $\displaystyle \sum_{k=1}^{n}\sum_{i=1}^{n} a_{k,i}= \sum_{i=1}^{n}\sum_{k=1}^{n} a_{k,i}$
    \item $\displaystyle \prod_{k=1}^{n} (a_{k}+b_k)=\prod_{k=1}^{n} a_{k}+\prod_{k=1}^{n} b_{k}$
    \item $\displaystyle \prod_{k=1}^{n} (a_{k} \times b_k)=\prod_{k=1}^{n} a_{k}\times \prod_{k=1}^{n} b_{k}$
\end{enumerate}
\end{multicols}

\vspace{1em}
\hrule
\vspace{1em}

\exo[2]{Sommes}
Calculer :
\begin{multicols}{2}
\begin{enumerate}
    \item $\displaystyle \sum_{k=1}^{n} \frac{k}{(k+1)!}$
    \item $\displaystyle \sum_{k=0}^{n} \frac{3^{2k}}{2^{k+2}}$
    \item $\displaystyle \sum_{k=1}^{n} \ln(1+\frac{1}{k})$
    \item $\displaystyle \sum_{k=0}^{n} (2k+3)$
    \item $\displaystyle \sum_{k=0}^{n} (-1)^k \binom{n}{k}2^{2k}5^{n-k}$
    \item $\displaystyle \sum_{k=0}^{n} \frac{1}{(k+1)(k+2)}\binom{n}{k}$
    \item $\displaystyle \sum_{0 \leq i, j \leq n}^{} i^2j^2$
    \item $\displaystyle \sum_{0 \leq i, j \leq n}^{} (i+j)^2$
    \item $\displaystyle \prod_{k=1}^{n} 2k$
    \item $\displaystyle \prod_{k=1}^{n} \frac{2k+1}{2k-1}$
    \item $\displaystyle \prod_{k=2}^{n} \frac{k^2-1}{k}$
    \item $\displaystyle \sum_{i=0}^n \prod_{j=1}^{n} (i+j)$
\end{enumerate}
\end{multicols}

\end{document}
