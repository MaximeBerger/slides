\documentclass[../PolyS1.tex]{subfiles}
\begin{document}

\section{Introduction}

Les complexes sont partout en science, vous les rencontrerez en mécanique, en électricité, dès qu'il s'agira de résoudre des équations polynomiales ou différentielles .. 
%En guise d'introduction, nous présentons un extrait du cours  de MPSI, PCSI, PTSI, TSI de messieurs Capaces, Soyeur et Vieillard-Baron proposé sur le site Les-Mathématiques.net.
%
%\includegraphics[scale=1]{COURS-PTSI-complexes}
%
%Cette définition du corps des complexes ne sera pas utilisée par la suite, mais on gardera en mémoire qu'un nombre complexe est un couple de nombre réels...
%
%\section{Rappels}

%%%%%%% FORME ALGEBRIQUE
\section{Différentes formes}
\subsection{Forme algébrique}

\begin{Def}{\textbf{Nombre complexe}}
    \vspace{1em}

On note $i$ un nombre imaginaire tel que $i^2 = -1$.

Un nombre complexe $z$ est un nombre qui s'écrit sous la forme (algébrique) $$z=a + ib$$ avec $a$ et $b$ réels.

L'ensemble des nombres complexes est noté $\C$.
\end{Def}

\begin{Def}{\textbf{Partie réelle, partie imaginaire}}
    \vspace{1em}

Lorsqu'on écrit $z = a + ib$, 
on dit que 
\begin{itemize}
    \item $a$ est la partie réelle de $z$
    \item $b$ est la partie imaginaire de $z$.
\end{itemize}
On note $a = \Re(z)$ ou $Re(z)$ et $b = \Im(z)$ ou $Im(z)$.
\end{Def}

\begin{Def}{\textbf{Nombre réel, imaginaire pur}}
    \vspace{1em}

Tout complexe $z$ tel que $b = 0$ s'écrit $z = a$ : c'est un nombre réel.

Tout complexe $z$ tel que $a = 0$ s'écrit $z = ib$ : c'est un nombre imaginaire pur.
\end{Def}


\begin{bclogo}[couleur = yellow!30, arrondi = 0.1, ombre = true, couleurOmbre=black!10, logo=\bccrayon]{~Addition}

L'addition et la multiplication dans $\C$ respectent les m\^emes règles de calcul que dans $\R$ en tenant compte du fait que $i^2 = -1$.
\end{bclogo}

\vspace{1em}
\hrule
\vspace{1em}

\exo[1]{Premier calcul algébrique}
On pose $z_1=1+2i$ et $z_2=2-i$.
Calculer sous forme algébrique
$$z_1+z_2, \qquad z_1z_2, \qquad \frac{z_1}{z_2} \qquad \text{et} \qquad 2z_1+3z_2.$$

\vspace{1em}
\hrule
\vspace{1em}

\exo[1]{Algébrique}
On pose $z_1=2+3i$ et $z_2=3-4i$.
Calculer sous forme algébrique
$$z_1+z_2, \qquad z_1z_2, \qquad \frac{z_1}{z_2} \qquad \text{et} \qquad 2z_1+3z_2.$$

\vspace{1em}
\hrule
\vspace{1em}

\exo[2]{Équations}
Résoudre dans $\C$ les équations suivantes:

\begin{enumerate}
 \item [$\bullet$] $z+2=i(z-1)$
 \item [$\bullet$] $iz-3=2i$
 \item [$\bullet$] $\frac{z+1}{z-1}=1+i$.
\end{enumerate}

\vspace{1em}
\hrule
\vspace{1em}

\exo[2]{Système}
Résoudre dans $\C$ le système 
$$\left\lbrace 
\begin{array}{rcrcl}
2z_1&+&z_2&=&4\\ 
-2iz_1&+&z_2&=&0
  \end{array}\right.$$ 

\vspace{1em}
\hrule
\vspace{1em}

\subsection{Représentation géométrique}
\noindent
Le plan complexe est le plan muni d'un repère orthonormé direct. %$\RE$.\\
Soit $z=a+ib$.
\begin{itemize}
 \item [$\bullet$]{$M\left(a;b \right)$ est {\bf l'image} du complexe $z=a+ib$
 
De même $\overrightarrow{OM}$ est le vecteur image de $z=a+ib$}
 \item [$\bullet$]$z=a+ib$ est {\bf l'affixe} du point $M$ ou du vecteur $\overrightarrow{OM}$
\end{itemize}
\begin{center}
 
\includegraphics[scale=0.5]{images/fig1.png}
\end{center}

%\newpage%gestion bclogo
\begin{Prop} \textbf{Nombre complexe et vecteur}
    \vspace{1em}

Soit $z_1$ et $z_2$ les complexes d'image les vecteurs $\overrightarrow{V_1}$ et $\overrightarrow{V_2}$ et soit $\alpha$ et $\beta$ deux réels :
 \begin{itemize}
  \item [$\bullet$] l'image de $z_1 +z_2$ est le vecteur $\overrightarrow{V_1} + \overrightarrow{V_2}$
 \item [$\bullet$]l'image de $\alpha z_1 $ est le vecteur $\alpha \overrightarrow{V_1}$
 \item [$\bullet$]l'image de $\alpha z_1 + \beta z_2$ est le vecteur $\alpha \overrightarrow{V_1}+ \beta \overrightarrow{V_2}$.
 \end{itemize}

\end{Prop}



\vspace{1em}
\hrule
\vspace{1em}
\exo[1]{Représentation}

Représenter les images $A$, $B$ et $C$ des complexes suivants :
$$z_{A}=2+3i\quad\quad z_{B}=3i-2\quad\quad z_{C}=-1-2i$$
\noindent Donner les affixes des milieux des segments $[AB]$ et $[BC]$.

\vspace{1em}
\hrule
\vspace{1em}

\subsection{Conjugué d'un complexe}

\begin{Def}\textbf{Nombre conjugué}
    \vspace{1em}

Le conjugué du complexe $z=a+ib$ est le complexe $\bar{z}=a-ib$
\end{Def}

 %\begin{Exs}
\newpage

 \exo[1]{Applications}
 \begin{multicols}{3}
 \begin{itemize}
 \item []$\overline{1+3i}=$%1-3i$
 \item []$\overline{-2i}=$%2i$
 \item []$\overline{i}=$%-i$
 \item []$\overline{3}=$%3$
  \item [] $\overline{-1}=$%-1$\\
 \end{itemize}
 \end{multicols}
% \end{Exs}


\begin{Prop}\textbf{Image}
    \vspace{1em}

 L'image de $\bar{z}$ est le symétrique de l'image de $z$ par rapport à l'axe des abscisses.
 
L'image de $-z$ est le symétrique de l'image de $z$ par rapport à l'origine $O$ du repère.
\end{Prop}

 \begin{center}
\includegraphics[scale=0.5]{images/fig2.png}
\end{center}
%\begin{thm}
\begin{Prop}\textbf{Opérations et nombre conjugué}
    \vspace{1em}

\begin{multicols}{2}
\begin{enumerate}%[(a)]
\item $z \bar{z}=(Re(z))^2+(Im(z))^2=\vert z \vert^2$
\item $z+\bar{z}=2Re(z)$
\item $z-\bar{z}=2i\,Im(z)$
\item $\bar{\bar{z}}=z$
\end{enumerate}
\begin{enumerate}%[(a')]
\item $\overline{z_1+z_2}=\overline{z_1}+\overline{z_2}$
\item $\overline{z_1 \times z_2}=\overline{z_1} \times \overline{z_2}$
\item si $z_2 \ne 0$ : $\overline{\left( \dfrac{z_1}{z_2}\right) }=\dfrac{\overline{z_1}}{\overline{z_2}}$\\ 
\end{enumerate}
\end{multicols}

\end{Prop}
%\end{thm}
%\vspace{0.5cm}

\vspace{1em}
\hrule
\vspace{1em}

\exo[2]{Démonstration}

 Faire la démonstration des propriétés précédentes.\\
%}
%
%\sol{
%\vspace{2cm}

\vspace{1em}
\hrule
\vspace{1em}
%
%{\textcolor{blue}{\hrulefill{} \ding{110}}}}
%
\newpage

\exo[1]{Conjugué}

 Donner les conjugués des complexes suivants :\\
 
$$z_1 = \left( 3-2i \right) \left( i-4 \right)\quad\quad z_2 = \frac{3-2i}{2i-4} \quad\quad z_3 = \frac{\left( 2-i\right) \left( 1+i\right) }{2i\left( 5+3i\right)}$$
%}
%

\vspace{1em}
\hrule
\vspace{1em}

%%%%%%%%% FORME TRIGONOMETRIQUE
\subsection{Forme trigonométrique  : module et argument }
\begin{Def}{Module et argument}
    \vspace{1em}
    
Soit $z=a+ib$ un complexe d'image $M$.\\

\noindent Le module de $z$, noté $|z|$, est la norme du vecteur $\overrightarrow{OM}$.\\
Si $z \ne 0$, un argument de $z$, noté $\arg(z)$, est une mesure en radian, de l'angle orienté $\left(\vec{i},\overrightarrow{OM} \right) $.\\
\begin{center}
\begin{tabular}{lll}
\multirow{9}*{
%\begin{pspicture*}(-1,-1.5)(7,6)
%	\def\xmin{0} \def\xmax{6} \def\ymin{0} \def\ymax{5}
%	\psaxes[linewidth=1pt,Dx=10,Dy=10]{->}(0,0)(\xmin,\ymin)(\xmax,\ymax)
%	\psset{algebraic=true,xunit=1cm,yunit=1cm}
%	%\psgrid[gridlabels=0pt,gridwidth=.3pt, gridcolor=gray, subgridwidth=.3pt, subgridcolor=gray, subgriddiv=1](0,0)(0,0)(8,8)
%	\uput[dl](0,0){O}\uput[ur](5,4){M($z$)}\uput[d](5,0){$a$}\uput[l](0,4){$b$}\uput[d](0.5,0){$\overrightarrow{i}$}\uput[l](0,0.5){$\overrightarrow{j}$}
%	\uput[ur](1,0.2){$\theta$}\uput[u](2,2){$\rho$}
%	\psline[linestyle=dashed](0,4)(5,4)\psline[linestyle=dashed](5,0)(5,4)\psline[linewidth=1.5pt]{->}(0,0)(5,4)\psline[linewidth=1.5pt]{->}(0,0)(1,0)\psline[linewidth=1.5pt]{->}(0,0)(0,1)
%	\psarc{->}(0,0){1}{0}{40}
%   \end{pspicture*}
\includegraphics[scale=0.4]{images/fig3.png}
}
			      &&$z=a+ib$ \\ 
			      &&\\
			      &&$\rho=|z|=OM=\| \overrightarrow{OM}\|=\sqrt{a^2+b^2}$\\&\\
			      &&$\theta = \arg(z)=\left(\vec{i},\overrightarrow{OM} \right)$\\
			      &&\\
			      &&$a= \rho \cos \theta$ \\
			      &&\\
			      &&$ b=\rho \sin \theta$  
\end{tabular}
\end{center}$\;$\\
\end{Def}
%
% \begin{Rem}
% - On impose  $z\neq0$ car sinon, l'angle orienté $\left(\vec{i},\overrightarrow{OM} \right) $ n'a aucun sens.\\

% - Si $\theta$ est un argument de $z$ alors tout nombre qui s'écrit $\theta+2k \pi$ o\`u $k$ appartient \`a $\Z$ est aussi un argument de $z$. On note $\theta =\arg(z) ~[2\pi ].$\\
% \end{Rem}

%\medskip
\exo[1]{Questions}
\begin{itemize}
\item Où se situe l'image d'un affixe dont le module est égal à $2$ ?\\
\item Où se situe l'image d'un affixe dont le module est égal à $0$ ?\\
\item Où se situe l'image d'un affixe dont le module est égal à $1$ ?\\
\item Où se situe l'image d'un affixe dont l'argument est égal à $0$ ?\\
\item Où se situe l'image d'un affixe dont l'argument est égal à $\dfrac{\pi}{2}$ ?\\
\end{itemize} 
\medskip
%

\vspace{1em}
\hrule
\vspace{1em}

\exo[1]{Argument et module}

Soit $z=2-2i\sqrt 3$.
 Calculer $|z|$ et $\arg(z)$.
%
%\sol{
%\vspace{2cm}%$|z|=\sqrt{2^2+(2\sqrt 3)^2}=\sqrt{4+12}=4$ et $\cos\theta=\frac{2}{4}=\frac{1}{2}$ et $\sin\theta=\frac{-2\sqrt 3}{4}=\frac{-\sqrt 3}{2}$, donc $\theta=-\frac{\pi}{3}\, [2\pi]$.
%
%{\textcolor{blue}{\hrulefill{} \ding{110}}}}
%%\newpage%gestion bclogo
%

\vspace{1em}
\hrule
\vspace{1em}

\begin{Prop}\textbf{Propriétés du module et de l'argument}
    \vspace{1em}

 \begin{enumerate}%[(i)]
  \item $|-z|=|z|$ et si $z\neq0$ alors $\arg(-z)=\pi+\arg(z) [2\pi]$
 \item $|\bar{z}|=|z|$ et si $z\neq0$ alors $\arg(\bar{z})=-\arg(z) [2\pi]$
  \item 
%  		{\obelix{$$si\, z\, est\, un\, réel\, : $$}}
		\begin{itemize}
                   \item [$\bullet$]$\arg(z)=0 ~[2\pi]$ si $\mathcal{R}e(z)>0$
		\item [$\bullet$]$\arg(z)=\pi~ [2\pi]$ si $\mathcal{R}e(z)<0$
%  		{\obelix{$$si\, z\, est\, un\, imaginaire\, pur\,: $$}}
                   \item [$\bullet$]$\arg(z)={\pi}/{2} ~[2\pi]$ si $\mathcal{I}m(z)>0$
		\item [$\bullet$]$\arg(z)=-{\pi}/{2} ~[2\pi]$ si $\mathcal{I}m(z)<0$
                  \end{itemize}
% \medskip
 \end{enumerate}
\end{Prop}
%\end{thm}
%
Dans ce qui suit on pose $z=a+ib$, avec $\rho=|z|=||\overrightarrow{OM}||$ et $\theta=\arg(z)= \hat{\left( \vec{i},\overrightarrow{OM}\right)}$.
%%
\begin{Def}\textbf{Forme trigonométrique}
    \vspace{1em}

On appelle forme trigonométrique de $z$ l'expression :
$$z=\rho(\cos\theta+i\sin\theta)$$
\end{Def}

\exo[1]{Placement}

Soit $z=i+\sqrt 3$.
Calculer $|z|$ et $\arg(z)$.

 Donner la forme trigonométrique. Placer le point image d'affixe $z$ dans le plan% $\RE$.\\
%
%\sol{
%\vspace{3cm}
%
%{\textcolor{blue}{\hrulefill{} \ding{110}}}

% \begin{Rem}Tout nombre complexe posséde une forme trigonométrique, encore faut-il \^etre sûr de l'avoir sous les yeux. On propose ci-dessous des formes que ne sont pas des formes trigonométriques mais que l'on transforme gr\^ace aux propriétés trigonométriques ou gr\^ace au positionnement du point image correspondant dans le plan $\RE$ afin d'obtenir la forme trigonométrique. \`A titre d'exercice, identifier le probléme par rapport \`a la définition d'une forme trigonométrique, puis repérer les propriétés utilisées pour la modification :$$\begin{array}{rrl}(a)&-2\bigg(\cos\left(\frac{\pi}{3}\right)+i\sin\left(\frac{\pi}{3}\right)\bigg)&=2\bigg(-\cos\left(\frac{\pi}{3}\right)-i\sin\left(\frac{\pi}{3}\right)\bigg)\\
% &&=2\bigg(\cos\left(\frac{\pi}{3}+\pi\right)+i\sin\left(\frac{\pi}{3}+\pi\right)\bigg)\\
% &&=2\bigg(\cos\left(\frac{4\pi}{3}\right)+i\sin\left(\frac{4\pi}{3}\right)\bigg)\\
% &&\\
% (b)&5\bigg(\cos\left(\frac{\pi}{3}\right)-i\sin\left(\frac{\pi}{3}\right)\bigg)&=5\bigg(\cos\left(-\frac{\pi}{3}\right)+i\sin\left(-\frac{\pi}{3}\right)\bigg)\end{array}$$
% \end{Rem}
%\newpage%%
%\begin{thm}

\begin{Def}\textbf{Distance et module}
    \vspace{1em}

Soient $A$ et $B$ deux points du plan complexe d'affixe respective $z_{A}$ et $z_{B}$, on a alors 
\vspace{1em}

\begin{minipage}[htbp]{0.3\linewidth}
$$AB=|z_{B}-z_{A}|$$
\end{minipage}
\begin{minipage}[htbp]{0.5\linewidth}
\begin{center}
 
\includegraphics[scale=0.3]{images/fig4.png}
\end{center}
% \begin{pspicture*}(-1,-1)(7,7)
%	\def\xmin{0} \def\xmax{8} \def\ymin{0} \def\ymax{8}
%	\psaxes[linewidth=1pt,Dx=10,Dy=10]{->}(0,0)(-1,-1)(6,6)
%	\psset{algebraic=true,xunit=1cm,yunit=1cm}
%	\psline[linewidth=1.5pt]{->}(0,0)(0,1)\psline[linewidth=1.5pt]{->}(0,0)(1,0)
%	\uput[d](0.5,0){$\overrightarrow{i}$}\uput[l](0,0.5){$\overrightarrow{j}$}
%	%\psgrid[gridlabels=0pt,gridwidth=.3pt, gridcolor=gray, subgridwidth=.3pt, subgridcolor=gray, subgriddiv=1](0,0)(0,0)(8,8)
%	\uput[dl](0,0){O}\uput[ur](5,5){A($z_{\tiny A}$)}\uput[ul](3,2){B($z_{\tiny B}$)}
%	\psline[linestyle=dashed](0,5)(5,5)\psline[linestyle=dashed](5,0)(5,5)\psline[linestyle=dashed](3,0)(3,2)\psline[linestyle=dashed](0,2)(3,2)
%	\uput[d](5,0){$x_A$}\uput[l](0,5){$y_A$}\uput[d](3,0){$x_B$}\uput[l](0,2){$y_B$}
%	\psline[linewidth=1.5pt]{-}(3,2)(5,5)
%   \end{pspicture*}
\end{minipage}
\end{Def}
%\end{thm}
%
%\exo{Démontrer la proposition 2.6}
%
%\sol{
%\vspace{2cm}
%
%{\textcolor{blue}{\hrulefill{} \ding{110}}}}
%
\exo[2]{Calculs}

Calculer d'une part $AB=\sqrt{\vert|\overrightarrow{AB}\vert|^2}$ et $|z_B-z_A|$ d'autre part.\\


\subsection{Forme exponentielle}

\begin{Def}\textbf{Forme exponentielle}
    \vspace{1em}

Pour tout réel $\theta$, on note $e^{i\theta}=\cos\theta+i\sin\theta$. 

Ainsi $e^{i\theta}$ est le complexe de module 1 et d'argument $\theta$.
\vspace{1em}

Tout complexe $z$ \textbf{non nul} de module $\rho$ et d'argument $\theta$ peut donc s'écrire $$z=\rho e^{i\theta}.$$
\end{Def}

% \begin{Rem}On en déduit que $$\overline{z}=\rho e^{-i\theta},\quad z^n=\rho^ne^{in\theta},\quad(\overline{z})^n=\rho^ne^{-in\theta}=\overline{z^n},\quad \vert z\vert ^n=\vert z^n\vert.$$
% \end{Rem}

\begin{Ex}Soit $z=3e^{i\frac{\pi}{6}}$, alors
$$ \overline{z}=3e^{-i\frac{\pi}{6}}\qquad z^2=9e^{i\frac{\pi}{3}}\qquad\overline{z}^2=9e^{-i\frac{\pi}{3}}$$
\end{Ex}

%\begin{thm}
\begin{Prop}\textbf{Formules d'Euler}
    \vspace{1em}

$$\cos \theta=\dfrac{e^{i\theta}+e^{-i\theta}}{2}\qquad \text{et} \qquad \sin \theta=\dfrac{e^{i\theta}-e^{-i\theta}}{2i}$$
\end{Prop}
%\end{thm}

% \begin{Dem}  \`A partir de $e^{i\theta}=\cos\theta+i\sin\theta$ et $e^{i(-\theta)}=\dots$.
% \vspace{2cm}

% \textcolor{red}{\hrulefill{} \ding{110}}
% \end{Dem}

%\begin{Prec}
Les formules d'Euler sont équivalentes à $$e^{i\theta}+e^{-i\theta}=2\cos \theta\qquad \text{ et }\qquad e^{i\theta}-e^{-i\theta}=2i\sin\theta$$
%\end{Prec}	      

\vspace{1em}
\hrule
\vspace{1em}

\exo[1]{Forme exponentielle}

\'Ecrire sous forme exponentielle les nombres complexes suivants :
$$1+i=\qquad 1-i=\qquad i=\qquad -i=$$
$$\dfrac{-1+i\sqrt 3}{1-i}=\qquad (1+i)^n=\qquad 7i=\qquad -7=$$
%}

%\sol{
%\textcolor{blue}{\hrulefill{} \ding{110}}}
%

\vspace{1em}
\hrule
\vspace{1em}

\newpage

\exo[2]{Forme exponentielle version 2}

\begin{enumerate}
      \item Mettre sous forme exponentielle les complexes $z_1=\sqrt{3}+i$ et $z_2=2i$.\\
      \item En déduire la forme exponentielle puis la forme algébrique de $z_1\times z_2$.\\
      \item Calculer directement la forme algébrique de $z_1\times z_2$.\\
     \end{enumerate}
%}
%
%\sol{
%\vspace{4cm}

%\textcolor{blue}{\hrulefill{} \ding{110}}}

\vspace{1em}
\hrule
\vspace{1em}

\exo[2]{Formules d'Euler}

En utilisant les formules d'Euler retrouver que $\sin (2\theta)=2\sin( \theta)\cos(\theta)$.

%\sol{
%\vspace{2cm}

%\textcolor{blue}{\hrulefill{} \ding{110}}}

\vspace{1em}
\hrule
\vspace{1em}

\exo[2]{Forme trigonométrique}

\'Ecrire les expressions suivantes en utilisant les fonctions sinus et cosinus :

\begin{itemize}
    \vspace{0.5em}
 \item[$\bullet$] $\dfrac{3e^{-i6x}+3e^{i6x}}{2} = $
 \vspace{0.5em}
 \item[$\bullet$] $\dfrac{e^{i3x}+7e^{i4x}+7e^{-i4x}+e^{-i3x}}{8} = $
 \vspace{0.5em}
 \item[$\bullet$] $\dfrac{e^{i4x}-5e^{i6x}+5e^{-i6x}-e^{-i4x}}{-4i} = $
\end{itemize}
%\vspace{0.5cm}

\vspace{1em}
\hrule
\vspace{1em}

%
%\sol{
%\textcolor{blue}{\hrulefill{} \ding{110}}}
%\newpage
\exo[3]{Linéarisation}

En utilisant les formules d'Euler, linéariser :

$$\sin (x)\cos (3x)\qquad \cos^4(x)\qquad \sin^4(x)\cos^2(x)$$

 Calculer 
$$\displaystyle\int \sin (x)\cos(3x)dx\qquad \int \cos^4(x)dx\qquad \int\sin^4(x)\cos^2(x)dx$$




%\sol{
%\vfill
%\includegraphics[scale=0.1]{exo_line_euler_corr}
%
%\textcolor{blue}{\hrulefill{} \ding{110}}}

%%%%%%%%%%%%%%%%%%%%%%%%
%

\newpage 

\section{Opérations}
\subsection{Somme et différence}
%
%\begin{thm}
\begin{Prop}\textbf{Somme}
    \vspace{1em}

Si $z_1$ a pour image $M_1$ et $z_2$ a pour image $M_2$, alors
%
\begin{center}$z_1+z_2$ a pour image $S$ o\`u  $\overrightarrow{OS}=\overrightarrow{OM_1}+\overrightarrow{OM_2}$.\end{center}
\end{Prop}
%\end{thm}
%
%\begin{Prec} 
Posons $z_1=x_1+iy_1$ et $z_2=x_2+iy_2$.
%\vspace{1em}
%
%\begin{center} \begin{pspicture*}(-1,-1)(8,7)
%	\def\xmin{0} \def\xmax{8} \def\ymin{0} \def\ymax{8}
%	\psset{algebraic=true,xunit=0.5cm,yunit=0.5cm}
%	\psaxes[linewidth=1pt,Dx=10,Dy=10]{->}(0,0)(-1,-1)(6,6)
%	\psline[linewidth=1.5pt]{->}(0,0)(0,1)\psline[linewidth=1.5pt]{->}(0,0)(1,0)
%	\uput[d](0.5,0){$\overrightarrow{i}$}\uput[l](0,0.5){$\overrightarrow{j}$}
%	%\psgrid[gridlabels=0pt,gridwidth=.3pt, gridcolor=gray, subgridwidth=.3pt, subgridcolor=gray, subgriddiv=1](0,0)(0,0)(8,8)
%	\uput[dl](0,0){O}\uput[ul](2,4){$M_1(z_1)$}\uput[r](3,2){$M_2(z_2)$}\uput[r](5,6){$S(z_1+z_2)$}
%	\psline[linewidth=1.5pt]{->}(0,0)(2,4)\psline[linewidth=1.5pt]{->}(0,0)(3,2)\psline[linewidth=1.5pt]{->}(0,0)(5,6)\psline[linewidth=1.5pt,linestyle=dashed]{->}(3,2)(5,6)\psline[linewidth=1.5pt,linestyle=dashed]{->}(2,4)(5,6)
%   \end{pspicture*}
%\end{center}
\begin{center}
 
\includegraphics[scale=0.6]{images/fig5.png}
\end{center}
%
Dans le triangle $OM_1S$, on a $OS\leq OM_1+M_1S$ et $M_1S=OM_2$, on en déduit $$OS\leq OM_1+OM_2$$
d'o\'u $$|z_1+z_2|\leq|z_1|+|z_2|$$
Cette inégalité est appelée \textbf{inégalité triangulaire.}\\
%\end{Prec}
%
%\begin{thm}
\begin{Prop}{\textbf{Différence et argument}}
    \vspace{1em}

Si $z_2-z_1$ a pour image $D$ avec $\overrightarrow{OD}=\overrightarrow{OM_2}-\overrightarrow{OM_1}=\overrightarrow{M_1M_2}$, alors
$$M_1M_2=||\overrightarrow{M_1M_2}||=|z_2-z_1|\quad et\quad \text{l'angle }\left( \vec{i},\overrightarrow{M_1M_2}\right)=\arg(z_2-z_1)\, [2\pi].$$
\end{Prop}
%\end{thm}
%
%\begin{Prec}\begin{center}
% \begin{pspicture*}(-1,-3)(8,7)
%	\def\xmin{0} \def\xmax{8} \def\ymin{-2} \def\ymax{8}
%	\psset{algebraic=true,xunit=0.5cm,yunit=0.5cm}
%	\psaxes[linewidth=1pt,Dx=10,Dy=10]{->}(0,0)(-1,-2)(6,6)
%	\psline[linewidth=1.5pt]{->}(0,0)(0,1)\psline[linewidth=1.5pt]{->}(0,0)(1,0)
%	\uput[d](0.5,0){$\overrightarrow{i}$}\uput[l](0,0.5){$\overrightarrow{j}$}
%	%\psgrid[gridlabels=0pt,gridwidth=.3pt, gridcolor=gray, subgridwidth=.3pt, subgridcolor=gray, subgriddiv=1](0,0)(0,0)(8,8)
%	\uput[dl](0,0){O}\uput[ul](2,4){$M_1(z_1)$}\uput[r](3,2){$M_2(z_2)$}\uput[r](1,-2){$D(z_2-z_1)$}
%	\psline[linewidth=1.5pt]{->}(0,0)(2,4)\psline[linewidth=1.5pt]{->}(0,0)(3,2)\psline[linewidth=1.5pt]{->}(0,0)(1,-2)\psline[linewidth=1.5pt,linestyle=dashed]{->}(3,2)(1,-2)\psline[linewidth=1.5pt,linestyle=dashed]{->}(2,4)(3,2)
%   \end{pspicture*}
%\end{center}
%\end{Prec}
%\vspace{1em}


\exo[2]{Distance et angle}

Dans le plan complexe, on donne 
\begin{itemize}
    \item le point $A$ d'affixe $z_a=-3+i$
    \item le point $B$ d'affixe $z_b=-2+2i$
\end{itemize}

 Calculer la distance $AB$ et une mesure de l'angle $\left( \vec{i},\overrightarrow{AB}\right)$.

%\sol{
%\vspace{2cm}

%{\textcolor{blue}{\hrulefill{} \ding{110}}}

\newpage

\subsection{Produit et quotient}

%\begin{thm}
\begin{Prop}\textbf{Produit, module et argument}
    \vspace{1em}

$$|z_1\times z_2|=|z_1|\times|z_2| \quad\text{ et }\quad\arg(z_1\times z_2)=\arg(z_1)+ \arg(z_2)\, [2\pi]$$
\end{Prop}

% \vspace{1em}
% \hrule
% \vspace{1em}
%\end{thm}
%
% \begin{Dem} 
% \` A partir des formes trigonométriques.

% Notons les formes trigonométriques de $z_1=\rho_1(\cos\theta_1+i\sin \theta_1)$ et $z_2=\rho_2(\cos\theta_2+i\sin \theta_2)$. Alors,
% $$\begin{array}{rl}
% z_1z_2&=\rho_1(\cos\theta_1+i\sin \theta_1)\rho_2(\cos\theta_2+i\sin \theta_2)\\
% &=\rho_1\rho_2(\cos\theta_1+i\sin \theta_1)(\cos\theta_2+i\sin \theta_2)\\
% &=\rho_1\rho_2(\cos\theta_1\cos\theta_2+i\cos\theta_1\sin \theta_2+i\sin\theta_1\cos\theta_2-\sin\theta_1\sin \theta_2)\\
% &=\rho_1\rho_2\bigg(\cos\theta_1\cos\theta_2-\sin\theta_1\sin \theta_2+i(\cos\theta_1\sin \theta_2+\sin\theta_1\cos\theta_2)\bigg)\\
% &=\rho_1\rho_2\bigg(\cos(\theta_1+\theta_2)+i(\sin(\theta_1+\theta_2)\bigg)
% \end{array}$$

% Finalement, $|z_1z_2|=\rho_1\rho_2$ et $\arg(z_1z_2)=\theta_1+\theta_2$.

% \textcolor{red}{\hrulefill{} \ding{110}}
% \end{Dem}
%
%\begin{thm}
\begin{Prop}\textbf{{~Inverse, module et argument}}
    \vspace{1em}

 Si $z\neq 0$, alors $$\left|\dfrac{1}{z}\right|=\dfrac{1}{|z|}\text{ et }\arg\left( \dfrac{1}{z}\right) =-\arg(z) [2\pi]$$
\end{Prop}
%\end{thm}
%
% \begin{Dem} En effet, si $z\neq0$ :

% \medskip
% $\left|1\right|=1=\left|z\right|\times\left|\dfrac{1}{z}\right|\Longrightarrow \left|\dfrac{1}{z}\right|=\dfrac{1}{|z|}$ et

% $\arg(1)=0=\arg\left(z\times\dfrac{1}{z}\right)=\arg(z)+\arg\left(\dfrac{1}{z}\right)\Longrightarrow \arg\left(\dfrac{1}{z}\right)=-\arg(z)$.

% \textcolor{red}{\hrulefill{} \ding{110}}
% \end{Dem}

%\begin{thm}
\begin{bclogo}[couleur = yellow!30, arrondi = 0.1, ombre = true, couleurOmbre=black!10, logo=\bccrayon]{Quotient,  module et argument}

 Si $z_2\neq 0$ alors $$\left|\dfrac{z_1}{z_2}\right|=\dfrac{|z_1|}{|z_2|}\text{ et }\quad \arg\left( \dfrac{z_1}{z_2}\right) =\arg(z_1)-\arg(z_2) \quad[2\pi]$$
\end{bclogo}
%\end{thm}
%
% \begin{Dem} %Utiliser les propriétés précédentes.
% \vspace{2cm}


% \textcolor{red}{\hrulefill{} \ding{110}}
% \end{Dem}
%\newpage
%


\exo[2]{Affixes}

Démontrer que si $z_a$, $z_b$, $z_c$ et $z_d$ sont respectivement les affixes des points $A$, $B$, $C$ et $D$
$$\left( \overrightarrow{AB},\overrightarrow{CD}\right)=\arg \left(\dfrac{z_d-z_c}{z_b-z_a}\right) \quad [2\pi]$$

Partir de $\left( \overrightarrow{AB},\overrightarrow{CD}\right)=\left( \overrightarrow{AB},\vec{i}\right)+\left( \vec{i},\overrightarrow{CD}\right)=-\left( \vec{i},\overrightarrow{AB}\right)+\left( \vec{i},\overrightarrow{CD}\right)$, puis utiliser la propriété précédente.\\
%
%\sol{
%\vspace{2cm}

%\textcolor{blue}{\hrulefill{} \ding{110}}}

\vspace{1em}
\hrule
\vspace{1em}

\exo[2]{Quatre points}

Dans le plan complexe, on donne les points $A,\, B,\, C$ et $D$, d'affixe respective 
$$z_a=-1+i,\qquad z_b=1+2i,\qquad z_c=2\qquad \text{ et }\qquad  z_d=-i$$
Calculer une mesure de chacun des angles
     $\left( \vec{i},\overrightarrow{AB}\right)$, $\left( \overrightarrow{BA},\overrightarrow{BC}\right)$ et 
$\left( \overrightarrow{AB},\overrightarrow{DC}\right)$.

Utiliser le résultat de l'exercice précédent, et les formes trigonométriques des rapports qui interviennent.

\vspace{1em}
\hrule
\vspace{1em}

%\sol{
%\vspace{2cm}
%
%\textcolor{blue}{\hrulefill{} \ding{110}}}
%
%\begin{thm}
\begin{Prop}\textbf{Puissance d'un nombre complexe, module et argument}
    \vspace{1em}

Pour tout complexe $z\neq 0$ et pour tout entier relatif $n$,
$$|z^n|=|z|^n \quad \text{ et }\quad \arg(z^n)=n\arg(z)\quad[2\pi]$$                             
\end{Prop}
%\end{thm}
% %
% \begin{Dem} Utiliser les propriétés précédentes.
% \vspace{3cm}

% \textcolor{red}{\hrulefill{} \ding{110}}
% \end{Dem}
%
%\begin{thm}
\begin{bclogo}[couleur = yellow!30, arrondi = 0.1, ombre = true, couleurOmbre=black!10, logo=\bccrayon]{~Formule de Moivre}
Pour tout nombre réel $\theta$ et pour tout entier relatif $n$,

$$\Big(\cos(\theta)+i\sin(\theta)\Big)^n=\cos (n\theta) + i \sin (n\theta)$$
\end{bclogo}
%\end{thm}

% \begin{Dem} On propose une démonstration par récurrence de la formule de Moivre.

% \begin{itemize}\item Pour $n=0$ : d'une part $(\cos\theta+i\sin\theta)^0=1$, d'autre part $\cos0\theta+i\sin0\theta=\cos 0+i\sin0=1$
% \item Supposons qu'il existe un rang $n$ tel que $$(\cos\theta+i\sin\theta)^n=\cos n\theta + i \sin n\theta$$ Alors $$\begin{array}{rl}(\cos\theta+i\sin\theta)^{n+1}&=(\cos\theta+i\sin\theta)^{n}(\cos\theta+i\sin\theta)\\&=(\cos n\theta + i \sin n\theta)(\cos\theta+i\sin\theta)\\&=\cos n\theta\cos\theta+i\cos n\theta\sin\theta + i \sin n\theta\cos\theta-\sin n\theta\sin\theta\\&=\cos n\theta\cos\theta-\sin n\theta\sin\theta+i(\cos n\theta\sin\theta + \sin n\theta\cos\theta)\\&=\cos(n\theta+\theta)+i\sin(n\theta+\theta)\\&=\cos(n+1)\theta+i\sin(n+1)\theta\end{array}$$
% La propriété est vraie au rang $n+1$
% \item Conclusion : $\forall n\in\N,  (\cos(\theta)+i\sin(\theta))^n=\cos n\theta + i \sin n\theta$\end{itemize}

% \textcolor{red}{\hrulefill{} \ding{110}}
% \end{Dem}

\vspace{1em}
\hrule
\vspace{1em}

\exo[2]{Formule de Moivre}
\begin{enumerate}
      \item \'Ecrire la formule de Moivre pour $n=2$.
      \item Développer $(\cos(\theta)+i\sin(\theta))^ 2$, et retrouver les formules de trigonométrie de $\cos(2\theta)$ et $\sin(2\theta)$.
     \end{enumerate}
     
\vspace{1em}
\hrule
\vspace{1em}
%
%\sol{
%\vspace{3cm}
%
%\textcolor{blue}{\hrulefill{} \ding{110}}}
%
%%%%%%%%%%%%%%%%%%%%%%%
%\newpage
%


\subsection{QCM}

\begin{enumerate}[label=\textbf{Q\arabic*.}]

\item La forme algébrique d'un nombre complexe \( z \) est :
\begin{multicols}{2}
\begin{itemize}
    \item[1.] \( z = a + bi \), avec \( a, b \in \mathbb{R} \)
    \item[2.] \( z = r \cos(\theta) + i r \sin(\theta) \)
    \item[3.] \( z = r\,e^{i\theta} \)
    \item[4.] \( z = |z| (\cos(\theta) + i \sin(\theta)) \)
\end{itemize}
\end{multicols}

\item Le module d'un nombre complexe \( z = a + ib \) est :
\begin{multicols}{2}
\begin{itemize}
    \item[1.] \( |z| = a^2 + b^2 \)
    \item[2.] \( |z| = \sqrt{a^2 + b^2} \)
    \item[3.] \( |z| = \sqrt{z \bar{z}} \)
    \item[4.] \( |z| = a + b \)
\end{itemize}
\end{multicols}

\item La forme trigonométrique d'un nombre complexe \( z \) est donnée par :
\begin{multicols}{2}
\begin{itemize}
    \item[1.] \( z = r(\cos\theta + i\sin\theta) \)
    \item[2.] \( z = a + ib \)
    \item[3.] \( r = |z| \), \( \theta = \arg(z) \)
    \item[4.] Elle est uniquement définie si \( r = 1 \)
\end{itemize}
\end{multicols}

\item La forme exponentielle de \( z \) s'écrit :
\begin{multicols}{2}
\begin{itemize}
    \item[1.] \( z = re^{i\theta} \)
    \item[2.] \( z = \ln(r) + i \theta \)
    \item[3.] \( z = |z|\,e^{i\arg(z)} \)
    \item[4.] \( z = e^{i(a + ib)} \)
\end{itemize}
\end{multicols}

\item Soit \( z = 2(\cos(\frac{\pi}{3}) + i\sin(\frac{\pi}{3})) \), alors :
\begin{multicols}{2}
\begin{itemize}
    \item[1.] \( z = 2e^{i\pi/3} \)
    \item[2.] \( |z| = 2 \)
    \item[3.] \( \arg(z) = \frac{\pi}{3} \)
    \item[4.] \( \text{Re}(z) = 1 \)
\end{itemize}
\end{multicols}

\item Soit \( z_1 = re^{i\theta_1} \), \( z_2 = se^{i\theta_2} \), alors :
\begin{multicols}{2}
\begin{itemize}
    \item[1.] \( z_1 z_2 = r s \,e^{i(\theta_1 + \theta_2)} \)
    \item[2.] \( \frac{z_1}{z_2} = \frac{r}{s} \,e^{i(\theta_1 - \theta_2)} \)
    \item[3.] \( \bar{z_1} = r \,e^{-i\theta_1} \)
    \item[4.] \( z_1 + z_2 = (r + s)\, e^{i(\theta_1 + \theta_2)} \)
\end{itemize}
\end{multicols}

\newpage

\item Le passage de la forme algébrique à la forme trigonométrique nécessite :
\begin{multicols}{2}
\begin{itemize}
    \item[1.] Le module de \( z \)
    \item[2.] L'argument de \( z \)
    \item[3.] Que \( z \neq 0 \)
    \item[4.] Que \( \text{Re}(z) = \text{Im}(z) \)
\end{itemize}
\end{multicols}

\item On a la formule d'Euler :
\begin{multicols}{2}
\begin{itemize}
    \item[1.] \( e^{i\theta} = \cos(\theta) + i\sin(\theta) \)
    \item[2.] \( e^{i\theta} = \tan(\theta) + i \)
    \item[3.] \( e^{i\pi} = -1 \)
    \item[4.] \( e^{2i\theta} = \cos(2\theta) + i \sin(2\theta) \)
\end{itemize}
\end{multicols}

\end{enumerate}



%%%%%%%%%%%%%%%%%%%%%%%% EQ 2 DEGRE
\section{Équation du second degré}
\subsection{Racines carrées complexes d'un nombre complexe}
%%
\begin{Def}\textbf{Racines carrées}
    \vspace{1em}

Soit $u\in\C$. On appelle racines carrées de $u$ les solutions de l'équation $$z^2=u,\quad z\in\C.$$
\end{Def}
%
\exo[2]{Racines}

Résoudre dans $\mathbb{C}$ :
 $$z^2=36\qquad z^2=-25\qquad z^2=2i$$

%\sol{
%\textcolor{blue}{\hrulefill{} \ding{110}}}
%
\begin{bclogo}[couleur = green!20, arrondi = 0.1, ombre = true, couleurOmbre=black!10, logo=\bcinfo]{Méthode dans le cas général}
    Soit $u=a+ib$ avec $a,b\in\R$, on cherche $z=x+iy$ tel que $z^2=u$.

\begin{enumerate}[itemsep=1em]%[$\bullet$]
\item $z=x+iy \Rightarrow z^2=x^2-y^2+2ixy\Rightarrow \left\{\begin{array}{l}x^2-y^2=a\\2xy=b\end{array}\right.$
\item $x^2+y^2=\vert z\vert^2=\vert z^2\vert=\vert u\vert=\sqrt{a^2+b^2}\Rightarrow\left\{\begin{array}{ll}x^2-y^2=a&(1)\\x^2+y^2=\sqrt{a^2+b^2}&(2)\\2xy=b&(3)\end{array}\right.$

\item En additionnant $(1)+(2)$, on déduit $2x^2=\sqrt{a^2+b^2}+a$ donc 

$$x=\sqrt{\dfrac{\sqrt{a^2+b^2}+a}{2}} \quad \text{ ou }\quad x=-\sqrt{\dfrac{\sqrt{a^2+b^2}+a}{2}}$$

\item si $x=\sqrt{\dfrac{\sqrt{a^2+b^2}+a}{2}}$ alors d'après $(3)$ : $$y=\dfrac{b}{\sqrt{2(\sqrt{a^2+b^2}+a)}}$$

et si $x=-\sqrt{\dfrac{\sqrt{a^2+b^2}+a}{2}}$ alors d'après $(3)$ : $$y=-\dfrac{b}{\sqrt{2(\sqrt{a^2+b^2}+a)}}$$
\item Finalement les deux solutions de l'équation $z^2=a+ib$ sont $$\sqrt{\dfrac{\sqrt{a^2+b^2}+a}{2}}+\dfrac{ib}{\sqrt{2(\sqrt{a^2+b^2}+a)}}$$
 et
$$-\sqrt{\dfrac{\sqrt{a^2+b^2}+a}{2}}-\dfrac{ib}{\sqrt{2(\sqrt{a^2+b^2}+a)}}$$
\end{enumerate}
\end{bclogo}
%
% \begin{Rem}On se gardera bien de retenir la forme générale des solutions, pour se concentrer sur le protocole \`a suivre.\end{Rem}\\
% %
\begin{Ex}On cherche $z$ tel que $z^2=-2+i2\sqrt3$.

\begin{enumerate}%[$\bullet$]
\item En posant $z=x+iy$ on obtient  $z^2=x^2-y^2+2ixy$, \newline 
donc $x$ et $y$ doivent vérifier le système  
$$
\left\{
\begin{array}{l}
x^2-y^2=-2\\
2xy=2\sqrt3
\end{array}
\right.
$$

\item On peut aussi calculer le module de $z$: $x^2+y^2=\vert z\vert^2=\vert z^2\vert=4$. \newline 
Nous obtenons donc les 3 équations :
$$
\left\{\begin{array}{ll}x^2-y^2=-2&(1)\\x^2+y^2=4&(2)\\xy=\sqrt3&(3)\end{array}\right.$$
\item En additionnant $(1)+(2)$, on déduit $2x^2=2$, c'est-à-dire $x^2=1$\newline 
donc  $x=1$ ou $x=-1$

\item D'après $(3)$, si $x=1$ alors $y=\sqrt3$\, et si $x=-1$ alors $y=-\sqrt3$
\end{enumerate}

Finalement les deux solutions de l'équation $z^2=-2+i2\sqrt3$ sont: $$1+i\sqrt3\quad \text{ et }\quad -1-i\sqrt3$$
\end{Ex}
%
%\begin{Rem}Si une racine carrée est évidente, on évite les calculs.\\ Par exemple, la racine carrée de $8+6i$ est $3+i$ ...  ha ! ha ! ha !\end{Rem}
%

\newpage

\subsection{Équation du second degré \`a coefficients complexe}
\begin{Prop}\textbf{Forme générale}
    \vspace{1em}

On appelle équation du second degré \`a coefficients complexe, une équation du type 
$$az^2+bz+c=0$$
avec $a\in\C^*,\ b\in\C,\ c\in\C.$
\end{Prop}
%
%\begin{thm}
\begin{Prop}\textbf{Forme des solutions}
    \vspace{1em}

Ce type d'équations admet toujours deux solutions complexes, qui sont :
$$z_1=\dfrac{-b-\delta}{2a}\quad\text{ et }\quad z_2=\dfrac{-b+\delta}{2a}$$
 avec $\delta^2=b^2-4ac.$
\end{Prop}
%\end{thm}
%
%\begin{Prec}
Pour déterminer $\delta$, on pose $\delta =x+iy$, puis on cherche $x,y$ tels que $\delta ^2=b^2-4ac$, en écrivant $b^2-4ac$ sous forme algébrique. On peut donc utiliser la méthode de résolution du type $z^2=u$.
%  \end{Prec}

\vspace{1em}
\hrule
\vspace{1em}

\exo[2]{Équation à coefficients réels}

Résoudre dans $\C : z^2-4z+5=0.$\\

%
%\sol{${\cal S}=\{2-i,2+i\}$
%
%\textcolor{blue}{\hrulefill{} \ding{110}}}

\vspace{1em}
\hrule
\vspace{1em}

\exo[3]{Équation à coefficients complexes}

Résoudre dans $\C : z^2-(4-i)z+5+i=0.$
%
%\sol{${\cal S}=\{1+i,3-2i\}$
%

%\textcolor{blue}{\hrulefill{} \ding{110}}
\vspace{1em}
\hrule
\vspace{1em}

\exo[3]{Racines de  $P$}

Soit $P(z)=2z^3-(1+4i)z^2+(4-6i)z-2+4i$.

\begin{enumerate}
    \item Montrer que $1/2$ est racine de $P(z)$.
    \item Factoriser $P(z)$ \textcolor{blue}{({\it \underline{rappel} : si $\alpha$ est une racine d'un polyn\^ome $P$ de degré $n$, alors $P(z)$ se factorise par $z-\alpha$, c'est-\`a-dire que $P(z)=(z-\alpha)Q(z)$ o\`u $Q$ est un polyn\^ome de degré $n-1$})}.
\item Résoudre dans $\C $ l'équation $P(z)=0$.
\end{enumerate}

\vspace{1em}
\hrule
\vspace{1em}

%\sol{${\cal S}=\{1/2,1-i,-1+3i\}$

%\textcolor{blue}{\hrulefill{} \ding{110}}}
%

\section{Racines $n^{\text{ièmes}}$ d'un nombre complexe}
\subsection{Cas général}

\begin{Def}\textbf{Racine $n^{\text{ièmes}}$}
    \vspace{1em}

Soit $u\in\C,\, n\in\N$ avec $n\geq2$. Une racine $n^{\text{ième}}$ du complexe $u$ est une solution de l'équation$$z^n=u.$$
\end{Def}

%\begin{Rem}
D\`es que $n\geq3$, la méthode algébrique est déconseillée et on pose $z=\rho e^{i\theta}$.
%\end{Rem}

\begin{Ex} Pour trouver les racines cubiques de $-8i$, il s'agit de résoudre $z^3=-8i$.

\begin{enumerate}%[$\bullet$]
\item On pose $z=\rho e^{i\theta}$, l'équation devient $$\rho^3e^{i3\theta}=8e^{-i\frac{\pi}{2}}$$

\item En identifiant module et argument, on obtient  
$$\left\{\begin{array}{l}\rho^3=8\\
3\theta=-\frac{\pi}{2}+2k\pi
\end{array}\right.\quad \text{ donc }
\left\{\begin{array}{l}\rho=2\\ \theta=-\frac{\pi}{6}+k\frac{2\pi}{3}\end{array}\right.$$

\item\underline{Solutions :}
\begin{enumerate}
    \item[] {$k=0 :$} $z_0=2e^{-i\frac{\pi}{6}}$, affixe de $M_0$
    \item[] {$k=1 :$} $z_1=2e^{i\frac{\pi}{2}}$, affixe de $M_1$
    \item[] {$k=2 :$} $z_2=2e^{i\frac{7\pi}{6}}$, affixe de $M_2$
\end{enumerate}

Le triangle $M_0M_1M_2$ est équilatéral.

%\hspace{1cm}\begin{minipage}[t]{0.5\linewidth}\begin{center}\underline{Interprétation géométrique :}\end{center}
%
%%\pstTriangle(0,2){A}(-1.73,-1){B}(1.73,-1){C}
%%\pstCircleABC{A}{B}{C}{O}
%%\psaxes[Dx=10,Dy=10,linewidth=1pt]{->}(0,0)(-3,-3)(3,3)
%
%\psset{xunit=1cm , yunit=3.5cm}
%\begin{pspicture}(3,3)(3,3)
%\CercleTriangle{1}
%\end{pspicture}\end{minipage}
\end{enumerate}
\end{Ex}

%\begin{Rem}
On peut aussi choisir comme valeurs de $k$, $-1,0,1$ : si bien que la racine $z_2$ ci-dessus sera donnée sous la forme $z_{-1}=2e^{-i\frac{5\pi}{6}}$.
%\end{Rem}
%\vspace{1em}
\begin{Prop}\textbf{Racines $n^{\text{ièmes}}$}
    \vspace{1em}

Pour tout $u\in\C^*,\, n\in\N$ avec $n\geq2$. Il existe toujours exactement $n$ solutions à équation$$z^n=u.$$
\end{Prop}

\exo[2]{Racines quatrièmes de -4 :}

Résoudre $z^4=-4$

%\sol{
% \vspace{4.5cm}

%\textcolor{blue}{\hrulefill{} \ding{110}}}

\subsection{Cas particulier : racines $n^{\text{ièmes}}$ de l'unité}

\begin{Def}\textbf{ Racine $n^{\text{ième}}$ de l'unité} 
    \vspace{1em}

Une racine $n^{\text{ième}}$ de $1$ est une solution de l'équation $$z^n=1.$$
\end{Def}

\noindent En posant $z=\rho e^{i\theta}$ : $$\left\{\begin{array}{l}\rho=1\\ \theta=\frac{2k\pi}{n}\end{array}\right..$$

\begin{Ex}
Pour trouver les racines cubiques de l'unité, il s'agit de résoudre $z^3=1$.

\begin{minipage}[t]{0.45\linewidth}\underline{Solutions :}
\vspace{1em}

{$k=0 :$} $z_0=1$
\vspace{1em}

{$k=1 :$} $z_1=e^{i\frac{2\pi}{3}}=j$
\vspace{1em}

{$k=2 :$} $z_2=e^{i\frac{4\pi}{3}}=j^2=\bar j$
\vspace{1em}

On notera que {$1+j+ j^2=0$.}\end{minipage}\hspace{1cm}
%\begin{minipage}[t]{0.5\linewidth}\begin{center}\underline{Interprétation géométrique :}\end{center}
%\psset{xunit=1cm , yunit=3.5cm}
%\begin{pspicture}(3,3)(3,3)
%\CercleTriangleDeux{1}
%\end{pspicture}\end{minipage}
\end{Ex}

\begin{Ex}\textbf{\emph{Résolution de $z^3=-8i$, par une technique se ramenant aux racines cubiques de l'unité.}}

\begin{enumerate}%[$\bullet$]
\item Une racine évidente : $2i$
\item donc on cherche $z$ tel que $z^3=(2i)^3\Longleftrightarrow\left(\dfrac{z}{2i}\right)^3=1$
\item On déduit que $\dfrac{z}{2i}=1$ ou $\dfrac{z}{2i}=j$ ou $\dfrac{z}{2i}=j^2$
\item c'est-\`a-dire $z=2i=2e^{i\frac{\pi}{2}}$ ou $z=2i j=2e^{i\frac{7\pi}{6}}$ ou $z=2ij^2=2e^{-i\frac{\pi}{6}}$
\end{enumerate}
\end{Ex}

\subsection{QCM}

\begin{enumerate}[label=\textbf{Q\arabic*.}]

\item Une équation \( az^2 + bz + c = 0 \), avec \( a, b, c \in \mathbb{C} \), a toujours :
\begin{multicols}{2}
\begin{itemize}
    \item[1.] Deux solutions dans \( \mathbb{R} \)
    \item[2.] Deux solutions dans \( \mathbb{C} \)
    \item[3.] Une seule solution si \( \Delta = 0 \)
    \item[4.] Deux solutions distinctes si \( \Delta \neq 0 \)
\end{itemize}
\end{multicols}

\item L'équation \( z^2 + 1 = 0 \) admet pour solutions :
\begin{multicols}{2}
\begin{itemize}
    \item[1.] \( i \)
    \item[2.] \( -i \)
    \item[3.] \( 1 \)
    \item[4.] \( \sqrt{2} \)
\end{itemize}
\end{multicols}

\item Le discriminant complexe \( \Delta \) de l'équation \( az^2 + bz + c = 0 \) est :
\begin{multicols}{2}
\begin{itemize}
    \item[1.] \( b^2 - 4ac \)
    \item[2.] Toujours réel
    \item[3.] Parfois complexe
    \item[4.] Nécessaire pour appliquer la formule des racines
\end{itemize}
\end{multicols}

\item Soit \( z \in \mathbb{C} \), alors les solutions de \( z^n = 1 \) sont :
\begin{multicols}{2}
\begin{itemize}
    \item[1.] Les racines \( n \)-ièmes de l'unité
    \item[2.] Réparties régulièrement sur le cercle de rayon 1
    \item[3.] Données par \( e^{2ik\pi/n} \), \( k = 0, 1, \ldots, n-1 \)
    \item[4.] Toutes réelles si \( n \) est pair
\end{itemize}
\end{multicols}

\item L'équation \( z^3 = 8 \) admet comme solutions :
\begin{multicols}{2}
\begin{itemize}
    \item[1.] \( 2 \)
    \item[2.] \( 2e^{2i\pi/3} \)
    \item[3.] \( 2e^{-2i\pi/3} \)
    \item[4.] Une infinité de solutions
\end{itemize}
\end{multicols}

\item Pour \( z_0 \in \mathbb{C} \), \( z_0 \neq 0 \), les solutions de \( z^n = z_0 \) sont données par :
\begin{multicols}{2}
\begin{itemize}
    \item[1.] \( \sqrt[n]{|z_0|} e^{i(\arg(z_0) + 2k\pi)/n} \)
    \item[2.] Une infinité de racines dans \( \mathbb{C} \)
    \item[3.] \( n \) racines distinctes si \( z_0 \neq 0 \)
    \item[4.] Une seule racine si \( z_0 \in \mathbb{R}_+^* \)
\end{itemize}
\end{multicols}

\item L'ensemble des racines \( n \)-ièmes de l'unité forme :
\begin{multicols}{2}
\begin{itemize}
    \item[1.] Un groupe multiplicatif
    \item[2.] Un sous-ensemble de \( \mathbb{R} \)
    \item[3.] Un polygone régulier à \( n \) sommets
    \item[4.] Une droite dans le plan complexe
\end{itemize}
\end{multicols}

\item Une équation complexe du second degré peut avoir :
\begin{multicols}{2}
\begin{itemize}
    \item[1.] Deux racines égales
    \item[2.] Une racine réelle et une imaginaire
    \item[3.] Deux racines imaginaires pures
    \item[4.] Des racines conjuguées
\end{itemize}
\end{multicols}

\end{enumerate}



\section{Retour sur les ensembles $\C[X]$ et $\R[X]$}
\subsection{Factorisation dans $\C[X]$ (polyn\^omes \`a coefficients complexes)}


\begin{Thm}\textbf{Théorème de d'Alembert}\
    \vspace{1em}

Tout polyn\^ome de $\C[X]$ peut s'écrire comme produit de polyn\^omes de degré 1 :$$\forall P\in\C[X]\quad \exists a\in\C\textrm{ et }(\alpha_i,m_i)_{i\in I}\in\C\times\N\quad\text{tels que}\quad P(X)=\ds a\prod_i (X-\alpha_i)^{m_i}$$
\end{Thm}

%
%\begin{Dem}
%Pas facile, facile...

%\textcolor{red}{\hrulefill{} \ding{110}}
%\end{Dem}

\'Etant donné un polyn\^ome $P$, on cherche ses racines.

%\begin{Exs}
\begin{enumerate}%[$\bullet$]
\item $P(X)=X^3-1$ : après la recherche des racines cubique de 1: 
$$P(X)=(X-1)(X-j)(X-\bar j)$$
\item $P(X)=X^6+1$ : cherchons les racines sous la forme $X=\rho e^{i\theta}$, l'équation donne alors 
$$\rho^6e^{i6\theta}=e^{i\pi}\quad \text{ donc } \quad 
\left\{\begin{array}{l}\rho^6=1\\ 6\theta=\pi+2k\pi\end{array}\right.$$
On touve finalement
$$\left\{\begin{array}{l}\rho=1\\ \theta=\frac{\pi}{6}+\frac{k\pi}{3}\end{array}\right.$$

Les 6 racines sont $\{e^{i\pi/6},\,e^{i\pi/2},\,e^{i5\pi/6},\,e^{i7\pi/6},\,e^{i3\pi/2},\,e^{i11\pi/6}\}$.

%\begin{center}
%\psset{xunit=1cm , yunit=4cm}
%\begin{pspicture}(-2,0.5)(2,2.5)
%\CercleHexagone{1}
%\end{pspicture}
%\end{center}
Et finalement 
$$P(X)=(X-e^{-i\pi/6})(X-e^{i\pi/6})(X-e^{i\pi/2})(X-e^{i5\pi/6})(X-e^{-i5\pi/6})(X-e^{-i\pi/2}).$$

\item $P(X)=X^6-2X^3+1$ : on remarque que 
$$P(X)=(X^3-1)^2=(X-1)^2(X-j)^2(X-\bar j)^2.$$
\end{enumerate}
%\end{Exs}

\subsection{Factorisation dans $\R[X]$ (polyn\^omes \`a coefficients réels)}

%\begin{thm}
\begin{Prop}\textbf{Racines sur $\R[X]$}
    \vspace{1em}

    \'Etant donné un polyn\^ome $P$ de $\R[X]$. Les racines de $P$ sont soit réelles, soit complexes conjuguées.
\end{Prop}
%\end{thm}

%\begin{Dem} 
On consid\` ere le polyn\^ome $P$ de $\R[X]$, $P(X)=a_nX^n+a_{n-1}X^{n-1}+\dots+a_0$.

$$\begin{array}{rll}\alpha\text{ racine de }P&\Longleftrightarrow&a_n\alpha^n+a_{n-1}\alpha^{n-1}+...+a_0=0\\
&&\\
&\Longleftrightarrow&\overline{a_n\alpha^n+a_{n-1}\alpha^{n-1}+...+a_0}=\bar0\\
&&\\
&\Longleftrightarrow&\overline{a_n\alpha^n}+\overline{a_{n-1}\alpha^{n-1}}+...+\overline{a_0}=0\\
&&\\
&\Longleftrightarrow&\overline{a_n}\overline{\alpha^n}+\overline{a_{n-1}}\overline{\alpha^{n-1}}+...+\overline{a_0}=0\\
&&\\
&\Longleftrightarrow&a_n\overline{\alpha^n}+a_{n-1}\overline{\alpha^{n-1}}+...+a_0=0\\
&&\\
&\Longleftrightarrow&a_n\overline{\alpha}^n+a_{n-1}\overline{\alpha}^{n-1}+...+a_0=0\\
&&\\
&\Longleftrightarrow&P(\overline\alpha)=0\\
&&\\
&\Longleftrightarrow&\overline\alpha\textrm{ racine de }P
\end{array}$$
%\textcolor{red}{\hrulefill{} \ding{110}}
%\end{Dem}
%
%\begin{Ex}$P(X)=X^2+\sqrt{3}X+2$\end{Ex}
%
%\begin{thm}
\begin{Prop}\textbf{Racines complexes sur $\C[X]$}
    \vspace{1em}

    \'Etant donné $\alpha\in\C$. Le produit $(X-\alpha)(X-\overline\alpha)$ est \`a coefficients réels.\\
\end{Prop}
%\end{thm}
%
%\begin{Dem} 
$$(X-\alpha)(X-\overline\alpha)=X^2-(\alpha+\overline\alpha)X+\alpha\overline\alpha=X^2-2Re(\alpha)X+\abs{\alpha}^2.$$

%\textcolor{red}{\hrulefill{} \ding{110}}
%\end{Dem}
%
%\begin{thm}
\begin{Prop}\textbf{Factorisation dans $\R[X]$}
    \vspace{1em}

Tout polyn\^ome de $\R[X]$ peut s'écrire comme produit de polyn\^omes de degré 1 et de polyn\^omes de degré 2 \`a discriminant strictement négatif (ou encore, aux racines complexes non réelles conjuguées).
\end{Prop}
%\end{thm}

% \begin{Dem} D'aprés le théoréme de d'Alembert, un polyn\^ome $P$ de $\C[X]$ peut s'écrire :$$P(X)=\ds\prod_i (X-a_i)^{m_i}$$
% o\`u $m_i$ est la multiplicité de $a_i$.

% Soit $a_i$ est une racine complexe de $P$ de multiplicité $m_i$, $\overline{a_i}$ est une racine complexe de $P$. Si $m_i\geq2$, $a_i$ est une racine de $P'$ ; et ainsi de suite jusqu'\`a $P^{(m_i-1)}$.

% Réciproquement, comme $P^{(m_i)}(a_i)\neq 0$, on a $P^{(m_i)}(\overline{a_i})$ ; on en déduit que $\overline{a_i}$ est une racine complexe de $P$ de multiplicité $m_i$.

% Si $a_i\in\R$ alors $\overline{a_i}=a_i\in\R$ ; par contre si $a_i\notin\R$ alors il existe $j\neq i$ tel que $a_j=\overline{a_i}$ et $m_j=m_i$. On a alors $$(X-a_i)^{m_i}(X-\overline{a_i})^{m_i}=(X^2-2Re(a_i)X+\abs{a_i}^2)^{m_i}$$
% et $X^2-2Re(a_i)X+\abs{a_i}^2$ est un polyn\^ome réel de degré 2 \`a discriminant négatif.\\

%\textcolor{red}{\hrulefill{} \ding{110}}
%\end{Dem}

%\begin{Exs}
\begin{enumerate}%[$\bullet$]
\item $P(X)=X^3-1=(X-1)(X-j)(X-\bar j)=(X-1)(X^2+X+1)$\\
\vspace{1em}

\item $P(X)=X^6+1\\\begin{array}{l}=(X-e^{-i\pi/6})(X-e^{i\pi/6})(X-e^{i\pi/2})(X-e^{-i\pi/2})(X-e^{i5\pi/6})(X-e^{-i5\pi/6})\\=(X^2-\sqrt3X+1)(X^2+1)(X^2+\sqrt3X+1)\end{array}$\end{enumerate}
%\end{Exs}

\begin{bclogo}[couleur = green!20, arrondi = 0.1, ombre = true, couleurOmbre=black!10, logo=\bcinfo]{~Méthode pour factoriser dans $\R[X]$}

 On peut, si nécessaire, rechercher les racines complexes du polyn\^ome $P$ de $\R[X]$,\\
  puis développer les produits du type $(X-a_i)(X-\overline{a_i})$ lorsque $a_i\in\C-\R$ pour retrouver les polyn\^omes de degré 2 irréductibles dans $\R[X]$ 
  et établir la factorisation de $P$ dans $\R[X]$.
\end{bclogo}

\exo[3]{Polynôme}

On consid\`ere le polyn\^ome $P(X)=X^5-1$.
\begin{enumerate}
\item Déterminer les racines complexes de $P$.
\item En déduire une factorisation de $P$ dans $\C[X]$, puis dans $\R[X]$.
\end{enumerate}

\subsection{Division de polyn\^omes}
Dans le paragraphe suivant, on propose de prendre connaissance d'une technique exposée dans le cours sur les polyn\^omes : la division euclidienne des polyn\^omes.
%\sect{La division euclidienne de polyn\^omes}
%
%\{
Si $A$ et $B$ sont deux polyn\^omes de $\C[X]$, avec $B$ non nul, il existe un unique couple $(Q, R)$ de polyn\^ omes de $\C[X]$ tel que : $$A = BQ + R \quad\text{avec}\quad deg(R) < deg(B)$$
%{\bf
\textbf{Divisibilité :} si $R=0$, c'est-\`a-dire que $A=BQ$ on dit que \og A est divisible par B\fg\, ou que \og B divise A\fg.
%}}
%%
%\exe{
%%
\vspace{1em}

\noindent\begin{minipage}[htb]{8cm}
\underline{(a) Division de $x^4-3x^3+x+1$ par $x^2+2$} :

\begin{equation*}
\renewcommand{\arraystretch}{1.2}
\renewcommand{\arraycolsep}{2pt}
  \begin{array}{rrrrr|rrr}
 x^4&-3x^3 &   &+x&+1  &x^2 &+2&  \\
\cline{6-7}
-x^4&&-2x^2 &   &  &x^2&-3x&-2\\
\cline{1-3}
    &-3x^3 &  &  & &  &   &  \\
    &3x^3&&+6x&  &   &   &  \\
    \cline{2-4}
    &     &-2x^2 &  & &  &   &  \\
    &     &+2x^2&&+4&   &   &  \\
              \cline{3-5}
    &     &   &7x&+5&   &   &  \\ 
  \end{array}
\end{equation*}
Finalement,

 $x^4-3x^3+x+1$
 
 $=(x^2+2)(x^2-3x-2)+7x+5$
\end{minipage}\hspace{1cm}\vline\hspace{0.5cm}
\begin{minipage}[htb]{5cm}
\underline{(b) Division de $x^3+x^2-1$ par $x-1$} :

\begin{equation*}
\renewcommand{\arraystretch}{1.2}
\renewcommand{\arraycolsep}{2pt}
  \begin{array}{rrrr|rrr}
 x^3&+x^2 &   &-1&x  &-1 &  \\
\cline{5-7}
-x^3&+x^2 &   &  &x^2&+2x&+2\\
\cline{1-2}
    &2x^2 &   &  &   &   &  \\
    &-2x^2&+2x&  &   &   &  \\
    \cline{2-3}
    &     &2x &  &   &   &  \\
    &     &-2x&+2&   &   &  \\
              \cline{3-4}
    &     &   &+1&   &   &  \\ 
  \end{array}
\end{equation*}
Finalement,

$x^3+x^2-1$

$=(x-1)(x^2+2x+2)+1$
\end{minipage}

\vspace{1em}
\hrule\hrule
\vspace{1em}

\noindent\begin{minipage}[htb]{7cm}
\underline{(c) Division de $x^3-x^2+x-1$ par $x-1$} :

\begin{equation*}
\renewcommand{\arraystretch}{1.2}
\renewcommand{\arraycolsep}{2pt}
  \begin{array}{rrrr|rr}
 x^3&-x^2 &+x&-1  &x &-1  \\
\cline{5-6}
-x^3&+x^2 &   &  &x^2&+1\\
\cline{1-3}
    & & x &  & &       \\
    &&-x&+1&  &      \\
    \cline{2-4}
    &     && 0 & &       \\
  \end{array}
\end{equation*}
donc,

$x^3-x^2+x-1=(x-1)(x^2+1)$
\end{minipage}\hspace{2.5cm}\vline\hspace{0.5cm}
\begin{minipage}[htb]{7cm}
il suit qu'en posant la division de

$x^2+1$ par $x-i$ on obtient :

\begin{equation*}
%\renewcommand{\arraystretch}{1.2}
%\renewcommand{\arraycolsep}{2pt}
  \begin{array}{rrr|rr}
x^2 &   &+1&x  &-i   \\
\cline{4-5}
-x^2 & +ix  &  &x&+i\\
\cline{1-2}
    &ix &   &   &    \\
    &-ix&-1&   &    \\
    \cline{2-3}
    &   &  0&   &  \\
  \end{array}
\end{equation*}
donc,

$x^2+1=(x-i)(x+i)$

Finalement, on obtient la forme factorisée :

$x^3-x^2+x-1$

$=(x-1)(x-i)(x+i)$
\end{minipage}
%}
%%
%%
%\newpage
\section{Exercices}
\vspace{1em}
\hrule
\vspace{1em}

\exo[1]{Forme algébrique - Somme et produits}


Mettre sous forme algébrique les nombres complexes suivants :
$$\begin{array}{lll}
\mathbf{1.}\ z_1=(2+5i)+(i+3)&\quad \mathbf{2.}\ z_2=4(-2+3i)+3(-5-8i)&\quad\mathbf{3.}\ z_3=(2-i)(3+8i)\\
\displaystyle\mathbf{4.}\ z_4=(1-i)\overline{(1+i)}&\quad\mathbf{5.}\ z_5=i(1-3i)^2& \quad\mathbf{6.}\ z_6=(1+i)^3
\end{array}
$$
Attention! Il y a un symbole de conjugaison dans $z_4$.


% Exercice 2902

\vspace{1em}
\hrule
\vspace{1em}

\newpage
\exo[1]{Forme algébrique - Quotients}

Mettre sous forme algébrique les nombres complexes suivants :
$$\begin{array}{lll}
\mathbf{1.}\ z_1=\frac1{1+i}&\quad{\mathbf 2.}\ z_2=\frac{-4}{1+i\sqrt 3}&
\quad\mathbf{3.}\ z_3=\frac{1-2i}{3+i}\\
\displaystyle{\mathbf 4.}\ z_4=\frac{(3+5i)^2}{1-2i}&\displaystyle\quad{\mathbf 5.}\ z_5=\left(\frac{1+i}{2-i}\right)^2+\frac{3+6i}{3-4i}\\
\end{array}
$$


% Exercice 2411
\vspace{1em}
\hrule
\vspace{1em}

\exo[2]{Systèmes}

Résoudre les syst\` emes suivants, d'inconnues les nombres complexes $z_1$ et $z_2$ :
$$\left\{
 \begin{array}{rcl}
  2z_1-z_2&=&i\\
  -2z_1+3iz_2&=&-17
 \end{array}\right.\qquad\qquad \left\{
 \begin{array}{rcl}
  3iz_1+iz_2&=&i+7\\
  iz_1+2z_2&=&11i
 \end{array}\right.$$

On donnera les résultats sous forme algébrique.\\


% Exercice 2
\vspace{1em}
\hrule
\vspace{1em}
\exo[1]{Forme exponentielle}

Mettre sous forme exponentielle les nombres complexes suivants :
$$\begin{array}{lll}
{\mathbf 1.}\ z_1=1+i\sqrt 3&\quad\mathbf 2.\ z_2=9i&\quad{\mathbf 3.}\ z_3=-3\\
\displaystyle{\mathbf 4.}\ z_4=\frac{-i\sqrt 2}{1+i}&\displaystyle \quad\mathbf{5.}\ z_5=\frac{(1+i\sqrt 3)^3}{(1-i)^5}&\quad{\mathbf 6.}\ z_6=\sin x+i\cos x.\\
\end{array}
$$


% Exercice 2930
\vspace{1em}
\hrule
\vspace{1em}

\exo[1]{Forme exponentielle, encore}

 On pose
  $z_1=4e^{i\frac{\pi}{4}},\;z_2=3ie^{i\frac{\pi}{6}},\;z_3=-2e^{i\frac{2\pi}{3}}$.
  
  \'Ecrire sous forme exponentielle les nombres complexes : $z_1$, $z_2$, $z_3$, $z_1z_2$,
  $\frac{z_1z_2}{z_3}$.
% Exercice 3
\vspace{1em}
\hrule
\vspace{1em}

\exo[2]{Les deux \`a la fois - avec application}

On consid\` ere les nombres complexes suivants :
$$z_1=1+i\sqrt 3,\qquad z_2=1+i\quad \textrm{ et }\quad z_3=\frac{z_1}{z_2}.$$
\begin{enumerate}
\item \'Ecrire $z_3$ sous forme algébrique.
\item \'Ecrire $z_3$ sous forme trigonométrique.
\item En déduire les valeurs exactes de $\cos\frac\pi{12}$ et $\sin\frac\pi{12}$.
\end{enumerate}


% Exercice 4
\vspace{1em}
\hrule
\vspace{1em}

\newpage

\exo[2]{Forme algébrique, le retour}

Déterminer la forme algébrique des nombres complexes suivants :
$$z_1=(2+2i)^6\qquad z_2=\left(\frac{1+i\sqrt 3}{1-i}\right)^{20}\qquad  z_3=\frac{(1+i)^{2000}}{(i-\sqrt 3)^{1000}}.$$


% Exercice 5

\vspace{1em}
\hrule
\vspace{1em}
\exo[2]{Réel positif}

Trouver les entiers $n\in\mathbb N$ tels que $(1+i\sqrt 3)^n$ soit un réel positif.\\


\vspace{1em}
\hrule
\vspace{1em}

\exo[2]{Forme exponentielle et formule d'Euler}

Soient $a,b\in]0,\pi[$. écrire sous forme exponentielle les nombres complexes suivants :
$$z_1=1+e^{ia}\qquad z_2=1-e^{ia}\qquad  z_3=e^{ia}+e^{ib}\qquad z_4=\frac{1+e^{ia}}{1+e^{ib}}.$$


\vspace{1em}
\hrule
\vspace{1em}
\exo[2]{Le m\^eme module}

Déterminer les nombres complexes non nuls $z$ tels que $z$, $\frac 1z$ et $1-z$ aient le m\^eme module.



\vspace{1em}
\hrule
\vspace{1em}
\exo[3]{Homographie}

Soit $z$ un nombre complexe, $z\neq 1$. Démontrer que :
$$|z|=1\iff \frac{1+z}{1-z}\in i\mathbb R.$$


\vspace{1em}
\hrule
\vspace{1em}
\exo[2]{Exponentielle}

Résoudre l'équation $e^z=3\sqrt 3-3i$.

\vspace{1em}
\hrule
\vspace{1em}

\exo[1]{Équations du premier degré}

Résoudre les équations suivantes, d'inconnue $z\in\mathbb C$ :
$$z+2i=iz-1\qquad\qquad(3+2i)(z-1)=i$$
$$(2-i)z+1=(3+2i)z-i\qquad\qquad(4-2i)z^2=(1+5i)z.$$

On écrira les solutions sous forme algébrique.


% Exercice 2517
\vspace{1em}
\hrule
\vspace{1em}

\exo[2]{Équations avec des conjugués}

Résoudre les équations suivantes :
$$2z+i=\overline z+1\qquad 2z+\overline z=2+3i\qquad  2z+2\overline z=2+3i.
$$

\vspace{1em}
\hrule
\vspace{1em}

% Exercice 15

\exo[1]{Racine carrée d'un nombre complexe}

Calculer les racines carrées des nombres complexes suivants :
$z_1=3+4i,\ z_2=8-6i.$

\vspace{1em}
\hrule
\vspace{1em}
% Exercice 16


\exo[2]{Racine carré de deux fa\c cons}

Déterminer les racines carrées de $Z=\sqrt 3+i$ sous forme algébrique, puis sous forme trigonométrique.\\
En déduire la valeur de $\cos\pi/12$.

\vspace{1em}
\hrule
\vspace{1em}

% Exercice 2971

\exo[2]{Racine carrée puis équation du second degré}

\begin{enumerate}
\item Calculer les racines carrées du nombre complexe $1+2\sqrt 2 i$ sous forme algébrique.
\item Résoudre dans $\mathbb C$ l'équation $z^2+iz-\frac 12-i\frac{\sqrt 2}2=0$.
\end{enumerate}

\vspace{1em}
\hrule
\vspace{1em}

% Exercice 18


\exo[2]{Racines $n$-i\` emes}

Résoudre les équations suivantes :
$$z^3=1+i\sqrt 3\qquad z^6=\frac{-4}{1+i\sqrt 3}
\qquad z^5=\frac{(1+i\sqrt 3)^4}{(1+i)^2}.$$

\vspace{1em}
\hrule
\vspace{1em}
% Exercice 19


\exo[3]{Presque des racines de l'unité}

Résoudre les équations suivantes : 
$$(z-1)^5=(z+1)^5\qquad  \left(\frac{z+1}{z-1}\right)^3+\left(\frac{z-1}{z+1}\right)^3=0
\qquad (z+i)^n=(z-i)^n.$$


% Exercice 21
\vspace{1em}
\hrule
\vspace{1em}

\newpage

\exo[3]{Degré plus grand}

On cherche \`a résoudre l'équation
$$z^3+(1+i)z^2+(i-1)z-i=0.$$
\begin{enumerate}
\item Rechercher une solution imaginaire pure $ai$ \` a l'équation.\\
\item Déterminer $b,c\in\mathbb R$ tels que 
$$z^3+(1+i)z^2+(i-1)z-i=(z-ai)(z^2+bz+c).$$
\item En déduire toutes les solutions de l'équation.\\
\item Sur le m\^eme mod\` ele, résoudre l'équation $z^3-(2+i)z^2+2(1+i)z-2i=0$.\\
\end{enumerate}

\vspace{1em}
\hrule
\vspace{1em}
% Exercice 26

\exo[3]{Lieu géométrique et arguments}

Le plan est rapporté \` a un rep\` ere orthonormé direct $(O,\vec u,\vec v)$. Déterminer l'ensemble des points $M$ dont l'affixe $z$ vérifie la relation demandée :
\begin{multicols}{3}
\begin{enumerate}[label=\alph*.]
\item $\arg(z-2)=\frac{\pi}2\ [2\pi]$
\item $\arg(z-2)=\frac{\pi}2\ [\pi]$
\item $\arg(iz)=\frac{\pi}{4}\ [\pi]$
\item $\arg\left(\frac{z}{1+i}\right)=\frac{\pi}2\ [2\pi]$
\item $\arg\left(\frac{z-2i}{z-1+i}\right)=\frac{\pi}2\ [\pi]$
\end{enumerate}
\end{multicols}

\vspace{1em}
\hrule
\vspace{1em}
% Exercice 24

\exo[3]{Lieux géométriques et module}

Déterminer le lieu géométrique des points $M$ dont l'affixe $z$ vérifie
$$
\begin{array}{ll}
\mathbf{1.}\ |z-i|=|z+i|&
\mathbf{2.}\ \displaystyle \frac{|z-3+i|}{|z+5-2i|}=1\\
\mathbf{3.}\ |(1+i)z-2i|=2&
\mathbf{4.}\ \displaystyle \ |3+iz|=|3-iz|
\end{array}$$

\vspace{1em}
\hrule
\vspace{1em}
% Exercice 27


\exo[3]{écriture complexe de transformations}
Déterminer la nature et les éléments caractéristiques des transformations géométriques données par l'écriture complexe suivante :
$$\begin{array}{ll}
\mathbf 1.\ z\mapsto \frac 1iz&\mathbf 2.\ z\mapsto z+(2+i)\\
\mathbf 3.\ z\mapsto (1+i\sqrt 3)z+\sqrt 3(1-i)&\mathbf 4.\ z\mapsto (1+i\tan\alpha )z-i\tan\alpha,\ \alpha\in [0,\pi/2[.\\
\end{array}$$


% Exercice 28

\vspace{1em}
\hrule
\vspace{1em}

\newpage

\exo[3]{Une coquille d'escargot}

Dans le plan muni d'un rep\` ere orthonormal $(O,\vec i,\vec j)$, on note $A_0$ le point d'affixe 6 et $S$ la similitude de centre $O$, de rapport $\frac{\sqrt 3}2$ et d'angle $\frac\pi 6$. On pose $A_{n+1}=S(A_n)$ pour $n\geq 1$.\\
\begin{enumerate}
\item Déterminer, en fonction de $n$, l'affixe du point $A_n$. En déduire que $A_{12}$ est sur la demi-droite $(O,\vec i)$.\\
\item Établir que le triangle $OA_nA_{n+1}$ est rectangle en $A_{n+1}$.\\
\item Calculer la longueur du segment $[A_0A_1]$. En déduire la longueur $\ell$ de la ligne polygonale
$A_0A_1A_2\dots A_{12}.$\\
\end{enumerate}


% Exercice 29
\vspace{1em}
\hrule
\vspace{1em}

\exo[2]{Lieux géométriques}

Pour chacune des conditions suivantes, déterminer et représenter le lieu géométrique des points $M$ d'affixe $z$ qui vérifient la condition.\\
\begin{enumerate}
\item $|z - 2 + i| = 3$;\\
\item $|z - 1| = |z + 3|$;\\
\item $\Re e(z + 2i) = 1$;\\
\item $|z - i| \leq 2$ et $\Im m(z) \geq 0$.\\
\end{enumerate}


% Exercice 3086
\vspace{1em}
\hrule
\vspace{1em}

\exo[2]{Linéariser!}

\begin{enumerate}
\item Établir la formule de trigonométrie 
$\cos^4(\theta)=\cos(4\theta)/8+\cos(2\theta)/2+3/8$.\\
\item Fournir une relation analogue pour $\sin^4(\theta)$. \\
\end{enumerate}

\vspace{1em}
\hrule
\vspace{1em}

% Exercice 57

\exo[2]{Linéariser à nouveau}

Linéariser $\cos^5 x$, $\sin^5 x$ et $\cos^2 x\sin^3 x$.\\

\vspace{1em}
\hrule
\vspace{1em}
% Exercice 3087

\newpage

\exo[2]{Addition}\

\begin{enumerate}
\item Démontrer la formule de trigonométrie 
$\cos(4\theta)=\cos^4(\theta)-6\cos^2(\theta)\sin^2(\theta)+\sin^4(\theta)$. \\
\item Fournir une relation analogue pour $\sin(4\theta)$.\\
\end{enumerate}

\vspace{1em}
\hrule
\vspace{1em}
% Exercice 58


\exo[2]{Addition à nouveau}

Exprimer $\cos(5x)$ et $\sin(5x)$ en fonction de $\cos x$ et $\sin x$.\\


% Exercice 59
\vspace{1em}
\hrule
\vspace{1em}

\exo[3]{Un calcul d'intégrale}

Calculer $\int_0^{\pi/2}\cos^4t\sin^2tdt$.\\


% Exercice 60

\vspace{1em}
\hrule
\vspace{1em}

\exo[3]{Sommes trigonométriques}

Soit $n\in\mathbb N^*$ et $x,y\in\mathbb R$. Calculer les sommes suivantes :
\begin{enumerate}
\item $\displaystyle \sum_{k=0}^n \binom{n}{k}\cos(x+ky)$;\\
\item $\displaystyle S=\sum_{k=0}^n \frac{\cos(kx)}{(\cos x)^k}\textrm{ et }T=\sum_{k=0}^n \frac{\sin(kx)}{(\cos x)^k},$\\
avec $x\neq\frac{\pi}2+k\pi$, $k\in\mathbb Z$;\\
\item $\displaystyle D_n=\sum_{k=-n}^n e^{ikx}$ et $\displaystyle K_n=\sum_{k=0}^n D_k$, avec $x\neq 0+2k\pi$, $k\in\mathbb Z$.\\
\end{enumerate}


%
\vspace{1em}
\hrule
\vspace{1em}

\exo[2]{Module}

Déterminer $z$ pour que $z, 1-z$ et $z^2$ aient le m\^eme module.

%\encad{SOLUTION :
%
%En posant $z=x+iy$, on déduite $1-z=1-x-iy$ et $z^2=x^2-y^2+i2xy$. Ces trois nombres complexes ont le méme module si $x, y$ vérifient les égalités $(1)$ et $(2)$ suivantes (écriture similaire à un systéme) :
%$$x^2+y^2\stackrel{(1)}{=}(1-x)^2+y^2\stackrel{(2)}{=}(x^2-y^2)^2+4x^2y^2$$
%
%- De l'égalité $(1)$, on déduit $x^2=1-2x+x^2$, soit $x=\frac{1}{2}$.
%
%- En remplaéant $x$ par $\frac{1}{2}$ dans l'égalité $(2)$, on déduit de $1-2x+x^2+y^2=x^4+y^4+2x^2y^2$ l'égalité
%$$\begin{array}{l}\frac{1}{4}+y^2=\frac{1}{16}+y^4+\dfrac{y^2}{2}\\y^4-\dfrac{y^2}{2}-\dfrac{3}{16}=0\\16y^4-8y^2-3=0\\16Y^2-8Y-3=0\quad\text{en posant } Y=y^2\end{array}$$
%
%Les solutions réelles é cette équation en $Y$ sont $-\dfrac{1}{4}$ et $\dfrac{3}{4}$ : on ne retient que $\dfrac{3}{4}$ pour $y^2$, soit $y=-\dfrac{\sqrt 3}{2}$ ou $y=\dfrac{\sqrt 3}{2}$.
%
%- Finalement, l'ensemble des solutions au probléme est $\left\{\dfrac{1}{2}+i\dfrac{\sqrt 3}{2},\dfrac{1}{2}-i\dfrac{\sqrt 3}{2}\right\}$
%}
\vspace{1em}
\hrule
\vspace{1em}

\exo[3]{Primitives}

\begin{enumerate}
\item \`A l'aide des formules d'Euler, linéariser $f(x)=\cos2x\sin3x$ et $g(x)=\cos x\sin^5x$.
\item En déduire les primitives $\ds\int f(x)\, dx$ et $\ds\int g(x)\, dx$.
\item Par le calcul direct de $\ds\int \cos x\sin^5x\, dx$ puis par linéarisation de la primitive, retrouver le résultat de la question 2.\\
\end{enumerate}

%\encad{SOLUTION :
%
%\begin{enumerate}
%\item D'une part $$\begin{array}{rl}
%f(x)=\cos2x\sin3x&=\left(\dfrac{e^{i2x}+e^{-i2x}}{2}\right)\left(\dfrac{e^{i3x}-e^{-i3x}}{2i}\right)\\
%&=\dfrac{1}{4i}(e^{i5x}-e^{-ix}+^{ix}+e^{-i5x})\\
%&\\
%&=\dfrac{1}{4i}(2i\sin 5x+2i\sin x)\\
%&\\
%&=\dfrac{1}{2}\sin 5x+\dfrac{1}{2}\sin x
%\end{array}$$
%
%D'autre part, en utilisant le développement $(a-b)^5=a^5-5a^4b+10a^3b^2-10a^2b^3+5ab^4-b^5$, $$\begin{array}{rl}
%g(x)=\cos x\sin^5 x&=\left(\dfrac{e^{ix}+e^{-ix}}{2}\right)\left(\dfrac{e^{ix}-e^{-ix}}{2i}\right)^5\\
%&=\dfrac{1}{64i}(e^{ix}+e^{-ix})(e^{5ix}-5e^{i3x}+10e^{ix}-10e^{-ix}+5e^{-i3x}-e^{-i5x})\\
%&\\
%&=\dfrac{1}{64i}(e^{i6x}-e^{-i6x}-4(e^{i4x}-e^{-i4x})+5(e^{i2x}-e^{-i2x})))\\
%&\\
%&=\dfrac{1}{64i}(2i\sin 6x-8i\sin 4x+10i\sin 2x)\\
%&\\
%&=\dfrac{1}{32}\sin 6x-\dfrac{1}{8}\sin 4x+\dfrac{5}{32}\sin 2x\\
%\end{array}$$
%\item On en déduit $\ds\int f(x)\, dx=-\dfrac{1}{10}\cos 5x-\dfrac{1}{2}\cos x+k,\quad k\in\R$ et $\ds\int g(x)\, dx=-\dfrac{1}{192}\cos 6x-\dfrac{1}{32}\cos 4x-\dfrac{5}{64}\cos 2x+k,\quad k\in\R$
%\item $\ds\int \cos x\sin^5x\, dx=\dfrac{1}{6}\sin^6 x+k$. Il ne reste plus qu'é (!) calculer $\sin^6 x=\left(\dfrac{e^{ix}-e^{-ix}}{2i}\right)^6$...
%\end{enumerate}
%}

\vspace{1em}
\hrule
\vspace{1em}

\exo[2]{Imaginaire pur ou réel?}


Soit le nombre complexe $T=\dfrac{e^{i\theta}-1}{i(e^{i\theta}+1)}$.

\'Etablir si $T$ est un imaginaire pur, ou un réel.

%\encad{SOLUTION :
%\begin{eqnarray*}
%T&=&\dfrac{e^{i\theta}-1}{i(e^{i\theta}+1)}=\dfrac{e^{i\frac{\theta}{2}}(e^{i\frac{\theta}{2}}-e^{-i\frac{\theta}{2}})}{ie^{i\frac{\theta}{2}}(e^{i\frac{\theta}{2}}+e^{-i\frac{\theta}{2}})}=\dfrac{2i\sin(\theta/2)}{i2\cos(\theta/2)}=\tan(\theta/2)\in\R\\
%\end{eqnarray*}
%
%}

\vspace{1em}
\hrule
\vspace{1em}


\exo[2]{Nouvelle équation}


Résoudre dans $\C$ l'équation : \quad $z^2+(2+2i)z-3+6i=0$

%\encad{SOLUTION :
%
% à partir de la méthode proposée en cours, on trouve deux solutions $$z_1=-3\quad\text{et}\quad z_2=1-2i$$
%}

\vspace{1em}
\hrule
\vspace{1em}

\exo[2]{Racines cinquièmes}


Déterminer les racines cinqui\` eme du complexe $16(1+i\sqrt 3)$.\\

%\encad{SOLUTION :
%
%\includegraphics[scale=0.85,trim=1cm 2cm 9.5cm 10.5cm, clip]{EG2_CA1_TD5corr-p1}
%}

\vspace{1em}
\hrule
\vspace{1em}

\exo[2]{Racines}

\begin{enumerate}
\item Résoudre dans $\C$ l'équation $Z^9=1$
\item En déduire les solutions de l'équation $\left(\dfrac{z-1}{z+1}\right)^9=1$
\end{enumerate}

\vspace{1em}
\hrule
\vspace{1em}
%\encad{SOLUTION :
%
%\includegraphics[scale=0.85,trim=1cm 0 0cm 2.5cm, clip]{EG2_CA1_TD5corr-p2}
%
%}

\end{document}
