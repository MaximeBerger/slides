\documentclass[../PolyS1.tex]{subfiles}
\begin{document}

\textbf{Dans tout ce chapitre, $\mathbb{K}$ désignera $\mathbb{R}$ ou $\mathbb{C}$.}

\section{Définitions - Premiers exemples}

\subsection{Définitions}

\begin{Def}\textbf{Système linéaire}
  \vspace{1em}

Soient $p$ et $n$ deux entiers non nuls. On appelle \textbf{système linéaire de $n$ équations à $p$ inconnues} un système de la forme :
\[
(S) : 
\left\{
\begin{array}{rcrcl}
a_{1,1}\,x_1 &+& \cdots + a_{1,p}\,x_p &=& b_1 \\
a_{2,1}\,x_1 &+& \cdots + a_{2,p}\,x_p &=& b_2 \\
\vdots && && \vdots \\
a_{n,1}\,x_1 &+& \cdots + a_{n,p}\,x_p &=& b_n \\
\end{array}
\right.
\]
Les scalaires $(a_{i,j})_{1 \le i \le n, 1 \le j \le p}$ sont appelés \textbf{coefficients} du système, \newline 
les $(b_i)_{1 \le i \le n}$ forment le \textbf{second membre}, \newline 
et $x_1, \ldots, x_p$ sont les \textbf{inconnues} du système.
\end{Def}

Résoudre un tel système consiste à trouver tous les $p$-uplets $(x_1, \ldots, x_p) \in \mathbb{K}^p$ qui vérifient simultanément les $n$ équations.


\begin{Def}\textbf{Systèmes équivalents, système incompatible}
  \vspace{1em}

\begin{itemize}
  \item Deux systèmes $(S)$ et $(S')$ sont dits \textbf{équivalents} s'ils ont les mêmes solutions.
  \item Un système $(S)$ est dit \textbf{incompatible} s'il n'admet aucune solution.
\end{itemize}
\end{Def}

\begin{Def}\textbf{Systèmes homogènes, système homogène associé}
  \vspace{1em}

\begin{itemize}
  \item Un système est dit \textbf{homogène} si son second membre est nul.
  \item Le \textbf{système homogène associé} à un système $(S)$ est le système $(H)$ obtenu en remplaçant tous les membres $b_i$ par $0$.
\end{itemize}
\end{Def}

\vspace{1em}
\hrule
\vspace{1em}

\exo[1]{Reconnaissance directe}

Déterminer si le système suivant est linéaire ou non. Si oui, expliciter les coefficients et le second membre. 

\[
\left\{
\begin{array}{rcrcl}
2x &+& 3y &=& 7 \\
5x &-& y &=& 1
\end{array}
\right.
\]
\vspace{1em}
\hrule
\vspace{1em}


\exo[1]{Présence d'un produit}

Les systèmes suivants sont-ils linéaires ? Si oui, expliciter les coefficients et le second membre. 

\[
\left\{
\begin{array}{rcrcl}
x^2 &+& y &=& 4 \\
x &+& 3y &=& 2
\end{array}
\right.
\qquad\quad 
\left\{
\begin{array}{rcrcl}
\sin(x) &+& y &=& 1 \\
2x &+& 3y &=& 5
\end{array}
\right.
\qquad \quad 
\left\{
\begin{array}{rcrcl}
xy &+& z &=& 3 \\
x &+& y &+& z = 1
\end{array}
\right.
\]

\vspace{1em}
\hrule
\vspace{1em}


\exo[1]{Système homogène}

Le système suivant est-il linéaire et homogène ?

\[
\left\{
\begin{array}{rcrcl}
x &-& y &+& 2z = 0 \\
3x &+& 0y &+& z = 0
\end{array}
\right.
\]

\vspace{1em}
\hrule
\vspace{1em}


\exo[1]{Racine}

Analyser la nature du système suivant :

\[
\left\{
\begin{array}{rcrcl}
\sqrt{x} &+& y &=& 2 \\
x &+& y &=& 3
\end{array}
\right.
\]
\vspace{1em}
\hrule
\vspace{1em}

\newpage

\exo[2]{Système avec paramètre}

Étudier la linéarité selon le paramètre \( a \in \mathbb{R} \) :

\[
\left\{
\begin{array}{rcrcl}
a x &+& y &=& 2 \\
x &+& a^2 y &=& 3
\end{array}
\right.
\]

\vspace{1em}
\hrule
\vspace{1em}


\exo[2]{Produit de fonction}

Le système suivant est-il linéaire ?

\[
\left\{
\begin{array}{rcrcl}
x &\cdot& \sin(y) &=& 0 \\
x &+& y &=& 1
\end{array}
\right.
\]

\vspace{1em}
\hrule
\vspace{1em}


\exo[2]{Trois équations, deux inconnues}

Le système suivant est-il linéaire ? Est-il incompatible ?

\[
\left\{
\begin{array}{rcrcl}
x &+& y  &=& 2 \\
2x &-& y &=& 3 \\
3x &+& 4y &=& 1
\end{array}
\right.
\]

\vspace{1em}
\hrule
\vspace{1em}


\subsection{Les différents cas possibles pour les systèmes linéaires}

\begin{Meth}
On distingue trois cas pour un système de $n$ équations à $p$ inconnues :

\begin{itemize}
  \item \textbf{Si $n > p$} : plus d'équations que d'inconnues. Deux cas possibles :
    \begin{itemize}
      \item le système est incompatible.
      \item certaines équations sont redondantes, et on revient au cas $n \le p$.
    \end{itemize}
  \item \textbf{Si $n = p$} : système carré d'ordre $n$. Trois cas possibles :
    \begin{itemize}
      \item une unique solution (système de \textbf{Cramer}) ;
      \item une infinité de solutions ;
      \item aucune solution (système incompatible).
    \end{itemize}
  \item \textbf{Si $n < p$} : deux cas possibles :
    \begin{itemize}
      \item une infinité de solutions ;
      \item aucune solution (système incompatible).
    \end{itemize}
\end{itemize}
\end{Meth}

\newpage

\subsection{Résolution dans des cas simples}

\subsubsection{Résolution par substitution}

\begin{Ex}Résolvons le système suivant par substitution :
\[
(S) :
\left\{
\begin{array}{rcrcl}
x &+& 3y &=& 2 \\
2x &+& 7y &=& 1
\end{array}
\right.
\]
\end{Ex}

\begin{Rmq}

Cette méthode devient vite fastidieuse lorsque le nombre d'équations ou d'inconnues dépasse $2$.
\end{Rmq}
\subsubsection{Systèmes triangulaires (ou échelonnés)}

\begin{Ex}
\[
(S) :
\left\{
\begin{array}{cccc}
x &+ 3y &+ 2z &= 2 \\
0x &+ 2y &+ 5z &= 4 \\
0x &+ 0y &+ 4z &= 8
\end{array}
\right.
\]
\end{Ex}
Ce système peut être résolu facilement par \textbf{remontée successive}.

\begin{Ex}
\[
(S) :
\left\{
\begin{array}{cccc}
-x &+ 2y &- z &= 0 \\
0x &+ y &- 2z &= 5
\end{array}
\right.
\]


Le nombre d'inconnues est supérieur au nombre d'équations.\vspace{1em}
Une inconnue (par exemple $z$) peut être choisie libre.\vspace{1em}
\end{Ex}


\newpage

\section{Pivot de Gauss}

\subsection{Opérations Élémentaires}

\begin{Def}\textbf{Opérations élémentaires}
  \vspace{1em}

Soit $(S)$ un système. On appelle \textbf{opérations élémentaires} les opérations suivantes :
\begin{enumerate}
  \item $L_i \leftrightarrow L_j$, $i \neq j$ : échange des lignes $i$ et $j$.
  \item $L_i \leftarrow a L_i$, $a \neq 0$ : multiplication d'une ligne par un scalaire non nul.
  \item $L_i \leftarrow L_i + b L_j$, $i \neq j$ : addition d'un multiple de la ligne $j$ à la ligne $i$.
  \item $L_i \leftarrow a L_i + b L_j$, $a \neq 0$ et $i \neq j$ : combinaison des deux précédentes.
\end{enumerate}
\end{Def}

\begin{Prop}\textbf{Opérations élémentaires et équivalence (admis)} 
  \vspace{1em}

Toute opération élémentaire appliquée à un système $(S)$ donne un système équivalent à $(S)$.
\end{Prop}

\subsection{Pivot de Gauss}

\begin{Meth}
Cette méthode procède en deux temps :
\begin{itemize}
  \item Élimination successive des inconnues à l'aide des opérations élémentaires pour passer du système initial à un système triangulaire équivalent.
  \item Remontée du système triangulaire obtenu.
\end{itemize}

\end{Meth}
\vspace{1em}
\hrule
\vspace{1em}
\exo[2]{Entraînement}
Résoudre les systèmes suivants par la méthode du pivot de Gauss.

\begin{equation*}
(S_1) :
\left\{
\begin{array}{rcrcrcl}
x &+& 2y &+& 3z &=& 2 \\
3x &+& y &+& 2z &=& 1 \\
2x &+& 3y &+& z &=& 0
\end{array}
\right.
\qquad 
(S_2) :
\left\{
\begin{array}{rcrcrcl}
3x &-& y &+& 2z &=& 1 \\
-x &+& 2y &-& 3z &=& 2 \\
x &+& 2y &+& z &=& 0
\end{array}
\right.
\end{equation*}

\begin{equation*}
(S_3) :
\left\{
\begin{array}{rcrcrcl}
&&y &-& z &=& 1 \\
2x &+& y &+& z &=& 3 \\
x &&&+& z &=& 1
\end{array}
\right.
\qquad
(S_4) :
\left\{
\begin{array}{rcrcrcl}
&&y &-& z &=& 1 \\
2x &+& y &+& z &=& 1 \\
x &&&+& z &=& 1
\end{array}
\right.
\end{equation*}

\vspace{1em}
\hrule
\vspace{1em}

\begin{Prop}\textbf{Caractérisation d'un système de Cramer par le pivot de Gauss}
  \vspace{1em}

Un système carré $(S)$ est un système de Cramer si on peut trouver un pivot non nul à chaque étape du pivot de Gauss.
\end{Prop}



\section{Exercices}

\vspace{1em}
\hrule
\vspace{1em}

\exo[2]{Ma\c cons}

Trois maçons et deux ouvriers polyvalents réalisent l'extension d'un bâtiment en $6$ jours et gagnent au total $2520$ \euro.
Pour un travail similaire, cinq maçons et trois ouvriers polyvalents gagnent $2720$ \euro \ et terminent le chantier en $4$ jours.

Calculer le salaire journalier d'un maçon et celui d'un ouvrier polyvalent.

\vspace{2em}

\hrule
\exo[2]{Chantier}

Le nombre de briques \(N\) sur un chantier (stock existant + livraisons) est représenté par l'équation :
\[
N = a \times D + e
\]
où :
\begin{itemize}
    \item \(D\) = nombre de jours,
    \item \(a\) = constante,
    \item \(e\) = stock existant avant le début des livraisons régulières.
\end{itemize}

Après \(2\) jours, le nombre de briques sur le chantier était de \(22\,000\), et après \(5\) jours, il était de \(40\,000\).

En supposant que les livraisons de briques sont régulières, calculer les valeurs de \(a\) et \(e\), puis déterminer le nombre de briques après \(8\) jours.
\vspace{1em}
\hrule
\vspace{1em}

% \exo{Équilibre d'une structure isostatique}
% Une structure triangulaire est soumise à trois forces extérieures en équilibre. Les composantes horizontales et verticales des forces $F_1$, $F_2$, et $F_3$ satisfont le système :
% \[
% \left\{
% \begin{array}{l}
% F_{1x} + F_{2x} + F_{3x} = 0 \\
% F_{1y} + F_{2y} + F_{3y} = 0 \\
% 3F_{1y} - 2F_{2x} + F_{3x} = 0
% \end{array}
% \right.
% \]
% Déterminer les composantes manquantes si certaines sont connues. Interpréter le système obtenu.
% \vspace{1em}
% \hrule
% \vspace{1em}

\exo[2]{Répartition de charges sur poutre}
Une poutre est appuyée sur trois points $A$, $B$ et $C$, et soumise à des charges. Les réactions $R_A$, $R_B$, $R_C$ vérifient :
\[
\left\{
\begin{array}{rcrcrcl}
R_A &+& R_B &+& R_C &=& 30 \\
2R_A &+& R_B &+& 0R_C &=& 20 \\
0R_A &+& R_B &+& 3R_C &=& 25
\end{array}
\right.
\]
Déterminer les réactions aux appuis. Interpréter le résultat.
\vspace{1em}
\hrule
\vspace{1em}

\newpage

\exo[1]{Dosage de béton}
Un dosage impose : 1 m$^3$ de béton est composé de gravier ($G$), sable ($S$), ciment ($C$) et eau ($E$), vérifiant :
\[
\left\{
\begin{array}{rcl}
G + S + C + E &=& 1 \\
G &=& 2S \\
C &=& 0.15 \\
E &=& 0.1
\end{array}
\right.
\]
Résoudre le système et donner les volumes de chaque composant. Vérifier que les proportions sont correctes pour un béton standard.
\vspace{1em}
\hrule
\vspace{1em}

\exo[3]{Stabilisation d'un mur de soutènement}
Un mur de soutènement exerce des réactions d'appui sur trois contreforts. Ces réactions vérifient :
\[
\left\{
\begin{array}{rcrcrcl}
R_1 &+& R_2 &+& R_3 &=& 150 \\
2R_1 &+& R_2 &+& 0R_3 &=& 110 \\
0R_1 &+& R_2 &+& R_3 &=& 100
\end{array}
\right.
\]
Calculer $R_1$, $R_2$, $R_3$ et discuter la stabilité du mur.
\vspace{1em}
\hrule
\vspace{1em}

\exo[2]{Répartition de coûts dans un chantier}
Trois entreprises $A$, $B$ et $C$ réalisent un chantier. Leur répartition de charges obéit aux équations :
\[
\left\{
\begin{array}{rcrcrcl}
x &+& y &+& z &=& 100 \quad \text{(total des coûts)} \\
2x &-& y &+& 0z &=& 10 \\
0x &+& y &+& z &=& 70
\end{array}
\right.
\]
Calculer les montants $x$, $y$, $z$ payés par chaque entreprise.

\vspace{1em}
\hrule
\vspace{1em}

\exo[2]{Résistance des matériaux}
Une structure métallique repose sur 3 barres qui doivent supporter une charge totale de 900 N. La répartition des efforts est modélisée par :
\[
\left\{
\begin{array}{rcrcrcl}
F_1 &+& F_2 &+& F_3 &=& 900 \\
2F_1 &-& F_2 &+& 0F_3 &=& 100 \\
0F_1 &+& F_2 &-& 2F_3 &=& 0
\end{array}
\right.
\]
Trouver la force exercée sur chaque barre.

\vspace{1em}
\hrule
\vspace{1em}

\newpage

\exo[2]{Approvisionnement d'un chantier}
Une centrale livre trois types de matériaux : sable ($x$), gravier ($y$), ciment ($z$). Un client commande trois mélanges $M_1$, $M_2$, $M_3$ définis par :
\[
\left\{
\begin{array}{rcrcrcl}
x &+& 2y &+& z &=& 30 \\
2x &+& y &+& 3z &=& 50 \\
x &+& y &+& z &=& 25
\end{array}
\right.
\]
Déterminer la quantité de chaque composant à livrer.

\vspace{1em}
\hrule
\vspace{1em}

\exo[2]{Ventilation dans un bâtiment}
L'air circule dans 3 conduits $x$, $y$, $z$ d'un réseau de ventilation. On impose :
\[
\left\{
\begin{array}{rcrcrcl}
x &+& y &=& 200 \\
x &+& 0y &+& z &=& 180 \\
0x &+& y &+& z &=& 220
\end{array}
\right.
\]
Trouver le débit d'air dans chaque conduit.

\vspace{1em}
\hrule
\vspace{1em}

\exo[3]{Budget d'un projet immobilier}
Trois catégories de dépenses : terrain ($T$), travaux ($W$), honoraires ($H$) avec contraintes :
\[
\left\{
\begin{array}{rcrcrcl}
T &+& W &+& H &=& 1\,000\,000 \\
0T &+& W &-& 3T &=& 0 \\
0T &+& 0W &+& H &=& 0.1(W + H)
\end{array}
\right.
\]
Résoudre le système pour connaître la répartition du budget.

\end{document}
