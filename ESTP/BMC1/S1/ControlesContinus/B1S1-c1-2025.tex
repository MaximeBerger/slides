\documentclass[a4paper,11pt]{article}

%	Tous les packages pour francisation compl�te
%\usepackage[applemac]{inputenc}
\usepackage[T1]{fontenc}
\usepackage{lmodern,textcomp}
\usepackage[frenchb]{babel}

\usepackage{amsfonts,amsmath,amssymb,mathrsfs}
\usepackage{geometry}
\usepackage{aeguill}
\usepackage{hyperref}
\usepackage{array}
\usepackage{multicol}
\usepackage{asymptote}
\usepackage{graphicx}
\usepackage{tabularx}
\usepackage{enumerate,tablists}
\usepackage{eurosym}
\usepackage{pst-solides3d}
\usepackage{multirow}
\usepackage{extsizes}
\usepackage{calc}
\usepackage{boites}
\usepackage{color,colortbl}
\usepackage{fancyhdr}
\usepackage{stmaryrd}
\usepackage{pifont}
%\usepackage{pst-all,pstricks,pstricks-add,pst-plot,pst-eucl,pst-text,pst-tree,pst-math,pst-eps}

%\usepackage{auto-pst-pdf}
\makeatletter
\let\Test@pr@shipout\pr@shipout%% save the original definition
\let\Test@shipout\shipout
\makeatother
\usepackage{tikz,tkz-tab}
\makeatletter
\AtBeginDocument{%
  \let\pr@shipout\Test@pr@shipout%% restore it 
  \let\shipout\Test@shipout
}
\makeatother

\newtheorem{definition}{D\'efinition}
\newtheorem{theoreme}{Th\'eor\`eme}

\newcommand{\C} {\ensuremath{\mathbb{C}}}
\newcommand{\R} {\ensuremath{\mathbb{R}}}
\newcommand{\Z} {\ensuremath{\mathbb{Z}}}
\newcommand{\N} {\ensuremath{\mathbb{N}}}
\newcommand{\ds} {\displaystyle}
\newcommand{\p}{\par}
\newcommand{\abs}[1]{\left |#1\right |}
\newcommand{\eq}[1]{\underset{#1}{\sim}}
\newcommand{\tend}[1]{\underset{#1}{\longrightarrow}}
\newcommand{\norme}[1]{\left\Vert #1\right\Vert}
\newcommand{\V}[1]{\overrightarrow{#1}} %vecteur

\definecolor{gris1}{gray}{0.85}
\definecolor{gris2}{gray}{0.65}

\setlength{\textwidth} {19cm}
\setlength{\textheight} {26cm}
\addtolength{\topmargin} {-2.5cm}
\setlength{\oddsidemargin} {-1.5cm}
\setlength{\evensidemargin} {-1.5cm}

\newcommand{\encad}[1]{\begin{center}%
\fcolorbox{black}{gray!20}{\begin{minipage}[t]{0.8\linewidth}%
#1\end{minipage}}\end{center}}


\pagestyle{fancy}
\fancyfoot[R]{\today}\renewcommand\headrulewidth{1pt}
\fancyhead[L]{ESTP / Bachelor 1 / \the\year}
\fancyhead[R]{Composition de math\'ematiques}
\renewcommand\footrulewidth{1pt}
%\fancyfoot[L]{Math�matiques : les polyn�mes}
%\fancyfoot[R]{\thepage}


\begin{document}
\begin{center}\bf {\Large\underline{Composition de MATH\'EMATIQUES (3h) : 35 points}}\end{center}
%\begin{center}\bf {{\small Rendre le sujet avec votre copie si le graphe de l'exercice 4 pr�sente la courbe de la fonction $f$.}}\end{center}
\hfill\break

%%%%%%%%%%
%\includegraphics[trim = gauche bas droit haut, clip, width=3cm]
%%%%%%%%%%
\underline {\ding{110}\, \bf{Exercice 1 : \'Equations} (8 points)}

\medskip

Résoudre dans $\mathbb{R}$ les équations ou systèmes suivants :
\begin{multicols}{2}
\begin{enumerate}
\item $e^{2x} - 2e^{x} + 1 = 0$\\
\item $\ln(4x^2) = 2$\\
\item $\cos\left ( 3x -\dfrac{\pi}{2}\right)=\dfrac{1}{2}$\\
\item $\cos (7x)+\sin(7x)=1$\\
\end{enumerate}
\end{multicols}


%{\color{blue}\underline{\bf SOLUTION :}
%\medskip
%
%\begin{enumerate}[1)]
%\item En posant $X=e^x$, on r�sout l'�quation $3X^2+X-4$. On trouve comme solutions $X_1=1$ et $X_2=-4/3$. On en d�duit la seule solution possible en $x$ � partir de $e^x=1$, soit 0.
%\item Il est n�cessaire que $x\neq 1$ et $x>0$.
%
%Il suffit alors que $\abs{\ln x}=1\Longleftrightarrow \ln x=1\text{ ou } \ln x=-1\Longleftrightarrow x=e \text{ ou } x=e^{-1}=1/e$.
%\item Il est n�cessaire que $x^2+x>0$, soit $x\in]-\infty, -1[\cup]0,+\infty[$.
%
%Il suffit alors que $x^2+x=e\Longleftrightarrow x^2+x-e=0\Longleftrightarrow x=\dfrac{-1-\sqrt{1+4e}}{2}\text{ ou }x=\dfrac{-1+\sqrt{1+4e}}{2}$. C'est deux solutions sont acceptables puisque $\dfrac{-1-\sqrt{1+4e}}{2}<-1$ et $\dfrac{-1+\sqrt{1+4e}}{2}>0$.
%\end{enumerate}
%
%}

\hfill\break
\hrule
\hfill\break


%Pour chaque fonction, préciser la nature des asymptotes (verticale, horizontale ou oblique) et tracer une ébauche du graphe de la fonction.

%
%Les courbes ne sont fournies qu'\`a titre indicatif afin d'orienter votre recherche et de valider vos r\'esultats.\\
%%
%\begin{minipage}{0.3\linewidth}
%$$f:x\mapsto \dfrac{x\sqrt x}{x^2+1}$$
%\end{minipage}
%\begin{minipage}{0.7\linewidth}
%\begin{center}
%\includegraphics[width=0.75\linewidth]{f2.png}
%\end{center}
%\end{minipage}
%%
%\hfill\break
%\hfill\break
%
%\begin{minipage}{0.3\linewidth}
%$$g:x\mapsto \dfrac{1}{2}x+2+\ln\left(\dfrac{x-1}{x+1}\right)$$
%\end{minipage}
%\begin{minipage}{0.7\linewidth}
%\begin{center}
%\includegraphics[width=0.75\linewidth]{f1.png}
%\end{center}
%\end{minipage}
%%
%
%\hfill\break
%\hfill\break
%
%\begin{minipage}{0.3\linewidth}
%$$h:x\mapsto (x+ 1)^2e^{-x}$$
%\end{minipage}
%\begin{minipage}{0.7\linewidth}
%\begin{center}
%\includegraphics[width=0.75\linewidth]{f3.png}
%\end{center}
%\end{minipage}
%%


\underline {\ding{110}\, \bf{Exercice 2 : Réponse d’un capteur à un changement de température (6 points)}}\\

Un capteur de température est initialement à l'équilibre thermique dans un environnement à $20^\circ$C. À l'instant $t = 0$, il est brusquement plongé dans un liquide maintenu à $80^\circ$C. 

On modélise la température $T(t)$ indiquée par le capteur à l'instant $t$ (en secondes) par la fonction :
\[
T(t) = 80 - 60 e^{-t/5}
\]
où $T(t)$ est exprimée en degrés Celsius.\\

\begin{enumerate}
  \item Quelle est la température initiale lue par le capteur ? Et la température à long terme ?\\
  \item À quelle date le capteur affiche-t-il une température de $50^\circ$C ?\\
  \item Déterminer la dérivée $T'(t)$, puis montrer que :
  \[
  T'(t) = \frac{12}{e^{t/5}}
  \]
  \item Interpréter le signe de $T'(t)$. Que peut-on dire de l’évolution de la température ?\\
  \item Déterminer le temps mis par le capteur pour atteindre $95\%$ de la température finale.\\
  %\item Quel est le temps caractéristique de ce système ? Justifier à partir de l'expression donnée.
\end{enumerate}

\hfill\break
\hrule
\hfill\break

\newpage

\underline {\ding{110}\, \bf{Exercice 3 : } Logique, sommes et produits (10 points)}\\

\begin{enumerate}
  \item Donner la table de vérité de la proposition logique :
  \[
  (P \Rightarrow Q)
  \]
  \item Donner la table de vérité de la proposition logique :
  \[
  (P \Rightarrow Q) \land (\lnot Q \Rightarrow \lnot P)
  \]
  \item En déduire une équivalence logique.\\
    \item Démontrer par récurrence que :
  \[
 (*) \qquad \sum_{k=1}^{n} k = \frac{n(n+1)}{2}
  \]
  \item Calculer les expressions suivantes :
  \begin{itemize}
    \item $\displaystyle \sum_{k=1}^{n} \ln\left(1+\dfrac{1}{k}\right)$
    \item $\displaystyle \prod_{k=1}^{n} e^{k-1}$, on pourra utiliser la relation $(*)$.\\
  \end{itemize}

\end{enumerate}

\hfill\break
\hrule
\hfill\break

\underline {\ding{110}\, \bf{Exercice 4 : }Vecteurs et géométrie plane (11 points)}\\

On considère dans le plan muni d’un repère orthonormé les points : $A(1,2,0)$, $B(0,4,6)$, $C(7,-1,2)$.\\

\begin{enumerate}
  \item Calculer les coordonnées des vecteurs $\overrightarrow{AB}$ et $\overrightarrow{AC}$.\\
  \item Calculer les longueurs $AB$ et $AC$.\\
  \item Calculer le produit scalaire entre les vecteurs $\overrightarrow{AB}$ et $\overrightarrow{AC}$ et le produit vectoriel vecteurs $\overrightarrow{AB}$ et $\overrightarrow{AC}$.\\
  \item En déduire le cosinus et le sinus de l'angle $\big(\overrightarrow{AB},\overrightarrow{AC}\big)$.\\
  \item Donner une équation paramétrique de la droite $(AB)$.\\
  \item Déterminer la distance du point $C$ à la droite $(AB)$.\\
\end{enumerate}


\end{document}
