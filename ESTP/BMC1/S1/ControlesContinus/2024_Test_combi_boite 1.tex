\documentclass[10pt]{article}

\usepackage[utf8]{inputenc}
\usepackage[french]{babel}
\usepackage{graphicx}
%\usepackage{wrapfig}
\usepackage{dsfont}
\usepackage[margin=2cm]{geometry}
\usepackage{amsmath,amssymb}
%\usepackage{tikz}
\usepackage{multicol}
\newcommand{\xB}{{\cal B}}
\newcommand{\xC}{{\cal C}}
\newcommand{\R}{\mathbf{R}}

\begin{document}

\pagestyle{empty}

\noindent
\begin{minipage}[l]{8cm}
  \scriptsize{ESTP, S3\\Ann\'ee universitaire 2024/2025}
\end{minipage}

\begin{center}
  {\large\textbf{Petit test \no 1} \\
20 minutes\\}
  \bigskip
  

 
\end{center}

\bigskip

\begin{center}
  \fbox{%
    \begin{minipage}{0.95\linewidth}
      NOM: \hspace*{5cm} Pr\'enom: \hspace*{4cm} Groupe: TD 
    \end{minipage}
  }
  \end{center}

\bigskip
 
 

\noindent\textbf{Exercice  : 5 points}\\




\begin{enumerate}
    \item 
Dans un bureau de vingt-cinq personnes, combien existe-t-il de façons
de choisir un président, un vice -président, un secrétaire et un trésorier ?\\

    \item

Quel est le nombre de choix possibles de 13 étudiants parmi un groupe
de 120 étudiants ?\\

    \item 
Une équipe de 8 personnes est amenée à occuper 8 postes de
travail distincts.\\
Combien peut-on envisager de répartitions distinctes des 8
personnes.\\
\end{enumerate}

\noindent\textbf{Exercice  : 5 points}\\


On considère la série statistique suivante:\\


 \begin{tabular}{|c|c|c|c|c|c|}
 \hline
 valeur&10&15&16&20&21\\
 \hline
 effectif &2& 3&7&7&1\\
 \hline
 \end{tabular}

\begin{enumerate}
    \item 
    Donner l'étendue.

    \item
 Donner l'écart inter-quartile.

    \item 
  Dessiner la boîte à moustaches.\\
\end{enumerate}

\end{document}
