\documentclass[a4paper,11pt]{article}

%	Tous les packages pour francisation compl�te
%\usepackage[applemac]{inputenc}
\usepackage[T1]{fontenc}
\usepackage{lmodern,textcomp}
\usepackage[frenchb]{babel}

\usepackage{amsfonts,amsmath,amssymb,mathrsfs}
\usepackage{geometry}
\usepackage{aeguill}
\usepackage{hyperref}
\usepackage{array}
\usepackage{multicol}
\usepackage{asymptote}
\usepackage{graphicx}
\usepackage{tabularx}
\usepackage{enumerate,tablists}
\usepackage{eurosym}
\usepackage{pst-solides3d}
\usepackage{multirow}
\usepackage{extsizes}
\usepackage{calc}
\usepackage{boites}
\usepackage{color,colortbl}
\usepackage{fancyhdr}
\usepackage{stmaryrd}
\usepackage{pifont}
%\usepackage{pst-all,pstricks,pstricks-add,pst-plot,pst-eucl,pst-text,pst-tree,pst-math,pst-eps}

%\usepackage{auto-pst-pdf}
\makeatletter
\let\Test@pr@shipout\pr@shipout%% save the original definition
\let\Test@shipout\shipout
\makeatother
\usepackage{tikz,tkz-tab}
\makeatletter
\AtBeginDocument{%
  \let\pr@shipout\Test@pr@shipout%% restore it 
  \let\shipout\Test@shipout
}
\makeatother

\newtheorem{definition}{D\'efinition}
\newtheorem{theoreme}{Th\'eor\`eme}

\newcommand{\C} {\ensuremath{\mathbb{C}}}
\newcommand{\R} {\ensuremath{\mathbb{R}}}
\newcommand{\Z} {\ensuremath{\mathbb{Z}}}
\newcommand{\N} {\ensuremath{\mathbb{N}}}
\newcommand{\ds} {\displaystyle}
\newcommand{\p}{\par}
\newcommand{\abs}[1]{\left |#1\right |}
\newcommand{\eq}[1]{\underset{#1}{\sim}}
\newcommand{\tend}[1]{\underset{#1}{\longrightarrow}}
\newcommand{\norme}[1]{\left\Vert #1\right\Vert}
\newcommand{\V}[1]{\overrightarrow{#1}} %vecteur

\definecolor{gris1}{gray}{0.85}
\definecolor{gris2}{gray}{0.65}

\setlength{\textwidth} {19cm}
\setlength{\textheight} {26cm}
\addtolength{\topmargin} {-2.5cm}
\setlength{\oddsidemargin} {-1.5cm}
\setlength{\evensidemargin} {-1.5cm}

\newcommand{\encad}[1]{\begin{center}%
\fcolorbox{black}{gray!20}{\begin{minipage}[t]{0.8\linewidth}%
#1\end{minipage}}\end{center}}


\pagestyle{fancy}
\fancyfoot[R]{\today}\renewcommand\headrulewidth{1pt}
\fancyhead[L]{ESTP / Bachelor 1 / \the\year}
\fancyhead[R]{Composition de math\'ematiques}
\renewcommand\footrulewidth{1pt}
%\fancyfoot[L]{Math�matiques : les polyn�mes}
%\fancyfoot[R]{\thepage}


\begin{document}
\begin{center}\bf {\Large\underline{Composition de MATH\'EMATIQUES (3h) : 35 points}}\end{center}
%\begin{center}\bf {{\small Rendre le sujet avec votre copie si le graphe de l'exercice 4 pr�sente la courbe de la fonction $f$.}}\end{center}
\hfill\break

%%%%%%%%%%
%\includegraphics[trim = gauche bas droit haut, clip, width=3cm]
%%%%%%%%%%
\underline {\ding{110}\, \bf{Exercice 1 : \'Equations} (8 points)}

\medskip

Résoudre dans $\mathbb{R}$ les équations ou systèmes suivants :
\begin{multicols}{2}
\begin{enumerate}
\item $e^{2x} - 2e^{x} + 1 = 0$\\
\item $\ln(4x^2) = 2$\\
\item $\cos\left ( 3x -\dfrac{\pi}{2}\right)=\dfrac{1}{2}$\\
\item $\cos (7x)+\sin(7x)=1$\\
\end{enumerate}
\end{multicols}



\medskip
{\color{blue}\underline{\bf Solution}\\


\begin{enumerate}
\item $e^{2x} - 2e^{x} + 1 = 0$
L'ensemble de définition est $\mathbb R$.\\
On pose \( X = e^x \), avec \( X > 0 \).  
L'équation devient : \( X^2 - 2X + 1 = 0 \iff (X - 1)^2 = 0 \iff X = 1 \)

Donc \( e^x = 1 \iff x = \ln(1) = 0 \)

\textbf{Ensemble solution :} \( \boxed{\{ 0\}} \)

\item \( \ln(4x^2) = 2 \)
L'ensemble de définition est $\mathbb R$.\\
On utilise la propriété \( \ln(ab) = \ln a + \ln b \) et \( \ln(e^2) = 2 \) :
\[
\ln(4x^2) = \ln 4 + \ln(x^2) 
\]


\[
 \ln(x^2) = \ln\left( \dfrac{e^2}{4} \right) \iff |x| = \dfrac{e}{2}
\]

\textbf{Ensemble solution :} \( \boxed{\left \{  -\dfrac{e}{2};\dfrac{e}{2}\right\}}  \)

\item \( \cos\left(3x - \dfrac{\pi}{2}\right) = \dfrac{1}{2} \)

On utilise la formule \( \cos\left(a - \dfrac{\pi}{2}\right) = \sin(a) \) :

\[
 \sin(3x) = \dfrac{1}{2}
\iff 3x = \dfrac{\pi}{6} + 2k\pi \quad \text{ou} \quad 3x = \dfrac{5\pi}{6} + 2k\pi
\]

\[
\iff x = \dfrac{\pi}{18} + \dfrac{2k\pi}{3} \quad \text{ou} \quad x = \dfrac{5\pi}{18} + \dfrac{2k\pi}{3}
\quad \text{avec } k \in \mathbb{Z}
\]

\textbf{Ensemble solution :}
\[
\boxed{x = \dfrac{\pi}{18} + \dfrac{2k\pi}{3} \quad \text{ou} \quad x = \dfrac{5\pi}{18} + \dfrac{2k\pi}{3},\; k \in \mathbb{Z}}
\]

\item \( \cos(7x) + \sin(7x) = 1 \)
L'ensemble de définition est $\mathbb R$.\\

On cherche à écrire cette somme sous la forme :
\[
\cos(7x) + \sin(7x) = R\cos(7x - \alpha)
\]

où \( R = \sqrt{a^2 + b^2} \) avec \( a = 1 \), \( b = 1 \), donc :
\[
R = \sqrt{1^2 + 1^2} = \sqrt{2}, \quad \cos \alpha = \dfrac{1}{\sqrt{2}}, \text{ et } \sin \alpha = \dfrac{1}{\sqrt{2}} \iff \alpha = \dfrac{\pi}{4}
\]

On a donc :
\[
\cos(7x) + \sin(7x) = \sqrt{2} \cos\left(7x - \dfrac{\pi}{4} \right)
\]

L'équation devient alors :
\[
\sqrt{2} \cos\left(7x - \dfrac{\pi}{4} \right) = 1 \iff
\cos\left(7x - \dfrac{\pi}{4} \right) = \dfrac{1}{\sqrt{2}}
\]

On résout :
\[
7x - \dfrac{\pi}{4} =  \dfrac{\pi}{4} + 2k\pi \text{ ou } 7x - \dfrac{\pi}{4} = - \dfrac{\pi}{4} + 2k\pi
\]

Deux cas :
\begin{itemize}
  \item \( 7x - \dfrac{\pi}{4} = \dfrac{\pi}{4} + 2k\pi \iff 7x = \dfrac{\pi}{2} + 2k\pi \iff x = \dfrac{\pi}{14} + \dfrac{2k\pi}{7} \)
  \item \( 7x - \dfrac{\pi}{4} = -\dfrac{\pi}{4} + 2k\pi \iff 7x = 2k\pi \iff x = \dfrac{2k\pi}{7} \)
\end{itemize}

\textbf{Solution :}  
\[
\boxed{x = \dfrac{2k\pi}{7} \quad \text{ou} \quad x = \dfrac{\pi}{14} + \dfrac{2k\pi}{7}, \quad k \in \mathbb{Z}}
\]


\end{enumerate}

}

\hfill\break
\hrule
\hfill\break


%Pour chaque fonction, préciser la nature des asymptotes (verticale, horizontale ou oblique) et tracer une ébauche du graphe de la fonction.

%
%Les courbes ne sont fournies qu'\`a titre indicatif afin d'orienter votre recherche et de valider vos r\'esultats.\\
%%
%\begin{minipage}{0.3\linewidth}
%$$f:x\mapsto \dfrac{x\sqrt x}{x^2+1}$$
%\end{minipage}
%\begin{minipage}{0.7\linewidth}
%\begin{center}
%\includegraphics[width=0.75\linewidth]{f2.png}
%\end{center}
%\end{minipage}
%%
%\hfill\break
%\hfill\break
%
%\begin{minipage}{0.3\linewidth}
%$$g:x\mapsto \dfrac{1}{2}x+2+\ln\left(\dfrac{x-1}{x+1}\right)$$
%\end{minipage}
%\begin{minipage}{0.7\linewidth}
%\begin{center}
%\includegraphics[width=0.75\linewidth]{f1.png}
%\end{center}
%\end{minipage}
%%
%
%\hfill\break
%\hfill\break
%
%\begin{minipage}{0.3\linewidth}
%$$h:x\mapsto (x+ 1)^2e^{-x}$$
%\end{minipage}
%\begin{minipage}{0.7\linewidth}
%\begin{center}
%\includegraphics[width=0.75\linewidth]{f3.png}
%\end{center}
%\end{minipage}
%%


\underline {\ding{110}\, \bf{Exercice 2 : Réponse d’un capteur à un changement de température (6 points)}}\\

Un capteur de température est initialement à l'équilibre thermique dans un environnement à $20^\circ$C. À l'instant $t = 0$, il est brusquement plongé dans un liquide maintenu à $80^\circ$C. 

On modélise la température $T(t)$ indiquée par le capteur à l'instant $t$ (en secondes) par la fonction :
\[
T(t) = 80 - 60 e^{-t/5}
\]
où $T(t)$ est exprimée en degrés Celsius.\\

\begin{enumerate}
  \item Quelle est la température initiale lue par le capteur ? Et la température à long terme ?\\
  \item À quelle date le capteur affiche-t-il une température de $50^\circ$C ?\\
  \item Déterminer la dérivée $T'(t)$, puis montrer que :
  \[
  T'(t) = \frac{12}{e^{t/5}}
  \]
  \item Interpréter le signe de $T'(t)$. Que peut-on dire de l’évolution de la température ?\\
  \item Déterminer le temps mis par le capteur pour atteindre $95\%$ de la température finale.\\
  %\item Quel est le temps caractéristique de ce système ? Justifier à partir de l'expression donnée.
\end{enumerate}



\medskip
{\color{blue}\underline{\bf Solution}\\

\begin{enumerate}
  \item Température initiale : 
  \[
  T(0) = 80 - 60e^{0} = 80 - 60 = 20^\circ\text{C}
  \]
  Température à long terme :
  \[
  \lim_{t \to +\infty} (T(t)) =\lim_{t \to +\infty} ( 80 - 60e^{-t/5})= 80
  \]

  \item On résout : \( T(t) = 50 \)\\
  \[
  50 = 80 - 60e^{-t/5} \iff 60e^{-t/5} = 30 \iff e^{-t/5} = \frac{1}{2}
  \iff t = 5\ln(2) 
  \]

  \item Dérivée de $T$, fonction dérivable sur $\mathbb R$ comme somme de fonctions dérivables:\\
  \[\forall t \in \mathbb R, \quad   T'(t) = -60 \cdot \left( \frac{d}{dt} e^{-t/5} \right) = -60 \cdot \left( -\frac{1}{5} e^{-t/5} \right)
  = \frac{60}{5} e^{-t/5} = 12e^{-t/5}
  \]
  Montrons :
  \[
  \boxed{T'(t) = \frac{12}{e^{t/5}}}
  \quad \text{car} \quad e^{-t/5} = \frac{1}{e^{t/5}}
  \]

  \item \( T'(t) > 0 \) pour tout \( t \geq 0 \) car une exponentielle réelle est toujours positive.\\
  Donc \( T \) est strictement croissante : la température indiquée par le capteur augmente.

  \item Température finale : \( T_f = 80^\circ\text{C} \). On cherche \( t \) tel que :
  \[
  T(t) = 0{,}95 \times 80 = 76
  \iff 76 = 80 - 60e^{-t/5} \iff 60e^{-t/5} = 4 \iff e^{-t/5} = \frac{1}{15}
  \iff t = -5 \ln\left( \frac{1}{15} \right)
  = 5 \ln(15) 
  \]
\end{enumerate}

}

\hfill\break
\hrule
\hfill\break



\underline {\ding{110}\, \bf{Exercice 3 : } Logique, sommes et produits (10 points)}\\

\begin{enumerate}
  \item Donner la table de vérité de la proposition logique :
  \[
  (P \Rightarrow Q)
  \]
  \item Donner la table de vérité de la proposition logique :
  \[
  (P \Rightarrow Q) \land (\lnot Q \Rightarrow \lnot P)
  \]
  \item En déduire une équivalence logique.\\
    \item Démontrer par récurrence que :
  \[
 (*) \qquad \sum_{k=1}^{n} k = \frac{n(n+1)}{2}
  \]
  \item Calculer les expressions suivantes :
  \begin{itemize}
    \item $\displaystyle \sum_{k=1}^{n} \ln\left(1+\dfrac{1}{k}\right)$
    \item $\displaystyle \prod_{k=1}^{n} e^{k-1}$, on pourra utiliser la relation $(*)$.\\
  \end{itemize}

\end{enumerate}


\medskip
{\color{blue}\underline{\bf Solution}\\

\begin{enumerate}
  
  \item Table de vérité de \( (P \Rightarrow Q) \) :

  \[
  \begin{array}{|c|c||c|}
  \hline
  P & Q & P \Rightarrow Q \\
  \hline
  V & V & V \\
  V & F & F \\
  F & V & V \\
  F & F & V \\
  \hline
  \end{array}
  \]

  \item Table de vérité de \( (P \Rightarrow Q) \land (\lnot Q \Rightarrow \lnot P) \) :

  \[
  \begin{array}{|c|c||c|c|c|}
  \hline
  P & Q & P \Rightarrow Q & \lnot Q \Rightarrow \lnot P & \text{Conjonction} \\
  \hline
  V & V & V & V & V \\
  V & F & F & F & F \\
  F & V & V & V & V \\
  F & F & V & V & V \\
  \hline
  \end{array}
  \]

  \item On en déduit :
  \[
  (P \Rightarrow Q) \Leftrightarrow (\lnot Q \Rightarrow \lnot P)
  \quad \text{(contraposée)}
  \]

  \item Montrons par récurrence que  pour tout entier naturel $n$ non nul
  \[
  P_n :  \sum_{k=1}^n k = \frac{n(n+1)}{2}
  \]
est vraie.\\

  \textbf{Initialisation :} pour \( n = 1 \), on a 
  \[
   \sum_{k=1}^1 k = 1 \text{ et }  \frac{1(1+1)}{2} = 1 
   \] : $P_1$ est vraie.\\.

  \textbf{Hérédité :} supposons la formule vraie au rang \( n \), montrons-la au rang \( n+1 \) :
  \[
  \sum_{k=1}^{n+1} k = \left( \sum_{k=1}^{n} k \right) + (n+1)
  = \frac{n(n+1)}{2} + (n+1) = \frac{n(n+1) + 2(n+1)}{2}
  = \frac{(n+1)(n+2)}{2}
  \]

  Conclusion : la propriété est vraie pour tout \( n \in \mathbb{N}^* \) par récurrence.

  \item \textbf{Calculs :}

  \begin{itemize}
    \item
    \[
    \sum_{k=1}^{n} \ln\left(1 + \frac{1}{k} \right)
    = \sum_{k=1}^{n} \ln\left( \frac{k+1}{k} \right)
    = \sum_{k=1}^{n} \left[ \ln(k+1) - \ln(k) \right]
    \]
    \[
    = \ln(n+1) - \ln(1) = \ln(n+1)
    \]

    \item
    \[
    \prod_{k=1}^{n} e^{k-1}
    = e^{\sum_{k=1}^{n} (k-1)} = e^{\sum_{k=0}^{n-1} k}
    = e^{\frac{(n-1)n}{2}} \quad \text{(formule (*) utilisée)}
    \]
  \end{itemize}
\end{enumerate}

  

}
\hfill\break
\hrule
\hfill\break

\underline {\ding{110}\, \bf{Exercice 4 : }Vecteurs et géométrie plane (11 points)}\\

On considère dans le plan muni d’un repère orthonormé les points : $A(1,2,0)$, $B(0,4,6)$, $C(7,-1,2)$.\\

\begin{enumerate}
  \item Calculer les coordonnées des vecteurs $\overrightarrow{AB}$ et $\overrightarrow{AC}$.\\
  \item Calculer les longueurs $AB$ et $AC$.\\
  \item Calculer le produit scalaire entre les vecteurs $\overrightarrow{AB}$ et $\overrightarrow{AC}$ et le produit vectoriel vecteurs $\overrightarrow{AB}$ et $\overrightarrow{AC}$.\\
  \item En déduire le cosinus et le sinus de l'angle $\big(\overrightarrow{AB},\overrightarrow{AC}\big)$.\\
  \item Donner une équation paramétrique de la droite $(AB)$.\\
  \item Déterminer la distance du point $C$ à la droite $(AB)$.\\
\end{enumerate}


\medskip
{\color{blue}\underline{\bf Solution}\\
\begin{enumerate}
  \item Calcul des vecteurs :
  \[
  \overrightarrow{AB} : (0 - 1,\ 4 - 2,\ 6 - 0) = (-1,\ 2,\ 6)
  \]
  \[
  \overrightarrow{AC}  : (7 - 1,\ -1 - 2,\ 2 - 0) = (6,\ -3,\ 2)
  \]

  \item Longueurs :
  \[
  AB = \| \overrightarrow{AB} \| = \sqrt{(-1)^2 + (2)^2 + (6)^2} = \sqrt{1 + 4 + 36} = \sqrt{41}
  \]
  \[
  AC = \| \overrightarrow{AC} \| = \sqrt{(6)^2 + (-3)^2 + (2)^2} = \sqrt{36 + 9 + 4} = \sqrt{49} = 7
  \]

  \item Produit scalaire :
  \[
  \overrightarrow{AB} \cdot \overrightarrow{AC} = (-1)(6) + 2(-3) + 6(2)  = 0
  \]

  Produit vectoriel :
  \[
  \overrightarrow{AB} \wedge \overrightarrow{AC} 
  = (2 \cdot 2 - 6 \cdot (-3)) \vec{i} - (-1 \cdot 2 - 6 \cdot 6) \vec{j} + (-1 \cdot (-3) - 2 \cdot 6) \vec{k}  = 22 \vec{i} + 38 \vec{j} - 9 \vec{k}
  \]

  \item Angle entre les vecteurs :

  Comme le produit scalaire est nul, on a :
  \[
  \cos(\theta) = 0 \iff \theta = \frac{\pi}{2}, \quad \text{donc } \sin(\theta) = 1
  \]

  \item Équation paramétrique de la droite $(AB)$ :

  Un point : \( A(1,2,0) \), un vecteur directeur \( \overrightarrow{AB} = (-1,2,6) \)

  \[
  \begin{cases}
  x = 1 - t \\
  y = 2 + 2t \\
  z = 6t
  \end{cases},\quad t \in \mathbb{R}
  \]

  \item Distance du point \( C \) à la droite \( (AB) \) :

  \[
  d(C,AB) = \frac{ \| \overrightarrow{AB} \wedge \overrightarrow{AC} \| }{ \| \overrightarrow{AB} \| }
  = \frac{ \sqrt{22^2 + 38^2 + (-9)^2} }{ \sqrt{41} }
  = \frac{ \sqrt{484 + 1444 + 81} }{ \sqrt{41} }
  = \frac{ \sqrt{2009} }{ \sqrt{41} } = \sqrt{ \frac{2009}{41} }
  \]
\end{enumerate}


}
\end{document}
