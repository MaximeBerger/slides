\documentclass[10pt]{article}

\usepackage[utf8]{inputenc}
\usepackage[french]{babel}
\usepackage{graphicx}
%\usepackage{wrapfig}
\usepackage{dsfont}
\usepackage[margin=2cm]{geometry}
\usepackage{amsmath,amssymb}
%\usepackage{tikz}
\usepackage{multicol}
\newcommand{\xB}{{\cal B}}
\newcommand{\xC}{{\cal C}}
\newcommand{\R}{\mathbf{R}}

\begin{document}

\pagestyle{empty}

\noindent
\begin{minipage}[l]{8cm}
  \scriptsize{ESTP, S1\\Ann\'ee universitaire 2024/2025}
\end{minipage}

\begin{center}
  {\large\textbf{Petit test } \\
10 minutes\\}
  \bigskip
  

 
\end{center}

\bigskip

\begin{center}
  \fbox{%
    \begin{minipage}{0.95\linewidth}
      NOM: \hspace*{5cm} Pr\'enom: \hspace*{4cm} Groupe: TD 
    \end{minipage}
  }
  \end{center}

\bigskip
 
 

\noindent\textbf{Exercice  : 5 points}\\

Considérons les nombres comples $z_1= 1+i$, $z_2=-4i$, $z_3=6i$ et $z_4=1-i\sqrt{3}$.\\


\begin{enumerate}
    \item 
    Calculer les conjugués de $z_1$, $z_2$, $z_3$ et $z_4$.
    \item
    Calculer $z_1\times z_4$.
    \item
    Donner la forme algébrique de $\dfrac{z_1}{z_4}$.
\end{enumerate}

\end{document}
