\documentclass[../PolyS2.tex]{subfiles}
\begin{document}
\setcounter{chapter}{6}
%%%%%%%%%%%%%%%%%%%%%%%%%%%%%%%%%%%%%%%%%%%%%%%%%%%%%%%%%
% FONCTIONS DE PLUSIEURS VARIABLES
%%%%%%%%%%%%%%%%%%%%%%%%%%%%%%%%%%%%%%%%%%%%%%%%%%%%%%%%%
\chapter{Fonctions de plusieurs variables}

\section{Définitions et représentation graphique}

Dans tout le chapitre, $D$ désigne un sous-ensemble de $\R^2$.

\subsection{Fonction de deux variables}

\begin{Def}[Fonction de deux variables]
On appelle \textbf{fonction de deux variables} toute application $f$ d'un sous-ensemble $D$ de $\R^2$ dans $\R$. $D$ est appelé \textbf{domaine de définition} de $f$.
\end{Def}

\begin{Ex}
Déterminer le domaine de définition des fonctions suivantes :
\begin{enumerate}
\item $f_1:(x,y)\mapsto \sqrt{9-x^2-y^2}$

\vspace{1em}

\textbf{Défini} si $9-x^2-y^2\geq 0$, c'est-à-dire $x^2+y^2\leq 9$ : le domaine de définition est le \textbf{disque} de centre O et de rayon 3.

\item $f_2:(x,y)\mapsto \dfrac{1}{x+y}$

\vspace{1em}

\textbf{Défini} si $x+y\neq 0$ : le domaine de définition est le plan privé de la \textbf{droite} d'équation $x+y=0$.

\item $f_3:(x,y)\mapsto \ln(x^2+y^2)$

\vspace{1em}

\textbf{Défini} si $x^2+y^2>0$ : le domaine de définition est le plan privé du \textbf{point} $O(0,0)$.
\end{enumerate}
\end{Ex}

\subsection{Lignes de niveau}

\begin{Def}[Ligne de niveau]
Soit $f\in\mathcal F(D,\mathbb R)$, et $k\in \mathbb R$. On appelle \textbf{ligne de niveau $k$} de $f$ le sous-ensemble de $\R^2$ défini par :
$$f^{-1}(k)=\{(x,y)\in D,\ \ f(x,y)=k\}$$
\end{Def}

\begin{Rmq}
Les lignes de niveau sont les \og courbes \fg{} obtenues en coupant la surface $z=f(x,y)$ par des plans horizontaux $z=k$.
\end{Rmq}

\begin{center}
\includegraphics[scale=0.5]{images/coupe.jpg} 
\end{center}

\begin{Ex}
Quelques exemples de fonctions et leurs lignes de niveau :
\end{Ex}

\begin{multicols}{2}
\paragraph{Fonction distance :}
$$f_1:(x,y)\mapsto x^2+y^2$$
Les lignes de niveau sont des \textbf{cercles} centrés en l'origine.
\begin{center}
\includegraphics[width=.6\linewidth]{images/ls-dist2.png}   
\end{center}

\paragraph{Fonction unidimensionnelle :}
$$f_2:(x,y)\mapsto x^3$$
Les lignes de niveau sont des \textbf{droites verticales}.
\begin{center}
\includegraphics[width=.6\linewidth]{images/ls-unid.png}   
\end{center}
\end{multicols}

\begin{multicols}{2}
\paragraph{Point selle :}
$$f_3:(x,y)\mapsto (x+y)^2-(x-y)^2=4xy$$
Les lignes de niveau sont des \textbf{hyperboles}.
\begin{center}
\includegraphics[width=.6\linewidth]{images/ls-selle.png}   
\end{center}

\paragraph{Polynôme de degré 2 :}
$$f_4:(x,y)\mapsto 2x-3y+3xy-x^2$$
\begin{center}
\includegraphics[width=.6\linewidth]{images/ls-poly2.png}   
\end{center}
\end{multicols}

\subsection*{QCM}

\begin{enumerate}
\item Une fonction de deux variables est une application :
\begin{enumerate}
\item de $\R$ dans $\R^2$
\item de $\R^2$ dans $\R$
\item de $\R^3$ dans $\R$
\item de $\R^2$ dans $\R^2$
\end{enumerate}

\item Les lignes de niveau de la fonction $f(x,y)=x^2+y^2$ sont :
\begin{enumerate}
\item des droites
\item des paraboles
\item des cercles
\item des hyperboles
\end{enumerate}

\item Géométriquement, une ligne de niveau correspond à :
\begin{enumerate}
\item une projection sur le plan $(x,y)$
\item l'intersection de la surface $z=f(x,y)$ avec un plan vertical
\item l'intersection de la surface $z=f(x,y)$ avec un plan horizontal
\item une coupe selon l'axe $Oz$
\end{enumerate}
\end{enumerate}


%%%%%%%%%%%%%%%%%%%%%%%%%%%%%%%%%%%%%%%%%%%%%%%%%%%%%%%%%
\section{Limite et continuité}

\subsection{Limite d'une fonction de deux variables}

\begin{Def}[Limite en un point]
Soit $f\in\mathcal F(D,\mathbb R)$ et $(a,b)\in D$. On définit la limite de $f$ au point $(a,b)$, et on écrit :
$$\Lim{(a,b)}{f(x,y)}=L$$ 
si les valeurs de $f(x,y)$ sont aussi proches que l'on veut de $L$ en choisissant $(x,y)$ suffisamment proche de $(a,b)$, sans y être égal.
\end{Def}

\begin{Rmq}
Ce qui est en jeu dans cette définition c'est \textbf{la distance entre $f(x,y)$ et $L$ en fonction de la distance entre $(x,y)$ et $(a,b)$}. Autrement dit, $f(x,y)$ doit tendre vers $L$, \textbf{quelque soit le chemin d'approche autour de $(a,b)$}. Sinon la limite n'existe pas.
\end{Rmq}

\begin{center}
\includegraphics[]{images/chemins.jpg}  
\end{center}

\begin{Meth}[Montrer qu'une limite n'existe pas]
Pour montrer qu'une limite n'existe pas, il suffit de trouver \textbf{deux chemins d'approche} donnant des limites différentes.
\end{Meth}

\begin{Ex}
Soit $f : (x,y)\mapsto \dfrac{x^2-y^2}{x^2+y^2}$ définie pour $(x,y)\neq(0,0)$. 

La limite $\Lim{(0,0)}{f(x,y)}$ existe-t-elle ?

\begin{itemize}
\item \textbf{Chemin $C_1$} selon l'axe $Ox$ : 
$$f(x,0)=\dfrac{x^2}{x^2}=1\quad (x\neq0)\quad\Rightarrow\quad f(x,y)\underset{(x,0) \rightarrow (0,0)}{\longrightarrow} 1$$

\item \textbf{Chemin $C_2$} selon l'axe $Oy$ : 
$$f(0,y)=\dfrac{-y^2}{y^2}=-1\quad (y\neq0)\quad\Rightarrow\quad f(x,y)\underset{(0,y) \rightarrow (0,0)}{\longrightarrow} -1$$

\item \textbf{Conclusion :} Les deux chemins donnent des limites différentes ($1\neq -1$), donc $f$ \textbf{n'a pas de limite} en $(0,0)$.
\end{itemize}
\end{Ex}

\begin{Ex}
Soit $f : (x,y)\mapsto \dfrac{xy}{x^2+y^2}$ définie pour $(x,y)\neq(0,0)$. 

La limite $\Lim{(0,0)}{f(x,y)}$ existe-t-elle ?

\begin{itemize}
\item \textbf{Chemin $C_1$} selon l'axe $Ox$ : 
$$f(x,0)=0\quad (x\neq0)\quad\Rightarrow\quad f(x,y)\underset{(x,0) \rightarrow (0,0)}{\longrightarrow} 0$$

\item \textbf{Chemin $C_2$} selon la droite $y=x$ : 
$$f(x,x)=\dfrac{x^2}{2x^2}=\dfrac{1}{2}\quad (x\neq0)\quad\Rightarrow\quad f(x,y)\underset{(x,x) \rightarrow (0,0)}{\longrightarrow} \dfrac{1}{2}$$

\item \textbf{Conclusion :} $f$ \textbf{n'a pas de limite} en $(0,0)$.
\end{itemize}
\end{Ex}

\begin{Meth}[Montrer qu'une limite existe]
Pour montrer qu'une limite existe (et vaut $L$), on peut utiliser :
\begin{enumerate}
\item Le \textbf{théorème d'encadrement} : montrer que $|f(x,y)-L|\leq g(x,y)$ avec $g(x,y)\to 0$
\item Le passage en \textbf{coordonnées polaires} : $x=r\cos\theta$, $y=r\sin\theta$
\end{enumerate}
\end{Meth}

\begin{Ex}
Soit $f : (x,y)\mapsto \dfrac{3x^2y}{x^2+y^2}$ définie pour $(x,y)\neq(0,0)$. 

La limite $\Lim{(0,0)}{f(x,y)}$ existe-t-elle ?

\begin{itemize}
\item Testons quelques chemins :
\begin{itemize}
\item Droite $y=mx$ : $f(x,mx)=\dfrac{3mx}{1+m^2}\to 0$
\item Courbe $y=x^2$ : $f(x,x^2)=\dfrac{3x^2}{1+x^2}\to 0$
\item Courbe $x=y^2$ : $f(y^2,y)=\dfrac{3y^3}{y^2+1}\to 0$
\end{itemize}

\item Il semble que la limite soit 0... mais il faut le démontrer \textbf{rigoureusement}.

\item \textbf{Démonstration par encadrement :}
$$\abs{f(x,y)-0}=\abs{\dfrac{3x^2y}{x^2+y^2}}=\dfrac{3x^2\abs{y}}{x^2+y^2}$$
Comme $x^2\leq x^2+y^2$, on a $\dfrac{x^2}{x^2+y^2}\leq 1$, donc :
$$0\leq\abs{f(x,y)}\leq 3\abs{y}$$
Comme $\Lim{(0,0)}{3\abs{y}}=0$, par le théorème d'encadrement : $\Lim{(0,0)}{f(x,y)}=0$.
\end{itemize}

\begin{center}
\includegraphics[scale=0.25]{images/courbe3D_1}
\end{center}
\end{Ex}

\subsection{Continuité}

\begin{Def}[Continuité en un point]
Soit $f\in \mathcal F (D,\R)$ et $(a,b)\in D$. On dit que $f$ est \textbf{continue} au point $(a,b)$ si :
$$\Lim{(a,b)}{f(x,y)}=f(a,b)$$
\end{Def}

\begin{Prop}
\begin{itemize}
\item Les fonctions \textbf{polynômes} sont continues sur $\R^2$.
\item Les fonctions \textbf{rationnelles} sont continues sur leur domaine de définition.
\item Les \textbf{composées} de fonctions continues sont continues.
\end{itemize}
\end{Prop}

\begin{Ex}
Soit $f$ définie par :
$$f(x,y)=\begin{cases} \dfrac{x^2-y^2}{x^2+y^2} & \text{si } (x,y)\neq (0,0)\\0 & \text{sinon}\end{cases}$$
D'après l'exemple précédent, $f$ n'a pas de limite en $(0,0)$, donc $f$ est \textbf{discontinue} en $(0,0)$.
\end{Ex}

\begin{Ex}
Soit $f$ définie par :
$$f(x,y)=\begin{cases} \dfrac{3x^2y}{x^2+y^2} & \text{si } (x,y)\neq (0,0)\\0 & \text{sinon}\end{cases}$$

\begin{itemize}
\item $f$ est continue sur $\R^2 \setminus \{(0,0)\}$ car c'est une fonction rationnelle.
\item On a montré que $\Lim{(0,0)}{f(x,y)}=0=f(0,0)$, donc $f$ est continue en $(0,0)$.
\item \textbf{Conclusion :} $f$ est continue sur $\R^2$.
\end{itemize}
\end{Ex}
\subsection*{QCM}

\begin{enumerate}
\item Une limite $\Lim{(a,b)}{f(x,y)}$ existe si :
\begin{enumerate}
\item la limite existe le long d'un chemin
\item toutes les limites le long des chemins sont égales
\item $f$ est continue en $(a,b)$
\item $f(a,b)$ est définie
\end{enumerate}

\item Pour montrer qu'une limite n'existe pas, il suffit de :
\begin{enumerate}
\item montrer que $f$ est discontinue
\item utiliser les coordonnées polaires
\item trouver deux chemins donnant des limites différentes
\item calculer une dérivée partielle
\end{enumerate}

\item Une fonction est continue en $(a,b)$ si :
\begin{enumerate}
\item elle est dérivable en $(a,b)$
\item $\Lim{(a,b)}{f(x,y)}$ existe
\item $f(a,b)=0$
\item $\Lim{(a,b)}{f(x,y)}=f(a,b)$
\end{enumerate}
\end{enumerate}

%%%%%%%%%%%%%%%%%%%%%%%%%%%%%%%%%%%%%%%%%%%%%%%%%%%%%%%%%
\section{Dérivées partielles}

\subsection{Définition}

\begin{Def}[Applications partielles]
Soit $f\in \mathcal F (D,\R)$ et $(a,b)\in D$. On définit les \textbf{applications partielles} associées à $f$ en $(a,b)$ :
$$f_{.,b}:x\mapsto f(x,b) \quad\text{et}\quad f_{a,.}:y\mapsto f(a,y)$$
\end{Def}

\begin{Rmq}
Les applications partielles sont des fonctions \textbf{d'une seule variable}.
\end{Rmq}

\begin{Def}[Dérivées partielles]
Soit $f\in \mathcal F(D,\R)$ et $(a,b)\in D$. 

\begin{itemize}
\item Si $f_{.,b}$ est dérivable en $a$, on appelle \textbf{dérivée partielle de $f$ par rapport à $x$} en $(a,b)$ :
$$\frac{\partial f}{\partial x}(a,b)=f'_x(a,b)=\limx{a}\dfrac{f(x,b)-f(a,b)}{x-a}$$

\item Si $f_{a,.}$ est dérivable en $b$, on appelle \textbf{dérivée partielle de $f$ par rapport à $y$} en $(a,b)$ :
$$\frac{\partial f}{\partial y}(a,b)=f'_y(a,b)=\limy{b}\dfrac{f(a,y)-f(a,b)}{y-b}$$
\end{itemize}
\end{Def}

\begin{Rmq}
En pratique, pour calculer $\dfrac{\partial f}{\partial x}$ :
\begin{itemize}
\item On considère $y$ comme une \textbf{constante}
\item On dérive par rapport à $x$ comme une fonction d'une variable
\end{itemize}
\end{Rmq}

\exo{\textbf{Dérivées partielles}}

Calculer les dérivées partielles des fonctions suivantes :
\begin{multicols}{2}
\begin{enumerate}
\item $f(x,y)=x^2+3xy-y^2$
\item $g(x,y)=\ln (x+y)+e^{x^2+2y}$
\item $h(x,y)=e^x\cos y$
\item $k(x,y)=(x^2-y^3)\cos(x-y)$
\item $l(x,y)=xy^2\ln(x^2+y)$
\end{enumerate}
\end{multicols}

\subsection{Interprétation géométrique}

\begin{center}
\includegraphics[scale=0.1]{images/PlanCoupeSurface}  
\end{center}

\begin{Prop}
\textbf{Interprétation géométrique.} L'ensemble des triplets $(x,y,z)$ où $z=f(x,y)$ est une \textbf{surface} de $\R^3$.

\begin{itemize}
\item \textbf{Fixer $y=b$} revient à couper la surface par le plan vertical $y=b$. La trace est la courbe $C_1 : z=f(x,b)$.

$\Rightarrow$ $\dfrac{\partial f}{\partial x}(a,b)$ est la \textbf{pente de la tangente} à $C_1$ au point $(a,b)$.

\item \textbf{Fixer $x=a$} revient à couper la surface par le plan vertical $x=a$. La trace est la courbe $C_2 : z=f(a,y)$.

$\Rightarrow$ $\dfrac{\partial f}{\partial y}(a,b)$ est la \textbf{pente de la tangente} à $C_2$ au point $(a,b)$.
\end{itemize}
\end{Prop}

\begin{Ex}
Soit $f(x,y)=4-x^2-2y^2$.

\textbf{Fixons $y=1$ :}
\begin{itemize}
\item La trace du plan $y=1$ sur la surface est la courbe $C_1 : z=f(x,1)=2-x^2$
\item $\dfrac{\partial f}{\partial x}(1,1)=-2x\big|_{x=1}=-2$ : pente de la tangente à $C_1$ au point $(1,1)$
\end{itemize}

\textbf{Fixons $x=1$ :}
\begin{itemize}
\item La trace du plan $x=1$ sur la surface est la courbe $C_2 : z=f(1,y)=3-2y^2$
\item $\dfrac{\partial f}{\partial y}(1,1)=-4y\big|_{y=1}=-4$ : pente de la tangente à $C_2$ au point $(1,1)$
\end{itemize}
\end{Ex}

\subsection{Plan tangent}

\begin{Thm}[Plan tangent]
Soit $S : z=f(x,y)$ une surface et $M(a,b,c)$ un point de $S$ où les dérivées partielles de $f$ sont continues.

Une équation du \textbf{plan tangent} à $S$ en $M$ est :
$$z-c=\frac{\partial f}{\partial x}(a,b)(x-a)+\frac{\partial f}{\partial y}(a,b)(y-b)$$
\end{Thm}

\begin{center}
\includegraphics[scale=0.12]{images/PlanTangent2}
\end{center}

\exo{\textbf{Plan tangent}}

\begin{enumerate}
\item Déterminer le plan tangent en $A(0,0,3)$ à la surface $S : z=\sqrt{9-x^2-y^2}$.
\item Déterminer le plan tangent en $A(1,1,3)$ au paraboloïde elliptique $S : z=2x^2+y^2$.
\end{enumerate}

%%%%%%%%%%%%%%%%%%%%%%%%%%%%%%%%%%%%%%%%%%%%%%%%%%%%%%%%%
\section{Différentielle totale}

\subsection{Définition}

\begin{Def}[Différentielle totale]
Soit $f$ une fonction de deux variables différentiable. La \textbf{différentielle totale} de $f$ est :
$$df=\dfrac{\partial f}{\partial x}dx+\dfrac{\partial f}{\partial y}dy$$
\end{Def}

\begin{Rmq}
La différentielle représente la \textbf{variation instantanée} d'une fonction pour une variation infinitésimale conjointe de $x$ et $y$.
\end{Rmq}

\exo{\textbf{Différentielle totale}}

Calculer la différentielle totale des fonctions suivantes :
\begin{enumerate}
\item $z=8x^2y^3-4xy^3+3xy^2+10$
\item $z=\dfrac{x}{x+y}$
\item $z=\dfrac{\cos (xy)}{xy^2}$
\end{enumerate}

\subsection{Approximation de la variation}

\begin{Prop}
\textbf{Approximation de la variation.} La différentielle permet d'approximer la variation d'une fonction pour de petits changements des variables :
$$\Delta f\approx\dfrac{\partial f}{\partial x}\Delta x+\dfrac{\partial f}{\partial y}\Delta y$$
\end{Prop}

\begin{Rmq}
Plus $\Delta x$ et $\Delta y$ sont petits, plus l'approximation est précise.
\end{Rmq}

\begin{Ex}
Calculer la variation de l'aire d'un rectangle si la longueur passe de $w = 5$ m à $w = 5,05$ m et si la hauteur passe de $h = 3$ m à $h = 3,02$ m.

L'aire est $A=wh$, donc :
$$\Delta A\approx\dfrac{\partial A}{\partial w}\Delta w+\dfrac{\partial A}{\partial h}\Delta h=h\Delta w+w\Delta h=3\times0,05+5\times0,02=0,25\text{ m}^2$$
\end{Ex}

\subsection{Calcul d'incertitude}

\begin{Prop}
\textbf{Calcul d'incertitude.} Pour une fonction $f(x,y)$ avec $x=x_0\pm\Delta x$ et $y=y_0\pm\Delta y$, l'incertitude sur $f$ est :
$$\Delta f=\abs{\dfrac{\partial f}{\partial x}\Delta x}+\abs{\dfrac{\partial f}{\partial y}\Delta y}$$
\end{Prop}

\begin{Ex}
La période d'oscillation d'un pendule est $T=2\pi\sqrt{\dfrac{l}{g}}$ où $l$ est la longueur et $g$ l'accélération gravitationnelle.

Calculer l'incertitude sur $T$ si $l = 126,6 \pm 0,3$ cm et $g = 9,81 \pm 0,01$ m/s$^2$.

On calcule :
$$\dfrac{\partial T}{\partial l}=\pi\sqrt{\dfrac{1}{gl}}\quad\text{et}\quad\dfrac{\partial T}{\partial g}=-\pi\sqrt{\dfrac{l}{g^3}}$$

D'où l'incertitude $\Delta T=\abs{\dfrac{\partial T}{\partial l}\Delta l}+\abs{\dfrac{\partial T}{\partial g}\Delta g}$.
\end{Ex}


\subsection*{QCM}

\begin{enumerate}
\item La différentielle totale d'une fonction $f(x,y)$ est :
\begin{enumerate}
\item $df=f(x,y)$
\item $\partial_x f + \partial_y f$
\item $\partial_x f\,dx+\partial_y f\,dy$
\item $\Delta x + \Delta y$
\end{enumerate}

\item La formule $\Delta f \approx \partial_x f\,\Delta x + \partial_y f\,\Delta y$ est :
\begin{enumerate}
\item exacte
\item valable pour de grandes variations
\item une approximation locale
\item indépendante du point considéré
\end{enumerate}

\item Le calcul d'incertitude repose sur :
\begin{enumerate}
\item une somme algébrique
\item une somme des valeurs absolues
\item un produit scalaire
\item une intégration
\end{enumerate}
\end{enumerate}

%%%%%%%%%%%%%%%%%%%%%%%%%%%%%%%%%%%%%%%%%%%%%%%%%%%%%%%%%
\section{Gradient et divergence}

\subsection{Le gradient}

\begin{Def}[Gradient d'une fonction à deux variables]
Soit $f$ une fonction de deux variables. Le \textbf{gradient} de $f$ est le vecteur :
$$\overrightarrow{\text{grad}}\,f(x,y)=\dfrac{\partial f}{\partial x}(x,y)\, \Vec{i}+\dfrac{\partial f}{\partial y}(x,y)\, \Vec{j}$$
\end{Def}

\begin{Rmq}
Notation compacte :
$$\overrightarrow{\nabla} f=\overrightarrow{\text{grad}}\,f=\begin{pmatrix}\dfrac{\partial f}{\partial x}\\ \\\dfrac{\partial f}{\partial y}\end{pmatrix}$$
\end{Rmq}

\begin{Prop}
\textbf{Propriétés du gradient.}
\begin{enumerate}
\item Le gradient indique la \textbf{direction de plus grande croissance} de $f$.
\item La norme $\Vert\overrightarrow{\nabla} f\Vert$ donne le \textbf{taux de variation maximal}.
\item Le gradient est \textbf{perpendiculaire aux lignes de niveau}.
\end{enumerate}
\end{Prop}

\begin{Prop}
\textbf{Dérivée directionnelle.} La dérivée de $f$ dans une direction $\vec u$ (vecteur unitaire) est :
$$f'_{\vec u}(x,y)=\overrightarrow{\nabla} f(x,y)\cdot\vec{u}=\Vert\overrightarrow{\nabla} f(x,y)\Vert\cos\theta$$
où $\theta$ est l'angle entre $\overrightarrow{\nabla} f$ et $\vec{u}$.
\end{Prop}

\begin{Ex}
Soit $f(x,y)=xe^y$.
\begin{enumerate}
\item Gradient au point $P(2,0)$ : $\nabla f(2,0)=(e^y,xe^y)\big|_{(2,0)}=(1,2)$

\item Dérivée dans la direction $\vec{u}=\left(-\frac{3}{5},\frac{4}{5}\right)$ :
$$f'_{\vec u}(2,0)=\nabla f(2,0)\cdot\vec{u}=(1,2)\cdot\left(-\frac{3}{5},\frac{4}{5}\right)=1$$

\item Taux de variation maximal : $\Vert\overrightarrow{\nabla} f(2,0)\Vert=\sqrt{1^2+2^2}=\sqrt 5$
\end{enumerate}
\end{Ex}

\begin{Rmq}
\textbf{Gradient en dimension 3.} Pour une fonction $f(x,y,z)$ :
$$\overrightarrow{\text{grad}}\,f=\begin{pmatrix}\dfrac{\partial f}{\partial x}\\ \\\dfrac{\partial f}{\partial y}\\\\\dfrac{\partial f}{\partial z}\end{pmatrix}$$
\end{Rmq}

\subsection{La divergence}

\begin{Def}[Divergence d'un champ de vecteurs]
Soit $\vec{F}=(F_x,F_y,F_z)$ un champ de vecteurs. La \textbf{divergence} de $\vec{F}$ est :
$$\text{div}(\vec{F})=\dfrac{\partial F_x}{\partial x}+\dfrac{\partial F_y}{\partial y}+\dfrac{\partial F_z}{\partial z}$$
\end{Def}

\begin{Rmq}
\begin{itemize}
\item La divergence mesure le \textbf{flux par unité de volume} du champ.
\item \textbf{Divergence positive} $\Rightarrow$ flux sortant (expansion)
\item \textbf{Divergence négative} $\Rightarrow$ flux entrant (compression)
\end{itemize}
\end{Rmq}

\subsection*{QCM}

\begin{enumerate}
\item Le gradient d'une fonction indique :
\begin{enumerate}
\item une direction quelconque
\item la direction de plus forte décroissance
\item la direction de plus forte croissance
\item une direction tangentielle
\end{enumerate}

\item Le gradient est :
\begin{enumerate}
\item tangent aux lignes de niveau
\item parallèle aux lignes de niveau
\item perpendiculaire aux lignes de niveau
\item nul sur les lignes de niveau
\end{enumerate}

\item Une divergence positive signifie :
\begin{enumerate}
\item un champ conservatif
\item un flux entrant
\item une rotation locale
\item un flux sortant
\end{enumerate}
\end{enumerate}


%%%%%%%%%%%%%%%%%%%%%%%%%%%%%%%%%%%%%%%%%%%%%%%%%%%%%%%%%
\section{Dérivation des fonctions composées}

\subsection{Cas d'une variable intermédiaire}

\begin{Prop}
\textbf{Règle de la chaîne.} Soit $z=f(x,y)$ avec $x=x(t)$ et $y=y(t)$. Si $f$ admet des dérivées partielles continues et $x,y$ sont dérivables par rapport à $t$ :
$$\dfrac{dz}{dt}=\frac{\partial z}{\partial x}\dfrac{dx}{dt}+\frac{\partial z}{\partial y}\dfrac{dy}{dt}$$
\end{Prop}

\begin{Ex}
Soit $z=x^2y$, $x=\sin t$ et $y=\cos t$.
$$ \dfrac{dz}{dt}=2xy\cos t+x^2(-\sin t)=2\sin t\cos^2 t-\sin^3 t$$

\vspace{1em}

\textbf{Vérification :} $z(t)=\sin^2t\cos t$, donc $z'(t)=2\sin t\cos^2 t-\sin^3 t$ \checkmark
\end{Ex}

\begin{Ex}
\textbf{Application en thermodynamique :} La pression $P$, le volume $V$ et la température $T$ d'une mole de gaz sont liés par $PV=8,31T$.

On veut déterminer $\dfrac{dP}{dt}$ quand $T=300$ K, $\dfrac{dT}{dt}=0,1$ K/s, $V=100$ L et $\dfrac{dV}{dt}=0,2$ L/s.

$$P=8,31\dfrac{T}{V}\quad\Rightarrow\quad\dfrac{dP}{dt}=\frac{8,31}{V}\dfrac{dT}{dt}-\frac{8,31T}{V^2}\dfrac{dV}{dt}$$
$$\dfrac{dP}{dt}=\frac{8,31}{100}\times0,1-\frac{8,31\times300}{100^2}\times0,2=-0,042\text{ kPa/s}$$
\end{Ex}

\subsection{Cas de deux variables intermédiaires}

\begin{Prop}
Soit $z=f(x,y)$ avec $x=x(s,t)$ et $y=y(s,t)$. Alors :
$$\begin{cases}\dfrac{\partial z}{\partial s}=\dfrac{\partial z}{\partial x}\dfrac{\partial x}{\partial s}+\dfrac{\partial z}{\partial y}\dfrac{\partial y}{\partial s}\\\\\dfrac{\partial z}{\partial t}=\dfrac{\partial z}{\partial x}\dfrac{\partial x}{\partial t}+\dfrac{\partial z}{\partial y}\dfrac{\partial y}{\partial t}\end{cases}$$
\end{Prop}

\exo{\textbf{Fonctions composées}}

Calculer $\dfrac{\partial z}{\partial s}$ et $\dfrac{\partial z}{\partial t}$ pour :
\begin{enumerate}
\item $z=e^x\sin y$, $x=st^2$, $y=s^2t$
\item $z=y^2\ln(x+y)$, $x=e^{st}$, $y=e^{s-t}$
\end{enumerate}
\subsection*{QCM}

\begin{enumerate}
\item La règle de la chaîne permet de :
\begin{enumerate}
\item dériver une fonction non dérivable
\item dériver une fonction composée
\item calculer un gradient
\item étudier une limite
\end{enumerate}

\item Si $z=f(x,y)$ avec $x=x(t)$ et $y=y(t)$, alors $\dfrac{dz}{dt}$ dépend :
\begin{enumerate}
\item uniquement de $f$
\item uniquement de $x$
\item de $x$, $y$ et de leurs dérivées
\item uniquement de $t$
\end{enumerate}

\item Dans le cas de deux variables intermédiaires, on calcule :
\begin{enumerate}
\item une dérivée totale
\item deux dérivées partielles
\item une dérivée directionnelle
\item une différentielle seconde
\end{enumerate}
\end{enumerate}

%%%%%%%%%%%%%%%%%%%%%%%%%%%%%%%%%%%%%%%%%%%%%%%%%%%%%%%%%
\section{Dérivée logarithmique}

\begin{Prop}
\textbf{Dérivée logarithmique.} Soient $u,v,w$ trois fonctions dérivables et non nulles sur un intervalle $I$ et $K,\alpha,\beta,\gamma$ des réels. Pour $f : t\longmapsto K\dfrac{ u(t)^\alpha v(t)^\beta}{w(t)^\gamma}$ :
$$\dfrac{df}{f}=\alpha\dfrac{du}{u}+\beta\dfrac{dv}{v}-\gamma\dfrac{dw}{w}$$
\end{Prop}

\begin{Rmq}
Dans une dérivée logarithmique :
\begin{itemize}
\item les \textbf{produits} se transforment en \textbf{sommes}
\item les \textbf{rapports} se transforment en \textbf{différences}
\item les \textbf{exposants} se transforment en \textbf{facteurs multiplicatifs}
\end{itemize}
\end{Rmq}

\exo{\textbf{Gaz parfait}}

Pour $n$ moles d'un gaz parfait : $PV = nRT$.

Établir les relations entre $dP$, $dV$ et $dT$ lorsque $P$, $V$ et $T$ varient de manière élémentaire :
\begin{enumerate}[label=\alph*)]
\item par différentiation simple
\item par différentiation logarithmique
\end{enumerate}
Vérifier l'équivalence des deux méthodes.

\subsection*{QCM}

\begin{enumerate}
\item La dérivée logarithmique transforme :
\begin{enumerate}
\item les sommes en produits
\item les produits en sommes
\item les dérivées en intégrales
\item les quotients en produits
\end{enumerate}

\item L'intérêt principal est de :
\begin{enumerate}
\item simplifier les calculs
\item éviter les dérivées partielles
\item calculer des limites
\item déterminer des extremums
\end{enumerate}

\item Elle est particulièrement adaptée aux expressions :
\begin{enumerate}
\item polynomiales
\item avec racines imbriquées
\item avec produits, quotients et puissances
\item trigonométriques
\end{enumerate}
\end{enumerate}


%%%%%%%%%%%%%%%%%%%%%%%%%%%%%%%%%%%%%%%%%%%%%%%%%%%%%%%%%
\section{Dérivées partielles successives}

\subsection{Définition}

\begin{Def}[Dérivées secondes]
Soit $f\in\mathcal F(D,\R)$ admettant des dérivées partielles premières. Les \textbf{dérivées secondes} de $f$ sont :

\begin{multicols}{2}
$$\frac{\partial^2 f}{\partial x^2}=\frac{\partial}{\partial x}\left(\frac{\partial f}{\partial x}\right)$$
$$\frac{\partial^2 f}{\partial y^2}=\frac{\partial}{\partial y}\left(\frac{\partial f}{\partial y}\right)$$
$$\frac{\partial^2 f}{\partial x\partial y}=\frac{\partial}{\partial x}\left(\frac{\partial f}{\partial y}\right)$$
$$\frac{\partial^2 f}{\partial y\partial x}=\frac{\partial}{\partial y}\left(\frac{\partial f}{\partial x}\right)$$
\end{multicols}
\end{Def}

\begin{Thm}[Théorème de Schwarz]
Soit $f\in \mathcal F(D,\mathbb R)$ admettant des dérivées partielles premières continues sur $D$ et des dérivées secondes croisées continues sur $D$. Alors :
$$\forall (x,y)\in D,\quad \frac{\partial^2 f}{\partial x\partial y}(x,y)=\frac{\partial^2 f}{\partial y\partial x}(x,y)$$
\end{Thm}

\begin{Rmq}
En pratique, pour les fonctions \og usuelles \fg{}, on peut intervertir l'ordre de dérivation.
\end{Rmq}

\exo{\textbf{Dérivées secondes}}

Calculer les dérivées partielles secondes pour :
\begin{multicols}{2}
\begin{enumerate}
\item $f(x,y)=x^2+xy-y^2$
\item $g(x,y)=e^{x^2+2y}+\frac{1}{x}$
\item $h(x,y)=e^{xy}$
\item $k(x,y)=\arctan{\left(\dfrac{x}{y}\right)}$
\item $l(x,y)=e^{x/y}$
\item $m(x,y)=3x^2y+\dfrac{\sin y}{xy}$
\end{enumerate}
\end{multicols}

\subsection{Contre-exemple au théorème de Schwarz}

\begin{Ex}
Soit $f:\R^2\to\R$ définie par :
$$f(x,y)=\begin{cases}
xy \dfrac{x^2-y^2}{x^2+y^2}& \text{si } (x,y)\neq(0,0)\\
0&\text{sinon}\end{cases}$$

\begin{enumerate}
\item Montrer que pour $(x,y)\neq(0,0)$ : $\dfrac{\partial^2 f}{\partial y\partial x}(x,y)\neq\dfrac{\partial^2 f}{\partial x\partial y}(x,y)$
\item Montrer que $f$ est de classe $C^1$.
\end{enumerate}

Le théorème de Schwarz n'est pas vérifié car les dérivées secondes ne sont pas continues en $(0,0)$.
\end{Ex}
\subsection*{QCM}

\begin{enumerate}
\item Les dérivées secondes croisées sont :
\begin{enumerate}
\item $\partial_{xx}$ et $\partial_{yy}$
\item $\partial_x(\partial_x f)$
\item $\partial_x(\partial_y f)$ et $\partial_y(\partial_x f)$
\item toujours nulles
\end{enumerate}

\item Le théorème de Schwarz garantit l'égalité des dérivées croisées si :
\begin{enumerate}
\item $f$ est continue
\item $f$ est dérivable
\item les dérivées secondes sont continues
\item $f$ est polynomiale
\end{enumerate}

\item En pratique, pour les fonctions usuelles :
\begin{enumerate}
\item l'ordre de dérivation est important
\item on ne peut jamais permuter
\item on peut permuter l'ordre
\item il faut vérifier numériquement
\end{enumerate}
\end{enumerate}

%%%%%%%%%%%%%%%%%%%%%%%%%%%%%%%%%%%%%%%%%%%%%%%%%%%%%%%%%
\section{Extremums des fonctions de deux variables}

\subsection{Définitions}

\begin{Def}[Maximum et minimum local]
Soit $f\in \mathcal F(D,\mathbb R)$. La fonction $f$ admet un \textbf{extremum local} en $(a,b)$ s'il existe un voisinage de $(a,b)$ sur lequel $f(x,y)-f(a,b)$ est de signe constant.
\begin{itemize}
\item \textbf{Maximum local :} $f(x,y)-f(a,b)\leq 0$ au voisinage de $(a,b)$
\item \textbf{Minimum local :} $f(x,y)-f(a,b)\geq 0$ au voisinage de $(a,b)$
\end{itemize}
\end{Def}

\begin{Ex}
\begin{multicols}{2}
Soit $f:(x,y)\mapsto \sqrt{1-x^2-y^2}$.

$x^2+y^2$ est minimum en $(0,0)$, donc $1-x^2-y^2$ est maximum en $(0,0)$.

$\Rightarrow$ $f$ est \textbf{maximum} en $(0,0)$ avec $f(0,0)=1$.

\begin{center}\includegraphics[scale=0.2]{images/Courbe3D_2.jpg}\end{center}
\end{multicols}
\end{Ex}

\begin{Def}[Point critique]
Soit $f\in\mathcal F(D,\mathbb R)$ admettant des dérivées partielles. Un point $(x,y)\in D$ est un \textbf{point critique} de $f$ si :
$$\begin{cases}
\dfrac{\partial f}{\partial x}(x,y)=0\\\\
\dfrac{\partial f}{\partial y}(x,y)=0
\end{cases}$$
\end{Def}

\begin{Def}[Point selle]
Un \textbf{point selle} (ou point col) est un point critique qui n'est pas un extremum : la fonction croît dans certaines directions et décroît dans d'autres.
\end{Def}

\begin{Ex}
\begin{multicols}{2}
Soit $f:(x,y)\mapsto xy$.

$(0,0)$ est un point critique car $f'_x=y=0$ et $f'_y=x=0$ en $(0,0)$.

Mais toute boule de centre $(0,0)$ contient des points où $f>0$ et des points où $f<0$.

$\Rightarrow$ $(0,0)$ est un \textbf{point selle}.

\begin{center}\includegraphics[scale=0.2]{images/Courbe3D_3.jpg}\end{center}
\end{multicols}
\end{Ex}

\subsection{Condition nécessaire d'extremum}

\begin{Thm}[Condition nécessaire]
Soit $f\in\mathcal F(D,\mathbb R)$ où $D$ est un \textbf{ouvert} de $\R^2$. Si $f$ admet un extremum local en $(a,b)$, alors :
$$\frac{\partial f}{\partial x}(a,b)=0\quad\text{et}\quad\frac{\partial f}{\partial y}(a,b)=0$$
\end{Thm}

\begin{Rmq}
\textbf{Attention :} C'est une condition \textbf{nécessaire mais pas suffisante}. Un point critique n'est pas forcément un extremum (exemple : point selle).
\end{Rmq}

\begin{center}
\includegraphics[scale=0.1,trim=0cm 65cm 0cm 0cm,clip=true]{images/extremum}
\end{center}

\subsection{Condition suffisante d'extremum}

\begin{Thm}[Condition suffisante - Critère de la Hessienne]
Soit $f\in\mathcal F(D,\mathbb R)$ de classe $C^2$ sur $D$ ouvert. Soit $(a,b)$ un point critique de $f$.

On pose (notations de Monge) :
$$r=\frac{\partial^2 f}{\partial x^2}(a,b),\quad s=\frac{\partial^2 f}{\partial x\partial y}(a,b),\quad t=\frac{\partial^2 f}{\partial y^2}(a,b)$$

Alors :
\begin{enumerate}
\item Si $rt-s^2>0$ et $r>0$ : $f$ admet un \textbf{minimum local} en $(a,b)$
\item Si $rt-s^2>0$ et $r<0$ : $f$ admet un \textbf{maximum local} en $(a,b)$
\item Si $rt-s^2<0$ : $f$ n'admet pas d'extremum, mais un \textbf{point selle} en $(a,b)$
\item Si $rt-s^2=0$ : on ne peut pas conclure
\end{enumerate}
\end{Thm}

\begin{Meth}[Recherche des extremums]
Pour rechercher les extremums locaux :
\begin{enumerate}
\item Calculer les dérivées partielles $\dfrac{\partial f}{\partial x}$ et $\dfrac{\partial f}{\partial y}$
\item Résoudre le système $\begin{cases}\dfrac{\partial f}{\partial x}(x,y)=0\\ \dfrac{\partial f}{\partial y}(x,y)=0\end{cases}$ pour trouver les points critiques
\item Calculer $r$, $s$, $t$ en chaque point critique
\item Appliquer le critère de la Hessienne
\end{enumerate}
\end{Meth}

\begin{Ex}
Soit $f(x,y)=x^3+y^3-3xy$.

\vspace{1em}

\textbf{Étape 1 :} Dérivées partielles
$$\frac{\partial f}{\partial x}=3x^2-3y,\quad \frac{\partial f}{\partial y}=3y^2-3x$$

\vspace{1em}

\textbf{Étape 2 :} Points critiques
$$\begin{cases}x^2=y\\y^2=x\end{cases}\Rightarrow x^4=x\Rightarrow x(x^3-1)=0$$
Points critiques : $(0,0)$ et $(1,1)$.

\vspace{1em}

\textbf{Étape 3-4 :} Classification
$$r=6x,\quad s=-3,\quad t=6y$$

En $(0,0)$ : $r=0$, $s=-3$, $t=0$ $\Rightarrow$ $rt-s^2=-9<0$ : \textbf{point selle}

En $(1,1)$ : $r=6$, $s=-3$, $t=6$ $\Rightarrow$ $rt-s^2=27>0$ et $r>0$ : \textbf{minimum local}
\end{Ex}

\subsection{Optimisation sous contrainte - Méthode de Lagrange}

\begin{Ex}
\textbf{Maximisation d'utilité sous contrainte budgétaire :} Un consommateur dispose d'une somme $b$, et peut acheter des oignons (prix $p_x$) ou des navets (prix $p_y$). L'utilité est $U(x,y)=x^{\alpha}y^{\beta}$. Comment maximiser cette utilité sous la contrainte $xp_x+yp_y=b$ ?

\vspace{1em}

\textbf{Méthode du Lagrangien :}
\begin{enumerate}
\item On pose $\mathcal L(x,y,\lambda)=x^\alpha y^\beta+\lambda (b-xp_x-yp_y)$

\item On résout :
$$\begin{cases}
\dfrac{\partial \mathcal L}{\partial x}=\alpha x^{\alpha-1}y^\beta-\lambda p_x=0\\\\
\dfrac{\partial \mathcal L}{\partial y}=\beta x^{\alpha}y^{\beta-1}-\lambda p_y=0\\\\
\dfrac{\partial \mathcal L}{\partial \lambda}=b-xp_x-yp_y=0
\end{cases}$$

\item Solution : $x=\dfrac{\alpha}{\alpha+\beta}\dfrac{b}{p_x}$ et $y=\dfrac{\beta}{\alpha+\beta}\dfrac{b}{p_y}$
\end{enumerate}
\end{Ex}

\subsection*{QCM}

\begin{enumerate}
\item Un point critique vérifie :
\begin{enumerate}
\item $f=0$
\item $\nabla f\neq 0$
\item $\partial_x f=\partial_y f=0$
\item $f$ est maximale
\end{enumerate}

\item Si $rt-s^2<0$, le point critique est :
\begin{enumerate}
\item un minimum
\item un maximum
\item un point selle
\item un point régulier
\end{enumerate}

\item La méthode de Lagrange sert à :
\begin{enumerate}
\item trouver des dérivées secondes
\item optimiser sans contrainte
\item optimiser sous contrainte
\item étudier la continuité
\end{enumerate}
\end{enumerate}


%%%%%%%%%%%%%%%%%%%%%%%%%%%%%%%%%%%%%%%%%%%%%%%%%%%%%%%%%
\section{Exercices}

\exo{\textbf{Ensembles de définition}}

Représenter les ensembles de définition des fonctions suivantes :
\begin{multicols}{2}
\begin{enumerate}
\item $f_1(x,y)=\ln(2x+y-2)$
\item $f_2(x,y)=\sqrt{1-xy}$
\item $f_3(x,y)=\dfrac{\ln(y-x)}{x}$
\item $f_4(x,y)=\dfrac{1}{\sqrt{x^2+y^2-1}}+\sqrt{4-x^2-y^2}$
\end{enumerate}
\end{multicols}

\medskip\hrule\medskip

\exo{\textbf{Lignes de niveau}}

Représenter les lignes de niveau pour :
\begin{enumerate}
\item $f_1(x,y)=y^2$, avec $k=-1$ et $k=1$
\item $f_2(x,y)=\dfrac{x^4+y^4}{8-x^2y^2}$, avec $k=2$
\end{enumerate}

\medskip\hrule\medskip

\exo{\textbf{Calcul de limites}}

\begin{enumerate}
\item Montrer que pour tous réels $x,y$ : $2|xy|\leq x^2+y^2$

\item Soit $f$ définie sur $A=\mathbb{R}^2\backslash\{(0,0)\}$ par $f(x,y)=\dfrac{3x^2+xy}{\sqrt{x^2+y^2}}$.

Montrer que $|f(x,y)|\leq 4\|(x,y)\|_2$ et en déduire que $f$ admet une limite en $(0,0)$.
\end{enumerate}

\medskip\hrule\medskip

\exo{\textbf{Existence de limites}}

Les fonctions suivantes ont-elles une limite finie en $(0,0)$ ?
\begin{enumerate}
\item $f(x,y)=(x+y)\sin\left(\dfrac{1}{x^2+y^2}\right)$
\item $f(x,y)=\dfrac{x^2-y^2}{x^2+y^2}$
\item $f(x,y)=\dfrac{|x+y|}{x^2+y^2}$
\end{enumerate}

\medskip\hrule\medskip

\exo{\textbf{Limites à paramètres}}

Soient $\alpha,\beta>0$. Déterminer, suivant les valeurs de $\alpha$ et $\beta$, si $f(x,y)=\dfrac{x^\alpha y^\beta}{x^2+y^2}$ admet une limite en $(0,0)$.

\medskip\hrule\medskip

\exo{\textbf{Continuité en un point}}

Étudier la continuité en $(0,0)$ des fonctions suivantes :
\begin{multicols}{2}
\begin{enumerate}
\item $f(x,y) = \begin{cases}
\dfrac{(x+y)^4}{x^4+y^4} & \text{si }(x,y) \neq (0,0) \\
1 & \text{sinon}
\end{cases}$

\item $f(x,y) = \begin{cases}
\dfrac{|x|^3|y|^5}{(x^2+y^2)^2} & \text{si }(x,y) \neq (0,0) \\
0 & \text{sinon}
\end{cases}$

\item $f(x,y) = \begin{cases}
\dfrac{e^{xy}-1}{x^2+y^2} & \text{si }(x,y) \neq (0,0) \\
0 & \text{sinon}
\end{cases}$

\item $f(x,y)=\begin{cases}
\dfrac{(x+2y)^3y^3}{x^4+y^4} & \text{si }(x,y) \neq (0,0) \\
0 & \text{sinon}
\end{cases}$
\end{enumerate}
\end{multicols}

\medskip\hrule\medskip

\exo{\textbf{Plan tangent et parallèle}}

\begin{enumerate}
\item Trouver les points sur le paraboloïde $z = 4x^2 + y^2$ où le plan tangent est parallèle au plan $x + 2y + z = 6$.
\item Même question avec le plan $3x + 5y - 2z = 3$.
\end{enumerate}

\medskip\hrule\medskip

\exo{\textbf{Plan tangent à une fonction}}

Soit $f$ définie sur $\mathbb{R}^2$ par $f (x, y) = x^2 - 2y^3$.
\begin{enumerate}
\item Déterminer l'équation du plan tangent $\mathcal P_{M_0}$ au graphe de $f$ en un point quelconque $M_0$.
\item Pour $M_0$ de coordonnées $(2, 1, 2)$, déterminer tous les points $M$ tels que le plan tangent en $M$ soit parallèle à $\mathcal P_{M_0}$.
\end{enumerate}

\medskip\hrule\medskip

\exo{\textbf{Approximation}}

Utiliser une approximation pour calculer une valeur approchée de :
\begin{enumerate}
\item $\exp[\sin(3,16) \cos(0,02)]$
\item $\arctan[\sqrt{4,03} \times 2\exp(0,01)]$
\end{enumerate}

\medskip\hrule\medskip

\exo{\textbf{Étude d'une fonction}}

Soit $f:\R^2\to\R$ définie par :
$$f(x,y)=\begin{cases}
\dfrac{x^2y}{x-y} & \text{si } x\neq y\\
x & \text{si } x=y
\end{cases}$$

\begin{enumerate}
\item Calculer $\dfrac{\partial f}{\partial x}(1,-2)$ et $\dfrac{\partial f}{\partial y}(1,-2)$.
\item Pour $v=(\cos \theta, \sin \theta)$, calculer $D_vf(1,-2)$. Pour quelles valeurs de $\theta\in [0,2\pi[$, $D_vf(1,-2)=0$ ?
\item Étudier la continuité de $f$ au point $(1,1)$.
\item Étudier la continuité de $f$ au point $(0,0)$.
\item Montrer que $\dfrac{\partial f}{\partial x}(0,0)$ et $\dfrac{\partial f}{\partial y}(0,0)$ existent et les déterminer.
\item Montrer que $D_vf(0,0)$ existe pour $v=(1,1)$, et la déterminer. Constater que $D_vf(0,0)\neq\partial_xf(0,0) + \partial_yf(0,0)$.
\end{enumerate}

\medskip\hrule\medskip

\exo{\textbf{Récapitulatif}}

Soit $f : \mathbb{R}^2 \to \mathbb{R}$ définie par :
$$f(x,y) = \begin{cases}
xy\dfrac{x^2-y^2}{x^2+y^2} & \text{si }(x,y) \neq (0,0) \\
0 & \text{sinon}
\end{cases}$$

\begin{enumerate}
\item Montrer que $f$ est continue sur $\mathbb{R}^2$.
\item Étudier l'existence des dérivées partielles premières de $f$ sur $\mathbb{R}^2$.
\item Étudier la continuité des dérivées partielles premières. En déduire où $f$ est de classe $C^1$.
\item Calculer $\dfrac{\partial f}{\partial x}(1, -2)$ et $\dfrac{\partial f}{\partial y}(1, -2)$ et en déduire le DL à l'ordre 1 de $f$ au point $(1, -2)$.
\end{enumerate}

\medskip\hrule\medskip

\exo{\textbf{Extremums}}

Soit $f(x,y)=x^2+y^2+xy+1$ et $g(x,y)=x^2+y^2+4xy-2$.
\begin{enumerate}
\item Déterminer les points critiques de $f$ et de $g$.
\item En reconnaissant le début du développement d'un carré, étudier les extremums locaux de $f$.
\item En étudiant les valeurs de $g$ sur deux droites bien choisies, étudier les extremums locaux de $g$.
\end{enumerate}

\medskip\hrule\medskip

\exo{\textbf{Extremums locaux}}

Déterminer les extremums locaux des fonctions $f:\mathbb{R}^2 \to \mathbb{R}$ suivantes :
\begin{enumerate}
\item $f(x,y) = x^2 + xy + y^2 - 3x - 6y$
\item $f(x,y) = x^2 + 2y^2 - 2xy - 2y + 1$
\item $f(x,y) = x^3 + y^3$
\item $f(x,y) = (x - y)^2 + (x + y)^3$
\end{enumerate}

\medskip\hrule\medskip

\exo{\textbf{Point de Torricelli/Fermat}}

Soit $A,B,C$ trois points non alignés. On pose $f(M)=AM+BM+CM$.
\begin{enumerate}
\item Étudier la différentiabilité de $g(M)=AM$ et calculer sa différentielle.
\item Démontrer que $f$ atteint son minimum en au moins un point dans le plan $(ABC)$.
\item Démontrer que $f$ est strictement convexe, et en déduire l'unicité du minimum.
\item Soit $F$ le point où $f$ atteint son minimum. Si $F$ est distinct de $A,B,C$, démontrer que :
$$\frac{1}{AF}\overrightarrow{AF}+\frac{1}{BF}\overrightarrow{BF}+\frac{1}{CF}\overrightarrow{CF}=\vec 0$$
\end{enumerate}

\medskip\hrule\medskip

\exo{\textbf{Extremums locaux et globaux}}

Déterminer les extremums locaux et globaux de :
\begin{enumerate}
\item $f(x,y)=y^2-x^2+\dfrac{x^4}{2}$
\item $f(x,y)=x^3+y^3-3xy$
\item $f(x,y)=x^4+y^4-4(x-y)^2$
\end{enumerate}

\medskip\hrule\medskip

\exo{\textbf{Extremums dégénérés}}

Déterminer les extremums locaux des fonctions suivantes. Sont-ce des extremums globaux ?
\begin{enumerate}
\item $f(x,y)=x^2+y^3$
\item $f(x,y)=x^4+y^3-3y-2$
\item $f(x,y)=x^3+xy^2-x^2y-y^3$
\end{enumerate}

\medskip\hrule\medskip

\exo{\textbf{En détails}}

Soit $f(x,y)=y^2-x^2y+x^2$ et $D=\{(x,y)\in\R^2;\ x^2-1\leq y\leq 1-x^2\}$.
\begin{enumerate}
\item Représenter $D$ et trouver une paramétrisation de $\Gamma$ (le bord de $D$).
\item Justifier que $f$ admet un maximum et un minimum sur $D$.
\item Déterminer les points critiques de $f$.
\item Déterminer le minimum et le maximum de $f$ sur $\Gamma$.
\item En déduire le minimum et le maximum de $f$ sur $D$.
\end{enumerate}

\medskip\hrule\medskip

\exo{\textbf{Extremums sur un compact}}

Pour chaque exemple, démontrer que $f$ admet un maximum sur $K$ et le déterminer.
\begin{enumerate}
\item $f(x,y)=xy(1-x-y)$ et $K=\{(x,y)\in\R^2;\ x,y\geq 0,\ x+y\leq 1\}$
\item $f(x,y)=x-y+x^3+y^3$ et $K=[0,1]\times [0,1]$
\item $f(x,y)=\sin x\sin y\sin(x+y)$ et $K=[0,\pi/2]^2$
\end{enumerate}

\medskip\hrule\medskip

\exo{\textbf{Polygone convexe de périmètre maximal}}

On considère un polygone convexe à $n$ côtés inscrit dans le cercle unité. On note $P$ son périmètre et $e^{ia_1},\dots,e^{ia_n}$ les affixes de ses sommets avec $0\leq a_1<\dots<a_n<2\pi$.
\begin{enumerate}
\item On pose $t_k=\frac{1}{2}(a_{k+1}-a_k)$ pour $k\in\{1,\dots,n-1\}$ et $t_n=\frac{1}{2}(a_1+2\pi-a_n)$. Montrer que $P=2\sum_{k=1}^n \sin(t_k)$.
\item Montrer que $P$ est maximal lorsque le polygone est régulier.
\end{enumerate}

\medskip\hrule\medskip

\exo{\textbf{Volume et surface d'une boîte}}

On veut fabriquer une boîte parallélépipédique rectangle, sans couvercle, de volume $0,5$ m$^3$ et de surface totale minimale. Quelles dimensions choisir ?

\medskip\hrule\medskip

\exo{\textbf{Extremums liés}}

Étudier les extremums de $f:(x,y)\mapsto \exp(axy)$, $a>0$, sous la contrainte $x^3+y^3+x+y-4=0$.

\medskip\hrule\medskip

\exo{\textbf{Inégalité arithmético-géométrique}}

Soit $n\geq 2$ et $f:(x_1,\dots,x_n)\mapsto x_1\cdots x_n$. On note $\Gamma=\{(x_1,\dots,x_n)\in\R_+^n;\ x_1+\dots+x_n=1\}$.
\begin{enumerate}
\item Démontrer que $f$ admet un maximum global sur $\Gamma$ et le déterminer.
\item En déduire l'inégalité arithmético-géométrique : pour tout $(x_1,\dots,x_n)\in\R_+^n$ :
$$\prod_{i=1}^n x_i^{1/n}\leq \frac{\sum_{i=1}^n x_i}{n}$$
\end{enumerate}

\medskip\hrule\medskip

\exo{\textbf{Étude complète des extremums}}

Soit $f(x,y) = \sin x + y^3 - 3y$.
\begin{enumerate}
\item Calculer les dérivées partielles premières de $f$.
\item Montrer que l'ensemble des points critiques est :
$$\mathcal{C} = \left(\mathbb{Z} + \frac{1}{2}\right)\pi \times \{-1, 1\}$$
\item Calculer les dérivées partielles secondes et la matrice Hessienne $H(x,y)$.
\item Donner $H(x,y)$ pour les points critiques.
\item Montrer qu'il n'y a que 4 matrices Hessiennes numériquement différentes.
\item Déterminer la nature de chaque point critique (minimum, maximum, selle).
\item Sur $D = [-\pi, \pi] \times [-2, 2]$ :
\begin{enumerate}
\item $D$ est-il fermé ? borné ? $f$ admet-elle des extremums globaux sur $D$ ?
\item Représenter les extremums locaux dans $\mathring{D}$.
\item Étudier les restrictions de $f$ sur chaque côté de la frontière de $D$.
\item Conclure sur les extremums globaux de $f$ sur $D$.
\end{enumerate}
\end{enumerate}

\medskip\hrule\medskip

\exo{\textbf{Avec la dérivée}}

Soit $f : \mathbb{R} \to \mathbb{R}$ de classe $C^1$. On définit $F : \mathbb{R}^2 \to \mathbb{R}$ par :
$$F(x,y) = \begin{cases}
\dfrac{f(x) - f(y)}{x - y} & \text{si } x \neq y \\
f'(x) & \text{si } x = y
\end{cases}$$

Démontrer que $F$ est continue sur $\mathbb{R}^2$.

\end{document}
