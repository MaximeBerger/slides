\documentclass[../main.tex]{subfiles}
\begin{document}

L'intégration correspond au processus inverse de la dérivation. C'est un outil fondamental en mathématiques et en sciences, permettant de calculer des aires, des volumes, ou de résoudre des équations différentielles.

\textit{Dans tout ce chapitre, $f$ et $g$ désignent des fonctions continues sur un intervalle $[a;b]$.}

\section{Primitives et intégrales}

\subsection{Définitions fondamentales}

\begin{Def}\textbf{Primitive d'une fonction}

On dit qu'une fonction $F$ est une \textbf{primitive} de la fonction $f$ sur l'intervalle $[a ; b]$ si la dérivée de $F$ est égale à $f$.

Autrement dit, pour tout $x \in[a ; b]$ :
$$F^{\prime}(x)=f(x)$$
\end{Def}

\begin{Rmq}
    Il faut garder en tête qu'une primitive est définie à constante près : 
    
    Si la fonction $F$ est une primitive de $f$, alors pour n'importe quel nombre réel $k$, la fonction $F+k$ est aussi une primitive de $f$. Une primitive est donc définie \textbf{à une constante près}.
\end{Rmq}

\begin{Def}\textbf{Intégrale}

On appelle \textbf{intégrale de $f$ entre $a$ et $b$} et on note
$$\int_a^b f(x) \, dx$$
l'aire algébrique délimitée par la courbe de $f$ et l'axe des abscisses entre les bornes $a$ et~$b$.
\end{Def}

\subsection{Théorème fondamental de l'analyse}

\begin{Thm}\textbf{Théorème fondamental}

Soit $F$ une primitive de $f$ sur $[a,b]$. Alors :
$$\int_a^b f(x) \, dx = \big[F(x)\big]_a^b = F(b) - F(a)$$

La valeur de l'intégrale ne dépend pas du choix de la primitive.
\end{Thm}

\subsection{Tableau des primitives usuelles}

\renewcommand{\arraystretch}{2.0}
\begin{center}
\begin{tabular}{|c|c||c|c|}
\hline
\textbf{Fonction $f$} & \textbf{Primitive $F$} & \textbf{Fonction $f$} & \textbf{Primitive $F$} \\
\hline
$k$ (constante) & $kx + C$ & $u'(x) \cdot u^n(x)$ & $\dfrac{u^{n+1}(x)}{n+1} + C$ \\
\hline
$x^n$ & $\dfrac{x^{n+1}}{n+1} + C$ & $\dfrac{u'(x)}{u(x)}$ & $\ln |u(x)| + C$ \\
\hline
$\dfrac{1}{x}$ & $\ln |x| + C$ & $u'(x) e^{u(x)}$ & $e^{u(x)} + C$ \\
\hline
$e^x$ & $e^x + C$ & $\dfrac{u'(x)}{\sqrt{u(x)}}$ & $2\sqrt{u(x)} + C$ \\
\hline
$\cos x$ & $\sin x + C$ & $u'(x) \cos(u(x))$ & $\sin(u(x)) + C$ \\
\hline
$\sin x$ & $-\cos x + C$ & $u'(x) \sin(u(x))$ & $-\cos(u(x)) + C$ \\
\hline
$\dfrac{1}{\sqrt{x}}$ & $2\sqrt{x} + C$ & $\dfrac{1}{1+x^2}$ & $\arctan x + C$ \\
\hline
\end{tabular}
\end{center}



\section{QCM d'entraînement}

\exo[1]{\textbf{QCM - Primitives}}

\begin{enumerate}
\item Une primitive de $f(x) = 3x^2 + 2x$ est :
\begin{multicols}{3}
\begin{enumerate}[label=\alph*.]
\item $6x + 2$
\item $x^3 + x^2$
\item $x^3 + x^2 + 5$
\end{enumerate}
\end{multicols}

\item $\displaystyle\int_0^1 e^{2x} \, dx$ vaut :
\begin{multicols}{3}
\begin{enumerate}[label=\alph*.]
\item $\dfrac{e^2 - 1}{2}$
\item $e^2 - 1$
\item $2(e^2 - 1)$
\end{enumerate}
\end{multicols}

\item $\displaystyle\int_1^e \frac{1}{x} \, dx$ vaut :
\begin{multicols}{3}
\begin{enumerate}[label=\alph*.]
\item $0$
\item $1$
\item $e - 1$
\end{enumerate}
\end{multicols}
\end{enumerate}

\vspace{1em}\hrule\vspace{1em}

\subsection{Propriétés de l'intégrale}

\begin{Prop}\textbf{Propriétés fondamentales}

\begin{enumerate}
\item \textbf{Intégrale sur un point :} $\displaystyle\int_a^a f(x) \, dx = 0$
\item \textbf{Changement de bornes :} $\displaystyle\int_a^b f(x) \, dx = -\int_b^a f(x) \, dx$
\item \textbf{Positivité :} Si $f \geq 0$ alors $\displaystyle\int_a^b f(x) \, dx \geq 0$
\item \textbf{Croissance :} Si $f \geq g$ alors $\displaystyle\int_a^b f(x) \, dx \geq \int_a^b g(x) \, dx$
\item \textbf{Inégalité triangulaire :} $\displaystyle\left|\int_a^b f(x) \, dx\right| \leq \int_a^b |f(x)| \, dx$
\item \textbf{Linéarité :} $\displaystyle\int_a^b \lambda f(x) + g(x) \, dx = \lambda \int_a^b f(x) \, dx + \int_a^b g(x) \, dx$
\item \textbf{Relation de Chasles :} $\displaystyle\int_a^b f(x) \, dx = \int_a^c f(x) \, dx + \int_c^b f(x) \, dx$
\end{enumerate}
\end{Prop}

\vspace{1em}\hrule\vspace{1em}

\section{Techniques d'intégration}

\subsection{Intégration par parties (IPP)}

Le principe est d'exploiter la règle de Leibniz pour dériver un produit :
$$(uv)' = u'v + uv'$$

\begin{Thm}\textbf{Formule d'intégration par parties}

Soient $f$ et $g$ deux fonctions dérivables de dérivées continues sur $[a,b]$. Alors :
$$\int_a^b f(x) g'(x) \, dx = \big[f(x) g(x)\big]_a^b - \int_a^b f'(x) g(x) \, dx$$
\end{Thm}

\begin{Rmq}
La difficulté réside dans le choix judicieux de $f$ et $g'$. En général :
\begin{itemize}
\item On choisit pour $f$ une fonction qui se simplifie en dérivant (polynôme, $\ln$, $\arctan$...)
\item On choisit pour $g'$ une fonction dont on connaît une primitive ($e^x$, $\sin$, $\cos$...)
\end{itemize}
\end{Rmq}

\begin{Ex}
Calculer $\displaystyle\int_0^1 x e^x \, dx$.

On pose $f(x) = x$ et $g'(x) = e^x$, donc $f'(x) = 1$ et $g(x) = e^x$.

$\displaystyle\int_0^1 x e^x \, dx = \big[x e^x\big]_0^1 - \int_0^1 e^x \, dx = e - \big[e^x\big]_0^1 = e - (e - 1) = 1$
\end{Ex}

\vspace{1em}\hrule\vspace{1em}

\subsection{Décomposition en éléments simples}

Cette technique permet de calculer les primitives de fractions rationnelles (quotients de polynômes).

\begin{Meth}\textbf{Principe de la méthode}
Pour intégrer $\dfrac{h(x)}{f(x) \cdot g(x)}$, on cherche des polynômes $A(x)$ et $B(x)$ tels que :
$$\frac{h(x)}{f(x) \cdot g(x)} = \frac{A(x)}{f(x)} + \frac{B(x)}{g(x)}$$

On procède par identification en multipliant par $f(x) \cdot g(x)$ :
$$h(x) = A(x) \cdot g(x) + B(x) \cdot f(x)$$
\end{Meth}

\begin{Ex}
Décomposer $\dfrac{1}{x^2-1}$ puis calculer $\displaystyle\int_2^4 \dfrac{1}{x^2-1} \, dx$.

On cherche $a$ et $b$ tels que $\dfrac{1}{x^2-1} = \dfrac{a}{x-1} + \dfrac{b}{x+1}$.

En multipliant par $(x-1)(x+1)$ : $1 = a(x+1) + b(x-1)$.
\begin{itemize}
\item Pour $x = 1$ : $1 = 2a$, donc $a = \dfrac{1}{2}$
\item Pour $x = -1$ : $1 = -2b$, donc $b = -\dfrac{1}{2}$
\end{itemize}

Ainsi : $\displaystyle\int_2^4 \dfrac{1}{x^2-1} \, dx = \dfrac{1}{2}\int_2^4 \dfrac{1}{x-1} \, dx - \dfrac{1}{2}\int_2^4 \dfrac{1}{x+1} \, dx = \dfrac{1}{2}\ln\dfrac{3}{5}$
\end{Ex}

\vspace{1em}\hrule\vspace{1em}

\subsection{Changement de variable}

\begin{Thm}\textbf{Formule de changement de variable}

Soit $\varphi : [a,b] \to I$ une fonction dérivable de dérivée continue. Alors :
$$\int_a^b f(\varphi(x)) \cdot \varphi'(x) \, dx = \int_{\varphi(a)}^{\varphi(b)} f(u) \, du$$

En posant $u = \varphi(x)$, on a $du = \varphi'(x) \, dx$.
\end{Thm}

\begin{Rmq}
Attention à bien transformer les bornes lors du changement de variable !
\end{Rmq}

\begin{Ex}
Calculer $\displaystyle\int_{-1}^{1} \sqrt{1-x^2} \, dx$ en posant $x = \sin u$.

On a $dx = \cos u \, du$ et $\sqrt{1-x^2} = \sqrt{1-\sin^2 u} = |\cos u| = \cos u$ (car $u \in [-\frac{\pi}{2}, \frac{\pi}{2}]$).

Les bornes : $x = -1 \Rightarrow u = -\frac{\pi}{2}$ et $x = 1 \Rightarrow u = \frac{\pi}{2}$.

$\displaystyle\int_{-1}^{1} \sqrt{1-x^2} \, dx = \int_{-\pi/2}^{\pi/2} \cos^2 u \, du = \int_{-\pi/2}^{\pi/2} \frac{1+\cos(2u)}{2} \, du = \frac{\pi}{2}$

On retrouve bien l'aire d'un demi-disque de rayon 1.
\end{Ex}

\vspace{1em}\hrule\vspace{1em}


\section{Exercices}

\exo[1]{\textbf{Calculs directs de primitives}}

Déterminer une primitive des fonctions suivantes :

\begin{multicols}{3}
\begin{enumerate}
\item $f(x) = 3x + 1$
\item $f(x) = 3x^2 - x + 1$
\item $f(x) = \dfrac{1}{x} + 2$
\item $f(x) = e^x + 1$
\item $f(x) = \dfrac{3}{x+1} - e^{2x+1}$
\item $f(t) = t \cdot e^{t^2+1}$
\item $f(t) = \dfrac{e^t}{e^t+1}$
\item $f(t) = \dfrac{e^{2t}-1}{e^t}$
\item $f(t) = \dfrac{t^2-t}{t^3}$
\end{enumerate}
\end{multicols}

\vspace{1em}\hrule\vspace{1em}

\exo[2]{\textbf{Intégration par parties}}

Calculer les intégrales suivantes :

\begin{multicols}{2}
\begin{enumerate}
\item $\displaystyle\int_0^1 \ln(x+1) \, dx$
\item $\displaystyle\int_1^{e^\pi} \sin(\ln x) \, dx$
\item $\displaystyle\int_0^{\pi/2} x^2 \cos x \, dx$
\item $\displaystyle\int_1^e x \ln x \, dx$
\end{enumerate}
\end{multicols}

\vspace{1em}\hrule\vspace{1em}

\exo[2]{\textbf{Changement de variable}}

\begin{enumerate}
\item En posant $u = \ln x$, calculer $\displaystyle\int_1^e \frac{1}{x\sqrt{\ln x + 1}} \, dx$
\item En posant $t = x + \sin x$, calculer $\displaystyle\int_{\pi/6}^{\pi/3} \frac{\cos^2(x/2)}{x + \sin x} \, dx$
\item En posant $t = \sin x - \cos x$, calculer $\displaystyle\int_{\pi/6}^{\pi/3} \frac{\sin x + \cos x}{2 - 2\sin(2x)} \, dx$
\end{enumerate}

\vspace{1em}\hrule\vspace{1em}

\exo[2]{\textbf{Décomposition en éléments simples}}

\begin{enumerate}
\item Déterminer $a$ et $b$ tels que $\dfrac{x}{2x^2+9x+9} = \dfrac{a}{x+3} + \dfrac{b}{2x+3}$.

En déduire $\displaystyle\int_0^1 \frac{x}{2x^2+9x+9} \, dx$.

\item Déterminer $a$, $b$ et $c$ tels que $\dfrac{x-2}{(x^2+1)(2x+1)} = \dfrac{ax+b}{x^2+1} + \dfrac{c}{2x+1}$.

En déduire $\displaystyle\int_0^1 \frac{x-2}{(x^2+1)(2x+1)} \, dx$.
\end{enumerate}

\vspace{1em}\hrule\vspace{1em}

\exo[3]{\textbf{Problème - Aire d'un logo}}

On s'intéresse au logo ci-dessous, symétrique par rapport à l'axe des ordonnées :

La feuille gauche du logo correspond à la partie du plan délimitée par les fonctions :
$$f(x) = \frac{0{,}2}{x} \quad \text{et} \quad g(x) = -x^2 + 0{,}2x + 1$$

L'unité choisie sur chacun des axes est de 2,5 cm.

\begin{enumerate}
\item Déterminer les points d'intersection des courbes $\mathcal{C}_f$ et $\mathcal{C}_g$.
\item Calculer l'aire de la feuille gauche du logo.
\item En déduire l'aire totale du logo en $cm^2$.
\end{enumerate}

\vspace{1em}\hrule\vspace{1em}

\exo[3]{\textbf{Application - Réserve d'eau}}

On prélève via un tuyau cylindrique de l'eau dans une réserve d'eau de pluie de 10 000 litres. Le débit d'eau, mesuré en litres par minute, varie en fonction du temps $t$ (en minutes) selon la formule :
$$D(t) = \frac{K}{(t+1)(t+2)}$$

où $K$ est une constante à déterminer.

Pour un prélèvement qui dure $x$ minutes, le volume prélevé est :
$$V(x) = \int_0^x D(t) \, dt$$

\begin{enumerate}
\item Décomposer $\dfrac{1}{(t+1)(t+2)}$ en éléments simples.
\item En déduire l'expression de $V(x)$ en fonction de $K$ et $x$.
\item Sachant que le prélèvement a duré 10 minutes et qu'il reste 8 000 litres dans la réserve, déterminer la constante $K$.
\item Combien de temps faudrait-il pour vider entièrement la réserve (limite théorique) ?
\end{enumerate}

\vspace{1em}\hrule\vspace{1em}

\exo[3]{\textbf{Application - Hangar à peindre}}

Un architecte conçoit un hangar dont la façade avant a la forme d'une courbe définie par :
$$f(x) = 80 - 20e^{0{,}025x} \quad \text{pour } x \in [0, 60]$$
où $x$ est exprimé en mètres.

\begin{enumerate}
\item Vérifier que $f(0) = 60$ m et calculer $f(60)$.
\item Montrer que $f$ est strictement décroissante sur $[0, 60]$.
\item Calculer l'aire de la façade avant (entre la courbe et l'axe des abscisses).
\item La peinture utilisée a un pouvoir couvrant de  $0,2 m^2$ par litre et est vendue en bidons de 68 litres. Combien de bidons faut-il pour peindre la façade ?
\end{enumerate}

\end{document}
