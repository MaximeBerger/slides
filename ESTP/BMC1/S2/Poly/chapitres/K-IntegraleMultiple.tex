\documentclass[../main.tex]{subfiles}
\begin{document}

L'intégrale double généralise la notion d'intégrale simple à des fonctions de deux variables. Elle permet de calculer des volumes, des aires, des masses, des centres d'inertie et bien d'autres grandeurs physiques.

\textit{Dans tout ce chapitre, l'espace est muni d'un repère orthonormé $(O;\vec i,\vec j,\vec k)$.}

\section{Rappel : calcul de volume par intégrale simple}

\begin{Prop}\textbf{Volume d'un solide}

On considère un solide délimité par les plans d'équations respectives $z = a$ et $z = b$.

On désigne par $B(z)$ la section plane de ce solide avec le plan perpendiculaire à $(Oz)$ de cote $z$ $(a \leq z \leq b)$.

On note $S(z)$ l'aire de la section $B(z)$.

Le volume $V$, en unités de volume, de ce solide est égal à :
$$V = \int_a^b S(z) \, dz$$
\end{Prop}

\begin{center}
\includegraphics{images/volume}
\end{center}

\begin{Ex}\textbf{Volume de la sphère}

\begin{minipage}{0.55\linewidth}
Considérons la sphère de centre $O$ et de rayon $R$. Elle est située entre les plans de cotes $-R$ et $R$.

Soit $z$ un réel de $[-R,R]$.

L'intersection de la sphère et du plan de cote $z$ est le disque $B(z)$ de rayon $r = \sqrt{R^2-z^2}$ dont l'aire est :
$$S(z) = \pi r^2 = \pi(R^2-z^2)$$

Le volume de la sphère est donc égal à :
$$V = \int_{-R}^R \pi(R^2-z^2) \, dz = \pi\left[R^2z - \frac{z^3}{3}\right]_{-R}^{R} = \frac{4}{3}\pi R^3$$
\end{minipage}
\hfill
\begin{minipage}{0.4\linewidth}
\begin{center}
\includegraphics[scale=0.6]{images/sphere.png}
\end{center}
\end{minipage}
\end{Ex}

\vspace{1em}\hrule\vspace{1em}

\exo[1]{\textbf{Volume d'un paraboloïde}}

Quel est le volume $V$ du solide ci-dessous, obtenu par révolution autour de l'axe $(Oz)$ du morceau de parabole d'équation (dans le plan $(yOz)$) $z = y^2$ (avec $0 \leq y \leq 2$) ?

\begin{center}
\includegraphics[scale=0.5]{images/fig1.png}
\end{center}

\vspace{1em}\hrule\vspace{1em}

\exo[1]{\textbf{Volume d'un cône}}

Déterminer le volume d'un cône circulaire de révolution, de hauteur $h$ et de génératrice d'équation (dans le plan $(yOz)$) $z = ay$ où $a$ est un nombre réel strictement positif donné.

\vspace{1em}\hrule\vspace{1em}

\section{Intégrale double}

\subsection{Définition et interprétation géométrique}

\begin{Def}\textbf{Intégrale double}

Soit $f$ une fonction continue sur un domaine fermé $D$ de $\mathbb{R}^2$. On dit que $f$ est intégrable sur $D$ si la limite suivante existe :
$$\lim_{m,n \to +\infty} \sum_{i=1}^m \sum_{j=1}^n f(a_i, b_j) \Delta x_i \Delta y_j$$

Cette limite, notée $\displaystyle\iint_D f(x,y) \, dxdy$, est appelée \textbf{intégrale double} de $f$ sur $D$.
\end{Def}

\begin{Rmq}\textbf{Interprétation géométrique}

Si $f(x,y) \geq 0$ sur $D$, alors $\displaystyle\iint_D f(x,y) \, dxdy$ représente le \textbf{volume} du solide délimité par :
\begin{itemize}
\item le plan $(xOy)$ en dessous
\item la surface $z = f(x,y)$ au-dessus
\item les "murs" verticaux au-dessus du bord de $D$
\end{itemize}
\end{Rmq}

\begin{center}
\begin{minipage}{0.45\linewidth}
\includegraphics[scale=0.05]{images/Expl1}
\end{minipage}
\hfill
\begin{minipage}{0.45\linewidth}
\includegraphics[scale=0.07]{images/Expl1Vol}
\end{minipage}
\end{center}

\subsection{Théorème de Fubini}

\begin{Thm}\textbf{Théorème de Fubini (sur un pavé)}

Soit $f$ une fonction continue sur le pavé $[a,b] \times [c,d]$. Alors :
$$\iint_{[a,b] \times [c,d]} f(x,y) \, dxdy = \int_a^b \left( \int_c^d f(x,y) \, dy \right) dx = \int_c^d \left( \int_a^b f(x,y) \, dx \right) dy$$
\end{Thm}

\begin{Rmq}
Le théorème de Fubini affirme que l'on peut calculer une intégrale double en effectuant \textbf{deux intégrales simples successives}, et que l'ordre d'intégration n'importe pas (le résultat est le même).
\end{Rmq}

\begin{center}
\begin{minipage}{0.45\linewidth}
\includegraphics[scale=0.07]{images/Fubini1}
\end{minipage}
\hfill
\begin{minipage}{0.45\linewidth}
\includegraphics[scale=0.07]{images/Fubini2}
\end{minipage}
\end{center}

\begin{Ex}
Calculer $\displaystyle\iint_{[0,4]^2} \left(-\frac{x}{2} - \frac{y}{4} + 7\right) dxdy$.

\begin{align*}
\iint_{[0,4]^2} \left(-\frac{x}{2} - \frac{y}{4} + 7\right) dxdy &= \int_0^4 \left( \int_0^4 \left(-\frac{x}{2} - \frac{y}{4} + 7\right) dx \right) dy \\
&= \int_0^4 \left[-\frac{x^2}{4} - \frac{yx}{4} + 7x\right]_0^4 dy \\
&= \int_0^4 (24-y) \, dy = \left[24y - \frac{y^2}{2}\right]_0^4 = 88
\end{align*}
\end{Ex}

\begin{Prop}\textbf{Cas des variables séparables}

Si $f(x,y) = h(x) \times k(y)$, alors :
$$\iint_{[a,b] \times [c,d]} h(x) k(y) \, dxdy = \left(\int_a^b h(x) \, dx\right) \times \left(\int_c^d k(y) \, dy\right)$$
\end{Prop}

\begin{Ex}
Calculer $\displaystyle\iint_{[0,2] \times [0,1]} x^2 y \, dxdy$.

$$\iint_{[0,2] \times [0,1]} x^2 y \, dxdy = \int_0^2 x^2 \, dx \times \int_0^1 y \, dy = \left[\frac{x^3}{3}\right]_0^2 \times \left[\frac{y^2}{2}\right]_0^1 = \frac{8}{3} \times \frac{1}{2} = \frac{4}{3}$$
\end{Ex}

\newpage 

\exo[1]{\textbf{QCM - Intégrales doubles}}

\begin{enumerate}
\item $\displaystyle\iint_{[0,1]^2} (x+y) \, dxdy$ vaut :
\begin{multicols}{3}
\begin{enumerate}[label=\alph*.]
\item $\dfrac{1}{2}$
\item $1$
\item $2$
\end{enumerate}
\end{multicols}

\item $\displaystyle\iint_{[0,\pi] \times [0,\pi]} \sin x \cos y \, dxdy$ vaut :
\begin{multicols}{3}
\begin{enumerate}[label=\alph*.]
\item $0$
\item $2$
\item $4$
\end{enumerate}
\end{multicols}

\item Si $D = [0,1]^2$, alors $\displaystyle\iint_D 1 \, dxdy$ vaut :
\begin{multicols}{3}
\begin{enumerate}[label=\alph*.]
\item $0$
\item $1$
\item $2$
\end{enumerate}
\end{multicols}
\end{enumerate}

\vspace{1em}\hrule\vspace{1em}

\subsection{Intégrale sur un domaine élémentaire}

\begin{Def}\textbf{Domaine élémentaire}

Un domaine $D$ de $\mathbb{R}^2$ est dit \textbf{élémentaire} si on peut l'écrire :
\begin{itemize}
\item $D = \{ (x,y) \in \mathbb{R}^2 \mid a \leq x \leq b \text{ et } \varphi_1(x) \leq y \leq \varphi_2(x) \}$

ou bien

\item $D = \{ (x,y) \in \mathbb{R}^2 \mid c \leq y \leq d \text{ et } \psi_1(y) \leq x \leq \psi_2(y) \}$
\end{itemize}
où les fonctions $\varphi_1, \varphi_2$ (resp. $\psi_1, \psi_2$) sont continues.
\end{Def}

\begin{center}
\includegraphics[scale=0.3]{images/fig3.png}
\end{center}

\begin{Thm}\textbf{Théorème de Fubini (sur un domaine élémentaire)}

Soit $f$ une fonction continue sur un domaine élémentaire $D$. Alors :
$$\iint_D f(x,y) \, dxdy = \int_a^b \left( \int_{\varphi_1(x)}^{\varphi_2(x)} f(x,y) \, dy \right) dx = \int_c^d \left( \int_{\psi_1(y)}^{\psi_2(y)} f(x,y) \, dx \right) dy$$
\end{Thm}

\begin{Meth}\textbf{En pratique}

\begin{enumerate}
\item \textbf{Dessiner le domaine} $D$ et identifier ses frontières.
\item \textbf{Choisir l'ordre d'intégration} :
\begin{itemize}
\item Intégration à "$x$ constant" : on fixe $x$, on fait varier $y$ entre les frontières
\item Intégration à "$y$ constant" : on fixe $y$, on fait varier $x$ entre les frontières
\end{itemize}
\item \textbf{Tracer une flèche} traversant le domaine pour visualiser les bornes.
\item \textbf{Calculer} l'intégrale intérieure, puis l'intégrale extérieure.
\end{enumerate}
\end{Meth}

\begin{center}
\includegraphics[scale=0.7]{images/fig4.png}
\end{center}

\begin{Ex}
Calculer $\displaystyle\iint_{D} y^2 \, dxdy$, avec $D = \{(x,y) \in \mathbb{R}^2 \mid y \geq 0, 0 \leq x \leq 2, y \leq x\}$.

\begin{minipage}{0.55\linewidth}
On intègre à $x$ constant. Pour $x$ fixé dans $[0,2]$, $y$ varie de $0$ à $x$.

\begin{align*}
\iint_{D} y^2 \, dxdy &= \int_0^2 \left( \int_0^x y^2 \, dy \right) dx \\
&= \int_0^2 \left[\frac{y^3}{3}\right]_0^x dx \\
&= \int_0^2 \frac{x^3}{3} \, dx \\
&= \left[\frac{x^4}{12}\right]_0^2 = \frac{4}{3}
\end{align*}
\end{minipage}
\hfill
\begin{minipage}{0.4\linewidth}
\begin{center}
\includegraphics[scale=0.7]{images/fig5.png}
\end{center}
\end{minipage}
\end{Ex}

\begin{Ex}
Calculer $\displaystyle\iint_{D} (x+2y) \, dxdy$, avec $D$ le domaine délimité par les paraboles $y = 2x^2$ et $y = 1+x^2$.

\begin{minipage}{0.55\linewidth}
Les paraboles se coupent en $2x^2 = 1+x^2$, soit $x^2 = 1$, donc $x = \pm 1$.

Pour $x \in [-1,1]$, $y$ varie de $2x^2$ à $1+x^2$.

\begin{align*}
&\iint_{D} (x+2y) \, dxdy = \int_{-1}^1 \left( \int_{2x^2}^{1+x^2} (x+2y) \, dy \right) dx \\
&= \int_{-1}^1 \left[xy+y^2\right]_{2x^2}^{1+x^2} dx \\
&= \int_{-1}^1 (-3x^4-x^3+2x^2+x+1) \, dx = \frac{32}{15}
\end{align*}
\end{minipage}
\hfill
\begin{minipage}{0.4\linewidth}
\begin{center}
\includegraphics[scale=0.7]{images/fig6.png}
\end{center}
\end{minipage}
\end{Ex}

\vspace{1em}\hrule\vspace{1em}

\exo[2]{\textbf{Calculs d'intégrales doubles}}

Calculer les intégrales doubles suivantes :

\begin{enumerate}
\item $\displaystyle\iint_D xy \, dxdy$ où $D = \{(x,y) \mid 0 \leq x \leq 1, x^2 \leq y \leq x\}$

\item $\displaystyle\iint_D (x^2+y^2) \, dxdy$ où $D = \{(x,y) \mid x \geq 0, y \geq 0, x+y \leq 1\}$

\item $\displaystyle\iint_D e^{x+y} \, dxdy$ où $D = \{(x,y) \mid 0 \leq x \leq 1, 0 \leq y \leq x\}$
\end{enumerate}

\vspace{1em}\hrule\vspace{1em}

\section{Aire d'un domaine}

\begin{Def}\textbf{Aire d'un domaine}

L'aire $S$ d'un domaine $D$ de $\mathbb{R}^2$ est définie par :
$$S = \iint_D dxdy = \iint_D 1 \, dxdy$$
\end{Def}

\begin{Rmq}
L'intégrale $\displaystyle\iint_D 1 \, dxdy$ représente le volume du cylindre de hauteur 1 et de base $D$, qui vaut bien $1 \times S = S$.
\end{Rmq}

\begin{Ex}
Calculer l'aire intérieure du triangle défini par les trois points $O(0,0)$, $A(2,2)$, $B(0,6)$.

\begin{minipage}{0.6\linewidth}
Le domaine est :
$D = \{(x,y) \in \mathbb{R}^2 \mid 0 \leq x \leq 2, x \leq y \leq -2x+6\}$

\begin{align*}
A &= \iint_D dxdy = \int_0^2 \left(\int_x^{-2x+6} dy\right) dx \\
&= \int_0^2 (-3x+6) \, dx = \left[-\frac{3}{2}x^2+6x\right]_0^2 = 6 \text{ u.a.}
\end{align*}

On retrouve ce résultat avec la formule $\frac{B \times h}{2} = \frac{6 \times 2}{2} = 6$.
\end{minipage}
\hfill
\begin{minipage}{0.35\linewidth}
\begin{center}
\includegraphics[scale=0.7]{images/fig10.png}
\end{center}
\end{minipage}
\end{Ex}

\vspace{1em}\hrule\vspace{1em}

\exo[2]{\textbf{Aire d'une ellipse}}

Calculer l'aire intérieure de l'ellipse d'équation $\dfrac{x^2}{a^2}+\dfrac{y^2}{b^2}=1$.

\textit{Indication : utiliser un changement de variable $x = a\cos\phi$.}

\vspace{1em}\hrule\vspace{1em}

\section{Intégration en coordonnées polaires}

\subsection{Rappel : coordonnées polaires}

\begin{center}
\begin{minipage}{0.45\linewidth}
\includegraphics[scale=0.6]{images/fig35.png}
\end{minipage}
\hfill
\begin{minipage}{0.45\linewidth}
$$\rho = OM = \sqrt{x^2+y^2}$$
$$\theta = (\vec{i}, \overrightarrow{OM})$$

\textbf{Formules de passage :}
$$x = \rho \cos \theta$$
$$y = \rho \sin \theta$$
\end{minipage}
\end{center}

\subsection{Formule d'intégration en polaires}

\begin{Thm}\textbf{Changement de variables en coordonnées polaires}

$$\colorbox{yellow}{$\displaystyle\iint_D f(x,y) \, dxdy = \iint_D f(\rho \cos \theta, \rho \sin \theta) \cdot \rho \, d\rho \, d\theta$}$$
\end{Thm}

\begin{Rmq}\textbf{Attention !}
\begin{enumerate}
\item L'élément différentiel $dxdy$ est remplacé par $\rho \, d\rho \, d\theta$ (facteur $\rho$ supplémentaire).
\item Après substitution, il faut \textbf{simplifier au maximum} en utilisant $\cos^2\theta + \sin^2\theta = 1$.
\item Ne pas oublier de \textbf{convertir les bornes} du domaine en coordonnées polaires.
\end{enumerate}
\end{Rmq}

\begin{Ex}
Calculer $I = \displaystyle\iint_D \dfrac{dxdy}{\sqrt{x^2+y^2}(1+x^2+y^2)}$, où $D$ est le disque de centre $O$ et de rayon $a$.

\begin{minipage}{0.55\linewidth}
En coordonnées polaires :
\begin{itemize}
\item $D = \{(\rho,\theta) \mid 0 \leq \rho \leq a, 0 \leq \theta \leq 2\pi\}$
\item $\sqrt{x^2+y^2} = \rho$ et $1+x^2+y^2 = 1+\rho^2$
\end{itemize}

\begin{align*}
I &= \int_0^{2\pi} \int_0^a \frac{\rho \, d\rho \, d\theta}{\rho(1+\rho^2)} \\
&= \int_0^{2\pi} \left(\int_0^a \frac{d\rho}{1+\rho^2}\right) d\theta \\
&= \int_0^{2\pi} \arctan(a) \, d\theta = 2\pi \arctan(a)
\end{align*}
\end{minipage}
\hfill
\begin{minipage}{0.4\linewidth}
\begin{center}
\includegraphics[scale=0.7]{images/fig25.png}
\end{center}
\end{minipage}
\end{Ex}

\begin{Ex}\textbf{Volume d'une demi-boule de rayon $R$}

\begin{minipage}{0.55\linewidth}
$D$ est le disque de centre $O$ et de rayon $R$.

L'équation de la demi-sphère est $z = \sqrt{R^2-\rho^2}$.

\begin{align*}
V &= \iint_D \sqrt{R^2-\rho^2} \cdot \rho \, d\rho \, d\theta \\
&= \int_0^{2\pi} d\theta \int_0^R \sqrt{R^2-\rho^2} \cdot \rho \, d\rho \\
&= 2\pi \times \left[-\frac{1}{3}(R^2-\rho^2)^{3/2}\right]_0^R \\
&= 2\pi \times \frac{R^3}{3} = \frac{2\pi R^3}{3}
\end{align*}
\end{minipage}
\hfill
\begin{minipage}{0.4\linewidth}
\begin{center}
\includegraphics[scale=0.7]{images/fig23.png}
\end{center}
\end{minipage}
\end{Ex}

\vspace{1em}\hrule\vspace{1em}

\exo[2]{\textbf{Intégrales en polaires}}

\begin{enumerate}
\item Calculer $\displaystyle\iint_D (x^2+y^2) \, dxdy$ où $D$ est le disque de centre $(0,1)$ et de rayon 1.

\item Calculer $\displaystyle\iint_D \sqrt{x^2+y^2} \, dxdy$ où $D = \{(x,y) \mid x^2+y^2-2x \leq 0\}$.

\item Calculer $\displaystyle\iint_D \frac{xy}{x^2+y^2} \, dxdy$ où $D$ est le quart de couronne $\{x \geq 0, y \geq 0, 1 \leq x^2+y^2 \leq 4\}$.
\end{enumerate}

\vspace{1em}\hrule\vspace{1em}

\exo[3]{\textbf{Intégrale de Gauss}}

On considère $J = \displaystyle\iint_D e^{-(x^2+y^2)} dxdy$ où $D$ est le quart de plan $x \geq 0$, $y \geq 0$.

\begin{enumerate}
\item Calculer $J$ en coordonnées polaires.
\item Calculer $J$ en coordonnées cartésiennes en fonction de $I = \displaystyle\int_0^{+\infty} e^{-x^2} dx$.
\item En déduire que $\displaystyle\int_0^{+\infty} e^{-x^2} dx = \frac{\sqrt{\pi}}{2}$.
\end{enumerate}

\vspace{1em}\hrule\vspace{1em}

\section{Applications de l'intégrale double}

\subsection{Masse d'un solide}

\begin{Def}\textbf{Masse}

Si un domaine $D$ possède une \textbf{densité surfacique} $\mu(x,y)$, la masse du solide est :
$$M = \iint_D \mu(x,y) \, dxdy$$
\end{Def}

\begin{Ex}
Calculer la masse d'un disque de centre $O$, de rayon $R$, sachant que la densité surfacique $\mu(x,y)$ au point $M(x,y)$ est proportionnelle à la distance de $M$ au centre.

$$M = \iint_D k\sqrt{x^2+y^2} \, dxdy = k\int_0^{2\pi} d\theta \int_0^R \rho^2 \, d\rho = \frac{2\pi kR^3}{3}$$
\end{Ex}

\subsection{Centre d'inertie d'un solide}

\begin{Def}\textbf{Centre d'inertie (ou centre de gravité)}

Pour une figure plane de densité surfacique $\mu(x,y)$, les coordonnées du centre de gravité $G$ sont :

$$x_G = \frac{\displaystyle\iint_D x \cdot \mu(x,y) \, dxdy}{\displaystyle\iint_D \mu(x,y) \, dxdy} \qquad y_G = \frac{\displaystyle\iint_D y \cdot \mu(x,y) \, dxdy}{\displaystyle\iint_D \mu(x,y) \, dxdy}$$

\vspace{1em}

\textbf{Cas homogène} ($\mu$ constante) :
$$x_G = \frac{\displaystyle\iint_D x \, dxdy}{\displaystyle\iint_D dxdy} \qquad y_G = \frac{\displaystyle\iint_D y \, dxdy}{\displaystyle\iint_D dxdy}$$
\end{Def}

\begin{Ex}
Déterminer les coordonnées du centre d'inertie du domaine homogène défini par $\{y \geq 0, y \leq 4-x^2\}$.

\begin{minipage}{0.55\linewidth}
\begin{enumerate}
\item \textbf{Aire :}
$$S = \int_{-2}^2 (4-x^2) \, dx = \left[4x-\frac{x^3}{3}\right]_{-2}^2 = \frac{32}{3}$$

\item \textbf{Par symétrie :} $x_G = 0$

\item \textbf{Calcul de $y_G$ :}
\begin{align*}
\iint_D y \, dxdy &= \int_{-2}^2 \left[\frac{y^2}{2}\right]_0^{4-x^2} dx \\
&= \frac{1}{2}\int_{-2}^2 (4-x^2)^2 \, dx = \frac{256}{15}
\end{align*}

Ainsi : $y_G = \dfrac{256/15}{32/3} = \dfrac{8}{5} = 1,6$
\end{enumerate}
\end{minipage}
\hfill
\begin{minipage}{0.4\linewidth}
\begin{center}
\includegraphics[scale=0.7]{images/fig13.png}

$G(0; 1,6)$
\end{center}
\end{minipage}
\end{Ex}

\subsection{Moment d'inertie d'un solide}

\begin{Def}\textbf{Moment d'inertie}

Le moment d'inertie caractérise la résistance d'un solide à la mise en rotation. Plus les masses sont éloignées de l'axe de rotation, plus le moment d'inertie est grand.

\begin{itemize}
\item \textbf{Par rapport à l'origine :}
$$I_O = \iint_D \mu(x,y)(x^2+y^2) \, dxdy$$

\item \textbf{Par rapport à l'axe $Ox$ :}
$$I_{Ox} = \iint_D \mu(x,y) y^2 \, dxdy$$

\item \textbf{Par rapport à l'axe $Oy$ :}
$$I_{Oy} = \iint_D \mu(x,y) x^2 \, dxdy$$

\item \textbf{Relation :} $I_O = I_{Ox} + I_{Oy}$
\end{itemize}
\end{Def}

\begin{Ex}
Calculer le moment d'inertie d'un disque homogène de rayon $r$ et de masse $M$ par rapport à son centre.

$$I_O = \frac{M}{\pi r^2} \iint_D (x^2+y^2) \, dxdy = \frac{M}{\pi r^2} \int_0^{2\pi} \int_0^r \rho^3 \, d\rho \, d\theta = \frac{M}{\pi r^2} \times 2\pi \times \frac{r^4}{4} = \frac{Mr^2}{2}$$
\end{Ex}

\begin{Thm}\textbf{Théorème de Huygens}

Soit $G$ le centre de gravité d'un solide de masse $M$. Si $I_G$ est le moment d'inertie par rapport à un axe passant par $G$, et $I_\Delta$ le moment d'inertie par rapport à un axe parallèle distant de $d$, alors :
$$I_\Delta = I_G + Md^2$$
\end{Thm}

\vspace{1em}\hrule\vspace{1em}

\exo[2]{\textbf{Centre d'inertie d'un demi-disque}}

Déterminer les coordonnées du centre d'inertie d'un demi-disque homogène de rayon $R$ (partie $y \geq 0$).

\vspace{1em}\hrule\vspace{1em}

\exo[2]{\textbf{Moment d'inertie d'un carré}}

Calculer le moment d'inertie d'un carré homogène de côté $a$ et de masse $M$ par rapport à un de ses sommets.

\vspace{1em}\hrule\vspace{1em}

\section{Changement de variables dans les intégrales doubles}

\subsection{Matrice jacobienne et jacobien}

\begin{Def}\textbf{Matrice jacobienne}

Soit $\varphi : U \to V$ un changement de variables défini par :
$$\varphi(u,v) = (x(u,v), y(u,v))$$

La \textbf{matrice jacobienne} de $\varphi$ au point $(u,v)$ est :
$$J_\varphi(u,v) = \begin{pmatrix} \dfrac{\partial x}{\partial u} & \dfrac{\partial x}{\partial v} \\ \\ \dfrac{\partial y}{\partial u} & \dfrac{\partial y}{\partial v} \end{pmatrix}$$

Le \textbf{jacobien} est le déterminant de cette matrice :
$$\text{Jac}_\varphi(u,v) = \det(J_\varphi) = \frac{\partial x}{\partial u} \cdot \frac{\partial y}{\partial v} - \frac{\partial x}{\partial v} \cdot \frac{\partial y}{\partial u}$$
\end{Def}

\begin{Thm}\textbf{Formule de changement de variables}

Soit $\varphi$ un $C^1$-difféomorphisme de $A$ sur $B$. Alors :
$$\iint_B f(x,y) \, dxdy = \iint_A f(x(u,v), y(u,v)) \cdot |\text{Jac}_\varphi(u,v)| \, dudv$$
\end{Thm}

\subsection{Coordonnées polaires}

\begin{Prop}\textbf{Jacobien en coordonnées polaires}

Pour $x = \rho \cos\theta$ et $y = \rho \sin\theta$ :
$$J_\varphi(\rho,\theta) = \begin{pmatrix} \cos\theta & -\rho\sin\theta \\ \sin\theta & \rho\cos\theta \end{pmatrix}$$
$$\text{Jac}_\varphi(\rho,\theta) = \rho\cos^2\theta + \rho\sin^2\theta = \rho$$
\end{Prop}

\subsection{Coordonnées sphériques (dans $\mathbb{R}^3$)}

\begin{Prop}\textbf{Coordonnées sphériques}

$$\begin{cases} x = r\cos\theta\cos\varphi \\ y = r\sin\theta\cos\varphi \\ z = r\sin\varphi \end{cases}$$

Le jacobien est : $\text{Jac} = r^2 \cos\varphi$
\end{Prop}

\subsection{Coordonnées cylindriques (dans $\mathbb{R}^3$)}

\begin{Prop}\textbf{Coordonnées cylindriques}

$$\begin{cases} x = r\cos\theta \\ y = r\sin\theta \\ z = z \end{cases}$$

Le jacobien est : $\text{Jac} = r$
\end{Prop}

\begin{Ex}
Calculer $I = \displaystyle\iint_D (x+y) \, dxdy$ où $D$ est le domaine délimité par $x = 0$, $y = x$, $y = 1-x$.

\vspace{1em}

\textbf{Méthode directe :}
$$I = \int_0^{0.5} \left(\int_x^{1-x} (x+y) \, dy\right) dx = \frac{1}{6}$$

\vspace{1em}

\textbf{Avec changement de variables :} $u = x+y$, $v = x-y$

On a $x = \frac{u+v}{2}$ et $y = \frac{u-v}{2}$.

Le jacobien vaut : $\left|\frac{\partial(x,y)}{\partial(u,v)}\right| = \left|\begin{vmatrix} 1/2 & 1/2 \\ 1/2 & -1/2 \end{vmatrix}\right| = \frac{1}{2}$

Le domaine devient : $D' = \{(u,v) \mid 0 \leq u \leq 1, -u \leq v \leq 0\}$

$$I = \int_0^1 \left(\int_{-u}^0 u \, dv\right) \frac{1}{2} \, du = \frac{1}{2} \int_0^1 u^2 \, du = \frac{1}{6}$$
\end{Ex}

\vspace{1em}\hrule\vspace{1em}

\section{Exercices récapitulatifs}

\exo[2]{\textbf{Domaine}}

Soit $D = \{(x,y) \mid x \geq 0, y \geq 0, x+y \leq 1\}$.

Calculer $\displaystyle\iint_D f(x,y) \, dxdy$ dans les cas suivants :
\begin{enumerate}
\item $f(x,y) = x^2 + y^2$
\item $f(x,y) = xy(x+y)$
\end{enumerate}

\vspace{1em}\hrule\vspace{1em}

\exo[2]{\textbf{Calculs d'intégrales doubles}}

Calculer $\displaystyle\iint_D f(x,y) \, dxdy$ dans les cas suivants :

\begin{enumerate}
\item $f(x,y) = x$ et $D = \{(x,y) \mid y \geq 0, x-y+1 \geq 0, x+2y-4 \leq 0\}$
\item $f(x,y) = x+y$ et $D = \{(x,y) \mid 0 \leq x \leq 1, x^2 \leq y \leq x\}$
\item $f(x,y) = \cos(xy)$ et $D = \{(x,y) \mid 1 \leq x \leq 2, 0 \leq xy \leq \frac{\pi}{2}\}$
\item $f(x,y) = \frac{1}{(x+y)^3}$ et $D = \{(x,y) \mid 1 < x < 3, y > 2, x+y < 5\}$
\end{enumerate}

\vspace{1em}\hrule\vspace{1em}

\exo[2]{\textbf{Aire d'un domaine}}

Soit $D = \{(x,y) \mid -1 \leq x \leq 1 \text{ et } x^2 \leq y \leq 4-x^3\}$.

Calculer l'aire de $D$.

\vspace{1em}\hrule\vspace{1em}

\exo[2]{\textbf{Coordonnées polaires}}

Calculer $\displaystyle\iint_\Delta \frac{1}{1+x^2+y^2} \, dxdy$ où :
$$\Delta = \{(x,y) \mid 0 \leq x \leq 1, 0 \leq y \leq 1, x^2+y^2 \leq 1\}$$

\vspace{1em}\hrule\vspace{1em}

\exo[2]{\textbf{Disque décentré}}

Soit $D = \{(x,y) \mid x^2+y^2-2x \leq 0\}$.
\begin{enumerate}
\item Montrer que $D$ est un disque (préciser son centre et son rayon).
\item Calculer $\displaystyle\iint_D \sqrt{x^2+y^2} \, dxdy$.
\end{enumerate}

\vspace{1em}\hrule\vspace{1em}

\exo[3]{\textbf{Changement de variables}}

Soit $D = \{(x,y) \mid x < y < 2x, x < y^2 < 2x\}$.

Calculer $\displaystyle\iint_D \frac{y}{x} \, dxdy$ en utilisant le changement de variables $u = x/y$ et $v = y^2/x$.

\vspace{1em}\hrule\vspace{1em}

\exo[3]{\textbf{Centre de gravité d'un quart d'ellipse}}

Soit $a, b > 0$. Calculer les coordonnées du centre de gravité du domaine :
$$D = \left\{(x,y) \mid \frac{x^2}{a^2}+\frac{y^2}{b^2} \leq 1, x \geq 0 \text{ et } y \geq 0\right\}$$

\vspace{1em}\hrule\vspace{1em}

\exo[3]{\textbf{Volume d'un ellipsoïde}}

Déterminer le volume intérieur à l'ellipsoïde d'équation :
$$\frac{x^2}{a^2}+\frac{y^2}{b^2}+\frac{z^2}{c^2} = 1$$
où $a$, $b$ et $c$ sont trois réels strictement positifs.

\vspace{1em}\hrule\vspace{1em}

\exo[3]{\textbf{Centre de gravité d'une demi-boule}}

Déterminer le centre de gravité d'une demi-boule homogène de rayon $R$.

\vspace{1em}\hrule\vspace{1em}

\exo[3]{\textbf{Intégrale triple}}

Calculer $\displaystyle\iiint_D f(x,y,z) \, dxdydz$ pour :

\begin{enumerate}
\item $f(x,y,z) = \cos x$ et $D = \{(x,y,z) \mid x^2+y^2+z^2 < 1\}$
\item $f(x,y,z) = \frac{z}{\sqrt{x^2+y^2}}$ et $D = \{(x,y,z) \mid x^2+y^2 \leq a^2 \text{ et } 0 < z < a\}$
\end{enumerate}

\vspace{1em}\hrule\vspace{1em}

\exo[3]{\textbf{Calcul d'une intégrale simple par intégrale double}}

Soit $I = \displaystyle\int_0^1 \frac{\ln(1+x)}{1+x^2} \, dx$.

\begin{enumerate}
\item Montrer que pour tout $x \in ]-1, +\infty[$, on a : $\ln(1+x) = \displaystyle\int_0^1 \frac{x \, dy}{1+xy}$.

En déduire que $I = \displaystyle\iint_D \frac{x}{(1+x^2)(1+xy)} \, dxdy$, où $D = [0,1]^2$.

\item En intervertissant les rôles de $x$ et $y$, montrer que :
$$2I = \iint_D \frac{x+y}{(1+x^2)(1+y^2)} \, dxdy$$

En déduire que $I = \frac{\pi}{8}\ln 2$.
\end{enumerate}

\end{document}
