
\section{Continuité}


On définit une fonction de cette manière :
\[f(x)=
\left\{
\begin{array}{cc}
(x-2)^2 &\text{ si } x<5\\
a&\text{ si } x=5\\
(2x-b)^2&\text{ si } x>5 \\
\end{array}
\right.
\]


\begin{enumerate}
     \item 
        Déterminer $a$ et $b$ pour que $f$ soit continue sur $\mathbb{R}$.\\
        
\vspace{1em}

        \textbf{Correction:}\\

        $ \underset{{x\overset{x<5}{\to} 5}}{\lim}((x-2)^2)=9$\\
        Pour que $f$ soit continue en $5$ à  gauche, il faut et il suffit que $a=9$.\\
        $ \underset{{x\overset{x>5}{\to} 5}}{\lim}((2x-b)^2)=(10-b)^2$\\
        Pour que $f$ soit continue en $5$ à  droite, il faut et il suffit que $(10-b)^2=9$.\\
        On obtient $b=7$ ou $b=13$.
        
        La fonction $f$ est prolongeable par continuité en $0$ en une focntion $\tilde{f}$ avec $\tilde{f}(5)=9$.
        
    \item La fonction $f$ est-elle dérivable au point $x=5$ ?\\
    
    Limite du taux d'accroissement à gauche:\\
     $ \underset{{x\overset{x<5}{\to} 5}}{\lim}\left(\dfrac{(x-2)^2-9}{x-5}\right)= \underset{{x\overset{x<5}{\to} 5}}{\lim}\left(\dfrac{(x+1)(x-5)}{x-5}\right)=6$\\
     
     Limite du taux d'accroissement à droite:\\
     $ \underset{{x\overset{x<5}{\to} 5}}{\lim}\left(\dfrac{(2x-b)^2-9}{x-5}\right)= \underset{{x\overset{x<5}{\to} 5}}{\lim}\left(\dfrac{(2x-b-3)(2x-b+3)}{x-5}\right)$\\
     
     Si $b=7$, $\underset{{x\overset{x<5}{\to} 5}}{\lim}\left(\dfrac{(2x-7-3)(2x-7+3)}{x-5}\right)=2\times 6=12$\\
          Si $b=13$, $\underset{{x\overset{x<5}{\to} 5}}{\lim}\left(\dfrac{(2x-13-3)(2x-13+3)}{x-5}\right)=-6\times 2=-12$\\
          
          
          $\tilde{f}$ n'est pas dérivable en $5$, comme les limites du taux d'accroisssement à gauche et à droite ne sont pas égales.\\
\end{enumerate}


\noindent\textbf{Exercice  : 2 points}\\



On considère les fonctions $f$ et $g$ définies par les expresssions ci-dessous:\\
$f(x)=\dfrac{1}{x-4}$ et $g(x)=\ln(2x-1)$

\begin{enumerate}
     \item 
        Déterminer les ensembles de définition de $f$ et $g$.\\
        
         La fonction $f$ est définie sur $\mathbb{R}- \{4\}$.  La fonction $f$ est définie sur $\left ] \dfrac{1}{2};+\infty\right[$
    \item Déterminer la fonction $f\circ g$ et son ensemble de définition.\\
    
    La fonction $f\circ g$ est définie par $f\circ g(x) =\dfrac{1}{\ln(2x-1)-4}$ sur l'ensemble des réels tels que $\ln(2x-1)-4\neq0$ et $2x-1\geq 0$.
\end{enumerate}

\section{Continuité, 2eme version}



On définit une fonction de cette manière :
\[f(x)=
\left\{
\begin{array}{cc}
\dfrac{x^2-4}{x-2} &\text{ si } x<2\\

x+2&\text{ si } x>2 \\
\end{array}
\right.
\]


\begin{enumerate}
     \item 
        Démontrer que $f$ est continue en $2$.
        
       
        
       % La fonction $f$ est prolongeable par continuité en $0$ en une focntion $\tilde{f}$ avec $\tilde{f}(5)=9$.
        
    \item La fonction $f$ est-elle dérivable au point $x=2$ ?\\
    
 
\end{enumerate}


\noindent\textbf{Exercice  : 2 points}\\



On considère les fonctions $f$ et $g$ définies par les expresssions ci-dessous:\\
$f(x)=\ln(3x-3)$ et $g(x)=\dfrac{1}{2x+5}$

\begin{enumerate}
     \item 
        Déterminer les ensembles de définition de $f$ et $g$.\\
        
 
    \item Déterminer la fonction $f\circ g$ et son ensemble de définition.\\
    
\end{enumerate}

\section{Equations Différentielles}


Résoudre les équations différentielles suivantes


\begin{enumerate}
    \item 
    $y''+y=0$.
    \item
    $z''-6z+9=0$.
    \item
    $3y''-4y'=0$.
\end{enumerate}

\section{Intégrale double}


Calculer l'intégrale suivante après avoir tracé le domaine

\begin{equation}
I = \int\int_D e^{-x-y} \,dx\,dy, \quad \text{où } D = \{(x, y) \in \mathbb{R}^2 \mid 0 \leq x \leq 1, \ 1 \leq y \leq x+3\}.
\end{equation}




%\section*{Correction}
%1 point tracé
%1 point ordre d'intégration
%1 point primitive
% 2 points fin de calcul
%\begin{center}
%\begin{tikzpicture}[scale=1.5]
%    % Axes
%    \draw[->] (-0.5,0) -- (2,0) node[right] {$x$};
%    \draw[->] (0,-0.5) -- (0,4.5) node[above] {$y$};
%    
%    % Domaine
%    \fill[blue!20, opacity=0.5] (0,1) -- (1,4) -- (1,1) -- cycle;
%    
%    % Bords du domaine
%    \draw[thick, blue] (0,1) -- (1,4);
%    \draw[thick, blue] (1,1) -- (1,4);
%    \draw[thick, blue] (0,1) -- (1,1);
%    
%    % Labels
%    \node[left] at (0,1) {1};
%    \node[below] at (1,0) {1};
%    %\node[right] at (1,4) {$(1,4)$};
%    
%    % Equation de la droite
%    \draw[dashed] (0,3) -- (1,4);
%    \node[right] at (0.5,3.5) {\small $y = x+3$};
%\end{tikzpicture}
%\end{center}

%\section*{Calcul de l'intégrale}
%
%L'intégrale double est donnée par :
%\begin{align*}
%I &= \int_0^1 \int_1^{x+3} e^{-x-y} \, dy \, dx.
%\end{align*}
%
%Calculons l'intégrale intérieure :
%\begin{align*}
%\int_1^{x+3} e^{-x-y} \, dy &= e^{-x} \int_1^{x+3} e^{-y} \, dy \\
%&= e^{-x} \left[ -e^{-y} \right]_1^{x+3} \\
%&= e^{-x} \left( -e^{-(x+3)} + e^{-1} \right) \\
%&= e^{-x} e^{-1} - e^{-x} e^{-(x+3)} \\
%&= e^{-x-1} - e^{-2x-3}.
%\end{align*}
%
%Intégrons maintenant par rapport à \( x \) :
%\begin{align*}
%I &= \int_0^1 \left( e^{-x-1} - e^{-2x-3} \right) dx \\
%&= e^{-1} \int_0^1 e^{-x} \,dx - e^{-3} \int_0^1 e^{-2x} \,dx.
%\end{align*}
%
%Les intégrales élémentaires donnent :
%\begin{align*}
%\int_0^1 e^{-x} \,dx &= \left[ -e^{-x} \right]_0^1 = 1 - e^{-1}, \\
%\int_0^1 e^{-2x} \,dx &= \left[ -\frac{e^{-2x}}{2} \right]_0^1 = \frac{1 - e^{-2}}{2}.
%\end{align*}
%
%Ainsi, on obtient :
%\begin{align*}
%I &= e^{-1} (1 - e^{-1}) - e^{-3} \frac{1 - e^{-2}}{2} \\
%&= e^{-1} - e^{-2} - \frac{e^{-3} (1 - e^{-2})}{2} \\
%&= e^{-1} - e^{-2} - \frac{e^{-3} - e^{-5}}{2}.
%\end{align*}

\section{}