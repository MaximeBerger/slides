\documentclass[10pt]{article}

\usepackage[utf8]{inputenc}
\usepackage[french]{babel}
\usepackage{graphicx}
%\usepackage{wrapfig}
\usepackage{dsfont}
\usepackage[margin=2cm]{geometry}
\usepackage{amsmath,amssymb}
%\usepackage{tikz}
\usepackage{multicol}
\newcommand{\xB}{{\cal B}}
\newcommand{\xC}{{\cal C}}
\newcommand{\R}{\mathbf{R}}

\begin{document}

\pagestyle{empty}

\noindent
\begin{minipage}[l]{8cm}
  \scriptsize{ESTP, S1\\Ann\'ee universitaire 2024/2025}
\end{minipage}

\begin{center}
  {\large\textbf{Petit test \no 2} \\
10 minutes\\}
  \bigskip
  

 
\end{center}

\bigskip

\begin{center}
  \fbox{%
    \begin{minipage}{0.95\linewidth}
      NOM: \hspace*{5cm} Pr\'enom: \hspace*{4cm} Groupe: TD 
    \end{minipage}
  }
  \end{center}

\bigskip
 
 

\noindent\textbf{Exercice 1  : 3 points}\\



On définit une fonction de cette manière :
\[f(x)=
\left\{
\begin{array}{cc}
\dfrac{x^2-4}{x-2} &\text{ si } x<2\\

x+2&\text{ si } x>2 \\
\end{array}
\right.
\]


\begin{enumerate}
     \item 
        Démontrer que $f$ est continue en $2$.
        
       
        
       % La fonction $f$ est prolongeable par continuité en $0$ en une focntion $\tilde{f}$ avec $\tilde{f}(5)=9$.
        
    \item La fonction $f$ est-elle dérivable au point $x=2$ ?\\
    
 
\end{enumerate}


\noindent\textbf{Exercice  : 2 points}\\



On considère les fonctions $f$ et $g$ définies par les expresssions ci-dessous:\\
$f(x)=\ln(3x-3)$ et $g(x)=\dfrac{1}{2x+5}$

\begin{enumerate}
     \item 
        Déterminer les ensembles de définition de $f$ et $g$.\\
        
 
    \item Déterminer la fonction $f\circ g$ et son ensemble de définition.\\
    
\end{enumerate}
\end{document}
