% Template pour les corrigés d'exercices
% Compiler avec : pdflatex corrige-template.tex

\documentclass[12pt,a4paper]{article}

% Langue et encodage
\usepackage[T1]{fontenc}
\usepackage{lmodern}
\usepackage[french]{babel}

% Mise en page compacte
\usepackage{geometry}
\geometry{top=1.5cm, bottom=1.5cm, left=2cm, right=2cm}

% Mathématiques
\usepackage{amsmath,amssymb}
\usepackage{multicol}

% Couleurs et style
\usepackage{xcolor}
\usepackage{fancyhdr}
\pagestyle{fancy}
\fancyhf{}
\renewcommand{\headrulewidth}{1pt}
\renewcommand{\headrule}{\hbox to\headwidth{\color{orange}\leaders\hrule height \headrulewidth\hfill}}

% === MÉTADONNÉES DU CORRIGÉ (À MODIFIER) ===
\newcommand{\numchapitre}{2}
\newcommand{\nomchapitre}{Intégration}
\newcommand{\numexo}{2}
\newcommand{\titreexo}{Calculs directs de primitives}
% ===========================================

\lhead{\textbf{Chapitre \numchapitre} -- \nomchapitre}
\rhead{Exercice \numchapitre.\numexo}
\cfoot{\thepage}

% Raccourcis mathématiques (copie de configPoly)
\newcommand{\R}{\mathbb{R}}
\newcommand{\ds}{\displaystyle}

\begin{document}

\begin{center}
\colorbox{orange!20}{\parbox{0.9\textwidth}{
\centering\large\bfseries
Corrigé -- Exercice \numchapitre.\numexo : \titreexo
}}
\end{center}

\vspace{1em}

% === CONTENU DU CORRIGÉ ===

\textbf{Énoncé :} Déterminer une primitive des fonctions suivantes.

\vspace{0.5em}

\begin{enumerate}
\item $f(x) = 3x + 1$

\textcolor{teal}{\textbf{Solution :}} Une primitive est $F(x) = \dfrac{3x^2}{2} + x + C$.

\item $f(x) = 3x^2 - x + 1$

\textcolor{teal}{\textbf{Solution :}} Une primitive est $F(x) = x^3 - \dfrac{x^2}{2} + x + C$.

\item $f(x) = \dfrac{1}{x} + 2$

\textcolor{teal}{\textbf{Solution :}} Une primitive est $F(x) = \ln|x| + 2x + C$ (pour $x \neq 0$).

\item $f(x) = e^x + 1$

\textcolor{teal}{\textbf{Solution :}} Une primitive est $F(x) = e^x + x + C$.

\item $f(x) = \dfrac{3}{x+1} - e^{2x+1}$

\textcolor{teal}{\textbf{Solution :}} Une primitive est $F(x) = 3\ln|x+1| - \dfrac{1}{2}e^{2x+1} + C$.

\item $f(t) = t \cdot e^{t^2+1}$

\textcolor{teal}{\textbf{Solution :}} On reconnaît la forme $u'e^u$ avec $u = t^2 + 1$ et $u' = 2t$.

Donc $F(t) = \dfrac{1}{2}e^{t^2+1} + C$.

\item $f(t) = \dfrac{e^t}{e^t+1}$

\textcolor{teal}{\textbf{Solution :}} On reconnaît la forme $\dfrac{u'}{u}$ avec $u = e^t + 1$ et $u' = e^t$.

Donc $F(t) = \ln(e^t + 1) + C$.

\item $f(t) = \dfrac{e^{2t}-1}{e^t}$

\textcolor{teal}{\textbf{Solution :}} On simplifie : $f(t) = e^t - e^{-t}$.

Donc $F(t) = e^t + e^{-t} + C$.

\item $f(t) = \dfrac{t^2-t}{t^3}$

\textcolor{teal}{\textbf{Solution :}} On simplifie : $f(t) = \dfrac{1}{t} - \dfrac{1}{t^2}$.

Donc $F(t) = \ln|t| + \dfrac{1}{t} + C$.

\end{enumerate}

\end{document}
