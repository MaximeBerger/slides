
\subsection{Des exemples}
Donner deux exemples différents dans chacune des situations suivantes :
\begin{enumerate}[label = $\square$, itemsep = 0.5em]
  \item une suite décroissante positive dont le terme général ne tend pas vers 0.

\ifthenelse{\boolean{showSolutions}}{
    \vspace{1em}

    \begin{mdframed}
    Par exemple, $u_n = 1+ \frac{1}{n}$.
    \end{mdframed}
}{}
  \item une suite bornée non convergente.

\ifthenelse{\boolean{showSolutions}}{
    \vspace{1em}

    \begin{mdframed}
    Par exemple, $u_n = (-1)^n$.
    \end{mdframed}
}{}
  \item une suite positive non bornée ne tendant pas vers $+\infty$.

\ifthenelse{\boolean{showSolutions}}{
    \vspace{1em}

    \begin{mdframed}
    Par exemple, $u_n = (-2)^n$.
    \end{mdframed}
}{}
  \item une suite non monotone qui tend vers 0.

\ifthenelse{\boolean{showSolutions}}{
    \vspace{1em}

    \begin{mdframed}
    Par exemple, $u_n = \dfrac{(-1)^n}{2^n}$.
    \end{mdframed}
}{}
  \item une suite positive qui tend vers 0 et qui n'est pas décroissante.

\ifthenelse{\boolean{showSolutions}}{
    \vspace{1em}

    \begin{mdframed}
    Par exemple, $u_n = -\dfrac{1}{n}$, ou $u_n = \dfrac{(-1)^n}{n}$.
    \end{mdframed}
}{}
\end{enumerate}



\vspace{1em}

\subsection{Vrai ou Faux ?}
  Dire si les assertions suivantes sont vraies ou fausses. On justifiera les réponses avec une démonstration ou un contre-exemple.
  \begin{enumerate}[label = $\square$, itemsep = 0.5em]
    \item Toute suite non-majorée tend vers $+\infty$.

    \ifthenelse{\boolean{showSolutions}}{
        \vspace{1em}

    \begin{mdframed}
    Faux, la suite $u_n = n (-1)^n$ n'est pas majorée et ne tend pas vers l'infini.
    \end{mdframed}
    }{
    }
    \item Soit $\left(u_n\right)_{n \geq 0}$ une suite à termes positifs convergeant vers $0$. Alors, $(u_n)$ est décroissante à partir d'un certain rang.

    \ifthenelse{\boolean{showSolutions}}{
        \vspace{1em}

    \begin{mdframed}
    Faux, on forme par exemple la suite 
    $$
    u_n = \begin{cases}
    \dfrac{1}{n} & \text{si } n \text{ est pair} \\
    \dfrac{1}{n^2} & \text{si } n \text{ est impair}
    \end{cases}
    $$
    qui converge vers $0$ mais n'est pas décroissante à partir d'un certain rang.
    \end{mdframed}
    }{
    }
    \item Si $(u_n)$ est une suite géométrique de raison $q \neq 0$, alors $\left(\frac{1}{u_n}\right)$ est une suite géométrique de raison~$1/q$.

    \ifthenelse{\boolean{showSolutions}}{
        \vspace{1em}

    \begin{mdframed}
    Vrai, si $u_n = u_0 q^n$, alors $\frac{1}{u_n} = \frac{1}{u_0} q^{-n} = \frac{1}{u_0} \left(\frac{1}{q}\right)^n$.
    \end{mdframed}
    }{
    }
    \item Soit $(u_n)$ une suite croissante et $\ell \in \mathbb{R}$. Si pour tout $N \in \mathbb{N}$, il existe $n_0 \geq N$ tel que $u_{n_0} > \ell$, alors $(u_n)$ ne converge pas vers $\ell$.

    \ifthenelse{\boolean{showSolutions}}{
        \vspace{1em}

    \begin{mdframed}
    Faux, la suite $u_n = \dfrac{(-1)^n}{n}$ converge vers $\ell = 0$, pourtant il existe une infinité de valeurs de $(u_n)$ qui sont strictement supérieures à $\ell$.
    \end{mdframed}
    }{
    }
    \item Si $f: \mathbb{R} \rightarrow \mathbb{R}$ est croissante et que $(u_n)$ vérifie $u_{n+1}=f\left(u_n\right)$ pour tout entier $n$, alors $(u_n)$ est croissante.
    
    \ifthenelse{\boolean{showSolutions}}{
        \vspace{1em}

    \begin{mdframed}
    Faux, on peut prendre par exemple $f(x) = x^2$ et $u_0 \in  ]0 , 1[$.
    \end{mdframed}
    }{
    }
    \item Si $(u_n)$ est divergente, alors $(u_n)$ est non bornée.

    \ifthenelse{\boolean{showSolutions}}{
        \vspace{1em}

    \begin{mdframed}
    Faux, la suite $u_n = (-1)^n$ est divergente mais bornée.
    \end{mdframed}
    }{
    }
    \item Si $u_n\to \ell$ et $f$ continue, alors $f(u_n)\to f(\ell)$

    \ifthenelse{\boolean{showSolutions}}{
        \vspace{1em}

    \begin{mdframed}
    Vrai, c'est la définition d'une fonction continue: $\lim_{x \to \ell} f(x) = f(\ell)$.
    \end{mdframed}
    }{
    }
  \end{enumerate}

  \vspace{1em}


  \subsection{Étude de suites}
  Étudier la nature des suites suivantes:
  \ifthenelse{\boolean{showSolutions}}{}{
  \begin{multicols}{3}}
  \begin{enumerate}[label = \alph*), itemsep = 0.5em]
    \item $\displaystyle u_n=\frac{\sin (n)+3 \cos \left(n^2\right)}{\sqrt{n}}$

    \ifthenelse{\boolean{showSolutions}}{
        \vspace{1em}

    \begin{mdframed}
    Le numérateur est borné car $|\sin(n)| \leq 1$ et $|3 \cos(n^2)| \leq 3$. Le dénominateur tend vers $+\infty$. Donc la suite tend vers 0.
    \end{mdframed}
    }{
    }
    \item $\displaystyle u_n=\frac{2 n+(-1)^n}{5 n+(-1)^{n+1}}$

    \ifthenelse{\boolean{showSolutions}}{
        \vspace{1em}

    \begin{mdframed}
        On conserve les termes dominants : 
        \[
        2n + (-1)^n = 2n \big( 1 + \frac{(-1)^n}{2n} \big) = 2n \big(1+ o(1)\big)
        \]
        $u_n$ est équivalente à $\frac{2 n}{5 n} = \frac{2}{5}$. donc la suite tend vers $\frac{2}{5}$.
    \end{mdframed}
    }{
    }
    \item $\displaystyle u_n=\frac{n^3+5 n}{4 n^2+\sin (n)+\ln (n)}$

    \ifthenelse{\boolean{showSolutions}}{
        \vspace{1em}

    \begin{mdframed}
        On fait de même, on conserve les termes dominants : 
        \[
        n^3 + 5n = n^3 \big( 1 + \frac{5}{n^2} \big) = n^3 \big(1 + o(1)\big)
        \]
        Pour le dénominateur : 
        \[
        4 n^2 + \sin(n) + \ln(n) = 4 n^2 \big( 1 + \frac{\sin(n)}{4 n^2} + \frac{\ln(n)}{4 n^2} \big) = 4 n^2 \big(1+ o(1)\big)
        \]
        $u_n$ est équivalente à $\frac{n^3}{4 n^2} = \frac{n}{4}$. donc la suite tend vers $+\infty$.
    \end{mdframed}
    }{
    }
    \item $\displaystyle u_n=\sqrt{2 n+1}-\sqrt{2 n-1}$

    \ifthenelse{\boolean{showSolutions}}{
        \vspace{1em}

    \begin{mdframed}
        On mutliplie en haut et en bas par la quantité conjuguée pour reconnaître une identité remarquable :
        \[
        u_n = \frac{\big(\sqrt{2 n+1}-\sqrt{2 n-1}\big)\big(\sqrt{2 n+1}+\sqrt{2 n-1}\big)}{\sqrt{2n + 1} + \sqrt{2n - 1}} 
        = \frac{2n + 1 - (2n - 1)}{\sqrt{2n + 1} + \sqrt{2n - 1}} = \frac{2}{\sqrt{2n + 1} + \sqrt{2n - 1}}
        \]
        Comme le dénominateur est équivalent à $2\sqrt{2n}$, on a $u_n$ est équivalent à $\frac{2}{2\sqrt{2n}} = \frac{1}{\sqrt{2n}}$. Donc la suite tend vers 0.
    \end{mdframed}
    }{
    }
    \item $\displaystyle u_n=3^n e^{-3 n}$.

    \ifthenelse{\boolean{showSolutions}}{
        \vspace{1em}

    \begin{mdframed}
        Par croissances comparées, l'exponentielle l'emporte sur les puissances donc la suite tend vers 0.
    \end{mdframed}
    }{
    }
    \item $\displaystyle u_n=\frac{n}{2^n}$

    \ifthenelse{\boolean{showSolutions}}{
        \vspace{1em}

        \begin{mdframed}
        Par croissances comparées, les exponentielles (ici $2^n$) l'emportent sur les puissances donc la suite tend vers 0.
    \end{mdframed}
    }{
    }
    \item $\displaystyle u_n=\frac{n!}{45^n}$

    \ifthenelse{\boolean{showSolutions}}{
        \vspace{1em}

        \begin{mdframed}
        Par croissances comparées, les factorielles l'emportent sur les puissances donc la suite tend vers l'infini.
    \end{mdframed}
    }{
    }
    \item $\displaystyle u_n=\frac{n!}{n^n}$

    \ifthenelse{\boolean{showSolutions}}{
        \vspace{1em}

        \begin{mdframed}
        On peut étudier le rapport $u_{n+1}/u_n$ :
        \[
        \frac{u_{n+1}}{u_n} = \frac{(n+1)!}{(n+1)^{n+1}} \cdot \frac{n^n}{n!} = \frac{n+1}{n+1} \cdot \left(\frac{n}{n+1}\right)^n = \left(\frac{n}{n+1}\right)^n
        \]
        Donc la suite est décroissante et tend vers 0.
    \end{mdframed}
    }{
    }
    \item $\displaystyle u_n=\frac{n^3+2^n}{n^2+3^n}$.

    \ifthenelse{\boolean{showSolutions}}{
        \vspace{1em}

        \begin{mdframed}
        On factorise par les termes dominants :
        \[
u_n = \frac{2^n(1 + \frac{n^3}{2^n})}{3^n(1 + \frac{n^2}{3^n})} = \frac{2^n(1 + o(1))}{3^n(1 + o(1))} 
        \]
        La suite est donc équivalente à $\frac{2^n}{3^n} = \left(\frac{2}{3}\right)^n$. Donc la suite tend vers 0.
    \end{mdframed}
    }{
    }
  \end{enumerate}
  \ifthenelse{\boolean{showSolutions}}{}{
  \end{multicols}}

  \vspace{1em}


  \subsection{*Plus difficile}
  Étudier la nature des suites suivantes, et déterminer un équivalent simple:
  \ifthenelse{\boolean{showSolutions}}{}{
  \begin{multicols}{2}}
  \begin{enumerate}[label = \alph*), itemsep = 1em]
    \item $u_n=\ln \left(2 n^2-n\right)-\ln (3 n+1)$

    \ifthenelse{\boolean{showSolutions}}{
        \vspace{1em}

        \begin{mdframed}
            On met en facteur le terme dominant dans chaque logarithme, de sorte que

            $$
            \begin{aligned}
            u_n & =\ln \left(2 n^2\left(1-\frac{1}{2 n}\right)\right)-\ln \left(3 n\left(1+\frac{1}{3 n}\right)\right) \\
            & =2 \ln n+\ln 2+\ln \left(1-\frac{1}{2 n}\right)-\ln (n)-\ln (3)-\ln \left(1+\frac{1}{3 n}\right) \\
            & =\ln n+\ln 2-\ln 3+v_n
            \end{aligned}
            $$
            
            où la suite $\left(v_n\right)$ tend vers 0 . On en déduit que $u_n$ tend vers $+\infty$.    \end{mdframed}
    }{
    }
    \item $u_n=\sqrt{n^2+n+1}-\sqrt{n^2-n+1}$

    \ifthenelse{\boolean{showSolutions}}{
        \vspace{1em}

        \begin{mdframed}
            On multiplie au numérateur et au dénominateur par la quantité conjuguée, de sorte que

            $$
            u_n=\frac{2 n}{\sqrt{n^2+n+1}+\sqrt{n^2-n+1}}
            $$
            
            
            On met encore en facteur, dans chaque racine carrée du dénominateur, le terme dominant (en $n^2$ ), et on trouve
            
            $$
            u_n=\frac{2}{\sqrt{1+\frac{1}{n}+\frac{1}{n^2}}+\sqrt{1-\frac{1}{n}+\frac{1}{n^2}}}
            $$
            
            
            Or, $\sqrt{1+\frac{1}{n}+\frac{1}{n^2}}$ tend vers 1 et $\sqrt{1-\frac{1}{n}+\frac{1}{n^2}}$ tend également vers 1 . On en déduit que $\left(u_n\right)$ converge vers 1.    \end{mdframed}
    }{
    }
    \item $_n=\frac{a^n-b^n}{a^n+b^n}, a, b \in] 0,+\infty[$

    \ifthenelse{\boolean{showSolutions}}{
        \vspace{1em}

        \begin{mdframed}
            Si $a=b$, alors $u_n=0$ pour tout $n$, et donc $\left(u_n\right)$ converge vers 0 . Si $a>b$, alors $a^n$ est prépondérant sur $b^n$ au sens que

            $$
            \frac{b^n}{a^n}=\left(\frac{b}{a}\right)^n \rightarrow 0
            $$
            
            puisque $|b / a|<1$. On factorise donc par $a^n$ au numérateur et au dénominateur :
            
            $$
            u_n=\frac{a^n\left(1-\left(\frac{b}{a}\right)^n\right)}{a^n\left(1+\left(\frac{b}{a}\right)^n\right)}=\frac{1-\left(\frac{b}{a}\right)^n}{1+\left(\frac{b}{a}\right)^n} .
            $$
            
            
            On en déduit que dans ce cas, ( $u_n$ ) converge vers 1. Si $b>a$, on factorise cette fois par $b^n$ et c'est $(a / b)^n$ qui converge vers 0 . On trouve :
            
            $$
            u_n=\frac{-1+\left(\frac{a}{b}\right)^n}{1+\left(\frac{a}{b}\right)^n} .
            $$
            
            ( $u_n$ ) converge donc vers -1 dans ce cas.    \end{mdframed}
    }{
    }
    \item $u_n=\frac{\ln \left(n+e^n\right)}{n}$

    \ifthenelse{\boolean{showSolutions}}{
        \vspace{1em}

        \begin{mdframed}
            On factorise par $e^n$ dans le logarithme. On obtient

            $$
            \begin{aligned}
            u_n & =\frac{\ln \left(e^n\left(1+n e^{-n}\right)\right)}{n} \\
            & =\frac{n+\ln \left(1+n e^{-n}\right)}{n} .
            \end{aligned}
            $$
            
            
            D'autre part, $n e^{-n}$ tend vers 0 (par exemple, on peut écrire $n e^{-n}=\frac{1}{\frac{e^n}{n}}$ et utiliser la comparaison des fonctions exponentielle et polynôme au voisinage de l'infini). Puisque la fonction ln est continue en 1 et $\ln (1)=0$, on en déduit que $\ln \left(1+n e^{-n}\right)$ tend vers 0. II vient $\ln \left(1+n e^{-n}\right) / n$ tend vers 0 , et donc la suite ( $u_n$ ) converge vers 1 .    \end{mdframed}
    }{
    }
    \item $u_n=\frac{\ln (1+\sqrt{n})}{\ln \left(1+n^2\right)}$.

    \ifthenelse{\boolean{showSolutions}}{
        \vspace{1em}

        \begin{mdframed}
            On factorise par le terme dominant dans chaque logarithme. On en déduit

            $$
            \begin{aligned}
            u_n & =\frac{\ln (\sqrt{n})+\ln \left(1+\frac{1}{\sqrt{n}}\right)}{\ln \left(n^2\right)+\ln \left(1+n^{-2}\right)} \\
            & =\frac{\frac{1}{2} \ln n+\ln \left(1+\frac{1}{\sqrt{n}}\right)}{2 \ln (n)+\ln \left(1+n^{-2}\right)} \\
            & =\frac{\frac{1}{2}+\frac{\ln \left(1+\frac{1}{\sqrt{n}}\right)}{\ln n}}{2+\frac{\ln \left(1+n^{-2}\right)}{\ln n}} .
            \end{aligned}
            $$
            
            
            Puisque $\ln \left(1+\frac{1}{\sqrt{n}}\right), \ln \left(1+n^{-2}\right)$ tendent vers $0,\left(u_n\right)$ converge vers $\frac{1}{4}$.    
        \end{mdframed}
    }{
    }
  \end{enumerate}
  \ifthenelse{\boolean{showSolutions}}{}{
  \end{multicols}}

  \vspace{1em}

  \subsection{Formule de Stirling}
  \begin{enumerate}[label = \alph*)]
    \item Soit $\left(x_n\right)$ une suite de réels et soit $\left(y_n\right)$ définie par $y_n=x_{n+1}-x_n$. \newline
    Démontrer que la série $\sum_n y_n$ et la suite $\left(x_n\right)$ sont de même nature.

    \ifthenelse{\boolean{showSolutions}}{
        \vspace{1em}

        \begin{mdframed}
        On a $y_n = x_{n+1} - x_n$. Donc
        $$\sum_{n=0}^N y_n = \sum_{n=0}^N (x_{n+1} - x_n) = x_{N+1} - x_0.$$
        Donc la série $\sum_n y_n$ et la suite $\left(x_n\right)$ sont de même nature.
    \end{mdframed}
    }{
    }
    \item On pose $(u_n)$ la suite définie par $u_n=\frac{n^n e^{-n} \sqrt{n}}{n!}$. \newline 
    A l'aide d'un développement limité, déterminer la nature de la série de terme général $v_n=\ln \left(\frac{u_{n+1}}{u_n}\right)$.

    \ifthenelse{\boolean{showSolutions}}{
        \vspace{1em}

        \begin{mdframed}
            2. Un petit calcul prouve que :

            $$
            v_n=\left(n+\frac{1}{2}\right) \ln \left(1+\frac{1}{n}\right)-1 .
            $$
            
            
            On effectue un développement limité de $v_n$-il faut pousser le développement du logarithme jusqu'à l'ordre 3 - et on a :
            
            $$
            v_n=\frac{1}{12 n^2}+o\left(\frac{1}{n^2}\right) .
            $$
            
            
            La série de terme général $v_n$ est donc convergente.
        \end{mdframed}
    }{
    }
    \item En déduire l'existence d'une constante $C>0$ telle que :
  \end{enumerate}
  $$
  n!\sim_{+\infty} C \sqrt{n} n^n e^{-n}
  $$

    \ifthenelse{\boolean{showSolutions}}{
        \vspace{1em}

        \begin{mdframed}
            On écrit $v_n=\ln \left(u_{n+1} / u_n\right)=\ln \left(u_{n+1}\right)-\ln \left(u_n\right)$. Puisque la série $\sum_n v_n$ converge, d'après la première question la suite $\left(\ln \left(u_n\right)\right)$ converge vers un réel $l_{\text {, }}$ et en passant à l'exponentielle, on trouve que la suite ( $u_n$ ) converge vers le réel $e^l$ qui est strictement positif. Revenant à la définition de $\left(u_n\right)$, ceci donne le résultat avec $C=e^{-l}$. 

    \end{mdframed}
    }{
    }

  \vspace{1em}


  \subsection{Télescopiques}
  \begin{enumerate}[label = \alph*)]
    \item Déterminer deux réels $a$ et $b$ tels que $\displaystyle
\frac{1}{k^2-1}=\frac{a}{k-1}+\frac{b}{k+1} .
$

    \ifthenelse{\boolean{showSolutions}}{
        \vspace{1em}

        \begin{mdframed}
            On met tout au même dénominateur, et on procède par identification :

            $$
            \frac{a}{k-1}+\frac{b}{k+1}=\frac{(a+b) k+(a-b)}{k^2-1} .
            $$
            
            
            On cherche donc $a$ et $b$ de sorte que
            
            $$
            \left\{\begin{array}{l}
            a+b=0 \\
            a-b=1
            \end{array}\right.
            $$
            
            
            On en déduit $a=1 / 2$ et $b=-1 / 2$.    \end{mdframed}
    }{
    }

    \item En déduire la limite de la suite $\displaystyle
u_n=\sum_{k=2}^n \frac{1}{k^2-1} .
$

    \ifthenelse{\boolean{showSolutions}}{
        \vspace{1em}

        \begin{mdframed}
            La somme est télescopique :

            $$
            \begin{aligned}
            u_n= & \frac{1}{2}\left(1+\frac{1}{2}+\cdots+\frac{1}{n-1}\right) \\
            & -\frac{1}{2}\left(\frac{1}{3}+\frac{1}{4}+\cdots+\frac{1}{n}+\frac{1}{n+1}\right)
            \end{aligned}
            $$
            
            soit
            
            $$
            u_n=\frac{1}{2}\left(1+\frac{1}{2}-\frac{1}{n}-\frac{1}{n+1}\right) .
            $$
            
            
            On en déduit que $\left(u_n\right)$ converge vers $\frac{3}{4}$.
            \end{mdframed}
    }{
    }

    \item Sur le même modèle, déterminer la limite de la suite $
v_n=\sum_{k=0}^n \frac{1}{k^2+3 k+2}
$

    \ifthenelse{\boolean{showSolutions}}{
        \vspace{1em}

        \begin{mdframed}
            L'idée est de factoriser $k^2+3 k+2=(k+1)(k+2)$. On cherche donc $a$ et $b$ tels que

            $$
            \frac{1}{k^2+3 k+2}=\frac{a}{k+1}+\frac{b}{k+2} .
            $$
            
            
            On trouve $a=1$ et $b=-1$. On en déduit que
            
            $$
            v_n=1-\frac{1}{n+2}
            $$
            
            et donc $\left(v_n\right)$ converge vers 1 .
        
        \end{mdframed}
    }{
    }


    \item Montrer que, pour tout $n \in \mathbb{N}^*$, on a $\sqrt{n+1}-\sqrt{n} \leq \frac{1}{2 \sqrt{n}}$

    \ifthenelse{\boolean{showSolutions}}{
        \vspace{1em}

        \begin{mdframed}
                Multipliant la différence de deux racines par la quantité conjuguée, on trouve
                $$\sqrt{n+1}-\sqrt{n}=\frac{(\sqrt{n+1}+\sqrt{n})(\sqrt{n+1}-\sqrt{n})}{\sqrt{n+1}+\sqrt{n}}=\frac{1}{\sqrt{n+1}+\sqrt{n}} \leq \frac{1}{2 \sqrt{n}} .$$
        \end{mdframed}
    }{
    }

    \item En déduire le comportement de la suite ( $u_n$ ) définie par $
u_n=1+\frac{1}{\sqrt{2}}+\cdots+\frac{1}{\sqrt{n}} .$
\ifthenelse{\boolean{showSolutions}}{
    \vspace{1em}

    \begin{mdframed}
        On somme alors ces inégalités, et les termes à gauche se télescopent :

        $$
        2(\sqrt{n+1}-\sqrt{1}) \leq u_n .
        $$
        
        
        Par le théorème de comparaison, on en déduit que ( $u_n$ ) diverge vers $+\infty$.
    \end{mdframed}
}{
}

  \end{enumerate}


\vspace{1em}

\section*{Séries numériques}
\subsection{Paramètres}
Soit $a, b \in \mathbb{R}$. Pour $n \geq 1$, on pose $u_n=\ln (n)+a \ln (n+1)+b \ln (n+2)$.
\begin{enumerate}[label = \alph*)]
  \item Pour quelle(s) valeur(s) de $(a, b)$ la série $\sum u_n$ est-elle convergente?



  \ifthenelse{\boolean{showSolutions}}{
        \vspace{1em}

        \begin{mdframed}
            Écrivons

            $$
            \ln (n+1)=\ln (n)+\ln \left(1+\frac{1}{n}\right)=\ln n+\frac{1}{n}+O\left(\frac{1}{n^2}\right)
            $$
            
            et
            
            $$
            \ln (n+2)=\ln (n)+\ln \left(1+\frac{2}{n}\right)=\ln n+\frac{2}{n}+O\left(\frac{1}{n^2}\right) .
            $$
            
            
            On obtient
            
            $$
            u_n=(1+a+b) \ln (n)+\frac{a+2 b}{n}++O\left(\frac{1}{n^2}\right) .
            $$
            
            
            Ainsi, si $1+a+b \neq 0, u_n \sim_{+\infty}(1+a+b) \ln (n)$ et la série diverge grossièrement. Si $1+a+b=0$ et $a+2 b \neq 0, u_n \sim_{+\infty} \frac{a+2 b}{n}$ et la série diverge par comparaison à une série de Riemann divergente. Si $1+a+b=0$ et $a+2 b=0$, alors $u_n=O\left(\frac{1}{n^2}\right)$ et la série converge absolument. Finalement, on a prouvé que $\sum u_n$ converge si et seulement si
            
            $$
            \left\{\begin{array} { r } 
            { 1 + a + b = 0 } \\
            { 2 a + b = 0 }
            \end{array} \Longleftrightarrow \left\{\begin{array}{l}
            a=-2 \\
            b=1
            \end{array}\right.\right.
            $$
            \end{mdframed}
    }{
    }

  \item Dans le(s) cas où la série converge, déterminer $\sum_{n=1}^{+\infty} u_n$.

  \ifthenelse{\boolean{showSolutions}}{
        \vspace{1em}

        \begin{mdframed}
            On traite donc le cas $a=-2$ et $b=1$. Notons $S_n=\sum_{k=1}^n u_k$. Alors

            $$
            \begin{aligned}
            S_n & =\sum_{k=1}^n \ln (k)-2 \sum_{k=1}^n \ln (k+1)+\sum_{k=1}^n \ln (k+2) \\
            & =\sum_{k=1}^n \ln (k)-\sum_{k=1}^n \ln (k+1)+\sum_{k=1}^n \ln (k+2)-\sum_{k=1}^n \ln (k+1) .
            \end{aligned}
            $$
            
            
            On reconnait deux couples de deux sommes télescopiques et on trouve
            
            $$
            S_n=\ln (1)-\ln (2)+\ln (n+2)-\ln (n+1) \rightarrow-\ln (2) .
            $$
            \end{mdframed}
    }{
    }
\end{enumerate}

\vspace{1em}

\subsection{Avec l'exponentielle}
Sachant que $e=\sum_{n \geq 0} \frac{1}{n!}$, déterminer la valeur des sommes suivantes :
  $$\sum_{n \geq 0} \frac{n+1}{n!}, \qquad \sum_{n \geq 0} \frac{n^2-2}{n!}, \qquad \sum_{n \geq 0} \frac{n^3}{n!}.$$


  \ifthenelse{\boolean{showSolutions}}{
        \vspace{1em}

        \begin{mdframed}
            1. La première somme ne pose pas de problèmes :

            $$
            \sum_{n \geq 0} \frac{n+1}{n!}=\sum_{n \geq 0} \frac{n}{n!}+\sum_{n \geq 0} \frac{1}{n!}=\sum_{n \geq 1} \frac{1}{(n-1)!}+e=2 e .
            $$
            
            2. La deuxième somme est plus compliquée à cause du terme en $n^2$. Pour qu'il se simplifie bien avec le $n!$, le plus commode est de l'écrire
            
            $$
            n^2=n(n-1)+n
            $$
            
            de sorte que
            
            $$
            \sum_{n \geq 0} \frac{n^2-2}{n!}=\sum_{n \geq 2} \frac{n(n-1)}{n!}+\sum_{n \geq 1} \frac{1}{(n-1)!}-2 \sum_{n \geq 0} \frac{1}{n!}=e+e-2 e=0 .
            $$
            
            3. La méthode est similaire. Dit de façon algébrique, on va décomposer le polynôme $X^3$ dans la base $1, X, X(X-1), X(X-1)(X-2)$. En raisonnant d'abord avec le terme de plus haut degré, puis celui juste après, etc..., on trouve :
            
            $$
            X^3=X(X-1)(X-2)+3 X(X-1)+X .
            $$
            
            
            On en déduit :
            
            $$
            \begin{aligned}
            \sum_{n \geq 1} \frac{n^3}{n!} & =\sum_{n \geq 1} \frac{n(n-1)(n-2)}{n!}+3 \sum_{n \geq 1} \frac{n(n-1)}{n!}+\sum_{n \geq 1} \frac{n}{n!} \\
            & =\sum_{n \geq 3} \frac{1}{(n-3)!}+\sum_{n \geq 2} \frac{3}{(n-2)!}+\sum_{n \geq 1} \frac{1}{(n-1)!} \\
            & =5 e .
            \end{aligned}
            $$
        
        \end{mdframed}
    }{
    }
