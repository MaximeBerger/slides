% Chapitre 3 : Indépendance

\section{Événements indépendants}

\begin{Def}
Deux événements $A$ et $B$ sont dits \textbf{indépendants} si :
$$\mathbb{P}(A \cap B) = \mathbb{P}(A) \times \mathbb{P}(B)$$
\end{Def}

\begin{Rmq}
C'est une définition \textbf{fondamentale} : la probabilité que $A$ soit réalisé ne change pas si on sait que $B$ est réalisé. En effet, si $A$ et $B$ sont indépendants et $\mathbb{P}(B) > 0$, alors :
$$\mathbb{P}(A | B) = \frac{\mathbb{P}(A \cap B)}{\mathbb{P}(B)} = \frac{\mathbb{P}(A) \mathbb{P}(B)}{\mathbb{P}(B)} = \mathbb{P}(A)$$
\end{Rmq}

\begin{Rmq}
Ce sera souvent difficile de prouver que deux événements sont indépendants ! Pour le faire, il faut :
\begin{enumerate}
    \item Calculer $\mathbb{P}(A)$
    \item Calculer $\mathbb{P}(B)$
    \item Calculer $\mathbb{P}(A \cap B)$
\end{enumerate}

Si le produit des deux premiers est égal au troisième, ils sont indépendants, sinon, ils ne le sont pas.
\end{Rmq}

\begin{Prop}
Si $A$ et $B$ sont indépendants, alors :
\begin{itemize}
    \item $A$ et $\overline{B}$ sont indépendants
    \item $\overline{A}$ et $B$ sont indépendants
    \item $\overline{A}$ et $\overline{B}$ sont indépendants
\end{itemize}
\end{Prop}

\begin{proof}
Montrons que $A$ et $\overline{B}$ sont indépendants :
\begin{align*}
\mathbb{P}(A \cap \overline{B}) &= \mathbb{P}(A) - \mathbb{P}(A \cap B) \\
&= \mathbb{P}(A) - \mathbb{P}(A) \mathbb{P}(B) \\
&= \mathbb{P}(A)(1 - \mathbb{P}(B)) \\
&= \mathbb{P}(A) \mathbb{P}(\overline{B})
\end{align*}
\end{proof}

\section{Indépendance mutuelle}

\begin{Def}
Les événements $A_1, A_2, \ldots, A_n$ sont dits \textbf{mutuellement indépendants} (ou simplement \textbf{indépendants}) si pour toute partie $I \subset \{1, 2, \ldots, n\}$ :
$$\mathbb{P}\left(\bigcap_{i \in I} A_i\right) = \prod_{i \in I} \mathbb{P}(A_i)$$
\end{Def}

\begin{Rmq}
Attention : l'indépendance deux à deux ne suffit pas pour avoir l'indépendance mutuelle. Il faut que toutes les intersections vérifient la propriété de factorisation.
\end{Rmq}

\begin{Ex}
On lance deux dés équilibrés. Notons :
\begin{itemize}
    \item $A$ : "le premier dé donne un nombre pair"
    \item $B$ : "le second dé donne un nombre pair"
    \item $C$ : "la somme des deux dés est paire"
\end{itemize}

Les événements $A$ et $B$ sont indépendants, $A$ et $C$ sont indépendants, $B$ et $C$ sont indépendants, mais $A$, $B$ et $C$ ne sont pas mutuellement indépendants car $\mathbb{P}(A \cap B \cap C) = \mathbb{P}(A \cap B) = 1/4$ alors que $\mathbb{P}(A) \mathbb{P}(B) \mathbb{P}(C) = (1/2)^3 = 1/8$.
\end{Ex}

