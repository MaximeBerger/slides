% Chapitre 5 : Variables aléatoires continues

\section{Définition}

\begin{Def}
Une variable aléatoire $X$ est dite \textbf{continue} si elle peut prendre toute valeur dans un intervalle (ou une réunion d'intervalles) de $\R$.
\end{Def}

\begin{Rmq}
Pour une variable continue, on ne peut généralement pas associer une probabilité à chaque valeur individuelle. On ne peut calculer que la probabilité d'obtenir une valeur comprise entre deux bornes.
\end{Rmq}

\section{Densité de probabilité}

\begin{Def}
Une variable aléatoire $X$ est dite \textbf{absolument continue} s'il existe une fonction $f_X : \R \to \R_+$ appelée \textbf{densité de probabilité} telle que pour tout intervalle $[a,b]$ :
$$\mathbb{P}(a \leq X \leq b) = \int_a^b f_X(x) \, dx$$
\end{Def}

\begin{Prop}
Si $X$ admet une densité $f_X$, alors :
\begin{itemize}
    \item $f_X(x) \geq 0$ pour tout $x \in \R$
    \item $\int_{-\infty}^{+\infty} f_X(x) \, dx = 1$
    \item Pour tout $x \in \R$ : $\mathbb{P}(X = x) = 0$
    \item La fonction de répartition est : $F_X(x) = \int_{-\infty}^x f_X(t) \, dt$
    \item Si $F_X$ est dérivable, alors $f_X(x) = F_X'(x)$ presque partout
\end{itemize}
\end{Prop}

\section{Fonction de répartition}

\begin{Def}
Pour une variable aléatoire continue $X$ de densité $f_X$, la \textbf{fonction de répartition} est :
$$F_X(x) = \mathbb{P}(X \leq x) = \int_{-\infty}^x f_X(t) \, dt$$
\end{Def}

\begin{Prop}
La fonction de répartition $F_X$ d'une variable continue vérifie :
\begin{itemize}
    \item $\lim_{x \to -\infty} F_X(x) = 0$
    \item $\lim_{x \to +\infty} F_X(x) = 1$
    \item $F_X$ est continue et croissante
    \item Pour $a < b$ : $\mathbb{P}(a < X < b) = F_X(b) - F_X(a) = \int_a^b f_X(x) \, dx$
\end{itemize}
\end{Prop}

\section{Espérance et variance}

\begin{Def}
L'\textbf{espérance} d'une variable aléatoire continue $X$ de densité $f_X$ est définie par :
$$\mathbb{E}[X] = \int_{-\infty}^{+\infty} x f_X(x) \, dx$$
à condition que cette intégrale soit absolument convergente.
\end{Def}

\begin{Def}
La \textbf{variance} d'une variable aléatoire continue $X$ est :
$$\Var(X) = \mathbb{E}[(X - \mathbb{E}[X])^2] = \int_{-\infty}^{+\infty} (x - \mathbb{E}[X])^2 f_X(x) \, dx = \mathbb{E}[X^2] - (\mathbb{E}[X])^2$$
\end{Def}

\begin{Def}
L'\textbf{écart-type} de $X$ est :
$$\sigma(X) = \sqrt{\Var(X)}$$
\end{Def}

\begin{Prop}[Linéarité de l'espérance]
Si $X$ et $Y$ sont deux variables aléatoires continues et $a, b \in \R$, alors :
$$\mathbb{E}[aX + bY] = a\mathbb{E}[X] + b\mathbb{E}[Y]$$
\end{Prop}

\section{Exemples de variables continues}

\begin{Ex}
\textbf{Loi uniforme sur $[a,b]$}

Une variable aléatoire $X$ suit une loi uniforme sur $[a,b]$ si sa densité est :
$$f_X(x) = \begin{cases}
\frac{1}{b-a} & \text{si } x \in [a,b] \\
0 & \text{sinon}
\end{cases}$$

On a :
\begin{itemize}
    \item $\mathbb{E}[X] = \frac{a+b}{2}$
    \item $\Var(X) = \frac{(b-a)^2}{12}$
\end{itemize}
\end{Ex}

\begin{Ex}
\textbf{Loi exponentielle de paramètre $\lambda > 0$}

Une variable aléatoire $X$ suit une loi exponentielle de paramètre $\lambda$ si sa densité est :
$$f_X(x) = \begin{cases}
\lambda e^{-\lambda x} & \text{si } x \geq 0 \\
0 & \text{si } x < 0
\end{cases}$$

On a :
\begin{itemize}
    \item $\mathbb{E}[X] = \frac{1}{\lambda}$
    \item $\Var(X) = \frac{1}{\lambda^2}$
    \item $F_X(x) = \begin{cases} 1 - e^{-\lambda x} & \text{si } x \geq 0 \\ 0 & \text{sinon} \end{cases}$
\end{itemize}
\end{Ex}

\begin{Ex}
\textbf{Loi normale (ou gaussienne) de paramètres $\mu$ et $\sigma^2$}

Une variable aléatoire $X$ suit une loi normale $\mathcal{N}(\mu, \sigma^2)$ si sa densité est :
$$f_X(x) = \frac{1}{\sigma\sqrt{2\pi}} \exp\left(-\frac{(x-\mu)^2}{2\sigma^2}\right)$$

On a :
\begin{itemize}
    \item $\mathbb{E}[X] = \mu$
    \item $\Var(X) = \sigma^2$
\end{itemize}
\end{Ex}

\section{Indépendance de variables aléatoires continues}

\begin{Def}
Deux variables aléatoires continues $X$ et $Y$ sont \textbf{indépendantes} si pour tous intervalles $I$ et $J$ de $\R$ :
$$\mathbb{P}(X \in I, Y \in J) = \mathbb{P}(X \in I) \times \mathbb{P}(Y \in J)$$
\end{Def}

\begin{Prop}
Si $X$ et $Y$ sont deux variables aléatoires continues indépendantes de densités respectives $f_X$ et $f_Y$, alors le couple $(X,Y)$ admet une densité jointe :
$$f_{(X,Y)}(x,y) = f_X(x) f_Y(y)$$
\end{Prop}

