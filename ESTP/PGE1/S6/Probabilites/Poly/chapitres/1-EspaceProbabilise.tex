% Chapitre 1 : Espace probabilisé

\section{Définitions de base}

\subsection{Univers et événements}

\begin{Def}
L'\textbf{univers} $\Omega$ est l'ensemble de toutes les issues possibles d'une expérience aléatoire.
\end{Def}

\begin{Def}
Un \textbf{événement} $A$ est une partie de l'univers $\Omega$, c'est-à-dire $A \subset \Omega$.
\end{Def}

\begin{Def}
Une \textbf{tribu} (ou $\sigma$-algèbre) $\mathcal{A}$ sur $\Omega$ est un ensemble de parties de $\Omega$ vérifiant :
\begin{itemize}
    \item $\Omega \in \mathcal{A}$
    \item Si $A \in \mathcal{A}$, alors $\overline{A} \in \mathcal{A}$ (stabilité par passage au complémentaire)
    \item Si $(A_n)_{n \in \N} \subset \mathcal{A}$, alors $\bigcup_{n \in \N} A_n \in \mathcal{A}$ (stabilité par réunion dénombrable)
\end{itemize}
\end{Def}

\subsection{Mesure de probabilité}

\begin{Def}
Une \textbf{mesure de probabilité} $\mathbb{P}$ sur $(\Omega, \mathcal{A})$ est une fonction $\mathbb{P} : \mathcal{A} \to [0,1]$ vérifiant :
\begin{itemize}
    \item $\mathbb{P}(\Omega) = 1$
    \item Pour toute famille dénombrable d'événements disjoints $(A_n)_{n \in \N}$, on a :
    $$\mathbb{P}\left(\bigcup_{n \in \N} A_n\right) = \sum_{n \in \N} \mathbb{P}(A_n)$$
    (propriété de $\sigma$-additivité)
\end{itemize}
\end{Def}

\begin{Def}
Un \textbf{espace probabilisé} est le triplet $(\Omega, \mathcal{A}, \mathbb{P})$ où :
\begin{itemize}
    \item $\Omega$ est l'univers
    \item $\mathcal{A}$ est une tribu sur $\Omega$
    \item $\mathbb{P}$ est une mesure de probabilité sur $(\Omega, \mathcal{A})$
\end{itemize}
\end{Def}

\section{Exemples d'espaces probabilisés}

\begin{Ex}
\textbf{Proportion de sol recouvert par de la peinture}

Vous jetez un pot de peinture par terre et mesurez la proportion de sol recouvert. L'univers est l'intervalle $\Omega = [0,1]$.

On ne peut pas associer une probabilité à chaque valeur individuelle, mais seulement calculer la probabilité d'obtenir une valeur comprise entre deux bornes. Par exemple, la probabilité de tomber sur un nombre entre $0.3$ et $0.7$ peut être définie comme la longueur de l'intervalle : $\mathbb{P}([0.3, 0.7]) = 0.4$.
\end{Ex}

\begin{Ex}
\textbf{Infinité de pièces de monnaie}

On lance une infinité de pièces de monnaie équilibrées. L'univers $\Omega$ est l'ensemble de toutes les suites infinies de résultats (Pile/Face).

Événements possibles :
\begin{itemize}
    \item Il y a une infinité de "Face"
    \item La suite commence par $n$ fois "Pile"
\end{itemize}

Événements non mesurables :
\begin{itemize}
    \item La proportion de Face ne tend pas vers $1/2$
\end{itemize}
\end{Ex}

\begin{Ex}
\textbf{Fissure dans un matériau}

L'univers est l'ensemble de toutes les fissures possibles sur une plaque soumise à un stress mécanique.

Événement possible : Une fissure traverse complètement la plaque.
\end{Ex}

\begin{Ex}
\textbf{Déformation d'un pont}

L'univers est l'ensemble des formes possibles de déformation d'un pont sous l'effet du trafic.

Événement possible : La déformation au centre du pont dépasse celle prévue par le cahier des charges.
\end{Ex}

\begin{Ex}
\textbf{Turbulences autour d'un avion}

L'univers est l'ensemble des champs de vitesse des écoulements de l'air autour de l'avion.

Événement possible : Le flux d'air entraîne une turbulence sentie par les passagers.
\end{Ex}

\begin{Ex}
\textbf{Chemin aléatoire dans Dijon}

L'univers est l'ensemble des positions d'un ivrogne immortel qui part de l'ESTP.

Événements possibles :
\begin{itemize}
    \item Au bout de $n$ pas, l'ivrogne se situe à plus de 100m de l'ESTP
    \item L'ivrogne repasse par l'ESTP plusieurs fois
\end{itemize}
\end{Ex}

\begin{Ex}
\textbf{Actifs financiers}

L'univers est l'ensemble des trajectoires possibles de l'actif. Si on se limite à un intervalle de temps $[0,T]$, $\Omega$ est l'ensemble des fonctions continues sur $[0,T]$.

Événement possible : Le cours de l'actif atteint 10.000 euros.
\end{Ex}

\section{Opérations sur les événements}

Soient $A$ et $B$ deux événements :
\begin{itemize}
    \item $A \cup B$ correspond à la réalisation de "$A$ ou $B$"
    \item $A \cap B$ correspond à la réalisation de "$A$ et $B$"
    \item $\overline{A}$ correspond à la non-réalisation de $A$
    \item $A \setminus B = A \cap \overline{B}$ correspond à "$A$ mais pas $B$"
\end{itemize}

