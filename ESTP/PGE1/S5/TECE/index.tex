% =====================================================================
% Template LaTeX – Traces distribuées aux étudiants
% Auteur : (à compléter)
% Compilation : pdflatex/xelatex (pdflatex recommandé ici)
% =====================================================================
\documentclass[11pt,a4paper]{report}

% -------------------- Encodage & langue --------------------
\usepackage[T1]{fontenc}
\usepackage[utf8]{inputenc}
\usepackage[french]{babel}
\usepackage{lmodern}
\usepackage{microtype}
\usepackage{amsmath, amssymb}
\usepackage{multicol}
\usepackage{enumitem}


% -------------------- Mise en page --------------------------
\usepackage[a4paper,margin=2cm]{geometry}
\usepackage{fancyhdr}
\usepackage{parskip}      % espace entre paragraphes
\setlength{\parindent}{0pt}

% -------------------- Couleurs & liens ----------------------
\usepackage{xcolor}
\definecolor{Theme}{HTML}{0E7490} % teal-700
\definecolor{ThemeLight}{HTML}{E0F2F1}
\definecolor{Accent}{HTML}{F59E0B} % amber-500
\definecolor{Gray}{HTML}{374151}
\usepackage[colorlinks=true,linkcolor=Theme,urlcolor=Theme,citecolor=Theme]{hyperref}

% -------------------- Graphiques / décor --------------------
\usepackage{tikz}
\usetikzlibrary{patterns,positioning,calc}
\usepackage{graphicx}
\usepackage{tcolorbox}
\tcbuselibrary{skins,breakable,hooks,most}
\usepackage{fontawesome5}

% -------------------- Titres -------------------------------
\usepackage{titlesec}
\titleformat{\chapter}[display]
  {\Huge\bfseries\color{Theme}}
  {\filright\rule{0.75\linewidth}{1.2pt}\\[3pt]{Algèbre linéaire - Chapitre~\thechapter}}
  {0.2ex}
  {\filright}
  [\vspace{0.1ex}\rule{0.35\linewidth}{1.2pt}]

\titleformat{\section}
  {\Large\bfseries\color{Gray}}
  {\thesection}{0.6em}{}

% -------------------- En-têtes / pieds ---------------------
\pagestyle{fancy}
\fancyhf{}
\fancyhead[L]{\color{Gray}\leftmark}
\fancyhead[R]{\color{Gray}\textit{TECE - 2025/2026}}
\fancyfoot[L]{\color{Gray}\small Auteur~: \textit{M. Berger}}
\fancyfoot[R]{\color{Gray}\small p.\ \thepage}
\renewcommand{\headrulewidth}{0pt}
\renewcommand{\footrulewidth}{0pt}

% -------------------- Macros utilitaires -------------------


% Tcolorboxes stylisées
\tcbset{tracebox/.style={breakable,enhanced,sharp corners,boxrule=0pt,frame hidden,arc=2mm,
  colback=white,coltitle=black,fonttitle=\bfseries\large,
  borderline west={2mm}{0pt}{Theme},
  before skip=8pt,after skip=8pt,drop fuzzy shadow}}

\newtcolorbox{resumeBox}{tracebox,title={\faStickyNote\quad Résumé des idées}}
\newtcolorbox{rappelsBox}{tracebox,title={\faRedo\quad Ce que je dois savoir }}
\newtcolorbox{exempleBox}{tracebox,title={\faChalkboardTeacher\quad Exemple vu ensemble}}

% Encadré « Formules & illustrations »
\newtcolorbox{formulesBox}{tracebox,title={\faCalculator\quad Formules \& illustrations},colback=ThemeLight}

% Astuce : puces clean
\newenvironment{niceitemize}{\begin{itemize}\setlength{\itemsep}{0.25em}\color{Gray}}{\end{itemize}}

% Raccourci pour une « Trace » complète
% Usage : \TraceSection{Titre}{Objectif court}
\newcommand{\TraceSection}[2]{%
  
}

% -------------------- Page de titre ------------------------
\title{\textbf{Traces de cours}\\\large (résumés, formules, exemples, mini-exercices)}
\author{ TECE - 2025/2026 }
\date{\today}


\makeatletter
\renewcommand{\thesubsection}{\arabic{subsection}}
\renewcommand{\p@subsection}{}% supprime le préfixe section/chapter dans \ref
% Si vous voulez la même chose pour les sous-sous-sections :
% \renewcommand{\thesubsubsection}{\arabic{subsubsection}}
% \renewcommand{\p@subsubsection}{}
\makeatother

\usepackage{mdframed}
\usepackage{ifthen}

% \usepackage[sf]{titlesec}
% Définition de la variable pour afficher les corrections
\newboolean{showSolutions}
% Décommentez la ligne suivante pour afficher les solutions
\input \jobname.adr
% -------------------- Document ----------------------------
\begin{document}

\begin{center}
    \textbf{\Large Calcul vectoriel, champs de vecteurs et opérateurs différentiels}
\end{center}

\vspace{2em}
% ================== Séquence 1 ==================

% 
\subsection{Des exemples}
Donner deux exemples différents dans chacune des situations suivantes :
\begin{enumerate}[label = $\square$, itemsep = 0.5em]
  \item une suite décroissante positive dont le terme général ne tend pas vers 0.

\ifthenelse{\boolean{showSolutions}}{
    \vspace{1em}

    \begin{mdframed}
    Par exemple, $u_n = 1+ \frac{1}{n}$.
    \end{mdframed}
}{}
  \item une suite bornée non convergente.

\ifthenelse{\boolean{showSolutions}}{
    \vspace{1em}

    \begin{mdframed}
    Par exemple, $u_n = (-1)^n$.
    \end{mdframed}
}{}
  \item une suite positive non bornée ne tendant pas vers $+\infty$.

\ifthenelse{\boolean{showSolutions}}{
    \vspace{1em}

    \begin{mdframed}
    Par exemple, $u_n = (-2)^n$.
    \end{mdframed}
}{}
  \item une suite non monotone qui tend vers 0.

\ifthenelse{\boolean{showSolutions}}{
    \vspace{1em}

    \begin{mdframed}
    Par exemple, $u_n = \dfrac{(-1)^n}{2^n}$.
    \end{mdframed}
}{}
  \item une suite positive qui tend vers 0 et qui n'est pas décroissante.

\ifthenelse{\boolean{showSolutions}}{
    \vspace{1em}

    \begin{mdframed}
    Par exemple, $u_n = -\dfrac{1}{n}$, ou $u_n = \dfrac{(-1)^n}{n}$.
    \end{mdframed}
}{}
\end{enumerate}



\vspace{1em}

\subsection{Vrai ou Faux ?}
  Dire si les assertions suivantes sont vraies ou fausses. On justifiera les réponses avec une démonstration ou un contre-exemple.
  \begin{enumerate}[label = $\square$, itemsep = 0.5em]
    \item Toute suite non-majorée tend vers $+\infty$.

    \ifthenelse{\boolean{showSolutions}}{
        \vspace{1em}

    \begin{mdframed}
    Faux, la suite $u_n = n (-1)^n$ n'est pas majorée et ne tend pas vers l'infini.
    \end{mdframed}
    }{
    }
    \item Soit $\left(u_n\right)_{n \geq 0}$ une suite à termes positifs convergeant vers $0$. Alors, $(u_n)$ est décroissante à partir d'un certain rang.

    \ifthenelse{\boolean{showSolutions}}{
        \vspace{1em}

    \begin{mdframed}
    Faux, on forme par exemple la suite 
    $$
    u_n = \begin{cases}
    \dfrac{1}{n} & \text{si } n \text{ est pair} \\
    \dfrac{1}{n^2} & \text{si } n \text{ est impair}
    \end{cases}
    $$
    qui converge vers $0$ mais n'est pas décroissante à partir d'un certain rang.
    \end{mdframed}
    }{
    }
    \item Si $(u_n)$ est une suite géométrique de raison $q \neq 0$, alors $\left(\frac{1}{u_n}\right)$ est une suite géométrique de raison~$1/q$.

    \ifthenelse{\boolean{showSolutions}}{
        \vspace{1em}

    \begin{mdframed}
    Vrai, si $u_n = u_0 q^n$, alors $\frac{1}{u_n} = \frac{1}{u_0} q^{-n} = \frac{1}{u_0} \left(\frac{1}{q}\right)^n$.
    \end{mdframed}
    }{
    }
    \item Soit $(u_n)$ une suite croissante et $\ell \in \mathbb{R}$. Si pour tout $N \in \mathbb{N}$, il existe $n_0 \geq N$ tel que $u_{n_0} > \ell$, alors $(u_n)$ ne converge pas vers $\ell$.

    \ifthenelse{\boolean{showSolutions}}{
        \vspace{1em}

    \begin{mdframed}
    Faux, la suite $u_n = \dfrac{(-1)^n}{n}$ converge vers $\ell = 0$, pourtant il existe une infinité de valeurs de $(u_n)$ qui sont strictement supérieures à $\ell$.
    \end{mdframed}
    }{
    }
    \item Si $f: \mathbb{R} \rightarrow \mathbb{R}$ est croissante et que $(u_n)$ vérifie $u_{n+1}=f\left(u_n\right)$ pour tout entier $n$, alors $(u_n)$ est croissante.
    
    \ifthenelse{\boolean{showSolutions}}{
        \vspace{1em}

    \begin{mdframed}
    Faux, on peut prendre par exemple $f(x) = x^2$ et $u_0 \in  ]0 , 1[$.
    \end{mdframed}
    }{
    }
    \item Si $(u_n)$ est divergente, alors $(u_n)$ est non bornée.

    \ifthenelse{\boolean{showSolutions}}{
        \vspace{1em}

    \begin{mdframed}
    Faux, la suite $u_n = (-1)^n$ est divergente mais bornée.
    \end{mdframed}
    }{
    }
    \item Si $u_n\to \ell$ et $f$ continue, alors $f(u_n)\to f(\ell)$

    \ifthenelse{\boolean{showSolutions}}{
        \vspace{1em}

    \begin{mdframed}
    Vrai, c'est la définition d'une fonction continue: $\lim_{x \to \ell} f(x) = f(\ell)$.
    \end{mdframed}
    }{
    }
  \end{enumerate}

  \vspace{1em}


  \subsection{Étude de suites}
  Étudier la nature des suites suivantes:
  \ifthenelse{\boolean{showSolutions}}{}{
  \begin{multicols}{3}}
  \begin{enumerate}[label = \alph*), itemsep = 0.5em]
    \item $\displaystyle u_n=\frac{\sin (n)+3 \cos \left(n^2\right)}{\sqrt{n}}$

    \ifthenelse{\boolean{showSolutions}}{
        \vspace{1em}

    \begin{mdframed}
    Le numérateur est borné car $|\sin(n)| \leq 1$ et $|3 \cos(n^2)| \leq 3$. Le dénominateur tend vers $+\infty$. Donc la suite tend vers 0.
    \end{mdframed}
    }{
    }
    \item $\displaystyle u_n=\frac{2 n+(-1)^n}{5 n+(-1)^{n+1}}$

    \ifthenelse{\boolean{showSolutions}}{
        \vspace{1em}

    \begin{mdframed}
        On conserve les termes dominants : 
        \[
        2n + (-1)^n = 2n \big( 1 + \frac{(-1)^n}{2n} \big) = 2n \big(1+ o(1)\big)
        \]
        $u_n$ est équivalente à $\frac{2 n}{5 n} = \frac{2}{5}$. donc la suite tend vers $\frac{2}{5}$.
    \end{mdframed}
    }{
    }
    \item $\displaystyle u_n=\frac{n^3+5 n}{4 n^2+\sin (n)+\ln (n)}$

    \ifthenelse{\boolean{showSolutions}}{
        \vspace{1em}

    \begin{mdframed}
        On fait de même, on conserve les termes dominants : 
        \[
        n^3 + 5n = n^3 \big( 1 + \frac{5}{n^2} \big) = n^3 \big(1 + o(1)\big)
        \]
        Pour le dénominateur : 
        \[
        4 n^2 + \sin(n) + \ln(n) = 4 n^2 \big( 1 + \frac{\sin(n)}{4 n^2} + \frac{\ln(n)}{4 n^2} \big) = 4 n^2 \big(1+ o(1)\big)
        \]
        $u_n$ est équivalente à $\frac{n^3}{4 n^2} = \frac{n}{4}$. donc la suite tend vers $+\infty$.
    \end{mdframed}
    }{
    }
    \item $\displaystyle u_n=\sqrt{2 n+1}-\sqrt{2 n-1}$

    \ifthenelse{\boolean{showSolutions}}{
        \vspace{1em}

    \begin{mdframed}
        On mutliplie en haut et en bas par la quantité conjuguée pour reconnaître une identité remarquable :
        \[
        u_n = \frac{\big(\sqrt{2 n+1}-\sqrt{2 n-1}\big)\big(\sqrt{2 n+1}+\sqrt{2 n-1}\big)}{\sqrt{2n + 1} + \sqrt{2n - 1}} 
        = \frac{2n + 1 - (2n - 1)}{\sqrt{2n + 1} + \sqrt{2n - 1}} = \frac{2}{\sqrt{2n + 1} + \sqrt{2n - 1}}
        \]
        Comme le dénominateur est équivalent à $2\sqrt{2n}$, on a $u_n$ est équivalent à $\frac{2}{2\sqrt{2n}} = \frac{1}{\sqrt{2n}}$. Donc la suite tend vers 0.
    \end{mdframed}
    }{
    }
    \item $\displaystyle u_n=3^n e^{-3 n}$.

    \ifthenelse{\boolean{showSolutions}}{
        \vspace{1em}

    \begin{mdframed}
        Par croissances comparées, l'exponentielle l'emporte sur les puissances donc la suite tend vers 0.
    \end{mdframed}
    }{
    }
    \item $\displaystyle u_n=\frac{n}{2^n}$

    \ifthenelse{\boolean{showSolutions}}{
        \vspace{1em}

        \begin{mdframed}
        Par croissances comparées, les exponentielles (ici $2^n$) l'emportent sur les puissances donc la suite tend vers 0.
    \end{mdframed}
    }{
    }
    \item $\displaystyle u_n=\frac{n!}{45^n}$

    \ifthenelse{\boolean{showSolutions}}{
        \vspace{1em}

        \begin{mdframed}
        Par croissances comparées, les factorielles l'emportent sur les puissances donc la suite tend vers l'infini.
    \end{mdframed}
    }{
    }
    \item $\displaystyle u_n=\frac{n!}{n^n}$

    \ifthenelse{\boolean{showSolutions}}{
        \vspace{1em}

        \begin{mdframed}
        On peut étudier le rapport $u_{n+1}/u_n$ :
        \[
        \frac{u_{n+1}}{u_n} = \frac{(n+1)!}{(n+1)^{n+1}} \cdot \frac{n^n}{n!} = \frac{n+1}{n+1} \cdot \left(\frac{n}{n+1}\right)^n = \left(\frac{n}{n+1}\right)^n
        \]
        Donc la suite est décroissante et tend vers 0.
    \end{mdframed}
    }{
    }
    \item $\displaystyle u_n=\frac{n^3+2^n}{n^2+3^n}$.

    \ifthenelse{\boolean{showSolutions}}{
        \vspace{1em}

        \begin{mdframed}
        On factorise par les termes dominants :
        \[
u_n = \frac{2^n(1 + \frac{n^3}{2^n})}{3^n(1 + \frac{n^2}{3^n})} = \frac{2^n(1 + o(1))}{3^n(1 + o(1))} 
        \]
        La suite est donc équivalente à $\frac{2^n}{3^n} = \left(\frac{2}{3}\right)^n$. Donc la suite tend vers 0.
    \end{mdframed}
    }{
    }
  \end{enumerate}
  \ifthenelse{\boolean{showSolutions}}{}{
  \end{multicols}}

  \vspace{1em}


  \subsection{*Plus difficile}
  Étudier la nature des suites suivantes, et déterminer un équivalent simple:
  \ifthenelse{\boolean{showSolutions}}{}{
  \begin{multicols}{2}}
  \begin{enumerate}[label = \alph*), itemsep = 1em]
    \item $u_n=\ln \left(2 n^2-n\right)-\ln (3 n+1)$

    \ifthenelse{\boolean{showSolutions}}{
        \vspace{1em}

        \begin{mdframed}
            On met en facteur le terme dominant dans chaque logarithme, de sorte que

            $$
            \begin{aligned}
            u_n & =\ln \left(2 n^2\left(1-\frac{1}{2 n}\right)\right)-\ln \left(3 n\left(1+\frac{1}{3 n}\right)\right) \\
            & =2 \ln n+\ln 2+\ln \left(1-\frac{1}{2 n}\right)-\ln (n)-\ln (3)-\ln \left(1+\frac{1}{3 n}\right) \\
            & =\ln n+\ln 2-\ln 3+v_n
            \end{aligned}
            $$
            
            où la suite $\left(v_n\right)$ tend vers 0 . On en déduit que $u_n$ tend vers $+\infty$.    \end{mdframed}
    }{
    }
    \item $u_n=\sqrt{n^2+n+1}-\sqrt{n^2-n+1}$

    \ifthenelse{\boolean{showSolutions}}{
        \vspace{1em}

        \begin{mdframed}
            On multiplie au numérateur et au dénominateur par la quantité conjuguée, de sorte que

            $$
            u_n=\frac{2 n}{\sqrt{n^2+n+1}+\sqrt{n^2-n+1}}
            $$
            
            
            On met encore en facteur, dans chaque racine carrée du dénominateur, le terme dominant (en $n^2$ ), et on trouve
            
            $$
            u_n=\frac{2}{\sqrt{1+\frac{1}{n}+\frac{1}{n^2}}+\sqrt{1-\frac{1}{n}+\frac{1}{n^2}}}
            $$
            
            
            Or, $\sqrt{1+\frac{1}{n}+\frac{1}{n^2}}$ tend vers 1 et $\sqrt{1-\frac{1}{n}+\frac{1}{n^2}}$ tend également vers 1 . On en déduit que $\left(u_n\right)$ converge vers 1.    \end{mdframed}
    }{
    }
    \item $_n=\frac{a^n-b^n}{a^n+b^n}, a, b \in] 0,+\infty[$

    \ifthenelse{\boolean{showSolutions}}{
        \vspace{1em}

        \begin{mdframed}
            Si $a=b$, alors $u_n=0$ pour tout $n$, et donc $\left(u_n\right)$ converge vers 0 . Si $a>b$, alors $a^n$ est prépondérant sur $b^n$ au sens que

            $$
            \frac{b^n}{a^n}=\left(\frac{b}{a}\right)^n \rightarrow 0
            $$
            
            puisque $|b / a|<1$. On factorise donc par $a^n$ au numérateur et au dénominateur :
            
            $$
            u_n=\frac{a^n\left(1-\left(\frac{b}{a}\right)^n\right)}{a^n\left(1+\left(\frac{b}{a}\right)^n\right)}=\frac{1-\left(\frac{b}{a}\right)^n}{1+\left(\frac{b}{a}\right)^n} .
            $$
            
            
            On en déduit que dans ce cas, ( $u_n$ ) converge vers 1. Si $b>a$, on factorise cette fois par $b^n$ et c'est $(a / b)^n$ qui converge vers 0 . On trouve :
            
            $$
            u_n=\frac{-1+\left(\frac{a}{b}\right)^n}{1+\left(\frac{a}{b}\right)^n} .
            $$
            
            ( $u_n$ ) converge donc vers -1 dans ce cas.    \end{mdframed}
    }{
    }
    \item $u_n=\frac{\ln \left(n+e^n\right)}{n}$

    \ifthenelse{\boolean{showSolutions}}{
        \vspace{1em}

        \begin{mdframed}
            On factorise par $e^n$ dans le logarithme. On obtient

            $$
            \begin{aligned}
            u_n & =\frac{\ln \left(e^n\left(1+n e^{-n}\right)\right)}{n} \\
            & =\frac{n+\ln \left(1+n e^{-n}\right)}{n} .
            \end{aligned}
            $$
            
            
            D'autre part, $n e^{-n}$ tend vers 0 (par exemple, on peut écrire $n e^{-n}=\frac{1}{\frac{e^n}{n}}$ et utiliser la comparaison des fonctions exponentielle et polynôme au voisinage de l'infini). Puisque la fonction ln est continue en 1 et $\ln (1)=0$, on en déduit que $\ln \left(1+n e^{-n}\right)$ tend vers 0. II vient $\ln \left(1+n e^{-n}\right) / n$ tend vers 0 , et donc la suite ( $u_n$ ) converge vers 1 .    \end{mdframed}
    }{
    }
    \item $u_n=\frac{\ln (1+\sqrt{n})}{\ln \left(1+n^2\right)}$.

    \ifthenelse{\boolean{showSolutions}}{
        \vspace{1em}

        \begin{mdframed}
            On factorise par le terme dominant dans chaque logarithme. On en déduit

            $$
            \begin{aligned}
            u_n & =\frac{\ln (\sqrt{n})+\ln \left(1+\frac{1}{\sqrt{n}}\right)}{\ln \left(n^2\right)+\ln \left(1+n^{-2}\right)} \\
            & =\frac{\frac{1}{2} \ln n+\ln \left(1+\frac{1}{\sqrt{n}}\right)}{2 \ln (n)+\ln \left(1+n^{-2}\right)} \\
            & =\frac{\frac{1}{2}+\frac{\ln \left(1+\frac{1}{\sqrt{n}}\right)}{\ln n}}{2+\frac{\ln \left(1+n^{-2}\right)}{\ln n}} .
            \end{aligned}
            $$
            
            
            Puisque $\ln \left(1+\frac{1}{\sqrt{n}}\right), \ln \left(1+n^{-2}\right)$ tendent vers $0,\left(u_n\right)$ converge vers $\frac{1}{4}$.    
        \end{mdframed}
    }{
    }
  \end{enumerate}
  \ifthenelse{\boolean{showSolutions}}{}{
  \end{multicols}}

  \vspace{1em}

  \subsection{Formule de Stirling}
  \begin{enumerate}[label = \alph*)]
    \item Soit $\left(x_n\right)$ une suite de réels et soit $\left(y_n\right)$ définie par $y_n=x_{n+1}-x_n$. \newline
    Démontrer que la série $\sum_n y_n$ et la suite $\left(x_n\right)$ sont de même nature.

    \ifthenelse{\boolean{showSolutions}}{
        \vspace{1em}

        \begin{mdframed}
        On a $y_n = x_{n+1} - x_n$. Donc
        $$\sum_{n=0}^N y_n = \sum_{n=0}^N (x_{n+1} - x_n) = x_{N+1} - x_0.$$
        Donc la série $\sum_n y_n$ et la suite $\left(x_n\right)$ sont de même nature.
    \end{mdframed}
    }{
    }
    \item On pose $(u_n)$ la suite définie par $u_n=\frac{n^n e^{-n} \sqrt{n}}{n!}$. \newline 
    A l'aide d'un développement limité, déterminer la nature de la série de terme général $v_n=\ln \left(\frac{u_{n+1}}{u_n}\right)$.

    \ifthenelse{\boolean{showSolutions}}{
        \vspace{1em}

        \begin{mdframed}
            2. Un petit calcul prouve que :

            $$
            v_n=\left(n+\frac{1}{2}\right) \ln \left(1+\frac{1}{n}\right)-1 .
            $$
            
            
            On effectue un développement limité de $v_n$-il faut pousser le développement du logarithme jusqu'à l'ordre 3 - et on a :
            
            $$
            v_n=\frac{1}{12 n^2}+o\left(\frac{1}{n^2}\right) .
            $$
            
            
            La série de terme général $v_n$ est donc convergente.
        \end{mdframed}
    }{
    }
    \item En déduire l'existence d'une constante $C>0$ telle que :
  \end{enumerate}
  $$
  n!\sim_{+\infty} C \sqrt{n} n^n e^{-n}
  $$

    \ifthenelse{\boolean{showSolutions}}{
        \vspace{1em}

        \begin{mdframed}
            On écrit $v_n=\ln \left(u_{n+1} / u_n\right)=\ln \left(u_{n+1}\right)-\ln \left(u_n\right)$. Puisque la série $\sum_n v_n$ converge, d'après la première question la suite $\left(\ln \left(u_n\right)\right)$ converge vers un réel $l_{\text {, }}$ et en passant à l'exponentielle, on trouve que la suite ( $u_n$ ) converge vers le réel $e^l$ qui est strictement positif. Revenant à la définition de $\left(u_n\right)$, ceci donne le résultat avec $C=e^{-l}$. 

    \end{mdframed}
    }{
    }

  \vspace{1em}


  \subsection{Télescopiques}
  \begin{enumerate}[label = \alph*)]
    \item Déterminer deux réels $a$ et $b$ tels que $\displaystyle
\frac{1}{k^2-1}=\frac{a}{k-1}+\frac{b}{k+1} .
$

    \ifthenelse{\boolean{showSolutions}}{
        \vspace{1em}

        \begin{mdframed}
            On met tout au même dénominateur, et on procède par identification :

            $$
            \frac{a}{k-1}+\frac{b}{k+1}=\frac{(a+b) k+(a-b)}{k^2-1} .
            $$
            
            
            On cherche donc $a$ et $b$ de sorte que
            
            $$
            \left\{\begin{array}{l}
            a+b=0 \\
            a-b=1
            \end{array}\right.
            $$
            
            
            On en déduit $a=1 / 2$ et $b=-1 / 2$.    \end{mdframed}
    }{
    }

    \item En déduire la limite de la suite $\displaystyle
u_n=\sum_{k=2}^n \frac{1}{k^2-1} .
$

    \ifthenelse{\boolean{showSolutions}}{
        \vspace{1em}

        \begin{mdframed}
            La somme est télescopique :

            $$
            \begin{aligned}
            u_n= & \frac{1}{2}\left(1+\frac{1}{2}+\cdots+\frac{1}{n-1}\right) \\
            & -\frac{1}{2}\left(\frac{1}{3}+\frac{1}{4}+\cdots+\frac{1}{n}+\frac{1}{n+1}\right)
            \end{aligned}
            $$
            
            soit
            
            $$
            u_n=\frac{1}{2}\left(1+\frac{1}{2}-\frac{1}{n}-\frac{1}{n+1}\right) .
            $$
            
            
            On en déduit que $\left(u_n\right)$ converge vers $\frac{3}{4}$.
            \end{mdframed}
    }{
    }

    \item Sur le même modèle, déterminer la limite de la suite $
v_n=\sum_{k=0}^n \frac{1}{k^2+3 k+2}
$

    \ifthenelse{\boolean{showSolutions}}{
        \vspace{1em}

        \begin{mdframed}
            L'idée est de factoriser $k^2+3 k+2=(k+1)(k+2)$. On cherche donc $a$ et $b$ tels que

            $$
            \frac{1}{k^2+3 k+2}=\frac{a}{k+1}+\frac{b}{k+2} .
            $$
            
            
            On trouve $a=1$ et $b=-1$. On en déduit que
            
            $$
            v_n=1-\frac{1}{n+2}
            $$
            
            et donc $\left(v_n\right)$ converge vers 1 .
        
        \end{mdframed}
    }{
    }


    \item Montrer que, pour tout $n \in \mathbb{N}^*$, on a $\sqrt{n+1}-\sqrt{n} \leq \frac{1}{2 \sqrt{n}}$

    \ifthenelse{\boolean{showSolutions}}{
        \vspace{1em}

        \begin{mdframed}
                Multipliant la différence de deux racines par la quantité conjuguée, on trouve
                $$\sqrt{n+1}-\sqrt{n}=\frac{(\sqrt{n+1}+\sqrt{n})(\sqrt{n+1}-\sqrt{n})}{\sqrt{n+1}+\sqrt{n}}=\frac{1}{\sqrt{n+1}+\sqrt{n}} \leq \frac{1}{2 \sqrt{n}} .$$
        \end{mdframed}
    }{
    }

    \item En déduire le comportement de la suite ( $u_n$ ) définie par $
u_n=1+\frac{1}{\sqrt{2}}+\cdots+\frac{1}{\sqrt{n}} .$
\ifthenelse{\boolean{showSolutions}}{
    \vspace{1em}

    \begin{mdframed}
        On somme alors ces inégalités, et les termes à gauche se télescopent :

        $$
        2(\sqrt{n+1}-\sqrt{1}) \leq u_n .
        $$
        
        
        Par le théorème de comparaison, on en déduit que ( $u_n$ ) diverge vers $+\infty$.
    \end{mdframed}
}{
}

  \end{enumerate}


\vspace{1em}

\section*{Séries numériques}
\subsection{Paramètres}
Soit $a, b \in \mathbb{R}$. Pour $n \geq 1$, on pose $u_n=\ln (n)+a \ln (n+1)+b \ln (n+2)$.
\begin{enumerate}[label = \alph*)]
  \item Pour quelle(s) valeur(s) de $(a, b)$ la série $\sum u_n$ est-elle convergente?



  \ifthenelse{\boolean{showSolutions}}{
        \vspace{1em}

        \begin{mdframed}
            Écrivons

            $$
            \ln (n+1)=\ln (n)+\ln \left(1+\frac{1}{n}\right)=\ln n+\frac{1}{n}+O\left(\frac{1}{n^2}\right)
            $$
            
            et
            
            $$
            \ln (n+2)=\ln (n)+\ln \left(1+\frac{2}{n}\right)=\ln n+\frac{2}{n}+O\left(\frac{1}{n^2}\right) .
            $$
            
            
            On obtient
            
            $$
            u_n=(1+a+b) \ln (n)+\frac{a+2 b}{n}++O\left(\frac{1}{n^2}\right) .
            $$
            
            
            Ainsi, si $1+a+b \neq 0, u_n \sim_{+\infty}(1+a+b) \ln (n)$ et la série diverge grossièrement. Si $1+a+b=0$ et $a+2 b \neq 0, u_n \sim_{+\infty} \frac{a+2 b}{n}$ et la série diverge par comparaison à une série de Riemann divergente. Si $1+a+b=0$ et $a+2 b=0$, alors $u_n=O\left(\frac{1}{n^2}\right)$ et la série converge absolument. Finalement, on a prouvé que $\sum u_n$ converge si et seulement si
            
            $$
            \left\{\begin{array} { r } 
            { 1 + a + b = 0 } \\
            { 2 a + b = 0 }
            \end{array} \Longleftrightarrow \left\{\begin{array}{l}
            a=-2 \\
            b=1
            \end{array}\right.\right.
            $$
            \end{mdframed}
    }{
    }

  \item Dans le(s) cas où la série converge, déterminer $\sum_{n=1}^{+\infty} u_n$.

  \ifthenelse{\boolean{showSolutions}}{
        \vspace{1em}

        \begin{mdframed}
            On traite donc le cas $a=-2$ et $b=1$. Notons $S_n=\sum_{k=1}^n u_k$. Alors

            $$
            \begin{aligned}
            S_n & =\sum_{k=1}^n \ln (k)-2 \sum_{k=1}^n \ln (k+1)+\sum_{k=1}^n \ln (k+2) \\
            & =\sum_{k=1}^n \ln (k)-\sum_{k=1}^n \ln (k+1)+\sum_{k=1}^n \ln (k+2)-\sum_{k=1}^n \ln (k+1) .
            \end{aligned}
            $$
            
            
            On reconnait deux couples de deux sommes télescopiques et on trouve
            
            $$
            S_n=\ln (1)-\ln (2)+\ln (n+2)-\ln (n+1) \rightarrow-\ln (2) .
            $$
            \end{mdframed}
    }{
    }
\end{enumerate}

\vspace{1em}

\subsection{Avec l'exponentielle}
Sachant que $e=\sum_{n \geq 0} \frac{1}{n!}$, déterminer la valeur des sommes suivantes :
  $$\sum_{n \geq 0} \frac{n+1}{n!}, \qquad \sum_{n \geq 0} \frac{n^2-2}{n!}, \qquad \sum_{n \geq 0} \frac{n^3}{n!}.$$


  \ifthenelse{\boolean{showSolutions}}{
        \vspace{1em}

        \begin{mdframed}
            1. La première somme ne pose pas de problèmes :

            $$
            \sum_{n \geq 0} \frac{n+1}{n!}=\sum_{n \geq 0} \frac{n}{n!}+\sum_{n \geq 0} \frac{1}{n!}=\sum_{n \geq 1} \frac{1}{(n-1)!}+e=2 e .
            $$
            
            2. La deuxième somme est plus compliquée à cause du terme en $n^2$. Pour qu'il se simplifie bien avec le $n!$, le plus commode est de l'écrire
            
            $$
            n^2=n(n-1)+n
            $$
            
            de sorte que
            
            $$
            \sum_{n \geq 0} \frac{n^2-2}{n!}=\sum_{n \geq 2} \frac{n(n-1)}{n!}+\sum_{n \geq 1} \frac{1}{(n-1)!}-2 \sum_{n \geq 0} \frac{1}{n!}=e+e-2 e=0 .
            $$
            
            3. La méthode est similaire. Dit de façon algébrique, on va décomposer le polynôme $X^3$ dans la base $1, X, X(X-1), X(X-1)(X-2)$. En raisonnant d'abord avec le terme de plus haut degré, puis celui juste après, etc..., on trouve :
            
            $$
            X^3=X(X-1)(X-2)+3 X(X-1)+X .
            $$
            
            
            On en déduit :
            
            $$
            \begin{aligned}
            \sum_{n \geq 1} \frac{n^3}{n!} & =\sum_{n \geq 1} \frac{n(n-1)(n-2)}{n!}+3 \sum_{n \geq 1} \frac{n(n-1)}{n!}+\sum_{n \geq 1} \frac{n}{n!} \\
            & =\sum_{n \geq 3} \frac{1}{(n-3)!}+\sum_{n \geq 2} \frac{3}{(n-2)!}+\sum_{n \geq 1} \frac{1}{(n-1)!} \\
            & =5 e .
            \end{aligned}
            $$
        
        \end{mdframed}
    }{
    }


\section*{Espaces Vectoriels}

\vspace{2em}

\subsection{parties de $\mathbb{R}^2$}

  Les parties suivantes sont-elles des sous-espaces vectoriels de $\mathbb{R}^2$ ?

  \ifthenelse{\boolean{showSolutions}}{}{\begin{multicols}{2} }
  \begin{enumerate}
    \item $A=\left\{(x, y) \in \mathbb{R}^2 \mid x \leqslant y\right\}$ 

    \ifthenelse{\boolean{showSolutions}}{
      \vspace{1em}
      \begin{mdframed}
        $A$ est un demi-plan de $\mathbb{R}^2$, ce n'est pas un espace vectoriel, si on prend un vecteur non nul et qu'on multiplie par un scalaire négatif, on obtient un vecteur qui n'est pas dans $A$.
      \end{mdframed}
    }{}
    \vspace{1em}

    \item $B=\left\{(x, y) \in \mathbb{R}^2 \mid x y=0\right\}$ 

    \ifthenelse{\boolean{showSolutions}}{
      \vspace{1em}
      \begin{mdframed}
        $B$ est une réunion de deux droites qui passent par l'origine. Si on prend un vecteur sur chaque droite et qu'on les additionne, on obtient un vecteur qui n'est pas dans $B$.
      \end{mdframed}
    }{}
    \vspace{1em}

    \item $C=\left\{(x, y) \in \mathbb{R}^2 \mid x=y\right\}$ 

    \ifthenelse{\boolean{showSolutions}}{
      \vspace{1em}
      \begin{mdframed}
        $C$ est une droite qui passe par l'origine. C'est un espace vectoriel.
      \end{mdframed}
    }{}
    \vspace{1em}

    \item $D=\left\{(x, y) \in \mathbb{R}^2 \mid x+y=1\right\}$

    \ifthenelse{\boolean{showSolutions}}{
      \vspace{1em}
      \begin{mdframed}
        Le vecteur nul n'est pas dans $D$, donc $D$ n'est pas un espace vectoriel.
      \end{mdframed}
    }{}
    \vspace{1em}

  \end{enumerate}
  \ifthenelse{\boolean{showSolutions}}{}{\end{multicols} }


  \vspace{2em}

\subsection{Dans $\mathbb{R}^n$}

On munit $\mathbb{R}^n$ des lois usuelles. Parmi les sous-ensembles suivants $F$ de $\mathbb{R}^n$, lesquels sont des espaces vectoriels?

\ifthenelse{\boolean{showSolutions}}{}{\begin{multicols}{2} }
\begin{enumerate}
  \item $\mathrm{F}=\left\{\left(\mathrm{x}_1, \ldots, \mathrm{x}_{\mathrm{n}}\right) \in \mathbb{R}^{\mathrm{n}} / \mathrm{x}_1=0\right\}$

  \ifthenelse{\boolean{showSolutions}}{
    \vspace{1em}
    \begin{mdframed}
      $F$ est bien un espace vectoriel. Il suffit pour le montrer de se poser les 3 questions : 
      \begin{itemize}
        \item Le vecteur nul est-il dans $F$ ? \newline 
        Oui, il s'agit du vecteur $(0,0,\dots,0)$.

        \item $F$ est-il stable par l'addition ? \newline 
        Oui, si $(x_1,x_2,\dots,x_n)$ et $(y_1,y_2,\dots,y_n)$ sont dans $F$, alors $x_1+y_1=0$ et $x_2+y_2=0$ et ainsi de suite. 
        
        Donc $(x_1+y_1,x_2+y_2,\dots,x_n+y_n)$ est dans $F$.
        \item $F$ est-il stable par la multiplication par un scalaire ? \newline 
        Oui, si $(x_1,x_2,\dots,x_n)$ est dans $F$ et $\lambda$ est un scalaire, alors $\lambda x_1=0$ et $\lambda x_2=0$ et ainsi de suite. 
        
        Donc $(\lambda x_1,\lambda x_2,\dots,\lambda x_n)$ est dans $F$.
      \end{itemize}
    \end{mdframed}
  }{}
  \vspace{1em}

  \item $\mathrm{F}=\left\{\left(\mathrm{x}_1, \ldots, \mathrm{x}_{\mathrm{n}}\right) \in \mathbb{R}^{\mathrm{n}} / \mathrm{x}_1=1\right\}$.

  \ifthenelse{\boolean{showSolutions}}{
    \vspace{1em}
    \begin{mdframed}
      $F$ n'est pas un espace vectoriel, le vecteur nul n'est pas dans $F$.
    \end{mdframed}
  }{}
  \vspace{1em}

  \item $F=\left\{\left(x_1, \ldots, x_n\right) \in \mathbb{R}^n / x_1=x_2\right\}$

  \ifthenelse{\boolean{showSolutions}}{
    \vspace{1em}
    \begin{mdframed}
      $F$ est bien un espace vectoriel.
    \end{mdframed}
  }{}
  \vspace{1em}

  \item $\mathrm{F}=\left\{\left(x_1, \ldots, x_{\mathrm{n}}\right) \in \mathbb{R}^{\mathrm{n}} / x_1+\ldots+x_{\mathrm{n}}=0\right\}$

  \ifthenelse{\boolean{showSolutions}}{
    \vspace{1em}
    \begin{mdframed}
      $F$ est bien un espace vectoriel.
    \end{mdframed}
  }{}
  \vspace{1em}

  \item $\mathrm{F}=\left\{\left(\mathrm{x}_1, \ldots, \mathrm{x}_{\mathrm{n}}\right) \in \mathbb{R}^{\mathrm{n}} / \mathrm{x}_1 \times \mathrm{x}_2=0\right\}$

  \ifthenelse{\boolean{showSolutions}}{
    \vspace{1em}

    \begin{mdframed}
      $F$ n'est pas un espace vectoriel, on peut trouver deux vecteurs de $F$ dont la somme n'est pas dans $F$.
    \end{mdframed}
  }{}
  \vspace{1em}

\end{enumerate}
\ifthenelse{\boolean{showSolutions}}{}{\end{multicols} }

\vspace{1em}
\section*{Calcul vectoriel}
\vspace{1em}

\subsection{Combinaisons linéaires}

\begin{itemize}
    \item Dans $\mathbb{R}^2$, $u=(1,2)$ est-il combinaison linéaire de $e_1=(1,-2)$ et $e_2=(2,3)$ ?
    \item Dans $\mathbb{R}^2$, $u=(1,2)$ est-il combinaison linéaire de $e_1=(1,-2), e_2=(2,3), e_3=(-4,5)$ ?
    \item Dans $\mathbb{R}^3$, $u=(2,5,3)$ est-il combinaison linéaire de $e_1=(1,3,2)$ et $e_2=(1,-1,4)$ ?
    \item Dans $\mathbb{R}^3$, $u=(3,1, m)$ est-il combinaison linéaire de $e_1=(1,3,2)$ et $e_2=(1,-1,4)$ ? \newline 
    (discuter suivant la valeur de $m$ ) 
\end{itemize}
Si oui, donner toutes les combinaisons linéaires possibles.

\ifthenelse{\boolean{showSolutions}}{
  \vspace{1em}

\begin{mdframed}

  \begin{enumerate}
    \item $u$ est combinaison linéaire de $e_1$ et $e_2$ si et seulement si il existe $a, b \in \mathbb{R}$ tels que $u=a e_1 + b e_2$.

    Trouver $a$ et $b$ nous conduit à un système linéaire : 

    \begin{align*}
      a + 2b &= 1 \\
      -2a + 3b &= 2
    \end{align*}

    On trouve $a=-1/7$ et $b=4/7$. Donc $u$ est combinaison linéaire de $e_1$ et $e_2$.

    \item $u$ est combinaison linéaire de $e_1, e_2$ et $e_3$ si et seulement si il existe $a, b, c \in \mathbb{R}$ tels que $u=a e_1 + b e_2 + c e_3$.

    Trouver $a$, $b$ et $c$ nous conduit à un système linéaire : 

    \begin{align*}
      a + 2b -4c &= 1 \\
      -2a + 3b + 5c &= 2 \\
    \end{align*}

    La première étape du pivot de gauss nous donne : 
    \begin{align*}
      a + 2b -4c &= 1 \\
      0 + 7b - 3c &= 3 \\
    \end{align*}
    On peut choisir $c$ comme on veut dans $\mathbb{R}$. $b$ et $a$ sont ensuite déterminés en fonction de $c$.

    \item $u$ est combinaison linéaire de $e_1$ et $e_2$ si et seulement si il existe $a, b \in \mathbb{R}$ tels que $u=a e_1 + b e_2$.

    Trouver $a$ et $b$ nous conduit à un système linéaire : 

    \begin{align*}
      a + b &= 2 \\
      3a - b &= 5 \\
      2a + 4 b &= 3
    \end{align*}

    Les premières étapes du pivot de Gauss nous donnent : 
    \begin{align*}
      a + b &= 2 \\
      0 - 4b &= 1 \\
      0 + 2b &= -1
    \end{align*}
    Le système est donc incompatible et $u$ n'est pas combinaison linéaire de $e_1$ et $e_2$.

    \item $u$ est combinaison linéaire de $e_1$ et $e_2$ si et seulement si il existe $a, b \in \mathbb{R}$ tels que $u=a e_1 + b e_2$.

    Trouver $a$ et $b$ nous conduit à un système linéaire : 

    \begin{align*}
      a + b &= 3 \\
      3a - b &= 1 \\
      2a + 4 b &= m
    \end{align*}

    Les premières étapes du pivot de Gauss nous donnent : 
    \begin{align*}
      a + b &= 3 \\
      0 - 4b &= -8 \\
      0 + 2b &= m-6
    \end{align*}
    Le système est compatible si et seulement si $m = 10$, dans ce cas, $u$ est combinaison linéaire de $e_1$ et $e_2$.

    Si $m \neq 10$, le système est incompatible et $u$ n'est pas combinaison linéaire de $e_1$ et $e_2$.
  \end{enumerate}
\end{mdframed}
}{}
\vspace{2em}
\subsection{Sous-espace engendré}


Dans $\mathbb{R}^3$, on pose $u_1=(1, -1, 2)$ et $u_2=(1, 1, -1)$.
\begin{itemize}
    \item Les vecteurs $v_1=(3, 1, 0)$ et $v_2=(1, 5, -1)$ sont-ils combinaison linéaire de $u_1$ et $u_2$ ?
    \item Soit $a, b, c \in \mathbb{R}$. Démontrer que $v=(a, b, c)$ est combinaison linéaire de $u_1$ et $u_2$ si et seulement si $-a+3 b+2 c=0$.
    \item En déduire un vecteur de $\mathbb{R}^3$ qui n'est pas combinaison linéaire de $u_1$ et de $u_2$.
\end{itemize}

\ifthenelse{\boolean{showSolutions}}{
  \vspace{1em}
  \begin{mdframed}
    \begin{enumerate}
      \item $v_1$ est combinaison linéaire de $u_1$ et $u_2$ si et seulement si il existe $a, b \in \mathbb{R}$ tels que $v_1=a u_1 + b u_2$.

      Trouver $a$ et $b$ nous conduit à un système linéaire : 

      \begin{align*}
        a + b &= 3 \\
        -a + b &= 1 \\
        2a - b &= 0
      \end{align*}

      Les premières étapes du pivot de Gauss nous donnent : 
      \begin{align*}
        a + b &= 3 \\
        0 + 2b &= 4 \\
        0 -3b &= -6
      \end{align*}
      Les deux dernières lignes correspondent à la même équation, le système possède donc une solution. 

      \item $v_2$ est combinaison linéaire de $u_1$ et $u_2$ si et seulement si il existe $a, b \in \mathbb{R}$ tels que $v_2=a u_1 + b u_2$.

      Trouver $a$ et $b$ nous conduit à un système linéaire : 

      \begin{align*}
        a + b &= 1 \\
        -a + b &= 5 \\
        2a - b &= -1
      \end{align*}
      Les premières étapes du pivot de Gauss nous donnent : 
      \begin{align*}
        a + b &= 1 \\
        0 + 2b &= 6 \\
        0 - 3b &= -3
      \end{align*}
      Le système est incompatible et $v_2$ n'est pas combinaison linéaire de $u_1$ et $u_2$.

      \item $v=(a, b, c)$ est combinaison linéaire de $u_1$ et $u_2$ si et seulement si le système suivant possède une solution : 
      \begin{align*}
        x + y &= a \\
        -x + y &= b \\
        2x - y &= c
      \end{align*}
      Les premières étapes du pivot de Gauss nous donnent : 
      \begin{align*}
        x + y &= a \\
        0 + 2y &= a + b \\
        0 - 3y &= c- 2a
      \end{align*}
      qu'on peut réécrire : 
      \begin{align*}
        x + y &= a \\
        y &= (a + b)/2 \\
        y &= (2a-c)/3
      \end{align*}
      Le système est compatible si et seulement si $(2a-c)/3 = (a+b)/2$, c'est-à-dire
      $$
      4a - 2c = 3a + 3b
      $$
      c'est-à-dire
      $$
      a - 3b - 2c = 0
      $$
      c'est bien l'équation de l'énoncé.

      \item Il suffit de trouver trois nombres $a, b, c$ qui ne vérifient pas l'équation de l'énoncé. Par exemple, $(1,0,0)$ n'est pas combinaison linéaire de $u_1$ et $u_2$.
    \end{enumerate}
  \end{mdframed}
}{}


\ifthenelse{\boolean{showSolutions}}{}{
    \newpage 
}

\section*{Familles}
\vspace{1em}
\subsection{Familles libres}

Les familles suivantes sont-elles libres dans $\mathbb{R}^3$ ?
\begin{itemize}
    \item $(u, v)$ avec $u=(1,2,3)$ et $v=(-1,4,6)$;
    \item $(u, v, w)$ avec $u=(1,2,-1), v=(1,0,1)$ et $w=(0,0,1)$;
    \item $(u, v, w)$ avec $u=(1,2,-1), v=(1,0,1)$ et $w=(-1,2,-3)$;
\end{itemize}

Sans calcul supplémentaire, dire si elles sont génératrices. 

\vspace{3em}
\subsection{Dimension}
On considère, dans $\mathbb{R}^4$, les vecteurs :

$$
v_1=(1,2,3,4), \quad v_2=(1,1,1,3), \quad v_3=(2,1,1,1), \quad v_4=(-1,0,-1,2), \quad v_5=(2,3,0,1) .
$$


Soit $F$ l'espace vectoriel engendré par $\left\{v_1, v_2, v_3\right\}$ et soit $G$ celui engendré par $\left\{v_4, v_5\right\}$. Calculer les dimensions respectives de $F, G, F \cap G, F+G$.


\ifthenelse{\boolean{showSolutions}}{
    \vspace{1em}

\begin{mdframed}


    \begin{enumerate}
    \item $G$ est engendré par deux vecteurs donc $\operatorname{dim} G \leqslant 2$. Clairement $v_4$ et $v_5$ ne sont pas liés donc $\operatorname{dim} G \geqslant 2$ c'est-à-dire $\operatorname{dim} G=2$.
    \item $F$ est engendré par trois vecteurs donc $\operatorname{dim} F \leqslant 3$. Un calcul montre que la famille $\left\{v_1, v_2, v_3\right\}$ est libre, d'où $\operatorname{dim} F \geqslant 3$ et donc $\operatorname{dim} F=3$.
    \item Essayons d'abord d'estimer la dimension de $F \cap G$.
    
    D'une part $F \cap G \subset G$ donc $\operatorname{dim}(F \cap G) \leqslant 2$. 
    
    Utilisons d'autre part la formule $\operatorname{dim}(F+G)=\operatorname{dim} F+\operatorname{dim} G-\operatorname{dim}(F \cap G)$. 
    
    Comme $F+G \subset \mathbb{R}^4$, on a $\operatorname{dim}(F+ G) \leqslant 4$ d'où on tire l'inégalité $\operatorname{dim}(F \cap G) \geqslant 1$. \newline 
    Donc soit $\operatorname{dim}(F \cap G)=1$ ou bien $\operatorname{dim}(F \cap G)=2$.

    Supposons que $\operatorname{dim}(F \cap G)$ soit égale à 2. 
    Comme $F \cap G \subset G$ on aurait dans ce cas $F \cap G=G$ et donc $G \subset F$.
    En particulier il existerait $\alpha, \beta, \gamma \in \mathbb{R}$ tels que $v_4=\alpha v_1+\beta v_2+\gamma v_3$. 
    On vérifie aisément que ce n'est pas le cas, ainsi $\operatorname{dim}(F \cap G)$ n'est pas égale à 2. 
    
    On peut donc conclure $\operatorname{dim}(F \cap G)=1$

    \item Par la formule $\operatorname{dim}(F+G)=\operatorname{dim} F+\operatorname{dim} G-\operatorname{dim}(F \cap G)$, on obtient $\operatorname{dim}(F+G)=2+3-1=4$. Cela entraîne $F+G=\mathbb{R}^4$.
    \end{enumerate}
\end{mdframed}
}{}

\section*{Champs de vecteurs}
\vspace{1em}

\subsection{Gradient et divergence}

Déterminer les coordonnées de $\operatorname{grad}(f)$ où $f$ est le champ scalaire suivant:
\begin{enumerate}[itemsep=0.5em]
    \item $f(x, y, z)=x y^2-y z^2$.
    \item $f(x, y, z)=x y z \sin (x y)$.
\end{enumerate}

\vspace{1em}
Déterminer $\operatorname{div}(f)$ où $f$ est le champ de vecteurs suivant:
\begin{enumerate}[itemsep=0.5em]
    \item $f(x, y, z)=\big(2 x^2 y, 2 x y^2, x y\big)$.
    \item $f(x, y, z)=\big(\sin (x y), 0, \cos (x z)\big)$.
    \item $f(x, y, z)=\big(x(2 y+z),-y(x+z), z(x-2 y)\big)$.
\end{enumerate}


\ifthenelse{\boolean{showSolutions}}{
    \vspace{2em}

\begin{mdframed}
Pour les gradients : 
\begin{enumerate}[itemsep=0.5em]
    \item $\operatorname{grad}(f)=\left(y^2, 2 x y-z^2,-2 y z\right)$.
    \item $\operatorname{grad}(f)=\left(y z \sin (x y)+x y^2 z \cos (x y), x z \sin (x y)+x^2 y z \cos (x y), x y \sin (x y)\right)$.
\end{enumerate}


\vspace{1em}
Pour les divergences : 
\begin{enumerate}[itemsep=0.5em]
    \item $\operatorname{div}(f)=8 x y$.
    \item $\operatorname{div}(f)=y \cos (x y)-x \sin (x z)$.
    \item $\operatorname{div}(f)=0$.
\end{enumerate}
\end{mdframed}
}{}

\vspace{2em}
\subsection{Potentiel scalaire}

On rappelle qu'on dit qu'un champ de vecteurs $F$ dérive d'un potentiel scalaire s'il existe un champ scalaire $f$ tel que $F=\operatorname{grad}(f)$. 
\begin{enumerate}[itemsep=0.5em]
    \item $F(x, y, z)=\left(2 x y+z^3, x^2, 3 x z^2\right)$, défini sur $\mathbb{R}^3$.
    \item $F(x, y)=\left(-\frac{y}{(x-y)^2}, \frac{x}{(x-y)^2}\right)$, défini sur $U=\left\{(x, y) \in \mathbb{R}^2, x>y\right\}$.
\end{enumerate}

Pour chacun des champs de vecteurs précédents, montrer qu'ils sont définis sur un ouvert étoilé, que leur rotationnel est nul, et qu'ils dérivent d'un potentiel scalaire. \newline 
 Déterminer tous les potentiels scalaires dont ils dérivent.   

\ifthenelse{\boolean{showSolutions}}{
    \vspace{1em}
    \begin{mdframed}

D'une part, on peut prouver l'existence théorique d'un tel champ scalaire, en observant que les champs de vecteurs sont définis sur des ouverts étoilés, et que leur rotationnel est nul. 

D'autre part, il est possible de les "intégrer". On cherche en effet:
\begin{enumerate}[itemsep=0.5em]
    \item $f$ de $\mathbb{R}^3$ dans $\mathbb{R}$ tel que $\frac{\partial f}{\partial x}=2 x y+z^3, \frac{\partial f}{\partial y}=x^2$ et $\frac{\partial f}{\partial z}=3 x z^2$. On résoud ce système d'équation aux dérivées partielles: la deuxième équation donne par exemple $f(x, y, z)=x^2 y+h(x, z)$, et utilisant les deux autres équations, on trouve:

$$
f(x, y, z)=x^2 y+x z^3+\text { cste. }
$$

    \item $f$ de $\mathbb{R}^2$ dans $\mathbb{R}$ tel que

$$
\frac{\partial f}{\partial x}=-\frac{y}{(x-y)^2} \text { et } \frac{\partial f}{\partial y}=\frac{x}{(x-y)^2} .
$$


Les fonctions $f$ qui conviennent sont donc $f(x, y)=\frac{y}{x-y}+c s t e$.
\end{enumerate}
\end{mdframed}
}{}
\end{document}
