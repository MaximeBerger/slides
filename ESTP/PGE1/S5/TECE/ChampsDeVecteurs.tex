\section*{Espaces Vectoriels}

\vspace{2em}

\subsection{parties de $\mathbb{R}^2$}

  Les parties suivantes sont-elles des sous-espaces vectoriels de $\mathbb{R}^2$ ?

  \ifthenelse{\boolean{showSolutions}}{}{\begin{multicols}{2} }
  \begin{enumerate}
    \item $A=\left\{(x, y) \in \mathbb{R}^2 \mid x \leqslant y\right\}$ 

    \ifthenelse{\boolean{showSolutions}}{
      \vspace{1em}
      \begin{mdframed}
        $A$ est un demi-plan de $\mathbb{R}^2$, ce n'est pas un espace vectoriel, si on prend un vecteur non nul et qu'on multiplie par un scalaire négatif, on obtient un vecteur qui n'est pas dans $A$.
      \end{mdframed}
    }{}
    \vspace{1em}

    \item $B=\left\{(x, y) \in \mathbb{R}^2 \mid x y=0\right\}$ 

    \ifthenelse{\boolean{showSolutions}}{
      \vspace{1em}
      \begin{mdframed}
        $B$ est une réunion de deux droites qui passent par l'origine. Si on prend un vecteur sur chaque droite et qu'on les additionne, on obtient un vecteur qui n'est pas dans $B$.
      \end{mdframed}
    }{}
    \vspace{1em}

    \item $C=\left\{(x, y) \in \mathbb{R}^2 \mid x=y\right\}$ 

    \ifthenelse{\boolean{showSolutions}}{
      \vspace{1em}
      \begin{mdframed}
        $C$ est une droite qui passe par l'origine. C'est un espace vectoriel.
      \end{mdframed}
    }{}
    \vspace{1em}

    \item $D=\left\{(x, y) \in \mathbb{R}^2 \mid x+y=1\right\}$

    \ifthenelse{\boolean{showSolutions}}{
      \vspace{1em}
      \begin{mdframed}
        Le vecteur nul n'est pas dans $D$, donc $D$ n'est pas un espace vectoriel.
      \end{mdframed}
    }{}
    \vspace{1em}

  \end{enumerate}
  \ifthenelse{\boolean{showSolutions}}{}{\end{multicols} }


  \vspace{2em}

\subsection{Dans $\mathbb{R}^n$}

On munit $\mathbb{R}^n$ des lois usuelles. Parmi les sous-ensembles suivants $F$ de $\mathbb{R}^n$, lesquels sont des espaces vectoriels?

\ifthenelse{\boolean{showSolutions}}{}{\begin{multicols}{2} }
\begin{enumerate}
  \item $\mathrm{F}=\left\{\left(\mathrm{x}_1, \ldots, \mathrm{x}_{\mathrm{n}}\right) \in \mathbb{R}^{\mathrm{n}} / \mathrm{x}_1=0\right\}$

  \ifthenelse{\boolean{showSolutions}}{
    \vspace{1em}
    \begin{mdframed}
      $F$ est bien un espace vectoriel. Il suffit pour le montrer de se poser les 3 questions : 
      \begin{itemize}
        \item Le vecteur nul est-il dans $F$ ? \newline 
        Oui, il s'agit du vecteur $(0,0,\dots,0)$.

        \item $F$ est-il stable par l'addition ? \newline 
        Oui, si $(x_1,x_2,\dots,x_n)$ et $(y_1,y_2,\dots,y_n)$ sont dans $F$, alors $x_1+y_1=0$ et $x_2+y_2=0$ et ainsi de suite. 
        
        Donc $(x_1+y_1,x_2+y_2,\dots,x_n+y_n)$ est dans $F$.
        \item $F$ est-il stable par la multiplication par un scalaire ? \newline 
        Oui, si $(x_1,x_2,\dots,x_n)$ est dans $F$ et $\lambda$ est un scalaire, alors $\lambda x_1=0$ et $\lambda x_2=0$ et ainsi de suite. 
        
        Donc $(\lambda x_1,\lambda x_2,\dots,\lambda x_n)$ est dans $F$.
      \end{itemize}
    \end{mdframed}
  }{}
  \vspace{1em}

  \item $\mathrm{F}=\left\{\left(\mathrm{x}_1, \ldots, \mathrm{x}_{\mathrm{n}}\right) \in \mathbb{R}^{\mathrm{n}} / \mathrm{x}_1=1\right\}$.

  \ifthenelse{\boolean{showSolutions}}{
    \vspace{1em}
    \begin{mdframed}
      $F$ n'est pas un espace vectoriel, le vecteur nul n'est pas dans $F$.
    \end{mdframed}
  }{}
  \vspace{1em}

  \item $F=\left\{\left(x_1, \ldots, x_n\right) \in \mathbb{R}^n / x_1=x_2\right\}$

  \ifthenelse{\boolean{showSolutions}}{
    \vspace{1em}
    \begin{mdframed}
      $F$ est bien un espace vectoriel.
    \end{mdframed}
  }{}
  \vspace{1em}

  \item $\mathrm{F}=\left\{\left(x_1, \ldots, x_{\mathrm{n}}\right) \in \mathbb{R}^{\mathrm{n}} / x_1+\ldots+x_{\mathrm{n}}=0\right\}$

  \ifthenelse{\boolean{showSolutions}}{
    \vspace{1em}
    \begin{mdframed}
      $F$ est bien un espace vectoriel.
    \end{mdframed}
  }{}
  \vspace{1em}

  \item $\mathrm{F}=\left\{\left(\mathrm{x}_1, \ldots, \mathrm{x}_{\mathrm{n}}\right) \in \mathbb{R}^{\mathrm{n}} / \mathrm{x}_1 \times \mathrm{x}_2=0\right\}$

  \ifthenelse{\boolean{showSolutions}}{
    \vspace{1em}

    \begin{mdframed}
      $F$ n'est pas un espace vectoriel, on peut trouver deux vecteurs de $F$ dont la somme n'est pas dans $F$.
    \end{mdframed}
  }{}
  \vspace{1em}

\end{enumerate}
\ifthenelse{\boolean{showSolutions}}{}{\end{multicols} }

\vspace{1em}
\section*{Calcul vectoriel}
\vspace{1em}

\subsection{Combinaisons linéaires}

\begin{itemize}
    \item Dans $\mathbb{R}^2$, $u=(1,2)$ est-il combinaison linéaire de $e_1=(1,-2)$ et $e_2=(2,3)$ ?
    \item Dans $\mathbb{R}^2$, $u=(1,2)$ est-il combinaison linéaire de $e_1=(1,-2), e_2=(2,3), e_3=(-4,5)$ ?
    \item Dans $\mathbb{R}^3$, $u=(2,5,3)$ est-il combinaison linéaire de $e_1=(1,3,2)$ et $e_2=(1,-1,4)$ ?
    \item Dans $\mathbb{R}^3$, $u=(3,1, m)$ est-il combinaison linéaire de $e_1=(1,3,2)$ et $e_2=(1,-1,4)$ ? \newline 
    (discuter suivant la valeur de $m$ ) 
\end{itemize}
Si oui, donner toutes les combinaisons linéaires possibles.

\ifthenelse{\boolean{showSolutions}}{
  \vspace{1em}

\begin{mdframed}

  \begin{enumerate}
    \item $u$ est combinaison linéaire de $e_1$ et $e_2$ si et seulement si il existe $a, b \in \mathbb{R}$ tels que $u=a e_1 + b e_2$.

    Trouver $a$ et $b$ nous conduit à un système linéaire : 

    \begin{align*}
      a + 2b &= 1 \\
      -2a + 3b &= 2
    \end{align*}

    On trouve $a=-1/7$ et $b=4/7$. Donc $u$ est combinaison linéaire de $e_1$ et $e_2$.

    \item $u$ est combinaison linéaire de $e_1, e_2$ et $e_3$ si et seulement si il existe $a, b, c \in \mathbb{R}$ tels que $u=a e_1 + b e_2 + c e_3$.

    Trouver $a$, $b$ et $c$ nous conduit à un système linéaire : 

    \begin{align*}
      a + 2b -4c &= 1 \\
      -2a + 3b + 5c &= 2 \\
    \end{align*}

    La première étape du pivot de gauss nous donne : 
    \begin{align*}
      a + 2b -4c &= 1 \\
      0 + 7b - 3c &= 3 \\
    \end{align*}
    On peut choisir $c$ comme on veut dans $\mathbb{R}$. $b$ et $a$ sont ensuite déterminés en fonction de $c$.

    \item $u$ est combinaison linéaire de $e_1$ et $e_2$ si et seulement si il existe $a, b \in \mathbb{R}$ tels que $u=a e_1 + b e_2$.

    Trouver $a$ et $b$ nous conduit à un système linéaire : 

    \begin{align*}
      a + b &= 2 \\
      3a - b &= 5 \\
      2a + 4 b &= 3
    \end{align*}

    Les premières étapes du pivot de Gauss nous donnent : 
    \begin{align*}
      a + b &= 2 \\
      0 - 4b &= 1 \\
      0 + 2b &= -1
    \end{align*}
    Le système est donc incompatible et $u$ n'est pas combinaison linéaire de $e_1$ et $e_2$.

    \item $u$ est combinaison linéaire de $e_1$ et $e_2$ si et seulement si il existe $a, b \in \mathbb{R}$ tels que $u=a e_1 + b e_2$.

    Trouver $a$ et $b$ nous conduit à un système linéaire : 

    \begin{align*}
      a + b &= 3 \\
      3a - b &= 1 \\
      2a + 4 b &= m
    \end{align*}

    Les premières étapes du pivot de Gauss nous donnent : 
    \begin{align*}
      a + b &= 3 \\
      0 - 4b &= -8 \\
      0 + 2b &= m-6
    \end{align*}
    Le système est compatible si et seulement si $m = 10$, dans ce cas, $u$ est combinaison linéaire de $e_1$ et $e_2$.

    Si $m \neq 10$, le système est incompatible et $u$ n'est pas combinaison linéaire de $e_1$ et $e_2$.
  \end{enumerate}
\end{mdframed}
}{}
\vspace{2em}
\subsection{Sous-espace engendré}


Dans $\mathbb{R}^3$, on pose $u_1=(1, -1, 2)$ et $u_2=(1, 1, -1)$.
\begin{itemize}
    \item Les vecteurs $v_1=(3, 1, 0)$ et $v_2=(1, 5, -1)$ sont-ils combinaison linéaire de $u_1$ et $u_2$ ?
    \item Soit $a, b, c \in \mathbb{R}$. Démontrer que $v=(a, b, c)$ est combinaison linéaire de $u_1$ et $u_2$ si et seulement si $-a+3 b+2 c=0$.
    \item En déduire un vecteur de $\mathbb{R}^3$ qui n'est pas combinaison linéaire de $u_1$ et de $u_2$.
\end{itemize}

\ifthenelse{\boolean{showSolutions}}{
  \vspace{1em}
  \begin{mdframed}
    \begin{enumerate}
      \item $v_1$ est combinaison linéaire de $u_1$ et $u_2$ si et seulement si il existe $a, b \in \mathbb{R}$ tels que $v_1=a u_1 + b u_2$.

      Trouver $a$ et $b$ nous conduit à un système linéaire : 

      \begin{align*}
        a + b &= 3 \\
        -a + b &= 1 \\
        2a - b &= 0
      \end{align*}

      Les premières étapes du pivot de Gauss nous donnent : 
      \begin{align*}
        a + b &= 3 \\
        0 + 2b &= 4 \\
        0 -3b &= -6
      \end{align*}
      Les deux dernières lignes correspondent à la même équation, le système possède donc une solution. 

      \item $v_2$ est combinaison linéaire de $u_1$ et $u_2$ si et seulement si il existe $a, b \in \mathbb{R}$ tels que $v_2=a u_1 + b u_2$.

      Trouver $a$ et $b$ nous conduit à un système linéaire : 

      \begin{align*}
        a + b &= 1 \\
        -a + b &= 5 \\
        2a - b &= -1
      \end{align*}
      Les premières étapes du pivot de Gauss nous donnent : 
      \begin{align*}
        a + b &= 1 \\
        0 + 2b &= 6 \\
        0 - 3b &= -3
      \end{align*}
      Le système est incompatible et $v_2$ n'est pas combinaison linéaire de $u_1$ et $u_2$.

      \item $v=(a, b, c)$ est combinaison linéaire de $u_1$ et $u_2$ si et seulement si le système suivant possède une solution : 
      \begin{align*}
        x + y &= a \\
        -x + y &= b \\
        2x - y &= c
      \end{align*}
      Les premières étapes du pivot de Gauss nous donnent : 
      \begin{align*}
        x + y &= a \\
        0 + 2y &= a + b \\
        0 - 3y &= c- 2a
      \end{align*}
      qu'on peut réécrire : 
      \begin{align*}
        x + y &= a \\
        y &= (a + b)/2 \\
        y &= (2a-c)/3
      \end{align*}
      Le système est compatible si et seulement si $(2a-c)/3 = (a+b)/2$, c'est-à-dire
      $$
      4a - 2c = 3a + 3b
      $$
      c'est-à-dire
      $$
      a - 3b - 2c = 0
      $$
      c'est bien l'équation de l'énoncé.

      \item Il suffit de trouver trois nombres $a, b, c$ qui ne vérifient pas l'équation de l'énoncé. Par exemple, $(1,0,0)$ n'est pas combinaison linéaire de $u_1$ et $u_2$.
    \end{enumerate}
  \end{mdframed}
}{}


\ifthenelse{\boolean{showSolutions}}{}{
    \newpage 
}

\section*{Familles}
\vspace{1em}
\subsection{Familles libres}

Les familles suivantes sont-elles libres dans $\mathbb{R}^3$ ?
\begin{itemize}
    \item $(u, v)$ avec $u=(1,2,3)$ et $v=(-1,4,6)$;
    \item $(u, v, w)$ avec $u=(1,2,-1), v=(1,0,1)$ et $w=(0,0,1)$;
    \item $(u, v, w)$ avec $u=(1,2,-1), v=(1,0,1)$ et $w=(-1,2,-3)$;
\end{itemize}

Sans calcul supplémentaire, dire si elles sont génératrices. 

\vspace{3em}
\subsection{Dimension}
On considère, dans $\mathbb{R}^4$, les vecteurs :

$$
v_1=(1,2,3,4), \quad v_2=(1,1,1,3), \quad v_3=(2,1,1,1), \quad v_4=(-1,0,-1,2), \quad v_5=(2,3,0,1) .
$$


Soit $F$ l'espace vectoriel engendré par $\left\{v_1, v_2, v_3\right\}$ et soit $G$ celui engendré par $\left\{v_4, v_5\right\}$. Calculer les dimensions respectives de $F, G, F \cap G, F+G$.


\ifthenelse{\boolean{showSolutions}}{
    \vspace{1em}

\begin{mdframed}


    \begin{enumerate}
    \item $G$ est engendré par deux vecteurs donc $\operatorname{dim} G \leqslant 2$. Clairement $v_4$ et $v_5$ ne sont pas liés donc $\operatorname{dim} G \geqslant 2$ c'est-à-dire $\operatorname{dim} G=2$.
    \item $F$ est engendré par trois vecteurs donc $\operatorname{dim} F \leqslant 3$. Un calcul montre que la famille $\left\{v_1, v_2, v_3\right\}$ est libre, d'où $\operatorname{dim} F \geqslant 3$ et donc $\operatorname{dim} F=3$.
    \item Essayons d'abord d'estimer la dimension de $F \cap G$.
    
    D'une part $F \cap G \subset G$ donc $\operatorname{dim}(F \cap G) \leqslant 2$. 
    
    Utilisons d'autre part la formule $\operatorname{dim}(F+G)=\operatorname{dim} F+\operatorname{dim} G-\operatorname{dim}(F \cap G)$. 
    
    Comme $F+G \subset \mathbb{R}^4$, on a $\operatorname{dim}(F+ G) \leqslant 4$ d'où on tire l'inégalité $\operatorname{dim}(F \cap G) \geqslant 1$. \newline 
    Donc soit $\operatorname{dim}(F \cap G)=1$ ou bien $\operatorname{dim}(F \cap G)=2$.

    Supposons que $\operatorname{dim}(F \cap G)$ soit égale à 2. 
    Comme $F \cap G \subset G$ on aurait dans ce cas $F \cap G=G$ et donc $G \subset F$.
    En particulier il existerait $\alpha, \beta, \gamma \in \mathbb{R}$ tels que $v_4=\alpha v_1+\beta v_2+\gamma v_3$. 
    On vérifie aisément que ce n'est pas le cas, ainsi $\operatorname{dim}(F \cap G)$ n'est pas égale à 2. 
    
    On peut donc conclure $\operatorname{dim}(F \cap G)=1$

    \item Par la formule $\operatorname{dim}(F+G)=\operatorname{dim} F+\operatorname{dim} G-\operatorname{dim}(F \cap G)$, on obtient $\operatorname{dim}(F+G)=2+3-1=4$. Cela entraîne $F+G=\mathbb{R}^4$.
    \end{enumerate}
\end{mdframed}
}{}

\section*{Champs de vecteurs}
\vspace{1em}

\subsection{Gradient et divergence}

Déterminer les coordonnées de $\operatorname{grad}(f)$ où $f$ est le champ scalaire suivant:
\begin{enumerate}[itemsep=0.5em]
    \item $f(x, y, z)=x y^2-y z^2$.
    \item $f(x, y, z)=x y z \sin (x y)$.
\end{enumerate}

\vspace{1em}
Déterminer $\operatorname{div}(f)$ où $f$ est le champ de vecteurs suivant:
\begin{enumerate}[itemsep=0.5em]
    \item $f(x, y, z)=\big(2 x^2 y, 2 x y^2, x y\big)$.
    \item $f(x, y, z)=\big(\sin (x y), 0, \cos (x z)\big)$.
    \item $f(x, y, z)=\big(x(2 y+z),-y(x+z), z(x-2 y)\big)$.
\end{enumerate}


\ifthenelse{\boolean{showSolutions}}{
    \vspace{2em}

\begin{mdframed}
Pour les gradients : 
\begin{enumerate}[itemsep=0.5em]
    \item $\operatorname{grad}(f)=\left(y^2, 2 x y-z^2,-2 y z\right)$.
    \item $\operatorname{grad}(f)=\left(y z \sin (x y)+x y^2 z \cos (x y), x z \sin (x y)+x^2 y z \cos (x y), x y \sin (x y)\right)$.
\end{enumerate}


\vspace{1em}
Pour les divergences : 
\begin{enumerate}[itemsep=0.5em]
    \item $\operatorname{div}(f)=8 x y$.
    \item $\operatorname{div}(f)=y \cos (x y)-x \sin (x z)$.
    \item $\operatorname{div}(f)=0$.
\end{enumerate}
\end{mdframed}
}{}

\vspace{2em}
\subsection{Potentiel scalaire}

On rappelle qu'on dit qu'un champ de vecteurs $F$ dérive d'un potentiel scalaire s'il existe un champ scalaire $f$ tel que $F=\operatorname{grad}(f)$. 
\begin{enumerate}[itemsep=0.5em]
    \item $F(x, y, z)=\left(2 x y+z^3, x^2, 3 x z^2\right)$, défini sur $\mathbb{R}^3$.
    \item $F(x, y)=\left(-\frac{y}{(x-y)^2}, \frac{x}{(x-y)^2}\right)$, défini sur $U=\left\{(x, y) \in \mathbb{R}^2, x>y\right\}$.
\end{enumerate}

Pour chacun des champs de vecteurs précédents, montrer qu'ils sont définis sur un ouvert étoilé, que leur rotationnel est nul, et qu'ils dérivent d'un potentiel scalaire. \newline 
 Déterminer tous les potentiels scalaires dont ils dérivent.   

\ifthenelse{\boolean{showSolutions}}{
    \vspace{1em}
    \begin{mdframed}

D'une part, on peut prouver l'existence théorique d'un tel champ scalaire, en observant que les champs de vecteurs sont définis sur des ouverts étoilés, et que leur rotationnel est nul. 

D'autre part, il est possible de les "intégrer". On cherche en effet:
\begin{enumerate}[itemsep=0.5em]
    \item $f$ de $\mathbb{R}^3$ dans $\mathbb{R}$ tel que $\frac{\partial f}{\partial x}=2 x y+z^3, \frac{\partial f}{\partial y}=x^2$ et $\frac{\partial f}{\partial z}=3 x z^2$. On résoud ce système d'équation aux dérivées partielles: la deuxième équation donne par exemple $f(x, y, z)=x^2 y+h(x, z)$, et utilisant les deux autres équations, on trouve:

$$
f(x, y, z)=x^2 y+x z^3+\text { cste. }
$$

    \item $f$ de $\mathbb{R}^2$ dans $\mathbb{R}$ tel que

$$
\frac{\partial f}{\partial x}=-\frac{y}{(x-y)^2} \text { et } \frac{\partial f}{\partial y}=\frac{x}{(x-y)^2} .
$$


Les fonctions $f$ qui conviennent sont donc $f(x, y)=\frac{y}{x-y}+c s t e$.
\end{enumerate}
\end{mdframed}
}{}