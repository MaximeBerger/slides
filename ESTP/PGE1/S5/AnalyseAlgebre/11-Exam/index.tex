\documentclass[12pt]{article}
\usepackage[french]{babel}
\usepackage[utf8]{inputenc}
\usepackage[T1]{fontenc}
\usepackage{lmodern}           % Police Latin Modern (plus nette)
\usepackage{charter}           % Police Charter (très lisible)
\usepackage[scaled=0.95]{inconsolata} % Police mono lisible
\usepackage{amsmath}
\usepackage{amsfonts}
\usepackage{amssymb}
\usepackage{amsthm}
\usepackage[version=4]{mhchem}
\usepackage{stmaryrd}
\usepackage[most]{tcolorbox}
\usepackage{xcolor}
\usepackage{geometry}
\geometry{margin=1.5cm}

\usepackage{mdframed}
\usepackage[sf]{titlesec}
\usepackage{array}
\usepackage{ifthen}
% Définition de la variable pour afficher les corrections
\newboolean{showSolutions}
% Décommentez la ligne suivante pour afficher les solutions
\input \jobname.adr

\title{Examen S5 - Mathématiques - Analyse et Algèbre }

\author{}
\date{}


\newenvironment{solution}
{
    \vspace{0.5em}
    \begin{mdframed}[backgroundcolor=ThemeLight,leftmargin=0,rightmargin=0,skipabove=0.2em,skipbelow=0.2em]
    \textbf{Solution.}\\[0.5em]
}
{
    \end{mdframed}
    \vspace{0.5em}
}
\begin{document}
\sffamily

\begin{center}
    \renewcommand{\arraystretch}{1.5} % Ajuste l'espacement vertical des lignes
    \begin{tabular}{|>{\centering\arraybackslash}m{4cm}|>{\centering\arraybackslash}m{6cm}|>{\centering\arraybackslash}m{4cm}|}
        \hline 
        \vspace{5mm} \hspace{5mm}\raisebox{-0.2\height}{\includegraphics[width=3cm]{Logo-ESTP.png}} \vspace{5mm}  & 
        \textbf{Contrôle de connaissances et de compétences} & 
        \textbf{FO-002-VLA-XX-001} \\
        \hline
        \textbf{16/01/2026}  &  & \textbf{Page 1/3} \\
        \hline
    \end{tabular}
\end{center}
\vspace{1em}

\begin{center}
    \renewcommand{\arraystretch}{1.5}
    \begin{tabular}{|c|m{10cm}|}
        \hline 
        \multicolumn{2}{|c|}{\textbf{ANNÉE SCOLAIRE 2025-2026 -- Semestre 1}} \\
        \hline 
        \textbf{Nom de l'enseignant} & Maxime Berger \\
        \hline 
        \textbf{Matière} & Mathématiques - Analyse et Algèbre \\
        \hline 
        \textbf{Promotion} & PGE1 - S5 \\
        \hline 
        \textbf{Durée de l'examen} & 2h00 \\
        \hline 
        \textbf{Consignes} & 
        \vspace{0.5em}
        \begin{itemize}
            \item Calculatrice \textbf{NON} autorisée
            \item Aucun document n'est autorisé \vspace{1em}
        \end{itemize}\\
        
        \hline
    \end{tabular}
\end{center}

\vspace{3em}
%==============================================================================
\section*{Exercice 1 : Calcul tensoriel (4 points)}
%==============================================================================

On se place dans $\mathbb{R}^3$ muni d'une base $\{\mathbf{e}_1, \mathbf{e}_2, \mathbf{e}_3\}$ qui n'est pas orthonormée. On donne les produits scalaires entre les vecteurs de la base :
\[
\langle \mathbf{e}_1, \mathbf{e}_1 \rangle = 2, \quad \langle \mathbf{e}_1, \mathbf{e}_2 \rangle = 1, \quad \langle \mathbf{e}_1, \mathbf{e}_3 \rangle = 0
\]
\[
\langle \mathbf{e}_2, \mathbf{e}_2 \rangle = 3, \quad \langle \mathbf{e}_2, \mathbf{e}_3 \rangle = -1, \quad \langle \mathbf{e}_3, \mathbf{e}_3 \rangle = 1
\]

\begin{enumerate}
    \item Former la matrice $(g_{ij})$ avec $g_{ij} = \langle \mathbf{e}_i, \mathbf{e}_j \rangle$. \textit{(1 pt)}
    
    \ifthenelse{\boolean{showSolutions}}{
    \begin{solution}
    Comme le produit scalaire est symétrique, la matrice $(g_{ij})$ est :
    \[
    (g_{ij}) = \begin{pmatrix} 2 & 1 & 0 \\ 1 & 3 & -1 \\ 0 & -1 & 1 \end{pmatrix}
    \]
    \end{solution}
    }{}
    
    \item Soit le vecteur $\mathbf{v} = (1, -1, 2)$ exprimé dans la base $\{\mathbf{e}_1, \mathbf{e}_2, \mathbf{e}_3\}$.
    Quelles sont les composantes contravariantes de $\mathbf{v}$ ? \textit{(0.5 pt)}
    
    \ifthenelse{\boolean{showSolutions}}{
    \begin{solution}
    Les composantes contravariantes sont directement les coordonnées données dans la base :
    \[
    v^1 = 1, \quad v^2 = -1, \quad v^3 = 2
    \]
    \end{solution}
    }{}
    
    \item Calculer les composantes covariantes $v_1$, $v_2$, $v_3$ de ce vecteur. \textit{(1 pt)}
    
    \ifthenelse{\boolean{showSolutions}}{
    \begin{solution}
    Les composantes covariantes se calculent avec le tenseur métrique : $v_i = g_{ij} v^j$.
    \[
    v_1 = g_{11}v^1 + g_{12}v^2 + g_{13}v^3 = 2 \cdot 1 + 1 \cdot (-1) + 0 \cdot 2 = 2 - 1 = 1
    \]
    \[
    v_2 = g_{21}v^1 + g_{22}v^2 + g_{23}v^3 = 1 \cdot 1 + 3 \cdot (-1) + (-1) \cdot 2 = 1 - 3 - 2 = -4
    \]
    \[
    v_3 = g_{31}v^1 + g_{32}v^2 + g_{33}v^3 = 0 \cdot 1 + (-1) \cdot (-1) + 1 \cdot 2 = 1 + 2 = 3
    \]
    \end{solution}
    }{}
    
    \item Calculer la norme du vecteur $\mathbf{v}$. \textit{(1 pt)}
    
    \ifthenelse{\boolean{showSolutions}}{
    \begin{solution}
    \[
    \|\mathbf{v}\|^2 = v^1 v_1 + v^2 v_2 + v^3 v_3 = 1 \cdot 1 + (-1) \cdot (-4) + 2 \cdot 3 = 1 + 4 + 6 = 11
    \]
    Donc $\|\mathbf{v}\| = \sqrt{11}$.
    \end{solution}
    }{}

    \item Rappelez la définition d'un tenseur, puis donner un exemple de quantité qui est un tenseur et une quantité qui n'est pas un tenseur. \textit{(0.5 pt)}
    \ifthenelse{\boolean{showSolutions}}{
    \begin{solution}
    Un tenseur est une quantité qui se transforme lors d'un changement de base de manière linéaire en multipliant par la matrice de passage ou son inverse plusieurs fois.
    Un exemple de quantité qui est un tenseur est le vecteur vitesse.
    Un exemple de quantité qui n'est pas un tenseur est le vecteur position.
    \end{solution}
    }{}
\end{enumerate}

\vspace{2em}


\newpage

\begin{center}
    \renewcommand{\arraystretch}{1.5} 
    \begin{tabular}{|>{\centering\arraybackslash}m{4cm}|>{\centering\arraybackslash}m{6cm}|>{\centering\arraybackslash}m{4cm}|}
        \hline
            \hspace{4cm}&\hspace{6cm} & \textbf{Page 2/3}\\
            \hline
    \end{tabular}
\end{center}

\vspace{1em}

%==============================================================================
\section*{Exercice 2 : Espaces $L^p$ (4 points)}
%==============================================================================

On considère les fonctions suivantes :
\[
f(x) = \frac{1}{x^2}, \qquad g(x) = e^{ix -2x}, \qquad h(x) = \sin(3x^2)
\]

\begin{enumerate}
    \item Déterminer si $f$ appartient à $L^1(]1, +\infty[)$. Si oui, calculer sa norme. \textit{(1.5 pts)}
    
    \ifthenelse{\boolean{showSolutions}}{
    \begin{solution}
    \[
    \int_1^{+\infty} \frac{1}{x^2} dx = \left[ -\frac{1}{x} \right]_1^{+\infty} = 0 - (-1) = 1
    \]
    L'intégrale converge, donc $f \in L^1(]1, +\infty[)$ et $\|f\|_1 = 1$.
    \end{solution}
    }{}
    
    \item Déterminer si $g$ appartient à $L^2(\mathbb{R}^*_+)$. Si oui, calculer sa norme. \textit{(1.5 pts)}
    
    \ifthenelse{\boolean{showSolutions}}{
    \begin{solution}
    \[
    \int_0^{+\infty} |e^{ix -2x}|^2 dx = \int_0^{+\infty} e^{-4x} dx = \left[ -\frac{e^{-4x}}{4} \right]_0^{+\infty} = \frac{1}{4}
    \]
    L'intégrale converge, donc $g \in L^2(\mathbb{R}^*_+)$ et $\|g\|_2 = \sqrt{1/4} = \frac{1}{2}$.
    \end{solution}
    }{}
    
    \item Déterminer si $h$ appartient à $L^\infty(\mathbb{R})$. Si oui, donner sa norme. \textit{(1 pt)}
    
    \ifthenelse{\boolean{showSolutions}}{
    \begin{solution}
    $|\sin(x)| \leq 1$ pour tout $x \in \mathbb{R}$, et cette borne est atteinte (par exemple en $x = \pi/2$).
    
    Donc $h \in L^\infty(\mathbb{R})$ et $\|h\|_\infty = 1$.
    \end{solution}
    }{}
    
\end{enumerate}
\vspace{2em}
%==============================================================================
\section*{Exercice 3 : Transformée de Laplace (4 points)}
%==============================================================================

On note $\mathcal{L}\{f\}(s)$ la transformée de Laplace de $f$.

\begin{enumerate}
    \item Calculer $\mathcal{L}\{e^{-3t}\}(s)$ directement à partir de la définition et préciser le domaine de convergence. \textit{(1 pt)}
    
    \ifthenelse{\boolean{showSolutions}}{
    \begin{solution}
    \[
    \mathcal{L}\{e^{-3t}\}(s) = \int_0^{+\infty} e^{-3t} e^{-st} dt = \int_0^{+\infty} e^{-(s+3)t} dt = \left[ -\frac{e^{-(s+3)t}}{s+3} \right]_0^{+\infty} = \frac{1}{s+3}
    \]
    Cette intégrale converge pour $\text{Re}(s) > -3$.
    \end{solution}
    }{}
    
    \item En utilisant la propriété de décalage en fréquence $\mathcal{L}\{e^{at}f(t)\} = F(s-a)$, calculer $\mathcal{L}\{t^2 e^{-3t}\}$. \textit{(1 pt)}
    
    \textit{On rappelle que $\mathcal{L}\{t^n\} = \frac{n!}{s^{n+1}}$.}
    
    \ifthenelse{\boolean{showSolutions}}{
    \begin{solution}
    On utilise le décalage avec $f(t) = t^2$ et $a = -3$.
    
    On a $\mathcal{L}\{t^2\} = \frac{2!}{s^3} = \frac{2}{s^3}$.
    
    Par décalage en fréquence :
    \[
    \mathcal{L}\{t^2 e^{-3t}\} = \frac{2}{(s+3)^3}
    \]
    \end{solution}
    }{}
    
    \item Résoudre l'équation différentielle $y' + 2y = e^{-t}$ avec $y(0) = 0$ en utilisant la transformée de Laplace. \textit{(2 pts)}
    
    \ifthenelse{\boolean{showSolutions}}{
    \begin{solution}
    \textbf{Étape 1 : Transformation.} En appliquant $\mathcal{L}$ à l'équation :
    \[
    sY(s) - y(0) + 2Y(s) = \frac{1}{s+1}
    \]
    Avec $y(0) = 0$ :
    \[
    (s+2)Y(s) = \frac{1}{s+1}
    \]
    
    \textbf{Étape 2 : Résolution algébrique.}
    \[
    Y(s) = \frac{1}{(s+1)(s+2)}
    \]
    
    Décomposition en éléments simples :
    \[
    \frac{1}{(s+1)(s+2)} = \frac{A}{s+1} + \frac{B}{s+2}
    \]
    \begin{itemize}
        \item $s = -1$ : $1 = A(1)$, donc $A = 1$
        \item $s = -2$ : $1 = B(-1)$, donc $B = -1$
    \end{itemize}
    
    Donc $Y(s) = \frac{1}{s+1} - \frac{1}{s+2}$.
    
    \textbf{Étape 3 : Transformation inverse.}
    \[
    y(t) = e^{-t} - e^{-2t}
    \]
    \end{solution}
    }{}
\end{enumerate}

\vspace{2em}

%==============================================================================
\section*{Exercice 4 : Équations aux dérivées partielles (4 points)}
%==============================================================================

\begin{enumerate}
    \item Soit $f(x, y) = e^{2x} \sin(3y)$. Calculer les dérivées partielles $\frac{\partial f}{\partial x}$, $\frac{\partial f}{\partial y}$, $\frac{\partial^2 f}{\partial x^2}$ et $\frac{\partial^2 f}{\partial y^2}$. \textit{(1 pt)}
    
    \ifthenelse{\boolean{showSolutions}}{
    \begin{solution}
    \[
    \frac{\partial f}{\partial x} = 2e^{2x}\sin(3y), \qquad \frac{\partial f}{\partial y} = 3e^{2x}\cos(3y)
    \]
    \[
    \frac{\partial^2 f}{\partial x^2} = 4e^{2x}\sin(3y), \qquad \frac{\partial^2 f}{\partial y^2} = -9e^{2x}\sin(3y)
    \]
    \end{solution}
    }{}
    
    \item Vérifier que $f$ est solution de l'équation $\frac{\partial^2 f}{\partial x^2} + \frac{\partial^2 f}{\partial y^2} = -5f$. \textit{(0.5 pt)}
    
    \ifthenelse{\boolean{showSolutions}}{
    \begin{solution}
    \[
    \frac{\partial^2 f}{\partial x^2} + \frac{\partial^2 f}{\partial y^2} = 4e^{2x}\sin(3y) - 9e^{2x}\sin(3y) = -5e^{2x}\sin(3y) = -5f
    \]
    L'équation est bien vérifiée.
    \end{solution}
    }{}
    
    \item On considère l'équation des ondes $\frac{\partial^2 u}{\partial t^2} = c^2 \frac{\partial^2 u}{\partial x^2}$ sur $[0, L] \times \mathbb{R}_+$ avec conditions aux bords $u(0,t) = u(L,t) = 0$.
    
    On cherche une solution sous la forme $u(x,t) = X(x)T(t)$. Montrer que l'équation devient :
    \[
    \frac{T''(t)}{c^2 T(t)} = \frac{X''(x)}{X(x)} = -\lambda
    \]
    où $\lambda$ est une constante. \textit{(1 pt)}
    
    \ifthenelse{\boolean{showSolutions}}{
    \begin{solution}
    En substituant $u(x,t) = X(x)T(t)$ :
    \[
    X(x)T''(t) = c^2 X''(x)T(t)
    \]
    En divisant par $c^2 X(x)T(t)$ :
    \[
    \frac{T''(t)}{c^2 T(t)} = \frac{X''(x)}{X(x)}
    \]
    Le membre de gauche ne dépend que de $t$, le membre de droite ne dépend que de $x$. Pour que l'égalité soit vraie pour tout $(x,t)$, les deux membres doivent être égaux à une constante $-\lambda$.
    \end{solution}
    }{}
    
    \item En déduire les deux équations différentielles vérifiées par $X$ et $T$. \textit{(0.5 pt)}
    
    \ifthenelse{\boolean{showSolutions}}{
    \begin{solution}
    \begin{itemize}
        \item Pour $T$ : $T'' + \lambda c^2 T = 0$
        \item Pour $X$ : $X'' + \lambda X = 0$
    \end{itemize}
    \end{solution}
    }{}
    
    \item En prenant $\lambda = \mu^2 > 0$, montrer que les conditions aux bords imposent $\mu = \frac{n\pi}{L}$ avec $n \in \mathbb{N}^*$, puis donner la forme générale de $T(t)$. \textit{(1 pt)}
    
    \ifthenelse{\boolean{showSolutions}}{
    \begin{solution}
    Pour $\lambda = \mu^2 > 0$, la solution générale de $X'' + \mu^2 X = 0$ est :
    \[
    X(x) = A\cos(\mu x) + B\sin(\mu x)
    \]
    
    Avec les conditions aux bords :
    \begin{itemize}
        \item $X(0) = 0$ : $A = 0$
        \item $X(L) = 0$ : $B\sin(\mu L) = 0$
    \end{itemize}
    
    Pour avoir $B \neq 0$ (solution non triviale), il faut $\sin(\mu L) = 0$, soit $\mu L = n\pi$ pour $n \in \mathbb{N}^*$.
    
    Donc $\mu_n = \frac{n\pi}{L}$.
    
    Pour $T$, l'équation $T'' + \mu_n^2 c^2 T = 0$ a pour solution générale :
    \[
    T_n(t) = C_n \cos\left(\frac{n\pi c}{L} t\right) + D_n \sin\left(\frac{n\pi c}{L} t\right)
    \]
    \end{solution}
    }{}
\end{enumerate}

\newpage

\begin{center}
    \renewcommand{\arraystretch}{1.5} 
    \begin{tabular}{|>{\centering\arraybackslash}m{4cm}|>{\centering\arraybackslash}m{6cm}|>{\centering\arraybackslash}m{4cm}|}
        \hline
            \hspace{4cm}&\hspace{6cm} & \textbf{Page 3/3}\\
            \hline
    \end{tabular}
\end{center}

\vspace{1em}

%==============================================================================
\section*{Exercice 5 : Distributions (4 points)}
%==============================================================================

On rappelle que la distribution de Dirac $\delta_a$ est définie par $\langle \delta_a, \varphi \rangle = \varphi(a)$ pour toute fonction test $\varphi$.

On note $H$ la fonction de Heaviside : $H(x) = 1$ si $x \geq 0$, $H(x) = 0$ si $x < 0$.

\begin{enumerate}
    
    
    \item Soit $f(x) = H(x) - H(x-3)$ (\og fonction porte \fg{} sur $[0,3]$). Calculer la dérivée de $f$ au sens des distributions. \textit{(1 pt)}
    
    \ifthenelse{\boolean{showSolutions}}{
    \begin{solution}
    La fonction $f$ vaut $1$ sur $[0,3[$ et $0$ ailleurs.
    
    \begin{itemize}
        \item La dérivée classique est nulle partout où $f$ est continue.
        \item Saut en $x = 0$ : $\sigma_0 = 1 - 0 = 1$
        \item Saut en $x = 3$ : $\sigma_3 = 0 - 1 = -1$
    \end{itemize}
    
    Donc $f' = \delta_0 - \delta_3$.
    \end{solution}
    }{}
    
    \item Soit $g$ la fonction définie par :
    \[
    g(x) = \begin{cases}
    x^2 & \text{si } x < 1 \\
    2x + 1 & \text{si } x \geq 1
    \end{cases}
    \]
    Calculer la dérivée de $g$ au sens des distributions en utilisant la formule des sauts. \textit{(1 pt)}
    
    \ifthenelse{\boolean{showSolutions}}{
    \begin{solution}
    On calcule d'abord le saut en $x = 1$ :
    \begin{itemize}
        \item Limite à gauche : $\lim_{x \to 1^-} g(x) = 1$
        \item Limite à droite : $\lim_{x \to 1^+} g(x) = 3$
        \item Saut : $\sigma_1 = 3 - 1 = 2$
    \end{itemize}
    
    La dérivée classique (partie régulière) est :
    \[
    \{g'\}(x) = \begin{cases}
    2x & \text{si } x < 1 \\
    2 & \text{si } x > 1
    \end{cases}
    \]
    
    Par la formule des sauts : $g' = \{g'\} + \sigma_1 \delta_1$
    
    \textbf{Donc :} $g'(x) = \{g'\}(x) + 2\delta_1$
    \end{solution}
    }{}
    
    \item Soit $h$ la fonction définie par :
    \[
    h(x) = \begin{cases}
    e^x & \text{si } x < 0 \\
    x + 1 & \text{si } 0 \leq x < 2 \\
    5 & \text{si } x \geq 2
    \end{cases}
    \]
    Calculer la dérivée de $h$ au sens des distributions et écrivez $\langle h', \varphi \rangle$ pour une fonction test $\varphi$. \textit{(2 pts)}
    
    \ifthenelse{\boolean{showSolutions}}{
    \begin{solution}
    On identifie les points de discontinuité et on calcule les sauts :
    
    \textbf{En $x = 0$ :}
    \begin{itemize}
        \item Limite à gauche : $\lim_{x \to 0^-} h(x) = e^0 = 1$
        \item Limite à droite : $\lim_{x \to 0^+} h(x) = 0 + 1 = 1$
        \item Saut : $\sigma_0 = 1 - 1 = 0$ (pas de discontinuité)
    \end{itemize}
    
    \textbf{En $x = 2$ :}
    \begin{itemize}
        \item Limite à gauche : $\lim_{x \to 2^-} h(x) = 2 + 1 = 3$
        \item Limite à droite : $\lim_{x \to 2^+} h(x) = 5$
        \item Saut : $\sigma_2 = 5 - 3 = 2$
    \end{itemize}
    
    La dérivée classique est :
    \[
    \{h'\}(x) = \begin{cases}
    e^x & \text{si } x < 0 \\
    1 & \text{si } 0 < x < 2 \\
    0 & \text{si } x > 2
    \end{cases}
    \]
    
    \textbf{Donc :} $h'(x) = \{h'\}(x) + 2\delta_2$

    Donc 
    $$\langle h', \varphi \rangle = \int_{-\infty}^{+\infty} \{h'\}(x) \varphi(x) dx + 2\varphi(2) = 
    \int_{-\infty}^{0} e^x \varphi(x) dx + \int_{0}^{2} \varphi(x) dx + 2\varphi(2)$$
    \end{solution}
    }{}
\end{enumerate}

\vspace{1em}

\subsection*{Transformées de Laplace usuelles}

\renewcommand{\arraystretch}{3}
\begin{center}
\begin{tabular}{|>{\centering\arraybackslash}m{3.5cm}|>{\centering\arraybackslash}m{5cm}|>{\centering\arraybackslash}m{3cm}|}
\hline
\textbf{Fonction} $f(t)$ & \textbf{Transformée} $\mathcal{L}\{f\}(s)$ & \textbf{Domaine} \\
\hline\hline
$1$ & $\displaystyle \frac{1}{s}$ & $\text{Re}(s) > 0$ \\
\hline
$t^n$ & $\displaystyle \frac{n!}{s^{n+1}}$ & $\text{Re}(s) > 0$ \\
\hline
$e^{at}$ & $\displaystyle \frac{1}{s-a}$ & $\text{Re}(s) > a$ \\
\hline
$\sin(\omega t)$ & $\displaystyle \frac{\omega}{s^2 + \omega^2}$ & $\text{Re}(s) > 0$ \\
\hline
$\cos(\omega t)$ & $\displaystyle \frac{s}{s^2 + \omega^2}$ & $\text{Re}(s) > 0$ \\
\hline
$u(t-a)$ (Heaviside) & $\displaystyle \frac{e^{-as}}{s}$ & $\text{Re}(s) > 0$ \\
\hline
\end{tabular}
\end{center}

\vspace{1em}


\end{document}