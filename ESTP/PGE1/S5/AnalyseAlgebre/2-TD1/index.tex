\documentclass[11pt,a4paper]{report}

% -------------------- Encodage & langue --------------------
\usepackage[T1]{fontenc}
\usepackage[utf8]{inputenc}
\usepackage[french]{babel}
\usepackage{lmodern}
\usepackage{microtype}
\usepackage{amsmath, amssymb}
\usepackage{multicol}
\usepackage{enumitem}

\usepackage{amsfonts}
\usepackage[version=4]{mhchem}
\usepackage{stmaryrd}
\usepackage{graphicx}
\usepackage[export]{adjustbox}
\usepackage{caption}
\usepackage{multirow}
\usepackage{tikz}
% -------------------- Mise en page --------------------------
\usepackage[a4paper,margin=2cm]{geometry}
\usepackage{fancyhdr}
\usepackage{parskip}      % espace entre paragraphes
\setlength{\parindent}{0pt}

% -------------------- Couleurs & liens ----------------------
\usepackage{xcolor}
\definecolor{Theme}{HTML}{0E7490} % teal-700
\definecolor{ThemeLight}{HTML}{E0F2F1}
\definecolor{Accent}{HTML}{F59E0B} % amber-500
\definecolor{Gray}{HTML}{374151}
\usepackage[colorlinks=true,linkcolor=Theme,urlcolor=Theme,citecolor=Theme]{hyperref}

% -------------------- Graphiques / décor --------------------
\usepackage{tikz}
\usetikzlibrary{patterns,positioning,calc}
\usepackage{graphicx}
\usepackage{tcolorbox}
\tcbuselibrary{skins,breakable,hooks,most}
\usepackage{fontawesome5}

% -------------------- Titres -------------------------------
\usepackage{titlesec}
\titleformat{\chapter}[display]
  {\Huge\bfseries\color{Theme}}
  {\filright\rule{0.75\linewidth}{1.2pt}\\[3pt]{Algèbre linéaire - Chapitre~\thechapter}}
  {0.2ex}
  {\filright}
  [\vspace{0.1ex}\rule{0.35\linewidth}{1.2pt}]

\titleformat{\section}
  {\Large\bfseries\color{Gray}}
  {\thesection}{0.6em}{}

% -------------------- En-têtes / pieds ---------------------
\pagestyle{fancy}
\fancyhf{}
\fancyhead[L]{\color{Gray}\leftmark}
\fancyhead[R]{\color{Gray}\textit{Analyse-Algèbre - 2025/2026}}
\fancyfoot[R]{\color{Gray}\small p.\ \thepage}
\fancyfoot[L]{\color{Gray}\small \textit{Maxime Berger}}
\renewcommand{\headrulewidth}{0pt}
\renewcommand{\footrulewidth}{0pt}

% -------------------- Macros utilitaires -------------------
\newenvironment{solution}
{
    \vspace{0.5em}
    \begin{mdframed}[backgroundcolor=ThemeLight,leftmargin=0,rightmargin=0,skipabove=0.2em,skipbelow=0.2em]
    \textbf{Solution.}\\[0.5em]
}
{
    \end{mdframed}
    \vspace{0.5em}
}



% -------------------- Page de titre ------------------------
\title{\textbf{Traces de cours}\\\large (résumés, formules, exemples, mini-exercices)}
\author{ Analyse-Algèbre - 2025/2026 }
\date{\today}


\makeatletter
\renewcommand{\thesubsection}{\arabic{subsection}}
\renewcommand{\p@subsection}{}% supprime le préfixe section/chapter dans \ref
% Si vous voulez la même chose pour les sous-sous-sections :
% \renewcommand{\thesubsubsection}{\arabic{subsubsection}}
% \renewcommand{\p@subsubsection}{}
\makeatother

\usepackage{mdframed}
\usepackage{ifthen}

% \usepackage[sf]{titlesec}
% Définition de la variable pour afficher les corrections
\newboolean{showSolutions}
% Décommentez la ligne suivante pour afficher les solutions
\input \jobname.adr
% -------------------- Document ----------------------------
\begin{document}

\begin{center}
    {\LARGE \textbf{Analyse et Algèbre - TD1}}\\[1em]
    {\large \textit{Calcul vectoriel et tensoriel}}
\end{center}



\section*{Exercice 1 : Application directe}
On se place dans \(\mathbb{R}^3\) muni de sa base canonique.
    On définit également le vecteur \(\textbf{v}=(2,1,-3)\).

    \begin{enumerate}
    \item Rappeler les définitions puis déterminer les composantes covariantes \(v_i\) et contravariantes \(v^i\) de ce vecteur.
        
        \ifthenelse{\boolean{showSolutions}}{
        \begin{solution}
        Dans une base orthonormée, les composantes contravariantes et covariantes sont identiques :
        \[
        v_1 = v^1 = 2, \quad v_2 = v^2 = 1, \quad v_3 = v^3 = -3
        \]
        \end{solution}
        }{\vspace{1em}}


    \noindent Considérons maintenant une autre base qui n'est plus orthonormée \(\{\textbf{e}_1, \textbf{e}_2, \textbf{e}_3\}\). On donne les produits scalaires entre les vecteurs de la base :
    \[
      \langle \textbf{e}_1, \textbf{e}_1 \rangle = 4, \quad \langle \textbf{e}_1, \textbf{e}_2 \rangle = 1, \quad \langle \textbf{e}_1, \textbf{e}_3 \rangle = -1, \quad \langle \textbf{e}_2, \textbf{e}_2 \rangle = 2, \quad \langle \textbf{e}_2, \textbf{e}_3 \rangle = 1, \quad \langle \textbf{e}_3, \textbf{e}_3 \rangle = 1
    \]
    
    \item Former la matrice \((g_{ij})\) avec \(g_{ij} = \langle \textbf{e}_i, \textbf{e}_j \rangle\).
      \ifthenelse{\boolean{showSolutions}}{
      \begin{solution}
      Comme le produit scalaire est symétrique, la matrice \((g_{ij})\) est :
      \[
      (g_{ij}) = \begin{pmatrix} 4 & 1 & -1 \\ 1 & 2 & 1 \\ -1 & 1 & 1 \end{pmatrix}
      \]
      \end{solution}
      }{}
    \item Donner une condition sur la matrice \((g_{ij})\) pour que la base soit orthogonale.

        \ifthenelse{\boolean{showSolutions}}{
        \begin{solution}
        Pour que la base soit orthogonale, il faut que \(g_{ij} = 0\) si \(i \neq j\), donc que la matrice \((g_{ij})\) soit diagonale.
        \end{solution}
        }{}

    \item Le tenseur représenté par la matrice $g$ est appelée tenseur métrique, il code la géométrie de l'espace et est très utilisé en physique relativiste. De quel ordre est ce tenseur ? 

        \ifthenelse{\boolean{showSolutions}}{
        \begin{solution}
        Le tenseur $g$ est décrit avec $n^2$ coefficients, il est donc d'ordre $2$.
        \end{solution}
        }{}

    \item Soit le vecteur \(\textbf{w}=(1,2,1)\) exprimé dans la base \(\{\textbf{e}_1, \textbf{e}_2, \textbf{e}_3\}\). \newline
        Calculer les composantes contravariantes \(w^i\) et covariantes \(w_i\) de ce vecteur.
        
        \ifthenelse{\boolean{showSolutions}}{
        \begin{solution}
        Les composantes contravariantes sont les coordonnées données :
        \[
        w^1 = 1, \quad w^2 = 2, \quad w^3 = 1
        \]
        Les composantes covariantes se calculent avec le tenseur métrique :
        \[
        w_1 = \langle w, e_1 \rangle = g_{11}w^1 + g_{12}w^2 + g_{13}w^3 = 4 \cdot 1 + 1 \cdot 2 + (-1) \cdot 1 = 4 + 2 - 1 = 5
        \]
        \[
        w_2 = \langle w, e_2 \rangle = g_{21}w^1 + g_{22}w^2 + g_{23}w^3 = 1 \cdot 1 + 2 \cdot 2 + 1 \cdot 1 = 1 + 4 + 1 = 6
        \]
        \[
        w_3 = \langle w, e_3 \rangle = g_{31}w^1 + g_{32}w^2 + g_{33}w^3 = (-1) \cdot 1 + 1 \cdot 2 + 1 \cdot 1 = -1 + 2 + 1 = 2
        \]
        \end{solution}
        }{}
    \item Si les vecteurs de la base sont divisés par deux, quelles sont les nouvelles composantes contravariantes et covariantes du vecteur \(\textbf{w}\)?

    \ifthenelse{\boolean{showSolutions}}{
    \begin{solution}
      Les composantes contravariantes changent de manière inverse par rapport à la base, elles sont multipliées par deux. 

      Les composantes covariantes changent comme la base, elles sont divisées par deux.
     \end{solution}
    }{}
    \item Sous cette transformation, comment est modifié le premier coefficient de la matrice $g$ ? \newline 
    On pourra noter $f_1$, $f_2$, $f_3$ les nouveaux vecteurs de la base. et $G_{ij}$ la matrice dans la nouvelle base.
     Le tenseur métrique est-il 
    \begin{itemize}[label=$\bullet$]
        \item d'ordre $(2, 0)$ (2 fois contravariant) ?
        \item d'ordre $(1, 1)$ (1 fois contravariant et 1 fois covariant) ?
        \item ou d'ordre $(0, 2)$ (2 fois covariant) ?
    \end{itemize}
    \ifthenelse{\boolean{showSolutions}}{
    \begin{solution}
      On a :
      \[
      G_{11} = \langle f_1, f_1 \rangle = \langle \frac{1}{2}e_1, \frac{1}{2}e_1 \rangle = \frac{1}{4} \langle e_1, e_1 \rangle = \frac{1}{4} g_{11}
      \]
      Donc pour obtenir la nouvelle matrice, il faut appliquer la même transformation subie par la base à la matrice $g$. Le tenseur métrique est d'ordre $(0,2)$, 2 fois covariant.
    \end{solution}
    }{}
\end{enumerate}

\section*{Exercice 2 : Dans une vraie base}
    On se place dans \(\mathbb{R}^3\) muni de son produit scalaire canonique, et de la base suivante donnée dans la base canonique : 
    \[
    \Big\{\textbf{e}_1=(1,1,1),\, \textbf{e}_2=(0,1,1),\, \textbf{e}_3=(0,0,1)\Big\}
    \]
    on considère \(\textbf{u}=\left(u^1, u^2, u^3\right)\) dans la base \(\{\textbf{e}_1, \textbf{e}_2, \textbf{e}_3\}\), 
    \begin{enumerate}[itemsep=0.3em]
        \item Quelles sont les composantes contravariantes de \(\textbf{u}\) ?
        
        \ifthenelse{\boolean{showSolutions}}{
        \begin{solution}
        Les composantes contravariantes sont les coordonnées données \(u^1, u^2, u^3\).
        \end{solution}
        }{}

        \item Déterminer les composantes covariantes $u_1, u_2, u_3$ de \(\textbf{u}\) en fonction de \(u^1, u^2, u^3\).
        
        \ifthenelse{\boolean{showSolutions}}{
        \begin{solution}
        Les composantes covariantes sont obtenues par produit scalaire avec la base :
        \[
        u_i = \langle u, \textbf{e}_i \rangle
        \]
        Donc :
        \[
        u_1 = \langle u, \textbf{e}_1 \rangle = u^1\langle \textbf{e}_1, \textbf{e}_1 \rangle + u^2\langle \textbf{e}_2, \textbf{e}_1 \rangle + u^3\langle \textbf{e}_3, \textbf{e}_1 \rangle = 3u^1 + 2u^2 + u^3
        \]
        \[
        u_2 = \langle u, \textbf{e}_2 \rangle = u^1\langle \textbf{e}_1, \textbf{e}_2 \rangle + u^2\langle \textbf{e}_2, \textbf{e}_2 \rangle + u^3\langle \textbf{e}_3, \textbf{e}_2 \rangle = 2u^1 + 2u^2 + u^3
        \]
        \[
        u_3 = \langle u, \textbf{e}_3 \rangle = u^1\langle \textbf{e}_1, \textbf{e}_3 \rangle + u^2\langle \textbf{e}_2, \textbf{e}_3 \rangle + u^3\langle \textbf{e}_3, \textbf{e}_3 \rangle = u^1 + u^2 + u^3
        \]
        \end{solution}
        }{}
        \item Montrer que le produit scalaire de deux vecteurs \(\langle \textbf{u}, \textbf{v} \rangle\) est égal à \(u^1 v_1 + u^2 v_2 + u^3 v_3\). 

        
        \ifthenelse{\boolean{showSolutions}}{
        \begin{solution}
        On décompose le vecteur \(\textbf{u}\) dans la base :
        \begin{align*}
        \langle \textbf{u}, \textbf{v} \rangle 
        &= \langle  u^1 e_1 + u^2 e_2 + u^3 e_3, v \rangle  \\
        \end{align*}
        On développe le produit scalaire par linéarité, les $u^1$, $u^2$, $u^3$ sont des nombres réels :
        \begin{align*}
        \langle \textbf{u}, \textbf{v} \rangle 
        &= u^1 \langle e_1, v \rangle + u^2 \langle e_2, v \rangle + u^3 \langle e_3, v \rangle \\
        \end{align*}
        Par définition des composantes covariantes, on a :
        \begin{align*}
        \langle e_1, v \rangle = \langle v, e_1 \rangle = v_1, \qquad 
        \langle e_2, v \rangle = \langle v, e_2 \rangle = v_2, \qquad \cdots        \end{align*}
        Donc finalement : 
        \begin{align*}
        \langle \textbf{u}, \textbf{v} \rangle 
        &= u^1 v_1 + u^2 v_2 + u^3 v_3
        \end{align*}
        \end{solution}
        }{}
        \item En utilisant toutes les questions précédentes, calculer le produit scalaire des deux vecteurs donnés dans la base \(\{e_1, e_2, e_3\}\) : \(\textbf{u} = (3,-1,2)\) et \(\textbf{v} = (1,3,5)\).
        
        \ifthenelse{\boolean{showSolutions}}{
        \begin{solution}
        Les composantes contravariantes de \(\textbf{u}\) sont \((3,-1,2)\). 

        Les composantes covariantes de \(\textbf{v}\) sont \((3\cdot 1 + 2 \cdot 3 + 5, 2 \cdot 1 + 2 \cdot 3 + 5, 1 \cdot 1 + 1 \cdot 3 + 5) = (14, 13, 9)\).

        Le produit scalaire est donc :
        \[
        \langle \textbf{u}, \textbf{v} \rangle = \sum_{i=1}^3 u^i v_i = 3 \cdot 14 + (-1) \cdot 13 + 2 \cdot 9 = 42 - 13 + 18 = 47\,.
        \]

        \end{solution}
        }{}
        \item Calculer la norme du vecteur \(\textbf{v}\).

        \ifthenelse{\boolean{showSolutions}}{
        \begin{solution}
        La norme du vecteur \(\textbf{v}\) est :
        \[
        \lVert \textbf{v} \rVert = \sqrt{\langle \textbf{v}, \textbf{v} \rangle} 
        \]
        En utilisant la formule précedente, on a : 
        \[
        \langle \textbf{v}, \textbf{v} \rangle = \sum_{i=1}^3 v^i v_i = 1\cdot 14 + 3\cdot 13 + 5\cdot 9 = 14 + 39 + 45 = 98
        \]
        Donc :
        \[
        \lVert \textbf{v} \rVert = \sqrt{98}
        \]
        \end{solution}
        }{}
    \end{enumerate}
 

    \ifthenelse{\boolean{showSolutions}}{}{\newpage}
    
\section*{Exercice 3 : Quelques exemples de tenseurs.}

On considère le plan muni d'une \textit{ancienne base} \(\mathcal{B} = (\vec{e}_1, \vec{e}_2)\) et d'une \textit{nouvelle base} \(\mathcal{B}' = (\vec{f}_1, \vec{f}_2)\). On note la \textit{matrice de passage} de la nouvelle base à l'ancienne

$$P =
\begin{pmatrix}
\dfrac{\partial f_1}{\partial e_1} & \dfrac{\partial f_2}{\partial e_1} \\
& \\
\dfrac{\partial f_1}{\partial e_2} & \dfrac{\partial f_2}{\partial e_2}
\end{pmatrix}
$$

Dans cet exercice, on passe d'un référentiel cartésien orthonormé aux coordonnées polaires.

\begin{enumerate}
  \item Ecrivez les lois de transformation qui permettent de passer du système $(x, y)$ au système $(r, \theta)$, puis leurs inverses.
  \ifthenelse{\boolean{showSolutions}}{
    \begin{solution}
      Les formules de changement de coordonnées du cartésien $(x, y)$ vers les coordonnées polaires $(r, \theta)$ sont :
      \[
      \left\{
        \begin{array}{l}
          x = r\cos\theta\\
          y = r\sin\theta
        \end{array}
      \right.
      \]
      Inversement, pour passer de $(x, y)$ à $(r, \theta)$ :
      \[
      \left\{
        \begin{array}{l}
          r = \sqrt{x^2 + y^2}\\
          \theta = \arctan\left(\dfrac{y}{x}\right)
        \end{array}
      \right.
      \]
      Ainsi, la transformation directe $(r, \theta) \to (x, y)$ consiste à utiliser les formules trigonométriques ci-dessus, et la transformation inverse $(x, y) \to (r, \theta)$ utilise le module et l’argument du point dans le plan.
    \end{solution}
  }{}

  \item En déduire l'expression de la matrice de changement de base.
  \ifthenelse{\boolean{showSolutions}}{
    \begin{solution}
      Pour obtenir la matrice de changement de base, il faut appliquer la transformation inverse à la base $\mathcal{B}$.
      On a :
      \[
      \vec{f}_1 = \cos\theta \vec{e}_1 + \sin\theta \vec{e}_2
      \]
      \[
      \vec{f}_2 = -\sin\theta \vec{e}_1 + \cos\theta \vec{e}_2
      \]
      Donc la matrice de changement de base est :
      \[
      P = \begin{pmatrix}
      \cos\theta & -\sin\theta \\
      \sin\theta & \cos\theta
      \end{pmatrix}
      \]
      Pour passer des anciens vecteurs aux nouveaux, on écrit : 
      \[
      \vec{f}_1 = P \vec{e}_1
      \]
      \[
      \vec{f}_2 = P \vec{e}_2
      \]
      \end{solution}
    }{}
    \noindent On considère une particule qui se déplace dans le plan avec le temps $t$ comme paramètre. \newline
  Soit $M(t)$ la position de la particule à l'instant $t$.
  On a donc le vecteur position de la particule $\vec{OM(t)} = x(t) \vec{e}_1 + y(t) \vec{e}_2$.
      \item Ecrivez le vecteur position dans la base \(\mathcal{B}'\). Le vecteur position est-il un tenseur ?
        \ifthenelse{\boolean{showSolutions}}{
            \begin{solution}
              Le vecteur position dans la base $\mathcal{B}'$ est :
              \[
              \vec{OM(t)} = r(t) \vec{f}_1
              \]
              écrit avec $x$ et $y$ : 
              \[
              \vec{OM(t)} = \sqrt{x(t)^2 + y(t)^2} \vec{f}_1 
              \]
              On ne peut pas passer de la description cartésienne à la description polaire de manière linéaire, donc le vecteur position n'est pas un tenseur.
            \end{solution}
        }{}

    \item Considérons maintenant le vecteur vitesse \(\vec{v} = \dfrac{d\vec{OM}}{dt}\).
    \begin{enumerate}
        \item Montrer comment se transforment les composantes du vecteur vitesse lors du changement de base.
        \ifthenelse{\boolean{showSolutions}}{
            \begin{solution}
            Dans la base cartésienne $\mathcal{B} = (\vec{e}_1, \vec{e}_2)$, le vecteur vitesse s'écrit :
            \[
            \vec{v} = \frac{d\vec{OM}}{dt} = \dot{x}(t) \vec{e}_1 + \dot{y}(t) \vec{e}_2
            \]
            où $\dot{x} = \frac{dx}{dt}$ et $\dot{y} = \frac{dy}{dt}$.

            Dans la base polaire $\mathcal{B}' = (\vec{f}_1, \vec{f}_2)$, le vecteur position est $\vec{OM} = r(t) \vec{f}_1$. 
            En dérivant par rapport au temps, on obtient :
            \[
            \vec{v} = \frac{d}{dt}(r \vec{f}_1) = \dot{r} \vec{f}_1 + r \frac{d\vec{f}_1}{dt}
            \]
            
            Or, $\vec{f}_1 = \cos\theta \vec{e}_1 + \sin\theta \vec{e}_2$ dépend de $\theta(t)$, donc :
            \[
            \frac{d\vec{f}_1}{dt} = \frac{d\vec{f}_1}{d\theta} \cdot \dot{\theta} = (-\sin\theta \vec{e}_1 + \cos\theta \vec{e}_2) \dot{\theta} = \dot{\theta} \vec{f}_2
            \]
            
            Ainsi, dans la base polaire :
            \[
            \vec{v} = \dot{r} \vec{f}_1 + r\dot{\theta} \vec{f}_2
            \]

            Pour obtenir la transformation des composantes, on utilise la matrice de passage $P$ :
            \[
            \begin{pmatrix} \dot{x} \\ \dot{y} \end{pmatrix} = P \begin{pmatrix} \dot{r} \\ r\dot{\theta} \end{pmatrix}
            \]
            où $P = \begin{pmatrix} \cos\theta & -\sin\theta \\ \sin\theta & \cos\theta \end{pmatrix}$.

            Inversement, les composantes dans la base polaire se transforment selon :
            \[
            \begin{pmatrix} \dot{r} \\ r\dot{\theta} \end{pmatrix} = P^{-1} \begin{pmatrix} \dot{x} \\ \dot{y} \end{pmatrix}
            \]
            avec $P^{-1} = \begin{pmatrix} \cos\theta & \sin\theta \\ -\sin\theta & \cos\theta \end{pmatrix}$.
            \end{solution}
        }{}
        \item Le vecteur vitesse est-il un tenseur ? Justifier brièvement.
        \ifthenelse{\boolean{showSolutions}}{
            \begin{solution}
            Oui, le vecteur vitesse est un tenseur (plus précisément un tenseur d'ordre 1, c'est-à-dire un vecteur).

            En effet, les composantes du vecteur vitesse se transforment de manière linéaire lors d'un changement de base via la matrice de passage $P$ :
            \[
            \begin{pmatrix} v'^1 \\ v'^2 \end{pmatrix} = P^{-1} \begin{pmatrix} v^1 \\ v^2 \end{pmatrix}
            \]
            où $(v^1, v^2)$ sont les composantes dans l'ancienne base et $(v'^1, v'^2)$ dans la nouvelle base.

            Cette transformation linéaire est exactement la définition d'un tenseur d'ordre 1 (vecteur contravariant). 
            Contrairement au vecteur position qui n'est pas un tenseur (car sa transformation n'est pas linéaire), 
            le vecteur vitesse, étant la dérivée du vecteur position, se comporte comme un tenseur.
             \end{solution}
        }{}

    \end{enumerate}

\end{enumerate}


\section*{Exercice 4 : Symboles de Christoffel}
    On se place dans l'espace euclidien \(\mathbb{R}^3\) muni d'une base quelconque \(\{\textbf{e}_1, \textbf{e}_2, \textbf{e}_3\}\).
    On se donne un champ de vecteurs  
    \[
    \textbf{A} = A^1 \textbf{e}_1 + A^2 \textbf{e}_2 + A^3 \textbf{e}_3
    \]
    Si l'on considère la dérivée de \(\textbf{A}\) le long de \(\textbf{e}_1\), on a :
    \[
    \nabla_{\textbf{e}_1} \textbf{A} = \partial_1 \textbf{A} = \partial_1 \Big(A^1 \textbf{e}_1 + A^2 \textbf{e}_2 + A^3 \textbf{e}_3\Big)
    \]
    Dans un repère quelconque, il faut dériver les vecteurs de la base :
    \[
    \partial_1 \textbf{A}
    = (\partial_1 A^1) \textbf{e}_1 + (\partial_1 A^2) \textbf{e}_2 + (\partial_1 A^3) \textbf{e}_3 + A^1 (\partial_1 \textbf{e}_1) + A^2 (\partial_1 \textbf{e}_2) + A^3 (\partial_1 \textbf{e}_3)
    \]

    Les symboles de Christoffel expriment les coordonnées de ces dérivées selon les vecteurs de la base :
    \[
    \partial_j \textbf{e}_i = \Gamma^1_{ij} \textbf{e}_1 + \Gamma^2_{ij} \textbf{e}_2 + \Gamma^3_{ij} \textbf{e}_3
    \]

    Plaçons-nous par exemple en coordonnées polaires, avec \(\textbf{e}_1 = \textbf{e}_r\) et \(\textbf{e}_2 = \textbf{e}_\theta\).

    \begin{enumerate}
        \item En vous ramenant au système de coordonnées cartésiennes \(\{\textbf{i}, \textbf{j}\}\), déterminer la dérivée de \(\textbf{e}_r\) par rapport à \(\theta\).


        \ifthenelse{\boolean{showSolutions}}{
        \begin{solution}
        En coordonnées cartésiennes, \(\textbf{e}_r = \cos(\theta) \textbf{i} + \sin(\theta) \textbf{j}\).
        La dérivée de \(\textbf{e}_r\) par rapport à \(\theta\) est :
        \[
        \partial_\theta \textbf{e}_r = -\sin(\theta) \textbf{i} + \cos(\theta) \textbf{j} = -\textbf{e}_\theta
        \]
        \end{solution}
        }{}

        \item Calculer tous les symboles de Christoffel pour les coordonnées cylindriques.

        \ifthenelse{\boolean{showSolutions}}{
        \begin{solution}
            La question précédente nous donne :
            \[
            \Gamma^1_{12} = 0, \qquad \Gamma^2_{12} = -1
            \]
            Les dérivées par rapport à \(r\) des vecteurs de base sont nulles donc :
            \[
            \Gamma^1_{11} = \Gamma^2_{11} = \Gamma^1_{21} = \Gamma^2_{21} = 0
            \]
            Par rapport à \(\theta\), on a :
            \[
            \partial_\theta \textbf{e}_\theta = -\cos(\theta) \textbf{i} - \sin(\theta) \textbf{j} = -\textbf{e}_r
            \]
            Ainsi :
            \[
            \Gamma^1_{12} = 0, \qquad \Gamma^2_{12} = -1. 
            \]
        \end{solution}
        }{}

    \end{enumerate}
\end{document}