\documentclass[11pt,a4paper]{report}

% -------------------- Encodage & langue --------------------
\usepackage[T1]{fontenc}
\usepackage[utf8]{inputenc}
\usepackage[french]{babel}
\usepackage{lmodern}
\usepackage{microtype}
\usepackage{amsmath, amssymb}
\usepackage{multicol}
\usepackage{enumitem}

\usepackage{amsfonts}
\usepackage[version=4]{mhchem}
\usepackage{stmaryrd}
\usepackage{graphicx}
\usepackage[export]{adjustbox}
\usepackage{caption}
\usepackage{multirow}
\usepackage{tikz}
% -------------------- Mise en page --------------------------
\usepackage[a4paper,margin=2cm]{geometry}
\usepackage{fancyhdr}
\usepackage{parskip}      % espace entre paragraphes
\setlength{\parindent}{0pt}

% -------------------- Couleurs & liens ----------------------
\usepackage{xcolor}
\definecolor{Theme}{HTML}{0E7490} % teal-700
\definecolor{ThemeLight}{HTML}{E0F2F1}
\definecolor{Accent}{HTML}{F59E0B} % amber-500
\definecolor{Gray}{HTML}{374151}
\usepackage[colorlinks=true,linkcolor=Theme,urlcolor=Theme,citecolor=Theme]{hyperref}

% -------------------- Graphiques / décor --------------------
\usepackage{tikz}
\usetikzlibrary{patterns,positioning,calc}
\usepackage{graphicx}
\usepackage{tcolorbox}
\tcbuselibrary{skins,breakable,hooks,most}
\usepackage{fontawesome5}

% -------------------- Titres -------------------------------
\usepackage{titlesec}
\titleformat{\chapter}[display]
  {\Huge\bfseries\color{Theme}}
  {\filright\rule{0.75\linewidth}{1.2pt}\\[3pt]{Algèbre linéaire - Chapitre~\thechapter}}
  {0.2ex}
  {\filright}
  [\vspace{0.1ex}\rule{0.35\linewidth}{1.2pt}]

\titleformat{\section}
  {\Large\bfseries\color{Gray}}
  {\thesection}{0.6em}{}

% -------------------- En-têtes / pieds ---------------------
\pagestyle{fancy}
\fancyhf{}
\fancyhead[L]{\color{Gray}\leftmark}
\fancyhead[R]{\color{Gray}\textit{MEF - 2025/2026}}
\fancyfoot[R]{\color{Gray}\small p.\ \thepage}
\renewcommand{\headrulewidth}{0pt}
\renewcommand{\footrulewidth}{0pt}

% -------------------- Macros utilitaires -------------------
\newenvironment{solution}
{
    \vspace{0.5em}
    \begin{mdframed}[backgroundcolor=ThemeLight,leftmargin=0,rightmargin=0,skipabove=0.2em,skipbelow=0.2em]
    \textbf{Solution.}\\[0.5em]
}
{
    \end{mdframed}
    \vspace{0.5em}
}



% -------------------- Page de titre ------------------------
\title{\textbf{Traces de cours}\\\large (résumés, formules, exemples, mini-exercices)}
\author{ MEF - 2025/2026 }
\date{\today}


\makeatletter
\renewcommand{\thesubsection}{\arabic{subsection}}
\renewcommand{\p@subsection}{}% supprime le préfixe section/chapter dans \ref
% Si vous voulez la même chose pour les sous-sous-sections :
% \renewcommand{\thesubsubsection}{\arabic{subsubsection}}
% \renewcommand{\p@subsubsection}{}
\makeatother

\usepackage{mdframed}
\usepackage{ifthen}

% \usepackage[sf]{titlesec}
% Définition de la variable pour afficher les corrections
\newboolean{showSolutions}
% Décommentez la ligne suivante pour afficher les solutions
\input \jobname.adr
% -------------------- Document ----------------------------
\begin{document}

\begin{center}
    {\LARGE \textbf{Analyse et Algèbre - TD1}}\\[1em]
    {\large \textit{Calcul vectoriel et tensoriel}}
\end{center}

\section*{Exercice 1 :}
Importante remarque: En coordonnées cartésiennes, les coordonnées ( $x_i$ ) d'un point sont aussi les composantes de son vecteur position $\underline{x}$. Ceci n'est plus vrai en coordonnées curvilignes! Il est faux d'écrire

$$
\underline{x}=r \underline{g}_r+\theta \underline{g}_\theta+z \underline{g}_z
$$

Nous allons maintenant rechercher l'expression de l'opérateur $\underline{\nabla}$ en coordonnées curvilignes. En coordonnées rectangulaires, nous l'avions défini comme suit

$$
\underline{\nabla}=\underline{e}_1 \frac{\partial}{\partial x_1}+\underline{e}_2 \frac{\partial}{\partial x_2}+\underline{e}_3 \frac{\partial}{\partial x_3} \quad \text { ou } \quad \underline{e}_i \frac{\partial}{\partial x_i} .
$$


Pour exprimer les opérateurs différentiels relatifs aux coordonnées cartésiennes ( $x_i$ ) en fonction des opérateurs différentiels relatifs aux coordonnées curvilignes $\left(\alpha_i\right)$, on applique la règle de dérivation des fonctions composées:

$$
\frac{\partial}{\partial x_i}=\frac{\partial \alpha_j}{\partial x_i} \frac{\partial}{\partial \alpha_j}
$$

\ifthenelse{\boolean{showSolutions}}{
  \begin{solution}
En appliquant cette règle, on trouve
En appliquant cette règle, on trouve
- pour les coordonnées cylindriques

$$
\begin{aligned}
\frac{\partial}{\partial x_1} & =(\cos \theta) \frac{\partial}{\partial r}+\left(-\frac{\sin \theta}{r}\right) \frac{\partial}{\partial \theta}+(0) \frac{\partial}{\partial z} \\
\frac{\partial}{\partial x_2} & =(\sin \theta) \frac{\partial}{\partial r}+\left(\frac{\cos \theta}{r}\right) \frac{\partial}{\partial \theta}+(0) \frac{\partial}{\partial z} \\
\frac{\partial}{\partial x_3} & =(0) \frac{\partial}{\partial r}+(0) \frac{\partial}{\partial \theta}+(1) \frac{\partial}{\partial z}
\end{aligned}
$$

- pour les coordonnées sphériques

$$
\begin{aligned}
\frac{\partial}{\partial x_1} & =(\sin \theta \cos \phi) \frac{\partial}{\partial r}+\left(\frac{\cos \theta \cos \phi}{r}\right) \frac{\partial}{\partial \theta}+\left(-\frac{\sin \phi}{r \sin \theta}\right) \frac{\partial}{\partial z} \\
\frac{\partial}{\partial x_2} & =(\sin \theta \sin \phi) \frac{\partial}{\partial r}+\left(\frac{\cos \theta \sin \phi}{r}\right) \frac{\partial}{\partial \theta}+\left(\frac{\cos \phi}{r \sin \theta}\right) \frac{\partial}{\partial z} \\
\frac{\partial}{\partial x_3} & =(\cos \theta) \frac{\partial}{\partial r}+\left(-\frac{\sin \theta}{r}\right) \frac{\partial}{\partial \theta}+(0) \frac{\partial}{\partial z}
\end{aligned}
$$


En injectant les expressions (3.6) et (4.15) dans l'expression (4.13) du gradient en coord cartésiennes, on obtient l'expression de $\underline{\nabla}$ en coordonnées cylindriques:

$$
\begin{aligned}
\underline{\nabla}= & \underline{e}_x \frac{\partial}{\partial x}+\underline{e}_y \frac{\partial}{\partial y}+\underline{e}_z \frac{\partial}{\partial z} \\
= & \left(\underline{e}_r \cos \theta-\underline{e}_\theta \sin \theta\right)\left((\cos \theta) \frac{\partial}{\partial r}+\left(-\frac{\sin \theta}{r}\right) \frac{\partial}{\partial \theta}\right) \\
& +\left(\underline{e}_r \sin \theta+\underline{e}_\theta \cos \theta\right)\left((\sin \theta) \frac{\partial}{\partial r}+\left(+\frac{\cos \theta}{r}\right) \frac{\partial}{\partial \theta}\right)+\underline{e}_z \frac{\partial}{\partial z}
\end{aligned}
$$

et, après avoir simplifié, on trouve finalement

$$
\underline{\nabla}=\underline{e}_r \frac{\partial}{\partial r}+\underline{e}_\theta \frac{1}{r} \frac{\partial}{\partial \theta}+\underline{e}_z \frac{\partial}{\partial z}
$$


Un développement analogue utilisant les expressions (3.9) et (4.16) permet de trouver l'e de $\underline{\nabla}$ en coordonnées sphériques

$$
\underline{\nabla}=\underline{e}_r \frac{\partial}{\partial r}+\underline{e}_\phi \frac{1}{r} \frac{\partial}{\partial \phi}+\underline{e}_\theta \frac{1}{r \sin \phi} \frac{\partial}{\partial \theta}
$$
\end{solution}
}{}


Donner des vecteurs et demander l'ordre d'un tenseur.


Produit tensoriel 

\end{document}