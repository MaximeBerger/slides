\documentclass[11pt,a4paper]{report}

% -------------------- Encodage & langue --------------------
\usepackage[T1]{fontenc}
\usepackage[utf8]{inputenc}
\usepackage[french]{babel}
\usepackage{lmodern}
\usepackage{microtype}
\usepackage{amsmath, amssymb}
\usepackage{multicol}
\usepackage{enumitem}

\usepackage{amsfonts}
\usepackage[version=4]{mhchem}
\usepackage{stmaryrd}
\usepackage{graphicx}
\usepackage[export]{adjustbox}
\usepackage{caption}
\usepackage{multirow, multicol}
\usepackage{tikz}
% -------------------- Mise en page --------------------------
\usepackage[a4paper,margin=2cm]{geometry}
\usepackage{fancyhdr}
\usepackage{parskip}      % espace entre paragraphes
\setlength{\parindent}{0pt}

% -------------------- Couleurs & liens ----------------------
\usepackage{xcolor}
\definecolor{Theme}{HTML}{0E7490} % teal-700
\definecolor{ThemeLight}{HTML}{E0F2F1}
\definecolor{Accent}{HTML}{F59E0B} % amber-500
\definecolor{Gray}{HTML}{374151}
\usepackage[colorlinks=true,linkcolor=Theme,urlcolor=Theme,citecolor=Theme]{hyperref}

% -------------------- Graphiques / décor --------------------
\usepackage{tikz}
\usetikzlibrary{patterns,positioning,calc}
\usepackage{graphicx}
\usepackage{tcolorbox}
\tcbuselibrary{skins,breakable,hooks,most}

% -------------------- Titres -------------------------------
\usepackage{titlesec}
\titleformat{\chapter}[display]
  {\Huge\bfseries\color{Theme}}
  {\filright\rule{0.75\linewidth}{1.2pt}\\[3pt]{Algèbre linéaire - Chapitre~\thechapter}}
  {0.2ex}
  {\filright}
  [\vspace{0.1ex}\rule{0.35\linewidth}{1.2pt}]

\titleformat{\section}
  {\Large\bfseries\color{Gray}}
  {\thesection}{0.6em}{}

% -------------------- En-têtes / pieds ---------------------
\pagestyle{fancy}
\fancyhf{}
\fancyhead[L]{\color{Gray}\leftmark}
\fancyhead[R]{\color{Gray}\textit{Analyse-Algèbre - 2025/2026}}
\fancyfoot[R]{\color{Gray}\small p.\ \thepage}
\fancyfoot[L]{\color{Gray}\small \textit{Maxime Berger}}
\renewcommand{\headrulewidth}{0pt}
\renewcommand{\footrulewidth}{0pt}

% -------------------- Macros utilitaires -------------------
\newenvironment{solution}
{
    \vspace{0.5em}
    \begin{mdframed}[backgroundcolor=ThemeLight,leftmargin=0,rightmargin=0,skipabove=0.2em,skipbelow=0.2em]
    \textbf{Solution.}\\[0.5em]
}
{
    \end{mdframed}
    \vspace{0.5em}
}

% Commande pour la transformée de Laplace
\newcommand{\Lap}{\mathcal{L}}
\newcommand{\Four}{\mathcal{F}}
\newcommand{\Distr}{\mathcal{D}}
\newcommand{\Schwartz}{\mathcal{S}}
\newcommand{\Sha}{\text{Ш}} % Peigne de Dirac

% -------------------- Page de titre ------------------------
\title{\textbf{Traces de cours}\\\large (résumés, formules, exemples, mini-exercices)}
\author{ Analyse-Algèbre - 2025/2026 }
\date{\today}


\makeatletter
\renewcommand{\thesubsection}{\arabic{subsection}}
\renewcommand{\p@subsection}{}% supprime le préfixe section/chapter dans \ref
\makeatother

\usepackage{mdframed}
\usepackage{ifthen}

% Définition de la variable pour afficher les corrections
\newboolean{showSolutions}
% Décommentez la ligne suivante pour afficher les solutions
\input \jobname.adr
% -------------------- Document ----------------------------
\begin{document}

\begin{center}
    {\LARGE \textbf{Analyse et Algèbre - TD5}}\\[1em]
    {\large \textit{Distributions}}
\end{center}

%============================================================
% PARTIE 1 : EXERCICES D'INTRODUCTION
%============================================================

\section*{Exercice 1 : Le Dirac et la fonction de Heaviside}

On rappelle que la distribution de Dirac $\delta_a$ est définie par $\langle \delta_a, \varphi \rangle = \varphi(a)$ pour toute fonction test $\varphi \in \mathcal{D}(\mathbb{R})$. \newline
(Rappel : une fonction test est une fonction $\varphi \in \mathcal{C}^\infty(\mathbb{R})$ à support compact.)


\textit{Les fonctions $\varphi$ suivantes n'appartiennent pas à $\mathcal{D}(\mathbb{R})$, l'objectif de cet exercice est purement pédagogique.}
\begin{enumerate}
    \item Calculer $\langle \delta_0, \varphi \rangle$ pour $\varphi(x) = x^2 + 3x + 1$.
    
    \ifthenelse{\boolean{showSolutions}}{
    \begin{solution}
        $\langle \delta_0, \varphi \rangle = \varphi(0) = 0^2 + 3 \times 0 + 1 = 1$.
    \end{solution}
    }{}
    
    \item Calculer $\langle \delta_2, \varphi \rangle$ pour $\varphi(x) = e^{-x^2}$.
    
    \ifthenelse{\boolean{showSolutions}}{
    \begin{solution}
        $\langle \delta_2, \varphi \rangle = \varphi(2) = e^{-4}$.
    \end{solution}
    }{}
    
    \item Calculer $\langle \delta_{-1} + 2\delta_3, \varphi \rangle$ pour $\varphi(x) = \cos(\pi x)$.
    
    \ifthenelse{\boolean{showSolutions}}{
    \begin{solution}
        $\langle \delta_{-1} + 2\delta_3, \varphi \rangle = \varphi(-1) + 2\varphi(3) = \cos(-\pi) + 2\cos(3\pi) = -1 + 2(-1) = -3$.
    \end{solution}
    }{}

    \vspace{1em}
    
    On note $H$ la fonction de Heaviside :
    \[
    H(x) = \begin{cases} 1 & \text{si } x \geq 0 \\ 0 & \text{si } x < 0 \end{cases}
    \]

    \item Calculer $\langle H, \varphi \rangle$ pour $\varphi(x) = e^{-x}$.

    \textit{Indication : on rappelle l'action d'une distribution sur une fonction test : $\langle T, \varphi \rangle = \int_{-\infty}^{+\infty} T(x) \varphi(x) dx$.}
   

    \ifthenelse{\boolean{showSolutions}}{
    \begin{solution}
        $\langle H, \varphi \rangle = \int_{-\infty}^{+\infty} H(x) \varphi(x) dx = \int_0^{+\infty} \varphi(x) dx = \int_0^{+\infty} e^{-x} dx = [e^{-x}]_0^{+\infty} = 1$.
    \end{solution}
    }{}

    \item Calculer $\langle H + 2\delta_{-\pi}, \varphi \rangle$ pour $\varphi(x) = \frac{1}{1+x^2}$.
    
    \ifthenelse{\boolean{showSolutions}}{
    \begin{solution}
        $\langle H + 2\delta_{-\pi}, \varphi \rangle = \langle H, \varphi \rangle + 2\langle \delta_{-\pi}, \varphi \rangle = \int_0^{+\infty} \frac{1}{1+x^2} dx + 2 \cdot \frac{1}{1+\pi^2} = \frac{\pi}{2} + \frac{2}{1+\pi^2}$.
    \end{solution}
    }{}

\end{enumerate}

\vspace{1em}
\section*{Exercice 2 : Dériver des fonctions discontinues}

Montrer que la dérivée (au sens des distributions) de la fonction de Heaviside $H$ est égale à $\delta_0$.
    
    \textit{Indication : Utiliser la définition $\langle H', \varphi \rangle = -\langle H, \varphi' \rangle$.}
    
    \ifthenelse{\boolean{showSolutions}}{
    \begin{solution}
        Par définition de la dérivée d'une distribution :
        \[
        \langle H', \varphi \rangle = -\langle H, \varphi' \rangle = -\int_{-\infty}^{+\infty} H(x) \varphi'(x) \, dx = -\int_0^{+\infty} \varphi'(x) \, dx
        \]
        
        Comme $\varphi$ est à support compact, $\varphi(x) \to 0$ quand $x \to +\infty$. Donc :
        \[
        \langle H', \varphi \rangle = -[\varphi(x)]_0^{+\infty} = -\left(0 - \varphi(0)\right) = \varphi(0) = \langle \delta_0, \varphi \rangle
        \]
        
        Ainsi $H' = \delta_0$.
    \end{solution}
 }{}

 \vspace{1em}

 On peut aussi calculer la dérivée des distributions en utilisant nos connaissances sur les dérivées classiques : On dérive quand c'est possible et ajoute des diracs aux points de discontinuité. Voir dans le dernier cours pour un exemple. 
\vspace{1em}

Pour chaque fonction $f$ suivante, calculer sa dérivée au sens des distributions en utilisant la formule des sauts :
\[
f' = \{f'\} + \sum_k \sigma_k \delta_{a_k}
\]
où $\{f'\}$ désigne la dérivée classique de $f$ là où elle existe, et $\sigma_k = f(a_k^+) - f(a_k^-)$ est le saut en $a_k$.

\begin{enumerate}
    \item $f(x) = H(x-1)$ (Heaviside translatée)
    
    \ifthenelse{\boolean{showSolutions}}{
    \begin{solution}
        La fonction $H(x-1)$ vaut $0$ pour $x < 1$ et $1$ pour $x \geq 1$.
        
        - La dérivée classique est nulle sur $]-\infty, 1[$ et $]1, +\infty[$.
        - Il y a un saut en $x = 1$ de valeur $\sigma = 1 - 0 = 1$.
        
        Donc $f' = \delta_1$.
    \end{solution}
    }{}
    
    \item $f(x) = H(x) - H(x-2)$ (fonction porte sur $[0,2]$)
    
    \ifthenelse{\boolean{showSolutions}}{
    \begin{solution}
        Cette fonction vaut $1$ sur $[0,2[$ et $0$ ailleurs.
        
        - La dérivée classique est nulle partout où $f$ est continue.
        - Saut en $x = 0$ : $\sigma_0 = 1 - 0 = 1$.
        - Saut en $x = 2$ : $\sigma_2 = 0 - 1 = -1$.
        
        Donc $f' = \delta_0 - \delta_2$.
    \end{solution}
    }{}
    
    \item $f(x) = xH(x)$ (rampe)
    
    \ifthenelse{\boolean{showSolutions}}{
    \begin{solution}
        - La dérivée classique est $0$ pour $x < 0$ et $1$ pour $x > 0$, soit $\{f'\} = H(x)$.
        - Il n'y a pas de saut en $x = 0$ car $f(0^-) = 0$ et $f(0^+) = 0$.
        
        Donc $f' = H(x)$ (pas de Dirac car la fonction est continue).
    \end{solution}
    }{}
    
    \item $f(x) = x^2H(x)$
    
    \ifthenelse{\boolean{showSolutions}}{
    \begin{solution}
        - La dérivée classique est $0$ pour $x < 0$ et $2x$ pour $x > 0$, soit $\{f'\} = 2x \cdot H(x)$.
        - Pas de saut en $x = 0$ car $f(0^-) = 0 = f(0^+)$.
        
        Donc $f' = 2x \cdot H(x)$.
    \end{solution}
    }{}
\end{enumerate}


%============================================================
% PARTIE 3 : ÉQUATIONS DIFFÉRENTIELLES AVEC DISTRIBUTIONS
%============================================================
\vspace{1em}
\section*{Exercice 3 : Équation différentielle avec second membre impulsionnel}

On cherche à résoudre l'équation différentielle au sens des distributions :
\[
y' + 2y = \delta_0
\]
où $y$ est une distribution.

\begin{enumerate}
    \item Quelle est l'interprétation physique de cette équation ?
    
    \ifthenelse{\boolean{showSolutions}}{
    \begin{solution}
        Cette équation modélise un système linéaire du premier ordre (circuit RC, système mécanique amorti) soumis à une impulsion instantanée à $t = 0$. Le terme $\delta_0$ représente une excitation de durée infiniment courte mais d'intensité infinie (comme un choc ou une décharge électrique).
    \end{solution}
    }{}
    
    \item On cherche $y$ sous la forme $y = f \cdot H$ où $f$ est une fonction continue et $H$ la fonction de Heaviside. Calculer $y'$ au sens des distributions.
    
    \ifthenelse{\boolean{showSolutions}}{
    \begin{solution}
        On utilise la règle de dérivation d'un produit au sens des distributions. Si $y = f \cdot H$ :
        \[
        y' = f' \cdot H + f \cdot H' = f' \cdot H + f(0) \delta_0
        \]
        car $H' = \delta_0$ et $f(x)\delta_0 = f(0)\delta_0$.
    \end{solution}
    }{}
    
    \item En substituant dans l'équation, montrer que $f$ doit vérifier $f' + 2f = 0$ sur $]0, +\infty[$ et déterminer $f(0)$.
    
    \ifthenelse{\boolean{showSolutions}}{
    \begin{solution}
        En substituant $y = fH$ dans l'équation :
        \[
        f'H + f(0)\delta_0 + 2fH = \delta_0
        \]
        Soit :
        \[
        (f' + 2f)H + f(0)\delta_0 = \delta_0
        \]
        
        En identifiant :
        \begin{itemize}
            \item La partie régulière (coefficient de $H$) : $f' + 2f = 0$ sur $]0, +\infty[$.
            \item La partie singulière : $f(0) = 1$.
        \end{itemize}
    \end{solution}
    }{}
    
    \item Résoudre l'équation différentielle $f' + 2f = 0$ et en déduire $y$.
    
    \ifthenelse{\boolean{showSolutions}}{
    \begin{solution}
        L'équation $f' + 2f = 0$ a pour solution générale $f(t) = Ce^{-2t}$.
        
        Avec la condition $f(0) = 1$, on obtient $C = 1$, donc $f(t) = e^{-2t}$.
        
        La solution est :
        \[
        \boxed{y(t) = e^{-2t} H(t)}
        \]
        
        C'est la réponse impulsionnelle du système : nulle pour $t < 0$, égale à $e^{-2t}$ pour $t \geq 0$.
    \end{solution}
    }{}
    
    \item Vérifier que cette solution est correcte en calculant $y' + 2y$.
    
    \ifthenelse{\boolean{showSolutions}}{
    \begin{solution}
        Calculons $y' + 2y$ avec $y = e^{-2t}H(t)$ :
        \[
        y' = -2e^{-2t}H + e^{-2 \cdot 0}\delta_0 = -2e^{-2t}H + \delta_0
        \]
        Donc :
        \[
        y' + 2y = -2e^{-2t}H + \delta_0 + 2e^{-2t}H = \delta_0 \quad \checkmark
        \]
    \end{solution}
    }{}
\end{enumerate}

\vspace{1em}
\section*{Exercice 4 : Équation du second ordre}

On cherche à résoudre l'équation différentielle au sens des distributions :
\[
y'' + y = \delta_0
\]

\begin{enumerate}
    \item On cherche $y$ sous la forme $y = f(t) H(t)$ où $f$ est de classe $\mathcal{C}^1$ sur $[0, +\infty[$. Calculer $y'$ et $y''$ au sens des distributions.
    
    \ifthenelse{\boolean{showSolutions}}{
    \begin{solution}
        $y' = f'H + f(0)\delta_0$
        
        Pour $y''$, on dérive $y'$ :
        \[
        y'' = f''H + f'(0)\delta_0 + f(0)\delta_0'
        \]
        où $\delta_0'$ est la dérivée du Dirac.
    \end{solution}
    }{}
    
    \item En substituant et en supposant $f(0) = 0$ (pour éviter le terme en $\delta_0'$), montrer que $f$ doit vérifier $f'' + f = 0$ avec $f(0) = 0$ et $f'(0) = 1$.
    
    \ifthenelse{\boolean{showSolutions}}{
    \begin{solution}
        En substituant :
        \[
        f''H + f'(0)\delta_0 + f(0)\delta_0' + fH = \delta_0
        \]
        
        Avec $f(0) = 0$, le terme en $\delta_0'$ disparaît :
        \[
        (f'' + f)H + f'(0)\delta_0 = \delta_0
        \]
        
        En identifiant :
        \begin{itemize}
            \item $f'' + f = 0$ sur $]0, +\infty[$
            \item $f'(0) = 1$
        \end{itemize}
    \end{solution}
    }{}
    
    \item En déduire la solution $y$.
    
    \ifthenelse{\boolean{showSolutions}}{
    \begin{solution}
        L'équation $f'' + f = 0$ a pour solution générale $f(t) = A\cos(t) + B\sin(t)$.
        
        Avec $f(0) = 0$ : $A = 0$.
        
        Avec $f'(0) = 1$ : $f'(t) = B\cos(t)$, donc $B = 1$.
        
        La solution est :
        \[
        \boxed{y(t) = \sin(t) \cdot H(t)}
        \]
    \end{solution}
    }{}
\end{enumerate}

\vspace{1em}
\section*{Exercice 5 : Manipulations de fonctions test}
On considère la fonction 
$$
\psi(x)= \begin{cases}
    \mathrm{e}^{-1 /(1-x^2)} & \text { si }|x|<1 \\ 
    0 & \text { sinon }
\end{cases}
$$
\begin{enumerate}
    \item Déterminer le support de la fonction $\psi$. La fonction $\psi$ est-elle dans $\mathbb{D}$ ? \newline
    \textit{On pourra étudier $\psi(-1+h)$ et $\psi(1-h)$, avec $h$ un réel positif qui tend vers $0$.}

    \ifthenelse{\boolean{showSolutions}}{
    \begin{solution}
        Pour $|x| < 1$, on a $\psi(x) = e^{-1/(1-x^2)} > 0$.
        
        Étudions le comportement aux bords :
        \begin{itemize}
            \item $\psi(1-h) = e^{-1/(1-(1-h)^2)} = e^{-1/(2h - h^2)} \to 0$ quand $h \to 0^+$ car $2h - h^2 \to 0^+$.
            \item De même, $\psi(-1+h) \to 0$.
        \end{itemize}
        
        Le support de $\psi$ est $[-1,1]$. 
        
        
        Pour vérifier que la fonction $\psi$ est de classe $\mathcal{C}^\infty$ à support compact, il faudrait montrer que toutes les dérivées de $\psi$ sont continues et que le support de $\psi$ est compact. 
        Il faudrait pour cela exprimer la dérivée $n$-ième de $\psi$ et vérifier que toutes les dérivées sont continues.

        Contentons-nous de vérifier que la dérivée première est continue:
        \[
        \psi'(x) = \frac{2x}{(1-x^2)^2} e^{-1/(1-x^2)}
        \]
        qui tend bien vers $0$ quand $x \to \pm 1$.


    \end{solution}
}{}
    \item En déduire une fonction $\varphi\in \mathbb{D}$ dont le maximum est $3$ et dont le support est $[-2,2]$.

    \ifthenelse{\boolean{showSolutions}}{
    \begin{solution}
        Le maximum de $\psi$ est atteint en $x=0$ et vaut $\psi(0) = e^{-1}$.
        
        On pose $\varphi(x) = 3e \cdot \psi(x/2)$.
        \begin{itemize}
            \item Le support de $\varphi$ est $[-2,2]$ (dilatation par 2).
            \item Le maximum vaut $\varphi(0) = 3e \cdot e^{-1} = 3$.
        \end{itemize}
    \end{solution}
}{}

On définit les distributions suivantes :
$$
T_1 = \delta_0, \quad T_2 = \delta_{-1}+\delta_1, \quad T_3(\varphi) = \int_{-\infty}^\infty x \varphi(x) dx, \quad T_4(\varphi) = \int_{5}^{10} \sqrt{x} \varphi(x) dx
$$

\item Quelles valeurs associent-elles à la fonction $\psi$ ?

\ifthenelse{\boolean{showSolutions}}{
\begin{solution}
    \begin{itemize}
        \item $T_1(\psi) = \psi(0) = e^{-1}$
        \item $T_2(\psi) = \psi(-1) + \psi(1) = 0 + 0 = 0$
        \item $T_3(\psi) = \int_{-1}^1 x \, e^{-1/(1-x^2)} dx = 0$ (fonction impaire)
        \item $T_4(\psi) = \int_5^{10} \sqrt{x} \psi(x) dx = 0$ (car $\psi = 0$ sur $[5,10]$)
    \end{itemize}
\end{solution}
}{}
\end{enumerate}




\end{document}
