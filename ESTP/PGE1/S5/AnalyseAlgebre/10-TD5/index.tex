\documentclass[11pt,a4paper]{report}

% -------------------- Encodage & langue --------------------
\usepackage[T1]{fontenc}
\usepackage[utf8]{inputenc}
\usepackage[french]{babel}
\usepackage{lmodern}
\usepackage{microtype}
\usepackage{amsmath, amssymb}
\usepackage{multicol}
\usepackage{enumitem}

\usepackage{amsfonts}
\usepackage[version=4]{mhchem}
\usepackage{stmaryrd}
\usepackage{graphicx}
\usepackage[export]{adjustbox}
\usepackage{caption}
\usepackage{multirow, multicol}
\usepackage{tikz}
% -------------------- Mise en page --------------------------
\usepackage[a4paper,margin=2cm]{geometry}
\usepackage{fancyhdr}
\usepackage{parskip}      % espace entre paragraphes
\setlength{\parindent}{0pt}

% -------------------- Couleurs & liens ----------------------
\usepackage{xcolor}
\definecolor{Theme}{HTML}{0E7490} % teal-700
\definecolor{ThemeLight}{HTML}{E0F2F1}
\definecolor{Accent}{HTML}{F59E0B} % amber-500
\definecolor{Gray}{HTML}{374151}
\usepackage[colorlinks=true,linkcolor=Theme,urlcolor=Theme,citecolor=Theme]{hyperref}

% -------------------- Graphiques / décor --------------------
\usepackage{tikz}
\usetikzlibrary{patterns,positioning,calc}
\usepackage{graphicx}
\usepackage{tcolorbox}
\tcbuselibrary{skins,breakable,hooks,most}

% -------------------- Titres -------------------------------
\usepackage{titlesec}
\titleformat{\chapter}[display]
  {\Huge\bfseries\color{Theme}}
  {\filright\rule{0.75\linewidth}{1.2pt}\\[3pt]{Algèbre linéaire - Chapitre~\thechapter}}
  {0.2ex}
  {\filright}
  [\vspace{0.1ex}\rule{0.35\linewidth}{1.2pt}]

\titleformat{\section}
  {\Large\bfseries\color{Gray}}
  {\thesection}{0.6em}{}

% -------------------- En-têtes / pieds ---------------------
\pagestyle{fancy}
\fancyhf{}
\fancyhead[L]{\color{Gray}\leftmark}
\fancyhead[R]{\color{Gray}\textit{Analyse-Algèbre - 2025/2026}}
\fancyfoot[R]{\color{Gray}\small p.\ \thepage}
\fancyfoot[L]{\color{Gray}\small \textit{Maxime Berger}}
\renewcommand{\headrulewidth}{0pt}
\renewcommand{\footrulewidth}{0pt}

% -------------------- Macros utilitaires -------------------
\newenvironment{solution}
{
    \vspace{0.5em}
    \begin{mdframed}[backgroundcolor=ThemeLight,leftmargin=0,rightmargin=0,skipabove=0.2em,skipbelow=0.2em]
    \textbf{Solution.}\\[0.5em]
}
{
    \end{mdframed}
    \vspace{0.5em}
}

% Commande pour la transformée de Laplace
\newcommand{\Lap}{\mathcal{L}}
\newcommand{\Four}{\mathcal{F}}
\newcommand{\Distr}{\mathcal{D}}
\newcommand{\Schwartz}{\mathcal{S}}
\newcommand{\Sha}{\text{Ш}} % Peigne de Dirac

% -------------------- Page de titre ------------------------
\title{\textbf{Traces de cours}\\\large (résumés, formules, exemples, mini-exercices)}
\author{ Analyse-Algèbre - 2025/2026 }
\date{\today}


\makeatletter
\renewcommand{\thesubsection}{\arabic{subsection}}
\renewcommand{\p@subsection}{}% supprime le préfixe section/chapter dans \ref
\makeatother

\usepackage{mdframed}
\usepackage{ifthen}

% Définition de la variable pour afficher les corrections
\newboolean{showSolutions}
% Décommentez la ligne suivante pour afficher les solutions
\input \jobname.adr
% -------------------- Document ----------------------------
\begin{document}

\begin{center}
    {\LARGE \textbf{Analyse et Algèbre - TD5}}\\[1em]
    {\large \textit{Distributions}}
\end{center}




\section*{Exercice 1 : Manipulation de distributions}
On considère la fonction 
$$
\psi(x)= \begin{cases}
    \mathrm{e}^{-1 /(1-x^2)} & \text { si }|x|<1 \\ 
    0 & \text { sinon }
\end{cases}
$$
\begin{enumerate}
    \item Déterminer le support de la fonction $\psi$. La fonction $\psi$ est-elle dans $\mathbb{D}$ ? \newline
    \textit{On pourra étudier $\psi(-1+h)$ et $\psi(1-h)$, avec $h$ un réel positif qui tend vers $0$.}

    \ifthenelse{\boolean{showSolutions}}{
    \begin{solution}
        On a $\psi(-1+h) = \mathrm{e}^{-1/(1-(1-h)^2)} = \mathrm{e}^{-1/(h^2)} \to 0$ quand $h \to 0$ car $-1/(h^2) \to -\infty$. Aussi $\psi(1-h) = 0$. Donc le support de $\psi$ est $[-1,1]$.
    \end{solution}
}{
}
    \item En déduire une fonction $\varphi\in \mathbb{D}$ dont le maximum est $3$ et dont le support est $[-2,2]$.

    \ifthenelse{\boolean{showSolutions}}{
    \begin{solution}
        Au point $x=0$, la fonction $\psi$ atteint son maximum qui vaut $\psi(0) = 1/e$, 
        On définit alors $\varphi(x) = 3e \psi(x/2)$ pour $x \in [-2,2]$, $0$ ailleurs. 
    \end{solution}
}{
}

On définit les distributions suivantes :
$$
T_1 = \delta_0, \quad T_2 = \delta_{-1}+\delta_1, \quad T_3(\varphi) = \int_{-\infty}^\infty x \varphi(x) dx, \quad T_4(\varphi) = \int_{5}^{10} \sqrt{x} \varphi(x) dx
$$

\item Quelles valeurs associent-elles à la fonction $\psi$ et à la fonction $\varphi$ ?

\ifthenelse{\boolean{showSolutions}}{
\begin{solution}
    On a 
    
    \[
    T_1(\psi) = \psi(0) = 1/e \quad \text{et} \quad T_1(\varphi) = \varphi(0) = 3e \psi(0) = 3e/e = 3
    \]
    Pour $T_2$, on a
    \[
    T_2(\psi) = \psi(-1) + \psi(1) = 0 \quad \text{et} \quad T_2(\varphi) = \varphi(-1) + \varphi(1) = 3e \psi(-1/2) + 3e \psi(1/2) = 6 e e^{-4/3} = 6 e^{-1/3}
    \]
    Pour $T_3$:
    \[
    T_3(\psi) = \int_{-\infty}^\infty x \psi(x) dx = \int_{-1}^1 x \mathrm{e}^{-1/(1-x^2)} dx = 0
    \]
    puisque la fonction intégrée est impaire.
    De même, $T_3(\varphi) = \int_{-\infty}^\infty x \varphi(x) dx = 0$.

    Pour $T_4$, on a
    \[
    T_4(\psi) = \int_5^{10} x \psi(x) dx = 0
    \]
    car $\psi$ est nulle en dehors de $[-1,1]$.
    De même, $T_4(\varphi) = \int_5^{10} x \varphi(x) dx = 0$.
\end{solution}
}{
}
\item Pour chaque distribution ci-dessus, existe-t-il une fonction $f$ localement intégrable telle que $T(\varphi) = \int f \varphi$ ?

\ifthenelse{\boolean{showSolutions}}{
\begin{solution}
    Les diracs ne peuvent pas être représentés par une fonction localement intégrable, il faudrait trouver une fonction $f$ telle que $\int_\mathbb{R} f \varphi = \varphi(0)$ pour tout $\varphi \in \mathbb{D}$. Donc que $f$ soit nulle en dehors du point $0$, donc nulle presque partout.

    Pour $T_3$, on a $T_3(\varphi) = \int_{-\infty}^\infty x \varphi(x) dx$, donc $T_3$ est construite à partir de la fonction $f(x) = x$.
    Pour $T_4$, on a $T_4(\varphi) = \int_5^{10} \sqrt{x} \varphi(x) dx$, donc $T_4$ est construite à partir de la fonction $f(x) = \sqrt{x}1_{[5,10]}(x)$.
\end{solution}
}{
}
\item Les fonctions $\mathbb{D}\rightarrow \mathbb{R}$ suivantes sont-elles des distributions ?
\[
T_1(\varphi) = \frac{1}{\varphi(0)}, \quad T_2(\varphi) = \int_{-\infty}^\infty \frac{\varphi(x)}{x^2} dx
\]
\ifthenelse{\boolean{showSolutions}}{
\begin{solution}
    La fonction $T_1$ n'est pas définie pour toutes les fonctions de $\mathbb{D}$, donc elle n'est pas une distribution.
    La fonction $T_2$ n'est pas une distribution non plus, elle conduit à une intégrale qui n'est pas définie si $\varphi$ est une fonction non nulle au voisinage de $0$.
\end{solution}
}{
}
\end{enumerate}
Il n'est pas facile de calculer les intégrales issues des distributions, le but de cette théorie n'est pas d'effectuer des calculs sur les fonctions $\mathbb{C}^\infty$ à support compact, on n'explicitera plus jamais de telle fonction $\varphi$.



\section*{Exercice 2 : Dériver une distribution}
\begin{enumerate}
    \item Soit $T_1$ la distribution associée à la fonction $f(x)=\operatorname{signe}(x)$. C'est-à-dire: si $\varphi$ est une fonction $\mathbb{C}^\infty$ à support compact,
    \[
    T_1(\varphi) = \int_{-\infty}^\infty \operatorname{signe}(x) \varphi(x) dx = -\int_{-\infty}^0 \varphi(x) dx + \int_0^\infty \varphi(x) dx
    \]
    Calculer la dérivée de $T_1$ de deux manières différentes:
    \begin{enumerate}
        \item En utilisant la définition de la dérivée d'une distribution.
        \item En utilisant la formule des sauts.
    \end{enumerate}

    \ifthenelse{\boolean{showSolutions}}{
    \begin{solution}
        \textbf{Par définition.}Prenons $\varphi \in \mathcal{D}(\mathbb{R})$. On a

$$
T_1^{\prime}\big(\varphi\big) = -T_1\big(\varphi^{\prime}\big)=\int_{-\infty}^0 \varphi^{\prime}(x) d x-\int_0^{+\infty} \varphi^{\prime}(x) d x=2 \varphi(0) .
$$


Ainsi, $T_1^{\prime}=2 \delta_0$.

        \textbf{Par la formule des sauts.} On a un seul saut dans cette fonction, en $x=0$. De plus, la dérivée sur les deux intervalles $]-\infty,0]$ et $[0,\infty[$ est nulle car la fonction est constante, la formule des sauts donne donc
\[
T_1^{\prime}= \sigma \delta_0
\]
où $\sigma$ est la valeur du saut en $x=0$ : $\sigma = \operatorname{signe}(0^+) - \operatorname{signe}(0^-) = 2$.
    \end{solution}
}{
}
    \item Soit $T_2$ la distribution associée à la fonction $g(x)=|x|$. Calculer la dérivée de $T_2$.

    \ifthenelse{\boolean{showSolutions}}{
    \begin{solution}
        Prenons $\varphi \in \mathcal{D}(\mathbb{R})$. On a

$$
T_2^{\prime}\big(\varphi\big) = -T_2\big(\varphi^{\prime}\big)=\int_{-\infty}^0 x \varphi^{\prime}(x) d x-\int_0^{+\infty} x \varphi^{\prime}(x) d x .
$$


On exprime alors chacune de ces intervalles à l'aide d'une intégration par parties. Par exemple, on a

$$
\int_0^{+\infty} x \varphi^{\prime}(x) d x=[x \varphi(x)]_0^{+\infty}-\int_0^{+\infty} \varphi(x) d x=-\int_0^{+\infty} \varphi(x) d x .
$$


On a de même

$$
\int_{-\infty}^0 x \varphi^{\prime}(x) d x=-\int_{-\infty}^0 \varphi(x) d x
$$

soit

$$
T_2^{\prime}\big(\varphi\big) = -\int_{-\infty}^0 \varphi(x) d x+\int_0^{+\infty} \varphi(x) d x
$$


Ainsi, $T_2^{\prime}$ est la distribution associée à la fonction $u=\mathbf{1}_{[0,+\infty[ }-\mathbf{1}_{]-\infty, 0]}$.
\end{solution}
}{
}
    \item Soit $T_3$ la distribution associée à la fonction $h(x)=x^2$. Calculer la dérivée de $T_3$.

    \ifthenelse{\boolean{showSolutions}}{
    \begin{solution}
        Prenons $\varphi \in \mathcal{D}(\mathbb{R})$. On a

\begin{align*}
T_3^{\prime}\big(\varphi\big) = -T_3\big(\varphi^{\prime}\big)
&=-\int_{-\infty}^\infty x^2 \varphi^{\prime}(x) d x = -\lim_{A \to \infty} \int_{-A}^A x^2 \varphi^{\prime}(x) d x \\
&= -\lim_{A \to \infty} \Big[ x^2 \varphi(x) \Big]_{-A}^A + \int_{-A}^A 2x \varphi(x) d x
\end{align*}
Quand $A$ tend vers l'infini, les deux termes tendent vers $0$ car $\varphi$ est à support compact. Donc la dérivée de $T_3$ est la distribution associée à la fonction $2x$:
\[
T_3^{\prime}(\varphi) = \int_{-\infty}^\infty 2x \varphi(x) d x 
\]


    \end{solution}
}{
}
\end{enumerate}


\section*{Exercice 3 : Multiplication par une fonction}
Soit $T$ une distribution et $\psi$ une fonction de classe $\mathcal{C}^\infty$, $\varphi$ une fonction $\mathbb{C}^\infty$ à support compact.
\begin{enumerate}
    \item Montrer que $S(\varphi) = T(\psi \varphi)$ est bien une application de $\mathbb{D}$ dans $\mathbb{R}$.
    \ifthenelse{\boolean{showSolutions}}{
    \begin{solution}
        On a $S(\varphi) = T(\psi \varphi)$, et $\psi \varphi$ est encore une fonction $\mathbb{C}^\infty$ à support compact, donc $S$ associe bien à chaque fonction de $\mathbb{D}$ un réel.
    \end{solution}
}{
}
    \item Que vaut alors $S'$ ? 
    \ifthenelse{\boolean{showSolutions}}{
    \begin{solution}
        On a $S'(\varphi) = -S(\varphi') = -T(\psi \varphi') $.
    \end{solution}
}{
}
    \vspace{1em}
    \item Soit $\left(T_n\right)$ une suite de distributions qui converge vers $T$, c'est-à-dire que $T_n\big(\varphi\big) \to T\big(\varphi\big)$ pour tout $\varphi \in \mathcal{D}(\mathbb{R})$, montrer que $(T_n^{\prime})$ converge vers $T^{\prime}$.
    \ifthenelse{\boolean{showSolutions}}{
    \begin{solution}
        Soit $\phi \in \mathcal{D}(\mathbb{R})$. On a :

        $$
        \left\langle T_n^{\prime}, \varphi\right\rangle=-\left\langle T_n, \varphi^{\prime}\right\rangle \rightarrow-\left\langle T, \varphi^{\prime}\right\rangle=\left\langle T^{\prime}, \varphi\right\rangle .
        $$
            \end{solution}
}{
}
\end{enumerate}


\section*{Exercice 4 : La valeur principale}
Le but de cet exercice est de déterminer la dérivée de la distribution associée à ln $|x|$. 
\begin{enumerate}
    \item Soit $\varphi \in \mathcal{D}(\mathbb{R})$ et $a>0$ tel que le support de $\varphi$ soit contenu dans $[-a, a]$. 
    Notons $T$ la distribution associée à la fonction $\ln |x|$. Explicitez l'intégrale définissant $T'(\varphi)$.

    \ifthenelse{\boolean{showSolutions}}{
    \begin{solution}
        $$
        \begin{aligned}
        \left\langle T^{\prime}, \varphi\right\rangle & =-\left\langle T, \varphi^{\prime}\right\rangle \\
        & =-\int_{-a}^a \varphi^{\prime}(x) \ln |x| d x \\
        \end{aligned}
        $$
        
    \end{solution}
}{
}
\item La fonction logarithme pose problème au voisinage de $0$, nous allons définir $\varepsilon$ et écrire
\[
    \int_{-a}^a \ln |x|\varphi'(x) d x = \lim_{\varepsilon \to 0^+} \bigg(\int_{-a}^{-\varepsilon} \ln (-x) \varphi'(x) d x + \int_{\varepsilon}^a \ln (x) \varphi'(x) d x\bigg)
\]
Traiter les deux intégrales séparément pour un $\varepsilon$ fixé et intégrer par parties.

    \ifthenelse{\boolean{showSolutions}}{
    \begin{solution}
        $$
        \int_{\varepsilon}^a \varphi^{\prime}(x) \ln (x) d x=[\varphi(x) \ln (x)]_{\varepsilon}^a-\int_{\varepsilon}^a \frac{\varphi(x)}{x} d x=-\varphi(\varepsilon) \ln (\varepsilon)-\int_{\varepsilon}^a \frac{\varphi(x)}{x} d x .
        $$
        
        et
        
        $$
        \int_{-a}^{-\varepsilon} \varphi^{\prime}(x) \ln (-x) d x=[\varphi(x) \ln (-x)]_{-a}^{-\varepsilon}-\int_{-a}^{-\varepsilon} \frac{\varphi(x)}{x} d x=\varphi(-\varepsilon) \ln (\varepsilon)-\int_{-a}^{-\varepsilon} \frac{\varphi(x)}{x} d x .
        $$
    \end{solution}
}{
}
    \item Certains termes se simplifient quand $\varepsilon$ tend vers $0$, simplifiez au maximum.
    \ifthenelse{\boolean{showSolutions}}{
        \begin{solution}
        
            
            $$
            (\varphi(-\varepsilon)-\varphi(\varepsilon)) \ln (\varepsilon)=\left(\varphi(0)-\varepsilon \varphi^{\prime}(0)-\varphi(0)-\varepsilon \varphi^{\prime}(0)+o(\varepsilon)\right) \ln (\varepsilon)=-2 \varphi^{\prime}(0) \varepsilon \ln (\varepsilon)+o(\varepsilon \ln (\varepsilon))
            $$
            
            
            Mais comme $\varepsilon \ln (\varepsilon) \rightarrow 0$ quand $\varepsilon \rightarrow 0$, on conclut finalement que
            
            $$
            \left\langle T^{\prime}, \varphi\right\rangle=\lim _{\varepsilon \rightarrow 0}\left(\int_{-a}^{-\varepsilon} \frac{\varphi(x)}{x} d x+\int_{\varepsilon}^a \frac{\varphi(x)}{x} d x\right)=\left\langle\operatorname{vp} \frac{1}{x}, \varphi\right\rangle .
            $$
            \end{solution}
        }{
        }
    \end{enumerate}

\textit{La distribution $S(\varphi) = \lim_{\epsilon \to 0^+} \int_{|x|>\epsilon} \frac{\varphi(x)}{x} dx$ est appelée la valeur principale.}


\section*{Exercice 5 : Encore plus de Diracs}
Calculer explicitement $\left\langle x^\alpha \partial^\beta \delta_p, \phi\right\rangle$ pour tout $\phi \in \mathcal{D}\left(\mathbb{R}^n\right)$, où $\alpha$ et $\beta$ sont des entiers et $\delta_p$ est la masse de Dirac au point $p \in \mathbb{R}$.

Quel est le support de $x^\alpha \partial^\beta \delta_p$ ?

\ifthenelse{\boolean{showSolutions}}{
\begin{solution}
    On a

    $$
    \begin{gathered}
    \left\langle x^\alpha \partial^\beta \delta_p, \phi\right\rangle=\left\langle\partial^\beta \delta_p, x^\alpha \phi\right\rangle=(-1)^\beta\left\langle\delta_p, \partial^\beta\left(x^\alpha \phi\right)\right\rangle \\
    \partial^\beta\left(x^\alpha \phi\right)=\sum_{k=0}^{\min (\alpha, \beta)}\binom{\beta}{k} \alpha \ldots(\alpha-k+1) x^{\alpha-k} \partial^{\beta-k} \phi
    \end{gathered}
    $$
    
    
    Donc,
    
    $$
    \left\langle x^\alpha \partial^\beta \delta_p, \phi\right\rangle=(-1)^\beta \sum_{k=0}^{\min (\alpha, \beta)}\binom{\beta}{k} \alpha \ldots(\alpha-k+1) p^{\alpha-k} \partial^{\beta-k} \phi(p) .
    $$
    
    
    En particulier, le support est $\{p\}$ si $p \neq 0$, il est vide si $p=0$ et $\alpha>\beta$.
\end{solution}
}{
}

\section*{Exercice 6 : Limites de distributions}

On considère la fonction $\chi: \mathbb{R} \longrightarrow \mathbb{R}$ définie par $\chi(x)=1$ si $x \in[-1,1], \chi(x)=0$ sinon.
\begin{enumerate}
    \item Dire pourquoi la fonction $\chi$ définit donc une distribution $T_\chi \in \mathcal{D}^{\prime}(\mathbb{R})$ et rappeler la définition de $T_\chi$.
    \item On définit la suite ( $\chi_n$ ) par $\chi_n(x)=\frac{n}{2} \chi(n x)$. Déterminer la limite de $\left(\chi_n\right)$ au sens des distributions (i.e trouver la limite dans $\mathcal{D}^{\prime}(\mathbb{R})$ de la suite de distributions $\left(T_{\chi_n}\right)$ associées à $\left(\chi_n\right)$ ).
    \item On définit la suite $\left(\xi_n\right)$ par $\xi_n(x)=\chi(x-n)$. Déterminer la limite de $\left(\xi_n\right)$ au sens des distributions (i.e trouver la limite dans $\mathcal{D}^{\prime}(\mathbb{R})$ de distributions $\left(T_{\xi_n}\right)$ associées à $\left(\xi_n\right)$ ).
\end{enumerate}
Soit $p_\epsilon$ définie par

$$
p_\epsilon=\left\{\begin{array}{l}
0 \text { si } x \in\left[0, \frac{1}{2}-\frac{\epsilon}{2}[\cup] \frac{1}{2}+\frac{\epsilon}{2}, 1\right] \\
\frac{1}{\epsilon} \text { si } x \in\left[\frac{1}{2}-\frac{\epsilon}{2}, \frac{1}{2}+\frac{\epsilon}{2}\right]
\end{array}\right.
$$


Montrer que $\lim _{\epsilon \rightarrow 0} p_\epsilon=\delta_{\frac{1}{2}}$ dans $\mathcal{D}^{\prime}([0,1[)$.
\end{document}
